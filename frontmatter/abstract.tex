\begin{abstract}
  Herein we develop category-theoretic tools for understanding network-style
  diagrammatic languages. The archetypal network-style diagrammatic language is
  that of electric circuits; other examples include signal flow graphs, Markov
  processes, automata, Petri nets, chemical reaction networks, and so on. The
  key feature is that the language is comprised of a number of `components' with
  multiple (input/output) terminals, each possibly labelled with some type, that
  may then be connected together along these terminals to form a larger
  `network'. The components form hyperedges between labelled vertices, and so a
  diagram in this language forms a hypergraph. We formalise the compositional
  structure by introducing the notion of a hypergraph category. Network-style
  diagrammatic languages and their semantics thus form hypergraph categories,
  and semantic interpretation gives a hypergraph functor. 
  
  The first part of this thesis develops the theory of hypergraph categories. In
  particular, we introduce the tools of decorated cospans and corelations.
  Decorated cospans allow straightforward construction of hypergraph categories
  from diagrammatic languages: the inputs, outputs, and their composition are
  modelled by the cospans, while the `decorations' specify the components
  themselves. Not all hypergraph categories can be constructed, however, through
  decorated cospans. Decorated corelations are a more powerful version that
  permits construction of all hypergraph categories and hypergraph functors.
  These are often useful for constructing the semantic categories of
  diagrammatic languages and functors from diagrams to the semantics. To
  illustrate these principles, the second part of this thesis details
  applications to linear time-invariant dynamical systems and passive linear
  networks.
\end{abstract}
