\documentclass[12pt%,twoside
]{ociamthesis}

%\includeonly{auxiliary/definitions,auxiliary/tikz,
%frontmatter/abstract,chapters/overview,chapters/introduction,
%chapters/1,
%chapters/2,
%chapters/3
%}

%% packages
\usepackage[centertags,sumlimits,intlimits,namelimits,reqno]{amsmath}
\usepackage{latexsym,amsfonts,amssymb,exscale,enumerate,amsthm,breakcites}
\usepackage[colorlinks,linkcolor=darkblue,citecolor=darkblue,urlcolor=darkblue,breaklinks=true,pagebackref]{hyperref}
\usepackage{tikz}
\usepackage{float,color,multirow,array,multicol}
\usepackage[all,cmtip,2cell]{xy}


%% citations
\definecolor{darkblue}{rgb}{0,0,0.7} 
\renewcommand*{\backref}[1]{(Referred to on page #1.)}

%% theorem environments

\newtheorem{theorem}{Theorem}[chapter]
\newtheorem{lemma}[theorem]{Lemma}
\newtheorem{proposition}[theorem]{Proposition}
\newtheorem{corollary}[theorem]{Corollary}

%\newtheorem{conjecture}{Conjecture}
%%\renewcommand{\theconj}{\Alph{conjecture}}  % numbered A, B, C etc
%
\theoremstyle{definition}
\newtheorem{definition}[theorem]{Definition}
%\newtheorem{defns}[theorem]{Definitions}
%
%\theoremstyle{remark} 
\newtheorem{example}[theorem]{Example}
\newtheorem{examples}[theorem]{Examples}
%\newtheorem{question}[theorem]{Question}
\newtheorem{remark}[theorem]{Remark}
\newtheorem{notation}[theorem]{Notation}

%% 
\newcommand\maps{{\colon}}
\newcommand{\define}[1]{{\bf \boldmath #1}}

%% from decorated cospans

  \newcommand{\R}{{\mathbb{R}}}
  \newcommand{\hooklongrightarrow}{\lhook\joinrel\longrightarrow}
  \newcommand{\linsub}{\operatorname{LinSub}}
  \newcommand{\lgraph}{\operatorname{Graph}}
  \newcommand{\res}{\operatorname{Res}}
  \newcommand{\FinSet}{\mathrm{FinSet}}
  \newcommand{\Set}{\mathrm{Set}}
  \newcommand{\opp}{\mathrm{opp}}

%% from decorated corelations
  \newcommand{\FinVect}{\mathrm{FinVect}}
  \newcommand{\Vect}{\mathrm{Vect}}
  \newcommand{\LinRel}{\mathrm{LinRel}}
  \DeclareMathOperator\corel{{Corel}}
  \DeclareMathOperator\cospan{{Cospan}}


%% from circuits

\newcommand\Z{{\mathbb Z}}   
\newcommand\C{{\mathbb C}}
\newcommand\F{{\mathbb F}}
\newcommand{\vect}[1]{{\F^{#1} \oplus {(\F^{#1})}^\ast}}
\renewcommand{\twoheadrightarrow}{\to \hspace{-8pt} \to}
\newcommand{\mc}{\mathcal}
\renewcommand{\a}{\alpha}
\newcommand{\id}{\mathrm{id}}
\newcommand{\s}{\sigma}
\newcommand{\ot}{\otimes}
\newcommand{\eps}{\epsilon}
\newcommand{\ob}{\mathrm{Ob}}
\newcommand{\im}{\mathrm{Im}\,}
\newcommand{\re}{\mathrm{Re}\,}
\newcommand{\mor}{\mathrm{Mor}}
\newcommand{\fgraph}{\operatorname{\mathbb{F}-Graph}}
\newcommand{\Circ}{\mathrm{Circ}}
\newcommand{\LagrRel}{\mathrm{LagrRel}}




\usetikzlibrary{backgrounds,circuits,circuits.ee.IEC,shapes,fit,matrix,automata,decorations.markings}
\UseTwocells

  \pgfdeclarelayer{edgelayer}
  \pgfdeclarelayer{nodelayer}
  \pgfdeclarelayer{background}
  \pgfsetlayers{background,edgelayer,nodelayer,main}

  \tikzset{font=\footnotesize}
  \tikzstyle{none}=[inner sep=0pt]

\tikzstyle{circ}=[circle,fill=black,draw,inner sep=3pt]
\tikzstyle{circ2}=[circle,fill=black,draw,inner sep=1pt]

\tikzstyle{sq}=[rectangle,draw,inner sep=1pt,minimum width=15pt,minimum height=15pt]
  

  \tikzstyle{connection}=[circuit symbol open,
    circuit symbol size=width .5 height .5,
    shape=circle ee,
  inner sep=0.25\pgflinewidth]
  \tikzset{->-/.style={decoration={
    markings,
    mark=at position .54 with {\arrow{latex}}},postaction={decorate}}}

%%fakesubsubsection generators
\newcommand{\mult}[1]
{
\begin{aligned}
    \resizebox{#1}{!}{
\begin{tikzpicture}
	\begin{pgfonlayer}{nodelayer}
		\node [style=none] (0) at (1, -0) {};
		\node [style=circ] (1) at (0.125, -0) {};
		\node [style=none] (2) at (-1, 0.5) {};
		\node [style=none] (3) at (-1, -0.5) {};
	\end{pgfonlayer}
	\begin{pgfonlayer}{edgelayer}
		\draw[line width=2pt] (0.center) to (1.center);
		\draw[line width=2pt] [in=0, out=120, looseness=1.20] (1.center) to (2.center);
		\draw[line width=2pt] [in=0, out=-120, looseness=1.20] (1.center) to (3.center);
	\end{pgfonlayer}
      \end{tikzpicture}}
\end{aligned}
}

\newcommand{\unit}[1]
{
  \begin{aligned}
    \resizebox{#1}{!}{
\begin{tikzpicture}
	\begin{pgfonlayer}{nodelayer}
		\node [style=none] (0) at (1, -0) {};
		\node [style=none] (1) at (-1, -0) {};
		\node [style=circ] (2) at (0, -0) {};
	\end{pgfonlayer}
	\begin{pgfonlayer}{edgelayer}
		\draw[line width=2pt] (0.center) to (2);
	\end{pgfonlayer}
      \end{tikzpicture}}
  \end{aligned}
}

\newcommand{\comult}[1]
{
\begin{aligned}
    \resizebox{#1}{!}{
\begin{tikzpicture}
	\begin{pgfonlayer}{nodelayer}
		\node [style=none] (0) at (-1, -0) {};
		\node [style=circ] (1) at (-0.125, -0) {};
		\node [style=none] (2) at (1, 0.5) {};
		\node [style=none] (3) at (1, -0.5) {};
	\end{pgfonlayer}
	\begin{pgfonlayer}{edgelayer}
		\draw[line width=2pt] (0.center) to (1.center);
		\draw[line width=2pt] [in=180, out=60, looseness=1.20] (1.center) to (2.center);
		\draw[line width=2pt] [in=180, out=-60, looseness=1.20] (1.center) to (3.center);
	\end{pgfonlayer}
      \end{tikzpicture}}
\end{aligned}
}

\newcommand{\counit}[1]
{
  \begin{aligned}
    \resizebox{#1}{!}{
\begin{tikzpicture}
	\begin{pgfonlayer}{nodelayer}
		\node [style=none] (0) at (-1, -0) {};
		\node [style=none] (1) at (1, -0) {};
		\node [style=circ] (2) at (0, -0) {};
	\end{pgfonlayer}
	\begin{pgfonlayer}{edgelayer}
		\draw[line width=2pt] (0.center) to (2);
	\end{pgfonlayer}
      \end{tikzpicture}}
  \end{aligned}
}


\newcommand{\idone}[1]
{
  \begin{aligned}
    \resizebox{#1}{!}{
     \begin{tikzpicture}
	\begin{pgfonlayer}{nodelayer}
		\node [style=none] (0) at (-1, -0) {};
		\node [style=none] (1) at (1, -0) {};
		\node [style=none] (2) at (0, 0.5) {};
		\node [style=none] (3) at (0, -0.5) {};
	\end{pgfonlayer}
	\begin{pgfonlayer}{edgelayer}
		\draw[line width=2pt] (1.center) to (0.center);
	\end{pgfonlayer}
\end{tikzpicture} 
    }
  \end{aligned}
}

\newcommand{\swap}[1]
{
  \begin{aligned}
    \resizebox{#1}{!}{
\begin{tikzpicture}
	\begin{pgfonlayer}{nodelayer}
		\node [style=none] (2) at (-0.5, -0.5) {};
		\node [style=none] (3) at (-2, 0.5) {};
		\node [style=none] (4) at (-0.5, 0.5) {};
		\node [style=none] (5) at (-2, -0.5) {};
	\end{pgfonlayer}
	\begin{pgfonlayer}{edgelayer}
		\draw[line width=2pt] [in=180, out=0, looseness=1.00] (3.center) to (2.center);
		\draw[line width=2pt] [in=0, out=180, looseness=1.00] (4.center) to (5.center);
	\end{pgfonlayer}
\end{tikzpicture}
    }
  \end{aligned}
}
%%fakesubsubsection associativity
\newcommand{\assocl}[1]
{
  \begin{aligned}
    \resizebox{#1}{!}{
\begin{tikzpicture}
	\begin{pgfonlayer}{nodelayer}
		\node [style=circ] (0) at (0.125, -0) {};
		\node [style=none] (1) at (-1, 0.5) {};
		\node [style=none] (2) at (-1, -0.5) {};
		\node [style=none] (3) at (0, -1) {};
		\node [style=none] (4) at (2.25, -0.5) {};
		\node [style=none] (5) at (0.25, -0) {};
		\node [style=circ] (6) at (1.25, -0.5) {};
		\node [style=none] (7) at (-1, -1) {};
	\end{pgfonlayer}
	\begin{pgfonlayer}{edgelayer}
		\draw[line width=2pt] [in=0, out=120, looseness=1.20] (0.center) to (1.center);
		\draw[line width=2pt] [in=0, out=-120, looseness=1.20] (0.center) to (2.center);
		\draw[line width=2pt] (4.center) to (6);
		\draw[line width=2pt] [in=0, out=120, looseness=1.20] (6) to (5.center);
		\draw[line width=2pt] [in=0, out=-120, looseness=1.20] (6) to (3.center);
		\draw[line width=2pt] (3.center) to (7.center);
	\end{pgfonlayer}
      \end{tikzpicture}}
  \end{aligned}
}


\newcommand{\assocr}[1]
{
  \begin{aligned}
    \resizebox{#1}{!}{
\begin{tikzpicture}
	\begin{pgfonlayer}{nodelayer}
		\node [style=circ] (0) at (0.125, -0.5) {};
		\node [style=none] (1) at (-1, -1) {};
		\node [style=none] (2) at (-1, 0) {};
		\node [style=none] (3) at (0, 0.5) {};
		\node [style=none] (4) at (2.25, 0) {};
		\node [style=none] (5) at (0.25, -0.5) {};
		\node [style=circ] (6) at (1.25, 0) {};
		\node [style=none] (7) at (-1, 0.5) {};
	\end{pgfonlayer}
	\begin{pgfonlayer}{edgelayer}
		\draw[line width=2pt] [in=0, out=-120, looseness=1.20] (0.center) to (1.center);
		\draw[line width=2pt] [in=0, out=120, looseness=1.20] (0.center) to (2.center);
		\draw[line width=2pt] (4.center) to (6);
		\draw[line width=2pt] [in=0, out=-120, looseness=1.20] (6) to (5.center);
		\draw[line width=2pt] [in=0, out=120, looseness=1.20] (6) to (3.center);
		\draw[line width=2pt] (3.center) to (7.center);
	\end{pgfonlayer}
      \end{tikzpicture}}
  \end{aligned}
}


\newcommand{\coassocl}[1]
{
  \begin{aligned}
    \resizebox{#1}{!}{
\begin{tikzpicture}
	\begin{pgfonlayer}{nodelayer}
		\node [style=circ] (0) at (1.125, -0.5) {};
		\node [style=none] (1) at (2.25, -1) {};
		\node [style=none] (2) at (2.25, 0) {};
		\node [style=none] (3) at (1.25, 0.5) {};
		\node [style=none] (4) at (-1, 0) {};
		\node [style=none] (5) at (1, -0.5) {};
		\node [style=circ] (6) at (0, 0) {};
		\node [style=none] (7) at (2.25, 0.5) {};
	\end{pgfonlayer}
	\begin{pgfonlayer}{edgelayer}
		\draw[line width=2pt] [in=180, out=-60, looseness=1.20] (0.center) to (1.center);
		\draw[line width=2pt] [in=180, out=60, looseness=1.20] (0.center) to (2.center);
		\draw[line width=2pt] (4.center) to (6);
		\draw[line width=2pt] [in=180, out=-60, looseness=1.20] (6) to (5.center);
		\draw[line width=2pt] [in=180, out=60, looseness=1.20] (6) to (3.center);
		\draw[line width=2pt] (3.center) to (7.center);
	\end{pgfonlayer}
      \end{tikzpicture}}
  \end{aligned}
}


\newcommand{\coassocr}[1]
{
  \begin{aligned}
    \resizebox{#1}{!}{
\begin{tikzpicture}
	\begin{pgfonlayer}{nodelayer}
		\node [style=circ] (0) at (1.125, 0) {};
		\node [style=none] (1) at (2.25, 0.5) {};
		\node [style=none] (2) at (2.25, -0.5) {};
		\node [style=none] (3) at (1.25, -1) {};
		\node [style=none] (4) at (-1, -0.5) {};
		\node [style=none] (5) at (1, 0) {};
		\node [style=circ] (6) at (0, -0.5) {};
		\node [style=none] (7) at (2.25, -1) {};
	\end{pgfonlayer}
	\begin{pgfonlayer}{edgelayer}
		\draw[line width=2pt] [in=180, out=60, looseness=1.20] (0.center) to (1.center);
		\draw[line width=2pt] [in=180, out=-60, looseness=1.20] (0.center) to (2.center);
		\draw[line width=2pt] (4.center) to (6);
		\draw[line width=2pt] [in=180, out=60, looseness=1.20] (6) to (5.center);
		\draw[line width=2pt] [in=180, out=-60, looseness=1.20] (6) to (3.center);
		\draw[line width=2pt] (3.center) to (7.center);
	\end{pgfonlayer}
      \end{tikzpicture}}
  \end{aligned}
}

%%fakesubsubsection unitality
\newcommand{\unitl}[1]
{
  \begin{aligned}
    \resizebox{#1}{!}{
\begin{tikzpicture}
	\begin{pgfonlayer}{nodelayer}
		\node [style=none] (0) at (1, -0) {};
		\node [style=circ] (1) at (0.125, -0) {};
		\node [style=circ] (2) at (-1, 0.5) {};
		\node [style=none] (3) at (-1, -0.5) {};
		\node [style=none] (4) at (-2, -0.5) {};
	\end{pgfonlayer}
	\begin{pgfonlayer}{edgelayer}
		\draw[line width=2pt] (0.center) to (1.center);
		\draw[line width=2pt] [in=0, out=120, looseness=1.20] (1.center) to (2.center);
		\draw[line width=2pt] [in=0, out=-120, looseness=1.20] (1.center) to (3.center);
		\draw[line width=2pt] (4.center) to (3.center);
	\end{pgfonlayer}
\end{tikzpicture}
    }
  \end{aligned}
}

\newcommand{\counitl}[1]
{
  \begin{aligned}
    \resizebox{#1}{!}{
\begin{tikzpicture}
	\begin{pgfonlayer}{nodelayer}
		\node [style=none] (0) at (-2, -0) {};
		\node [style=circ] (1) at (-1.125, -0) {};
		\node [style=circ] (2) at (0, 0.5) {};
		\node [style=none] (3) at (0, -0.5) {};
		\node [style=none] (4) at (1, -0.5) {};
	\end{pgfonlayer}
	\begin{pgfonlayer}{edgelayer}
		\draw[line width=2pt] (0.center) to (1.center);
		\draw[line width=2pt] [in=180, out=60, looseness=1.20] (1.center) to (2.center);
		\draw[line width=2pt] [in=180, out=-60, looseness=1.20] (1.center) to (3.center);
		\draw[line width=2pt] (4.center) to (3.center);
	\end{pgfonlayer}
\end{tikzpicture}
    }
  \end{aligned}
}

%%fakesubsubsection commutativity
\newcommand{\commute}[1]
{
  \begin{aligned}
    \resizebox{#1}{!}{
\begin{tikzpicture}
	\begin{pgfonlayer}{nodelayer}
		\node [style=none] (0) at (1.25, -0) {};
		\node [style=circ] (1) at (0.375, -0) {};
		\node [style=none] (2) at (-0.5, -0.5) {};
		\node [style=none] (3) at (-2, 0.5) {};
		\node [style=none] (4) at (-0.5, 0.5) {};
		\node [style=none] (5) at (-2, -0.5) {};
	\end{pgfonlayer}
	\begin{pgfonlayer}{edgelayer}
		\draw[line width=2pt] (0.center) to (1.center);
		\draw[line width=2pt] [in=0, out=-120, looseness=1.20] (1.center) to (2.center);
		\draw[line width=2pt] [in=180, out=0, looseness=1.00] (3.center) to (2.center);
		\draw[line width=2pt] [in=0, out=120, looseness=1.20] (1.center) to (4.center);
		\draw[line width=2pt] [in=0, out=180, looseness=1.00] (4.center) to (5.center);
	\end{pgfonlayer}
\end{tikzpicture}
    }
  \end{aligned}
}


\newcommand{\cocommute}[1]
{
  \begin{aligned}
    \resizebox{#1}{!}{
\begin{tikzpicture}
	\begin{pgfonlayer}{nodelayer}
		\node [style=none] (0) at (-2, -0) {};
		\node [style=circ] (1) at (-1.125, -0) {};
		\node [style=none] (2) at (-0.25, -0.5) {};
		\node [style=none] (3) at (1.25, 0.5) {};
		\node [style=none] (4) at (-0.25, 0.5) {};
		\node [style=none] (5) at (1.25, -0.5) {};
	\end{pgfonlayer}
	\begin{pgfonlayer}{edgelayer}
		\draw[line width=2pt] (0.center) to (1.center);
		\draw[line width=2pt] [in=180, out=-60, looseness=1.20] (1.center) to (2.center);
		\draw[line width=2pt] [in=0, out=180, looseness=1.00] (3.center) to (2.center);
		\draw[line width=2pt] [in=180, out=60, looseness=1.20] (1.center) to (4.center);
		\draw[line width=2pt] [in=180, out=0, looseness=1.00] (4.center) to (5.center);
	\end{pgfonlayer}
\end{tikzpicture}
    }
  \end{aligned}
}
%%fakesubsubsection frobenius


\newcommand{\frobs}[1]
{
  \begin{aligned}
    \resizebox{#1}{!}{
\begin{tikzpicture}
	\begin{pgfonlayer}{nodelayer}
		\node [style=none] (0) at (-1.5, 0.5) {};
		\node [style=circ] (1) at (-0.75, 0.5) {};
		\node [style=none] (2) at (0.25, -0) {};
		\node [style=none] (3) at (0.25, 1) {};
		\node [style=circ] (4) at (1, -0.5) {};
		\node [style=none] (5) at (0, -0) {};
		\node [style=none] (6) at (1.75, -0.5) {};
		\node [style=none] (7) at (0, -1) {};
		\node [style=none] (8) at (1.75, 1) {};
		\node [style=none] (9) at (-1.5, -1) {};
	\end{pgfonlayer}
	\begin{pgfonlayer}{edgelayer}
		\draw[line width=2pt] [in=180, out=-60, looseness=1.20] (1) to (2.center);
		\draw[line width=2pt] [in=180, out=60, looseness=1.20] (1) to (3.center);
		\draw[line width=2pt] (0.center) to (1);
		\draw[line width=2pt] (6.center) to (4);
		\draw[line width=2pt] [in=0, out=120, looseness=1.20] (4) to (5.center);
		\draw[line width=2pt] [in=0, out=-120, looseness=1.20] (4) to (7.center);
		\draw[line width=2pt] (3.center) to (8.center);
		\draw[line width=2pt] (7.center) to (9.center);
	\end{pgfonlayer}
\end{tikzpicture}
    }
  \end{aligned}
}

\newcommand{\frobx}[1]
{
  \begin{aligned}
    \resizebox{#1}{!}{
\begin{tikzpicture}
	\begin{pgfonlayer}{nodelayer}
		\node [style=circ] (0) at (-0.5, -0) {};
		\node [style=none] (1) at (-1.5, -0.5) {};
		\node [style=none] (2) at (-1.5, 0.5) {};
		\node [style=circ] (3) at (0.5, -0) {};
		\node [style=none] (4) at (1.5, -0.5) {};
		\node [style=none] (5) at (1.5, 0.5) {};
	\end{pgfonlayer}
	\begin{pgfonlayer}{edgelayer}
		\draw[line width=2pt, in=0, out=-120, looseness=1.20] (0.center) to (1.center);
		\draw[line width=2pt, in=0, out=120, looseness=1.20] (0.center) to (2.center);
		\draw[line width=2pt, in=180, out=-60, looseness=1.20] (3) to (4.center);
		\draw[line width=2pt, in=180, out=60, looseness=1.20] (3) to (5.center);
		\draw[line width=2pt] (0) to (3);
	\end{pgfonlayer}
\end{tikzpicture}
    }
  \end{aligned}
}

\newcommand{\frobz}[1]
{
  \begin{aligned}
    \resizebox{#1}{!}{
\begin{tikzpicture}
	\begin{pgfonlayer}{nodelayer}
		\node [style=none] (0) at (1.75, 0.5) {};
		\node [style=circ] (1) at (1, 0.5) {};
		\node [style=none] (2) at (0, -0) {};
		\node [style=none] (3) at (0, 1) {};
		\node [style=circ] (4) at (-0.75, -0.5) {};
		\node [style=none] (5) at (0.25, -0) {};
		\node [style=none] (6) at (-1.5, -0.5) {};
		\node [style=none] (7) at (0.25, -1) {};
		\node [style=none] (8) at (-1.5, 1) {};
		\node [style=none] (9) at (1.75, -1) {};
	\end{pgfonlayer}
	\begin{pgfonlayer}{edgelayer}
		\draw[line width=2pt] [in=0, out=-120, looseness=1.20] (1) to (2.center);
		\draw[line width=2pt] [in=0, out=120, looseness=1.20] (1) to (3.center);
		\draw[line width=2pt] (0.center) to (1);
		\draw[line width=2pt] (6.center) to (4);
		\draw[line width=2pt] [in=180, out=60, looseness=1.20] (4) to (5.center);
		\draw[line width=2pt] [in=180, out=-60, looseness=1.20] (4) to (7.center);
		\draw[line width=2pt] (3.center) to (8.center);
		\draw[line width=2pt] (7.center) to (9.center);
	\end{pgfonlayer}
\end{tikzpicture}
    }
  \end{aligned}
}
%%fakesubsubsection bimonoid

\newcommand{\bi}[1]
{
  \begin{aligned}
    \resizebox{#1}{!}{
\begin{tikzpicture}
	\begin{pgfonlayer}{nodelayer}
		\node [style=none] (0) at (-2, 0.75) {};
		\node [style=none] (1) at (-2, -0.5) {};
		\node [style=circ] (2) at (-1, 0.75) {};
		\node [style=circ] (3) at (-1, -0.5) {};
		\node [style=none] (4) at (0, 1.25) {};
		\node [style=none] (5) at (-0.25, 0.25) {};
		\node [style=none] (6) at (-0.25, -0) {};
		\node [style=none] (7) at (0, -1) {};
		\node [style=none] (8) at (0, 1.25) {};
		\node [style=none] (9) at (0.25, 0.25) {};
		\node [style=none] (10) at (0.25, -0) {};
		\node [style=none] (11) at (0, -1) {};
		\node [style=circ] (12) at (1, 0.75) {};
		\node [style=circ] (13) at (1, -0.5) {};
		\node [style=none] (14) at (2, 0.75) {};
		\node [style=none] (15) at (2, -0.5) {};
	\end{pgfonlayer}
	\begin{pgfonlayer}{edgelayer}
		\draw[line width=2pt] (0.center) to (2);
		\draw[line width=2pt] (1.center) to (3);
		\draw[line width=2pt] (12) to (14.center);
		\draw[line width=2pt] (13) to (15.center);
		\draw[line width=2pt] [in=180, out=60, looseness=1.20] (2) to (4.center);
		\draw[line width=2pt] [in=180, out=-60, looseness=1.20] (2) to (5.center);
		\draw[line width=2pt] [in=180, out=60, looseness=1.20] (3) to (6.center);
		\draw[line width=2pt] [in=180, out=-60, looseness=1.20] (3) to (7.center);
		\draw[line width=2pt] [in=120, out=0, looseness=1.20] (8.center) to (12);
		\draw[line width=2pt] [in=-120, out=0, looseness=1.20] (9.center) to (12);
		\draw[line width=2pt] [in=120, out=0, looseness=1.20] (10.center) to (13);
		\draw[line width=2pt] [in=-120, out=0, looseness=1.20] (11.center) to (13);
		\draw[line width=2pt] (4.center) to (8.center);
		\draw[line width=2pt] [in=150, out=-30, looseness=1.00] (5.center) to (10.center);
		\draw[line width=2pt] [in=-150, out=30, looseness=1.00] (6.center) to (9.center);
		\draw[line width=2pt] (7.center) to (11.center);
	\end{pgfonlayer}
\end{tikzpicture}
    }
  \end{aligned}
}

\newcommand{\bimultl}[1]
{
\begin{aligned}
    \resizebox{#1}{!}{
\begin{tikzpicture}
	\begin{pgfonlayer}{nodelayer}
		\node [style=circ] (0) at (1, -0) {};
		\node [style=circ] (1) at (0.125, -0) {};
		\node [style=none] (2) at (-1, 0.5) {};
		\node [style=none] (3) at (-1, -0.5) {};
	\end{pgfonlayer}
	\begin{pgfonlayer}{edgelayer}
		\draw[line width=2pt] (0.center) to (1.center);
		\draw[line width=2pt] [in=0, out=120, looseness=1.20] (1.center) to (2.center);
		\draw[line width=2pt] [in=0, out=-120, looseness=1.20] (1.center) to (3.center);
	\end{pgfonlayer}
      \end{tikzpicture}}
\end{aligned}
}

\newcommand{\bimultr}[1]
{
\begin{aligned}
    \resizebox{#1}{!}{
\begin{tikzpicture}
	\begin{pgfonlayer}{nodelayer}
		\node [style=none] (e) at (1, -0) {};
		\node [style=circ] (0) at (0.125, 0.5) {};
		\node [style=circ] (1) at (0.125, -0.5) {};
		\node [style=none] (2) at (-1, 0.5) {};
		\node [style=none] (3) at (-1, -0.5) {};
	\end{pgfonlayer}
	\begin{pgfonlayer}{edgelayer}
		\draw[line width=2pt] (2.center) to (0.center);
		\draw[line width=2pt] (3.center) to (1.center);
	\end{pgfonlayer}
      \end{tikzpicture}}
\end{aligned}
}

\newcommand{\bicomultl}[1]
{
\begin{aligned}
    \resizebox{#1}{!}{
\begin{tikzpicture}
	\begin{pgfonlayer}{nodelayer}
		\node [style=circ] (0) at (-1, -0) {};
		\node [style=circ] (1) at (-0.125, -0) {};
		\node [style=none] (2) at (1, 0.5) {};
		\node [style=none] (3) at (1, -0.5) {};
	\end{pgfonlayer}
	\begin{pgfonlayer}{edgelayer}
		\draw[line width=2pt] (0.center) to (1.center);
		\draw[line width=2pt] [in=180, out=60, looseness=1.20] (1.center) to (2.center);
		\draw[line width=2pt] [in=180, out=-60, looseness=1.20] (1.center) to (3.center);
	\end{pgfonlayer}
      \end{tikzpicture}}
\end{aligned}
}

\newcommand{\bicomultr}[1]
{
\begin{aligned}
    \resizebox{#1}{!}{
\begin{tikzpicture}
	\begin{pgfonlayer}{nodelayer}
		\node [style=none] (e) at (-1, -0) {};
		\node [style=circ] (0) at (-0.125, 0.5) {};
		\node [style=circ] (1) at (-0.125, -0.5) {};
		\node [style=none] (2) at (1, 0.5) {};
		\node [style=none] (3) at (1, -0.5) {};
	\end{pgfonlayer}
	\begin{pgfonlayer}{edgelayer}
		\draw[line width=2pt] (0.center) to (2.center);
		\draw[line width=2pt] (1.center) to (3.center);
	\end{pgfonlayer}
      \end{tikzpicture}}
\end{aligned}
}
%%fakesubsubsection speciality

\newcommand{\spec}[1]
{
  \begin{aligned}
    \resizebox{#1}{!}{
\begin{tikzpicture}
	\begin{pgfonlayer}{nodelayer}
		\node [style=none] (0) at (1.75, -0) {};
		\node [style=circ] (1) at (0.75, -0) {};
		\node [style=none] (2) at (0, -0.5) {};
		\node [style=none] (3) at (0, 0.5) {};
		\node [style=circ] (4) at (-0.75, -0) {};
		\node [style=none] (5) at (0, -0.5) {};
		\node [style=none] (6) at (-1.75, -0) {};
		\node [style=none] (7) at (0, 0.5) {};
	\end{pgfonlayer}
	\begin{pgfonlayer}{edgelayer}
		\draw[line width=2pt] (0.center) to (1.center);
		\draw[line width=2pt] [in=0, out=-120, looseness=1.20] (1.center) to (2.center);
		\draw[line width=2pt] [in=0, out=120, looseness=1.20] (1.center) to (3.center);
		\draw[line width=2pt] (6.center) to (4);
		\draw[line width=2pt] [in=180, out=-60, looseness=1.20] (4) to (5.center);
		\draw[line width=2pt] [in=180, out=60, looseness=1.20] (4) to (7.center);
	\end{pgfonlayer}
\end{tikzpicture}
    }
  \end{aligned}
}

\newcommand{\extral}[1]
{
  \begin{aligned}
    \resizebox{#1}{!}{
\begin{tikzpicture}
	\begin{pgfonlayer}{nodelayer}
		\node [style=none] (0) at (1.75, -0) {};
		\node [style=circ] (1) at (0.75, -0) {};
		\node [style=circ] (4) at (-0.75, -0) {};
		\node [style=none] (6) at (-1.75, -0) {};
	\end{pgfonlayer}
	\begin{pgfonlayer}{edgelayer}
	  \draw[line width=2pt] (1.center) to (4.center);
	\end{pgfonlayer}
\end{tikzpicture}
    }
  \end{aligned}
}

\newcommand{\extrar}[1]
{
  \begin{aligned}
    \resizebox{#1}{!}{
\begin{tikzpicture}
	\begin{pgfonlayer}{nodelayer}
		\node [style=none] (0) at (1.75, -0) {};
		\node [style=none] (6) at (-1.75, -0) {};
	\end{pgfonlayer}
\end{tikzpicture}
    }
  \end{aligned}
}






\title{The Algebra of Open and \\[1ex]
	Interconnected Systems}   %note \\[1ex] is a line break in the title
\author{Brendan Fong}     
\college{Hertford College}
%\renewcommand{\submittedtext}{change the default text here if needed}
\degree{Doctor of Philosophy in Computer Science} 
\degreedate{Trinity 2016}    

\begin{document}
\baselineskip=18pt plus1pt
\setcounter{secnumdepth}{3} %set the number of sectioning levels that get number and appear in the contents
\setcounter{tocdepth}{3}

\maketitle                 
\pagestyle{empty}
%
\begin{quotation}
\textit{For all those who have prepared food so I could eat and created homes so
  I could live over the past four years. You too have laboured to produce this; I
hope I have done your labours justice.}
\end{quotation}

 
%\section*{Acknowledgements}

My supervisors
Bob Coecke
John Baez
Rob Ghrist

Jamie Vicary

My coauthors
Brandon Coya
Hugo Nava-Kopp
Blake Pollard
Pawe\l Soboc\'inski
Paolo Rapisarda

Writing adviser Rashmi Kumar

My fellow mathematicians and friends
Ross Atkins
Fiona Skerman
Kati Cohn-Gordon
Greg Henselman
GRST 

Shiori Shakuto
Bahar Tuncgenc
Mustak Ayub
Adam Berrington


Stephanie Lin

Mum, Dad, Justin, Calvin,

  
\begin{abstract}
(300 words)
\end{abstract}
 

\pagestyle{plain}
\begin{romanpages} 
\tableofcontents
\phantomsection
\addcontentsline{toc}{chapter}{Layintroduction}
\chapter*{Layintroduction}

\textit{This section is for all the family, friends, and strangers who have ever
asked me what is it that I actually do, only to come away more confused than
before.}

Welcome! Thanks for opening up my thesis. I'm aware that most people reading
this will never make it past the introduction, and a number will never really
make it through the introduction. (That's okay---I get stuck at the introduction
more often than not too.) These first two-and-a-half pages are a race to sneak
you an impression of my goals for this thesis before the covers are closed. 

Every thesis starts with a question. The question here, in the vague form that
occupies my dreams, is simply `Why do we use diagrams?'. More precisely, I'm
after an `algebra of interconnection' (see the title), whatever that means.
(Read on to find out =).)

Every thesis has a field, and this field tells us what an answer looks like. I
won't try to make my question more precise here, but let me quickly say a few
words about what the answer looks like. This is what I hear the cab-driver 
ask when she says `How do you do research in mathematics? Isn't it all solved
already?'.

In my corner of mathematics the main theme is abstraction and unification. Let
me tell a quick history lesson my supervisor John Baez once told me, from the
very beginning of mathematics. How did we invent number systems?

With civilization came trade, and with trade a need for writing down quantities.
By the 4th millennium BCE, Sumerian innovators had responded admirably to this
task, and many bureaucrats were now skilled in drawing up contracts trading
livestock, or sheaves of wheat, or jars of oil. But there was a problem: each
community had arrived at different ways to represent numbers. So the man
experienced in trading wheat had no ability to describe quantities of cows.It
took another thousand years to unify these systems into single concept of 

This looked like this.\footnote{You can read more about this here.}

Finding the right representation took us all many more years, but was important:
the Arabic system is much easier for many purposes (try to describe addition or
multiplication!) than the Roman system (I, II, III, IV, V) or the Chinese system
(.) among others.

By now this is a pattern we know well. 

Diagrams! Are just the same. In the abstract I mentioned a bunch of different
types. What is the unified framework? 

Language. networks have `words' (like a battery or resistor), and a `grammar':
this is a well-formed circuit diagram, while this is not. Networks also have
meaning: each well-formed circuit diagram can be understood as, well, a circuit.
So part of this project is a linguistic one, in this sense.

The dream here is to contribute to the invention of richer reasoning systems in
the same vein.

Circuit diagrams.

Why bother to do this? Because I really enjoy it: it's fun to understand and
beautiful to see how things fit together. (I hope those who read further agree!)
But I'm optimistic that once we understand why we choose to represent certain
things as networks, and how all our different 

Let me conclude with three different perspectives on where this might lead:

\begin{itemize}
  \item Mathematical physicist Eugene Wigner once described an unreasonable effectiveness of mathematics (see
citation). Biologists hardly feel the same.
Why? Reductionist themes in science and mathematics. Networks and interconnection are better suited to biology.

\item Causal thinking. 
People really like functions. They are causal, they are operational, they situate ourselves in the mathematics: if we do this, then that will happen. Input--output. They are also unphysical. Russell, Willems, Dijkstra have all been skeptical of causality. So what do we do? Relations.

\item Where will it go? Formal tools for reasoning about systems and their
  interconnection.  richer structures for interconnection, and reasoning about
  interconnection Programming. The pioneering computer scientist E.\ W.\
  Dijkstra: languages are too easy to write, not easy enough to read. Coding is
  99\% debugging. \footnote{Citation On the cruelty of really teaching computer
  science.}
\end{itemize}

But we must take small steps towards these grand ideas. What follows is just one
small contribution.

Lastly, some credit to Piper Harron\footnote{blog \cite{Har15}} and her delightful
thesis---this layintroduction is inspired by her `laysplanations'. I want to also acknowledge that
despite the (hopefully) polished final product, this thesis has been the result
of years of procrastination, feelings of inadequacy, blank staring at a page, 

While I will refrain from a running commentary on mathematical culture in my
thesis, let me echo her: mathematics is a human activity that we are all capable
of, and in a better world technical details and jargon and bad experiences would
not scare us all away. The key ingredients to understanding are curiosity,
patience, and perseverence. Smash the cult of genius! 


\phantomsection
\addcontentsline{toc}{chapter}{Introduction}
\chapter*{Introduction}

This is a thesis in the mathematical sciences, with emphasis on the mathematics.
But before we get to the category theory, I want to say a few, brief words about
the scientific tradition this thesis draws from.

Mathematics is the language of science. New science, new scientific paradigms,
requires new language.

This dissertation represents the beginnings of an attempt at a category
theoretic framework for general systems theory. A loosely organised body of
research dating back to biologist Ludwig von Bertalanffy in the mid-20th
century \cite{Ber}, general systems theory represents the study of systems built
from rich interconnections of simple component systems. Our central example in
this dissertation will be passive linear circuits: systems built from linear
resistors, inductors, and capacitors.  While each component here is simple,
networks built from such components are complex enough to form the foundation of
modern electronics.

General system theory seeks to talk about composition
\begin{quotation}
  While in the past, science tried to explain observable phenomena by reducing
  them to an interplay of elementary units investigable independently of each
  other, conceptions appear in contemporary science that are concerned with what
  is somewhat vaguely termed `wholeness', i.e. problems of organization,
  phenomena not resolvable into local events, dynamic interactions manifest in
  difference of behaviour of parts when isolated or in a higher configuration,
  etc.; in short, `systems' of various order not understandable by investigation
  of their respective parts in isolation. 
\end{quotation}

%http://vhpark.hyperbody.nl/images/a/aa/Bertalanffy-The_Theory_of_Open_Systems_in_Physics_and_Biology.pdf

From closed systems to open
\begin{quotation}
  Conventional physics deals only with closed systems, i.e. systems which are
  considered to be isolated from their environment.

  However, we find systems which by their very nature and definition are not
  closed systems. Every living organism is essentially an open system. It
  maintains itself in a continuous inflow and outflow, a building up and
  breaking down of components, never being, so long as it is alive, in a state
  of chemical and thermodynamic equilibrium but maintained in a so-called steady
  state which is distinct from the latter.  
\end{quotation}

It has always had a unification bent to it \cite{Ber50}

\begin{quotation}
  Not only are general aspects and viewpoints alike in different sciences;
  frequently we find formally identical or isomorphic laws in different fields.
  In many cases, isomorphic laws hold for certain classes or subclasses of
  'systems', irrespective of the nature of the entities involved. There appear
  to exist general system laws which apply to any system of a certain type,
  irrespective if the particular properties of the system and of the elements
  involved.
\end{quotation}
We will return to these ideas about isomorphic laws in Chapter 5.

This vision extended into the social sciences and humanities, influencing
economics, urban planning, sociology, psychology, management, philosophy. Eg
Forrester. These aspects of system theory are beyond the scope of the present
work.

Part of this vision, most notably in von Bertalanffy's home field of
systems biology, has been realised through 


    ``Systems biology...is about putting together rather than taking apart,
    integration rather than reduction. It requires that we develop ways of
    thinking about integration that are as rigorous as our reductionist
    programmes, but different....It means changing our philosophy, in the full
    sense of the term'' (Denis Noble).

Feedback; Wiener's cybernetics was often viewed as identical in agenda, 

More technical progress has been realised: cybernetics, catastrophe theory,
chaos theory, complex systems, network analysis

The goal here is to investigate these ideas from an algebraic perspective,
creating structural and compositional techniques for modelling
interconnection, open systems, and formal analogies, or isomorphisms, between
different fields.

\paragraph{A diagrammatic approach.}

In this program Odum, diagrams. universal language SBGN, UML

what is the algebra of network languages?

hypergraph categories.

graph example

black-boxing, information compression.

examples examples.

Example: trijunction

signal flow

circuits

automata, markov processes, flow networks.

The Yoneda lemma and representability.


Syntactically, by system we mean a `box' with finitely many `ports' through which it
interfaces with the external world. These ports may be of different types. These
systems may be connected together, along ports of the same type, to form larger
systems. Examples of such systems abound; a motivating source of them is
network-style diagrammatic languages, such as the aforementioned electrical
circuit diagrams, but also including chemical reaction networks, Petri nets,
automata, and Markov processes.

We capture this syntactic conception of a system is formally through the notion
of a hypergraph category---a symmetric monoidal category in which every object
is equipped with a special commutative Frobenius monoid in a way compatible with
the monoidal product. This framework provides precise language for describing
how we manipulate and interact with systems.


But what \emph{is} a system, and what do these manipulations and
interconnections mean? Taking inspiration from Willems \cite{Wi} and Deutsch
\cite{D} among others, we consider a system to be merely the set of all possible
different observations one might make by measuring all relevant variables at all
ports of the `box' we use to represent it.  We call this set the
\emph{behaviour} of the system. This arises from a view of physical laws as a
mechanism for simply partitioning the set of all trajectories of a system, the
so-called \emph{universum}, into possible and impossible trajectories.

For example, consider a resistor of resistance $r$. This has two ports---the two
ends of the resistor---and at each port we may measure the potential, and the
current flowing into the port. Now the resistor is governed by Kirchhoff's
current law, or conservation of charge, and Ohm's law. Conservation of charge
states that the current flowing into one port must equal the current flowing out
of the other port, while Ohm's law states that this current will be proportional
to the potential difference, with constant of proportionality $1/r$. Thus the
behaviour of the resistor is the set 
\[
  \big\{\big(\phi_1,\phi_2,
    -\tfrac1r(\phi_2-\phi_1),\tfrac1r(\phi_2-\phi_1)\big)\,\big\vert\,
    \phi_1,\phi_2 \in \mathbb{R}\big\}.
\]
Here the universum is the set
$\mathbb{R}\oplus\mathbb{R}\oplus\mathbb{R}\oplus\mathbb{R}$, where the
summands represent respectively the potentials and currents at each of the two
terminals.

% universum $\mathcal U$ of trajectories, behaviour $\mathcal P(\mathcal U)$,
% principle $\mathcal P(\mathcal P(\mathcal U))$,

The variables associated to each port are determined by the type of the port.
Interconnection of ports then, simply asserts the identification of the
variables at the connected ports. On the level of behaviours, this becomes a
generalised version of composition of relations.

In this thesis I argue that this general framework has wide applicability to
applied science and engineering, and with appropriate additional mathematical
tools allows one to formalise certain diagrammatic languages and their
relationships.

The central message of decorated corelations is that hypergraph structure
requires some sort of uniformity in the composition rule, and it is easier to
work by acknowledging this structure, defining them as algebras over some
theory. But we can weaken this notion of theory.

Influence of computer science. syntax semantics. rewrite rules/local rather than
global analysis.

Inspired by Coecke Abramsky CQM

\section{Outline}
Chapter 1 introduces hypergraph categories. Then decorated cospans and decorated
corelations. Colimits?

Then applications. First to control theory from this behavioural perspective,
demonstrating the ideas of black boxing and corelations, open systems and
control by interconnection. We give a new characterisation of controllability.
Second to passive linear networks, which speaks to the idea that we can find
isomorphisms across disciplines, and explore the insights in new ways.

\section{Related work}

Category-theoretic frameworks for general systems theory have been developed
before. Notably Goguen and Rosen led efforts. Goguen from a more computer
science perspective, Rosen more in biology.



Spivak 

Zanasi, Soboc\'inski, Bonchi

Most similar is the work of Walters, together with Sabadini and Rosebrugh. In
particular, automata.

\section{Contributions of collaborators}
%Where some part of the thesis is not solely the work of the candidate or has been carried out in collaboration with one or more persons, the candidate shall submit a clear statement of the extent of his or her own contribution.

The first three chapters are my own work. The applications chapters were
developed with collaborators. Chapter 4 arises from a weekly seminar with Paolo
Rapisarda and Pawe\l\ Soboc\'inski. I developed the corelation formalism and provided a
first draft of the paper. Pawe\l\ provided much expertise in signal flow graphs,
significantly revising the text and contributing the section on operational
semantics. Paolo contributed comparisons to classical methods in control theory. 
Much of the text is taken from the paper \cite{FonRapSob16}.

Chapter 5 is joint work with John Baez. John proposed the topic of research, and
supplied some notes on Ohm's law and the principle of minimum power. I
was responsible for producing the first draft of the chapter from this start.
John guided and assisted revisions of this text for publication \cite{BaeFon16}.

I thank my collaborators.

\end{romanpages}

\pagestyle{fancy}
\part{Mathematical Foundations}

\chapter[Hypergraph categories: the algebra of interconnection]{Hypergraph
categories: the algebra of interconnection} \label{ch.hypcats}

In this chapter we introduce hypergraph categories, giving a definition,
coherence theorem, and graphical language. We then explore a fundamental example
of hypergraph categories: categories of cospans.

We assume basic familiarity with category theory and symmetric monoidal
categories; although we give a sparse overview of the latter for reference. A
proper introduction to both can be found in Mac Lane \cite{Mac98}.

\section{The algebra of interconnection}

Our aim is to algebraicise network diagrams. A network diagram is built from
pieces like so:
\[
  \begin{tikzpicture}
    \node [thick, circle, draw] (0) at (0, -0) {};
    \node [style=none] (1) at (-0.75, 1.5) {};
    \node [style=none] (2) at (-1.75, -0) {};
    \node [style=none] (3) at (-0.75, -1.75) {};
    \node [style=none] (4) at (1.25, -1.5) {};
    \node [style=none] (5) at (1.75, 0.75) {};
    \draw (0) to (5);
    \draw [dashed] (0) to (4);
    \draw (0) to (3);
    \draw [line width=2pt, draw=gray] (0) to (2);
    \draw (0) to (1);
  \end{tikzpicture}
\]
These represent open systems, concrete or abstract; for example a resistor, a
chemical reaction, or a linear transformation. The essential feature, for
openness and for networking, is that the system may have terminals, perhaps of
different `types', each one depicted by a line radiating from the central body.
In the case of a resistor each terminal might represent a wire, for chemical
reactions a chemical species, for linear transformations a variable in the
domain or codomain.  Network diagrams are formed by connecting terminals of
systems to build larger systems.

A network-style diagrammatic language is a collection of network diagrams
together with the stipulation that if we take some of these network diagrams,
and connect terminals of the same type in any way we like, then we form
another diagram in the collection.  The point of this chapter is that hypergraph
categories provide a precise formalisation of network-style diagrammatic
languages.  

\begin{figure}
\[
\begin{aligned}
\begin{tikzpicture}
	\begin{pgfonlayer}{nodelayer}
		\node [thick, circle, draw] (0) at (0, -0) {};
		\node [style=none] (1) at (-0.75, 1.5) {};
		\node [style=none] (2) at (-1.75, -0) {};
		\node [style=none] (3) at (-0.75, -1.75) {};
		\node [style=none] (4) at (1.25, -1.5) {};
		\node [style=none] (5) at (1.75, 0.75) {};
		\node [thick, circle, draw=gray, fill=gray] (6) at (-1, -4.25) {};
		\node [style=none] (7) at (-1, -2.5) {};
		\node [style=none] (8) at (0.25, -5) {};
		\node [style=none] (9) at (-2.25, -4.5) {};
		\node [style=none] (10) at (0.5, -2.75) {};
		\node [thick, circle, draw=gray, fill=gray] (11) at (3.5, -3.75) {};
		\node [style=none] (12) at (3.5, -2.25) {};
		\node [style=none] (13) at (2, -4) {};
		\node [style=none] (14) at (4.25, -5) {};
		\node [thick, circle, draw, fill=black] (15) at (4, -0) {};
		\node [style=none] (16) at (2.5, 0.75) {};
		\node [style=none] (17) at (2.5, -0) {};
		\node [style=none] (18) at (3.75, -1.75) {};
		\node [style=none] (19) at (2, -2.25) {};
	\end{pgfonlayer}
	\begin{pgfonlayer}{edgelayer}
		\draw (0) to (5);
		\draw [dashed] (0) to (4);
		\draw (0) to (3);
		\draw [line width=2pt, draw=gray] (0) to (2);
		\draw (0) to (1);
		\draw (6) to (7);
		\draw [line width=2pt, draw=gray] (6) to (9);
		\draw [dashed] (6) to (10);
		\draw [dotted] (6) to (8);
		\draw [dotted] (11) to (13);
		\draw [line width=2pt, draw=gray] (11) to (12);
		\draw (11) to (14);
		\draw [line width=2pt, draw=gray] (15) to (18);
		\draw (15) to (17);
		\draw (15) to (16);
		\draw [dashed] (11) to (19);
	\end{pgfonlayer}
\end{tikzpicture}
\end{aligned}
\qquad
\Rightarrow
\qquad
\begin{aligned}
\begin{tikzpicture}
	\begin{pgfonlayer}{nodelayer}
		\node [draw, circle, thick] (0) at (0, -0) {};
		\node [style=none] (1) at (-0.75, 1.5) {};
		\node [style=none] (2) at (-1.75, -0) {};
		\node [style=none] (3) at (1.25, -0.25) {};
		\node [draw=gray, circle, thick, fill=gray] (4) at (-0.25, -3) {};
		\node [style=none] (5) at (1.25, -3.5) {};
		\node [style=none] (6) at (-1.5, -3.25) {};
		\node [draw=gray, circle, thick, fill=gray] (7) at (2.75, -2.25) {};
		\node [style=none] (8) at (3.5, -3.5) {};
		\node [draw, circle, thick, fill=black] (9) at (2.5, -0.5) {};
		\node [style=none] (10) at (1, -1.75) {};
	\end{pgfonlayer}
	\begin{pgfonlayer}{edgelayer}
		\draw (0) to (3);
		\draw [line width=2pt, draw=gray] (0) to (2);
		\draw (0) to (1);
		\draw [line width=2pt, draw=gray] (4) to (6);
		\draw [dotted] (4) to (7);
		\draw (7) to (8);
		\draw [dashed] (0) to (10);
		\draw [dashed] (4) to (10);
		\draw [dashed] (7) to (10);
		\draw [bend right, looseness=1.25] (9) to (3);
		\draw [bend left, looseness=1.25] (9) to (3);
		\draw [line width=2pt, draw=gray] (9) to (7);
		\draw (0) to (4);
	\end{pgfonlayer}
\end{tikzpicture}
\end{aligned}
\]
\caption{Interconnection of network diagrams. Note that we only connect
terminals of the same type, but we can connect as many as we like.}
\end{figure}

In jargon, a hypergraph category is a symmetric monoidal category in
which every object is equipped with a special commutative Frobenius monoid in a
way compatible with the monoidal product. We will walk through these terms in
detail, illustrating them with examples and a few theorems. 

The key data comprising a hypergraph category are its objects, morphisms,
composition rule, monoidal product, and Frobenius maps. Each of these model a
feature of network diagrams and their interconnection. The objects model the
terminal types, while the morphisms model the network diagrams themselves. The
composition, monoidal product, and Frobenius maps model different aspects of
interconnection: composition models the interconnection of two terminals of the
same type, the monoidal product models the network formed by taking two networks
without interconnecting any terminals, while the Frobenius maps model
multi-terminal interconnection.

These Frobenius maps are the distinguishing feature of hypergraph categories as
compared to other structured monoidal categories, and are crucial for
formalising the intuitive concept of network languages detailed above. In the
case of electric circuits the Frobenius maps model the `branching' of wires; in
the case when diagrams simply model an abstract system of equations and
terminals variables in these equations, the Frobenius maps allow variables to be
shared between many systems of equations.

Examining these correspondences, it is worthwhile to ask whether hypergraph
categories permit too much structure to be specified, given that the
interconnection rule is now divided into three different aspects, and features
such as domains and codomains of network diagrams, rather than just a collection
of terminals, exist. The answer is given by examining the additional coherence
laws that these data must obey. For example, in the case of the domain and
codomain, we shall see that hypergraph categories are all compact closed
categories, and so there is ultimately only a formal distinction between domain
and codomain objects. One way to think of these data is as scaffolding. We could
compare it to the use of matrices and bases to provide language for talking
about linear transformations and vector spaces.  They are not part of the target
structure, but nonetheless useful paraphenalia for constructing it.

\begin{figure}
  \begin{center}
  \begin{tabular}{c|c}
    Networks & Hypergraph categories \\
    \hline 
    list of terminal types & object \\
    network diagram & morphism \\
    series connection & composition \\
    juxtaposition & monoidal product \\
    branching & Frobenius maps
  \end{tabular}
  \end{center}
  \caption{Corresponding features of networks and hypergraph categories.}
\end{figure}

Network languages are not only syntactic entities: as befitting the descriptor
`language', they typically have some associated semantics. Circuits diagrams, for
instance, not only depict wire circuits that may be constructed, they also
represent the electrical behaviour of that circuit. Such semantics considers the
circuits 
\[
  \begin{aligned}
  \begin{tikzpicture}[circuit ee IEC, set resistor graphic=var resistor IEC graphic]
    \node[contact] (I1) at (0,0) {};
    \node[circle, minimum width = 3pt, inner sep = 0pt, fill=black] (int) at (3,0) {};
    \node[contact] (O1) at (6,0) {};
    \draw (I1) 	to [resistor] node [label={[label distance=3pt]90:{$1 \Omega$}}] {} (int)
    to [resistor] node [label={[label distance=3pt]90:{$1 \Omega$}}] {} (O1);
  \end{tikzpicture}
  \end{aligned}
  \qquad
  \mbox{and}
  \qquad
  \begin{aligned}
  \begin{tikzpicture}[circuit ee IEC, set resistor graphic=var resistor IEC graphic]
    \node[contact] (I1) at (0,0) {};
    \node[contact] (O1) at (3,0) {};
    \draw (I1) 	to [resistor] node [label={[label distance=3pt]90:{$2 \Omega$}}]
    {} (O1);
  \end{tikzpicture}
  \end{aligned}
\]
the same, even though as `syntactic' diagrams they are distinct. A cornerstone
of the utility of the hypergraph formalism is the ability to also realise the
semantics of these diagrams as morphisms of another hypergraph category. This
`semantic' hypergraph category, as a hypergraph category, still permits the rich
`networking' interconnection structure, and a so-called hypergraph functor
implies that the syntactic category provides a sound framework for depicting
these morphisms. Network languages syntactically are often `free' hypergraph
categories, and much of the interesting structure lies in their functors to
their semantic hypergraph categories.


\section{Symmetric monoidal categories}
Suppose we have some tiles with inputs and outputs of various types like so:
\[
    \tikzset{every path/.style={line width=1.1pt}}
  \begin{tikzpicture}
	\begin{pgfonlayer}{nodelayer}
		\node [style=none] (0) at (-0.25, 0.375) {};
		\node [style=none] (1) at (0.5, 0.375) {};
		\node [style=none] (2) at (-0.25, -1.375) {};
		\node [style=none] (3) at (0.5, -1.375) {};
		\node [style=none] (4) at (0.5, 0.25) {};
		\node [style=none] (5) at (0.5, -1.25) {};
		\node [style=none] (6) at (1.25, 0.25) {};
		\node [style=none] (7) at (1.25, -1.25) {};
		\node [style=none] (8) at (0.125, -0.5) {$f$};
		\node [style=none] (9) at (1.5, 0.25) {$Y_1$};
		\node [style=none] (10) at (1.5, -1.25) {$Y_m$};
		\node [style=none] (11) at (1.25, -0.25) {};
		\node [style=none] (12) at (1.5, -0.25) {$Y_2$};
		\node [style=none] (13) at (0.5, -0.25) {};
		\node [style=none] (14) at (1, -0.75) {$\vdots$};
		\node [style=none] (15) at (-1, -1.25) {};
		\node [style=none] (16) at (-0.25, -1.25) {};
		\node [style=none] (17) at (-0.75, -0.75) {$\vdots$};
		\node [style=none] (18) at (-1.25, -1.25) {$X_n$};
		\node [style=none] (19) at (-0.25, -0.25) {};
		\node [style=none] (20) at (-1.25, 0.25) {$X_1$};
		\node [style=none] (21) at (-1, 0.25) {};
		\node [style=none] (22) at (-0.25, 0.25) {};
		\node [style=none] (23) at (-1, -0.25) {};
		\node [style=none] (24) at (-1.25, -0.25) {$X_2$};
	\end{pgfonlayer}
	\begin{pgfonlayer}{edgelayer}
		\draw (0.center) to (1.center);
		\draw (1.center) to (3.center);
		\draw (3.center) to (2.center);
		\draw (2.center) to (0.center);
		\draw (4.center) to (6.center);
		\draw (5.center) to (7.center);
		\draw (13.center) to (11.center);
		\draw (22.center) to (21.center);
		\draw (16.center) to (15.center);
		\draw (19.center) to (23.center);
	\end{pgfonlayer}
\end{tikzpicture}
\]
These tiles may vary in height and width. We can place these tiles above and
below each other, and to the left and right, so long as the inputs on the right
tile match the outputs on the left. Suppose also that some arrangements of tiles
are equal to other arrangements of tiles. How do we formalise this structure
algebraically? The theory of monoidal categories provides an answer. 

Hypergraph categories are first monoidal categories, indeed symmetric monoidal
categories.  Monoidal categories are categories with two notions of composition:
ordinary categorical composition and monoidal composition, with the monoidal
composition only associative and unital up to natural isomorphism. They are the
algebra of processes that may occur simultaneously as well as sequentially.
First defined by B\'enabou and Mac Lane in the 1960s \cite{Ben63, Mac63}, their
theory and their links with graphical representation are well explored
\cite{JS91, Sel11}. We bootstrap on this, using monoidal categories to define
hypergraph categories, and so immediately arriving at an understanding of how
hypergraph categories formalise our network languages. 

Moreover, symmetric monoidal functors play a key role in our framework for
defining and working with hypergraph categories: decorated cospans and
corelations constructions. For this reason we provide, for quick reference, a
definition of symmetric monoidal categories.


\subsection{Monoidal categories}
A \define{monoidal category} $(\c, \ot)$ consists of a category $\c$, a
functor $\ot: \c \times \c \to \c$, a distinguished object $I$, and natural
isomorphisms $\a_{A,B,C}: (A \ot B) \ot C \to A \ot (B \ot C)$,
$\rho_A: A \ot I  \to A$, and $\lambda_A: I \ot A \to A$ such that for all
$A,B,C,D$ in $\mc C$ the following two diagrams commute: 
\[
  \xymatrixcolsep{3pc}
  \xymatrix{
    \big((A \ot B) \ot C\big) \ot D \ar[d]_{\a_{A,B,C}\ot\idn_D} \ar[rr]^{\a_{(A\ot B),C,D}} 
    &&(A \ot B) \ot (C \ot D) \ar[d]^{\a_{A,B,(C\ot D)}} \\
    \big(A \ot (B\ot C)\big) \ot D \ar[r]_{\a_{A,(B\ot C),D}} 
    & A\ot\big((B \ot C)\ot D\big)\ar[r]_{\idn_A \ot \a_{B,C,D}}
    &A \ot \big(B \ot (C \ot D)\big)
  }
\]
\[
  \xymatrix{
    (A\ot I)\ot B  \ar[rr]^{\a_{A,I,B}} \ar[dr]_{\rho_{A}\ot \idn_B} && A \ot (I \ot B) \ar[dl]^{\idn_A\ot \lambda_B}\\
    & A \ot B \\
  }
\]
We call $\ot$ the \define{monoidal product}, $I$ the \define{monoidal unit},
$\alpha$ the \define{associator}, $\rho$ and $\lambda$ the \define{right} and
\define{left unitor} respectively. The associator and unitors are known
collectively as the \define{coherence maps}.

By Mac Lane's coherence theorem, these two axioms are equivalent to requiring
that `all formal diagrams'---that is, all diagrams in which the morphism are
built from identity morphisms and the coherence maps using composition and the
monoidal product---commute. Consequently, between any two products of the same
ordered list of objects up to instances of the monoidal unit, such as $((A \ot
I) \ot B) \ot C$ and $A \ot ((B \ot C) \ot (I \ot I))$, there is a unique
so-called \define{canonical} map. See Mac Lane \cite[Corollary of Theorem
VII.2.1]{Mac98} for a precise statement and proof.

A \define{lax monoidal functor} $(F, \varphi): (\c,\otimes) \to (\c',\boxtimes)$
between monoidal categories consists of a functor $F: \c \to \c'$, and natural
transformations $\varphi_{A,B}: FA \boxtimes FB \to F(A \ot B)$ and $\varphi_1:
1_{\c'} \to F1_{\c}$, such that for all $A,B,C \in \c$ the three diagrams
\[
  \xymatrixcolsep{4pc}
  \xymatrix{
    (FA \ot FB) \ot FC \ar[d]_{\a_{FA,FB,FC}} \ar[r]^{\varphi_{A,B} \ot \idn_{FC}} &
    F(A \ot B) \ot FC \ar[r]^{\varphi_{A\ot B,C}} & F((A \ot B) \ot C) \ar[d]^{F\a_{A,B,C}}\\
    FA \ot (FB \ot FC) \ar[r]_{\idn_{FA} \ot \varphi_{B,C}} & FA \ot F(B \ot C)
    \ar[r]_{\varphi_{A,B\ot C}} & F(A \ot (B \ot C))
  }
\]
\[
  \xymatrixcolsep{3pc}
  \xymatrixrowsep{3pc}
  \xymatrix{
    F(A) \ot I' \ar[d]_{\idn \ot \varphi_1} \ar[r]^{\rho} & F(A) \\
    F(A) \ot F(I) \ar[r]_{\varphi_{A,I}} & F(A \ot I) \ar[u]_{F\rho} 
  }
  \qquad
  \xymatrix{
    I' \ot F(A) \ar[d]_{\varphi_1 \ot \idn} \ar[r]^{\lambda} & F(A) \\
    F(I) \ot F(A) \ar[r]_{\varphi_{I,A}} & F(I \ot A) \ar[u]_{F\lambda} 
  }
\]
commute. We further say a monoidal functor is a \define{strong monoidal functor}
if the $\varphi$ are isomorphisms, and a \define{strict monoidal functor} if the
$\varphi$ are identities. 

A \define{monoidal natural transformation} $\theta: (F,\varphi) \Rightarrow
(G,\gamma)$ between two monoidal functors $F$ and $G$ is a natural
transformation $\theta: F \Rightarrow G$ such that
\[
  \begin{aligned}
    \xymatrix{
      F1_{\c} \ar[rr]^{\theta_I}&& G1_{\c} \\
      & 1_{\c'} \ar[ul]^{\varphi_1} \ar[ur]_{\gamma_1}
    } 
  \end{aligned} 
  \qquad 
  \mbox{and}
  \qquad
  \begin{aligned}
    \xymatrixcolsep{3pc}
    \xymatrixrowsep{3pc}
    \xymatrix{
      FA \boxtimes FB \ar[r]^{\theta_A \ot \theta_B} \ar[d]_{\varphi_{A,B}} 
      & GA \boxtimes GB \ar[d]^{\gamma_{A,B}}\\
      F(A \ot B) \ar[r]_{\theta_{A\ot B}} & G(A \ot B)
    }
  \end{aligned} 
\]
commute for all objects $A,B$.

Two monoidal categories $\mc C, \mc D$ are \define{monoidally equivalent} if
there exist strong monoidal functors $F\maps \mc C \to \mc D$ and $G\maps \mc D
\to \mc C$ such that the composites $FG$ and $GF$ are monoidally naturally
isomorphic to the identity functors. (Note that identity functors are
immediately strict monoidal functors.)

\subsection{String diagrams}
A \define{strict monoidal category} category is a monoidal category in which the
associators and unitors are all identity maps. In this case then any two objects
that can be related by associators and unitors are equal, and so we may write
objects without parentheses and units without ambiguity. An equivalent statement
of Mac Lane's coherence theorem is that every monoidal category is monoidally
equivalent to strict monoidal category. 

Yet another equivalent statement of the coherence theorem is the existence of a
graphical calculus for monoidal categories. As discussed above, monoidal
categories figure strongly in our current investigations precisely because of
this. We leave the details to discussions elsewhere. The main point is that we
shall be free to assume our monoidal categories are strict, writing $X_1 \otimes
X_2 \otimes \dots \otimes X_n$ for objects in $(\mathcal C,\otimes)$ without a
care for parentheses. We then depict a morphism $f\maps X_1 \otimes X_2 \otimes
\dots \otimes X_n \to Y_1 \otimes Y_2 \otimes \dots \otimes Y_n$ with the
diagram:
\[
  f \quad = \quad
  \begin{aligned}
    \tikzset{every path/.style={line width=1.1pt}}
  \begin{tikzpicture}
	\begin{pgfonlayer}{nodelayer}
		\node [style=none] (0) at (-0.25, 0.375) {};
		\node [style=none] (1) at (0.5, 0.375) {};
		\node [style=none] (2) at (-0.25, -1.375) {};
		\node [style=none] (3) at (0.5, -1.375) {};
		\node [style=none] (4) at (0.5, 0.25) {};
		\node [style=none] (5) at (0.5, -1.25) {};
		\node [style=none] (6) at (1.25, 0.25) {};
		\node [style=none] (7) at (1.25, -1.25) {};
		\node [style=none] (8) at (0.125, -0.5) {$f$};
		\node [style=none] (9) at (1.5, 0.25) {$Y_1$};
		\node [style=none] (10) at (1.5, -1.25) {$Y_m$};
		\node [style=none] (11) at (1.25, -0.25) {};
		\node [style=none] (12) at (1.5, -0.25) {$Y_2$};
		\node [style=none] (13) at (0.5, -0.25) {};
		\node [style=none] (14) at (1, -0.75) {$\vdots$};
		\node [style=none] (15) at (-1, -1.25) {};
		\node [style=none] (16) at (-0.25, -1.25) {};
		\node [style=none] (17) at (-0.75, -0.75) {$\vdots$};
		\node [style=none] (18) at (-1.25, -1.25) {$X_n$};
		\node [style=none] (19) at (-0.25, -0.25) {};
		\node [style=none] (20) at (-1.25, 0.25) {$X_1$};
		\node [style=none] (21) at (-1, 0.25) {};
		\node [style=none] (22) at (-0.25, 0.25) {};
		\node [style=none] (23) at (-1, -0.25) {};
		\node [style=none] (24) at (-1.25, -0.25) {$X_2$};
	\end{pgfonlayer}
	\begin{pgfonlayer}{edgelayer}
		\draw (0.center) to (1.center);
		\draw (1.center) to (3.center);
		\draw (3.center) to (2.center);
		\draw (2.center) to (0.center);
		\draw (4.center) to (6.center);
		\draw (5.center) to (7.center);
		\draw (13.center) to (11.center);
		\draw (22.center) to (21.center);
		\draw (16.center) to (15.center);
		\draw (19.center) to (23.center);
	\end{pgfonlayer}
\end{tikzpicture}.
\end{aligned}
\]
Identity morphisms are depicted by `wires':
\[
  \idn_X \quad = \quad
  \begin{aligned}
    \tikzset{every path/.style={line width=1.1pt}}
\begin{tikzpicture}
	\begin{pgfonlayer}{nodelayer}
		\node [style=none] (0) at (1.25, 0.25) {};
		\node [style=none] (1) at (1.5, 0.25) {$X$};
		\node [style=none] (2) at (-1.25, 0.25) {$X$};
		\node [style=none] (3) at (-1, 0.25) {};
	\end{pgfonlayer}
	\begin{pgfonlayer}{edgelayer}
		\draw (3.center) to (0.center);
	\end{pgfonlayer}
\end{tikzpicture}
\end{aligned}
\]
and the monoidal unit is not depicted at all:
\[
\idn_I\quad = \quad
  \begin{aligned}
    \tikzset{every path/.style={line width=1.1pt}}
\begin{tikzpicture}
		\node [style=none] (1) at (1.5, 0.25) {};
		\node [style=none] (2) at (-1.25, 0.25) {};
\end{tikzpicture}
\end{aligned}
\]
Composition of morphisms is depicted by connecting the relevant `wires':
\[
    \tikzset{every path/.style={line width=1.1pt}}
  \begin{aligned}
    \begin{tikzpicture}
	\begin{pgfonlayer}{nodelayer}
		\node [style=none] (0) at (0.25, -0) {$Y_1$};
		\node [style=none] (1) at (0.5, -0) {};
		\node [style=none] (2) at (2.75, -0.75) {};
		\node [style=none] (3) at (3, -0.75) {$Y_2$};
		\node [style=none] (4) at (0.5, -0.75) {};
		\node [style=none] (5) at (2, -0.365) {};
		\node [style=none] (6) at (2.75, 0.25) {};
		\node [style=none] (7) at (3, 0.25) {$Z_1$};
		\node [style=none] (8) at (1.25, 0.375) {};
		\node [style=none] (9) at (1.25, -0) {};
		\node [style=none] (10) at (1.25, -0.365) {};
		\node [style=none] (11) at (2, 0.25) {};
		\node [style=none] (12) at (2, 0.375) {};
		\node [style=none] (13) at (1.625, -0) {$g$};
		\node [style=none] (14) at (2, -0.25) {};
		\node [style=none] (15) at (2.75, -0.25) {};
		\node [style=none] (16) at (3, -0.25) {$Z_2$};
		\node [style=none] (17) at (0.25, -0.75) {$Y_2$};
	\end{pgfonlayer}
	\begin{pgfonlayer}{edgelayer}
		\draw (4.center) to (2.center);
		\draw (8.center) to (12.center);
		\draw (5.center) to (10.center);
		\draw (11.center) to (6.center);
		\draw (12.center) to (5.center);
		\draw (10.center) to (8.center);
		\draw (14.center) to (15.center);
		\draw (1.center) to (9.center);
	\end{pgfonlayer}
\end{tikzpicture}
\end{aligned}
  \circ
  \begin{aligned}
  \begin{tikzpicture}
	\begin{pgfonlayer}{nodelayer}
		\node [style=none] (0) at (-0.25, 0.375) {};
		\node [style=none] (1) at (0.5, 0.375) {};
		\node [style=none] (2) at (-0.25, -.875) {};
		\node [style=none] (3) at (0.5, -.875) {};
		\node [style=none] (4) at (0.5, 0.125) {};
		\node [style=none] (5) at (1.25, 0.125) {};
		\node [style=none] (6) at (0.125, -0.25) {$f$};
		\node [style=none] (7) at (1.5, 0.125) {$Y_1$};
		\node [style=none] (8) at (1.25, -0.625) {};
		\node [style=none] (9) at (1.5, -0.625) {$Y_2$};
		\node [style=none] (10) at (0.5, -0.625) {};
		\node [style=none] (11) at (-0.25, -0.75) {};
		\node [style=none] (12) at (-1.25, -0.75) {$X_3$};
		\node [style=none] (13) at (-0.25, -0.25) {};
		\node [style=none] (14) at (-1.25, 0.25) {$X_1$};
		\node [style=none] (15) at (-1, 0.25) {};
		\node [style=none] (16) at (-0.25, 0.25) {};
		\node [style=none] (17) at (-1, -0.25) {};
		\node [style=none] (18) at (-1.25, -0.25) {$X_2$};
		\node [style=none] (19) at (-1, -0.75) {};
	\end{pgfonlayer}
	\begin{pgfonlayer}{edgelayer}
		\draw (0.center) to (1.center);
		\draw (1.center) to (3.center);
		\draw (3.center) to (2.center);
		\draw (2.center) to (0.center);
		\draw (4.center) to (5.center);
		\draw (10.center) to (8.center);
		\draw (16.center) to (15.center);
		\draw (13.center) to (17.center);
		\draw (11.center) to (19.center);
	\end{pgfonlayer}
\end{tikzpicture}
\end{aligned}
\quad = \quad
\begin{aligned}
\begin{tikzpicture}
	\begin{pgfonlayer}{nodelayer}
		\node [style=none] (0) at (-0.25, 0.375) {};
		\node [style=none] (1) at (0.5, 0.375) {};
		\node [style=none] (2) at (-0.25, -0.875) {};
		\node [style=none] (3) at (0.5, -0.875) {};
		\node [style=none] (4) at (0.5, 0.125) {};
		\node [style=none] (5) at (0.125, -0.25) {$f$};
		\node [style=none] (6) at (2.75, -0.625) {};
		\node [style=none] (7) at (3, -0.625) {$Y_2$};
		\node [style=none] (8) at (0.5, -0.625) {};
		\node [style=none] (9) at (-0.25, -0.75) {};
		\node [style=none] (10) at (-1.25, -0.75) {$X_3$};
		\node [style=none] (11) at (-0.25, -0.25) {};
		\node [style=none] (12) at (-1.25, 0.25) {$X_1$};
		\node [style=none] (13) at (-1, 0.25) {};
		\node [style=none] (14) at (-0.25, 0.25) {};
		\node [style=none] (15) at (-1, -0.25) {};
		\node [style=none] (16) at (-1.25, -0.25) {$X_2$};
		\node [style=none] (17) at (-1, -0.75) {};
		\node [style=none] (18) at (2, -0.25) {};
		\node [style=none] (19) at (2.75, 0.375) {};
		\node [style=none] (20) at (3, 0.375) {$Z_1$};
		\node [style=none] (21) at (1.25, 0.5) {};
		\node [style=none] (22) at (1.25, 0.125) {};
		\node [style=none] (23) at (1.25, -0.25) {};
		\node [style=none] (24) at (2, 0.375) {};
		\node [style=none] (25) at (2, 0.5) {};
		\node [style=none] (26) at (1.625, 0.125) {$g$};
		\node [style=none] (27) at (2, -0.125) {};
		\node [style=none] (28) at (2.75, -0.125) {};
		\node [style=none] (29) at (3, -0.125) {$Z_2$};
	\end{pgfonlayer}
	\begin{pgfonlayer}{edgelayer}
		\draw (0.center) to (1.center);
		\draw (1.center) to (3.center);
		\draw (3.center) to (2.center);
		\draw (2.center) to (0.center);
		\draw (8.center) to (6.center);
		\draw (14.center) to (13.center);
		\draw (11.center) to (15.center);
		\draw (9.center) to (17.center);
		\draw (21.center) to (25.center);
		\draw (18.center) to (23.center);
		\draw (24.center) to (19.center);
		\draw (25.center) to (18.center);
		\draw (23.center) to (21.center);
		\draw (27.center) to (28.center);
		\draw (4.center) to (22.center);
	\end{pgfonlayer}
\end{tikzpicture}
\end{aligned}
\]
while monoidal composition is just juxtaposition:
\[
    \tikzset{every path/.style={line width=1.1pt}}
  \begin{aligned}
    \begin{tikzpicture}
	\begin{pgfonlayer}{nodelayer}
		\node [style=none] (0) at (-0.25, 0.375) {};
		\node [style=none] (1) at (0.5, 0.375) {};
		\node [style=none] (2) at (-0.25, -0.375) {};
		\node [style=none] (3) at (0.5, -0.375) {};
		\node [style=none] (4) at (0.5, -0) {};
		\node [style=none] (5) at (1.25, -0) {};
		\node [style=none] (6) at (0.125, -0) {$h$};
		\node [style=none] (7) at (1.5, -0) {$Y_1$};
		\node [style=none] (8) at (-0.25, -0.25) {};
		\node [style=none] (9) at (-1.25, 0.25) {$X_1$};
		\node [style=none] (10) at (-1, 0.25) {};
		\node [style=none] (11) at (-0.25, 0.25) {};
		\node [style=none] (12) at (-1, -0.25) {};
		\node [style=none] (13) at (-1.25, -0.25) {$X_2$};
	\end{pgfonlayer}
	\begin{pgfonlayer}{edgelayer}
		\draw (0.center) to (1.center);
		\draw (3.center) to (2.center);
		\draw (2.center) to (0.center);
		\draw (4.center) to (5.center);
		\draw (11.center) to (10.center);
		\draw (8.center) to (12.center);
		\draw (1.center) to (3.center);
	\end{pgfonlayer}
\end{tikzpicture}
  \end{aligned}
  \ot \quad
  \begin{aligned}
    \begin{tikzpicture}
	\begin{pgfonlayer}{nodelayer}
		\node [style=none] (0) at (1.5, -0.75) {$Y_2$};
		\node [style=none] (1) at (-0.25, -1.375) {};
		\node [style=none] (2) at (1.25, -0.75) {};
		\node [style=none] (3) at (1.5, -1.25) {$Y_3$};
		\node [style=none] (4) at (1.25, -1.25) {};
		\node [style=none] (5) at (0.125, -1) {$k$};
		\node [style=none] (6) at (0.5, -0.75) {};
		\node [style=none] (7) at (0.5, -1.375) {};
		\node [style=none] (8) at (-0.25, -0.625) {};
		\node [style=none] (9) at (0.5, -1.25) {};
		\node [style=none] (10) at (0.5, -0.625) {};
	\end{pgfonlayer}
	\begin{pgfonlayer}{edgelayer}
		\draw (8.center) to (10.center);
		\draw (6.center) to (2.center);
		\draw (9.center) to (4.center);
		\draw (10.center) to (7.center);
		\draw (7.center) to (1.center);
		\draw (8.center) to (1.center);
	\end{pgfonlayer}
\end{tikzpicture}
  \end{aligned}
  \quad = \quad
  \begin{aligned}
    \begin{tikzpicture}
	\begin{pgfonlayer}{nodelayer}
		\node [style=none] (0) at (-0.25, 0.375) {};
		\node [style=none] (1) at (0.5, 0.375) {};
		\node [style=none] (2) at (-0.25, -0.375) {};
		\node [style=none] (3) at (0.5, -0.375) {};
		\node [style=none] (4) at (0.5, -0) {};
		\node [style=none] (5) at (1.25, -0) {};
		\node [style=none] (6) at (0.125, -0) {$h$};
		\node [style=none] (7) at (1.5, -0) {$Y_1$};
		\node [style=none] (8) at (-0.25, -0.25) {};
		\node [style=none] (9) at (-1.25, 0.25) {$X_1$};
		\node [style=none] (10) at (-1, 0.25) {};
		\node [style=none] (11) at (-0.25, 0.25) {};
		\node [style=none] (12) at (-1, -0.25) {};
		\node [style=none] (13) at (-1.25, -0.25) {$X_2$};
		\node [style=none] (14) at (0.5, -0.75) {};
		\node [style=none] (15) at (1.25, -0.75) {};
		\node [style=none] (16) at (0.125, -1) {$k$};
		\node [style=none] (17) at (0.5, -0.625) {};
		\node [style=none] (18) at (-0.25, -1.375) {};
		\node [style=none] (19) at (1.5, -1.25) {$Y_3$};
		\node [style=none] (20) at (0.5, -1.25) {};
		\node [style=none] (21) at (-0.25, -0.625) {};
		\node [style=none] (22) at (1.25, -1.25) {};
		\node [style=none] (23) at (1.5, -0.75) {$Y_2$};
		\node [style=none] (24) at (0.5, -1.375) {};
	\end{pgfonlayer}
	\begin{pgfonlayer}{edgelayer}
		\draw (0.center) to (1.center);
		\draw (3.center) to (2.center);
		\draw (2.center) to (0.center);
		\draw (4.center) to (5.center);
		\draw (11.center) to (10.center);
		\draw (8.center) to (12.center);
		\draw (21.center) to (17.center);
		\draw (14.center) to (15.center);
		\draw (20.center) to (22.center);
		\draw (1.center) to (3.center);
		\draw (21.center) to (18.center);
		\draw (17.center) to (24.center);
		\draw (24.center) to (18.center);
	\end{pgfonlayer}
\end{tikzpicture}
  \end{aligned}
\]

Only the `topology' of the diagrams matters: if two diagrams with the same
domain and codomain are equivalent up to isotopy, they represent the same
morphism.
On the other hand, two algebraic expressions might have the same diagrammatic
representation. For example, the equivalent diagrams
\[
    \tikzset{every path/.style={line width=1.1pt}}
\begin{aligned}
\begin{tikzpicture}
	\begin{pgfonlayer}{nodelayer}
		\node [style=none] (0) at (-0.25, 0.375) {};
		\node [style=none] (1) at (0.5, 0.375) {};
		\node [style=none] (2) at (-0.25, -0.875) {};
		\node [style=none] (3) at (0.5, -0.875) {};
		\node [style=none] (4) at (0.5, 0.125) {};
		\node [style=none] (5) at (0.125, -0.25) {$f$};
		\node [style=none] (6) at (2.75, -0.625) {};
		\node [style=none] (7) at (3, -0.625) {$Y_2$};
		\node [style=none] (8) at (0.5, -0.625) {};
		\node [style=none] (9) at (-0.25, -0.75) {};
		\node [style=none] (10) at (-1.25, -0.75) {$X_3$};
		\node [style=none] (11) at (-0.25, -0.25) {};
		\node [style=none] (12) at (-1.25, 0.25) {$X_1$};
		\node [style=none] (13) at (-1, 0.25) {};
		\node [style=none] (14) at (-0.25, 0.25) {};
		\node [style=none] (15) at (-1, -0.25) {};
		\node [style=none] (16) at (-1.25, -0.25) {$X_2$};
		\node [style=none] (17) at (-1, -0.75) {};
		\node [style=none] (18) at (2, -0.25) {};
		\node [style=none] (19) at (2.75, 0.375) {};
		\node [style=none] (20) at (3, 0.375) {$Z_1$};
		\node [style=none] (21) at (1.25, 0.5) {};
		\node [style=none] (22) at (1.25, 0.125) {};
		\node [style=none] (23) at (1.25, -0.25) {};
		\node [style=none] (24) at (2, 0.375) {};
		\node [style=none] (25) at (2, 0.5) {};
		\node [style=none] (26) at (1.625, 0.125) {$g$};
		\node [style=none] (27) at (2, -0.125) {};
		\node [style=none] (28) at (2.75, -0.125) {};
		\node [style=none] (29) at (3, -0.125) {$Z_2$};
		\node [style=none] (30) at (1.25, -1.75) {};
		\node [style=none] (31) at (2, -1.625) {};
		\node [style=none] (32) at (2.75, -1.125) {};
		\node [style=none] (33) at (3, -1.625) {$Y_3$};
		\node [style=none] (34) at (2.75, -1.625) {};
		\node [style=none] (35) at (1.625, -1.375) {$k$};
		\node [style=none] (36) at (3, -1.125) {$Y_2$};
		\node [style=none] (37) at (2, -1.125) {};
		\node [style=none] (38) at (2, -1.75) {};
		\node [style=none] (39) at (2, -1) {};
		\node [style=none] (40) at (1.25, -1) {};
	\end{pgfonlayer}
	\begin{pgfonlayer}{edgelayer}
		\draw (0.center) to (1.center);
		\draw (1.center) to (3.center);
		\draw (3.center) to (2.center);
		\draw (2.center) to (0.center);
		\draw (8.center) to (6.center);
		\draw (14.center) to (13.center);
		\draw (11.center) to (15.center);
		\draw (9.center) to (17.center);
		\draw (21.center) to (25.center);
		\draw (18.center) to (23.center);
		\draw (24.center) to (19.center);
		\draw (25.center) to (18.center);
		\draw (23.center) to (21.center);
		\draw (27.center) to (28.center);
		\draw (4.center) to (22.center);
		\draw (40.center) to (39.center);
		\draw (37.center) to (32.center);
		\draw (31.center) to (34.center);
		\draw (39.center) to (38.center);
		\draw (38.center) to (30.center);
		\draw (40.center) to (30.center);
	\end{pgfonlayer}
\end{tikzpicture}
\end{aligned}
=
\begin{aligned}
  \begin{tikzpicture}
	\begin{pgfonlayer}{nodelayer}
		\node [style=none] (0) at (0, 1) {};
		\node [style=none] (1) at (1, 1) {};
		\node [style=none] (2) at (0, -1.25) {};
		\node [style=none] (3) at (1, -1.25) {};
		\node [style=none] (4) at (1, -0.25) {};
		\node [style=none] (5) at (0.5, -0) {$f$};
		\node [style=none] (6) at (2.75, -0.5) {};
		\node [style=none] (7) at (3, -0.5) {$Y_2$};
		\node [style=none] (8) at (1, -0.75) {};
		\node [style=none] (9) at (0, -0.75) {};
		\node [style=none] (10) at (-1.25, -0.75) {$X_3$};
		\node [style=none] (11) at (0, -0.25) {};
		\node [style=none] (12) at (-1.25, 0.25) {$X_1$};
		\node [style=none] (13) at (-1, 0.25) {};
		\node [style=none] (14) at (0, 0.5) {};
		\node [style=none] (15) at (-1, -0.25) {};
		\node [style=none] (16) at (-1.25, -0.25) {$X_2$};
		\node [style=none] (17) at (-1, -0.75) {};
		\node [style=none] (18) at (2, -0.25) {};
		\node [style=none] (19) at (2.75, 0.375) {};
		\node [style=none] (20) at (3, 0.375) {$Z_1$};
		\node [style=none] (21) at (1.5, 0.25) {};
		\node [style=none] (22) at (1.375, -0) {};
		\node [style=none] (23) at (1.25, -0.25) {};
		\node [style=none] (24) at (2, 0.375) {};
		\node [style=none] (25) at (2, 0.5) {};
		\node [style=none] (26) at (1.75, -0) {$g$};
		\node [style=none] (27) at (2, -0.125) {};
		\node [style=none] (28) at (2.75, -0.125) {};
		\node [style=none] (29) at (3, -0.125) {$Z_2$};
		\node [style=none] (30) at (-0.5, -2.5) {};
		\node [style=none] (31) at (0.25, -1.75) {};
		\node [style=none] (32) at (2.75, -1.5) {};
		\node [style=none] (33) at (3, -1.75) {$Y_3$};
		\node [style=none] (34) at (2.75, -1.75) {};
		\node [style=none] (35) at (-0.125, -1.75) {$k$};
		\node [style=none] (36) at (3, -1.5) {$Y_2$};
		\node [style=none] (37) at (0.25, -1.5) {};
		\node [style=none] (38) at (0.25, -2.25) {};
		\node [style=none] (39) at (0.25, -1.375) {};
		\node [style=none] (40) at (-0.5, -1.375) {};
	\end{pgfonlayer}
	\begin{pgfonlayer}{edgelayer}
		\draw (0.center) to (1.center);
		\draw (1.center) to (3.center);
		\draw (3.center) to (2.center);
		\draw (2.center) to (0.center);
		\draw (8.center) to (6.center);
		\draw (14.center) to (13.center);
		\draw (11.center) to (15.center);
		\draw (9.center) to (17.center);
		\draw (21.center) to (25.center);
		\draw (18.center) to (23.center);
		\draw (24.center) to (19.center);
		\draw (25.center) to (18.center);
		\draw (23.center) to (21.center);
		\draw (27.center) to (28.center);
		\draw (4.center) to (22.center);
		\draw (40.center) to (39.center);
		\draw (37.center) to (32.center);
		\draw (31.center) to (34.center);
		\draw (39.center) to (38.center);
		\draw (38.center) to (30.center);
		\draw (40.center) to (30.center);
	\end{pgfonlayer}
\end{tikzpicture}
\end{aligned}
\]
read as all of the equivalent algebraic expressions 
\[
  ((g \ot \idn_{Y_2}) \ot k) \circ f = 
  (g \ot ((\idn_{Y_2} \ot k)) \circ \rho \circ (f \ot \idn_I) = 
  (g \ot \idn_{Y_2}) \circ f \circ (\idn_{X_1 \ot (X_2 \ot X_3)} \ot k) 
\]
and so on. The coherence theorem says that this does not matter: if two
algebraic expressions have the same diagrammatic representation, then the
algebraic expressions are equal. In more formal language, the graphical calculus
is sound and complete for the axioms of monoidal categories. See Joyal--Street
for details \cite{JS91}.


The coherence theorem thus implies that the graphical calculi goes beyond
visualisations of morphisms: it can provide provide bona-fide proofs of
equalities of morphisms. As a general principle, string diagrams are more
intuitive than the conventional algebraic language for understanding monoidal
categories.

\subsection{Symmetry}
A symmetric braiding in a monoidal category provides the ability to permute
objects or, equivalently, cross wires. We define symmetric monoidal categories
making use of the graphical notation outlined above, but introducing a new,
special symbol $\swap{.04\textwidth}$.

A \define{symmetric monoidal category} is a monoidal category $(\mc C,\ot)$
together with natural isomorphisms 
\[
    \tikzset{every path/.style={line width=1.1pt}}
    \xymatrixrowsep{0pt}
  \xymatrix{
\begin{tikzpicture}
	\begin{pgfonlayer}{nodelayer}
		\node [style=none] (0) at (1, 0.25) {$B$};
		\node [style=none] (1) at (0.5, -0.25) {};
		\node [style=none] (2) at (-1, 0.25) {$A$};
		\node [style=none] (3) at (0.5, 0.25) {};
		\node [style=none] (4) at (-0.5, -0.25) {};
		\node [style=none] (5) at (-1, -0.25) {$B$};
		\node [style=none] (6) at (1, -0.25) {$A$};
		\node [style=none] (7) at (-0.5, 0.25) {};
		\node [style=none] (8) at (-0.75, 0.25) {};
		\node [style=none] (9) at (-0.75, -0.25) {};
		\node [style=none] (10) at (0.75, 0.25) {};
		\node [style=none] (11) at (0.75, -0.25) {};
	\end{pgfonlayer}
	\begin{pgfonlayer}{edgelayer}
		\draw [in=180, out=0, looseness=1.00] (7.center) to (1.center);
		\draw [in=180, out=0, looseness=1.00] (4.center) to (3.center);
		\draw (7.center) to (8.center);
		\draw (4.center) to (9.center);
		\draw (3.center) to (10.center);
		\draw (1.center) to (11.center);
	\end{pgfonlayer}
\end{tikzpicture}
    \\
    \s_{A,B}: A \ot B \to B \ot A
  }
\]
such that
\[
    \tikzset{every path/.style={line width=1.1pt}}
  \begin{aligned}
    \begin{tikzpicture}
	\begin{pgfonlayer}{nodelayer}
		\node [style=none] (0) at (2, -0.25) {$B$};
		\node [style=none] (1) at (0.5, -0.25) {};
		\node [style=none] (2) at (-1, 0.25) {$A$};
		\node [style=none] (3) at (0.5, 0.25) {};
		\node [style=none] (4) at (-0.5, -0.25) {};
		\node [style=none] (5) at (-1, -0.25) {$B$};
		\node [style=none] (6) at (2, 0.25) {$A$};
		\node [style=none] (7) at (-0.5, 0.25) {};
		\node [style=none] (8) at (-0.75, 0.25) {};
		\node [style=none] (9) at (-0.75, -0.25) {};
		\node [style=none] (10) at (1.75, 0.25) {};
		\node [style=none] (11) at (1.75, -0.25) {};
		\node [style=none] (12) at (1.5, 0.25) {};
		\node [style=none] (13) at (1.5, -0.25) {};
	\end{pgfonlayer}
	\begin{pgfonlayer}{edgelayer}
		\draw [in=180, out=0, looseness=1.00] (7.center) to (1.center);
		\draw [in=180, out=0, looseness=1.00] (4.center) to (3.center);
		\draw (7.center) to (8.center);
		\draw (4.center) to (9.center);
		\draw [in=180, out=0, looseness=1.00] (3.center) to (13.center);
		\draw [in=180, out=-3, looseness=1.00] (1.center) to (12.center);
		\draw (13.center) to (11.center);
		\draw (12.center) to (10.center);
	\end{pgfonlayer}
\end{tikzpicture}
  \end{aligned}
\quad = \quad
  \begin{aligned}
\begin{tikzpicture}
	\begin{pgfonlayer}{nodelayer}
		\node [style=none] (0) at (1, -0.25) {$B$};
		\node [style=none] (1) at (-1, 0.25) {$A$};
		\node [style=none] (2) at (0.75, -0.25) {};
		\node [style=none] (3) at (-1, -0.25) {$B$};
		\node [style=none] (4) at (1, 0.25) {$A$};
		\node [style=none] (5) at (0.75, 0.25) {};
		\node [style=none] (6) at (-0.75, 0.25) {};
		\node [style=none] (7) at (-0.75, -0.25) {};
	\end{pgfonlayer}
	\begin{pgfonlayer}{edgelayer}
		\draw (5.center) to (6.center);
		\draw (2.center) to (7.center);
	\end{pgfonlayer}
\end{tikzpicture}
  \end{aligned}
\]
and
\[
    \tikzset{every path/.style={line width=1.1pt}}
  \begin{aligned}
    \begin{tikzpicture}
	\begin{pgfonlayer}{nodelayer}
		\node [style=none] (0) at (2, -0.25) {$C$};
		\node [style=none] (1) at (-1, 0.25) {$A$};
		\node [style=none] (2) at (-0.5, -0.25) {};
		\node [style=none] (3) at (-1, -0.25) {$B$};
		\node [style=none] (4) at (2, 0.25) {$B$};
		\node [style=none] (5) at (-0.5, 0.25) {};
		\node [style=none] (6) at (-0.75, 0.25) {};
		\node [style=none] (7) at (-0.75, -0.25) {};
		\node [style=none] (8) at (1.75, -0.25) {};
		\node [style=none] (9) at (1.75, 0.25) {};
		\node [style=none] (10) at (0.5, 0.25) {};
		\node [style=none] (11) at (1.5, -0.25) {};
		\node [style=none] (12) at (0.5, -0.25) {};
		\node [style=none] (13) at (-1, -0.75) {$C$};
		\node [style=none] (14) at (-0.75, -0.75) {};
		\node [style=none] (15) at (-0.5, -0.75) {};
		\node [style=none] (16) at (0.5, -0.75) {};
		\node [style=none] (17) at (1.5, -0.75) {};
		\node [style=none] (18) at (1.75, -0.75) {};
		\node [style=none] (19) at (2, -0.75) {$A$};
	\end{pgfonlayer}
	\begin{pgfonlayer}{edgelayer}
		\draw (11.center) to (8.center);
		\draw [in=180, out=0, looseness=1.00] (2.center) to (10.center);
		\draw [in=180, out=0, looseness=1.00] (5.center) to (12.center);
		\draw (10.center) to (9.center);
		\draw (14.center) to (15.center);
		\draw (15.center) to (16.center);
		\draw [in=180, out=0, looseness=1.00] (16.center) to (11.center);
		\draw [in=180, out=0, looseness=1.00] (12.center) to (17.center);
		\draw (17.center) to (18.center);
		\draw (6.center) to (5.center);
		\draw (2.center) to (7.center);
	\end{pgfonlayer}
\end{tikzpicture}
  \end{aligned}
\quad = \quad
  \begin{aligned}
\begin{tikzpicture}
	\begin{pgfonlayer}{nodelayer}
		\node [style=none] (0) at (1.35, 0.25) {$B\ot C$};
		\node [style=none] (1) at (0.5, -0.25) {};
		\node [style=none] (2) at (-1, 0.25) {$A$};
		\node [style=none] (3) at (0.5, 0.25) {};
		\node [style=none] (4) at (-0.5, -0.25) {};
		\node [style=none] (5) at (-1.35, -0.25) {$B\ot C$};
		\node [style=none] (6) at (1, -0.25) {$A$};
		\node [style=none] (7) at (-0.5, 0.25) {};
		\node [style=none] (8) at (-0.75, 0.25) {};
		\node [style=none] (9) at (-0.75, -0.25) {};
		\node [style=none] (10) at (0.75, 0.25) {};
		\node [style=none] (11) at (0.75, -0.25) {};
	\end{pgfonlayer}
	\begin{pgfonlayer}{edgelayer}
		\draw [in=180, out=0, looseness=1.00] (7.center) to (1.center);
		\draw [in=180, out=0, looseness=1.00] (4.center) to (3.center);
		\draw (7.center) to (8.center);
		\draw (4.center) to (9.center);
		\draw (3.center) to (10.center);
		\draw (1.center) to (11.center);
	\end{pgfonlayer}
\end{tikzpicture}
  \end{aligned}
\]
for all $A,B,C$ in $\mc C$.  We call $\s$ the \define{braiding}. We will also
talk, somewhat incidentally, of braided monoidal categories in the next
chapter; a \define{braided monoidal category} is a monoidal category with a
braiding that only obeys the latter axiom.

A \define{(lax/strong) symmetric monoidal functor} is a (lax/strong) monoidal
functor that further obeys
\[
\xymatrixcolsep{3pc}
\xymatrixrowsep{3pc}
\xymatrix{
FA \ot FB \ar[r]^{\varphi_{A,B}} \ar[d]_{\s'_{FA,FB}} & F(A \ot B)\ar[d]^{F\s_{A,B}}\\
FB \ot FA \ar[r]_{\varphi_{B,A}} & F(B \ot A)
}
\]
Morphisms between symmetric monoidal functors are simply monoidal natural
transformations. Thus two symmetric monoidal categories are \define{symmetric
monoidally equivalent} if they are monoidally equivalent by strong
\emph{symmetric} monoidal functors. If our categories are merely braided, we
refer to these functors as \define{braided monoidal functors}.

The coherence theorem for symmetric monoidal categories, with respect to string
diagrams, states that two morphisms in a symmetric monoidal category are equal
according to the axioms of symmetric monoidal categories if and only if their
diagrams are equal up to homotopy equivalence and applications of the defining
graphical identities above. See Joyal--Street \cite[Theorem 2.3]{JoyStr} for
more precision and details.

\section{Hypergraph categories}
Just as symmetric monoidal categories equip monoidal categories with precisely
enough extra structure to model crossing of strings in the graphical calculus,
hypergraph categories equip symmetric monoidal categories with precisely enough
extra structure to model multi-input multi-output interconnections of strings of
the same type. For this, we require each object to be equipped with a so-called
special commutative Frobenius monoid, which provides chosen maps to model this
interaction. These have a coherence result, known as the `spider theorem', that
says exactly how we use the maps to describe the connection of strings does not
matter: all that matters is that the strings are connected. 

\subsection{Frobenius monoids}
A Frobenius monoid comprises a monoid and comonoid on the same object that
interact according to the so-called Frobenius law.
\begin{definition}
  A \define{special commutative Frobenius monoid} $(X,\mu,\eta,\delta,\epsilon)$
  in a symmetric monoidal category $(\mathcal C, \otimes)$ is an object $X$ of
  $\mathcal C$ together with maps 
\[
  \xymatrixrowsep{1pt}
  \xymatrixcolsep{20pt}
  \xymatrix{
    \mult{.075\textwidth} & & \unit{.075\textwidth} & & 
    \comult{.075\textwidth} & & \counit{.075\textwidth} \\
    \mu\maps X\otimes X \to X & & \eta\maps I \to X & & 
    \delta\maps X\to X \otimes X & & \epsilon\maps X \to I
  }
\]
obeying the commutative monoid axioms
\[
  \xymatrixrowsep{1pt}
  \xymatrixcolsep{25pt}
  \xymatrix{
    \assocl{.1\textwidth} = \assocr{.1\textwidth} & \unitl{.1\textwidth} =
    \idone{.1\textwidth} & \commute{.1\textwidth} = \mult{.07\textwidth} \\
    \textrm{(associativity)} & \textrm{(unitality)} & \textrm{(commutativity)}
  }
\]
the cocommutative comonoid axioms
\[
  \xymatrixrowsep{1pt}
  \xymatrixcolsep{25pt}
  \xymatrix{
    \coassocl{.1\textwidth} = \coassocr{.1\textwidth} & \counitl{.1\textwidth} =
    \idone{.1\textwidth} & \cocommute{.1\textwidth} = \comult{.07\textwidth} \\
    \textrm{(coassociativity)} & \textrm{(counitality)} &
    \textrm{(cocommutativity)}
  }
\]
and the Frobenius and special axioms
  \[
  \xymatrixrowsep{1pt}
  \xymatrixcolsep{25pt}
  \xymatrix{
    \frobs{.1\textwidth} = \frobx{.1\textwidth} = \frobz{.1\textwidth} & \spec{.1\textwidth} =
    \idone{.1\textwidth} \\
    \textrm{(Frobenius)} & \textrm{(special)} 
  }
  \]
  We call $\mu$ the \define{multiplication}, $\eta$ the \define{unit}, $\delta$
  the \define{comultiplication}, and $\epsilon$ the \define{counit}.
\end{definition}

Special commutative Frobenius monoids were first formulated by Carboni and
Walters, under the name commutative separable algebras. The Frobenius law and
the special law were termed the S=X law and the diamond=1 law respectively
\cite{CarWal,RosSabWal}.

Alternate axiomatisations are possible. In addition to the `upper' unitality law
above, the mirror image `lower' unitality law also holds, due to commutativity
and the naturality of the braiding. While we write two equations for the
Frobenius law, this is redundant: given the other axioms, the equality of any
two of the diagrams implies the equality of all three.  Further, note that a
monoid and comonoid obeying the Frobenius law is commutative if and only if it
is cocommutative. Thus while a commutative and cocommutative Frobenius monoid
might more properly be called a bicommutative Frobenius monoid, there is no
ambiguity if we only say commutative.

The common feature to these equations is that each side describes a different
way of using the generators to connect some chosen set of inputs to some chosen
set of outputs. This observation provides a `coherence' type result for special
commutative Frobenius monoids, known as the `spider theorem'.
\begin{theorem}
  Let $(X,\mu,\eta,\delta,\epsilon)$ be a special commutative Frobenius monoid,
  and let $f,g\maps X^{\ot n} \to X^{\ot m}$ be map constructed, using
  composition and the monoidal product, from $\mu$, $\eta$, $\delta$, $\epsilon$, the
  coherence maps and braiding, and the identity map on $X$. Then $f$ and $g$ are
  equal if and only if given their string diagrams in the above notation, there
  exists a bijection between the connected components of the two diagrams such
  that corresponding connected components connect the exact same sets of inputs
  and outputs.
\end{theorem}%
See \cite{Kis16,FonCoy16} \cite{CK,CPP} for further details.

\subsection{Hypergraph categories}

\begin{definition}
  A \define{hypergraph category} is a symmetric monoidal category in which each
  object $X$ is equipped with a special commutative Frobenius structure
  $(X,\mu_X,\delta_X,\eta_X,\epsilon_X)$ such that 
  \[
    \tikzset{every path/.style={line width=1.1pt}}
    \xymatrixcolsep{1.8ex}
    \xymatrixrowsep{1ex}
    \xymatrix{
    \begin{aligned}
      \begin{tikzpicture}[scale=.65]
	\begin{pgfonlayer}{nodelayer}
		\node [style=none] (0) at (-0.25, 0.5) {};
		\node [style=dot] (1) at (0.5, -0) {};
		\node [style=none] (2) at (-0.25, -0.5) {};
		\node [style=none] (3) at (1.25, -0) {};
		\node [style=none] (4) at (-1.5, 0.5) {$X \otimes Y$};
		\node [style=none] (5) at (-1.5, -0.5) {$X \otimes Y$};
		\node [style=none] (6) at (2, -0) {$X \otimes Y$};
		\node [style=none] (7) at (-0.75, 0.5) {};
		\node [style=none] (8) at (-0.75, -0.5) {};
	\end{pgfonlayer}
	\begin{pgfonlayer}{edgelayer}
		\draw [in=90, out=0, looseness=0.90] (0.center) to (1.center);
		\draw [in=-90, out=0, looseness=0.90] (2.center) to (1.center);
		\draw (1.center) to (3.center);
		\draw (7.center) to (0.center);
		\draw (8.center) to (2.center);
	\end{pgfonlayer}
\end{tikzpicture}
\end{aligned}
=
\begin{aligned}
  \begin{tikzpicture}[scale=.65]
	\begin{pgfonlayer}{nodelayer}
		\node [style=none] (0) at (-0.25, 0.5) {};
		\node [style=dot] (1) at (0.5, -0) {};
		\node [style=none] (2) at (-0.25, -0.5) {};
		\node [style=none] (3) at (1.25, -0) {};
		\node [style=none] (4) at (-1.75, 0.5) {$X$};
		\node [style=none] (5) at (-1.75, -1.25) {$X$};
		\node [style=none] (6) at (1.5, -0) {$X$};
		\node [style=none] (7) at (-1.5, 0.5) {};
		\node [style=none] (8) at (-1.5, -1.25) {};
		\node [style=none] (9) at (1.25, -1.75) {};
		\node [style=none] (10) at (-1.5, -2.25) {};
		\node [style=none] (11) at (1.5, -1.75) {$Y$};
		\node [style=none] (12) at (-1.5, -0.5) {};
		\node [style=none] (13) at (-0.25, -2.25) {};
		\node [style=dot] (14) at (0.5, -1.75) {};
		\node [style=none] (15) at (-0.25, -1.25) {};
		\node [style=none] (16) at (-1.75, -0.5) {$Y$};
		\node [style=none] (17) at (-1.75, -2.25) {$Y$};
	\end{pgfonlayer}
	\begin{pgfonlayer}{edgelayer}
		\draw [in=90, out=0, looseness=0.90] (0.center) to (1.center);
		\draw [in=-90, out=0, looseness=0.90] (2.center) to (1.center);
		\draw (1.center) to (3.center);
		\draw (7.center) to (0.center);
		\draw [in=180, out=0, looseness=1.00] (8.center) to (2.center);
		\draw [in=90, out=0, looseness=0.90] (15.center) to (14);
		\draw [in=-90, out=0, looseness=0.90] (13.center) to (14);
		\draw (14) to (9.center);
		\draw [in=180, out=0, looseness=1.00] (12.center) to (15.center);
		\draw (10.center) to (13.center);
	\end{pgfonlayer}
\end{tikzpicture}
\end{aligned}
& &   
\qquad
  \begin{aligned}
    \begin{tikzpicture}[scale=.65]
	\begin{pgfonlayer}{nodelayer}
		\node [style=dot] (0) at (-0.5, 0.5) {};
		\node [style=none] (1) at (1.5, 0.5) {$X\otimes Y$};
		\node [style=none] (2) at (0.75, 0.5) {};
	\end{pgfonlayer}
	\begin{pgfonlayer}{edgelayer}
		\draw (2.center) to (0.center);
	\end{pgfonlayer}
\end{tikzpicture}
  \end{aligned}
  = \qquad
  \begin{aligned}
    \begin{tikzpicture}[scale=.65]
	\begin{pgfonlayer}{nodelayer}
		\node [style=dot] (0) at (0, 0.5) {};
		\node [style=none] (1) at (1.5, 0.5) {$X$};
		\node [style=none] (2) at (1.25, 0.5) {};
		\node [style=none] (3) at (1.25, -0.25) {};
		\node [style=dot] (4) at (0, -0.25) {};
		\node [style=none] (5) at (1.5, -0.25) {$Y$};
	\end{pgfonlayer}
	\begin{pgfonlayer}{edgelayer}
		\draw (2.center) to (0.center);
		\draw (3.center) to (4.center);
	\end{pgfonlayer}
\end{tikzpicture}
  \end{aligned}
\\
    \begin{aligned}
\begin{tikzpicture}[scale=.65]
	\begin{pgfonlayer}{nodelayer}
		\node [style=none] (0) at (0.75, 0.5) {};
		\node [style=dot] (1) at (0, -0) {};
		\node [style=none] (2) at (0.75, -0.5) {};
		\node [style=none] (3) at (-0.75, -0) {};
		\node [style=none] (4) at (2, 0.5) {$X \otimes Y$};
		\node [style=none] (5) at (2, -0.5) {$X \otimes Y$};
		\node [style=none] (6) at (-1.5, -0) {$X \otimes Y$};
		\node [style=none] (7) at (1.25, 0.5) {};
		\node [style=none] (8) at (1.25, -0.5) {};
	\end{pgfonlayer}
	\begin{pgfonlayer}{edgelayer}
		\draw [in=90, out=180, looseness=0.90] (0.center) to (1.center);
		\draw [in=-90, out=180, looseness=0.90] (2.center) to (1.center);
		\draw (1.center) to (3.center);
		\draw (7.center) to (0.center);
		\draw (8.center) to (2.center);
	\end{pgfonlayer}
\end{tikzpicture}
\end{aligned}
=
\begin{aligned}
\begin{tikzpicture}[scale=.65]
	\begin{pgfonlayer}{nodelayer}
		\node [style=none] (0) at (0, 0.5) {};
		\node [style=dot] (1) at (-0.75, -0) {};
		\node [style=none] (2) at (0, -0.5) {};
		\node [style=none] (3) at (-1.5, -0) {};
		\node [style=none] (4) at (1.5, 0.5) {$X$};
		\node [style=none] (5) at (1.5, -1.25) {$X$};
		\node [style=none] (6) at (-1.75, -0) {$X$};
		\node [style=none] (7) at (1.25, 0.5) {};
		\node [style=none] (8) at (1.25, -1.25) {};
		\node [style=none] (9) at (-1.5, -1.75) {};
		\node [style=none] (10) at (1.25, -2.25) {};
		\node [style=none] (11) at (-1.75, -1.75) {$Y$};
		\node [style=none] (12) at (1.25, -0.5) {};
		\node [style=none] (13) at (0, -2.25) {};
		\node [style=dot] (14) at (-0.75, -1.75) {};
		\node [style=none] (15) at (0, -1.25) {};
		\node [style=none] (16) at (1.5, -0.5) {$Y$};
		\node [style=none] (17) at (1.5, -2.25) {$Y$};
	\end{pgfonlayer}
	\begin{pgfonlayer}{edgelayer}
		\draw [in=90, out=180, looseness=0.90] (0.center) to (1.center);
		\draw [in=-90, out=180, looseness=0.90] (2.center) to (1.center);
		\draw (1.center) to (3.center);
		\draw (7.center) to (0.center);
		\draw [in=0, out=180, looseness=1.00] (8.center) to (2.center);
		\draw [in=90, out=180, looseness=0.90] (15.center) to (14);
		\draw [in=-90, out=180, looseness=0.90] (13.center) to (14);
		\draw (14) to (9.center);
		\draw [in=0, out=180, looseness=1.00] (12.center) to (15.center);
		\draw (10.center) to (13.center);
	\end{pgfonlayer}
\end{tikzpicture}
\end{aligned}
& &
  \begin{aligned}
    \begin{tikzpicture}[scale=.65]
	\begin{pgfonlayer}{nodelayer}
		\node [style=dot] (0) at (1.5, 0.5) {};
		\node [style=none] (1) at (-0.5, 0.5) {$X\otimes Y$};
		\node [style=none] (2) at (0.25, 0.5) {};
	\end{pgfonlayer}
	\begin{pgfonlayer}{edgelayer}
		\draw (2.center) to (0.center);
	\end{pgfonlayer}
\end{tikzpicture}
  \end{aligned}
  \qquad \quad =
  \begin{aligned}
    \begin{tikzpicture}[scale=.65]
	\begin{pgfonlayer}{nodelayer}
		\node [style=dot] (0) at (1.5, 0.5) {};
		\node [style=none] (1) at (0, 0.5) {$X$};
		\node [style=none] (2) at (0.25, 0.5) {};
		\node [style=none] (3) at (0.25, -0.25) {};
		\node [style=dot] (4) at (1.5, -0.25) {};
		\node [style=none] (5) at (0, -0.25) {$Y$};
	\end{pgfonlayer}
	\begin{pgfonlayer}{edgelayer}
		\draw (2.center) to (0.center);
		\draw (3.center) to (4.center);
	\end{pgfonlayer}
\end{tikzpicture}
\qquad \quad
  \end{aligned}
}
\]
\end{definition}

Note that we do \emph{not} require these Frobenius morphisms to be natural in
$X$. While morphisms in a hypergraph category need not interact with the Frobenius
structure in any particular way, we do require functors between hypergraph
categories to preserve it.

\begin{definition}
A functor $(F,\varphi)$ of hypergraph categories, or \define{hypergraph
functor}, is a strong symmetric monoidal functor $(F,\varphi)$ such that for
each object $X$ the following diagrams commute:
\[
  \xymatrix{
    FX\boxtimes FX \ar[rr]^{\mu_{FX}} \ar[dr]_{\varphi} && FX \\
    & F(X \ot X) \ar[ur]_{F\mu_X}
  }
  \qquad
  \xymatrix{
    1_{\mc D} \ar[rr]^{\eta_{FX}} \ar[dr]_{\varphi_1} && FX \\
    & F1_{\mc C} \ar[ur]_{F\eta_X}
  }
\]
\[
  \xymatrix{
    FX \ar[rr]^{\delta_{FX}} \ar[dr]_{F\delta_X} && FX \boxtimes FX\\
    & F(X \ot X) \ar[ur]_{\varphi^{-1}}
  }
  \qquad
  \xymatrix{
    FX \ar[rr]^{\epsilon_{FX}} \ar[dr]_{F\epsilon_X} && 1_{\mc D} \\
    & F1_{\mc C} \ar[ur]_{\varphi^{-1}}
  }
\]
\end{definition}

Equivalently, a strong symmetric monoidal functor $F$ is a hypergraph functor if
for every $X$ the special commutative Frobenius structure on $FX$ is
\[
  (FX,\enspace F\mu_X \circ \varphi_{X,X},\enspace  \varphi^{-1}_{X,X} \circ F\delta_X,\enspace  F\eta_X \circ
  \varphi_1,\enspace  \varphi_1^{-1} \circ F\epsilon_X).
\]

Just as monoidal natural transformations themselves are enough as morphisms
between symmetric monoidal functors, so too they suffice as morphisms between
hypergraph functors. Two hypergraph categories are \define{hypergraph
equivalent} if there exist hypergraph functors with monoidal natural
transformations to the identity functors. 
  
The term hypergraph category was introduced recently \cite{Fon15,Kis16}, in
reference to the fact that these special commutative Frobenius monoids provide
precisely the structure required to draw graphs with `hyperedges': edges
connecting any number of inputs to any number of outputs. Again first defined by
Walters and Carboni \cite{Car91}, under the name well-supported compact closed
categories, in recent years hypergraph categories have been rediscovered a
number of times, also appearing under the names dungeon categories \cite{Mor12}
and dgs-monoidal categories \cite{Gad}. 

%Nonetheless, this flexibility is a strength of hypergraph categories. In the
%category of vector spaces, for example, a special commutative Frobenius monoid
%is a basis \cite{CoePavVic}. This allows us to talk about the hypergraph category of vector spaces with bases .



\subsection{Hypergraph categories are self-dual compact closed}

Note that if an object $X$ is equipped with a Frobenius monoid structure then
the maps 
\[
    \xymatrixrowsep{0pt}
    \xymatrix{
  \begin{aligned}
      \resizebox{.09\textwidth}{!}{
	\begin{tikzpicture}
	  \begin{pgfonlayer}{nodelayer}
	    \node [style=circ] (0) at (0.75, -0) {};
	    \node [style=circ] (1) at (0.125, -0) {};
	    \node [style=none] (2) at (-1, 0.5) {};
	    \node [style=none] (3) at (-1, -0.5) {};
	  \end{pgfonlayer}
	  \begin{pgfonlayer}{edgelayer}
	    \draw [line width=2pt] (0.center) to (1.center);
	    \draw [line width=2pt, in=0, out=120, looseness=1.20] (1.center) to (2.center);
	    \draw [line width=2pt, in=0, out=-120, looseness=1.20] (1.center) to (3.center);
	  \end{pgfonlayer}
	\end{tikzpicture} 
    }
  \end{aligned}
  & \quad \mbox{and} \quad&
  \begin{aligned}
      \resizebox{.09\textwidth}{!}{
	\begin{tikzpicture}
	  \begin{pgfonlayer}{nodelayer}
	    \node [style=circ] (0) at (-1, 0) {};
	    \node [style=circ] (1) at (-0.375, 0) {};
	    \node [style=none] (2) at (0.75, -0.5) {};
	    \node [style=none] (3) at (0.75, 0.5) {};
	  \end{pgfonlayer}
	  \begin{pgfonlayer}{edgelayer}
	    \draw [line width=2pt] (0.center) to (1.center);
	    \draw [line width=2pt, in=180, out=-60, looseness=1.20] (1.center) to (2.center);
	    \draw [line width=2pt, in=180, out=60, looseness=1.20] (1.center) to (3.center);
	  \end{pgfonlayer}
	\end{tikzpicture}
      } 
  \end{aligned} \\
      \epsilon \circ \mu\maps X \ot X \to 1 & &
      \delta \circ \eta\maps 1 \to X \ot X
    }
\]
obey both
\[
  \begin{aligned}
    \resizebox{3cm}{!}{
      \begin{tikzpicture}
	\begin{pgfonlayer}{nodelayer}
	  \node [style=circ] (0) at (-1.5, 0.5) {};
	  \node [style=circ] (1) at (-0.75, 0.5) {};
	  \node [style=none] (2) at (0.25, -0) {};
	  \node [style=none] (3) at (0.25, 1) {};
	  \node [style=circ] (4) at (1, -0.5) {};
	  \node [style=none] (5) at (0, -0) {};
	  \node [style=circ] (6) at (1.75, -0.5) {};
	  \node [style=none] (7) at (0, -1) {};
	  \node [style=none] (8) at (2.5, 1) {};
	  \node [style=none] (9) at (-2.5, -1) {};
	\end{pgfonlayer}
	\begin{pgfonlayer}{edgelayer}
	  \draw [line width=2pt, in=180, out=-60, looseness=1.20] (1) to (2.center);
	  \draw [line width=2pt, in=180, out=60, looseness=1.20] (1) to (3.center);
	  \draw [line width=2pt] (0.center) to (1);
	  \draw [line width=2pt] (6.center) to (4);
	  \draw [line width=2pt, in=0, out=120, looseness=1.20] (4) to (5.center);
	  \draw [line width=2pt, in=0, out=-120, looseness=1.20] (4) to (7.center);
	  \draw [line width=2pt] (3.center) to (8.center);
	  \draw [line width=2pt] (7.center) to (9.center);
	\end{pgfonlayer}
      \end{tikzpicture}
    }
  \end{aligned}
  \quad = \quad
  \begin{aligned}
    \resizebox{3cm}{!}{
      \begin{tikzpicture}
	\begin{pgfonlayer}{nodelayer}
	  \node [style=circ] (0) at (-0.5, -0) {};
	  \node [style=none] (1) at (-1.5, -0.5) {};
	  \node [style=circ] (2) at (-1.5, 0.5) {};
	  \node [style=circ] (3) at (0.5, -0) {};
	  \node [style=circ] (4) at (1.5, -0.5) {};
	  \node [style=none] (5) at (1.5, 0.5) {};
	  \node [style=none] (6) at (2.5, 0.5) {};
	  \node [style=none] (7) at (-2.5, -0.5) {};
	\end{pgfonlayer}
	\begin{pgfonlayer}{edgelayer}
	  \draw [line width=2pt, in=0, out=-120, looseness=1.20] (0.center) to (1.center);
	  \draw [line width=2pt, in=0, out=120, looseness=1.20] (0.center) to (2.center);
	  \draw [line width=2pt, in=180, out=-60, looseness=1.20] (3) to (4.center);
	  \draw [line width=2pt, in=180, out=60, looseness=1.20] (3) to (5.center);
	  \draw [line width=2pt] (0) to (3);
	  \draw [line width=2pt] (7.center) to (1.center);
	  \draw [line width=2pt] (5.center) to (6.center);
	\end{pgfonlayer}
      \end{tikzpicture}
    }
  \end{aligned}
  \quad = \quad
  \begin{aligned}
    \resizebox{2cm}{!}{
      \begin{tikzpicture}
	\begin{pgfonlayer}{nodelayer}
	  \node [style=none] (0) at (2, -0) {};
	  \node [style=none] (1) at (-2, -0) {};
	  \node [style=none] (2) at (0, -0.5) {};
	  \node [style=none] (3) at (0, 0.5) {};
	\end{pgfonlayer}
	\begin{pgfonlayer}{edgelayer}
	  \draw [line width=2pt](1.center) to (0.center);
	\end{pgfonlayer}
      \end{tikzpicture}
    }
  \end{aligned}
\]
and the reflected equations. Thus if an object carries a Frobenius monoid it is
also self-dual, and any hypergraph category is a fortiori self-dual compact
closed. 

We introduce the notation
  \[
    \tikzset{every path/.style={line width=1.1pt}}
    \begin{aligned}
      \begin{tikzpicture}[scale=.65]
	\begin{pgfonlayer}{nodelayer}
		\node [style=none] (0) at (0.75, 0.5) {};
		\node [style=none] (1) at (0, -0) {};
		\node [style=none] (2) at (0.75, -0.5) {};
		\node [style=none] (3) at (1.25, 0.5) {};
		\node [style=none] (4) at (1.25, -0.5) {};
	\end{pgfonlayer}
	\begin{pgfonlayer}{edgelayer}
		\draw [in=90, out=180, looseness=0.90] (0.center) to (1.center);
		\draw [in=-90, out=180, looseness=0.90] (2.center) to (1.center);
		\draw (3.center) to (0.center);
		\draw (4.center) to (2.center);
	\end{pgfonlayer}
\end{tikzpicture}
    \end{aligned}
    :=
    \begin{aligned}
      \begin{tikzpicture}[scale=.65]
	\begin{pgfonlayer}{nodelayer}
		\node [style=none] (0) at (0.75, 0.5) {};
		\node [style=dot] (1) at (0, -0) {};
		\node [style=none] (2) at (0.75, -0.5) {};
		\node [style=none] (3) at (1.25, 0.5) {};
		\node [style=none] (4) at (1.25, -0.5) {};
		\node [style=dot] (5) at (-0.5, -0) {};
	\end{pgfonlayer}
	\begin{pgfonlayer}{edgelayer}
		\draw [in=90, out=180, looseness=0.90] (0.center) to (1.center);
		\draw [in=-90, out=180, looseness=0.90] (2.center) to (1.center);
		\draw (3.center) to (0.center);
		\draw (4.center) to (2.center);
		\draw (5.center) to (1.center);
	\end{pgfonlayer}
\end{tikzpicture}
    \end{aligned}
    \qquad
    \qquad
    \begin{aligned}
      \begin{tikzpicture}[scale=.65]
	\begin{pgfonlayer}{nodelayer}
		\node [style=none] (0) at (0, 0.5) {};
		\node [style=none] (1) at (0.75, -0) {};
		\node [style=none] (2) at (0, -0.5) {};
		\node [style=none] (3) at (-0.5, 0.5) {};
		\node [style=none] (4) at (-0.5, -0.5) {};
	\end{pgfonlayer}
	\begin{pgfonlayer}{edgelayer}
		\draw [in=90, out=0, looseness=0.90] (0.center) to (1.center);
		\draw [in=-90, out=0, looseness=0.90] (2.center) to (1.center);
		\draw (3.center) to (0.center);
		\draw (4.center) to (2.center);
	\end{pgfonlayer}
\end{tikzpicture}
    \end{aligned}
    :=
    \begin{aligned}
      \begin{tikzpicture}[scale=.65]
	\begin{pgfonlayer}{nodelayer}
		\node [style=none] (0) at (0, 0.5) {};
		\node [style=dot] (1) at (0.75, -0) {};
		\node [style=none] (2) at (0, -0.5) {};
		\node [style=none] (3) at (-0.5, 0.5) {};
		\node [style=none] (4) at (-0.5, -0.5) {};
		\node [style=dot] (5) at (1.25, -0) {};
	\end{pgfonlayer}
	\begin{pgfonlayer}{edgelayer}
		\draw [in=90, out=0, looseness=0.90] (0.center) to (1.center);
		\draw [in=-90, out=0, looseness=0.90] (2.center) to (1.center);
		\draw (3.center) to (0.center);
		\draw (4.center) to (2.center);
		\draw (5.center) to (1.center);
	\end{pgfonlayer}
\end{tikzpicture}
    \end{aligned}
  \]
As in any self-dual compact closed category, mapping each morphism 
$
    \tikzset{every path/.style={line width=1.1pt}}
    \begin{aligned}
  \begin{tikzpicture}[scale=.65]
	\begin{pgfonlayer}{nodelayer}
		\node [style=none] (0) at (0.5, -0.5) {};
		\node [style=none] (1) at (-0.25, -0.5) {};
		\node [style=none] (2) at (-1, -0.5) {};
		\node [style=none] (3) at (-1.75, -0.5) {};
		\node [style=none] (4) at (-1, -0.125) {};
		\node [style=none] (5) at (-1, -0.875) {};
		\node [style=none] (6) at (-0.25, -0.875) {};
		\node [style=none] (7) at (-0.25, -0.125) {};
		\node [style=none] (8) at (-0.625, -0.5) {$f$};
		\node [style=none] (9) at (-2, -0.5) {$X$};
		\node [style=none] (10) at (0.75, -0.5) {$Y$};
	\end{pgfonlayer}
	\begin{pgfonlayer}{edgelayer}
		\draw (1.center) to (0.center);
		\draw (2.center) to (3.center);
		\draw (4.center) to (5.center);
		\draw (5.center) to (6.center);
		\draw (6.center) to (7.center);
		\draw (7.center) to (4.center);
	\end{pgfonlayer}
\end{tikzpicture}
    \end{aligned}
$
to its dual morphism
\[
    \tikzset{every path/.style={line width=1.1pt}}
\begin{tikzpicture}[scale=.65]
	\begin{pgfonlayer}{nodelayer}
		\node [style=none] (0) at (0, 0.5) {};
		\node [style=none] (1) at (0.75, -0) {};
		\node [style=none] (2) at (0, -0.5) {};
		\node [style=none] (3) at (-2.5, 0.5) {};
		\node [style=none] (4) at (-0.25, -0.5) {};
		\node [style=none] (5) at (-2, -1) {};
		\node [style=none] (6) at (-1.25, -1.5) {};
		\node [style=none] (7) at (-1, -0.5) {};
		\node [style=none] (8) at (1.25, -1.5) {};
		\node [style=none] (9) at (-1.25, -0.5) {};
		\node [style=none] (10) at (-1, -0.125) {};
		\node [style=none] (11) at (-1, -0.875) {};
		\node [style=none] (12) at (-0.25, -0.875) {};
		\node [style=none] (13) at (-0.25, -0.125) {};
		\node [style=none] (14) at (-0.625, -0.5) {$f$};
		\node [style=none] (15) at (1.5, -1.5) {$X$};
		\node [style=none] (16) at (-2.75, 0.5) {$Y$};
	\end{pgfonlayer}
	\begin{pgfonlayer}{edgelayer}
		\draw [in=90, out=0, looseness=0.90] (0.center) to (1.center);
		\draw [in=-90, out=0, looseness=0.90] (2.center) to (1.center);
		\draw (3.center) to (0.center);
		\draw (4.center) to (2.center);
		\draw [in=90, out=180, looseness=0.90] (9.center) to (5.center);
		\draw [in=-90, out=180, looseness=0.90] (6.center) to (5.center);
		\draw (7.center) to (9.center);
		\draw (8.center) to (6.center);
		\draw (10.center) to (11.center);
		\draw (11.center) to (12.center);
		\draw (12.center) to (13.center);
		\draw (13.center) to (10.center);
	\end{pgfonlayer}
\end{tikzpicture}
\]
further equips each hypergraph category with a so-called dagger functor---an
involutive contravariant endofunctor that is the identity on objects---such that
the category is a dagger compact category. Dagger compact categories were first
introduced in the context of categorical quantum mechanics \cite{AC}, under the
name strongly compact closed category, and have been demonstrated to be a key
structure in diagrammatic reasoning and the logic of quantum mechanics.

Compactness allows us to blur the distinction between composition and the
monoidal product of morphisms. Firstly, there is a one-to-one correspondence
between morphisms $X \to Y$ and morphisms $1 \to X\ot Y$ given by taking 
$
    \tikzset{every path/.style={line width=1.1pt}}
    \begin{aligned}
  \begin{tikzpicture}[scale=.65]
	\begin{pgfonlayer}{nodelayer}
		\node [style=none] (0) at (0.5, -0.5) {};
		\node [style=none] (1) at (-0.25, -0.5) {};
		\node [style=none] (2) at (-1, -0.5) {};
		\node [style=none] (3) at (-1.75, -0.5) {};
		\node [style=none] (4) at (-1, -0.125) {};
		\node [style=none] (5) at (-1, -0.875) {};
		\node [style=none] (6) at (-0.25, -0.875) {};
		\node [style=none] (7) at (-0.25, -0.125) {};
		\node [style=none] (8) at (-0.625, -0.5) {$f$};
		\node [style=none] (9) at (-2, -0.5) {$X$};
		\node [style=none] (10) at (0.75, -0.5) {$Y$};
	\end{pgfonlayer}
	\begin{pgfonlayer}{edgelayer}
		\draw (1.center) to (0.center);
		\draw (2.center) to (3.center);
		\draw (4.center) to (5.center);
		\draw (5.center) to (6.center);
		\draw (6.center) to (7.center);
		\draw (7.center) to (4.center);
	\end{pgfonlayer}
\end{tikzpicture}
    \end{aligned}
$
to its so-called \define{name}
\[
    \tikzset{every path/.style={line width=1.1pt}}
  \begin{tikzpicture}[scale=.65]
	\begin{pgfonlayer}{nodelayer}
		\node [style=none] (0) at (0.5, -0.5) {};
		\node [style=none] (1) at (-0.25, -0.5) {};
		\node [style=none] (2) at (-2, -0) {};
		\node [style=none] (3) at (-1.25, -0.5) {};
		\node [style=none] (4) at (0.5, 0.5) {};
		\node [style=none] (5) at (-1, -0.5) {};
		\node [style=none] (6) at (-1.25, 0.5) {};
		\node [style=none] (7) at (-1, -0.125) {};
		\node [style=none] (8) at (-1, -0.875) {};
		\node [style=none] (9) at (-0.25, -0.875) {};
		\node [style=none] (10) at (-0.25, -0.125) {};
		\node [style=none] (11) at (-0.625, -0.5) {$f$};
		\node [style=none] (12) at (0.75, 0.5) {$X$};
		\node [style=none] (13) at (0.75, -0.5) {$Y$};
	\end{pgfonlayer}
	\begin{pgfonlayer}{edgelayer}
		\draw (1.center) to (0.center);
		\draw [in=90, out=180, looseness=0.90] (6.center) to (2.center);
		\draw [in=-90, out=180, looseness=0.90] (3.center) to (2.center);
		\draw (4.center) to (6.center);
		\draw (5.center) to (3.center);
		\draw (7.center) to (8.center);
		\draw (8.center) to (9.center);
		\draw (9.center) to (10.center);
		\draw (10.center) to (7.center);
	\end{pgfonlayer}
\end{tikzpicture}
\]
By compactness, we have the equation
\[
    \tikzset{every path/.style={line width=1.1pt}}
  \begin{aligned}
\begin{tikzpicture}[scale=.65]
	\begin{pgfonlayer}{nodelayer}
		\node [style=none] (0) at (0, -0.5) {};
		\node [style=none] (1) at (-0.25, -0.5) {};
		\node [style=none] (2) at (-2, -0) {};
		\node [style=none] (3) at (-1.25, -0.5) {};
		\node [style=none] (4) at (1.5, 0.5) {};
		\node [style=none] (5) at (-1, -0.5) {};
		\node [style=none] (6) at (-1.25, 0.5) {};
		\node [style=none] (7) at (-1, -0.125) {};
		\node [style=none] (8) at (-1, -0.875) {};
		\node [style=none] (9) at (-0.25, -0.875) {};
		\node [style=none] (10) at (-0.25, -0.125) {};
		\node [style=none] (11) at (-0.625, -0.5) {$f$};
		\node [style=none] (12) at (1.75, 0.5) {$X$};
		\node [style=none] (13) at (-2, -2) {};
		\node [style=none] (14) at (-0.25, -2.125) {};
		\node [style=none] (15) at (-0.25, -2.5) {};
		\node [style=none] (16) at (-1, -2.5) {};
		\node [style=none] (17) at (1.5, -2.5) {};
		\node [style=none] (18) at (-0.25, -2.875) {};
		\node [style=none] (19) at (-1.25, -2.5) {};
		\node [style=none] (20) at (0, -1.5) {};
		\node [style=none] (21) at (-0.625, -2.5) {$g$};
		\node [style=none] (22) at (-1, -2.125) {};
		\node [style=none] (23) at (-1.25, -1.5) {};
		\node [style=none] (24) at (-1, -2.875) {};
		\node [style=none] (25) at (1.75, -2.5) {$Z$};
		\node [style=none] (26) at (1, -1) {};
	\end{pgfonlayer}
	\begin{pgfonlayer}{edgelayer}
		\draw (1.center) to (0.center);
		\draw [in=90, out=180, looseness=0.90] (6.center) to (2.center);
		\draw [in=-90, out=180, looseness=0.90] (3.center) to (2.center);
		\draw (4.center) to (6.center);
		\draw (5.center) to (3.center);
		\draw (7.center) to (8.center);
		\draw (8.center) to (9.center);
		\draw (9.center) to (10.center);
		\draw (10.center) to (7.center);
		\draw (15.center) to (17.center);
		\draw [in=90, out=180, looseness=0.90] (23.center) to (13.center);
		\draw [in=-90, out=180, looseness=0.90] (19.center) to (13.center);
		\draw (20.center) to (23.center);
		\draw (16.center) to (19.center);
		\draw (22.center) to (24.center);
		\draw (24.center) to (18.center);
		\draw (18.center) to (14.center);
		\draw (14.center) to (22.center);
		\draw [in=90, out=0, looseness=0.90] (0.center) to (26.center);
		\draw [in=0, out=-90, looseness=0.90] (26.center) to (20.center);
	\end{pgfonlayer}
\end{tikzpicture}
  \end{aligned}
  \qquad
  =
  \qquad
  \begin{aligned}
\begin{tikzpicture}[scale=.65]
	\begin{pgfonlayer}{nodelayer}
		\node [style=none] (0) at (-0.25, -0.5) {};
		\node [style=none] (1) at (-2, -0) {};
		\node [style=none] (2) at (-1.25, -0.5) {};
		\node [style=none] (3) at (1.5, 0.5) {};
		\node [style=none] (4) at (-1, -0.5) {};
		\node [style=none] (5) at (-1.25, 0.5) {};
		\node [style=none] (6) at (-1, -0.125) {};
		\node [style=none] (7) at (-1, -0.875) {};
		\node [style=none] (8) at (-0.25, -0.875) {};
		\node [style=none] (9) at (-0.25, -0.125) {};
		\node [style=none] (10) at (-0.625, -0.5) {$f$};
		\node [style=none] (11) at (1.75, 0.5) {$X$};
		\node [style=none] (12) at (1, -0.125) {};
		\node [style=none] (13) at (1, -0.5) {};
		\node [style=none] (14) at (0.25, -0.5) {};
		\node [style=none] (15) at (1.5, -0.5) {};
		\node [style=none] (16) at (1, -.875) {};
		\node [style=none] (17) at (0.625, -0.5) {$g$};
		\node [style=none] (18) at (0.25, -0.125) {};
		\node [style=none] (19) at (0.25, -.875) {};
		\node [style=none] (20) at (1.75, -0.5) {$Z$};
	\end{pgfonlayer}
	\begin{pgfonlayer}{edgelayer}
		\draw [in=90, out=180, looseness=0.90] (5.center) to (1.center);
		\draw [in=-90, out=180, looseness=0.90] (2.center) to (1.center);
		\draw (3.center) to (5.center);
		\draw (4.center) to (2.center);
		\draw (6.center) to (7.center);
		\draw (7.center) to (8.center);
		\draw (8.center) to (9.center);
		\draw (9.center) to (6.center);
		\draw (13.center) to (15.center);
		\draw (18.center) to (19.center);
		\draw (19.center) to (16.center);
		\draw (16.center) to (12.center);
		\draw (12.center) to (18.center);
		\draw (0.center) to (14.center);
	\end{pgfonlayer}
\end{tikzpicture}
  \end{aligned}
\]
Here the right hand side is the name of the composite $f \circ g$, while the
left hand side is the monoidal product post-composed with the map
\[
    \tikzset{every path/.style={line width=1.1pt}}
\begin{tikzpicture}[scale=.65]
	\begin{pgfonlayer}{nodelayer}
		\node [style=none] (0) at (0, -0.5) {};
		\node [style=none] (1) at (1.5, 0.5) {};
		\node [style=none] (2) at (1.75, 0.5) {$X$};
		\node [style=none] (3) at (-0.5, -0.5) {$Y$};
		\node [style=none] (4) at (-0.25, -2.5) {};
		\node [style=none] (5) at (1.5, -2.5) {};
		\node [style=none] (6) at (0, -1.5) {};
		\node [style=none] (7) at (1.75, -2.5) {$Z$};
		\node [style=none] (8) at (1, -1) {};
		\node [style=none] (9) at (-0.25, 0.5) {};
		\node [style=none] (10) at (-0.5, 0.5) {$X$};
		\node [style=none] (11) at (-0.5, -2.5) {$Z$};
		\node [style=none] (12) at (-0.5, -1.5) {$Y$};
		\node [style=none] (13) at (-0.25, -1.5) {};
		\node [style=none] (14) at (-0.25, -0.5) {};
	\end{pgfonlayer}
	\begin{pgfonlayer}{edgelayer}
		\draw (4.center) to (5.center);
		\draw [in=90, out=0, looseness=0.90] (0.center) to (8.center);
		\draw [in=0, out=-90, looseness=0.90] (8.center) to (6.center);
		\draw (1.center) to (9.center);
		\draw (6.center) to (13.center);
		\draw (14.center) to (0.center);
	\end{pgfonlayer}
\end{tikzpicture}
\]
Thus this morphism, the product of a cap and two identity maps, enacts the
categorical composition on monoidal products of names. We will make liberal use
of this fact.

\subsection{Coherence}

The lack of naturality of the Frobenius maps in hypergraph categories affects
some common properties of structured categories. For example, it is not always
possible to construct a skeletal hypergraph category hypergraph equivalent to a
given hypergraph category: isomorphic objects may be equipped with `different'
Frobenius monoids.  Similarly, a fully faithful, essentially surjective
hypergraph functor does not necessarily define a hypergraph equivalence of
categories. 

Nonetheless, in this section we prove that every hypergraph category is
hypergraph equivalent to a strict hypergraph category. This coherence result
will be important in proving that every hypergraph category can be constructed
using decorated corelations.

\begin{theorem} \label{thm.stricthypergraphs}
  Every hypergraph category is hypergraph equivalent to a strict hypergraph
  category. Moreover, the objects of this strict hypergraph category form a free
  monoid.
\end{theorem}
\begin{proof}
  Let $(\mc H,\ot)$ be a hypergraph category. As $\mc H$ is a fortiori a
  symmetric monoidal category, a standard construction (see Mac Lane
  \cite[Theorem]{Mac98}) gives an equivalent strict symmetric monoidal category
  $(\mc H_{\mathrm{str}}, \cdot)$ with objects finite lists $[x_1,\dots,x_n]$ of
  objects of $\mc H$ and morphisms $[x_1,\dots,x_n] \to [y_1,\dots,y_m]$ those
  morphisms from $((x_1 \ot x_2) \ot \dots) \ot x_n \to ((y_1 \ot y_2) \ot
  \dots) \ot y_m$ in $\mc H$.  Composition is given by composition in $\mc H$.
  
  The monoidal structure is given as follows. Given a list $X$ of objects in
  $\mc H$, write $PX$ for the corresponding monoidal product in $\mc H$ with all
  open parenthesis at the front.  The monoidal product of objects in $\mc
  H_{\mathrm{str}}$ is given by concatenation $\cdot$ of lists; the monoidal
  unit is the empty list. The monoidal product of two morphisms is given by
  their monoidal product in $\mc H$ pre- and post-composed with the necessary
  canonical maps: given $f\maps X \to Y$ and $g\maps Z \to W$, their product
  $f\cdot g\maps X\cdot Y \to Z \cdot
  W$ is \[ P(X \cdot Y) \longrightarrow PX \ot PY \stackrel{f \ot
  g}{\longrightarrow} PZ \ot PW \longrightarrow P(Z \cdot W).  \] By design, the
  associators and unitors are simply identity maps. The braiding $X \cdot Y \to
  Y \cdot X$ is given by the braiding $PX \ot PY \to PY \ot PX$ in $\mc H$,
  similarly pre- and post-composed with the necessary canonical maps. This
  defines a strict symmetric monoidal category \cite{Mac98}.

  To make $\mc H_{\mathrm{str}}$ into a hypergraph category, we make each
  object $[x_1,\dots,x_n]$ into a special commutative Frobenius monoid in a
  similar way. For example, the multiplication on $[x_1,\dots,x_n]$ is given by 
  \[
    \xymatrixcolsep{-4pt}
    \xymatrix{
      P([x_1,\dots,x_n]\cdot[x_1,&\dots,x_n])=
      ((((((x_1 \ot x_2) \ot \dots) \ot x_n) \ot x_1) \ot x_2) \ot \dots) \ot
      x_n \ar[d] \\
      &(((x_1 \ot x_1) \ot (x_2 \ot x_2)) \ot \dots) \ot (x_n \ot x_n)
      \ar[d]^{((\mu_{x_1} \ot \mu_{x_2}) \ot \dots ) \ot \mu_{x_n}} \\
      &P([x_1,\dots,x_n])=((x_1 \ot x_2) \ot \dots ) \ot x_n 
      \phantom{P([x_1,\dots,x_n])=}
    }
  \]
  where the first map is the canonical map such that each pair of $x_i$'s remains
  in the same order. It is straightforward to check that this defines a
  hypergraph category.
  \begin{center}
    \begin{tabular}{| c | p{.65\textwidth} |}
      \hline
      \multicolumn{2}{|c|}{The strict hypergraph category $(\mc H_{\mathrm{str}},
      \cdot)$} \\
      \hline
      \textbf{objects} & finite lists $[x_1, \dots, x_n]$ of objects of
      $\mathcal H$ \\ 
      \textbf{morphisms} & $\mathrm{hom}_{\mc H_{\mathrm{str}}}\big([x_1, \dots,
      x_n],[y_1, \dots, y_m]\big)$ \newline $= \mathrm{hom}_{\mc H}\big(((x_1 \ot x_2) \ot
      \dots) \ot x_n, ((y_1 \ot y_2) \ot \dots) \ot y_m\big)$\\ 
      \textbf{composition} & composition of corresponding maps in $\mc H$ \\
      \textbf{monoidal product} & concatenation of lists \\
      \textbf{coherence maps} & associators and unitors are strict; braiding is
      inherited from $\mc H$  \\
      \textbf{hypergraph maps} & lists of hypergraph maps in $\mc H$ \\
      \hline
    \end{tabular}
  \end{center}

  Our standard construction further gives strong symmetric monoidal functors
  $P\maps \mc H_{\mathrm{str}} \to \mc H$ extending the map $P$ above, and
  $S\maps \mc H \to \mc H_{\mathrm{str}}$ sending $x \in \mc H$ to the string
  $[x]$ of length 1 in $\mc H_{\mathrm{str}}$. These extend to hypergraph
  functors.

  In detail, the functor $P$ is given on morphisms by taking a map in
  $\mathrm{hom}_{\mc H_{\mathrm{str}}}(X,Y)$ to the same map considered now as a
  map in $\mathrm{hom}_{\mc H}(PX,PY)$; its coherence maps are given by the
  canonical maps $PX \ot PY \to P(X\cdot Y)$. The functor $S$ is even easier to
  define: a morphism $x \to y$ in $\mc H$ is by definition a morphism $[x] \to
  [y]$ in $\mc H_{\mathrm{str}}$, so $S$ is a monoidal embedding of $\mc H$ into
  $\mc H_{\mathrm{str}}$. 
  
  By Mac Lane's proof of the coherence theorem for monoidal categories these are
  both strong monoidal functors; by inspection they also preserve hypergraph
  structure, and so are hypergraph functors.  As they already witness an
  equivalence of symmetric monoidal categories, thus $\mc H$ and $\mc
  H_{\mathrm{str}}$ are equivalent as hypergraph categories.
\end{proof}

\section{Example: cospan categories}

A central example of a hypergraph category is the category
$\mathrm{Cospan(\mathcal C)}$ of cospans in any category $\mathcal C$ with
finite colimits. We will later see that decorated cospan categories are a
generalisation of such categories, and each inherits a hypergraph structure
from such. 

We first recall the basic definitions. Let $\mc C$ be a category with finite
colimits, writing the coproduct $+$. A \define{cospan}
\[
  \xymatrix{
    & N \\
    X \ar[ur]^{i} && Y \ar[ul]_{o}
  }
\]
from $X$ to $Y$ in $\mathcal C$ is a pair of morphisms with common codomain. We
refer to $X$ and $Y$ as the \define{feet}, and $N$ as the \define{apex}.  Given
two cospans $X \stackrel{i}{\longrightarrow} N \stackrel{o}{\longleftarrow} Y$
and $X \stackrel{i'}{\longrightarrow} N' \stackrel{o'}{\longleftarrow} Y$ with
the same feet, a \define{map of cospans} is a morphism $n\colon  N \to N'$ in
$\mathcal C$ between the apices such that
\[
  \xymatrix{
    & N \ar[dd]^n  \\
    X \ar[ur]^{i} \ar[dr]_{i'} && Y \ar[ul]_{o} \ar[dl]^{o'}\\
    & N'
  }
\]
commutes.

Cospans may be composed using the pushout from the common
foot: given cospans $X \stackrel{i_X}{\longrightarrow} N
\stackrel{o_Y}{\longleftarrow} Y$ and $Y \stackrel{i_Y}{\longrightarrow} M
\stackrel{o_Z}{\longleftarrow} Z$, their composite cospan is $X \stackrel{j_N
\circ i_X}{\longrightarrow} N+_YM \stackrel{j_M\circ i_Z}{\longleftarrow} Z$,
where the top part of
\[
  \xymatrix{
    && N+_YM \\
    & N \ar[ur]^{j_N} && M \ar[ul]_{j_M} \\
    \quad X \quad \ar[ur]^{i_X} && Y \ar[ul]_{o_Y} \ar[ur]^{i_Y} && \quad Z \quad \ar[ul]_{o_Z}
  }
\]
is a pushout square. This composition rule is associative up to isomorphism, and
so we may define a category, in fact a symmetric monoidal bicategory,
$\mathrm{Cospan}(\mathcal C)$ with objects the objects of $\mathcal C$ and
morphisms isomorphism classes of cospans \cite{Ben}.

The symmetric monoidal structure is `inherited' from $\mc C$. Indeed, we shall
consider any category $\mc C$ with finite colimits a symmetric monoidal category
as follows. Given maps $f \maps A \to C$, $g \maps B \to C$ with common
codomain, the universal property of the coproduct gives a unique map $A+B \to
C$. We call this the \define{copairing} of $f$ and $g$, and write it $[f,g]$. The
monoidal product on $\mc C$ is then given by the coproduct $+$, with monoidal
unit the initial object $\varnothing$ and coherence maps given by copairing the
appropriate identity, inclusion, and initial object maps. For example, the
braiding is given by $[\iota_X,\iota_Y]\maps X+Y \to Y+X$ where $\iota_X\maps X
\to Y+X$ and $\iota_Y\maps Y \to Y+X$ are the inclusion maps into the coproduct
$Y+X$.

The category $\mathrm{Cospan(\mathcal C)}$ inherits this symmetric monoidal
structure from $\mathcal C$ as follows. Call a subcategory $\mathcal C$ of a
category $\mathcal D$ \define{wide} if $\mathcal C$ contains all objects of
$\mathcal D$, and call a functor that is faithful and bijective-on-objects a
\define{wide embedding}.  Note then that we have a wide embedding
\[
  \mathcal C \hooklongrightarrow \mathrm{Cospan(\mathcal C)}
\]
that takes each object of $\mathcal C$ to itself as an object of
$\mathrm{Cospan(\mathcal C)}$, and each morphism $f\colon  X \to Y$ in $\mathcal
C$ to the cospan
\[
  \xymatrix{
    & Y \\
    X \ar[ur]^{f} && Y, \ar@{=}[ul]
  }
\]
where the extended `equals' sign denotes an identity morphism. Now since the
monoidal product $+\colon \mathcal C \times \mathcal C \to \mathcal C$ is left
adjoint to the diagram functor, it preserves colimits, and so extends to a
functor $+\colon \mathrm{Cospan(\mathcal C)} \times \mathrm{Cospan(\mathcal C)}
\to \mathrm{Cospan(\mathcal C)}$. The coherence maps are just the images of the
coherence maps in $\mathcal C$ under this wide embedding; checking naturality is
routine, and clearly they still obey the required axioms.

Write $\FinSet$ for the category of finite sets and functions. Replacing
$\FinSet$ with its equivalent strict skeleton, the following proposition is
well-known. 
\begin{proposition} \label{prop.cospanscfm}
  Special commutative Frobenius monoids in a strict symmetric monoidal category
$\mc C$ are in one-to-one correspondence with strict symmetric monoidal functors
$\cospan(\FinSet) \to \mc C$.
\end{proposition}
\begin{proof}
  The special commutative Frobenius monoid is given by the image of the one
  element set. See Lack \cite{Lac04}.
\end{proof}
Over the next few chapters we will further explore this deep link between
cospans and special commutative Frobenius monoids and hypergraph categories. To
begin, we detail a natural hypergraph structure on $\cospan(\mc C)$. 

This hypergraph structure also comes from copairings of identity morphisms.
Call cospans 
\[
  \xymatrix{
    & N \\
    X \ar[ur]^{i} && Y \ar[ul]_{o}
  }
  \qquad \xymatrix@R=8pt{\\\textrm{and}} \qquad 
  \xymatrix{
    & N \\
    Y \ar[ur]^{o} && X \ar[ul]_{i}
  }
\]
that are reflections of each other \define{opposite} cospans. Given any object
$X$ in $\mathcal C$, the copairing $[1_X,1_X]\colon  X + X \to X$ of two identity
maps on $X$, together with the unique map $!\colon  \varnothing \to X$ from the
initial object to $X$, define a monoid structure on $X$. Considering these
maps as morphisms in $\mathrm{Cospan(\mathcal C)}$, we may take them together
with their opposites to give a special commutative Frobenius structure on $X$.
It is easily verified that this gives a hypergraph category.

Given $f \maps X \to Y$ in $\mc C$, abuse notation by writing $f \in
\mathrm{Cospan}(\mc C)$ for the cospan $X \stackrel{f}\to Y
\stackrel{1_Y}\leftarrow Y$, and $f^\opp$ for the cospan $Y \stackrel{1_Y}\to Y
\stackrel{f}\leftarrow X$. To summarise:
  \begin{center}
  \begin{tabular}{ |c| p{.65\textwidth}|}
      \hline
      \multicolumn{2}{|c|}{The hypergraph category $(\mathrm{Cospan(\mc C)},+)$} \\
    \hline
    \textbf{objects} & the objects of $\mathcal C$ \\ 
    \textbf{morphisms} & isomorphism classes of cospans in
    $\mathcal C$\\ 
  \textbf{composition} & given by pushout \\
  \textbf{monoidal product} & the coproduct in $\mathcal C$. \\
  \textbf{coherence maps} & inherited from $(\mc C,+)$\\
  \textbf{hypergraph maps} & $\mu = [1,1]$, $\eta = !$,
  $\delta = [1,1]^\opp$, $\epsilon = !^\opp$. \\
      \hline
  \end{tabular}
\end{center}
   
We will often abuse our terminology and refer to cospans themselves as
morphisms in some cospan category $\mathrm{Cospan}(\mathcal C)$; we of course
refer instead to the isomorphism class of the said cospan.
 
It is not difficult to show that we in fact have a functor from the category
having categories with finite colimits as objects and colimit preserving
functors as morphisms to the category of hypergraph categories and hypergraph
functors. In the next chapter we show that this extends to a functor from the
category of symmetric lax monoidal presheaves over categories with finite
colimits. This is known as the decorated cospans construction.

%Walters: cospan graph is the generic special commutative Frobenius monoid.  The
%free hypergraph category on a single object in the category of cospans in the
%category of finite sets.

The decorated cospans theorem then leads to the decorated corelations
construction, which gives a way to build every hypergraph category using
cospans. Write $\mathrm{Mon}_C$ for the category with objects functions $[n]=\{1,2,
\dots,n \} \to C$, where $C$ is some set, and morphisms functions $[n] \to [m]$
that commute over $C$. We will prove the following.
\begin{theorem}
  The category of hypergraph categories is equivalent to the category of
  lax symmetric monoidal functors
  \[
    (\mathrm{Cospan}(\FinSet_{\mc O}),+) \to (\Set,\times).
  \]
  varying over sets $\mc O$.
\end{theorem}
This theorem, couched in the language of algebras for operads, is also the
subject of a forthcoming paper by Vagner, Spivak, and Schultz \cite{VagSpiSch}.


%\begin{proposition}
%  The monoidal product in a hypergraph category is a coproduct for something if
%  and only if every morphism is a Frobenius monoid homomorphism. (ie take
%  subcategory of all objects, monoid maps, and monoid homs. This is cocomplete.)
%\end{proposition}
%Clearly not true for circuits for example.


\chapter{Decorated cospans: from closed systems to open}

In the previous chapter we discussed hypergraph categories as a model for
network languages, and showed that cospans provide a method of constructing such
categories. In many situations, however, we wish to model our systems not just
with cospans, but `decorated' cospans, where the apex of each cospan is equipped
with some extra structure. In this chapter we detail a method for composing such
decorated cospans. 

%\begin{theorem}
%  The category of symmetric lax monoidal presheaves over finitely complete
%  categories embeds faithfully into $\mathrm{HypCat}$.
%\end{theorem}

This chapter is based on \cite{fong_decorated_2015}.


\section{From closed systems to open}

The key idea in this chapter is that we may take our network diagrams, mark
points of interconnection, or terminals, using cospans, and connect these
terminals with others using pushouts. This marking of terminals defines a
boundary of the `system', turning it from a closed system to an open system.
This process `freely' constructs a hypergraph category from a library of network
pieces.

It has been recognised for some time that spans and cospans provide an intuitive
framework for composing network diagrams \cite{KSW}, and the material we develop
here is a variant on this theme. In the case of finite graphs, the intuition
reflected is this: given two graphs, we may construct a third by gluing chosen
vertices of the first with chosen vertices of the second. It is our goal in this
article to view this process as composition of morphisms in a category, in a way
that also facilitates the construction of a composition rule for any semantics
associated to the diagrams, and a functor between these two resulting
categories.

To see how this works, let us start with the following graph:
\begin{center}
  \begin{tikzpicture}[auto,scale=2.3]
    \node[circle,draw,inner sep=1pt,fill]         (A) at (0,0) {};
    \node[circle,draw,inner sep=1pt,fill]         (B) at (1,0) {};
    \node[circle,draw,inner sep=1pt,fill]         (C) at (0.5,-.86) {};
    \path (B) edge  [bend right,->-] node[above] {0.2} (A);
    \path (A) edge  [bend right,->-] node[below] {1.3} (B);
    \path (A) edge  [->-] node[left] {0.8} (C);
    \path (C) edge  [->-] node[right] {2.0} (B);
  \end{tikzpicture}
\end{center}
We shall work with labelled, directed graphs, as the additional data help
highlight the relationships between diagrams. Now, for this graph to be a
morphism, we must equip it with some notion of `input' and `output'. We do this by
marking vertices using functions from finite sets:
\begin{center}
  \begin{tikzpicture}[auto,scale=2.15]
    \node[circle,draw,inner sep=1pt,fill=gray,color=gray]         (x) at (-1.4,-.43) {};
    \node at (-1.4,-.9) {$X$};
    \node[circle,draw,inner sep=1pt,fill]         (A) at (0,0) {};
    \node[circle,draw,inner sep=1pt,fill]         (B) at (1,0) {};
    \node[circle,draw,inner sep=1pt,fill]         (C) at (0.5,-.86) {};
    \node[circle,draw,inner sep=1pt,fill=gray,color=gray]         (y1) at (2.4,-.25) {};
    \node[circle,draw,inner sep=1pt,fill=gray,color=gray]         (y2) at (2.4,-.61) {};
    \node at (2.4,-.9) {$Y$};
    \path (B) edge  [bend right,->-] node[above] {0.2} (A);
    \path (A) edge  [bend right,->-] node[below] {1.3} (B);
    \path (A) edge  [->-] node[left] {0.8} (C);
    \path (C) edge  [->-] node[right] {2.0} (B);
    \path[color=gray, very thick, shorten >=10pt, shorten <=5pt, ->, >=stealth] (x) edge (A);
    \path[color=gray, very thick, shorten >=10pt, shorten <=5pt, ->, >=stealth] (y1) edge (B);
    \path[color=gray, very thick, shorten >=10pt, shorten <=5pt, ->, >=stealth] (y2) edge (B);
  \end{tikzpicture}
\end{center}
Let $N$ be the set of vertices of the graph. Here the finite sets $X$, $Y$, and
$N$ comprise one, two, and three elements respectively, drawn as points, and the
values of the functions $X \to N$ and $Y \to N$ are indicated by the grey
arrows. This forms a cospan in the category of finite sets, one with the set at
the apex decorated by our given graph.

Given another such decorated cospan with input set equal to the output of the
above cospan
\begin{center}
  \begin{tikzpicture}[auto,scale=2.15]
    \node[circle,draw,inner sep=1pt,fill=gray,color=gray]         (y1) at (-1.4,-.25) {};
    \node[circle,draw,inner sep=1pt,fill=gray,color=gray]         (y2) at (-1.4,-.61) {};
    \node at (-1.4,-.9) {$Y$};
    \node[circle,draw,inner sep=1pt,fill]         (A) at (0,0) {};
    \node[circle,draw,inner sep=1pt,fill]         (B) at (1,0) {};
    \node[circle,draw,inner sep=1pt,fill]         (C) at (0.5,-.86) {};
    \node[circle,draw,inner sep=1pt,fill=gray,color=gray]         (z1) at (2.4,-.25) {};
    \node[circle,draw,inner sep=1pt,fill=gray,color=gray]         (z2) at (2.4,-.61) {};
    \node at (2.4,-.9) {$Z$};
    \path (A) edge  [->-] node[above] {1.7} (B);
    \path (C) edge  [->-] node[right] {0.3} (B);
    \path[color=gray, very thick, shorten >=10pt, shorten <=5pt, ->, >=stealth] (y1) edge (A);
    \path[color=gray, very thick, shorten >=10pt, shorten <=5pt, ->, >=stealth] (y2)
    edge (C);
    \path[color=gray, very thick, shorten >=10pt, shorten <=5pt, ->, >=stealth] (z1) edge (B);
    \path[color=gray, very thick, shorten >=10pt, shorten <=5pt, ->, >=stealth] (z2) edge (C);
  \end{tikzpicture}
\end{center}
composition involves gluing the graphs along the identifications
\begin{center}
  \begin{tikzpicture}[auto,scale=2.15]
    \node[circle,draw,inner sep=1pt,fill=gray,color=gray]         (x) at (-1.3,-.43) {};
    \node at (-1.3,-.9) {$X$};
    \node[circle,draw,inner sep=1pt,fill]         (A) at (0,0) {};
    \node[circle,draw,inner sep=1pt,fill]         (B) at (1,0) {};
    \node[circle,draw,inner sep=1pt,fill]         (C) at (0.5,-.86) {};
    \node[circle,draw,inner sep=1pt,fill=gray,color=gray]         (y1) at (2.3,-.25) {};
    \node[circle,draw,inner sep=1pt,fill=gray,color=gray]         (y2) at (2.3,-.61) {};
    \node at (2.3,-.9) {$Y$};
    \path (B) edge  [bend right,->-] node[above] {0.2} (A);
    \path (A) edge  [bend right,->-] node[below] {1.3} (B);
    \path (A) edge  [->-] node[left] {0.8} (C);
    \path (C) edge  [->-] node[right] {2.0} (B);
    \path[color=gray, very thick, shorten >=10pt, shorten <=5pt, ->, >=stealth] (x) edge (A);
    \path[color=gray, very thick, shorten >=10pt, shorten <=5pt, ->, >=stealth] (y1) edge (B);
    \path[color=gray, very thick, shorten >=10pt, shorten <=5pt, ->, >=stealth] (y2) edge (B);
    \node[circle,draw,inner sep=1pt,fill]         (A') at (3.6,0) {};
    \node[circle,draw,inner sep=1pt,fill]         (B') at (4.6,0) {};
    \node[circle,draw,inner sep=1pt,fill]         (C') at (4.1,-.86) {};
    \node[circle,draw,inner sep=1pt,fill=gray,color=gray]         (z1) at (5.9,-.25) {};
    \node[circle,draw,inner sep=1pt,fill=gray,color=gray]         (z2) at (5.9,-.61) {};
    \node at (5.9,-.9) {$Z$};
    \path (A') edge  [->-] node[above] {1.7} (B');
    \path (C') edge  [->-] node[right] {0.3} (B');
    \path[color=gray, very thick, shorten >=10pt, shorten <=5pt, ->, >=stealth] (y1) edge (A');
    \path[color=gray, very thick, shorten >=10pt, shorten <=5pt, ->, >=stealth] (y2)
    edge (C');
    \path[color=gray, very thick, shorten >=10pt, shorten <=5pt, ->, >=stealth] (z1) edge (B');
    \path[color=gray, very thick, shorten >=10pt, shorten <=5pt, ->, >=stealth] (z2) edge (C');
  \end{tikzpicture}
\end{center}
specified by the shared foot of the two cospans. This results in the decorated
cospan
\begin{center}
  \begin{tikzpicture}[auto,scale=2.15]
    \node[circle,draw,inner sep=1pt,fill=gray,color=gray]         (x) at (-1.4,-.43) {};
    \node at (-1.4,-.9) {$X$};
    \node[circle,draw,inner sep=1pt,fill]         (A) at (0,0) {};
    \node[circle,draw,inner sep=1pt,fill]         (B) at (1,0) {};
    \node[circle,draw,inner sep=1pt,fill]         (C) at (0.5,-.86) {};
    \node[circle,draw,inner sep=1pt,fill]         (D) at (2,0) {};
    \node[circle,draw,inner sep=1pt,fill=gray,color=gray]         (z1) at (3.4,-.25) {};
    \node[circle,draw,inner sep=1pt,fill=gray,color=gray]         (z2) at (3.4,-.61) {};
    \node at (3.4,-.9) {$Z$};
    \path (B) edge  [bend right,->-] node[above] {0.2} (A);
    \path (A) edge  [bend right,->-] node[below] {1.3} (B);
    \path (A) edge  [->-] node[left] {0.8} (C);
    \path (C) edge  [->-] node[right] {2.0} (B);
    \path (B) edge  [bend left,->-] node[above] {1.7} (D);
    \path (B) edge  [bend right,->-] node[below] {0.3} (D);
    \path[color=gray, very thick, shorten >=10pt, shorten <=5pt, ->, >=stealth] (x) edge (A);
    \path[color=gray, very thick, shorten >=10pt, shorten <=5pt, ->, >=stealth] (z1)
    edge (D);
    \path[bend left, color=gray, very thick, shorten >=10pt, shorten <=5pt, ->, >=stealth] (z2)
    edge (B);
  \end{tikzpicture}
\end{center}
The decorated cospan framework generalises this intuitive construction.

More precisely: fix a set $L$. Then given a finite set $N$, we may talk of the
collection of finite $L$-labelled directed multigraphs, to us just $L$-graphs
or simply graphs, that have $N$ as their set of vertices. Write such a graph
$(N,E,s,t,r)$, where $E$ is a finite set of edges, $s\colon E \to N$ and $t\colon E \to N$
are functions giving the source and target of each edge respectively, and $r\colon  E
\to L$ equips each edge with a label from the set $L$.  Next, given a function
$f\colon N \to M$, we may define a function from graphs on $N$ to graphs on $M$
mapping $(N,E,s,t,r)$ to $(M,E,f \circ s,f \circ t, r)$.  After dealing
appropriately with size issues, this gives a lax monoidal functor from
$(\FinSet,+)$ to $(\Set,\times)$.\footnote{Here $(\FinSet,+)$ is the monoidal
  category of finite sets and functions with disjoint union as monoidal
  product, and $(\Set,\times)$ is the category of sets and functions with
  cartesian product as monoidal product. One might ensure the collection of
  graphs forms a set in a number of ways. One such method is as follows: the
  categories of finite sets and finite graphs are essentially small; replace
  them with equivalent small categories. We then constrain the graphs
  $(N,E,s,t,r)$ to be drawn only from the objects of our small category of finite
  graphs.}  

Now, taking any lax monoidal functor $(F,\varphi)\colon  (\mathcal C,+) \to (\mathcal
D,\otimes)$ with $\mathcal C$ having finite colimits and coproduct written $+$,
the decorated cospan category associated to $F$ has as objects the objects of
$\mathcal C$, and as morphisms pairs comprising a cospan in $\mathcal C$
together with some morphism $1 \to FN$, where $1$ is the unit in $(\mathcal
D,\otimes)$ and $N$ is the apex of the cospan. In the case of our graph
functor, this additional data is equivalent to equipping the apex $N$ of the
cospan with a graph. We thus think of our morphisms as having two distinct
parts: an instance of our chosen structure on the apex, and a cospan describing
interfaces to this structure. Our first theorem says that when $(\mathcal
D,\otimes)$ is braided monoidal and $(F,\varphi)$ lax braided monoidal, we may
further give this data a composition rule and monoidal product such that the
resulting `decorated cospan category' is hypergraph.  

Suppose now we have two such lax monoidal functors; we then have two such
decorated cospan categories. Our second theorem is that, given also a monoidal
natural transformation between these functors, we may construct a strict
hypergraph functor between their corresponding decorated cospan categories.  These
natural transformations can often be specified by some semantics associated to
some type of topological diagram. A trivial case of such is assigning to a
finite graph its number of vertices, but richer examples abound, including
assigning to a directed graph with edges labelled by rates its depicted Markov
process, or assigning to an electrical circuit diagram the current--voltage
relationship such a circuit would impose.

The structure of this chapter is straightforward: in the following section we
review some basic background material, which then allows us to give the
constructions of decorated cospan categories and their functors in Sections
\ref{sec:dcc} and \ref{sec:dcf} respectively. We then explicate these
definitions through some examples in Section \ref{sec:ex}. 

\section{Decorated cospan categories} \label{sec:dcc}
Motivated by the previous section, we make the following definition.
\begin{definition} \label{def:fcospans}
  Let $\mathcal C$ be a category with finite colimits, and
  \[
    (F,\varphi)\colon  (\mathcal C,+) \longrightarrow (\mathcal D, \otimes)
  \]
  be a lax monoidal functor. We define a \define{decorated cospan}, or more
  precisely an $F$-decorated cospan, to be a pair 
  \[
    \left(
    \begin{aligned}
      \xymatrix{
	& N \\  
	X \ar[ur]^{i} && Y \ar[ul]_{o}
      }
    \end{aligned}
    ,
    \qquad
    \begin{aligned}
      \xymatrix{
	FN \\
	1 \ar[u]_{s}
      }
    \end{aligned}
    \right)
  \]
  comprising a cospan $X \stackrel{i}\rightarrow N \stackrel{o}\leftarrow Y$ in
  $\mathcal C$ together with an element $1 \stackrel{s}\rightarrow FN$ of
  the $F$-image $FN$ of the apex of the cospan. We shall call the element $1
  \stackrel{s}\rightarrow FN$ the \define{decoration} of the decorated
  cospan. A \define{morphism of decorated cospans}
  \[
    n\colon  \big(X \stackrel{i_X}\longrightarrow N \stackrel{o_Y}\longleftarrow
    Y,\enspace 1 \stackrel{s}\longrightarrow FN\big) \longrightarrow \big(X
    \stackrel{i'_X}\longrightarrow N' \stackrel{o'_Y}\longleftarrow Y,\enspace 1
    \stackrel{s'}\longrightarrow FN'\big)
  \]
  is a morphism $n\colon  N \to N'$ of cospans such that $Fn \circ s = s'$.
\end{definition}
In this section the task is to prove that decorated cospans can be used to
construct hypergraph categories.

\subsection{Composing decorated cospans}

First, we show that isomorphism classes of decorated cospans form the morphisms
of a category.

\begin{proposition}
  There is a category $F\mathrm{Cospan}$ of $F$-decorated cospans, with objects
  the objects of $\mathcal C$, and morphisms isomorphism classes of
  $F$-decorated cospans. On representatives of the isomorphism classes,
  composition in this category is given by pushout of cospans in $\mathcal C$
  \[
    \xymatrix{
      && N+_YM \\
      & N \ar[ur]^{j_N} && M \ar[ul]_{j_M} \\
      \quad X \quad \ar[ur]^{i_X} && Y \ar[ul]_{o_Y} \ar[ur]^{i_Y} && \quad Z
      \quad \ar[ul]_{o_Z}
    }
  \]
  paired with the composite
  \[
    1 \stackrel{\lambda^{-1}}\longrightarrow 1 \otimes 1 \stackrel{s \otimes
    t}\longrightarrow FN \otimes FM \stackrel{\varphi_{N,M}}\longrightarrow
    F(N+M) \stackrel{F[j_N,j_M]}\longrightarrow F(N+_YM)
  \]
  of the tensor product of the decorations with the $F$-image of the copairing
  of the pushout maps.
\end{proposition}

\begin{proof}
  The identity morphism on an object $X$ in a decorated cospan category is
  simply the identity cospan decorated as follows:
  \[
    \left(
    \begin{aligned}
      \xymatrix{
	& X \\  
	X \ar@{=}[ur] && X \ar@{=}[ul]
      }
    \end{aligned}
    ,
    \qquad
    \begin{aligned}
      \xymatrixrowsep{.5pc}
      \xymatrix{
	FX \\
	F\varnothing \ar[u]_{F!} \\
	1 \ar[u]_{\varphi_1}
      }
    \end{aligned}
    \right).
  \]
  We must check that the composition defined is well-defined on isomorphism
  classes, is associative, and, with the above identity maps, obeys the
  unitality axiom. These are straightforward, but lengthy, exercises in
  using the available colimits and monoidal structure to show that
  the relevant diagrams of decorations commute.

\paragraph{Representation independence for composition of isomorphism
classes of decorated cospans.}
Let 
\[
  n\colon  \big(X \stackrel{i_X}\longrightarrow N \stackrel{o_Y}\longleftarrow Y,\enspace 1
\stackrel{s}\longrightarrow FN\big) \longrightarrow \big(X \stackrel{i'_X}\longrightarrow N'
\stackrel{o'_Y}\longleftarrow Y,\enspace 1 \stackrel{s'}\longrightarrow FN'\big)
\]
and
\[
  m\colon  \big(Y \stackrel{i_Y}\longrightarrow M \stackrel{o_Z}\longleftarrow Z,\enspace 1
\stackrel{t}\longrightarrow FM\big) \longrightarrow \big(Y \stackrel{i'_Y}\longrightarrow M'
\stackrel{o'_Z}\longleftarrow Z,\enspace 1 \stackrel{t'}\longrightarrow FM'\big)
\]
be isomorphisms of decorated cospans. We wish to show that the composite of the
decorated cospans on the left is isomorphic to the composite of the decorated
cospans on the right. As discussed, it is well-known that the composite cospans
are isomorphic, and it remains to us to check the decorations agree too. Let $p\colon 
N+_YM \to N'+_YM'$ be the isomorphism given by the universal property of the
pushout and the isomorphisms $n\colon N \to N'$ and $m\colon  M \to M'$. Then the two
decorations in question are given by the top and bottom rows of the following
diagram.
\[
  \xymatrixrowsep{1pc}
  \xymatrixcolsep{1pc}
  \xymatrix{
    &&&& FN \otimes FM \ar[dd]^{Fn \otimes Fm}_\sim
    \ar[rr]^{\varphi_{N,M}} && F(N+M) \ar[dd]^{F(n+m)}_\sim
    \ar[rr]^{F[j_N,j_M]} && F(N+_YM) \ar[dd]^{Fp}_\sim \\ 
    1 \ar[rr]^(.4){\lambda^{-1}} && 1\otimes 1 \ar[urr]^{s \otimes t}
    \ar[drr]_{s' \otimes t'} &
    \qquad\textrm{\tiny(I)} && \textrm{\tiny(F)} && \textrm{\tiny(C)}\\ 
    &&&& FN' \otimes FM' \ar[rr]_{\varphi_{N',M'}} && F(N'+M')
    \ar[rr]_{F[j_{N'},j_{M'}]} && F(N'+_YM')
  }
\]
The triangle (I) commutes as $n$ and $m$ are morphisms of decorated cospans and
$- \otimes -$ is functorial, (F) commutes by the monoidality of $F$, and (C)
commutes by properties of colimits in $\mathcal C$ and the functoriality of $F$.
This proves the claim.

\paragraph{Associativity.}
Suppose we have morphisms
\[
  (X \stackrel{i_X}\longrightarrow N \stackrel{o_Y}\longleftarrow Y,\enspace 1
  \stackrel{s}\longrightarrow FN),
\]
\[
  (Y \stackrel{i_Y}\longrightarrow M \stackrel{o_Z}\longleftarrow Z,\enspace 1
  \stackrel{t}\longrightarrow FM), 
\]
\[
  (Z \stackrel{i_Z}\longrightarrow P \stackrel{o_W}\longleftarrow W,\enspace 1
  \stackrel{u}\longrightarrow FP).
\]
It is well-known that composition of isomorphism classes of cospans via
pushout of representatives is associative; this follows from the universal
properties of the relevant colimit. We must check that the pushforward of the
decorations is also an associative process. Write 
\[
  \tilde a\colon  (N+_YM)+_ZP \longrightarrow N+_Y(M+_ZP)
\]
for the unique isomorphism between the two pairwise pushouts constructions
from the above three cospans. Consider then the following diagram, with leftmost
column the decoration obtained by taking the composite of the first two morphisms
first, and the rightmost column the decoration obtained by taking the composite
of the last two morphisms first.
\[
  \xymatrixcolsep{-.6pc}
  \xymatrix{ 
    \scriptstyle F((N+_YM)+_ZP) \ar[rrrrrrrr]^{F\tilde a} &&&&&&&&
    \scriptstyle F(N+_Y(M+_ZP)) \\
    &&&& \textrm{\tiny(C)} \\
    \scriptstyle F((N+_YM)+P) \ar[uu]^{F[j_{N+_YM},j_P]} &&&&&&&& \scriptstyle
    F(N+(M+_ZP)) \ar[uu]_{F[j_N,j_{M+_ZP}]} \\
    && \scriptstyle F((N+M)+P) \ar[ull]_(.35)*+<6pt>_{\scriptstyle F([j_N,J_M]+1_P)}
    \ar[rrrr]^{Fa} &&&& \scriptstyle F(N+(M+P))
    \ar[urr]^(.35)*+<6pt>^{\scriptstyle F(1_N+[j_M,j_P])} \\
    & \textrm{\tiny(F2)} &&&&&& \textrm{\tiny(F3)} \\
    \scriptstyle F(N+_YM)\otimes FP \ar[uuu]^{\varphi_{N+_YM,P}} &&&& 
    \textrm{\tiny(F1)} &&&& \scriptstyle FN\otimes F(M+_ZP)
    \ar[uuu]_{\varphi_{N,M+_ZP}} \\
    && \scriptstyle F(N+M)\otimes FP \ar[ull]^(.6)*+<6pt>^{\scriptstyle
      F[j_N,j_M] \otimes 1_{FP}} \ar[uuu]_{\varphi_{N+M,P}} &&&& \scriptstyle
      FN \otimes F(M+P) \ar[urr]_(.6)*+<6pt>_{\scriptstyle 1_{FN} \otimes
    F[j_M,j_P]} \ar[uuu]^{\varphi_{N,M+P}} \\
    &&& \scriptstyle (FN \otimes FM) \otimes FP \ar[ul]^{\varphi_{N,M} \otimes
    1_{FP}} \ar[rr]^{a} && \scriptstyle FN \otimes (FM \otimes FP)
    \ar[ur]_{\phantom{1}1_{FN} \otimes \varphi_{M,P}} \\ 
    &&&& \textrm{\tiny(D2)} \\
    &&& \scriptstyle (1 \otimes 1) \otimes 1 \ar[uu]^{(s \otimes t) \otimes u}
    \ar[rr]^{a} && \scriptstyle 1 \otimes (1 \otimes 1) \ar[uu]_{s
    \otimes (t \otimes u)} \\
    &&&& \textrm{\tiny(D1)} \\
    &&&& \scriptstyle 1 \otimes 1 \ar[uul]^{\rho^{-1} \otimes 1} \ar[uur]_{1
      \otimes \lambda^{-1}} \\
      &&&& \scriptstyle 1 \ar[u]^{\lambda^{-1}}
  }
\]
This diagram commutes as (D1) is the triangle coherence equation for the
monoidal category $(\mathcal D,\otimes)$, (D2) is naturality for the associator
$a$, (F1) is the associativity condition for the monoidal functor $F$, (F2)
and (F3) commute by the naturality of $\varphi$, and (C) commutes as it is the
$F$-image of a hexagon describing the associativity of the pushout.  This
shows that the two decorations obtained by the two different orders of
composition of our three morphisms are equal up to the unique isomorphism
$\tilde a$ between the two different pushouts that may be obtained. Our
composition rule is hence associative.

\paragraph{Identity morphisms.} 
We shall show that the claimed identity morphism on $Y$, the decorated cospan
\[
  (Y \stackrel{1_Y}\longrightarrow Y \stackrel{1_Y}\longleftarrow Y,\enspace 1
\stackrel{F!\circ \varphi_1}\longrightarrow FY),
\]
is an identity for composition
on the right; the case for composition on the left is similar. The cospan in
this pair is known to be the identity cospan in $\mathrm{Cospan}(\mathcal C)$.
We thus need to check that, given a morphism 
\[
  (X \stackrel{i}\longrightarrow N
\stackrel{o}\longleftarrow Y,\enspace 1 \stackrel{s}\longrightarrow FN),
\] 
the composite of the product $s \otimes (F! \circ \varphi_1)$ with the $F$-image
of the copairing $[1_N,i_Y]\colon  N+Y \to N$ of the pushout maps is again the same
element $s$; this composite being, by definition, the decoration of the
composite of the given morphism and the claimed identity map. This is shown by
the commutativity of the diagram below, with the path along the lower edge equal
to the aforementioned pushforward.
\[
  \xymatrixcolsep{1.1pc}
  \xymatrix{
    1 \ar[rrrrrrrrrr]^s \ar[ddrr]_{\rho_1^{-1}} &&&&&&&&&& FN \\
    &&&& \textrm{\tiny(D1)} &&&& \textrm{\tiny(F1)}\\
    && 1 \otimes 1 \ar[rr]^{s\otimes 1} \ar[ddrrrr]_{s \otimes (F! \circ
    \varphi_1)} && FN \otimes 1 \ar[uurrrrrr]^{\rho_{FN}} \ar[rr]^(.57){1_{FN}
  \otimes \varphi_1}
    && FN \otimes F\varnothing \ar[rr]^{\varphi_{N,\varnothing}} \ar[dd]_{1_{FN}
  \otimes F!} && F(N+\varnothing) \ar[uurr]^(.2){F\rho_N=F[1_N,!]}
      \ar[dd]_{F(1_N+!)} \quad \textrm{\tiny(C)}\\
      &&&&& \textrm{\tiny(D2)} && \textrm{\tiny(F2)}\\
      &&&&&& FN \otimes FY \ar[rr]_{\varphi_{N,M}} && F(N+Y)
      \ar[uuuurr]_{F[1_N,i_Y]}
  }
\]
This diagram commutes as each subdiagram commutes: (D1) commutes by the
naturality of $\rho$, (D2) by the functoriality of the monoidal product in
$\mathcal D$, (F1) by the unit axiom for the monoidal functor $F$, (F2) by the
naturality of $\varphi$, and (C) due to the properties of colimits in $\mathcal
C$ and the functoriality of $F$.

We thus have a category.
\end{proof}

\begin{remark}
While at first glance it might seem surprising that we can construct a
composition rule for decorations $s\colon  1\to FN$ and $t\colon  1 \to FM$ just from
monoidal structure, the copair $[j_N,j_M]\colon  N+M \to N+_YM$ of the pushout maps
contains the data necessary to compose them. Indeed, this is the key insight of the
decorated cospan construction. To wit, the coherence maps for the lax monoidal
functor allow us to construct an element of $F(N+M)$ from the monoidal product
$s \otimes t$ of the decorations, and we may then post-compose with $F[j_N,j_M]$ to
arrive at an element of $F(N+_YM)$. The map $[j_N,j_M]$ encodes the
identification of the image of $Y$ in $N$ with the image of the same in $M$, and
so describes merging the `overlap' of the two decorations.
\end{remark}


\subsection{The hypergraph structure}
Our main theorem is that when \emph{braided} monoidal structure is present, the
category of decorated cospans is a hypergraph category, and moreover one into
which the category of `undecorated' cospans widely embeds.  This embedding
motivates the monoidal and hypergraph structures we put on $F\mathrm{Cospan}$.

\begin{theorem} \label{thm:fcospans}
  Let $\mathcal C$ be a category with finite colimits, $(\mathcal D, \otimes)$ a
  braided monoidal category, and $(F,\varphi)\colon  (\mathcal C,+) \to (\mathcal D,
  \otimes)$ be a lax braided monoidal functor. Then we may give
  $F\mathrm{Cospan}$ a symmetric monoidal and hypergraph structure such that
  there is a wide embedding of hypergraph categories
  \[
    \mathrm{Cospan(\mathcal C)} \hooklongrightarrow F\mathrm{Cospan}.
  \]
\end{theorem}

We first prove a lemma.  Recall that the identity decorated cospan has apex
decorated by $1 \stackrel{\varphi_1}\longrightarrow F\varnothing
\stackrel{F!}\longrightarrow FX$. Given any cospan $X \to N \leftarrow Y$, we
call the decoration $1 \stackrel{\varphi_1}\longrightarrow F\varnothing
\stackrel{F!}\longrightarrow FN$ the \define{empty decoration} on $N$. Empty
decorations are indeed empty: when composed with other decorations, they have no
impact.

\begin{lemma} \label{lem.emptydecorations}
  Let
\[
  (X \stackrel{i_X}\longrightarrow N
\stackrel{o_Y}\longleftarrow Y,\enspace 1 \stackrel{s}\longrightarrow FN),
\]
be a decorated cospan, and suppose we have an empty-decorated cospan 
\[
  (Y \stackrel{i_Y}\longrightarrow M \stackrel{o_Z}\longleftarrow Z,\enspace 1
  \stackrel{\varphi\circ F!}\longrightarrow FM).
\]
Then the composite of these decorated cospans is 
\[
  \big(X \xrightarrow{j_N \circ i_X} N+_YM \xleftarrow{j_M \circ o_Z} Z,\enspace 
  1 \xrightarrow{Fj_N \circ s} F(N+_YM)\big).
\]
In particular, the decoration on the composite is the decoration $s$ pushed
forward along the $F$-image of the map $j_N\colon N \to N+_YM$ to become an
$F$-decoration on $N+_YM$. The analogous statement also holds for composition
with an empty-decorated cospan on the left.
\end{lemma}
\begin{proof}
As is now familiar, a statement of this sort is proved by a large commutative
diagram:
\[
  \xymatrixcolsep{.63pc}
  \xymatrix{
    1 \ar[rrrrrr]^s \ar[ddrr]_{\rho_1^{-1}} &&&&&& FN \ar[rrrr]^{Fj_N} &&&& F(N+_YM) \\
    &&& \textrm{\tiny(D1)} &&& \textrm{\tiny(F1)} && \textrm{\tiny(C1)} \\
    && 1 \otimes 1 \ar[rr]^{s\otimes 1} \ar[ddrrrr]_{s \otimes (F! \circ
    \varphi_1)} && FN \otimes 1 \ar[uurr]^{\rho_{FN}} \ar[rr]^{1_{FN}
  \otimes \varphi_1}
    && FN \otimes F\varnothing \ar[rr]^{\varphi_{N,\varnothing}} \ar[dd]_{1_{FN}
  \otimes F!} && F(N+\varnothing) \ar[uull]_{F\rho_N=F[1_N,!]}
  \ar[uurr]^{F[j_N,!]}
      \ar[dd]_{F(1_N+!)} \quad \textrm{\tiny(C2)}\\
      &&&&& \textrm{\tiny(D2)} && \textrm{\tiny(F2)}\\
      &&&&&& FN \otimes FM \ar[rr]_{\varphi_{N,M}} && F(N+M)
      \ar[uuuurr]_{F[j_N,j_M]}
  }
\]
The subdiagrams in this diagram commute for the same reasons as their
corresponding regions in the previous diagram for identity morphisms. 
\end{proof}

\begin{proof}[Proof of Theorem \ref{thm:fcospans}]
We first prove that we have a wide embedding of categories, and then use this to
transfer monoidal and hypergraph structure onto $F\mathrm{Cospan}$.

\paragraph{Embedding.} 
  We define a functor 
  \[
    \mathrm{Cospan(\mathcal C)} \hooklongrightarrow F\mathrm{Cospan}.
  \]
  mapping each object of $\mathrm{Cospan(\mathcal C)}$ to itself as an object
  of $F\mathrm{Cospan}$, and each cospan in $\mathcal C$ to the same cospan
  decorated with the empty decoration on its apex. 
  
  To check this is well-defined, we must check that the composite of two
  empty-decorated cospans is again empty-decorated. This is an immediate
  consquence of Lemma \ref{lem.emptydecorations}: the composite of two
  empty-decorated cospans is again empty-decorated. This implies the
  functoriality of the wide embedding $\mathrm{Cospan(\mathcal{C})}
  \hookrightarrow F\mathrm{Cospan}$.

\paragraph{Monoidal structure.} 

  We define the monoidal product of objects $X$ and $Y$ of $F\mathrm{Cospan}$ to
  be their coproduct $X+Y$ in $\mathcal C$, and define the monoidal product of
  decorated cospans $(X \stackrel{i_X}\longrightarrow N
  \stackrel{o_Y}\longleftarrow Y,\enspace 1 \stackrel{s}\longrightarrow FN)$ and
  $(X' \stackrel{i_{X'}}\longrightarrow N' \stackrel{o_{Y'}}\longleftarrow
  Y',\enspace 1 \stackrel{t}\longrightarrow FN')$ to be 
  \[
    \left(
    \begin{aligned}
      \xymatrix{
	& N+N' \\  
	X+X' \ar[ur]^{i_X+i_{X'}} && Y+Y' \ar[ul]_{o_Y+o_{Y'}}
      }
    \end{aligned}
    ,
    \qquad
    \begin{aligned}
      \xymatrixrowsep{.8pc}
      \xymatrix{
	F(N+N') \\
	FN \otimes FN' \ar[u]_{\varphi_{N,N'}}\\
	1 \otimes 1 \ar[u]_{s \otimes t} \\
	1 \ar[u]_{\lambda^{-1}}
      }
    \end{aligned}
    \right).
  \]
  Using the braiding in $\mathcal D$, we can show that this proposed monoidal
  product is functorial. 
 
  Indeed, suppose we have decorated cospans
\[
  (X \stackrel{i_X}\longrightarrow N \stackrel{o_Y}\longleftarrow Y,\enspace 1
  \stackrel{s}\longrightarrow FN),
  \qquad
  (Y \stackrel{i_Y}\longrightarrow M \stackrel{o_Z}\longleftarrow Z,\enspace 1
  \stackrel{t}\longrightarrow FM), 
\]
\[
  (U \stackrel{i_U}\longrightarrow P \stackrel{o_V}\longleftarrow V,\enspace 1
  \stackrel{u}\longrightarrow FP),
  \qquad
  (V \stackrel{i_V}\longrightarrow Q \stackrel{o_W}\longleftarrow W,\enspace 1
  \stackrel{v}\longrightarrow FQ). 
\]
We must check the so-called interchange law: that the composite of the
column-wise monoidal products is equal to the monoidal product of the row-wise
composites.

Again, for the cospans we take this equality as familiar fact. Write 
\[
  b\colon  (N+P)+_{(Y+V)}(M+Q) \stackrel{\sim}{\longrightarrow} (N+_YM)+(P+_{V}Q).
\]
for the isomorphism between the two resulting representatives of the isomorphism
class of cospans. The two resulting decorations are then given by the leftmost 
and rightmost columns respectively of the diagram below.
\[
  \xymatrixcolsep{1.4pc}
  \xymatrix{ 
    \scriptstyle F((N+P)+_{(Y+V)}(M+Q)) \ar[rrrr]^{Fb}_{\sim} &&&&
    \scriptstyle F((N+_YM)+(P+_VQ)) \\
    & \textsc{\tiny(C)} \\
    \scriptstyle F((N+P)+(M+Q)) \ar[uu]^{F[j_{N+P},j_{M+Q}]} \ar@{.>}[uurrrr] &&&
    \textsc{\tiny(F2)} & \scriptstyle F(N+_YM)\otimes F(P+_VQ)
    \ar[uu]_{\varphi_{N+_YM,P+_VQ}} \\
    \\
    \scriptstyle F(N+P)\otimes F(M+Q) \ar[uu]^{\varphi_{N+P,M+Q}} &&&&
    \scriptstyle F(N+M)\otimes F(P+Q) \ar@{.>}[uullll]
    \ar[uu]_{F[j_N,j_M] \otimes F[j_P,j_Q]} \\
    && \textsc{\tiny(F1)} \\
    \scriptstyle (FN \otimes FP) \otimes (FM \otimes FQ) \ar[uu]^{\varphi_{N,P} \otimes
    \varphi_{M,Q}} \ar@{.>}[rrrr] &&&& \scriptstyle (FN \otimes FM) \otimes (FP \otimes FQ)
    \ar[uu]_{\varphi_{N,M} \otimes \varphi_{P,Q}} \\
    && \textsc{\tiny(D)} \\
    && \scriptstyle (1\otimes1)\otimes(1\otimes1) \ar[uull]^{(s\otimes u) \otimes (t \otimes
    v)} \ar[uurr]_{(s \otimes t) \otimes (u \otimes v)} \\
    && \scriptstyle 1 \ar[u]_{(\lambda^{-1}\otimes\lambda^{-1})\circ\lambda^{-1}}
  }
\]
These two decorations are related by the isomorphism $b$ as the diagram
commutes. We argue this more briefly than before, as the basic structure of
these arguments is now familiar to us. Briefly then, there exist dotted arrows
of the above types such that the subdiagram (D) commutes by the naturality of
the associators and braiding in $\mathcal D$, (F1) commutes by the coherence
diagrams for the braided monoidal functor $F$, (F2) commutes by the naturality of
the coherence map $\varphi$ for $F$, and (C) commutes by the properties of
colimits in $\mathcal C$ and the functoriality of $F$. 

Using now routine methods, it also is straightforward to show that the monoidal
product of identity decorated cospans on objects $X$ and $Y$ is the identity
decorated cospan on $X+Y$; for the decorations this amounts to the observation
that the monoidal product of empty decorations is again an empty decoration.

  Choosing associator, unitors, and braiding in $F\mathrm{Cospan}$ to be the
  images of those in $\mathrm{Cospan(\mathcal{C})}$, we have a symmetric
  monoidal category. These transformations remain natural transformations when
  viewed in the category of $F$-decorated cospans as they have empty
  decorations.
  
We consider the case of the left unitor in detail; the naturality of the right unitor,
associator, and braiding follows similarly, using the relevant axiom where here
we use the left unitality axiom.

Given a decorated cospan $(X \stackrel{i}\longrightarrow N
\stackrel{o}\longleftarrow Y,\enspace 1 \stackrel{s}\longrightarrow FN)$, we
must show that the diagram of decorated cospans
\[
  \xymatrix{
    X+\varnothing \ar[r]^{i+1} \ar[d]_{\lambda_{\mathcal C}} & N+\varnothing & Y+\varnothing
    \ar[l]_{i+1} \ar[d]^{\lambda_{\mathcal C}} \\
    X \ar[r]_{i} & N & Y \ar[l]^{o} 
  }
\]
commutes, where the $\lambda_{\mathcal C}$ are the maps of the left unitor in $\mathcal C$
considered as empty-decorated cospans, and where the top cospan has decoration 
\[
  1 \stackrel{\lambda^{-1}_{\mathcal D}}{\longrightarrow} 1 \otimes 1 \stackrel{s \otimes
  \varphi_1}\longrightarrow FN \otimes F\varnothing
  \stackrel{\varphi_{N,\varnothing}}\longrightarrow F(N+\varnothing),
\]
and the lower cospan simply has decoration $1 \stackrel{s}{\longrightarrow} FN$. 

Now as the $\lambda$ are isomorphisms in $\mathcal C$ and have empty
decorations, the composite through the upper right corner is isomorphic to the
decorated cospan
\[
  \big(X+\varnothing \stackrel{i+1}\longrightarrow N+\varnothing
  \stackrel{(o+1)\circ \lambda^{-1}_{\mathcal C}}\longleftarrow Y,\enspace 1
  \stackrel{\varphi_{N,\varnothing} \circ (s \otimes \varphi_1) \circ
  \lambda^{-1}_{\mathcal D}}{\xrightarrow{\hspace*{2.5cm}}} F(N+\varnothing)\big),
\]
while the composite through the lower left corner is isomorphic to the decorated
cospan
\[
  (X+\varnothing \stackrel{[i,!]}\longrightarrow N
\stackrel{o}\longleftarrow Y,\enspace 1 \stackrel{s}\longrightarrow FN).
\]
Furthermore, $\lambda_{\mathcal C}\colon  N+\varnothing \rightarrow N$ gives an isomorphism
between these two cospans, and the naturality of the left unitor and the left
unitality axiom in $\mathcal D$ imply that this is in fact an isomorphism of
decorated cospans:
\[
  \xymatrix{
    1 \ar[rrr]^{s} \ar[d]_{\lambda^{-1}_{\mathcal D}} &&& FN \\
    1 \otimes 1 \ar[r]_{s \otimes 1} & FN \otimes 1 \ar[r]_{1 \otimes \varphi_1}
    \ar[urr]^{\lambda_{\mathcal D}} & FN \otimes F\varnothing
    \ar[r]_{\varphi_{N,\varnothing}} & F(N+\varnothing).
    \ar[u]_{F\lambda_{\mathcal C}}
  }
\]
This the coherence maps are indeed natural transformations. Moreover, they obey
the required coherence laws as they are images of maps that obey these laws in
$\mathrm{Cospan(\mathcal{C})}$. 

\paragraph{Hypergraph structure.}
Similarly, to arrive at the hypergraph structure on $F\mathrm{Cospan}$, we
  simply equip each object $X$ with the image of the special commutative
  Frobenius monoid specified by the hypergraph structure of
  $\mathrm{Cospan(\mathcal{C})}$. It is evident that this choice of structures
  implies the above wide embedding is a hypergraph functor.
\end{proof}

In summary, the hypergraph category $F\mathrm{Cospan}$ comprises:

\begin{center}
  \begin{tabular}{| c | p{.65\textwidth} |}
    \hline
    \multicolumn{2}{|c|}{The hypergraph category $(F\mathrm{Cospan},+)$} \\
    \hline
    \textbf{objects} & the objects of $\mathcal C$ \\ 
    \textbf{morphisms} & isomorphism classes of $F$-decorated cospans in
    $\mathcal C$\\ 
    \textbf{composition} & given by pushout \\
    \textbf{monoidal product} & the coproduct in $\mathcal C$ \\
    \textbf{coherence maps} & empty decorated maps from $\cospan(\mc C)$ \\
    \textbf{hypergraph maps} & empty decorated maps from $\cospan(\mc C)$ \\
    \hline
  \end{tabular}
\end{center}

Note that if the monoidal unit in $(\mathcal D,\otimes)$ is the initial object,
then each object only has one possible decoration: the empty decoration. This
immediately implies the following corollary.
\begin{corollary}
  Let $1_{\mathcal C}\colon (\mathcal C,+) \to (\mathcal C,+)$ be the identity functor
  on a category $\mathcal C$ with finite colimits. Then
  $\mathrm{Cospan}(\mathcal C)$ and $1_{\mathcal C}\mathrm{Cospan}$ are
  isomorphic as hypergraph categories.
\end{corollary}

Thus we see that there is always a hypergraph functor between decorated cospan
categories $1_{\mathcal C}\mathrm{Cospan} \rightarrow F\mathrm{Cospan}$. This
provides an example of a more general way to construct hypergraph functors
between decorated cospan categories. We detail this in the next section.

\subsection{A bicategory of decorated cospans}
Just as cospans form the morphisms of a symmetric monoidal bicategory, so too do
decorated cospans. In this bicategory the 2-morphisms are given by morphisms of
decorated cospans: morphisms $n$
in $\mc C$ such that the diagrams
\[
  \begin{aligned}
    \xymatrix{
      & N \ar[dd]^n \\  
      X \ar[ur]^{i} \ar[dr]_{i'} && Y \ar[ul]_{o} \ar[dl]^{o'}\\
      & N'
    }
  \end{aligned}
  \qquad \mbox{and}
  \qquad
  \begin{aligned}
    \xymatrix{
      & FN \ar[dd]^{Fn} \\
      1 \ar[ur]^{s} \ar[dr]_{s'} \\
      & FN'
    }
  \end{aligned}
\]
commute. For more details see Courser \cite{Cou16}. These 2-morphisms could be
employed to model transformations, coarse grainings, rewrite rules, and so on of
open systems and their representations.

\section{Functors between decorated cospan categories} \label{sec:dcf}

Decorated cospans provide a setting for formulating various operations that we
might wish to enact on the decorations, including the composition of these
decorations, both sequential and monoidal, as well as dagger, dualising, and
other operations afforded by the hypergraph structure. We now observe that these
operations are formulated in a systematic way, so that transformations of the
decorating structure---that is, monoidal transformations between the lax
monoidal functors defining decorated cospan categories---respect these
operations. 

\begin{theorem} \label{thm:decoratedfunctors}
  Let $\mathcal C$, $\mathcal C'$ be categories with finite colimits, abusing
  notation to write the coproduct in each category $+$, and $(\mathcal D,
  \otimes)$, $(\mathcal D',\boxtimes)$ be braided monoidal categories. Further let
  \[
    (F,\varphi)\colon  (\mathcal C,+) \longrightarrow (\mathcal D,\otimes)
    \qquad \mbox{and} \qquad
    (G,\gamma)\colon  (\mathcal C',+) \longrightarrow (\mathcal D',\boxtimes)
  \]
  be lax braided monoidal functors. This gives rise to decorated cospan
  categories $F\mathrm{Cospan}$ and $G\mathrm{Cospan}$. 

  Suppose then that we have a finite colimit-preserving functor $A\colon  \mathcal C
  \to \mathcal C'$ with accompanying natural isomorphism $\alpha\colon  A(-)+A(-)
  \Rightarrow A(-+-)$, a lax monoidal functor $(B,\beta)\colon  (\mathcal D, \otimes)
  \to (\mathcal D', \boxtimes)$, and a monoidal natural transformation $\theta\colon 
  (B \circ F, B\varphi\circ\beta) \Rightarrow (G \circ A, G\alpha\circ\gamma)$.
  This may be depicted by the diagram:
  \[
    \xymatrixcolsep{3pc}
    \xymatrixrowsep{3pc}
    \xymatrix{
      (\mathcal C,+) \ar^{(F,\varphi)}[r] \ar_{(A,\alpha)}[d] \drtwocell
      \omit{_\:\theta} & (\mathcal D,\otimes) \ar^{(B,\beta)}[d]  \\
      (\mathcal C',+) \ar_{(G,\gamma)}[r] & (\mathcal D',\boxtimes).
    }
  \]

  Then we may construct a hypergraph functor 
  \[
    (T, \tau)\colon  F\mathrm{Cospan} \longrightarrow G\mathrm{Cospan}
  \]
  mapping objects $X \in F\mathrm{Cospan}$ to $AX \in G\mathrm{Cospan}$, and
  morphisms 
  \[
    \left(
    \begin{aligned}
      \xymatrix{
	& N \\  
	X \ar[ur]^{i} && Y \ar[ul]_{o}
      }
    \end{aligned}
    ,
    \quad
    \begin{aligned}
      \xymatrix{
	FN \\
	1_{\mathcal D} \ar[u]_{s}
      }
    \end{aligned}
    \right)
    \qquad
  to
  \qquad
    \left(
    \begin{aligned}
      \xymatrix{
	& AN \\  
	AX \ar[ur]^{Ai} && AY \ar[ul]_{Ao}
      }
    \end{aligned}
    ,
    \quad
    \begin{aligned}
      \xymatrixrowsep{.8pc}
      \xymatrix{
	GAN \\
	BFN \ar[u]_{\theta_N}\\
	B1_{\mathcal D} \ar[u]_{Bs} \\
	1_{\mathcal D'} \ar[u]_{\beta_1}
      }
    \end{aligned}
    \right).
  \]
  Moreover, $(T,\tau)$ is a strict monoidal functor if and only if $(A,\alpha)$
  is.
\end{theorem}

\begin{proof}
  We must prove that $(T,\tau)$ is a functor, is strong symmetric monoidal, and
  that it preserves the special commutative Frobenius structure on each object.

  \paragraph{Functoriality.}
  Checking the functoriality of $T$ is again an exercise in applying the
  properties of structure available---in this case the colimit-preserving nature
  of $A$ and the monoidality of $(\mathcal D,\boxtimes)$, $(B,\beta)$, and
  $\theta$---to show that the relevant diagrams of decorations commute. 
  
  In detail, let 
\[
  (X \stackrel{i_X}\longrightarrow N \stackrel{o_Y}\longleftarrow Y, \enspace 1
  \stackrel{s}\longrightarrow FN)
  \qquad \mbox{and} \qquad
  (Y \stackrel{i_Y}\longrightarrow M \stackrel{o_Z}\longleftarrow Z, \enspace 1
  \stackrel{t}\longrightarrow FM), 
\]
be morphisms in $F\mathrm{Cospan}$. As the composition of the cospan part is by
pushout in $\mathcal C$ in both cases, and as $T$ acts as the colimit preserving
functor $A$ on these cospans, it is clear that $T$ preserves composition of
isomorphism classes of cospans. Write
\[
  c\colon  AX+_{AY}AZ \stackrel\sim\longrightarrow A(X+_YZ)
\]
for the isomorphism from the cospan obtained by composing the $A$-images of the
above two decorated cospans to the cospan obtained by taking the $A$-image of their
composite. To see that this extends to an isomorphism of decorated cospans,
observe that the decorations of these two cospans are given by the rightmost and
leftmost columns respectively in the following diagram:
\[
  \xymatrixcolsep{.5pc}
  \xymatrixrowsep{.9pc}
  \xymatrix{ 
    GA(N+_YM) &&&&&&&& \ar[llllllll]_{Gc}^{\sim} G(AN+_{AY}AM) \\
    &&&&& \textsc{\tiny(A)} \\
    BF(N+_YM) \ar[uu]^{\theta_{N+YM}} && \textsc{\tiny(T2)} && GA(N+M)
    \ar[uullll]_{GA[j_N,j_M]} &&&& G(AN+AM) \ar[uu]_{G[j_{AN},j_{AM}]}
    \ar[llll]^{\sim}_{G\alpha} \\
    \\
    BF(N+M) \ar[uu]^{BF[j_N,j_M]} \ar[uurrrr]_{\theta_{N,M}} &&&&
    \textsc{\tiny(T1)} &&&& GAN \boxtimes GAM \ar[uu]_{\gamma_{AN,AM}} \\
    \\
    B(FN \otimes FM) \ar[uu]^{B\varphi_{N,M}} &&&&&&&& BFN \boxtimes BFM
    \ar[uu]_{\theta_N \boxtimes \theta_M} \ar[llllllll]_{\beta_{FN,FM}} \\
    &&&& \textsc{\tiny(B2)} \\
    B(1_{\mathcal D} \otimes 1_{\mathcal D}) \ar[uu]^{B(s \otimes t)} &&&&&&&&
    B1_{\mathcal D} \boxtimes B1_{\mathcal D} \ar[uu]_{Bs \boxtimes Bt}
    \ar[llllllll]_{\beta_{1,1}} \\
    &&& \textsc{\tiny(B1)} &&&& \textsc{\tiny(D2)} \\
    B1_{\mathcal D} \ar[uu]^{\beta\lambda^{-1}_1} \ar[rrrr]^{\lambda^{-1}_{B1}}
    &&&& 1_{\mathcal D'} \boxtimes B1_{\mathcal D} \ar[uurrrr]^{\beta_1 \boxtimes 1}
    &&&& 1_{\mathcal D'} \boxtimes 1_{\mathcal D'} \ar[uu]^{\beta_1 \boxtimes \beta_1}
    \ar[llll]_{1 \boxtimes \beta_1} \\
    &&&& \textsc{\tiny(D1)} \\
    &&&& 1_{\mathcal D'} \ar[uullll]^{\beta_1} \ar[uurrrr]_{\lambda^{-1}_1}
  }
\]
From bottom to top, \textsc{(D1)} commutes by the naturality of $\lambda$,
\textsc{(D2)} by the functoriality of the monoidal product $-\boxtimes-$,
\textsc{(B1)} by the unit law for $(B,\beta)$, \textsc{(B2)} by the
naturality of $\beta$, \textsc{(T1)} by the monoidality of the natural
transformation $\theta$, \textsc{(T2)} by the naturality of $\theta$, and
\textsc{(A)} by the colimit preserving property of $A$ and the functoriality of
$G$.


We must also show that identity morphisms are mapped to identity morphisms. Let 
\[
  (X \stackrel{1_X}\longrightarrow X \stackrel{1_X}\longleftarrow X, \enspace 1
  \stackrel{F!\circ \varphi_1}\longrightarrow FX)
\]
be the identity morphism on some object $X$ in the category of $F$-decorated
cospans. Now this morphism has $T$-image
\[
  (AX \stackrel{1_{AX}}\longrightarrow AX \stackrel{1_{AX}}\longleftarrow AX, \enspace 1
  \stackrel{\theta_X \circ B(F!\circ \varphi_1) \circ
  \beta_1}{\xrightarrow{\hspace*{2.3cm}}} GAX).
\]
But we have the following diagram
\[
  \xymatrixcolsep{2pc}
  \xymatrixrowsep{.3pc}
  \xymatrix{ 
    &&&& BF\varnothing_{\mathcal C} \ar[ddrr]^{BF!}
    \ar[dddd]^{\theta_\varnothing} \\
    \\
    && B1_{\mathcal D} \ar[uurr]^{B\varphi_1} &&&& BFX \ar[ddrr]^{\theta_X} \\
    &&& \textsc{\tiny (T1)} && \textsc{\tiny (T2)} \\
    1_{\mathcal D'} \ar[uurr]^{\beta_1} \ar[rr]_{\gamma_1}
    \ar[ddddrrrr]_{\gamma_1} && G\varnothing_{\mathcal C'} \ar[rr]_{G\alpha_1}
    && GA\varnothing_{\mathcal C} \ar[rrrr]_{GA!} &&&& GAX \\
    &&& \textsc{\tiny (A1)} && \textsc{\tiny (A2)} \\
    \\
    \\
    &&&& G\varnothing_{\mathcal C'} \ar[uuuu]^{\sim}_{G!} \ar[uuuurrrr]_{G!}
  }
\]
Here \textsc{(A1)} and \textsc{(A2)} commute by the fact $A$ preserves colimits,
\textsc{(T1)} commutes by the unit law for the monoidal natural transformation
$\theta$, and \textsc{(T2)} commutes by the naturality of $\theta$.

Thus
we have the equality of decorations $\theta_X \circ B(F! \circ \varphi_1) \circ
\beta_1
= G! \circ \gamma_1\colon  1 \to GAX$, and so $T$ sends identity morphisms to
identity morphisms.

\paragraph{Monoidality.} The coherence maps of the functor are given by the coherence maps for the
  monoidal functor $A$, viewed now as cospans with the empty decoration. That
  is, we define the coherence maps $\tau$ to be the collection of isomorphisms
  \[
    \tau_1 = 
    \left(
    \begin{aligned}
      \xymatrix{
	& A\varnothing_{\mathcal C} \\  
	\varnothing_{\mathcal C'} \ar[ur]^{\alpha_1} && A\varnothing_{\mathcal
	C} \ar@{=}[ul]
      }
    \end{aligned}
    ,
    \qquad
    \begin{aligned}
      \xymatrixrowsep{.9pc}
      \xymatrix{
	GA\varnothing_{\mathcal C} \\
	G\varnothing_{\mathcal C'} \ar[u]_{G!} \\
	1_{\mathcal D'} \ar[u]_{\gamma_1}
      }
    \end{aligned}
    \right),
  \]
  \[
    \tau_{X,Y}=
    \left(
    \begin{aligned}
      \xymatrix{
	& A(X+Y) \\  
	AX+AY \ar[ur]^{\alpha_{X,Y}} && A(X+Y) \ar@{=}[ul]
      }
    \end{aligned}
    ,
    \qquad
    \begin{aligned}
      \xymatrixrowsep{.9pc}
      \xymatrix{
	GA(X+Y) \\
	G\varnothing_{\mathcal C'} \ar[u]_{G!} \\
	1_{\mathcal D} \ar[u]_{\gamma_1}
      }
    \end{aligned}
    \right),
  \]
  where $X$, $Y$ are objects of $G\mathrm{Cospan}$. As $(A,\alpha)$ is already
  strong symmetric monoidal and $\tau$ merely views these maps in $\mathcal C$
  as empty-decorated cospans in $G\mathrm{Cospan}$, $\tau$ is natural in $X$ and
  $Y$, and obeys the required coherence axioms for $(T,\tau)$ to also be strong
  symmetric monoidal. 

  Indeed, the monoidality of a functor $(T,\tau)$ has two aspects: the naturality
of the transformation $\tau$, and the coherence axioms. We discuss the former;
since $\tau$ is just an empty-decorated version of $\alpha$, the latter then
immediately follow from the coherence of $\alpha$.

The naturality of $\tau$ may be proved via the same method as that employed for
the naturality of the coherence maps of decorated cospan categories: we first
use the composition of empty decorations to compute the two paths around the
naturality square, and then use the naturality of the coherence map $\alpha$ to
show that these two decorated cospans are isomorphic.

In slightly more detail, suppose we have decorated cospans
\[
  (X \stackrel{i_X}\longrightarrow N
  \stackrel{o_Z}\longleftarrow Z,\enspace 1 \stackrel{s}\longrightarrow FN) \quad
  \textrm{and} \quad (Y \stackrel{i_Y}\longrightarrow M
  \stackrel{o_W}\longleftarrow W,\enspace 1 \stackrel{t}\longrightarrow FM).
\]
Then naturality demands that the cospans
\[
  AX+AY \stackrel{Ai_X+Ai_Y}{\xrightarrow{\hspace*{1.5cm}}} AN+AM
  \stackrel{(o_Z+o_W)\circ\alpha^{-1}}{\xleftarrow{\hspace*{1.5cm}}} A(Z+W)
\]
and
\[
  AX+AY \stackrel{A(i_X+i_Y)\circ\alpha}{\xrightarrow{\hspace*{1.5cm}}} A(N+M)
  \stackrel{A(o_Z+o_W)}{\xleftarrow{\hspace*{1.5cm}}} A(Z+W)
\]
are isomorphic as decorated cospans, with decorations the top and bottom rows of
the diagram below respectively.
\[
  \xymatrixrowsep{0.5pc}
  \xymatrix{
    & 1 \otimes 1 \ar[r]^(.4){\beta_1 \otimes \beta_1} & B1 \otimes B1 \ar[r]^(.4){Bs
    \otimes Bt} & BFN \otimes BFM \ar[r]^{\theta_N \otimes \theta_M} & GAN \otimes
    GAM \ar[r]^{\gamma_{AN,AM}} & G(AN+AM) \ar[dd]^{G\alpha_{N,M}} \\
    1 \ar[ur]^{\lambda^{-1}} \ar[dr]_{\beta_1} \\
    & B1 \ar[r]_(.4){B\lambda^{-1}} & B(1 \otimes 1) \ar[r]_(.4){B(s\otimes t)} & B(FN
    \otimes FM) \ar[r]_{B\varphi_{N,M}} & B(F(N+M))
    \ar[r]_{\theta_{N,M}} & G(A(N+M)).
  }
\]
As it is a subdiagram of the large functoriality commutative diagram, this
diagram commutes. The diagrams required for $\alpha_{N,M}$ to be a morphism of
cospans also commute, so our decorated cospans are indeed isomorphic. This
proves $\tau$ is a natural tranfomation.  

Moreover, as $A$ is coproduct-preserving and the Frobenius structures on
$F\mathrm{Cospan}$ and $G\mathrm{Cospan}$ are built using various copairings
of the identity map, $(T,\tau)$ preserves the hypergraph structure.

Finally, it is straightforward to observe that the maps $\tau$ are identity
maps if and only if the maps $\alpha$ are, so $(T,\tau)$ is a strict monoidal
functor if and only $(A,\alpha)$ is.
\end{proof}

When the decorating structure comprises some notion of topological diagram, such
as a graph, these natural transformations $\theta$ might describe some semantic
interpretation of the decorating structure. In this setting the above theorem
constructs functorial semantics for the decorated cospan category of diagrams.

\section{Decorations in $\Set$ are general}

So far we have let decorations lie in any braided monoidal category. In
conversations with the author, Sam Staton pointed out that it is general enough
to let decorations lie in the symmetric monoidal category $(\Set,\times)$ of
sets, functions, and the categorical product.

The key observation is that decorated cospans only make use of the sets of
monoidal elements $x\maps I \to X$ in the decorating category $\mc D$. To
extract this information, we define the monoidal global sections functor.

\begin{proposition}
Let $(\mathcal D, \otimes)$ be a braided monoidal category. Then there exists a
lax braided monoidal functor, the \define{monoidal global sections functor}
$\mathcal D(I,-)\maps (\mc D,\ot) \to (\Set,\times)$ taking each object $X$ in
$\mc D$ to the homset $\mc D(I,X)$, and each morphism $f\maps X \to Y$ to the
function 
\begin{align*}
  \mc D(I,f)\maps \mc D(I,X) &\longrightarrow \mc D(I,Y); \\ 
  x &\longmapsto f \circ x.
\end{align*}
The coherence maps are given by $d_1\maps 1 \to \mc D(I,I);
\ast \mapsto \idn_I$ and
\begin{align*}
  d_{X,Y} \maps \mc D(I,X) \times \mc D(I,Y) &\longrightarrow \mc D(I, X \ot
  Y);\\
  (x,y) &\longmapsto (I \xrightarrow{\rho^{-1}} I \ot I \xrightarrow{x \ot y} X \ot
  Y).
\end{align*}
\end{proposition}
\begin{proof}
  This functor is the covariant hom functor on the monoidal unit. The
  coherence axioms immediately follow from the coherence of the braided monoidal
  category $\mc D$.
\end{proof}

Composing a decorating functor $F$ with this global sections functor produces an
isomorphic decorated cospans category.

\begin{proposition} \label{prop.setdecorations}
  Let $F\maps (\mathcal C, +) \to (\mathcal D, \otimes)$ be a braided lax
  monoidal functor. Then $F\mathrm{Cospan}$ and $(\mathcal D(I,-)\circ
  F)\mathrm{Cospan}$ are isomorphic as hypergraph categories.
\end{proposition}
\begin{proof}
  Note that the global sections functor is always lax braided monoidal. We have
  the commutative-by-definition triangle of braided lax monoidal functors
  \[
    \xymatrixrowsep{2ex}
    \xymatrix{
      && (\mathcal D,\otimes) \ar[dd]^{\mathcal D(I,-)} \\
      (\mathcal C,+) \ar[urr]^{F} \ar[drr]_{\mathcal D(I,-)\circ F} \\
      && (\Set, \times)
    }
  \]
  By Theorem \ref{thm:decoratedfunctors}, this gives a strict hypergraph functor
  $F\mathrm{Cospan} \to (\mathcal D(I,-)\circ F)\mathrm{Cospan}$. It is easily
  checked that this functor is bijective-on-objects, full, and faithful.
\end{proof}

Similarly, taking $\Set$ to be our decorating category does not impinge upon the
functors that can be constructed using decorated cospans.
\begin{proposition}
  Let $T\maps F\mathrm{Cospan} \to G\mathrm{Cospan}$ be a decorated cospans
  functor. Then there is a decorated cospans functor $U$ such that the square of
  hypergraph functors 
  \[
    \xymatrixcolsep{3pc}
    \xymatrixrowsep{3pc}
    \xymatrix{
      F\mathrm{Cospan} \ar[r]^T \ar[d]_{\sim} & G\mathrm{Cospan} \ar[d]^{\sim}\\ 
      (\mathcal D(I,-)\circ F)\mathrm{Cospan} \ar[r]^U & (\mathcal D'(I,-)\circ
      G)\mathrm{Cospan}
    }
  \]
  commutes, where the vertical maps are the isomorphisms given by the previous
  proposition.
\end{proposition}
\begin{proof}
  Write 
  \[
    \xymatrixcolsep{3pc}
    \xymatrixrowsep{3pc}
    \xymatrix{
      (\mathcal C,+) \ar^{(F,\varphi)}[r] \ar_{(A,\alpha)}[d] \drtwocell
      \omit{_\:\theta} & (\mathcal D,\otimes) \ar^{(B,\beta)}[d]  \\
      (\mathcal C',+) \ar_{(G,\gamma)}[r] & (\mathcal D',\boxtimes).
    }
  \]
  for the monoidal natural transformation yielding $T$, and write
  \begin{align*}
    \overline\beta_X\maps \mc D(I,X) &\longrightarrow \mc D'(I,BX); \\
    (I_{\mc D} \xrightarrow{x} X) &\longmapsto (I_{\mc D'} \xrightarrow{\beta_I}
    BI_{\mc D} \xrightarrow{Bx} BX)
  \end{align*}
  for all $X \in \mc D$. This defines a monoidal natural transformation
  $\overline\beta\maps \mc D(I,-) \Rightarrow \mc D'(I,B-)$, with the
  monoidality following from the monoidality of the functor $(B,\beta)$.

  Next, define $U$ as the functor resulting from the decorated cospan
  construction applied to the horizontal composition of monoidal natural
  transformations
  \[
    \xymatrixcolsep{3pc}
    \xymatrixrowsep{3pc}
    \xymatrix{
      (\mathcal C,+) \ar^{(F,\varphi)}[r] \ar_{(A,\alpha)}[d] \drtwocell
      \omit{_\:\theta} & (\mathcal D,\otimes) \ar[r]^{\mc D(I,-)}
      \ar^{(B,\beta)}[d] \drtwocell \omit{_\:\overline{\beta}} & (\Set,\times)
      \ar@{=}[d] \\
      (\mathcal C',+) \ar_{(G,\gamma)}[r] & (\mathcal D',\boxtimes) \ar[r]_{\mc
      D'(I,-)} & (\Set,\times).
    }
  \]
  To see that the required square of hypergraph functors commutes, observe that
  both functors $F\mathrm{Cospan} \to (\mc D'(I,-)\circ G)\mathrm{Cospan}$ take
  objects $X$ to $AX$ and morphisms 
  \[
    (X \stackrel{i}\longrightarrow N \stackrel{o}\longleftarrow Y, \enspace 1
    \stackrel{s}\longrightarrow FN)
  \]
  to 
  \[
    \big(AX \stackrel{Ai}\longrightarrow AN
    \stackrel{Ao}\longleftarrow AY, \enspace \theta_N \circ Bs\circ \beta_1\in \mc
    D'(I,GAN)\big). \qedhere
  \]
\end{proof}

Note that the braided monoidal categories $(\mc C,+)$ and $(\Set,\times)$ are
both symmetric monoidal categories. Thus when working with the decorating
category $\Set$ we may refer to the decorating functors $F$ as symmetric lax
monoidal, rather than merely braided lax monoidal.

\section{Examples} \label{sec:ex}
In this final section we outline two constructions of decorated cospan
categories, based on labelled graphs and linear subspaces respectively, and a
functor between these two categories interpreting each graph as an electrical
circuit. We shall see that the decorated cospan framework allows us to take a
notion of closed system and construct a corresponding notion of open or
composable system, together with functorial semantics for these systems.

This electrical circuits example is the motivating application for the
decorated cospan construction, and its shortcomings motivate the further
theoretical development of decorated cospans in the next chapter. Armed with
these additional tools, we will return to the application for a full discussion
in Chapter \ref{ch.circuits}.

\subsection{Labelled graphs}

Recall that a \define{$[0,\infty)$-graph} $(N,E,s,t,r)$ comprises a finite set
$N$ of vertices (or nodes), a finite set $E$ of edges, functions $s,t\colon  E \to N$
describing the source and target of each edge, and a function $r\colon  E \to
[0,\infty)$ labelling each edge. The decorated cospan framework allows us to
construct a category with, roughly speaking, these graphs as morphisms. More
precisely, our morphisms will consist of these graphs, together with subsets of
the nodes marked, with multiplicity, as `input' and `output' connection points.

Pick small categories equivalent to the category of $[0,\infty)$-graphs such
that we may talk about the set of all $[0,\infty)$-graphs on each finite set
$N$.  Then we may consider the functor
\[
  \lgraph\colon  (\FinSet,+) \longrightarrow (\Set,\times)
\]
taking a finite set $N$ to the set $\lgraph(N)$ of $[0,\infty)$-graphs
$(N,E,s,t,r)$ with set of nodes $N$. On
morphisms let it take a function $f\colon N \to M$ to the function that pushes
labelled graph structures on a set $N$ forward onto the set $M$:
\begin{align*}
  \lgraph(f)\colon  \lgraph(N) &\longrightarrow
  \lgraph(M); \\
  (N,E,s,t,r) &\longmapsto (M,E,f \circ s, f \circ t, r).
\end{align*}
As this map simply acts by post-composition, our map $\lgraph$ is indeed
functorial.

We then arrive at a lax braided monoidal functor $(\lgraph,\zeta)$ by equipping
this functor with the natural transformation 
\begin{align*}
  \zeta_{N,M}\colon  \lgraph(N) \times \lgraph(M)
  &\longrightarrow \lgraph(N+M); \\
  \big( (N,E,s,t,r), (M,F,s',t',r') \big) &\longmapsto
  \big(N+M,E+F,s+s',t+t',[r,r']\big),
\end{align*}
together with the unit map
\begin{align*}
  \zeta_1\colon  1=\{\bullet\} &\longrightarrow \lgraph(\varnothing); \\
  \bullet &\longmapsto
  (\varnothing,\varnothing,!,!,!),
\end{align*}
where we remind ourselves
that we write $[r,r']$ for the copairing of the functions $r$ and $r'$. The
naturality of this collection of morphisms, as well as the coherence laws for
lax braided monoidal functors, follow from the universal property of the coproduct.

Theorem \ref{thm:fcospans} thus allows us to construct a hypergraph category
$\mathrm{GraphCospan}$.  For an intuitive visual understanding of the morphisms
of this category and its composition rule, see this paper's introduction.

\subsection{Linear relations} 
Another example of a decorated cospan category arising from a functor $(\FinSet,+)
\to (\Set, \times)$ is closely related to the category of linear relations. Here
we decorate each cospan in $\Set$ with a linear subspace of $\R^N \oplus
(\R^N)^\ast$, the sum of the vector space generated by the apex $N$ over $\R$
and its vector space dual.

First let us recall some facts about relations. Let $R \subseteq X\times Y$ be
a relation; we write this also as $R\colon  X \to Y$. The opposite relation $R^\opp\colon 
Y\to X$, is the subset $R^\opp \subseteq Y\times X$ such that $(y,x) \in R^\opp$
if and only if $(x,y) \in R$. We say that the image of a subset $S \subseteq X$
under a relation $R\colon  X \to Y$ is the subset of all elements of the codomain $Y$
related by $R$ to an element of $S$. Note that if $X$ and $Y$ are vector spaces
and $S$ and $R$ are both linear subspaces, then the image $R(S)$ of $S$ under
$R$ is again a linear subspace.

Now any function $f\colon N \to M$ induces a linear map $f^\ast\colon  \R^M \to
\R^N$ by precomposition. This linear map $f^\ast$ itself induces a dual map
$f_\ast\colon (\R^N)^\ast \to (\R^M)^\ast$ by precomposition. Furthermore
$f^\ast$ has, as a linear relation $f^\ast \subseteq \R^M \oplus \R^N$, an
opposite linear relation $(f^\ast)^\opp\colon  \R^N \to \R^M$.  Define the
functor 
\[
  \linsub\colon  (\FinSet,+) \longrightarrow (\Set,\times)
\]
taking a finite set $N$ to set of linear subspaces of the vector space $\R^N
\oplus (\R^N)^\ast$, and taking a function $f\colon  N \to M$ to the function
$\linsub(N) \to \linsub(M)$ induced by the sum of these two relations:
\begin{align*}
  \linsub(f)\colon  \linsub(N) &\longrightarrow \linsub(M); \\
  L &\longmapsto \big((f^\ast)^\opp \oplus f_\ast\big)(L).
\end{align*}
The above operations on $f$ used in the construction of this map are functorial,
and so it is readily observed that $\linsub$ is indeed a functor.

It is moreover lax braided monoidal as the sum of linear subspace of $\R^N
\oplus (\R^N)^\ast$ and a linear subspace of $\R^M \oplus (\R^M)^\ast$ may be
viewed as a subspace of $\R^N \oplus (\R^N)^\ast \oplus \R^M \oplus (\R^M)^\ast
\cong \R^{N+M} \oplus (\R^{N+M})^\ast \cong \R^{M+N} \oplus (\R^{M+N})^\ast$,
and the empty subspace is a linear subspace of each $\R^N \oplus (\R^N)^\ast$.

We thus have a hypergraph category $\mathrm{LinSubCospan}$.

\subsection{Electrical circuits} 
Electrical circuits and their diagrams are the motivating application for the
decorated cospan construction. Specialising to the case of networks of linear
resistors, we detail here how we may use the category $\mathrm{LinSubCospan}$ to
provide semantics for the morphisms of $\mathrm{GraphCospan}$ as diagrams of
networks of linear resistors.

Intuitively, after choosing a unit of resistance, say ohms ($\Omega$), each
$[0,\infty)$-graph can be viewed as a network of linear resistors, with the
$[0,\infty)$-graph of the introduction now more commonly depicted as
\begin{center}
  \begin{tikzpicture}[circuit ee IEC, set resistor graphic=var resistor IEC graphic]
    \node[contact]         (A) at (0,0) {};
    \node[contact]         (B) at (3,0) {};
    \node[contact]         (C) at (1.5,-2.6) {};
    \coordinate         (ua) at (.5,.25) {};
    \coordinate         (ub) at (2.5,.25) {};
    \coordinate         (la) at (.5,-.25) {};
    \coordinate         (lb) at (2.5,-.25) {};
    \path (A) edge (ua);
    \path (A) edge (la);
    \path (B) edge (ub);
    \path (B) edge (lb);
    \path (ua) edge  [resistor] node[label={[label distance=1pt]90:{$0.2\Omega$}}] {} (ub);
    \path (la) edge  [resistor] node[label={[label distance=1pt]270:{$1.3\Omega$}}] {} (lb);
    \path (A) edge  [resistor] node[label={[label distance=2pt]180:{$0.8\Omega$}}] {} (C);
    \path (C) edge  [resistor] node[label={[label distance=2pt]0:{$2.0\Omega$}}] {} (B);
  \end{tikzpicture}
\end{center}
$\mathrm{GraphCospan}$ may then be viewed as a category with morphisms
circuits of linear resistors equipped with chosen input and output terminals.

The suitability of this language is seen in the way the different categorical
structures of $\mathrm{GraphCospan}$ capture different operations that can be
performed with circuits. To wit, the sequential composition expresses the fact
that we can connect the outputs of one circuit to the inputs of the next, while
the monoidal composition models the placement of circuits side-by-side.
Furthermore, the symmetric monoidal structure allows us reorder input and output
wires, the compactness captures the interchangeability between input and
output terminals of circuits---that is, the fact that we can choose any input
terminal to our circuit and consider it instead as an output terminal, and vice
versa---and the Frobenius structure expresses the fact that we may wire any
node of the circuit to as many additional components as we like.

Moreover, Theorem \ref{thm:decoratedfunctors} provides semantics. Each node in a
network of resistors can be assigned an electric potential and a net current
outflow at that node, and so the set $N$ of vertices of a $[0,\infty)$-graph can
be seen as generating a space $\R^N \oplus (\R^N)^\ast$ of electrical states of
the network. We define a natural transformation 
\[
  \res\colon  \lgraph \Longrightarrow \linsub
\]
mapping each $[0,\infty)$-graph on $N$, viewed as a network of resistors, to the
linear subspace of $\R^N \oplus (\R^N)^\ast$ of electrical states permitted by
Ohm's law.\footnote{Note that these states need not obey Kirchhoff's current
law.} In detail, let $\psi \in \R^N$. We define the power $Q\colon  \R^N \to \R$
corresponding to a $[0,\infty)$-graph $(N,E,s,t,r)$ to be the function
\[
  Q(\psi) = \sum_{e \in E} \frac1{r(e)}
  \Big(\psi\big(t(e)\big)-\psi\big(s(e)\big)\Big)^2.
\]
Then the states of a network of resistors are given by a potential $\phi$ on the
nodes and the gradient of the power at this potential:
\begin{align*}
  \res_N\colon  \lgraph(N) &\longrightarrow \linsub(N)\\
  (N,E,s,t,r) &\longmapsto \{(\phi,\nabla Q_\phi) \mid \phi \in \R^N\}.
\end{align*}
This defines a monoidal natural transformation. Hence, by Theorem
\ref{thm:decoratedfunctors}, we obtain a hypergraph functor
$\mathrm{GraphCospan} \to \mathrm{LinSubCospan}$. 

The semantics provided by this functor match the standard interpretation of
networks of linear resistors. The maps of the Frobenius monoid take on the
interpretation of perfectly conductive wires, forcing the potentials at all
nodes they connect to be equal, and the sum of incoming currents to equal the
sum of outgoing currents---precisely the behaviour implied by Kirchhoff's laws.
More generally, let $(X \stackrel{i}{\rightarrow} N
\stackrel{o}{\leftarrow} Y, \, (N,E,s,t,r))$ be a morphism of
$\mathrm{GraphCospan}$, with $Q$ the power function corresponding to the graph
$\Gamma = (N,E,s,t,r)$. The image of this decorated cospan in
$\mathrm{LinSubCospan}$ is the decorated cospan $(X \stackrel{i}{\rightarrow} N
\stackrel{o}{\leftarrow} Y, \, \{(\phi,\nabla Q_\phi) \mid \phi \in \R^N\})$.
Then it is straightforward to check that the subspace 
\[
  \linsub[i,o]\big(\res_N(\Gamma)\big)\subseteq \R^{X+Y} \oplus (\R^{X+Y})^\ast
\]
is the subspace of electrical states on the terminals $X+Y$ such that currents
and potentials can be chosen across the network of resistors $(N,E,s,t,r)$ that
obey Ohm's and, on its interior, Kirchhoff's laws. In particular, after passing
to a subspace of the terminals in this way, composition in
$\mathrm{LinSubCospan}$ corresponds to enforcing Kirchhoff's laws on the shared
terminals of the two networks. 

This behaviour at the terminals $X+Y$ is often all we are interested in: for
example, in a large electrical network, substituting a subcircuit for a
different subcircuit with the same terminal behaviour will not affect the rest
of the network. Yet this terminal behaviour is not yet encoded directly in the
categorical structure. The images of circuits in $\mathrm{GraphCospan}$ in
$\mathrm{LinSubCospan}$ are cospans decorated by a subspace of $\R^N \oplus
(\R^N)^\ast$, keeping track of \emph{all} the internal behaviour as well. This
can be undesirable, for reasons of information compression as well as reasoning
about equivalence. The next chapter addresses this issue by introducing corelations.


\chapter{Corelations: a tool for black boxing} \label{ch.corelations}

As we have seen, cospans are a useful tool for describing finite colimits, and
so for describing the interconnection of systems. In the previous chapter
we defined decorated cospans to take advantage of this fact and provide
a tool for composing structures that had no inherent composition law, like
graphs and subspaces. In many situations, however, the colimit contains more
information than we care about.  Rather than concerning ourselves with all the
internal structure of a system, we only find a certain aspect of it
relevant---roughly, the part that affects what happens at the boundary. This is
better modelled by corelations. In this chapter we develop the theory of
corelations.

As usual, we begin by giving an intuitive overview of the aims of this chapter
in \textsection\ref{sec.blackboxing}. We then give the formal details of
corelation categories (\textsection\ref{sec.corels}) and their functors
(\textsection\ref{sec.corelfunctors}), before concluding in
\textsection\ref{sec.corelexs} with two important
examples: equivalence relations as epi-mono corelations in $(\Set,+)$, and
linear relations as epi-mono corelations in $(\Vect,\oplus)$. This sets us up
for a decorated version of this theory in Chapter \ref{ch.deccorels}.

\section{The idea of black boxing} \label{sec.blackboxing}

Thus far we have argued that network languages should be modelled using
hypergraph categories, and shown that cospans provide a good language for
talking about interconnection. We then developed the theory of decorated
cospans, which allows us to take diagrams, mark `inputs' and `outputs' using
cospans, and then compose these diagrams using pushouts. This turns a notion of
closed system into a notion of open one.

A serious limitation of using cospans alone, however, is that cospans
indiscrimately accumulate information. For example, suppose we consider the
morphisms of $\mathrm{GraphCospan}$ as open electrical circuits as in
\textsection\ref{ssec.exelectricalcircuits}. Pursuing this, let us depict a
graph decorated cospan from $(X \xrightarrow{i} N \xleftarrow{o} Y, \enspace
(N,E,s,t,r))$ by 
\[
\resizebox{.35\textwidth}{!}{
    \tikzset{every path/.style={line width=1.1pt}}
  \begin{tikzpicture}[circuit ee IEC, set resistor graphic=var resistor IEC graphic]
	\begin{pgfonlayer}{nodelayer}
		\node [style=dot] (0) at (-2.5, 0.75) {};
		\coordinate (1) at (-2, -2) {};
		\coordinate (2) at (1.5, 2) {};
		\coordinate (A) at (-2, 0.75) {};
		\node [style=dot] (4) at (2, 1.25) {};
		\coordinate (B) at (1.5, 0.75) {};
		\node [style=dot] (6) at (2, 0.25) {};
		\coordinate (ub) at (0.75, 1) {};
		\coordinate (la) at (-1.25, 0.5) {};
		\coordinate (C) at (-0.25, -1.375) {};
		\coordinate (lb) at (0.75, 0.5) {};
		\coordinate (ua) at (-1.25, 1) {};
	\end{pgfonlayer}
	\begin{pgfonlayer}{edgelayer}
		\draw (0.center) to (A);
		\draw (6) to (B);
		\draw (4) to (B);
    \path (A) edge (ua);
    \path (A) edge (la);
    \path (B) edge (ub);
    \path (B) edge (lb);
    \path (ua) edge  [resistor] (ub);
    \path (la) edge  [resistor] (lb);
    \path (A) edge  [resistor]  (C);
    \path (C) edge  [resistor]  (B);
	\end{pgfonlayer}
	\begin{pgfonlayer}{background}
	  \filldraw [fill=black!5!white, draw=black!40!white] (1) rectangle (2);
	\end{pgfonlayer}
\end{tikzpicture}
}
\]
where the bullets $\bullet$ on the left represent the elements of the set $X$,
those on the right represent the elements of $Y$, and we draw a line from $x \in
X$ or $y \in Y$ to a node $n$ in the graph when $i(x) = n$ or $o(y)=n$.  We omit
the labels (resistances) on the edges as these are not essential to the main
point here.  

We also depict an example of composition of these open circuits using decorated
cospans:
\[
\begin{aligned}
\resizebox{.45\textwidth}{!}{
    \tikzset{every path/.style={line width=1.1pt}}
  \begin{tikzpicture}[circuit ee IEC, set resistor graphic=var resistor IEC graphic]
	\begin{pgfonlayer}{nodelayer}
		\node [style=dotbig] (0) at (-2.25, 4) {};
		\coordinate (1) at (-1.75, 1.25) {};
		\coordinate (2) at (1.75, 5.25) {};
		\coordinate (3) at (-1.75, 4) {};
		\node [style=dotbig] (4) at (2.25, 4.5) {};
		\coordinate (5) at (1.75, 4) {};
		\node [style=dotbig] (6) at (2.25, 3.5) {};
		\coordinate (7) at (1, 4.25) {};
		\coordinate (8) at (-1, 3.75) {};
		\coordinate (9) at (0, 1.75) {};
		\coordinate (10) at (1, 3.75) {};
		\coordinate (11) at (-1, 4.25) {};
		\node [style=dotbig] (12) at (7.25, 4) {};
		\coordinate (13) at (-3, -1.5) {};
		\coordinate (14) at (6, 2) {};
		\coordinate (15) at (3.5, 2) {};
		\coordinate (16) at (4.75, -0) {};
		\coordinate (17) at (3, 3.5) {};
		\coordinate (18) at (3, 4.5) {};
		\node [style=dotbig] (19) at (2.5, -1.5) {};
		\coordinate (20) at (6.75, 4) {};
		\node [style=dotbig] (21) at (7.25, -1.5) {};
		\node [style=dotbig] (22) at (-6.5, -1) {};
		\node [style=dotbig] (23) at (7.25, 1.5) {};
		\node [style=dotbig] (24) at (-6.5, 4.5) {};
		\node [style=dotbig] (25) at (-6.5, 3) {};
		\coordinate (26) at (-6, -2) {};
		\coordinate (27) at (-3, 5.25) {};
		\coordinate (28) at (1.75, 1) {};
		\coordinate (29) at (-1.75, -2) {};
		\coordinate (30) at (3, -2) {};
		\coordinate (31) at (-6, 4.5) {};
		\coordinate (32) at (-6, 3) {};
		\coordinate (33) at (-3, 4) {};
		\coordinate (34) at (6.75, 2.5) {};
		\coordinate (35) at (3, 2.75) {};
		\coordinate (36) at (6.75, 5.25) {};
		\node [style=dotbig] (37) at (7.25, 0.5) {};
		\node [style=dotbig] (38) at (-2.25, -1.5) {};
		\node [style=dotbig] (39) at (-2.5, 4) {};
		\node [style=dotbig] (40) at (2.5, 4.5) {};
		\node [style=dotbig] (41) at (2.5, 3.5) {};
		\node [style=dotbig] (42) at (-2.25, 0.5) {};
		\node [style=dotbig] (43) at (-2.5, -1.5) {};
		\node [style=dotbig] (44) at (-2.5, 0.5) {};
		\node [style=dotbig] (45) at (2.25, -1.5) {};
		\node [style=dotbig] (46) at (2.25, -0) {};
		\node [style=dotbig] (47) at (2.5, -0) {};
		\node [style=dotbig] (48) at (-2.5, -0.5) {};
		\node [style=dotbig] (49) at (-2.25, -0.5) {};
		\coordinate (50) at (-6, -1) {};
		\coordinate (51) at (-5, -0) {};
		\coordinate (52) at (-3, 0.5) {};
		\coordinate (53) at (-3, -0.5) {};
		\coordinate (54) at (-1.75, 0.5) {};
		\coordinate (55) at (-1.75, -0.5) {};
		\coordinate (56) at (0, -0) {};
		\coordinate (57) at (1.75, -0) {};
		\coordinate (58) at (3, -0) {};
		\coordinate (59) at (-5.75, 1.25) {};
		\coordinate (60) at (4.75, 1) {};
		\coordinate (61) at (6.75, 1.5) {};
		\coordinate (62) at (6.75, 0.5) {};
		\coordinate (63) at (-1.75, -1.5) {};
		\coordinate (64) at (1.75, -1.5) {};
		\coordinate (65) at (3, -1.5) {};
		\coordinate (66) at (6.75, -1.5) {};
		\coordinate (67) at (-5.5, 1.75) {};
		\coordinate (68) at (-5.25, 1) {};
		\coordinate (69) at (-3.75, 2.5) {};
		\coordinate (70) at (-3.5, 1.75) {};
		\coordinate (71) at (-3.25, 2.25) {};
	\end{pgfonlayer}
	\begin{pgfonlayer}{edgelayer}
		\path (50) edge [resistor] (13);
		\draw (24) to (31);
		\draw (61) to (23);
		\draw (62) to (37);
		\draw (25) to (32);
		\draw (22) to (50);
		\draw (33) to (39);
		\draw (0) to (3);
		\draw (52) to (44);
		\draw (42) to (54);
		\draw (53) to (48);
		\draw (49) to (55);
		\path (31) edge [resistor] (33);
		\path (32) edge [resistor] (33);
		\draw (3) to (11);
		\draw (3) to (8);
		\path (11) edge [resistor] (7);
		\path (8) edge [resistor] (10);
		\draw (7) to (5);
		\draw (10) to (5);
		\draw (5) to (4);
		\draw (5) to (6);
		\draw (41) to (17);
		\draw (40) to (18);
		\path (18) edge [resistor] (20);
		\path (17) edge [resistor] (20);
		\draw (20) to (12);
		\path (15) edge [resistor] (14);
		\path (54) edge [resistor] (56);
		\path (55) edge [resistor] (56);
		\path (56) edge [resistor] (57);
		\path (51) edge [resistor] (52);
		\path (51) edge [resistor] (53);
		\draw (57) to (46);
		\draw (47) to (58);
		\path (58) edge [resistor] (16);
		\path (3) edge [resistor] (9);
		\path (9) edge [resistor] (5);
		\draw (13) to (43);
		\path (60) edge [resistor] (61);
		\path (60) edge [resistor] (62);
		\draw (38) to (63);
		\path (63) edge [resistor] (64);
		\draw (64) to (45);
		\draw (19) to (65);
		\path (65) edge [resistor] (66);
		\draw (66) to (21);
		\path (67) edge [resistor] (69);
		\path (68) edge [resistor] (70);
		\draw (59) to (67);
		\draw (59) to (68);
		\draw (69) to (71);
		\draw (71) to (70);
	\end{pgfonlayer}
	\begin{pgfonlayer}{background}
	  \filldraw [fill=black!5!white, draw=black!40!white] (1) rectangle (2);
	  \filldraw [fill=black!5!white, draw=black!40!white] (26) rectangle (27);
	  \filldraw [fill=black!5!white, draw=black!40!white] (29) rectangle (28);
	  \filldraw [fill=black!5!white, draw=black!40!white] (35) rectangle (36);
	  \filldraw [fill=black!5!white, draw=black!40!white] (30) rectangle (34);
	\end{pgfonlayer}
\end{tikzpicture}
}
\end{aligned}
\quad
\mapsto
\enspace
\begin{aligned}
\resizebox{.4\textwidth}{!}{
    \tikzset{every path/.style={line width=1.1pt}}
  \begin{tikzpicture}[circuit ee IEC, set resistor graphic=var resistor IEC graphic]
	\begin{pgfonlayer}{nodelayer}
		\coordinate (0) at (-1.75, 4) {};
		\coordinate (1) at (1.75, 4) {};
		\coordinate (2) at (1, 4.25) {};
		\coordinate (3) at (-1, 3.75) {};
		\coordinate (4) at (0, 1.75) {};
		\coordinate (5) at (1, 3.75) {};
		\coordinate (6) at (-1, 4.25) {};
		\node [style=dotbig] (7) at (6.25, 4) {};
		\coordinate (8) at (5, 2) {};
		\coordinate (9) at (2.5, 2) {};
		\coordinate (10) at (3.75, -0) {};
		\coordinate (11) at (2.5, 3.75) {};
		\coordinate (12) at (2.5, 4.25) {};
		\coordinate (13) at (5.75, 4) {};
		\node [style=dotbig] (14) at (6.25, -1.5) {};
		\node [style=dotbig] (15) at (-5.5, -1) {};
		\node [style=dotbig] (16) at (6.25, 1.5) {};
		\node [style=dotbig] (17) at (-5.5, 4.5) {};
		\node [style=dotbig] (18) at (-5.5, 3) {};
		\coordinate (19) at (-5, -2) {};
		\coordinate (20) at (-5, 4.5) {};
		\coordinate (21) at (-5, 3) {};
		\coordinate (22) at (5.75, 5.25) {};
		\node [style=dotbig] (23) at (6.25, 0.5) {};
		\coordinate (24) at (-5, -1) {};
		\coordinate (25) at (-4, -0) {};
		\coordinate (26) at (-1.75, 0.5) {};
		\coordinate (27) at (-1.75, -0.5) {};
		\coordinate (28) at (0, -0) {};
		\coordinate (29) at (1.75, -0) {};
		\coordinate (30) at (-4.75, 1.25) {};
		\coordinate (31) at (3.75, 1) {};
		\coordinate (32) at (5.75, 1.5) {};
		\coordinate (33) at (5.75, 0.5) {};
		\coordinate (34) at (-1.75, -1.5) {};
		\coordinate (35) at (1.75, -1.5) {};
		\coordinate (36) at (5.75, -1.5) {};
		\coordinate (37) at (-4.25, 1.75) {};
		\coordinate (38) at (-4.25, 1) {};
		\coordinate (39) at (-2.75, 2.5) {};
		\coordinate (40) at (-2.5, 1.75) {};
		\coordinate (41) at (-2.25, 2.25) {};
		\coordinate (42) at (5, 4.25) {};
		\coordinate (43) at (5, 3.75) {};
	\end{pgfonlayer}
	\begin{pgfonlayer}{edgelayer}
		\draw (17) to (20);
		\draw (18) to (21);
		\draw (15) to (24);
		\draw (13) to (7);
		\draw (36) to (14);
		\draw (32) to (16);
		\draw (33) to (23);
		\draw (1) to (12);
		\draw (1) to (11);
		\draw (2) to (1);
		\draw (5) to (1);
		\draw (0) to (6);
		\draw (0) to (3);
		\draw (30) to (37);
		\draw (30) to (38);
		\draw (39) to (41);
		\draw (41) to (40);
		\draw (42) to (13);
		\draw (43) to (13);
		\path (6) edge [resistor] (2);
		\path (3) edge [resistor] (5);
		\path (9) edge [resistor] (8);
		\path (26) edge [resistor] (28);
		\path (27) edge [resistor] (28);
		\path (28) edge [resistor] (29);
		\path (0) edge [resistor] (4);
		\path (4) edge [resistor] (1);
		\path (31) edge [resistor] (32);
		\path (31) edge [resistor] (33);
		\path (34) edge [resistor] (35);
		\path (37) edge [resistor] (39);
		\path (38) edge [resistor] (40);
		\path (20) edge [resistor] (0);
		\path (21) edge [resistor] (0);
		\path (25) edge [resistor] (26);
		\path (25) edge [resistor] (27);
		\path (24) edge [resistor] (34);
		\path (29) edge [resistor] (10);
		\path (35) edge [resistor] (36);
		\path (12) edge [resistor] (42);
		\path (11) edge [resistor] (43);
	\end{pgfonlayer}
	\begin{pgfonlayer}{background}
	  \filldraw [fill=black!5!white, draw=black!40!white] (19) rectangle (22);
	\end{pgfonlayer}
\end{tikzpicture}
}
\end{aligned}
\]
Note in particular that the composite of these open circuits contains a unique
resistor for every resistor in the factors. If we are interested in describing
the syntax of a diagrammatic language, then this is useful: composition builds
given expressions into a larger one. If we are only interested in the semantics,
however, this is often unnecessary and thus often wildly inefficient.

Indeed, suppose our semantics for open circuits is given by the information that
can be gleaned by connecting other open circuits, such as measurement devices,
to the terminals. In these semantics we consider two open circuits equivalent
if, should they be encased, but for their terminals, in a black box
\[
\resizebox{.4\textwidth}{!}{
    \tikzset{every path/.style={line width=1.1pt}}
  \begin{tikzpicture}
    \begin{pgfonlayer}{nodelayer}
		\node [style=dotbig] (14) at (-5.5, 4.5) {};
		\node [style=dotbig] (15) at (-5.5, 3) {};
		\node [style=dotbig] (12) at (-5.5, -1) {};
		\node [style=dotbig] (7) at (6.25, 4) {};
		\node [style=dotbig] (13) at (6.25, 1.5) {};
		\node [style=dotbig] (20) at (6.25, 0.5) {};
		\node [style=dotbig] (11) at (6.25, -1.5) {};
		\coordinate (17) at (-5, 4.5) {};
		\coordinate (18) at (-5, 3) {};
		\coordinate (21) at (-5, -1) {};
		\coordinate (10) at (5.75, 4) {};
		\coordinate (23) at (5.75, 1.5) {};
		\coordinate (24) at (5.75, 0.5) {};
		\coordinate (27) at (5.75, -1.5) {};
		\coordinate (16) at (-5, -2) {};
		\coordinate (19) at (5.75, 5.25) {};
	\end{pgfonlayer}
	\begin{pgfonlayer}{edgelayer}
		\draw (14) to (17);
		\draw (15) to (18);
		\draw (12) to (21);
		\draw (10) to (7);
		\draw (27) to (11);
		\draw (23) to (13);
		\draw (24) to (20);
	\end{pgfonlayer}
	\begin{pgfonlayer}{background}
	  \filldraw [fill=black!80!white, draw=black!40!white] (16) rectangle (19);
	\end{pgfonlayer}
\end{tikzpicture}
}
\]
we would be unable to distinguish them through our electrical investigations. In
this case, at the very least, the previous circuit is equivalent to the circuit
\[
\resizebox{.4\textwidth}{!}{
    \tikzset{every path/.style={line width=1.1pt}}
  \begin{tikzpicture}[circuit ee IEC, set resistor graphic=var resistor IEC graphic]
	\begin{pgfonlayer}{nodelayer}
		\coordinate (0) at (-1.75, 4) {};
		\coordinate (1) at (1.75, 4) {};
		\coordinate (2) at (1, 4.25) {};
		\coordinate (3) at (-1, 3.75) {};
		\coordinate (4) at (0, 1.75) {};
		\coordinate (5) at (1, 3.75) {};
		\coordinate (6) at (-1, 4.25) {};
		\node [style=dotbig] (7) at (6.25, 4) {};
		\coordinate (8) at (2.5, 3.75) {};
		\coordinate (9) at (2.5, 4.25) {};
		\coordinate (10) at (5.75, 4) {};
		\node [style=dotbig] (11) at (6.25, -1.5) {};
		\node [style=dotbig] (12) at (-5.5, -1) {};
		\node [style=dotbig] (13) at (6.25, 1.5) {};
		\node [style=dotbig] (14) at (-5.5, 4.5) {};
		\node [style=dotbig] (15) at (-5.5, 3) {};
		\coordinate (16) at (-5, -2) {};
		\coordinate (17) at (-5, 4.5) {};
		\coordinate (18) at (-5, 3) {};
		\coordinate (19) at (5.75, 5.25) {};
		\node [style=dotbig] (20) at (6.25, 0.5) {};
		\coordinate (21) at (-5, -1) {};
		\coordinate (22) at (3.75, 1) {};
		\coordinate (23) at (5.75, 1.5) {};
		\coordinate (24) at (5.75, 0.5) {};
		\coordinate (25) at (-1.75, -1.5) {};
		\coordinate (26) at (1.75, -1.5) {};
		\coordinate (27) at (5.75, -1.5) {};
		\coordinate (28) at (5, 4.25) {};
		\coordinate (29) at (5, 3.75) {};
	\end{pgfonlayer}
	\begin{pgfonlayer}{edgelayer}
		\draw (14) to (17);
		\draw (15) to (18);
		\draw (12) to (21);
		\draw (10) to (7);
		\draw (27) to (11);
		\draw (23) to (13);
		\draw (24) to (20);
		\draw (1) to (9);
		\draw (1) to (8);
		\draw (2) to (1);
		\draw (5) to (1);
		\draw (0) to (6);
		\draw (0) to (3);
		\draw (28) to (10);
		\draw (29) to (10);
		\path (6) edge [resistor] (2);
		\path (3) edge [resistor] (5);
		\path (0) edge [resistor] (4);
		\path (4) edge [resistor] (1);
		\path (22) edge [resistor] (23);
		\path (22) edge [resistor] (24);
		\path (25) edge [resistor] (26);
		\path (17) edge [resistor] (0);
		\path (18) edge [resistor] (0);
		\path (21) edge [resistor] (25);
		\path (26) edge [resistor] (27);
		\path (9) edge [resistor] (28);
		\path (8) edge [resistor] (29);
	\end{pgfonlayer}
	\begin{pgfonlayer}{background}
	  \filldraw [fill=black!5!white, draw=black!40!white] (16) rectangle (19);
	\end{pgfonlayer}
\end{tikzpicture}
}
\]
where we have removed circuitry not connected to the terminals.  Moreover, this
second circuit is a much more efficient representation, as it does not model
inaccessible, internal structure. If we wish to construct a hypergraph category
modelling the semantics of open circuits, we require circuit representations and
a composition rule that only retain the information relevant to the black boxed
circuit.

Dealing with this problem fully requires further discussion of the semantics of
circuit diagrams, and the introduction of structures more flexible than graphs,
such as linear relations.  We deal with this in depth in Chapter
\ref{ch.circuits}. In this chapter, however, we are able to contribute an
important piece of the puzzle: corelations.

Corelations allow us to pursue a notion of composition that discards extraneous
information as we compose our systems. Consider, for example, the category
$\cospan(\FinSet)$ of cospans in the category of finite sets and functions. A
morphism is then a finite set $N$ together with functions $X \to N$ and $Y \to
N$. We depict these like so:
\begin{center}
  \begin{tikzpicture}[auto,scale=2]
    \node[circle,draw,inner sep=1pt,fill=gray,color=gray] (x1) at (-1.5,.2) {};
    \node[circle,draw,inner sep=1pt,fill=gray,color=gray] (x2) at (-1.5,-.2) {};
    \node at (-1.5,-.7) {$X$};
    \node[circle,draw,inner sep=1pt,fill]         (A) at (0,.4) {};
    \node[circle,draw,inner sep=1pt,fill]         (B) at (0,.133) {};
    \node[circle,draw,inner sep=1pt,fill]         (C) at (0,-.133) {};
    \node[circle,draw,inner sep=1pt,fill]         (D) at (0,-.4) {};
    \node at (0,-.7) {$N$};
    \node[circle,draw,inner sep=1pt,fill=gray,color=gray] (y1) at (1.5,.3) {};
    \node[circle,draw,inner sep=1pt,fill=gray,color=gray] (y2) at (1.5,.1) {};
    \node[circle,draw,inner sep=1pt,fill=gray,color=gray] (y3) at (1.5,-.1) {};
    \node[circle,draw,inner sep=1pt,fill=gray,color=gray] (y4) at (1.5,-.3) {};
    \node at (1.5,-.7) {$Y$};
    \path[color=gray, very thick, shorten >=10pt, shorten <=5pt, ->, >=stealth]
    (x1) edge (C);
    \path[color=gray, very thick, shorten >=10pt, shorten <=5pt, ->, >=stealth]
    (x2) edge (C);
    \path[color=gray, very thick, shorten >=10pt, shorten <=5pt, ->, >=stealth]
    (y1) edge (B);
    \path[color=gray, very thick, shorten >=10pt, shorten <=5pt, ->, >=stealth]
    (y2) edge (C);
    \path[color=gray, very thick, shorten >=10pt, shorten <=5pt, ->, >=stealth]
    (y3) edge (D);
    \path[color=gray, very thick, shorten >=10pt, shorten <=5pt, ->, >=stealth]
    (y4) edge (D);
  \end{tikzpicture}
\end{center}
Here $X$ is a two element set, while $N$ and $Y$ are four element sets.

Suppose we have a pair of cospans $X \to N \leftarrow Y$, $Y \to M \leftarrow
Z$. By definition, their composite has apex the pushout $N+_YM$ which, roughly
speaking, is the union of $N$ and $M$ with two points identified if they are
both images of the same element of $Y$. For example, the following pair of
cospans:
\begin{center}
  \begin{tikzpicture}[auto,scale=2]
    \node[circle,draw,inner sep=1pt,fill=gray,color=gray] (x1) at (-1.5,.2) {};
    \node[circle,draw,inner sep=1pt,fill=gray,color=gray] (x2) at (-1.5,-.2) {};
    \node at (-1.5,-.8) {$X$};
    \node[circle,draw,inner sep=1pt,fill]         (A) at (0,.4) {};
    \node[circle,draw,inner sep=1pt,fill]         (B) at (0,.133) {};
    \node[circle,draw,inner sep=1pt,fill]         (C) at (0,-.133) {};
    \node[circle,draw,inner sep=1pt,fill]         (D) at (0,-.4) {};
    \node at (0,-.8) {$N$};
    \node[circle,draw,inner sep=1pt,fill=gray,color=gray] (y1) at (1.5,.3) {};
    \node[circle,draw,inner sep=1pt,fill=gray,color=gray] (y2) at (1.5,.1) {};
    \node[circle,draw,inner sep=1pt,fill=gray,color=gray] (y3) at (1.5,-.1) {};
    \node[circle,draw,inner sep=1pt,fill=gray,color=gray] (y4) at (1.5,-.3) {};
    \node at (1.5,-.8) {$Y$};
    \node[circle,draw,inner sep=1pt,fill]         (A') at (3,.5) {};
    \node[circle,draw,inner sep=1pt,fill]         (B') at (3,.25) {};
    \node[circle,draw,inner sep=1pt,fill]         (C') at (3,0) {};
    \node[circle,draw,inner sep=1pt,fill]         (D') at (3,-.25) {};
    \node[circle,draw,inner sep=1pt,fill]         (E') at (3,-.5) {};
    \node at (3,-.8) {$M$};
    \node[circle,draw,inner sep=1pt,fill=gray,color=gray] (z1) at (4.5,.2) {};
    \node[circle,draw,inner sep=1pt,fill=gray,color=gray] (z2) at (4.5,-.2) {};
    \node at (4.5,-.8) {$Z$};
    \path[color=gray, very thick, shorten >=10pt, shorten <=5pt, ->, >=stealth]
    (x1) edge (C);
    \path[color=gray, very thick, shorten >=10pt, shorten <=5pt, ->, >=stealth]
    (x2) edge (C);
    \path[color=gray, very thick, shorten >=10pt, shorten <=5pt, ->, >=stealth]
    (y1) edge (B);
    \path[color=gray, very thick, shorten >=10pt, shorten <=5pt, ->, >=stealth]
    (y2) edge (C);
    \path[color=gray, very thick, shorten >=10pt, shorten <=5pt, ->, >=stealth]
    (y3) edge (D);
    \path[color=gray, very thick, shorten >=10pt, shorten <=5pt, ->, >=stealth]
    (y4) edge (D);
    \path[color=gray, very thick, shorten >=10pt, shorten <=5pt, ->, >=stealth]
    (y1) edge (A');
    \path[color=gray, very thick, shorten >=10pt, shorten <=5pt, ->, >=stealth]
    (y2) edge (B');
    \path[color=gray, very thick, shorten >=10pt, shorten <=5pt, ->, >=stealth]
    (y3) edge (C');
    \path[color=gray, very thick, shorten >=10pt, shorten <=5pt, ->, >=stealth]
    (y4) edge (D');
    \path[color=gray, very thick, shorten >=10pt, shorten <=5pt, ->, >=stealth]
    (z1) edge (B');
    \path[color=gray, very thick, shorten >=10pt, shorten <=5pt, ->, >=stealth]
    (z2) edge (E');
  \end{tikzpicture}
\end{center}
becomes
\begin{center}
  \begin{tikzpicture}[auto,scale=2]
    \node[circle,draw,inner sep=1pt,fill=gray,color=gray] (x1) at (1.5,.2) {};
    \node[circle,draw,inner sep=1pt,fill=gray,color=gray] (x2) at (1.5,-.2) {};
    \node at (1.5,-.8) {$X$};
    \node[circle,draw,inner sep=1pt,fill]         (A) at (3,.5) {};
    \node[circle,draw,inner sep=1pt,fill]         (B) at (3,.25) {};
    \node[circle,draw,inner sep=1pt,fill]         (C) at (3,0) {};
    \node[circle,draw,inner sep=1pt,fill]         (D) at (3,-.25) {};
    \node[circle,draw,inner sep=1pt,fill]         (E) at (3,-.5) {};
    \node at (3,-.8) {$N+_YM$};
    \node[circle,draw,inner sep=1pt,fill=gray,color=gray] (z1) at (4.5,.2) {};
    \node[circle,draw,inner sep=1pt,fill=gray,color=gray] (z2) at (4.5,-.2) {};
    \node at (4.5,-.8) {$Z$};
    \path[color=gray, very thick, shorten >=10pt, shorten <=5pt, ->, >=stealth]
    (x1) edge (C);
    \path[color=gray, very thick, shorten >=10pt, shorten <=5pt, ->, >=stealth]
    (x2) edge (C);
    \path[color=gray, very thick, shorten >=10pt, shorten <=5pt, ->, >=stealth]
    (z1) edge (C);
    \path[color=gray, very thick, shorten >=10pt, shorten <=5pt, ->, >=stealth]
    (z2) edge (E);
  \end{tikzpicture}
\end{center}
Here we see essentially the same phenomenon as we described for circuits above:
the apex of the cospan is much larger than the image of the maps from the feet.

Corelations address this with what is known as a $(\mc E,\mc M)$-factorisation
system. A factorisation system comprises subcategories $\mc E$ and $\mc M$ of
$\mc C$ such that every morphism in $\mc C$ factors, in a coherent way, as the
composite of a morphism in $\mc E$ followed by a morphism in $\mc M$. An
example, known as the epi-mono factorisation system on $\Set$, is yielded by the
observation that every function can be written as a surjection followed by an
injection.

Corelations, or more precisely $(\mc E,\mc M)$-corelations, are cospans $X
\to N \leftarrow Y$ such that the copairing $X+Y \to N$ of the two maps is an
element of the first factor $\mc E$ of the factorisation system. Composition of
corelations proceeds first as composition of cospans, but then takes only the
so-called $\mc E$-part of the composite cospan, to ensure the composite is again
a corelation. If we take the $\mc E$-part of a cospan $X \to N \leftarrow Y$, we
write the new apex $\overline{N}$, and so the resulting corelation $X \to
\overline{N} \leftarrow Y$. 

Mapping the above two cospans to epi-mono corelations in $\FinSet$ they become 
\begin{center}
  \begin{tikzpicture}[auto,scale=2]
    \node[circle,draw,inner sep=1pt,fill=gray,color=gray] (x1) at (-1.5,.2) {};
    \node[circle,draw,inner sep=1pt,fill=gray,color=gray] (x2) at (-1.5,-.2) {};
    \node at (-1.5,-.8) {$X$};
    \node[circle,draw,inner sep=1pt,fill]         (B) at (0,.3) {};
    \node[circle,draw,inner sep=1pt,fill]         (C) at (0,0) {};
    \node[circle,draw,inner sep=1pt,fill]         (D) at (0,-.3) {};
    \node at (0,-.8) {$\overline{N}$};
    \node[circle,draw,inner sep=1pt,fill=gray,color=gray] (y1) at (1.5,.3) {};
    \node[circle,draw,inner sep=1pt,fill=gray,color=gray] (y2) at (1.5,.1) {};
    \node[circle,draw,inner sep=1pt,fill=gray,color=gray] (y3) at (1.5,-.1) {};
    \node[circle,draw,inner sep=1pt,fill=gray,color=gray] (y4) at (1.5,-.3) {};
    \node at (1.5,-.8) {$Y$};
    \node[circle,draw,inner sep=1pt,fill]         (A') at (3,.5) {};
    \node[circle,draw,inner sep=1pt,fill]         (B') at (3,.25) {};
    \node[circle,draw,inner sep=1pt,fill]         (C') at (3,0) {};
    \node[circle,draw,inner sep=1pt,fill]         (D') at (3,-.25) {};
    \node[circle,draw,inner sep=1pt,fill]         (E') at (3,-.5) {};
    \node at (3,-.8) {$M=\overline{M}$};
    \node[circle,draw,inner sep=1pt,fill=gray,color=gray] (z1) at (4.5,.2) {};
    \node[circle,draw,inner sep=1pt,fill=gray,color=gray] (z2) at (4.5,-.2) {};
    \node at (4.5,-.8) {$Z$,};
    \path[color=gray, very thick, shorten >=10pt, shorten <=5pt, ->, >=stealth]
    (x1) edge (C);
    \path[color=gray, very thick, shorten >=10pt, shorten <=5pt, ->, >=stealth]
    (x2) edge (C);
    \path[color=gray, very thick, shorten >=10pt, shorten <=5pt, ->, >=stealth]
    (y1) edge (B);
    \path[color=gray, very thick, shorten >=10pt, shorten <=5pt, ->, >=stealth]
    (y2) edge (C);
    \path[color=gray, very thick, shorten >=10pt, shorten <=5pt, ->, >=stealth]
    (y3) edge (D);
    \path[color=gray, very thick, shorten >=10pt, shorten <=5pt, ->, >=stealth]
    (y4) edge (D);
    \path[color=gray, very thick, shorten >=10pt, shorten <=5pt, ->, >=stealth]
    (y1) edge (A');
    \path[color=gray, very thick, shorten >=10pt, shorten <=5pt, ->, >=stealth]
    (y2) edge (B');
    \path[color=gray, very thick, shorten >=10pt, shorten <=5pt, ->, >=stealth]
    (y3) edge (C');
    \path[color=gray, very thick, shorten >=10pt, shorten <=5pt, ->, >=stealth]
    (y4) edge (D');
    \path[color=gray, very thick, shorten >=10pt, shorten <=5pt, ->, >=stealth]
    (z1) edge (B');
    \path[color=gray, very thick, shorten >=10pt, shorten <=5pt, ->, >=stealth]
    (z2) edge (E');
  \end{tikzpicture}
\end{center}
with composite
\begin{center}
  \begin{tikzpicture}[auto,scale=2]
    \node[circle,draw,inner sep=1pt,fill=gray,color=gray] (x1) at (1.5,.2) {};
    \node[circle,draw,inner sep=1pt,fill=gray,color=gray] (x2) at (1.5,-.2) {};
    \node at (1.5,-.45) {$X$};
    \node[circle,draw,inner sep=1pt,fill]         (A) at (3,.2) {};
    \node[circle,draw,inner sep=1pt,fill]         (B) at (3,-.2) {};
    \node at (3,-.45) {$\overline{N+_YM}$};
    \node[circle,draw,inner sep=1pt,fill=gray,color=gray] (z1) at (4.5,.2) {};
    \node[circle,draw,inner sep=1pt,fill=gray,color=gray] (z2) at (4.5,-.2) {};
    \node at (4.5,-.45) {$Z$.};
    \path[color=gray, very thick, shorten >=10pt, shorten <=5pt, ->, >=stealth]
    (x1) edge (A);
    \path[color=gray, very thick, shorten >=10pt, shorten <=5pt, ->, >=stealth]
    (x2) edge (A);
    \path[color=gray, very thick, shorten >=10pt, shorten <=5pt, ->, >=stealth]
    (z1) edge (A);
    \path[color=gray, very thick, shorten >=10pt, shorten <=5pt, ->, >=stealth]
    (z2) edge (B);
  \end{tikzpicture}
\end{center}
Note that the apex of the composite corelation is the subset of the apex of the
composite cospan comprising those elements that are in the image of the maps
from the feet. The intuition, again, is that composition of corelations discards
irrelevant information---of course, exactly what it discards depends on our
choice of factorisation system.

In this chapter we show that given a category $\mc C$ with finite colimits and a
factorisation system $(\mc E,\mc M)$, if $\mc M$ obeys a condition known as
`stability under pushout', then corelations in $\mc C$ form a hypergraph
category. We also show that given a colimit-preserving functor $A$ between such
categories $\mathcal C$, $\mathcal C'$ with factorisation systems $(\mathcal E,
\mathcal M)$, $(\mathcal E', \mathcal M')$, $A$ induces a hypergraph functor
between their corelation categories if the image of $\mathcal M$ lies in
$\mathcal M'$.

\section{Corelations} \label{sec.corels}

Given sets $X$, $Y$, a relation $X \to Y$ is a subset of the product $X
\times Y$. Note that by the universal property of the product, spans $X
\leftarrow N \to Y$ are in one-to-one correspondence with functions $N \to X
\times Y$. When this map is monic, we say that the span is \emph{jointly monic}.
More abstractly then, we might say a relation is an isomorphism class of jointly
monic spans in the category of sets. Here we generalise the dual concept: these
are our so-called corelations.

Relations and, equivalently, multi-valued functions have long been a structure
of mathematical interest, with their formalisation going back to De Morgan
\cite{DeM60} and Pierce \cite{Pie70} in the 1800s. One hundred years later,
Klein \cite{Kle70} provided a category theoretic generalisation of the
concept---albeit one slightly more restrictive than that which we use here---and
by now the categorical theory of relations is well studied \cite{Mil00, JW00}.
The key insight is the definition of a factorisation system. We begin this section
by introducing this idea, before developing the theory of not just categories
of relations, but monoidal categories of (co)relations.

\subsection{Factorisation systems}
The relevant properties of jointly monic spans come from the fact that
monomorphisms form one half of a factorisation system. A factorisation system
allows any morphism in a category to be factored into the composite of two
morphisms in a coherent way. This subsection introduces factorisation systems
and monoidal factorisation systems.

\begin{definition}
  A \define{factorisation system} $(\mathcal E,\mathcal M)$ in a category
  $\mathcal C$ comprises subcategories $\mathcal E$, $\mathcal M$ of $\mathcal
  C$ such that
  \begin{enumerate}[(i)]
    \item $\mathcal E$ and $\mathcal M$ contain all isomorphisms of $\mathcal
      C$.
    \item  every morphism $f \in \mathcal C$ admits a factorisation $f=m \circ
      e$, $e \in \mathcal E$, $m \in \mathcal M$.
\item given morphisms $f,f'$, with factorisations $f = m \circ e$, $f' = m' \circ
  e'$ of the above sort, for every $u$, $v$ such that the square
  \[
    \xymatrixcolsep{3pc}
    \xymatrixrowsep{3pc}
    \xymatrix{
       \ar[r]^f \ar[d]_u &  \ar[d]^v \\
       \ar[r]_{f'} & 
    }
  \]
  commutes, there exists a unique morphism $s$ such that
  \[
    \xymatrixcolsep{3pc}
    \xymatrixrowsep{3pc}
    \xymatrix{
      \ar[r]^e \ar[d]_u & \ar[r]^m \ar@{.>}[d]^{\exists! s} &  \ar[d]^v \\
       \ar[r]_{e'}& \ar[r]_{m'} & 
    }
  \]
  commutes.
  \end{enumerate}
\end{definition}

\begin{examples} \label{ex.factsysts}\ 

  \begin{itemize}
    \item Write $\mathcal I_{\mathcal C}$ for the wide subcategory of
      $\mathcal C$ containing exactly the isomorphisms of $\mathcal C$. Then
      $(\mathcal I_{\mathcal C}, \mathcal C)$ and $(\mathcal C, \mathcal
      I_{\mathcal C})$ are both factorisation systems in $\mathcal C$. While
      these may seem too trivial to mention, we will see they are of central
      importance in what follows.
    
    \item The prototypical example of a factorisation system is the epi-mono
      factorisation system $(\mathrm{Sur},\mathrm{Inj})$ in $\Set$. Here we
      write $\mathrm{Sur}$ for the subcategory of surjections in $\Set$, and
      $\mathrm{Inj}$ for the subcategory of injections. 
      
      Recall that split monomorphism is a map $m\colon X\to Y$ such that there
      exists a one-sided inverse, i.e. a map $m'\colon Y\to X$ such that
      $m'm=\idn_X$. Observe that all monos in $\Set$ split.  One way of proving
      the above is a factorisation system on $\Set$ is via the more general
      fact, true in any category: if every arrow can be factorised as an epi
      followed by a split mono, then epimorphisms and split monomorphisms form
      the factors of a factorisation system.  The only non-trivial part to check
      is the uniqueness condition: given epis $e_1,e_2$, split monos $m_1,m_2$,
      and commutative diagram
      \[
	\xymatrixcolsep{3pc}
	\xymatrixrowsep{3pc}
	\xymatrix{
	  \ar[r]^{e_1} \ar[d]_u & \ar[r]^{m_1} \ar@{.>}[d]^{\exists! t} &
	  \ar[d]^v \\
	  \ar[r]_{e_2}& \ar[r]_{m_2} & 
	}
      \]
      we must show that there is a unique $t$ that makes the diagram commute.
      Indeed let $t= m_2'vm_1$ where $m_2'$ satisfies $m_2'm_2=id$. 
      To see that the right square commutes, observe
      \[
	m_2 t e_1 =  m_2 m_2' v m_1 e_1 = m_2 m_2' m_2 e_2 u = m_2 e_2 u = v m_1 e_1
      \]
      and since $e_1$ is epi we have $m_2 t = v m_1$. For the left square,
      \[
	t e_1 = m_2' v m_1 e_1 = m_2' m_2 e_2 u = e_2 u.
      \] 
      Uniqueness is immediate, since, $e_1$ is epi and $m_2$ is mono. 
    
    \item Recall that a regular epimorphism in an epimorphism that is the
      coequaliser for some pair of parallel morphisms. (Dually, a regular
      monomorphism is an equaliser for some pair of parallel morphisms.) In any
      so-called regular category the regular epimorphisms and monomorphisms form
      a factorisation system.  Examples of regular categories include $\Set$,
      toposes, and abelian categories. Regular categories were introduced by
      Barr and Grillet; for more details see \cite{Bar71,Gri71}.
\end{itemize}
More details and further examples can be found in \cite[\textsection 14]{AHS}.
\end{examples}

As we are concerned with building monoidal categories of corelations, it will be
important that our factorisation systems are monoidal factorisation systems.

\begin{definition}
Call a factorisation system $(\mc E,\mc M)$ in a monoidal category $(\mc C,\ot)$
a \define{monoidal factorisation system} if $(\mc E,\ot)$ is a monoidal
category.
\end{definition}

One might wonder why $\mc M$ does not appear in the above definition. To give a
touch more intuition for this definition, we quote a theorem of Ambler. Recall a
symmetric monoidal closed category is one in which each functor $- \ot X$ has a
specified right adjoint $[X,-]$. 
\begin{proposition}
  Let $(\mc E,\mc M)$ be a factorisation system in a symmetric monoidal
  closed category $(\mc C,\ot)$. Then the following are equivalent:
  \begin{enumerate}[(i)]
    \item $(\mc E,\mc M)$ is a monoidal factorisation system.
    \item $\mc E$ is closed under $- \ot X$ for all $X \in \mc C$.
    \item $\mc M$ is closed under $[X,-]$ for all $X \in \mc C$.
\end{enumerate}
\end{proposition}
\begin{proof}
  See Ambler for proof and further details \cite[Lemma 5.2.2]{Am}.
\end{proof}


We need not worry too much, however, about the distinction between factorisation
systems and monoidal factorisation systems. The reason is that for our
purposes---where the underlying category $\mc C$ has finite colimits and the
monoidal product is the coproduct---\emph{all} factorisation systems are
monoidal factorisation systems. This is implied by the following lemma.

\begin{lemma} \label{lem.monfact}
  Let $\mc C$ be a category with finite coproducts, and let $(\mc E, \mc M)$ be a
  factorisation system on $\mc C$. Then $(\mc E,+)$ is a symmetric monoidal
  category.
\end{lemma}
\begin{proof}
  The only thing to check is that $\mc E$ is closed under $+$. That is, given
  $f\maps A \to B$ and $g\maps C \to D$ in $\mc E$, we wish to show that
  $f+g\maps A+C \to B+D$, defined in $\mc C$, is also a morphism in $\mc E$. 

  Let $f+g$ have factorisation $A+C \stackrel{e}\longrightarrow \overline{B+D}
  \stackrel{m}\longrightarrow B+D$, where $e \in \mc E$ and $m \in \mc
  M$. We will prove that $m$ is an isomorphism. To construct an inverse, recall
  that by definition, as $f$ and $g$ lie in $\mc E$, there exist morphisms
  $x\maps B \to \overline{B+D}$ and $y\maps D \to \overline{B+D}$ such that
  \[ \label{eq.coreltensor}
    \xymatrixcolsep{2pc}
    \xymatrixrowsep{2pc}
    \xymatrix{
      A \ar[r]^f \ar[d] & B \ar@{=}[r] \ar@{.>}[d]^x & B
      \ar[d] \\
      A+C \ar[r]_{e}&\overline{B+D} \ar[r]_{m} & B+D
    }
    \qquad \mbox{and} \qquad
    \xymatrixcolsep{2pc}
    \xymatrixrowsep{2pc}
    \xymatrix{
      C \ar[r]^g \ar[d] & D \ar@{=}[r] \ar@{.>}[d]^y & D
      \ar[d] \\
      A+C \ar[r]_{e}&\overline{B+D} \ar[r]_{m} & B+D
    }
    \tag{1}
  \]
  The copairing $[x,y]$ is an inverse to $m$. 
  
  Indeed, taking the coproduct of the top rows of the two diagrams above and the
  copairings of the vertical maps gives the commutative diagram
  \[
    \xymatrix{
      A+C \ar[r]^{f+g} \ar@{=}[d] & B+D \ar@{=}[r] \ar[d]_{[x,y]} & B+D \ar@{=}[d] \\
      A+C \ar[r]^{e} & \overline{B+D} \ar[r]^{m} & B+D
    }
  \]
  Reading the right-hand square immediately gives $m \circ [x,y] =1$.
  
  Conversely, to see that $[x,y] \circ m = 1$, remember that by definition $f+g
  = m \circ e$. So the left-hand square above implies that
  \[
    \xymatrixcolsep{2pc}
    \xymatrixrowsep{2pc}
    \xymatrix{
      A+C \ar[r]^e \ar@{=}[d] & \overline{B+D} \ar[d]^{[x,y] \circ m} \\
      A+C \ar[r]_{e}&\overline{B+D} 
    }
  \]
  commutes. But by the universal property of factorisation systems, there is a
  unique map $\overline{B+D} \to \overline{B+D}$ such that this diagram
  commutes, and clearly the identity map also suffices. Thus $[x,y] \circ m =
  1$.
\end{proof}

\subsection{Corelations}
Now that we have introduced factorisation systems, observe that relations are
just cospans $X \leftarrow N \to Y$ in $\Set$ such that $N \to X \times Y$ is an
element of $\mathrm{Inj}$, the right factor in the factorisation system
$(\mathrm{Sur},\mathrm{Inj})$. Relations may thus be generalised as spans such
that the span maps jointly belong to some class $\mc M$ of an $(\mc E,\mc
M)$-factorisation system. We define corelations in the dual manner.

\begin{definition}
  Let $\mathcal C$ be a category with finite colimits, and let $(\mathcal E,
  \mathcal M)$ be a factorisation system on $\mathcal C$. An $(\mathcal
  E,\mathcal M)$\define{-corelation} $X \to Y$ is a cospan $X
  \stackrel{i}\longrightarrow N \stackrel{o}\longleftarrow Y$ in $\mc C$ such
  that the copairing $[i,o]\maps X+Y \to N$ lies in $\mathcal E$.
\end{definition}

When the factorisation system is clear from context, we simply call $(\mathcal
E,\mathcal M)$-corelations `corelations'.

We also say that a cospan $X \stackrel{i}\longrightarrow N
\stackrel{o}\longleftarrow Y$ with the property that the copairing $[i,o]\maps
X+Y \to N$ lies in $\mathcal E$ is \define{jointly} $\mathcal E$\define{-like}.
Note that if a cospan is jointly $\mc E$-like then so are all isomorphic
cospans. Thus the property of being a corelation is closed under isomorphism of
cospans, and we again are often lazy with out language, referring to both
jointly $\mc E$-like cospans and their isomorphism classes as corelations. 

If $f\maps A \to N$ is a morphism with factorisation $f = m \circ e$, write
$\overline N$ for the object such that $e\maps A \to \overline N$ and $m\maps
\overline N \to N$. Now, given a cospan $X \stackrel{i_X}{\longrightarrow} N
\stackrel{o_Y}{\longleftarrow} Y$, we may use the factorisation system to write
the copairing $[i_X,o_Y]\maps X+Y \to N$ as
\[
  X+Y \stackrel{e}{\longrightarrow} \overline{N} \stackrel{m}{\longrightarrow}
  N.
\]
From the universal property of the coproduct, we also have maps $\iota_X\maps X
\to X+Y$ and $\iota_Y\maps Y \to X+Y$. We then call the corelation 
\[
  X \stackrel{e\circ \iota_X}{\longrightarrow} \overline{N} \stackrel{e \circ
  \iota_Y}{\longleftarrow} Y
\]
the $\mathcal E$\define{-part} of the above cospan. On occasion we will also
write $e\maps X+Y \to \overline N$ for the same corelation.

\begin{examples} \label{ex.corels}
  Many examples of corelations are already familiar.
  \begin{itemize}
    \item For the morphism-isomorphism factorisation system $(\mc C,\mc I_{\mc
      C})$, corelations are just cospans.

    \item For the isomorphism-morphism factorisation $(\mc I_{\mc C}, \mc C)$,
      jointly $\mc I_{\mc C}$-like cospans $X \to Y$ are simply isomorphisms
      $X+Y \stackrel\sim\to N$. Thus there is a unique isomorphism class of
      corelations between any two objects.

    \item Note that the category $\Set$ has finite colimits and an epi-mono
      factorisation system $(\mathrm{Sur},\mathrm{Inj})$. Epi-mono corelations
      from $X \to Y$ in $\mathrm{Set}$ surjective functions $X+Y \to N$; thus
      their isomorphism classes are partitions, or equivalence relations on
      $X+Y$. 

      While this compositional structure on equivalence relations has not the
      prominence of that on relations, these corelations have been recognised as
      an important structure. Ellerman gives a detailed treatment from a logic
      viewpoint in \cite{Ell14}, while basic category theoretic aspects can be
      found in Lawvere and Rosebrugh \cite{LR}.  Note that in these and in other
      sources, including \cite{CF,BF}, the term corelation is used to solely
      refer to these $(\mathrm{Sur},\mathrm{Inj})$-corelations in $\Set$, while
      here we use the term corelation primarily in the generalised sense.
  \end{itemize}
\end{examples}

Our next task is to define a composition rule on corelations.

\subsection{Categories of corelations}
We begin this subsection by defining a composition rule on isomorphism classes
of corelations. The end goal, however, is to define a hypergraph category in
which the morphism are corelations. Here we will explain how to define such a
category, introducing and exploring the important condition that $\mc M$ is
stable under pushout. We leave the proof that we have truly defined a hypergraph
category for the next subsection.

We compose corelations by taking the $\mathcal E$-part of their composite
cospan. That is, given corelations $X \stackrel{i_X}{\longrightarrow} N
\stackrel{o_Y}{\longleftarrow} Y$ and $Y \stackrel{i_Y}{\longrightarrow} M
\stackrel{o_Z}{\longleftarrow} Z$, their composite is given by the cospan $X
\xrightarrow{e\circ\iota_X} \overline{N+_YM} \xleftarrow{e \circ \iota_Z} Z$ in the commutative diagram
\[
  \begin{tikzcd}[row sep=7ex,column sep=7ex]
    && N+_YM \\
    && \overline{N+_YM} \arrow[u,pos=.4,"m"] \\
    & N \arrow[uur,"j_N"] & X+Z \arrow[from=dll,pos=.35,swap,"\iota_X"]
    \arrow[from=drr,pos=.35,"\iota_Z"]
    \arrow[u,"e"] & 
    M \arrow[uul,swap,"j_M"] \\
    X \arrow[ur,"i_X"] && Y
    \arrow[ul,crossing over,pos=.35,"o_Y"] \arrow[ur,crossing
    over,pos=.35,swap,"i_Y"] && Z,
    \arrow[ul,swap,"o_Z"]
  \end{tikzcd}
\]
where $m \circ e$ is the $(\mc E,\mc M)$-factorisation of $[j_N\circ i_X,j_M
\circ o_Z]\maps X+Z
\to N+_YM$. 

It is well known that this composite is unique up to isomorphism, and that when
$\mc M$ is well behaved it defines a category with isomorphism classes of
corelations as morphisms. For instance, a bicategorical version of the dual
theorem, for spans and relations, can be found in \cite{JW00}. Nonetheless, for
the sake of completeness we will explain all the details and sketch our own
argument here. The first fact to show is that, as we have just stated, the
composite of corelations is unique up to isomorphism.

\begin{proposition} \label{prop.corelcomp}
  Let $\mc C$ be a category with finite colimits and with a factorisation system
  $(\mc E,\mc M)$. Then the above is a well-defined composition rule on
  isomorphism classes of corelations.
\end{proposition}
\begin{proof}
  Let
  $(X \stackrel{i_X}{\longrightarrow} N \stackrel{o_Y}{\longleftarrow} Y$,
  $X \stackrel{i_X'}{\longrightarrow} N' \stackrel{o_Y'}{\longleftarrow} Y)$
%  \[
%    X \stackrel{i_X}{\longrightarrow} N \stackrel{o_Y}{\longleftarrow} Y 
%    \qquad \mbox{and} \qquad 
%    X \stackrel{i_X'}{\longrightarrow} N' \stackrel{o_Y'}{\longleftarrow} Y
%  \]
  and
  $(Y \stackrel{i_Y}{\longrightarrow} M \stackrel{o_Z}{\longleftarrow} Z$, $Y
  \stackrel{i_Y'}{\longrightarrow} M' \stackrel{o_Z'}{\longleftarrow} Z)$
%  \[
%    Y \stackrel{i_Y}{\longrightarrow} M \stackrel{o_Z}{\longleftarrow} Z 
%    \qquad \mbox{and} \qquad 
%    Y \stackrel{i_Y'}{\longrightarrow} M' \stackrel{o_Z'}{\longleftarrow} Z
%  \]
  be pairs of isomorphic jointly $\mathcal E$-like cospans. By Proposition
  \ref{prop.composingdeccospans} their composites \emph{as cospans} 
  % $(X \longrightarrow N+_YM \longleftarrow Z)$ and $(X
  %\longrightarrow N'+_YM' \longleftarrow Z)$
%  \[
%    X \longrightarrow N+_YM \longleftarrow Z 
%    \qquad \mbox{and} \qquad 
%    X \longrightarrow N'+_YM' \longleftarrow Z
%  \]
  are isomorphic via an isomorphism $p\maps N+_YM \to N'+_YM'$. The
  factorisation system then gives an isomorphism $s$ such that the diagram
  \[
    \xymatrixcolsep{3pc}
    \xymatrixrowsep{2pc}
    \xymatrix{
      X+Z \ar[r]^e \ar@{=}[d] & \overline{N+_YM} \ar[r]^m \ar@{.>}[d]^{s}_\sim & N+_YM
      \ar[d]_\sim^{p} \\
      X+Z \ar[r]_{e'}&\overline{N'+_YM'} \ar[r]_{m'} & N'+_YM'
    }
  \]
  commutes. Thus $s$ is an isomorphism of the composite corelations.
\end{proof}

As we have said, this composition rule only gives a category when $\mc M$ is
well behaved. The reason is that composition of corelations is \emph{not}
associative in general. It is, however, associative when $\mathcal M$ is stable
under pushout. We now define this, provide some examples, and prove a crucial
lemma.

\begin{definition}
  Given a category $\mc C$, we say that a subcategory $\mc M$ is \define{stable
  under pushout} if for every pushout square
  \[
    \xymatrixcolsep{3pc}
    \xymatrixrowsep{3pc}
    \xymatrix{
      \ar[r]^j & \\
      \ar[u] \ar[r]^m &  \ar[u]
    }
  \]
  such that $m \in \mathcal M$, we also have that $j \in \mathcal M$. 
\end{definition}

\begin{examples}
  There are many examples of $(\mc E,\mc M)$-factorisation systems with $\mc M$
  stable under pushout, including the factorisation systems $(\mc I_{\mc C},\mc
  C)$ and $(\mc C,\mc I_{\mc C})$ in $\mc C$, and $(\mathrm{Sur},\mathrm{Inj})$
  in $\Set$.

  This last example generalises to any topos. Indeed, Lack and Soboci\'nski
  showed that monomorphisms are stable under pushout in any adhesive category
  \cite{LS04}.  Since any topos is both a regular category and an adhesive
  category \cite{LS06,Lac11}, the regular epimorphism-monomorphism factorisation
  system in any topos is an $(\mc E,\mc M)$-factorisation system with $\mc M$
  stable under pushout.
  
  Another example is the dual of any regular category. Such a category is known
  as a coregular category, and is by definition a category that has finite
  colimits and an epimorphism-regular monomorphism factorisation system with
  regular monomorphisms stable under pushout. Examples of these include the
  category of topological spaces and continuous maps, as well as $\Set^\opp$,
  any cotopos, and so on.
\end{examples}

Stability under pushout is a powerful property. A key corollary, both for
associativity and in general, is that it implies $\mc M$ is also closed under
$+$. 
%For our proof sketch we rely on the following useful lemma.

\begin{lemma} \label{lem.mcoproductsmc}
  Let $\mathcal C$ be a category with finite colimits, and let $\mathcal M$ be a
  subcategory of $\mathcal C$ stable under pushouts and containing all
  isomorphisms. Then $(\mc M,+)$ is a symmetric monoidal category.
\end{lemma}
\begin{proof}
  It is enough to show that for all morphisms $m,m' \in \mc M$ we have $m+m'$ in
  $\mc M$. Since $\mc M$ contains all isomorphisms, the coherence maps are
  inherited from $\mc C$. The required axioms---the functoriality of the tensor
  product, the naturality of the coherence maps, and the coherence laws---are
  also inherited as they hold in $\mc C$.

  To see $m+m'$ is in $\mc M$, simply observe that we have the pushout square
  \[
  \xymatrixcolsep{3pc}
  \xymatrixrowsep{2pc}
    \xymatrix{
      A+C \ar[r]^{m+1} & B+C \\
      A \ar[r]^m \ar[u]^{\iota} & B \ar[u]_{\iota} \\
    }
  \]
  in $\mc C$. As $\mc M$ is stable under pushout, $m+1 \in \mc M$. Similarly,
  $1+m' \in \mc M$. Thus their composite $m+m'$ lies in $\mc M$, as required.
\end{proof}

An analogous argument shows that pushouts of maps $m+_Ym'$ also lie in $\mc M$.
Using this lemma it is not difficult to show associativity---the key point is
that factorisation `commutes' with pushouts, and that we have a category
$\corel_{(\mc E,\mc M)}(\mc C)$. Again, this is all well known, and can be found
in \cite{JW00}. We will incidentally reprove these facts in the following,
while pursuing richer structure.

%\begin{proposition}
%  Let $\mc C$ be a category with finite colimits and a factorisation system
%  $(\mc E,\mc M)$ such that $\mc M$ is stable under pushouts. Then composition
%  of corelations is associative.
%\end{proposition}
%\begin{proof}[Proof (sketch)]
%  The key concern is whether factorisation `commutes' with taking pushouts.
%  Using the above lemma and the fact that $\mc M$ is stable under pushouts, the
%  pushout square 
%  \[
%  \xymatrixcolsep{3pc}
%  \xymatrixrowsep{2pc}
%  \xymatrix{
%    N+_Y\overline{M} \ar[r]^{m'} & N+_YM \\
%    N+\overline{M} \ar[u] \ar[r]^{1+m} & N+M  \ar[u]
%  }
%  \]
%  shows that the map $m'\maps N+_Y\overline{M} \to N+_YM$ is in $\mc M$
%  whenever $m\maps \overline{N} \to N$ is. It can then be shown that the
%  composite of any number of corelations can be given by taking the composite of
%  them all as cospans, and then taking the jointly $\mc E$-like part.
%\end{proof}

%The identity axioms for corelations are self-evident. We thus have a category
%$\corel_{(\mc E,\mc M)}(\mc C)$.  Again, we will drop explicit reference to the
%factorisation system when context allows, simply writing $\mathrm{Corel}(\mc
%C)$.
%
%\begin{remark}
%An aside: proving associativity is not logically necessary in the development
%of this chapter. Associativity will follow from the existence of the functor
%from $\cospan(\mc C)$ to $\corel(\mc C)$, like the other necessary coherence
%results for hypergraph categories. Nonetheless, it is terminologically easier to
%establish that $\corel(\mc C)$ is indeed a category at this point, before we
%define hypergraph structure and functors on it.
%\end{remark}
%

Indeed, for modelling networks, we require not just a category, but a hypergraph
category. Corelation categories come equipped with this extra structure.  Recall
that we gave decorated cospan categories a hypergraph structure by defining a
wide embedding $\mathrm{Cospan}(\mc C) \hookrightarrow F\mathrm{Cospan}$, via
which $F\mathrm{Cospan}$ inherited the coherence and Frobenius maps (Theorem
\ref{thm:fcospans}). We will argue similarly here, after showing that the map
\[
  \cospan(\mc C) \longrightarrow \corel(\mc C)
\]
taking each cospan to its jointly $\mc E$-like part is functorial. Indeed, we
define the coherence and Frobenius maps of $\corel(\mc C)$ to be their image
under this map. For the monoidal product we again use the coproduct in $\mc C$;
the monoidal product of two corelations is their monoidal product as cospans.

\begin{theorem} \label{thm.cospantocorel}
  Let $\mathcal C$ be a category with finite colimits, and let $(\mathcal E,
  \mathcal M)$ be factorisation system on $\mathcal C$ such that $\mathcal M$ is
  stable under pushout. Then there exists a hypergraph category
  $\mathrm{Corel}_{(\mc E,\mc M)}(\mathcal C)$ with 
  \smallskip

  \begin{center}
    \begin{tabular}{| c | p{.65\textwidth} |}
      \hline
      \multicolumn{2}{|c|}{The hypergraph category $(\mathrm{Corel}_{(\mc E,\mc M)}(\mc C),+)$} \\
      \hline
      \textbf{objects} & the objects of $\mathcal C$ \\ 
      \textbf{morphisms} & isomorphism classes of $(\mc E,\mc M)$-corelations in $\mathcal C$\\ 
      \textbf{composition} & given by the $\mc E$-part of pushout \\
      \textbf{monoidal product} & the coproduct in $\mathcal C$ \\
      \textbf{coherence maps} & inherited from $\cospan(\mc C)$  \\
      \textbf{hypergraph maps} & inherited from $\cospan(\mc C)$ \\
      \hline
    \end{tabular}
  \end{center}  
  \smallskip
\end{theorem}

Again, we will drop explicit reference to the factorisation system when context allows, simply writing $\mathrm{Corel}(\mc C)$.

\begin{examples} \label{ex.corelcats}
In each factorisation system of Examples \ref{ex.corels} the right factor
$\mathcal M$ is stable under pushout. The hypergraph category 
$\corel_{(\mc C,\mc I_{\mc C})}(\mc C)$ is the just the hypergraph category of
cospans in $\mc C$. In the hypergraph category $\corel_{(\mc I_{\mc C},\mc
C)}(\mc C)$, the only morphism between any two objects $X$ and $Y$ is the
isomorphism class of the corelation $X \xrightarrow{\iota_X} X+Y
\xleftarrow{\iota_Y} Y$ corresponding to the identity map $X+Y \to X+Y$ in $\mc
I_{\mc C}$. Thus $\corel_{(\mc I_{\mc C},\mc C)}(\mc C)$ is the indiscrete
category on the objects of $\mc C$. 

We discussed epi-mono corelations in $\FinSet$ informally in our motivation
section \textsection\ref{sec.blackboxing}. These are equivalence relations
between finite sets. This example is perhaps the most instructive for
black-boxing open systems, and we will return to it in
\textsection\ref{ssec.equivrels}.
\end{examples}

Proposition \ref{prop.corelcomp} shows that our composition rule is a
well-defined function; Lemma \ref{lem.monfact} shows likewise for the monoidal
product $+\maps \corel(\mc C)\times \corel(\mc C) \to \corel(\mc C)$. Thus we
have the required data for a hypergraph category. It remains to check a
number of axioms: associativity and unitality of the categorical composition,
functoriality of the monoidal product, naturality of the coherence maps, the
coherence axioms for symmetric monoidal categories, the Frobenius laws.

Our strategy for this will be to show that the surjective function from cospans
to corelations defined by taking a cospan to its jointly $\mc E$-part preserves
both composition and the monoidal product. This then implies that to evaluate an
expression in the monoidal category of corelations, we may simply evaluate it in
the monoidal category of cospans, and then take the $\mc E$-part. Thus if an
equation is true for cospans, it is true for corelations.

Instead of proving just this, however, we will prove a generalisation regarding
an analogous map between any two corelation categories. Such a map exists
whenever we have two corelation categories $\corel_{(\mc E,\mc M)}(\mc C)$ and
$\corel_{(\mc E',\mc M')}(\mc C')$ and a colimit
preserving functor $A\maps \mc C \to \mc C'$ such that the image of $\mc M$ lies
in $\mc M'$. As $(\mc C,\mc I_{\mc C})$-corelations are just cospans, this
reduces to the desired special case by taking the domain to be the category of
$(\mc C,\mc I_{\mc C})$-corelations, $\mc C'$ to be equal to $\mc C$, and $A$ to
be the identity functor. But the generality is not spurious: it has the
advantage of proving the existence of a class of hypergraph functors between
corelation categories in the same fell swoop.

Although a touch convoluted, this strategy is worth the pause for thought. We
will use it once again for \emph{decorated} corelations, to great economy.

%\begin{lemma}
%  The map $(-)+(-)\maps \corel(\mc C)\times\corel(\mc C) \to \corel(\mc C)$
%  induced by the coproduct in $\mc C$ is functorial.
%\end{lemma}
%\begin{proof}
%As a preliminary, note that $+$ is a well-defined function: Lemma
%\ref{lem.monfact} implies that the coproduct of two corelations is again a
%corelation. Our task is to show that, given the four corelations 
%\[
%  \xymatrixrowsep{0pt}
%  \xymatrix{
%  f= X \longrightarrow N \longleftarrow Y  & &
%  g= Y \longrightarrow M \longleftarrow Z \\
%  h = X' \longrightarrow N' \longleftarrow Y' & &
% k= Y' \longrightarrow M' \longleftarrow Z'
%}
%\]
%we have $(g \circ f) + (k \circ h) = (g + k) \circ (f + h)$. The left and
%right side of this expression are respectively given by the first arrow in the
%upper and lower rows of the commutative diagram
%\[
%  \xymatrix{
%    (X+Z)+(X'+Z') \ar[r]^{\mc E+\mc E} \ar[d]^{\sim} & 
%    (\overline{N+_YM})+(\overline{N'+_{Y'}M'}) \ar[r]^{\mc M+\mc M} \ar@{.}[d] &
%    (N+_YM)+(N'+_{Y'}M') \ar[d]^{\sim} \\
%    (X+X')+(Z+Z') \ar[r]^{\mc E} &
%    \overline{(N+N')+_{Y+Y'}(M+M')} \ar[r]^{\mc M} & 
%    (N+N')+_{Y+Y'}(M+M'). \\
%  }
%\]
%The leftmost and rightmost vertical arrows are isomorphisms by properties of
%colimits. The upper row is an $(\mc E,\mc M)$-factorisation as the first map is
%the coproduct of two maps in $\mc E$ and the second map is the coproduct of two
%maps in $\mc M$, both of which are monoidal with respect to the coproduct
%(Lemmas \ref{lem.monfact} and \ref{lem.mcoproductsmc}). The lower row is an
%$(\mc E,\mc M)$-factorisation by definition. Thus, by the properties of
%factorisation systems, the dotted map $s$ is an isomorphism, and hence $+$ is
%functorial. 
%\end{proof}
%
%To prove Theorem \ref{thm.cospantocorel} it remains to show that our proposed
%data for $\mathrm{Corel}(\mc C)$ obey the necessary axioms: the symmetric
%monoidal coherence laws, the special commutative Frobenius monoid laws. Our
%proof strategy will be a touch complicated. Again, recall that in Theorem
%\ref{thm:fcospans} we proved $F\mathrm{Cospan}$ had hypergraph structure via
%a functor from $\cospan(\mc C)$. Instead of proving each axiom directly, we will
%again leverage the fact that we already know $\cospan(\mc C)$ is a hypergraph
%category, and show that $\corel(\mc C)$ is the image of $\cospan(\mc C)$ under a
%composition-preserving map that also respects (indeed, defines) the monoidal and
%hypergraph structure.
%
%Note that $\mc C$ is trivially stable under pushout. By definition we have the
%equality of hypergraph categories $\mathrm{Corel}_{(\mc C,\mc I_{\mc C})}(\mc
%C)= \cospan(\mc C)$. Thus the functor $\cospan(\mc C) \to \mathrm{Corel}(\mc C)$
%is a special case of functors between corelation categories. Taking advantage of
%this, we will discuss functors between corelation categories in general, before
%specialising to this case to prove that $\mathrm{Corel}(\mc C)$ indeed is a
%well-defined hypergraph category.
%
%This will be done in the next subsection.



\section{Functors between corelation categories} \label{sec.corelfunctors}
We have seen that to construct a functor between cospan categories one may start
with a colimit-preserving functor between the underlying categories. Corelation
categories are similar to cospan categories, but we remove the part of each
cospan that lies in $\mc M$. Hence for functors between corelation categories,
we require not just a colimit-preserving functor between the underlying
categories with finite colimits, but also that the part that is removed from the
image is greater than the part removed in the source category.

We devote the next few pages to proving the following proposition. Along the way
we prove, as promised, that corelation categories are well-defined hypergraph
categories.

\begin{proposition} \label{prop.corelfunctors}
  Let $\mathcal C$, $\mathcal C'$ have finite colimits and respective
  factorisation systems $(\mathcal E, \mathcal M)$, $(\mathcal E', \mathcal M')$,
  such that $\mathcal M$ and $\mathcal M'$ are stable under pushout. Further let
  $A\maps \mathcal C \to \mathcal C'$ be a functor that preserves finite colimits
  and such that the image of $\mathcal M$ lies in $\mathcal M'$.

  Then we may define a hypergraph functor $\square\maps \corel(\mathcal C) \to
  \corel(\mathcal C')$ sending each object $X$ in $\corel(\mathcal C)$ to $AX$ in
  $\corel(\mc C')$ and each corelation 
  \[
    X \stackrel{i_X}{\longrightarrow} N \stackrel{o_Y}{\longleftarrow} Y 
  \]
  to the $\mc E'$-part
  \[
    AX \stackrel{Ai_X}{\longrightarrow} \overline{AN}
    \stackrel{Ao_Y}{\longleftarrow} AY.
  \]
  of the image cospan. The coherence maps are the $\mc E'$-part
  $\overline{\kappa_{X,Y}}$ of the isomorphisms $\kappa_{X,Y}\maps AX+AY \to
  A(X+Y)$ given as $A$ preserves colimits.
\end{proposition}

As discussed, we still have to prove that $\corel(\mc C)$ is a hypergraph
category. We address this first with two lemmas regarding these proposed
functors.

\begin{lemma} \label{lem.corelfuncomposition}
  The above function $\square\maps \corel(\mc C) \to \corel(\mc C')$ preserves
  composition.
\end{lemma}
\begin{proof}
  Let $f = (X \longrightarrow N \longleftarrow Y)$ and $g= (Y \longrightarrow M
  \longleftarrow Z)$ be corelations in $\mathcal C$. By definition, the
  corelations $\square(g) \circ \square(f)$ and $\square(g \circ f)$ are given
  by the first arrows in the top and bottom row respectively of the diagram:
  \[ \label{diag.eparts}
    \begin{aligned}
      \xymatrixcolsep{5.5pc}
      \xymatrixrowsep{2pc}
      \xymatrix{
	\scriptstyle AX+AZ \ar[r]^{\mc E'} \ar@{=}[d] & \scriptstyle \overline{\overline{AN}+_{AY}\overline{AM}}
	\ar[r]^{\mc M'} \ar@{<.>}[d]^{n} & \scriptstyle \overline{AN}+_{AY}\overline{AM}
	\ar[r]^{m'_{AN}+_{AY}m'_{AM}} & \scriptstyle
	AN+_{AY}AM \\
	\scriptstyle AX+AZ \ar[r]^{\mc E'} & \scriptstyle \overline{A(\overline{N+_YM})} \ar[r]^{\mc M'} & \scriptstyle
	A(\overline{N+_YM}) \ar[r]^{Am_{N+_YM}} & \scriptstyle A(N+_YM) \ar@{<->}[u]_{\sim}
      }
    \end{aligned}
    \tag{$\ast$}
  \]
  The morphisms labelled $\mc E'$ lie in $\mc E'$, and similarly for $\mc M'$;
  these are given by the factorisation system on $\mc C'$.  The maps
  $Am_{N+_YM}$ and $m'_{AN}+_{AY}m'_{AM}$ lie in $\mc M'$ too: $Am_{N+_YM}$ as
  it is in the image of $\mc M$, and $m'_{AN}+_{AY}m'_{AM}$ as $\mc M'$ is
  stable under pushout. 

  Moreover, the diagram commutes as both maps $AX+AZ \to AN+_{AY}AM$ compose to
  that given by the pushout of the images of $f$ and $g$ over $AY$.  Thus the
  diagram represents two $(\mc E', \mc M')$ factorisations of the same morphism,
  and there exists an isomorphism $n$ between the corelations $\square(g) \circ
  \square(f)$ and $\square(g\circ f)$. This proves that $\square$ preserves
  composition.
\end{proof}
This first lemma allows us to verify the associativity and unit laws for
$\corel(\mc C)$.
\begin{corollary}
  $\corel(\mc C)$ is well defined as a category.
\end{corollary}
\begin{proof}
  Consider the case of Proposition \ref{prop.corelfunctors} with $\mc C = \mc
  C'$, $(\mc E,\mc M) = (\mc C, \mc I_{\mc C})$, and $A = 1_{\mc C}$. Then the
  domain of $\square$ is $\cospan(\mc C)$ by definition. In this case, the
  function $\square\maps \cospan(\mc C) \to \corel(\mc C)$ is
  bijective-on-objects and surjective-on-morphisms. Thus to compute the
  composite of any two corelations, we may consider them as cospans, compute
  their composite \emph{as cospans}, and then take the $\mc E$-part of the
  result. Since composition of cospans is associative and unital, so is
  composition of corelations, with the identity corelation just the image of the
  identity cospan.
\end{proof}

Note that the identity in $\corel(\mc C)$ may not be the identity cospan itself.
For example, with the factorisation system $(\mc I_{\mc C}, \mc C)$ the $\mc
I_{\mc C}$-part of the identity cospan is simply $X \xrightarrow{\iota_{X_1}}
X+X \xleftarrow{\iota_{X_2}} X$, where $\iota_{X_i}$ is the inclusion of
$X$ into the $X_i$ factor of the coproduct $X+X$.

This first lemma is also useful in proving the second important lemma: the
naturality of $\overline{\kappa}$.

\begin{lemma} \label{lem.corelfunmonoidal}
  The maps $\overline{\kappa_{X,Y}}$ above are natural.
\end{lemma}
\begin{proof}
  Let $f = (X \longrightarrow N \longleftarrow Y)$, $g= (Z \longrightarrow M
  \longleftarrow W)$ be corelations in $\mc C$. We wish to show that
  \[
    \xymatrixcolsep{4pc}
    \xymatrixrowsep{3pc}
    \xymatrix{
      AX+AY \ar[r]^{\square(f)+\square(g)}
      \ar[d]_{\overline{\kappa_{X,Y}}} & 
      AZ+AW \ar[d]^{\overline{\kappa_{Z,W}}} \\
      A(X+Y) \ar[r]^{\square(f+g)} & A(Z+W)
    }
  \]
  commutes in $\corel(\mc C')$. 

  Consider the following commutative diagram in $\mc C'$, with the outside
  square equivalent to the naturality square for the coherence maps of the
  monoidal functor \linebreak $\cospan(\mc C) \to \cospan(\mc C')$:
  \[ \label{diag.natural}
    \begin{aligned}
      \xymatrixcolsep{4pc}
      \xymatrixrowsep{2.5pc}
      \xymatrix{
	(AX+AY)+(AZ+AW) \ar[r]^(.65){\mc E'+\mc E'}
	\ar[d]_{\kappa_{X,Y}+\kappa_{Z,W}} & 
	\overline{AN}+\overline{AM} \ar[r]^{\mc M'+\mc M'} \ar@{.>}[d]^{p} & 
	AN+AM \ar[d]^{\kappa_{N,M}}\\
	A(X+Y)+A(Z+W) \ar[r]^{\mc E'} & \overline{A(N+M)} \ar[r]^{\mc M'} & A(N+M)
      }
    \end{aligned}
    \tag{$\#$}
  \]
  We have factored the top edge as the coproduct of the respective
  factorisations of $f$ and $g$, and the bottom edge simply as the factorisation
  of the coproduct $f+g$. 

  Note that by Lemma \ref{lem.monfact} the coproduct of two maps in $\mc E'$ is
  again in $\mc E'$, while Lemma \ref{lem.mcoproductsmc} implies the same for
  $\mc M'$. Thus the top edge is an $(\mc E',\mc M')$-factorisation, and the
  uniqueness of factorisations gives the isomorphism $n$. 
  Given that the map reducing cospans to corelations is functorial, the
  commutative square
  \[
    \xymatrixcolsep{2.5pc}
    \xymatrixrowsep{2.5pc}
    \xymatrix{
      (AX+AY)+A(Z+W) \ar[r]^{1+\kappa_{Z,W}^{-1}} \ar@{=}[d] & (AX+AY)+(AZ+AW)
      \ar[r]^(.65){\mc E'+\mc E'} & 
      \overline{AN}+\overline{AM} \ar[d]^{n} \\
      (AX+AY)+A(Z+W) \ar[r]^{\kappa_{X,Y}+1} & A(X+Y)+A(Z+W) \ar[r]^(.6){\mc E'} & 
      \overline{A(N+M)}
    }
  \]
  then implies the naturality of the maps $\overline{\kappa}$.
\end{proof}

These lemmas now imply that $\corel(\mc C)$ is a well-defined hypergraph
category.
\begin{proof}[Proof of Theorem \ref{thm.cospantocorel}]
  To complete the proof then, again consider the case of Proposition
  \ref{prop.corelfunctors} with $\mc C = \mc C'$, $(\mc E,\mc M) = (\mc C, \mc
  I_{\mc C})$, and $A = 1_{\mc C}$. Note that by definition this function maps
  the coherence and hypergraph maps of $\cospan(\mc C)$ onto the corresponding
  maps of $\corel(\mc C)$. As $\cospan(\mc C)$ is a hypergraph, and $\square$
  preserves composition and respects the monoidal and hypergraph structure,
  $\corel(\mc C)$ is also a hypergraph category. 
  
  For instance, suppose we want to check the functoriality of the monoidal
  product $+$. We then wish to show $(g \circ f) + (k \circ h) = (g + k) \circ
  (f + h)$ for corelations of the appropriate types.  But $\square$ preserves
  composition, and the naturality of $\kappa$, here the identity map, implies
  that for any two cospans the $\mc E$-part of their coproduct is equal to the
  coproduct of their $\mc E$-parts. Thus we may compute these two expressions by
  viewing $f$, $g$, $h$, and $k$ as cospans, evaluating them in the category of
  cospans, and then taking their $\mc E$-parts. Since the equality holds in the
  category of cospans, it holds in the category of corelations.
\end{proof}

\begin{corollary}
  There is a strict hypergraph functor 
  \[
    \square\maps \mathrm{Cospan}(\mathcal C) \longrightarrow \mathrm{Corel}(\mathcal C)
  \]
  that takes each object of $\cospan(\mathcal C)$ to itself as an object of
  $\corel(\mathcal C)$ and each cospan to its $\mathcal E$-part.
\end{corollary}

  Finally, we complete the proof that $\square$ is always a hypergraph functor.

\begin{proof}[Proof of Proposition \ref{prop.corelfunctors}] 
  We show $\square$ is a functor, a symmetric monoidal functor, and then finally
  a hypergraph functor.

  \paragraph{Functoriality.} First, recall that $\square$ preserves composition
  (Lemma \ref{lem.corelfuncomposition}). Thus to prove $\square$ is a functor it
  remains to show identities are mapped to identities. The general idea for this
  and for similar axioms is to recall that the special maps are given by reduced
  versions of particular colimits, and that $(\mc E',\mc M')$ reduces maps more
  than $(\mc E,\mc M)$. 

  In this case, recall the identity corelation is given by the $\mc E$-part $X+X
  \to \overline{X}$ of $[1,1]\maps X+X \to X$. Thus the image of the identity on
  $X$ and the identity on $AX$ are given by the top and bottom rows of the
  commuting square
  \[
    \xymatrixcolsep{4pc}
    \xymatrixrowsep{2pc}
    \xymatrix{
      A(X+X) \ar[d]^{\kappa^{-1}}_{\sim} \ar[r]^{\mc E'} &
      \overline{A\overline{X}} \ar[r]^{\mc M'} \ar@{.>}[d]^{n} &
      A\overline{X} \ar[r]^{A\mc M} & AX \ar@{=}[d]\\
      AX+AX \ar[r]^{\mc E'} & \overline{AX} \ar[rr]^{\mc M'} && AX
    }
  \]
  The outside square commutes as we know $A$ maps the identity cospan of $\mc C$
  to the identity cospan of $\mc C'$. The top row is the image under $A$ of the
  identity cospan in $\mc C$, factored first in $\mc C$, and then in $\mc C'$.
  The bottom row is just the factored identity cospan on $AX$ in $\mc C'$. As
  $A$ maps $\mc M$ into $\mc M'$, the map marked $A\mc M$ lies in $\mc M'$. Thus
  both rows are $(\mc E',\mc M')$-factorisations, and so we have the isomorphism
  $n$. Thus $\square$ preserves identities.

  \paragraph{Strong monoidality.} We proved in Lemma \ref{lem.corelfunmonoidal} that
    our proposed coherence maps are natural. The rest of the properties follow
    from the composition preserving map $\cospan(\mc C') \to \corel(\mc C')$.
    Since the $\kappa$ obey all the required axioms as cospans, they obey them
    as corelations too.

  \paragraph{Hypergraph structure.} The proof of preservation of the hypergraph
  structure follows the same pattern as the identity maps. 
\end{proof}

\begin{example}
Note that if both $(\mathcal E, \mathcal M)$ and $(\mathcal E', \mathcal M')$
are epi-split mono factorisations, then we always have that $F(\mathcal M)
\subseteq \mathcal M'$. Indeed, if an (one-sided) inverse exists in the domain
category, it exists in the codomain category. Thus colimit-preserving functors
between categories with finite colimits and epi-split mono factorisation systems
also induce a functor between the epi-split mono corelation categories. We will
use this in Chapter \ref{ch.sigflow}.
\end{example}

\begin{remark} \label{rem.corelposet}
On any category $\mc C$ with finite colimits, reverse inclusions of the right factor
$\mc M$ defines a partial order on the set of factorisation systems $(\mc E,\mc
M)$ with $\mc M$ stable under pushout. That is, we write $(\mc E,\mc M) \ge (\mc
E,\mc M')$ whenever $\mc M \subseteq \mc M'$.  The trivial factorisation
systems $(\mc C,\mc I_{\mc C})$ and $(\mc I_{\mc C},\mc C)$ are the top and
bottom elements of this poset respectively.

Corelation categories realise this poset as a subcategory of the category of
hypergraph categories. One way to understand this is that corelations are
cospans with the $\mc M$-part `factored out'. Using an morphism-isomorphism
factorisation system nothing is factored out, so cospans result. Using the
isomorphism-morphism factorisation system everything is factored out, so there
is a unique corelation between any two objects.

We can construct a decorated corelation functor between two corelation
categories if the codomain factors out more than the domain: i.e. if the
codomain is less than the domain in the poset. In particular, this implies
there is always a functor from the cospan category $\corel_{(\mc C,\mc I_{\mc
C})}(\mc C)= \cospan(\mc C)$ to any other corelation category $\corel_{(\mc
E,\mc M)}(\mc C)$, and from $\corel_{(\mc E,\mc M)}(\mc C)$ any corelation
category to the indiscrete category $\corel_{(\mc I_{\mc C},\mc C)}(\mc C)$ on
the objects of $\mc C$.
\end{remark}

\section{Examples} \label{sec.corelexs}
We conclude this chapter with two examples of corelation categories. These both
play a central role in the applications of Part \ref{part.apps}: the first as
an algebra of ideal wires, and the second as semantics for signal flow graphs.

\subsection{Equivalence relations as corelations in $\Set$}
\label{ssec.equivrels}
   
As we saw in Examples \ref{ex.corelcats}, an epi-mono corelation $X \to Y$ in
$\Set$ is an equivalence relation on $X+Y$. We might depict these as follows
\[
  \begin{tikzpicture}[circuit ee IEC]
	\begin{pgfonlayer}{nodelayer}
		\node [contact, outer sep=5pt] (0) at (-2, 1) {};
		\node [contact, outer sep=5pt] (1) at (-2, 0.5) {};
		\node [contact, outer sep=5pt] (2) at (-2, -0) {};
		\node [contact, outer sep=5pt] (3) at (-2, -0.5) {};
		\node [contact, outer sep=5pt] (4) at (-2, -1) {};
		\node [contact, outer sep=5pt] (5) at (1, 1.25) {};
		\node [contact, outer sep=5pt] (6) at (1, 0.75) {};
		\node [contact, outer sep=5pt] (7) at (1, 0.25) {};
		\node [contact, outer sep=5pt] (8) at (1, -0.25) {};
		\node [contact, outer sep=5pt] (9) at (1, -0.75) {};
		\node [contact, outer sep=5pt] (10) at (1, -1.25) {};
		\node [style=none] (11) at (-2.75, -0) {$X$};
		\node [style=none] (12) at (1.75, -0) {$Y$};
	\end{pgfonlayer}
	\begin{pgfonlayer}{edgelayer}
		\draw [rounded corners=5pt, dashed] 
   (node cs:name=0, anchor=north west) --
   (node cs:name=1, anchor=south west) --
   (node cs:name=6, anchor=south east) --
   (node cs:name=5, anchor=north east) --
   cycle;
		\draw [rounded corners=5pt, dashed] 
   (node cs:name=2, anchor=north west) --
   (node cs:name=3, anchor=south west) --
   (node cs:name=3, anchor=south east) --
   (node cs:name=2, anchor=north east) --
   cycle;
		\draw [rounded corners=5pt, dashed] 
   (node cs:name=4, anchor=north west) --
   (node cs:name=4, anchor=south west) --
   (node cs:name=10, anchor=south east) --
   (node cs:name=9, anchor=north east) --
   cycle;
   		\draw [rounded corners=5pt, dashed] 
   (node cs:name=7, anchor=north west) --
   (node cs:name=7, anchor=south west) --
   (node cs:name=7, anchor=south east) --
   (node cs:name=7, anchor=north east) --
   cycle;
   		\draw [rounded corners=5pt, dashed] 
   (node cs:name=8, anchor=north west) --
   (node cs:name=8, anchor=south west) --
   (node cs:name=8, anchor=south east) --
   (node cs:name=8, anchor=north east) --
   cycle;
	\end{pgfonlayer}
\end{tikzpicture}
\]
Here we have a corelation from a set $X$ of five elements to a set $Y$ of six
elements. Elements belonging to the same equivalence class of $X+Y$ are grouped
(`connected') by a dashed line.

Composition of corelations first takes the transitive closure of the two
partitions (the pushout in $\Set$), before restricting the partition to the new
domain and codomain (restricting to the jointly epic part). For example,
suppose in addition to the corelation $\alpha\maps X \to Y$ above we have
another corelation $\beta\maps Y \to Z$
\[
\begin{tikzpicture}[circuit ee IEC]
	\begin{pgfonlayer}{nodelayer}
		\node [style=none] (0) at (-2.75, -0) {$Y$};
		\node [style=none] (1) at (1.75, 0) {$Z$};
		\node [contact, outer sep=5pt] (2) at (-2, 1.25) {};
		\node [contact, outer sep=5pt] (3) at (-2, 0.75) {};
		\node [contact, outer sep=5pt] (4) at (-2, 0.25) {};
		\node [contact, outer sep=5pt] (5) at (-2, -0.25) {};
		\node [contact, outer sep=5pt] (6) at (-2, -0.75) {};
		\node [contact, outer sep=5pt] (7) at (-2, -1.25) {};
		\node [contact, outer sep=5pt] (8) at (1, 1) {};
		\node [contact, outer sep=5pt] (9) at (1, 0.5) {};
		\node [contact, outer sep=5pt] (10) at (1, -0) {};
		\node [contact, outer sep=5pt] (11) at (1, -0.5) {};
		\node [contact, outer sep=5pt] (12) at (1, -1) {};
	\end{pgfonlayer}
		\draw [rounded corners=5pt, dashed] 
   (node cs:name=2, anchor=north west) --
   (node cs:name=3, anchor=south west) --
   (node cs:name=8, anchor=south east) --
   (node cs:name=8, anchor=north east) --
   cycle;
		\draw [rounded corners=5pt, dashed] 
   (node cs:name=4, anchor=north west) --
   (node cs:name=4, anchor=south west) --
   (node cs:name=4, anchor=south east) --
   (node cs:name=4, anchor=north east) --
   cycle;
		\draw [rounded corners=5pt, dashed] 
   (node cs:name=5, anchor=north west) --
   (node cs:name=6, anchor=south west) --
   (node cs:name=11, anchor=south east) --
   (node cs:name=10, anchor=north east) --
   cycle;
		\draw [rounded corners=5pt, dashed] 
   (node cs:name=7, anchor=north west) --
   (node cs:name=7, anchor=south west) --
   (node cs:name=12, anchor=south east) --
   (node cs:name=12, anchor=north east) --
   cycle;
		\draw [rounded corners=5pt, dashed] 
   (node cs:name=9, anchor=north west) --
   (node cs:name=9, anchor=south west) --
   (node cs:name=9, anchor=south east) --
   (node cs:name=9, anchor=north east) --
   cycle;
\end{tikzpicture}
\]
Then the composite $\beta\circ\alpha$ of our two corelations is given by
\vspace{-1ex}
\[
  \begin{aligned}
\begin{tikzpicture}[circuit ee IEC]
	\begin{pgfonlayer}{nodelayer}
		\node [contact, outer sep=5pt] (-2) at (1, 1.25) {};
		\node [contact, outer sep=5pt] (-1) at (1, 0.75) {};
		\node [contact, outer sep=5pt] (0) at (1, 0.25) {};
		\node [contact, outer sep=5pt] (1) at (1, -0.25) {};
		\node [contact, outer sep=5pt] (2) at (1, -0.75) {};
		\node [contact, outer sep=5pt] (3) at (1, -1.25) {};
		\node [style=none] (4) at (-2.75, -0) {$X$};
		\node [style=none] (5) at (4.75, -0) {$Z$};
		\node [contact, outer sep=5pt] (6) at (-2, 1) {};
		\node [contact, outer sep=5pt] (7) at (-2, -0.5) {};
		\node [contact, outer sep=5pt] (8) at (-2, 0.5) {};
		\node [contact, outer sep=5pt] (9) at (-2, -0) {};
		\node [contact, outer sep=5pt] (10) at (-2, -1) {};
		\node [contact, outer sep=5pt] (11) at (4, -0) {};
		\node [contact, outer sep=5pt] (12) at (4, -1) {};
		\node [contact, outer sep=5pt] (13) at (4, -0.5) {};
		\node [contact, outer sep=5pt] (14) at (4, 0.5) {};
		\node [contact, outer sep=5pt] (19) at (4, 1) {};
		\node [style=none] (20) at (1, -1.75) {$Y$};
		\node [style=none] (21) at (1, 1.75) {\phantom{$Y$}};
	\end{pgfonlayer}
	\begin{pgfonlayer}{edgelayer}
		\draw [rounded corners=5pt, dashed] 
   (node cs:name=6, anchor=north west) --
   (node cs:name=8, anchor=south west) --
   (node cs:name=-1, anchor=south east) --
   (node cs:name=-2, anchor=north east) --
   cycle;
		\draw [rounded corners=5pt, dashed] 
   (node cs:name=9, anchor=north west) --
   (node cs:name=7, anchor=south west) --
   (node cs:name=7, anchor=south east) --
   (node cs:name=9, anchor=north east) --
   cycle;
		\draw [rounded corners=5pt, dashed] 
   (node cs:name=10, anchor=north west) --
   (node cs:name=10, anchor=south west) --
   (node cs:name=3, anchor=south east) --
   (node cs:name=2, anchor=north east) --
   cycle;
		\draw [rounded corners=5pt, dashed] 
   (node cs:name=-2, anchor=north west) --
   (node cs:name=-1, anchor=south west) --
   (node cs:name=19, anchor=south east) --
   (node cs:name=19, anchor=north east) --
   cycle;
		\draw [rounded corners=5pt, dashed] 
   (node cs:name=0, anchor=north west) --
   (node cs:name=0, anchor=south west) --
   (node cs:name=0, anchor=south east) --
   (node cs:name=0, anchor=north east) --
   cycle;
		\draw [rounded corners=5pt, dashed] 
   (node cs:name=1, anchor=north west) --
   (node cs:name=1, anchor=south west) --
   (node cs:name=1, anchor=south east) --
   (node cs:name=1, anchor=north east) --
   cycle;
		\draw [rounded corners=5pt, dashed] 
   (node cs:name=1, anchor=north west) --
   (node cs:name=2, anchor=south west) --
   (node cs:name=13, anchor=south east) --
   (node cs:name=11, anchor=north east) --
   cycle;
		\draw [rounded corners=5pt, dashed] 
   (node cs:name=3, anchor=north west) --
   (node cs:name=3, anchor=south west) --
   (node cs:name=12, anchor=south east) --
   (node cs:name=12, anchor=north east) --
   cycle;
		\draw [rounded corners=5pt, dashed] 
   (node cs:name=14, anchor=north west) --
   (node cs:name=14, anchor=south west) --
   (node cs:name=14, anchor=south east) --
   (node cs:name=14, anchor=north east) --
   cycle;
	\end{pgfonlayer}
\end{tikzpicture}
\end{aligned}
\:
  =
\:
\begin{aligned}
\begin{tikzpicture}[circuit ee IEC]
	\begin{pgfonlayer}{nodelayer}
		\node [style=none] (0) at (-2.75, -0) {$X$};
		\node [style=none] (1) at (1.75, -0) {$Z$};
		\node [contact, outer sep=5pt] (2) at (-2, 1) {};
		\node [contact, outer sep=5pt] (3) at (-2, -0.5) {};
		\node [contact, outer sep=5pt] (4) at (-2, 0.5) {};
		\node [contact, outer sep=5pt] (5) at (-2, -0) {};
		\node [contact, outer sep=5pt] (6) at (-2, -1) {};
		\node [contact, outer sep=5pt] (7) at (1, -0) {};
		\node [contact, outer sep=5pt] (8) at (1, -1) {};
		\node [contact, outer sep=5pt] (9) at (1, -0.5) {};
		\node [contact, outer sep=5pt] (10) at (1, 0.5) {};
		\node [contact, outer sep=5pt] (13) at (1, 1) {};
		\node [style=none] (20) at (1, -1.75) {\phantom{$Y$}};
		\node [style=none] (21) at (1, 1.75) {\phantom{$Y$}};
	\end{pgfonlayer}
	\begin{pgfonlayer}{edgelayer}
		\draw [rounded corners=5pt, dashed] 
   (node cs:name=2, anchor=north west) --
   (node cs:name=4, anchor=south west) --
   (node cs:name=13, anchor=south east) --
   (node cs:name=13, anchor=north east) --
   cycle;
		\draw [rounded corners=5pt, dashed] 
   (node cs:name=5, anchor=north west) --
   (node cs:name=3, anchor=south west) --
   (node cs:name=3, anchor=south east) --
   (node cs:name=5, anchor=north east) --
   cycle;
		\draw [rounded corners=5pt, dashed] 
   (node cs:name=6, anchor=north west) --
   (node cs:name=6, anchor=south west) --
   (node cs:name=8, anchor=south east) --
   (node cs:name=7, anchor=north east) --
   cycle;
		\draw [rounded corners=5pt, dashed] 
   (node cs:name=10, anchor=north west) --
   (node cs:name=10, anchor=south west) --
   (node cs:name=10, anchor=south east) --
   (node cs:name=10, anchor=north east) --
   cycle;
	\end{pgfonlayer}
\end{tikzpicture}
\end{aligned}
\]
Informally, this captures the idea that two elements of $X+Z$ are `connected' if
we may travel from one to the other staying within connected components of
$\alpha$ and $\beta$.
  
For a greater resemblance of the diagrams in the motivating comments of
\textsection\ref{sec.blackboxing}, epi-mono corelations in $\Set$ can also be
visualised as terminals connected by junctions of ideal wires. We draw these by
marking each equivalence class with a point (the `junction'), and then
connecting each element of the domain and codomain to their equivalence class
with a `wire'. The junction serves as the apex of the corelation, the terminals
as the feet, and the wires depict a function from the feet to the apex.
Composition then involves collapsing connected junctions down to a point.
\vspace{-1ex}
\[
  \begin{aligned}
\begin{tikzpicture}[circuit ee IEC]
	\begin{pgfonlayer}{nodelayer}
		\node [contact, outer sep=5pt] (6) at (-2, 1) {};
		\node [contact, outer sep=5pt] (7) at (-2, -0.5) {};
		\node [contact, outer sep=5pt] (8) at (-2, 0.5) {};
		\node [contact, outer sep=5pt] (9) at (-2, -0) {};
		\node [contact, outer sep=5pt] (10) at (-2, -1) {};
		\node [style=none] (15) at (-0.5, 0.875) {};
		\node [style=none] (28) at (-0.5, 0.25) {};
		\node [style=none] (16) at (-0.5, -0.125) {};
		\node [style=none] (29) at (-0.5, -0.375) {};
		\node [style=none] (17) at (-0.5, -1) {};
		\node [contact, outer sep=5pt] (-2) at (1, 1.25) {};
		\node [contact, outer sep=5pt] (-1) at (1, 0.75) {};
		\node [contact, outer sep=5pt] (0) at (1, 0.25) {};
		\node [contact, outer sep=5pt] (1) at (1, -0.25) {};
		\node [contact, outer sep=5pt] (2) at (1, -0.75) {};
		\node [contact, outer sep=5pt] (3) at (1, -1.25) {};
		\node [style=none] (18) at (2.5, -1.125) {};
		\node [style=none] (21) at (2.5, 1) {};
		\node [style=none] (22) at (2.5, -0.375) {};
		\node [style=none] (23) at (2.5, 0.475) {};
		\node [style=none] (24) at (2.5, 0.25) {};
		\node [contact, outer sep=5pt] (19) at (4, 1) {};
		\node [contact, outer sep=5pt] (14) at (4, 0.5) {};
		\node [contact, outer sep=5pt] (11) at (4, -0) {};
		\node [contact, outer sep=5pt] (13) at (4, -0.5) {};
		\node [contact, outer sep=5pt] (12) at (4, -1) {};
		\node [style=none] (4) at (-2.75, -0) {$X$};
		\node [style=none] (5) at (4.75, -0) {$Z$};
		\node [style=none] (20) at (1, -1.75) {$Y$};
		\node [style=none] (30) at (1, 1.75) {\phantom{$Y$}};
	\end{pgfonlayer}
	\begin{pgfonlayer}{edgelayer}
		\draw [thick] (6.center) to (15.center);
		\draw [thick] (8.center) to (15.center);
		\draw [thick] (-2.center) to (15.center);
		\draw [thick] (-1.center) to (15.center);
		\draw [thick] (9.center) to (16.center);
		\draw [thick] (7.center) to (16.center);
		\draw [thick] (10.center) to (17.center);
		\draw [thick] (17.center) to (2.center);
		\draw [thick] (17.center) to (3.center);
		\draw [thick] (3.center) to (18.center);
		\draw [thick] (18.center) to (12.center);
		\draw [thick] (-2.center) to (21.center);
		\draw [thick] (-1.center) to (21.center);
		\draw [thick] (21.center) to (19.center);
		\draw [thick] (1.center) to (22.center);
		\draw [thick] (2.center) to (22.center);
		\draw [thick] (22.center) to (11.center);
		\draw [thick] (22.center) to (13.center);
		\draw [thick] (23.center) to (14.center);
		\draw [thick] (28.center) to (0.center);
		\draw [thick] (0.center) to (24.center);
		\draw [thick] (29.center) to (1.center);
		\draw [rounded corners=5pt, dashed, color=gray] 
   (node cs:name=6, anchor=north west) --
   (node cs:name=8, anchor=south west) --
   (node cs:name=-1, anchor=south east) --
   (node cs:name=-2, anchor=north east) --
   cycle;
		\draw [rounded corners=5pt, dashed, color=gray] 
   (node cs:name=9, anchor=north west) --
   (node cs:name=7, anchor=south west) --
   (node cs:name=7, anchor=south east) --
   (node cs:name=9, anchor=north east) --
   cycle;
		\draw [rounded corners=5pt, dashed, color=gray] 
   (node cs:name=10, anchor=north west) --
   (node cs:name=10, anchor=south west) --
   (node cs:name=3, anchor=south east) --
   (node cs:name=2, anchor=north east) --
   cycle;
		\draw [rounded corners=5pt, dashed, color=gray] 
   (node cs:name=-2, anchor=north west) --
   (node cs:name=-1, anchor=south west) --
   (node cs:name=19, anchor=south east) --
   (node cs:name=19, anchor=north east) --
   cycle;
		\draw [rounded corners=5pt, dashed, color=gray] 
   (node cs:name=0, anchor=north west) --
   (node cs:name=0, anchor=south west) --
   (node cs:name=0, anchor=south east) --
   (node cs:name=0, anchor=north east) --
   cycle;
		\draw [rounded corners=5pt, dashed, color=gray] 
   (node cs:name=1, anchor=north west) --
   (node cs:name=1, anchor=south west) --
   (node cs:name=1, anchor=south east) --
   (node cs:name=1, anchor=north east) --
   cycle;
		\draw [rounded corners=5pt, dashed, color=gray] 
   (node cs:name=1, anchor=north west) --
   (node cs:name=2, anchor=south west) --
   (node cs:name=13, anchor=south east) --
   (node cs:name=11, anchor=north east) --
   cycle;
		\draw [rounded corners=5pt, dashed, color=gray] 
   (node cs:name=3, anchor=north west) --
   (node cs:name=3, anchor=south west) --
   (node cs:name=12, anchor=south east) --
   (node cs:name=12, anchor=north east) --
   cycle;
		\draw [rounded corners=5pt, dashed, color=gray] 
   (node cs:name=14, anchor=north west) --
   (node cs:name=14, anchor=south west) --
   (node cs:name=14, anchor=south east) --
   (node cs:name=14, anchor=north east) --
   cycle;
	\end{pgfonlayer}
\end{tikzpicture}
\end{aligned}
\:
  =
\:
\begin{aligned}
\begin{tikzpicture}[circuit ee IEC]
	\begin{pgfonlayer}{nodelayer}
		\node [style=none] (0) at (-2.75, -0) {$X$};
		\node [style=none] (1) at (1.75, -0) {$Z$};
		\node [contact, outer sep=5pt] (2) at (-2, 1) {};
		\node [contact, outer sep=5pt] (3) at (-2, -0.5) {};
		\node [contact, outer sep=5pt] (4) at (-2, 0.5) {};
		\node [contact, outer sep=5pt] (5) at (-2, -0) {};
		\node [contact, outer sep=5pt] (6) at (-2, -1) {};
		\node [contact, outer sep=5pt] (7) at (1, -0) {};
		\node [contact, outer sep=5pt] (8) at (1, -1) {};
		\node [contact, outer sep=5pt] (9) at (1, -0.5) {};
		\node [contact, outer sep=5pt] (10) at (1, 0.5) {};
		\node [style=none] (11) at (-0.5, 0.875) {};
		\node [style=none] (12) at (-0.5, 0.3) {};
		\node [contact, outer sep=5pt] (13) at (1, 1) {};
		\node [style=none] (14) at (-0.5, -0.2) {};
		\node [style=none] (15) at (-0.5, -0.6) {};
	\end{pgfonlayer}
	\begin{pgfonlayer}{edgelayer}
		\draw [thick] (2.center) to (11.center);
		\draw [thick] (4.center) to (11.center);
		\draw [thick] (11.center) to (13.center);
		\draw [thick] (5.center) to (14.center);
		\draw [thick] (3.center) to (14.center);
		\draw [thick] (15.center) to (7.center);
		\draw [thick] (15.center) to (9.center);
		\draw [thick] (6.center) to (15.center);
		\draw [thick] (15.center) to (8.center);
		\draw [thick] (12.center) to (10.center);
		\draw [rounded corners=5pt, dashed, color=gray] 
   (node cs:name=2, anchor=north west) --
   (node cs:name=4, anchor=south west) --
   (node cs:name=13, anchor=south east) --
   (node cs:name=13, anchor=north east) --
   cycle;
		\draw [rounded corners=5pt, dashed, color=gray] 
   (node cs:name=5, anchor=north west) --
   (node cs:name=3, anchor=south west) --
   (node cs:name=3, anchor=south east) --
   (node cs:name=5, anchor=north east) --
   cycle;
		\draw [rounded corners=5pt, dashed, color=gray] 
   (node cs:name=10, anchor=north west) --
   (node cs:name=10, anchor=south west) --
   (node cs:name=10, anchor=south east) --
   (node cs:name=10, anchor=north east) --
   cycle;
		\draw [rounded corners=5pt, dashed, color=gray] 
   (node cs:name=6, anchor=north west) --
   (node cs:name=6, anchor=south west) --
   (node cs:name=8, anchor=south east) --
   (node cs:name=7, anchor=north east) --
   cycle;
	\end{pgfonlayer}
\end{tikzpicture}
\end{aligned}
\]
The composition law captures the idea that connectivity is all that
matters: as long as the wires are `ideal', the exact path does not matter. 

In Coya--Fong \cite{CF} we formalise this idea by saying that corelations are
the prop for extraspecial commutative Frobenius monoids. An \define{extraspecial
commutative Frobenius monoid} $(X,\mu,\eta,\delta,\epsilon)$ in a monoidal
category $(\mathcal C, \otimes)$ is a special commutative Frobenius monoid that
further obeys the extra law
  \[
    \extral{.1\textwidth} = \extrar{.1\textwidth}
  \]

Two morphisms built from the generators of an extraspecial commutative Frobenius
monoid are equal and if and only if their diagrams impose the same connectivity
relations on the disjoint union of the domain and codomain. This is an extension
of the spider theorem for special commutative Frobenius monoids. 


\subsection{Linear relations as corelations in $\Vect$} \label{ssec.linrel}

Recall that a linear relation $L\maps U \leadsto V$ is a subspace $L \subseteq
U \oplus V$. We compose linear relations as we do relations, and vector spaces
and linear relations form a category $\LinRel$. It is well known that this
category can be constructed as the category of relations in the category $\Vect$
of vector spaces and linear maps with respect to epi-mono factorisations. We
show that they may also be constructed as corelations in $\Vect$ with respect to
epi-mono factorisations.

If we restrict to the full subcategory $\FinVect$ of finite dimensional vector
spaces this is easy to see: after picking a basis for each vector space the
transpose yields an equivalence of $\FinVect$ with its opposite category, so
the category of $(\mathcal E,\mathcal M)$-corelations (jointly epic cospans)
is isomorphic to the category of $(\mathcal E,\mathcal M)$-relations
(jointly monic spans) in $\FinVect$. This fact has been fundamental in work on
finite dimensional linear systems and signal flow diagrams \cite{BE,BSZ,FRS16}.

We prove the general case in detail. To begin, note $\mathrm{Vect}$ has an
epi-mono factorisation system with monos stable under pushouts. This
factorisation system is inherited from $\Set$: the epimorphisms in $\Vect$ are
precisely the surjective linear maps, the monomorphisms are the injective
linear maps, and the image of a linear map is always a subspace of the
codomain, and so itself a vector space. Monos are stable under pushout as the
pushout of a diagram $V \xleftarrow{f} U \xrightarrow{m} W$ is $V \oplus
W/\im[f\; -g]$. The map $m'\maps V \to V \oplus W/\im[f\; -g]$ into the pushout
has kernel $f(\ker m)$. Thus when $m$ is a monomorphism, $m'$ is too.

Thus we have a category of corelations $\corel(\Vect)$. We show that the map
$\corel(\Vect) \to \LinRel$ sending each vector space to itself and each
corelation
\[
  U \stackrel{f}\longrightarrow A \stackrel{g}\longleftarrow V
\]
to the linear subspace $\ker[f\;-g]$ is a full, faithful, and
bijective-on-objects functor.

Indeed, corelations $U \xrightarrow{f} A \xleftarrow{g} V$ are in one-to-one
correspondence with surjective linear maps $U\oplus V \to A$, which are in
turn, by the isomorphism theorem, in one-to-one correspondence with subspaces
of $U\oplus V$. These correspondences are described by the kernel construction
above. Thus our map is evidently full, faithful, and bijective-on-objects. It
also maps identities to identities. It remains to check that it preserves
composition.

Suppose we have corelations $U \xrightarrow{f} A \xleftarrow{g} V$
and $V \xrightarrow{h} B \xleftarrow{k} W$. Then their pushout is given by
$P=A \oplus B/\im[g\;-h]$, and we may draw the pushout diagram
\[
  \xymatrix{
    U \ar[dr]_{f} & & V \ar[dl]^{g}  
    \ar[dr]_{h} & & W \ar[dl]^{k} {} 
    \\
    & A \ar[dr]_{\iota_A} & & B \ar[dl]^{\iota_B}  \\
    & & P {\save*!<0cm,-.5cm>[dl]@^{|-}\restore}
  }
\]
We wish to show the equality of relations
\[
  \ker[f\;-g];\ker[h\;-k] = \ker[\iota_A f\; -\iota_B g].
\]
Now $(\mathbf{u},\mathbf{w}) \in U \oplus W$ lies in the composite relation
$\ker[f\;-g];\ker[h\;-k]$ iff there exists $\mathbf{v} \in V$ such that
$f\mathbf{u} = g\mathbf{v}$ and $h\mathbf{v} = k\mathbf{w}$. But as $P$ is the
pushout, this is true iff 
\[
  \iota_A f \mathbf{u} = \iota_A g \mathbf{v} = \iota_B h \mathbf{v} =
  \iota_B k \mathbf{w}.
\]
This in turn is true iff $(\mathbf{u}, \mathbf{w}) \in \ker[\iota_Af\;
-\iota_Bk]$, as required. 

This corelations perspective is important as it fits the relational picture into
our philosophy of black boxing. In Chapter \ref{ch.sigflow} we will see it is
the corelation construction of $\LinRel$ that correctly generalises to the case
of non-controllable systems.
\smallskip

In the next chapter we discuss decorations on corelations.


%    \tikzset{every path/.style={line width=1.1pt}}
%\begin{tikzpicture}
%	\begin{pgfonlayer}{nodelayer}
%		\node [style=dot] (0) at (-2.5, 1.5) {};
%		\node [style=dot] (1) at (-2.5, -0) {};
%		\node [style=dot] (2) at (-2.5, -1.5) {};
%		\node [style=none] (3) at (-2, -2) {};
%		\node [style=none] (4) at (1.5, 2) {};
%		\node [style=none] (5) at (-2, -0) {};
%		\node [style=none] (6) at (-2, -1.5) {};
%		\node [style=none] (7) at (-2, 1.5) {};
%		\node [style=dot] (8) at (-2.5, -0.75) {};
%		\node [style=none] (9) at (-2, -0.75) {};
%		\node [style=dot] (10) at (-2.5, 0.75) {};
%		\node [style=none] (11) at (-2, 0.75) {};
%		\node [style=dot] (12) at (2, -1.5) {};
%		\node [style=none] (13) at (1.5, -1.5) {};
%		\node [style=dot] (14) at (2, -0.75) {};
%		\node [style=none] (15) at (1.5, 1.5) {};
%		\node [style=dot] (16) at (2, 1.5) {};
%		\node [style=none] (17) at (1.5, 0.75) {};
%		\node [style=dot] (18) at (2, -0) {};
%		\node [style=dot] (19) at (2, 0.75) {};
%		\node [style=none] (20) at (1.5, -0) {};
%		\node [style=none] (21) at (1.5, -0.75) {};
%		\node [style=none] (22) at (-0.75, 1.25) {};
%		\node [style=none] (23) at (0, 0.25) {};
%		\node [style=none] (24) at (-1, -0.25) {};
%		\node [style=none] (25) at (-0.5, -0.75) {};
%		\node [style=none] (26) at (-1.25, -1.25) {};
%		\node [style=none] (27) at (0, 0.75) {};
%		\node [style=none] (28) at (0.5, 1.25) {};
%		\node [style=none] (29) at (-1.25, 0.5) {};
%		\node [style=none] (30) at (-1.25, 1.5) {};
%		\node [style=none] (31) at (0.5, -1.5) {};
%		\node [style=none] (32) at (-1.25, 1) {};
%		\node [style=none] (33) at (-1.25, -0.75) {};
%		\node [style=none] (34) at (-1.75, -0.25) {};
%		\node [style=none] (35) at (0, -1.25) {};
%		\node [style=none] (36) at (0, -0.25) {};
%		\node [style=none] (37) at (1, -1.5) {};
%		\node [style=none] (38) at (-1.75, 1.25) {};
%		\node [style=none] (39) at (0.5, 0.25) {};
%		\node [style=none] (40) at (1, 0.5) {};
%		\node [style=none] (41) at (-0.5, 1.75) {};
%		\node [style=none] (42) at (0.75, 1.75) {};
%		\node [style=none] (43) at (0.25, -1.25) {};
%		\node [style=none] (44) at (1.25, -1) {};
%		\node [style=none] (45) at (-1.75, -1.75) {};
%		\node [style=none] (46) at (-1, -1.75) {};
%		\node [style=none] (47) at (1.25, -1.7) {};
%		\node [style=none] (48) at (-0.5, -1.5) {};
%	\end{pgfonlayer}
%	\begin{pgfonlayer}{edgelayer}
%		\draw (0.center) to (7.center);
%		\draw (1.center) to (5.center);
%		\draw (2.center) to (6.center);
%		\draw (8) to (9.center);
%		\draw (10) to (11.center);
%		\draw (16) to (15.center);
%		\draw (18) to (20.center);
%		\draw (12) to (13.center);
%		\draw (14) to (21.center);
%		\draw (19) to (17.center);
%		\draw (6.center) to (26.center);
%		\draw [in=180, out=-11, looseness=2.00] (26.center) to (31.center);
%		\draw [in=-165, out=15, looseness=1.25] (26.center) to (25.center);
%		\draw [in=1, out=180, looseness=1.00] (24.center) to (5.center);
%		\draw [in=-105, out=0, looseness=1.00] (24.center) to (23.center);
%		\draw [in=108, out=-135, looseness=2.25] (29.center) to (24.center);
%		\draw [in=-169, out=-30, looseness=1.25] (29.center) to (27.center);
%		\draw [in=165, out=60, looseness=1.25] (23.center) to (17.center);
%		\draw (27.center) to (23.center);
%		\draw [bend right, looseness=1.00] (23.center) to (20.center);
%		\draw (28.center) to (27.center);
%		\draw [in=150, out=0, looseness=1.25] (22.center) to (28.center);
%		\draw (28.center) to (15.center);
%		\draw (22.center) to (30.center);
%		\draw (30.center) to (7.center);
%		\draw (11.center) to (32.center);
%		\draw (38.center) to (11.center);
%		\draw (38.center) to (32.center);
%		\draw [in=-78, out=-48, looseness=1.00] (39.center) to (40.center);
%		\draw (31.center) to (37.center);
%		\draw (37.center) to (13.center);
%		\draw (9.center) to (34.center);
%		\draw (34.center) to (33.center);
%		\draw [in=-158, out=22, looseness=1.00] (33.center) to (36.center);
%		\draw (36.center) to (21.center);
%		\draw [in=86, out=15, looseness=2.00] (25.center) to (35.center);
%		\draw (24.center) to (27.center);
%		\draw [in=56, out=-60, looseness=1.00] (22.center) to (29.center);
%		\draw (17.center) to (20.center);
%		\draw [bend left=75, looseness=1.00] (39.center) to (40.center);
%		\draw (41.center) to (42.center);
%		\draw (45.center) to (46.center);
%		\draw (46.center) to (48.center);
%		\draw (46.center) to (47.center);
%		\draw (43.center) to (44.center);
%		\draw (28.center) to (17.center);
%	\end{pgfonlayer}
%	\begin{pgfonlayer}{background}
%	  \filldraw [fill=black!5!white, draw=black!40!white] (3) rectangle (4);
%	\end{pgfonlayer}
%\end{tikzpicture}



%
%\[
%    \tikzset{every path/.style={line width=1.1pt}}
%\begin{tikzpicture}
%	\begin{pgfonlayer}{nodelayer}
%		\node [style=dot] (0) at (-2.5, 1.5) {};
%		\node [style=dot] (1) at (-2.5, -0) {};
%		\node [style=dot] (2) at (-2.5, -1.5) {};
%		\node [style=none] (3) at (-2, -2) {};
%		\node [style=none] (4) at (1.5, 2) {};
%		\node [style=none] (5) at (-2, -0) {};
%		\node [style=none] (6) at (-2, -1.5) {};
%		\node [style=none] (7) at (-2, 1.5) {};
%		\node [style=dot] (8) at (-2.5, -0.75) {};
%		\node [style=none] (9) at (-2, -0.75) {};
%		\node [style=dot] (10) at (-2.5, 0.75) {};
%		\node [style=none] (11) at (-2, 0.75) {};
%		\node [style=dot] (12) at (2, -1.5) {};
%		\node [style=none] (13) at (1.5, -1.5) {};
%		\node [style=dot] (14) at (2, -0.75) {};
%		\node [style=none] (15) at (1.5, 1.5) {};
%		\node [style=dot] (16) at (2, 1.5) {};
%		\node [style=none] (17) at (1.5, 0.75) {};
%		\node [style=dot] (18) at (2, -0) {};
%		\node [style=dot] (19) at (2, 0.75) {};
%		\node [style=none] (20) at (1.5, -0) {};
%		\node [style=none] (21) at (1.5, -0.75) {};
%		\node [style=none] (22) at (-0.25, 0.75) {};
%	\end{pgfonlayer}
%	\begin{pgfonlayer}{edgelayer}
%		\draw (0) to (7);
%		\draw (1) to (5);
%		\draw (2) to (6);
%		\draw (8) to (9);
%		\draw (10) to (11);
%		\draw (16) to (15);
%		\draw (18) to (20);
%		\draw (12) to (13);
%		\draw (14) to (21);
%		\draw (19) to (17);
%		\draw (9) to (21);
%		\draw (6) to (13);
%		\draw (7) to (22);
%		\draw (5) to (22);
%		\draw (22) to (15);
%		\draw (22) to (17);
%		\draw (22) to (20);
%	\end{pgfonlayer}
%	\begin{pgfonlayer}{background}
%	  \filldraw [fill=black!5!white, draw=black!40!white] (3) rectangle (4);
%	\end{pgfonlayer}
%\end{tikzpicture}
%\]


%\[
%    \tikzset{every path/.style={line width=1.1pt}}
%    \begin{tikzpicture}
%	\begin{pgfonlayer}{nodelayer}
%		\node [style=dot] (0) at (-2.5, 1.5) {};
%		\node [style=dot] (1) at (-2.5, -0) {};
%		\node [style=dot] (2) at (-2.5, -1.5) {};
%		\node [style=none] (3) at (-2, -2) {};
%		\node [style=none] (4) at (1.5, 2) {};
%		\node [style=none] (5) at (-2, -0) {};
%		\node [style=none] (6) at (-2, -1.5) {};
%		\node [style=none] (7) at (-2, 1.5) {};
%		\node [style=dot] (8) at (-2.5, -0.75) {};
%		\node [style=none] (9) at (-2, -0.75) {};
%		\node [style=dot] (10) at (-2.5, 0.75) {};
%		\node [style=none] (11) at (-2, 0.75) {};
%		\node [style=dot] (12) at (2, -1.5) {};
%		\node [style=none] (13) at (1.5, -1.5) {};
%		\node [style=dot] (14) at (2, -0.75) {};
%		\node [style=none] (15) at (1.5, 1.5) {};
%		\node [style=dot] (16) at (2, 1.5) {};
%		\node [style=none] (17) at (1.5, 0.75) {};
%		\node [style=dot] (18) at (2, -0) {};
%		\node [style=dot] (19) at (2, 0.75) {};
%		\node [style=none] (20) at (1.5, -0) {};
%		\node [style=none] (21) at (1.5, -0.75) {};
%	\end{pgfonlayer}
%	\begin{pgfonlayer}{edgelayer}
%	  \filldraw [fill=black!80!white] (3) rectangle (4);
%		\draw (0.center) to (7.center);
%		\draw (1.center) to (5.center);
%		\draw (2.center) to (6.center);
%		\draw (8) to (9.center);
%		\draw (10) to (11.center);
%		\draw (16) to (15.center);
%		\draw (18) to (20.center);
%		\draw (12) to (13.center);
%		\draw (14) to (21.center);
%		\draw (19) to (17.center);
%	\end{pgfonlayer}
%\end{tikzpicture}
%  ,
%\]


\part{Applications}

\chapter{Signal flow diagrams}



\chapter{Passive linear networks} \label{ch.circuits}


\section{Introduction}\label{sec:intro}
%%fakesubsection
In late 1940s, just as Feynman was developing his diagrams for processes in particle physics, Eilenberg and Mac Lane initiated their work on category theory.  Over the subsequent decades, and especially in the work of Joyal and Street in the 1980s \cite{JS1,JS2}, it became clear that these developments were profoundly linked: monoidal categories have a precise graphical representation in terms of string diagrams, and conversely monoidal categories provide an algebraic foundation for the intuitions behind Feynman diagrams.  The key insight is the use of categories where morphisms describe physical processes, rather than structure-preserving maps between mathematical objects \cite{BaezStay,CP}.

In work on fundamental physics, the cutting edge has moved from categories
to higher categories \cite{BL}.  But the same techniques have filtered into more immediate applications, particularly in computation and quantum computation \cite{AC,Ba1,Se}.  This paper is part of a still nascent program of applying string diagrams to engineering, with the aim of giving diverse diagram languages a unified foundation based on category theory \cite{BE,BSZ,KSW,RSW,Sp}. 

Indeed, even before physicists began using Feynman diagrams, various branches of engineering were using diagrams that in retrospect are closely related.   Foremost among these are the ubiquitous electrical circuit diagrams. Although less well-known, similar diagrams are used to describe networks consisting of mechanical, hydraulic, thermodynamic and chemical systems.   Further work, pioneered in particular by 
Forrester \cite{Fo} and Odum \cite{Od}, applies similar diagrammatic methods to biology, ecology, and economics.

As discussed in detail by Olsen \cite{Ol}, Paynter \cite{Pa} and others, there are mathematically precise analogies between these different systems.  In each case, the system's state is described by variables that come in pairs, with one variable in each pair playing the role of  `displacement' and the other playing the role of `momentum'.  In engineering, the time derivatives of these variables are sometimes called `flow' and `effort'.    In classical mechanics, this pairing of variables is well understood using
symplectic geometry.  Thus, any mathematical formulation of the diagrams used to
describe networks in engineering needs to take symplectic geometry as well as category
theory into account. 

\vskip 1em
\begin{small}
\begin{center}
\begin{tabular}{|c||c|c|c|c|}
\hline
& displacement  &  flow & momentum & effort \\
& $q$ & $\dot{q}$ & $p$ & $\dot{p}$ \\
\hline\hline
Electronics & charge & current & flux linkage & voltage\\
\hline
Mechanics (translation) & position & velocity & momentum & force\\
\hline
Mechanics (rotation) & angle & angular velocity & angular momentum & torque\\
\hline
Hydraulics & volume & flow & pressure momentum & pressure\\
\hline
Thermodynamics & entropy & entropy flow & temperature momentum & temperature \\
\hline
Chemistry & moles & molar flow & chemical momentum & chemical potential \\
\hline
\end{tabular}
\end{center}
\end{small}

While diagrams of networks have been independently introduced in many disciplines, we do not expect formalizing these diagrams to immediately help the practitioners of these disciplines.  At first the flow of information will mainly go in the other direction: by translating ideas from these disciplines into the language of modern mathematics, we can provide mathematicians with food for thought and interesting new problems to solve.  We hope that in the long run mathematicians can return the favor by bringing new insights to the table.

Although we shall keep the broad applicability of network diagrams in the back of our minds, we couch our discussion in terms of electrical circuits, for the sake of familiarity. In this paper our goal is somewhat limited.  We only study circuits built from `passive' components: that is, those that do not produce energy.  Thus, we exclude batteries and current sources.  We only consider components that respond linearly to an applied voltage.   Thus, we exclude components such as nonlinear resistors or diodes.  Finally, we only consider components with one input and one output, so that a circuit can be described as a graph with edges labeled by components.  Thus, we also exclude transformers.  The most familiar components our framework covers are linear resistors, capacitors and inductors.

While we hope to expand our scope in future work, the class of circuits made from these components has appealing mathematical properties, and is worthy of deep study.  Indeed, this class has been studied intensively for many decades by electrical engineers \cite{AV,Budak,Slepian}.  Even circuits made exclusively of resistors have inspired work by mathematicians of the caliber of Weyl \cite{Weyl} and Smale \cite{Smale}.  

Our work relies on this research.  All we are adding is an emphasis on symplectic geometry and an explicitly `compositional' framework, which clarifies the way a larger circuit can be built from smaller pieces.  This is where monoidal categories become important: the main operations for building circuits from pieces are composition and tensoring.
 
Our strategy is most easily illustrated for circuits made of linear resistors.  Such a resistor dissipates power, turning useful energy into heat at a rate determined by the voltage across the resistor.  However, a remarkable fact is that a circuit made of these resistors always acts to \emph{minimize} the power dissipated this way.  This `principle of minimum power' can be seen as the reason symplectic geometry becomes important in understanding circuits made of resistors, just as the principle of least action leads to the role of symplectic geometry in classical mechanics.  

Here is a circuit made of linear resistors:
\[
\begin{tikzpicture}[circuit ee IEC, set resistor graphic=var resistor IEC graphic]
\node[contact] (I1) at (0,2) {};
\node[contact] (I2) at (0,0) {};
\node[contact] (O1) at (5.83,1) {};
\node(input) at (-2,1) {\small{\textsf{inputs}}};
\node(output) at (7.83,1) {\small{\textsf{outputs}}};
\draw (I1) 	to [resistor] node [label={[label distance=2pt]85:{$3\Omega$}}] {} (2.83,1);
\draw (I2)	to [resistor] node [label={[label distance=2pt]275:{$1\Omega$}}] {} (2.83,1)
				to [resistor] node [label={[label distance=3pt]90:{$4\Omega$}}] {} (O1);
\path[color=gray, very thick, shorten >=10pt, ->, >=stealth, bend left] (input) edge (I1);		\path[color=gray, very thick, shorten >=10pt, ->, >=stealth, bend right] (input) edge (I2);		
\path[color=gray, very thick, shorten >=10pt, ->, >=stealth] (output) edge (O1);
\end{tikzpicture}
\]
The wiggly lines are resistors, and their resistances are written beside them: for example,
$3\Omega$ means 3 ohms, an `ohm' being a unit of resistance.  To formalize this, define a circuit of linear resistors to consist of:
\begin{itemize}
\item a set $N$ of nodes,
\item a set $E$ of edges, 
\item maps $s,t \maps E \to N$ sending each edge to its source and target node,
\item a map $r\maps E \to (0,\infty)$ specifying the resistance of the resistor 
labelling each edge, 
\item maps $i \maps X \to N$, $o \maps Y \to N$ specifying the
inputs and outputs of the circuit.
\end{itemize}

When we run electric current through such a circuit, each node $n \in N$ gets
a `potential' $\phi(n)$.  The `voltage' across an edge $e \in E$ is defined as the 
change in potential as we move from to the source of $e$ to its target, $\phi(t(e)) - 
\phi(s(e))$, and the power dissipated by the resistor on this edge equals
\[      
\frac{1}{r(e)}\big(\phi(t(e))-\phi(s(e))\big)^2. 
\]
The total power dissipated by the circuit is therefore twice
\[   
P(\phi) = \frac{1}{2}\sum_{e \in E} \frac{1}{r(e)}\big(\phi(t(e))-\phi(s(e))\big)^2.
\]
The factor of $\frac{1}{2}$ is convenient in some later calculations.  
Note that $P$ is a nonnegative quadratic form on the vector space $\R^N$.
However, not every nonnegative definite quadratic form on $\R^N$ arises in this way from some circuit of linear resistors with $N$ as its set of nodes.  The quadratic forms that do arise are called `Dirichlet forms'.  They have been extensively investigated \cite{Fukushima,MR,Sabot1997,Sabot2004}, and they play a major role in our work.

We write $\partial N = i(X) \cup o(Y)$ for the set of `terminals': that is,
nodes corresponding to inputs and outputs.    The principle of minimum
power says that if we fix the potential at the terminals, the circuit will choose
the potential at other nodes to minimize the total power dissipated.   
An element $\psi$ of the vector space $\R^{\partial N}$ assigns a potential 
to each terminal.   Thus, if we fix $\psi$, the total power dissipated will be twice
\[
  Q(\psi) = \min_{\substack{ \phi \in \R^N \\ \phi\vert_{\partial N} = \psi}} \; P(\phi)  
\]
The function $Q \maps \R^{\partial N} \to \R$ is again a Dirichlet form.  We call it the `power functional' of the circuit.  

Now, suppose we are unable to see the internal workings of a circuit, and can only observe its `external behavior': that is, the potentials at its terminals and the currents flowing into or out of these terminals.   Remarkably, this behavior is completely determined by the power functional $Q$.  The reason is that the current at any terminal can be obtained by differentiating $Q$ with respect to the potential at this terminal, and relations of this form are \emph{all} the relations that hold between potentials and currents at the terminals.

The Laplace transform allows us to generalize this immediately to circuits that
can also contain linear inductors and capacitors, simply by changing the field we work over, replacing $\R$ by the field $\R(s)$ of rational functions of a single real variable,
and talking of `impedance' where we previously talked of resistance.  We obtain
a category $\Circ$ where an object is a finite set, a morphism $f \maps X \to Y$ is a circuit with input set $X$ and output set $Y$, and composition is given by identifying the outputs of one circuit with the inputs of the next, and taking the resulting union of labelled graphs.  Each such circuit gives rise to a Dirichlet form, now defined over
$\R(s)$, and this Dirichlet form completely describes the externally observable
behavior of the circuit.  

We can take equivalence classes of circuits, where two circuits count as the
same if they have the same Dirichlet form.  We wish for these equivalence classes of circuits to form a category. Although
there is a notion of composition for Dirichlet forms, we find that it lacks
identity morphisms or, equivalently, it lacks morphisms representing ideal wires
of zero impedance. To address this we turn to Lagrangian subspaces of
symplectic vector spaces.  These generalize quadratic forms via the map
\[
  \Big(Q\maps \F^{\partial N} \to \F\Big) \longmapsto \Big(\mathrm{Graph}(dQ) =
  \{(\psi, dQ_\psi) \mid \psi \in \F^{\partial N} \} \subseteq \F^{\partial
  N} \oplus (\F^{\partial N})^\ast\Big)
\]
taking a quadratic form $Q$ on the vector space $\F^{\partial N}$
over the field $\F$ to the graph
of its differential $dQ$. Here we think of the symplectic vector space
$\F^{\partial N} \oplus (\F^{\partial N})^\ast$ as the state space of the
circuit, and the subspace $\mathrm{Graph}(dQ)$ as the subspace of attainable
states, with $\psi \in \F^{\partial N}$ describing the potentials at the
terminals, and $dQ_\psi \in (\F^{\partial N})^\ast$ the currents. 

This construction is well-known in classical mechanics \cite{Weinstein}, where the principle of least action plays a role analogous to that of the principle of minimum power here.   The set of Lagrangian subspaces is actually an algebraic variety,
the `Lagrangian Grassmannian', which serves as a compactification of the
space of quadratic forms.  The Lagrangian Grassmannian has already played a
role in Sabot's work on circuits made of resistors \cite{Sabot1997,Sabot2004}.
For us, its importance it that we can find identity morphisms
for the composition of Dirichlet forms by taking circuits made of parallel resistors
and letting their resistances tend to zero: the limit is not a Dirichlet form, but
it exists in the Lagrangian Grassmannian.    Indeed, 
there exists a category $\LagrRel$ with finite dimensional
symplectic vector spaces as objects and `Lagrangian relations' as morphisms: 
that is, linear relations from $V$ to $W$ that are given by Lagrangian subspaces of $\overline{V} \oplus W$, where $\overline{V}$ is the symplectic vector space conjugate to $V$.   

To move from the Lagrangian subspace defined by the graph of the differential of
the power functional to a morphism in the category $\LagrRel$---that
is, to a Lagrangian relation---we must treat seriously the input and output
functions of the circuit. These express the circuit as built upon a cospan   
\[
  \xymatrix{
    & N \\
    X \ar[ur]^{i} && Y. \ar[ul]_o
  }
\]
Applicable far more broadly than this present formalization of circuits, cospans
model systems with two `ends', an input and output end, albeit without any
connotation of directionality: we might just as well exchange the role of the
inputs and outputs by taking the mirror image of the above diagram. The role of
the input and output functions, as we have discussed, is to mark the terminals
we may glue onto the terminals of another circuit, and the pushout of cospans
gives formal precision to this gluing construction.

One upshot of this cospan framework is that we may consider circuits with elements
of $N$ that are both inputs and outputs, such as this one:
\[
  \begin{tikzpicture}[circuit ee IEC, set resistor graphic=var resistor iec graphic]
    \node[contact] (c1) at (0,2) {};
    \node[contact] (c2) at (0,0) {};
    \node(input) at (-2,1) {\small{\textsf{inputs}}};
    \node(output) at (2,1) {\small{\textsf{outputs}}};
    \path[color=gray, very thick, shorten >=10pt, ->, >=stealth, bend left]
    (input) edge (c1);		
    \path[color=gray, very thick, shorten >=10pt, ->, >=stealth, bend right]
    (input) edge (c2);	
    \path[color=gray, very thick, shorten >=10pt, ->, >=stealth, bend right]
    (output) edge (c1);
    \path[color=gray, very thick, shorten >=10pt, ->, >=stealth, bend left]
    (output) edge (c2);
  \end{tikzpicture}
\]
This corresponds to the identity morphism on the finite set with two elements.
Another is that some points may be considered an input or output multiple
times; we draw this:
\[
  \begin{tikzpicture}[circuit ee IEC, set resistor graphic=var resistor IEC graphic]
    \node[contact] (I1) at (0,0) {};
    \node[contact] (O1) at (3,0) {};
    \node(input) at (-2,0) {\small{\textsf{inputs}}};
    \node(output) at (5,0) {\small{\textsf{outputs}}};
    \draw (I1) 	to [resistor] node [label={[label distance=3pt]90:{$1\Omega$}}]
    {} (O1);
    \path[color=gray, very thick, shorten >=10pt, ->, >=stealth, bend left] (input)
    edge (I1);		
    \path[color=gray, very thick, shorten >=10pt, ->,
    >=stealth, bend right] (input) edge (I1);		
    \path[color=gray, very thick, shorten >=10pt, ->, >=stealth] (output) edge (O1);
  \end{tikzpicture}
\]
This allows us to connect two distinct outputs to the above double
input.

Given a set $X$ of inputs or outputs, we understand the electrical behavior on this set 
by considering the symplectic vector space $\vectf{X}$, the direct sum of the space
$\F^X$ of potentials and the space ${(\F^X)}^\ast$ of currents at these points.
A Lagrangian relation specifies which states of the output space $\vectf{Y}$ are
allowed for each state of the input space $\vectf{X}$.
Turning the Lagrangian subspace $\mathrm{Graph}(dQ)$ of a circuit into this
information requires that we understand the `symplectification' 
\[  Sf\maps \vectf{B} \to \vectf{A} \] 
and `twisted symplectification'
\[  S^tf\maps \vectf{B} \to \overline{\vectf{A}}\]
of a function $f\maps A \to B$ between finite sets.  In particular we need to understand how these apply to the input and output functions with codomain restricted to $\partial N$; abusing notation, we also write these $i\maps X \to \partial N$ and $o\maps Y \to \partial N$.

The symplectification is a Lagrangian relation, and the catch
phrase is that it `copies voltages' and `splits currents'.  More precisely,
for any given potential-current pair $(\psi,\iota)$ in $\vectf{B}$, its image
under $Sf$ comprises all elements of $(\psi', \iota') \in \vectf{A}$ such that
the potential at $a \in A$ is equal to the potential at $f(a) \in B$, and such
that, for each fixed $b \in B$, collectively the currents at the $a \in
f^{-1}(b)$ sum to the current at $b$.  We use the symplectification $So$ of the
output function to relate the state on $\partial N$ to that on the
outputs $Y$. As our current framework is set up to report the current \emph{out}
of each node, to describe input currents we define the twisted symplectification
$S^tf\maps \vectf{B} \to \overline{\vectf{A}}$ almost identically to the above, except that we flip the sign of the currents $\iota' \in (\F^A)^\ast$.  We use the twisted symplectification $S^ti$ of the input function to relate the state on $\partial N$
to that on the inputs.

The Lagrangian relation corresponding to a circuit is then the set of all
potential--current pairs that are possible at the inputs and outputs of that circuit. 
For instance, consider a resistor of resistance $r$, with one end considered as an
input and the other as an output:
\[
  \begin{tikzpicture}[circuit ee IEC, set resistor graphic=var resistor IEC graphic]
    \node[contact] (I1) at (0,0) {};
    \node[contact] (O1) at (3,0) {};
    \node(input) at (-2,0) {\small{\textsf{input}}};
    \node(output) at (5,0) {\small{\textsf{output}}};
    \draw (I1) 	to [resistor] node [label={[label distance=3pt]90:{$r$}}] {} (O1);
    \path[color=gray, very thick, shorten >=10pt, ->, >=stealth] (input)
    edge (I1);
    \path[color=gray, very thick, shorten >=10pt, ->, >=stealth] (output) edge (O1);
  \end{tikzpicture}
\]
To obtain the corresponding Lagrangian relation, we must first specify domain and
codomain symplectic vector spaces. In this case, as the input and output sets
each consist of a single point, these vector spaces are both $\F \oplus \F^\ast$,
where the first summand is understood as the space of potentials, and the second
the space of currents.

Now, the resistor has power functional $Q\maps \F^2 \to \F$ given by
\[   Q(\psi_1,\psi_2) = \frac1{2r}(\psi_2-\psi_1)^2, \]
and the graph of the differential of $Q$ is
\[
  \mathrm{Graph}(dQ) = \big\{\big(\psi_1,\psi_2,
  \tfrac1r(\psi_1-\psi_2),\tfrac1r(\psi_2-\psi_1)\big) \,\big|\, \psi_1,\psi_2 \in
  \F\big\} \subseteq \F^2 \oplus (\F^2)^\ast.
\]
In this example the input and output functions $i,o$ are simply the identity
functions on a one element set, so the symplectification of the output function
is simply the identity linear transformation, and the twisted symplectification
of the input function is the isomorphism  between conjugate
symplectic vector spaces $\F\oplus\F^\ast \to \overline{\F\oplus\F^\ast}$ mapping $(\phi,i)$ to $(\phi,-i)$ This implies that the behavior associated to this
circuit is the Lagrangian relation
\[
  \big\{(\psi_1,i,\psi_2,i) \,\big|\, \psi_1,\psi_2 \in \F, i =
  \tfrac1r(\psi_2-\psi_1)\big\}\subseteq \overline{\F \oplus \F^\ast} \oplus \F
    \oplus \F^\ast.
\]
This is precisely the set of potential-current pairs that are allowed at the
input and output of a resistor of resistance $r$.  In particular, the relation
$i = \tfrac1r(\psi_2-\psi_1)$ is well-known in electrical engineering: it is
`Ohm's law'.

A crucial fact is that the process of mapping a circuit to its corresponding
Lagrangian relation identifies distinct circuits.  For example, a single 2-ohm resistor:
\[
  \begin{tikzpicture}[circuit ee IEC, set resistor graphic=var resistor IEC graphic]
    \node[contact] (I1) at (0,0) {};
    \node[contact] (O1) at (3,0) {};
    \node(input) at (-2,0) {\small{\textsf{input}}};
    \node(output) at (5,0) {\small{\textsf{output}}};
    \draw (I1) 	to [resistor] node [label={[label distance=3pt]90:{$2\Omega$}}] {} (O1);
    \path[color=gray, very thick, shorten >=10pt, ->, >=stealth] (input)
    edge (I1);
    \path[color=gray, very thick, shorten >=10pt, ->, >=stealth] (output) edge (O1);
  \end{tikzpicture}
\]
has the same Lagrangian relation as two 1-ohm resistors in series:
\[
  \begin{tikzpicture}[circuit ee IEC, set resistor graphic=var resistor IEC graphic]
    \node[contact] (I1) at (0,0) {};
    \node[circle, minimum width = 3pt, inner sep = 0pt, fill=black] (int) at
    (3,0) {};
    \node[contact] (O1) at (6,0) {};
    \node(input) at (-2,0) {\small{\textsf{input}}};
    \node(output) at (8,0) {\small{\textsf{output}}};
    \draw (I1) 	to [resistor] node [label={[label distance=3pt]90:{$1\Omega$}}] {} (int)
    to [resistor] node [label={[label distance=3pt]90:{$1\Omega$}}] {} (O1);
    \path[color=gray, very thick, shorten >=10pt, ->, >=stealth] (input)
    edge (I1);
    \path[color=gray, very thick, shorten >=10pt, ->, >=stealth] (output) edge (O1);
  \end{tikzpicture}
\]
The Lagrangian relation does not shed any light on the internal workings of a
circuit.  Thus, we call the process of computing this relation `black boxing':
it is like encasing the circuit in an opaque box, leaving only its terminals
accessible. Fortunately, the Lagrangian relation of a circuit is enough to
completely characterize its external behavior, including how it interacts when
connected with other circuits. 

Put more precisely, the black boxing process is \emph{functorial}: we can 
compute the black boxed version of a circuit made of parts by computing the
black boxed versions of the parts and then composing them.   In fact we shall 
prove that $\Circ$ and $\LagrRel$ are dagger compact categories, and
the black box functor preserves all this extra structure:

\begin{theorem} \label{main_theorem}
  There exists a symmetric monoidal dagger functor, the \define{black box functor}   
  \[ \blacksquare\maps \Circ \to \LagrRel, \]
   mapping a finite set $X$ to the symplectic vector space
  $\vectf{X}$ it generates, and a circuit $\big((N,E,s,t,r),i,o\big)$ to the Lagrangian     
  relation 
  \[
    \bigcup_{v \in \mathrm{Graph}(dQ)} S^ti(v) \times So(v)
    \subseteq \overline{\F^X \oplus (\F^X)^\ast} \oplus \F^Y \oplus (\F^Y)^\ast,
  \]
  where $Q$ is the circuit's power functional.
\end{theorem}

The goal of this paper is to prove and explain this result. The proof itself is
more tricky than one might first expect, but our approach introduces various
concepts that should be useful throughout the study of networks, such as
`decorated cospans' and `corelations'.  These provide a general framework for
discussing open networked systems---and not only the passive linear systems
discussed here, but also others, such as Markov processes \cite{BFP}.

Cospans are already familiar as a formalism for making entities with an
arbitrarily designated `input end' and `output end' into the morphisms of a
category.  For example, in topological quantum field theory we use special
cospans called `cobordisms' to describe pieces of spacetime \cite{BL,BaezStay}.
Earlier we introduced `decorated cospans' to describe
circuits with specified inputs and outputs.  Later, with the help of machinery
developed in a companion paper \cite{Fon}, we use decorated cospans to set up
several functors that appear as factors of the black box functor, as steps in 
proving its functoriality.

Just as a relation is an isomorphism class of spans of a special sort, namely
jointly monic spans, a `corelation' is an isomorphism class of jointly epic
cospans.  While corelations are less widely used than relations, they turn out
to be perfectly suited for describing circuits of ideal perfectly conductive
wires---precisely the class of circuits that cannot be modeled simply as
Dirichlet forms, since the power functional becomes infinite when it includes
terms where the resistance is zero.  We introduce corelations in Section
\ref{sec:corel}, and they too play a key role in constructing the black box
functor.

With these tools in hand, the black box functor turns out to rely on a tight
relationship between Kirchhoff's laws, the minimization of Dirichlet forms, and
the `symplectification' of corelations. It is well-known that away from the terminals,
a circuit must obey two rules known as Kirchhoff's laws.  We have already noted that 
the principle of minimum power states that a circuit will `choose' potentials on its interior
that minimize the power functional.  We clarify the relation between these points
in Theorems \ref{thm:realizablepotentials} and \ref{thm:dirichletminimization}, 
which together show that minimizing a Dirichlet form over some subset amounts to
assuming that the corresponding circuit obeys Kirchhoff's laws on that subset.

We have also mentioned the symplectification of functions above.  Extending this to
allow symplectification of corelations, this process gives a map sending corelations
to Lagrangian relations that describe the behavior of ideal perfectly conductive wires. 
We prove that these symplectified corelations simultaneously impose Kirchhoff's 
laws (Proposition \ref{prop:sympfunctor} and Example \ref{ex:sympfunction}) and 
accomplish the minimization of Dirichlet forms (Theorem \ref{thm:sympmin}).  

Together, our results show that these three concepts---Kirchhoff's laws from circuit theory, 
the analytic idea of minimizing power dissipation, and the algebraic idea of symplectification of corelations---are merely different faces of one law: the law of composition of circuits.

\subsection{Finding your way through this paper}
This paper is split into three parts, addressing in turn the questions:
\begin{enumerate}[I.]
  \item What do circuit diagrams mean?
  \item How do we interact with circuit diagrams?
  \item How is meaning preserved under these interactions?
\end{enumerate}

We begin Part , on the semantics of circuit diagrams, with
a discussion of circuits of linear resistors, developing the intuition for the
governing laws of passive linear circuits---Ohm's law, Kirchhoff's voltage law,
and Kirchhoff's current law---in a time-independent setting (Section
\ref{sec:resistors}). This allows us to develop the concept of
Dirichlet form as a representation of power consumption, and understand their
composition as minimizing power, an expression of the current law. In Section
\ref{sec:plcs}, the Laplace transform then allows us to recapitulate these ideas
after introducing inductors and capacitors, speaking of impedance where we
formerly spoke of resistance, and generalizing Dirichlet forms from the field
$\R$ to the field $\R(s)$ of real rational functions. While in this setting the
principle of minimum power is replaced by a variational principle,
the intuitions gained from circuits of resistors still remain useful. 

At the end of this part, we show that Dirichlet forms alone do not provide the 
flexibility to construct a category representing the semantics of circuit diagrams. 
The fundamental reason is that they do not allow us to describe circuits involving 
ideal perfectly conductive wires.  This motivates the development of more powerful machinery.

Part , on the syntax of circuit diagrams, contains the main
technical contributions of the paper. It begins with Section,
which develops machinery to construct what we call categories of decorated
cospans. These are categories where the objects are finite sets and the morphisms
are cospans in the category of finite sets together with some extra structure on
the apex. Circuits, as defined above, are naturally an example of such a
construction, and Section \ref{sec:circdef} lays out the details of this.
In Section \ref{sec:circlagr} we then review the basic theory of linear
Lagrangian relations, giving details to the correspondence we have defined
between Dirichlet forms, and hence passive linear circuits, and Lagrangian
relations. Section \ref{sec:corel} then takes immediate advantage of the added
flexibility of Lagrangian relations, discussing the `trivial' circuits
comprising only perfectly conductive wires, which mediate the notion of composition
of circuits.

Having introduced these prerequisites, we get to the point in Part .
In Section \ref{sec:blackbox} we introduce the black box functor,  and  in Section \ref{sec:proof} we prove our main result.

\section{Circuits of linear resistors} \label{sec:resistors}
%%fakesubsection
In this part we review the meaning of circuit diagrams comprising resistors,
inductors, and capacitors, giving an answer to the question ``What do circuit
diagrams mean?''. 

To elaborate, while circuit diagrams model electric circuits according to their
physical form, another, often more relevant, way to understand a circuit is by
its external behavior. This means the following. To an electric circuit we
associate two quantities to each edge: voltage and current. We are not free,
however, to choose these quantities as we like; circuits are subject to
governing laws that imply voltages and currents must obey certain relationships.
From the perspective of control theory we are particularly interested in the
values these quantities take at the so-called terminals, and how altering one
value will affect the other values. We call two circuits equivalent when they
determine the same relationship. Our main task in this first part is to explore
when two circuits are equivalent.

In order to let physical intuition lead the way, we begin by specialising to the
case of linear resistors. In this section we describe how to find the function
of a circuit from its form, advocating in particular the perspective of the
principle of minimum power. This allows us to identify the external behavior of a
circuit with a so-called Dirichlet form representing the dependence of its power
consumption on potentials at its terminals.

\subsection{Circuits as labelled graphs}

The concept of an abstract open electrical circuit made of linear resistors is
well-known in electrical engineering, but we shall need to formalize it with
more precision than usual.  The basic idea is that a circuit of linear resistors
is a graph whose edges are labelled by positive real numbers called
`resistances', and whose sets of vertices is equipped with two subsets: the
`inputs' and `outputs'. This unfolds as follows.

A (closed) circuit of resistors looks like this: 
\[
\begin{tikzpicture}[circuit ee IEC, set resistor graphic=var resistor IEC graphic]
\node (I1) at (0,0) {};
\node (I2) at (0,2) {};
\node (O1) at (5.83,1) {};
\draw (I1) 	to [resistor] node [label={[label distance=2pt]275:{$1\Omega$}}] {} (2.83,1);
\draw (I2)	to [resistor] node [label={[label distance=2pt]85:{$1\Omega$}}] {} (2.83,1)
				to [resistor] node [label={[label distance=3pt]90:{$2\Omega$}}] {} (O1);
\end{tikzpicture}
\]
We can consider this a labelled graph, with each resistor an edge of the graph,
its resistance its label, and the vertices of the graph the points at which
resistors are connected. 

A circuit is `open' if it can be connected to other circuits. To do this we
first mark points at which connections can be made by denoting some vertices as
input and output terminals:
\[
\begin{tikzpicture}[circuit ee IEC, set resistor graphic=var resistor IEC graphic]
\node[contact] (I1) at (0,2) {};
\node[contact] (I2) at (0,0) {};
\node[contact] (O1) at (5.83,1) {};
\node(input) at (-2,1) {\small{\textsf{inputs}}};
\node(output) at (7.83,1) {\small{\textsf{outputs}}};
\draw (I1) 	to [resistor] node [label={[label distance=2pt]85:{$1\Omega$}}] {} (2.83,1);
\draw (I2)	to [resistor] node [label={[label distance=2pt]275:{$1\Omega$}}] {} (2.83,1)
				to [resistor] node [label={[label distance=3pt]90:{$2\Omega$}}] {} (O1);
\path[color=gray, very thick, shorten >=10pt, ->, >=stealth, bend left] (input) edge (I1);		\path[color=gray, very thick, shorten >=10pt, ->, >=stealth, bend right] (input) edge (I2);		
\path[color=gray, very thick, shorten >=10pt, ->, >=stealth] (output) edge (O1);
\end{tikzpicture}
\]
Then, given a second circuit, we may choose a relation between the output set of
the first and the input set of this second circuit, such as the simple relation
of the single output vertex of the circuit above with the single input vertex of
the circuit below.
\[
\begin{tikzpicture}[circuit ee IEC, set resistor graphic=var resistor IEC graphic]
\node[contact] (I1) at (0,1) {};
\node[contact] (O1) at (2.83,2) {};
\node[contact] (O2) at (2.83,0) {};
\node (input) at (-2,1) {\small{\textsf{inputs}}};
\node (output) at (4.83,1) {\small{\textsf{outputs}}};
\draw (I1) 	to [resistor] node [label={[label distance=2pt]95:{$1\Omega$}}] {} (O1);
\draw (I1)		to [resistor] node [label={[label distance=2pt]265:{$3\Omega$}}] {} (O2);
\path[color=gray, very thick, shorten >=10pt, ->, >=stealth] (input) edge (I1);		\path[color=gray, very thick, shorten >=10pt, ->, >=stealth, bend right] (output) edge (O1);
\path[color=gray, very thick, shorten >=10pt, ->, >=stealth, bend left] (output) edge (O2);
\end{tikzpicture}
\]
We connect the two circuits by identifying output and input vertices according
to this relation, giving in this case the composite circuit:
\[
\begin{tikzpicture}[circuit ee IEC, set resistor graphic=var resistor IEC graphic]
\node[contact] (I1) at (0,2) {};
\node[contact] (I2) at (0,0) {};
\coordinate (int1) at (2.83,1) {};
\coordinate (int2) at (5.83,1) {};
\node[contact] (O1) at (8.66,2) {};
\node[contact] (O2) at (8.66,0) {};
\node (input) at (-2,1) {\small{\textsf{inputs}}};
\node (output) at (10.66,1) {\small{\textsf{outputs}}};
\draw (I1) 	to [resistor] node [label={[label distance=2pt]85:{$1\Omega$}}] {} (int1);
\draw (I2)	to [resistor] node [label={[label distance=2pt]275:{$1\Omega$}}] {} (int1)
				to [resistor] node [label={[label distance=3pt]90:{$2\Omega$}}] {} (int2);
\draw (int2) 	to [resistor] node [label={[label distance=2pt]95:{$1\Omega$}}] {} (O1);
\draw (int2)		to [resistor] node [label={[label distance=2pt]265:{$3\Omega$}}] {} (O2);
\path[color=gray, very thick, shorten >=10pt, ->, >=stealth, bend left] (input) edge (I1);		\path[color=gray, very thick, shorten >=10pt, ->, >=stealth, bend right] (input) edge (I2);		
\path[color=gray, very thick, shorten >=10pt, ->, >=stealth, bend right] (output) edge (O1);
\path[color=gray, very thick, shorten >=10pt, ->, >=stealth, bend left] (output) edge (O2);
\end{tikzpicture}
\]

\vskip 1em

More formally, we define a \define{graph}\footnote{In this paper we refer
to directed multigraphs simply as graphs.} to be a pair of functions $s,t\maps E \to N$ where $E$ and $N$ are finite sets.  We call elements of $E$ \define{edges} and elements of $N$ \define{vertices} or
\define{nodes}.  We say that the edge $e \in E$ has \define{source} $s(e)$ and
\define{target} $t(e)$, and also say that $e$ is an edge \define{from} $s(e)$
\define{to} $t(e)$.

To study circuits we need graphs with labelled edges:

\begin{definition}
Given a set $L$ of \define{labels}, an \define{$L$-graph} is a graph equipped with a function $r\maps E \to L$:
\[
\xymatrix{
L & E \ar@<2.5pt>[r]^{s} \ar@<-2.5pt>[r]_{t} \ar[l]_{r} & N.
}
\]
\end{definition}

For circuits made of resistors we take $L = (0,\infty)$, but later we shall
take $L$ to be a set of positive elements in some more general field.  In either
case, a circuit will be an $L$-graph with some extra structure:

\begin{definition} \label{def_circuit}
Given a set $L$, a \define{circuit over $L$} is an $L$-graph $\xymatrix{
L & E \ar@<2.5pt>[r]^{s} \ar@<-2.5pt>[r]_{t} \ar[l]_{r} & N}$ together with finite sets $X$, $Y$, and functions $i \maps X \to N$ and $o\maps Y \to  N$. We call the sets $i(X)$, $o(Y)$, and $\partial N = i(X) \cup o(Y)$ the \define{inputs},  \define{outputs}, and \define{terminals} or \define{boundary} of the circuit, respectively.
\end{definition}

We will later make use of the notion of connectedness in graphs. Recall that
given two vertices $v, w \in N$ of a graph, a \define{path from $v$ to $w$} is a
finite sequence of vertices $v = v_0, v_1, \dots , v_n = w$ and edges $e_1,
\dots , e_n$ such that for each $1 \le i \le n$, either $e_i$ is an edge from
$v_i$ to $v_{i+1}$, or an edge from $v_{i+1}$ to $v_i$. A subset $S$ of the
vertices of a graph is \define{connected} if, for each pair of vertices in $S$,
there is a path from one to the other. A \define{connected component} of a graph
is a maximal connected subset of its vertices.\footnote{In the theory of
directed graphs the qualifier `weakly' is commonly used before the word
`connected' in these two definitions, in distinction from a stronger notion of
connectedness requiring paths to respect edge directions. As we never consider
any other sort of connectedness, we omit this qualifier.}

In the rest of this section we take $L = (0,\infty) \subseteq \R$ and fix a circuit over 
$(0,\infty)$.  The edges of this circuit should be thought of as `wires'.  The label 
$r_e \in (0,\infty)$ stands for the \define{resistance} of the resistor on the wire $e$.   
There will also be a voltage and current on each wire.  In this section, these will
be specified by functions $V \in \R^E$ and $I \in \R^E$.  Here, as customary in
engineering, we use $I$ for `intensity of current', following Amp\`ere.  

\subsection{Ohm's law, Kirchhoff's laws, and the principle of minimum power}

In 1827, Georg Ohm published a book which included a relation between the voltage
and current for circuits made of resistors \cite{O}.  At the time, the critical
reception was harsh: one contemporary called Ohm's work ``a web of naked
fancies, which can never find the semblance of support from even the most
superficial of observations'', and the German Minister of Education said that a
professor who preached such heresies was unworthy to teach science \cite{D,H}.
However, a simplified version of his relation is now widely used under the name
of `Ohm's law'. We say that \define{Ohm's law} holds if for all edges $e \in
E$ the voltage and current functions of a circuit obey:
\[ 
V(e) = r(e) I(e).  \label{ohm}  
\]

Kirchhoff's laws date to Gustav Kirchhoff in 1845, generalising Ohm's work. They
were in turn generalized into Maxwell's equations a few decades later. We say
\define{Kirchhoff's voltage law} holds if there exists $\phi \in \R^N$ such that
\[
V(e) = \phi(t(e)) - \phi(s(e)).
\]
We call the function $\phi$ a \define{potential}, and think of it as assigning
an electrical potential to each node in the circuit. The voltage then arises as
the differences in potentials between adjacent nodes. If Kirchhoff's voltage law
holds for some voltage $V$, the potential $\phi$ is unique only in the trivial
case of the empty circuit: when the set of nodes $N$ is empty. Indeed, two
potentials define the same voltage function if and only if their difference is
constant on each connected component of the graph $\Gamma$.

We say \define{Kirchhoff's current law} holds if for all nonterminal nodes $n
\in N\setminus \partial N$ we have
\[ 
\sum_{s(e) = n} I(e) = \sum_{t(e) = n} I(e).  \label{kcl}  
\]  
This is an expression of conservation of charge within the circuit; it says that
the total current flowing in or out of any nonterminal node is zero. Even when
Kirchhoff's current law is obeyed, terminals need not be sites of zero net
current; we call the function $\iota \in \R^{\partial N}$ that takes a terminal
to the difference between the outward and inward flowing currents,
\begin{align*}
\iota:\partial N &\longrightarrow \R \\
n &\longmapsto \sum_{t(e) = n} I(e) -\sum_{s(e) = n} I(e),
\end{align*}
the \define{boundary current} for $I$.

A \define{boundary potential} is also a function in $\R^{\partial N}$, but
instead thought of as specifying potentials on the terminals of a
circuit. As we think of our circuits as open circuits, with the terminals points
of interaction with the external world, we shall think of these potentials as
variables that are free for us to choose. Using the above three
principles---Ohm's law, Kirchhoff's voltage law, and Kirchhoff's current
law---it is possible to show that choosing a boundary potential determines
unique voltage and current functions on that circuit. 

The so-called `principle of minimum power' gives some insight into how this
occurs, by describing a way potentials on the terminals might determine
potentials at all nodes. From this, Kirchhoff's voltage law then gives rise to a
voltage function on the edges, and Ohm's law gives us a current function too. We
shall show, in fact, that a potential satisfies the principle of minimum power
for a given boundary potential if and only if this current obeys Kirchhoff's
current law.

A circuit with current $I$ and voltage $V$ dissipates energy at a rate
equal to
\[
 \sum_{e \in E} I(e)V(e).
\]  
Ohm's law allows us to rewrite $I$ as $V/r$, while Kirchhoff's voltage law gives
us a potential $\phi$ such that $V(e)$ can be written as
$\phi(t(e))-\phi(s(e))$, so for a circuit obeying these two laws the power can
also be expressed in terms of this potential. We thus arrive at a functional
mapping potentials $\phi$ to the power dissipated by the circuit when Ohm's law
and Kirchhoff's voltage law are obeyed for $\phi$. 

\begin{definition}
The \define{extended power functional} $P\maps \R^N \to \R$ of a circuit is
defined by
\[
P(\phi) =\frac{1}{2} \sum_{e \in E} \frac{1}{r(e)}\big(\phi(t(e))-\phi(s(e))\big)^2.
\]
\end{definition}

\noindent
The factor of $\frac{1}{2}$ is inserted to cancel the factor of 2 that appears when
we differentiate this expression.  We call $P$ the \emph{extended} power functional as we shall see that it is defined even on potentials that are not compatible with the three governing laws of electric circuits. We shall later restrict the domain of this functional so that it is defined precisely on those potentials that \emph{are} compatible with the
governing laws. Note that this functional does not depend on the directions
chosen for the edges of the circuit.

This expression lets us formulate the `principle of minimum power', which gives
us information about the potential $\phi$ given its restriction to the boundary
of $\Gamma$. Call a potential $\phi \in \R^N$ an \define{extension} of a
boundary potential $\psi \in \R^{\partial N}$ if $\phi$ is equal to $\psi$ when
restricted to $\R^{\partial N}$---that is, if $\phi|_{\partial N} = \psi$. 

\begin{definition}
We say a potential $\phi \in \R^{N}$ \define{obeys the principle of minimum
power} for a boundary potential $\psi \in \R^{\partial N}$ if $\phi$ minimizes
the extended power functional $P$ subject to the constraint that  $\phi$ is an
extension of $\psi$. 
\end{definition}

As promised, in the presence of Ohm's law and Kirchhoff's voltage law, the
principle of minimum power is equivalent to Kirchhoff's current law.

\begin{proposition} \label{minimum_power_implies_kirchhoff_current}
Let $\phi$ be a potential extending some boundary potential $\psi$. Then $\phi$
obeys the principle of minimum power for $\psi$ if and only if the 
current 
\[  I(e) = \frac1{r(e)}(\phi(t(e))-\phi(s(e))) \] 
obeys Kirchhoff's current law.
\end{proposition}

\begin{proof}
Fixing the potentials at the terminals to be those given by the boundary
potential $\psi$, the power is a nonnegative quadratic function of the
potentials at the nonterminals. This implies that an extension $\phi$ of $\psi$
minimizes $P$ precisely when 
\[ \left. \frac{\partial P(\varphi)}{\partial \varphi(n)}\right|_{\varphi = \phi} = 0 \]
for all nonterminals $n \in N \setminus \partial N$. Note that the
partial derivative of the power with respect to the potential at $n$ is given by 
\begin{align*}
  \frac{\partial P}{\partial \varphi(n)}\bigg|_{\varphi = \phi} 
  &= \sum_{t(e) = n} \frac1{r(e)}\big(\phi(t(e))-\phi(s(e))\big) - \sum_{s(e) =
  n} \frac1{r(e)}\big(\phi(t(e))-\phi(s(e))\big) \\
  &= \sum_{t(e) = n} I(e) - \sum_{s(e) = n} I(e).
\end{align*}
Thus $\phi$ obeys the principle of minimum power for $\psi$ if and only if
\[ \sum_{s(e) = n} I(e) = \sum_{t(e) = n} I(e)\] 
for all $n \in N \setminus \partial N$, and so if and only if Kirchhoff's current law holds.
\end{proof}

\subsection{A Dirichlet problem}

We remind ourselves that we are in the midst of understanding circuits as objects that define relationships between boundary potentials and boundary currents. This relationship is defined by the stipulation that voltage--current pairs on a circuit must obey Ohm's law and Kirchhoff's laws---or equivalently, Ohm's law, Kirchhoff's voltage law, and the principle of minimum power. In this subsection we show these conditions imply that for each boundary potential $\psi$ on the circuit there exists a potential $\phi$ on the circuit extending $\psi$, unique up to what may be interpreted as a choice of reference potential on each connected component of the circuit. From this potential $\phi$ we can then compute the unique voltage, current, and boundary current functions compatible with the given boundary potential.

Fix again a circuit with extended power functional $P\maps \R^N \to \R$. Let $\nabla\maps \R^{N} \to \R^{N}$ be the operator that maps a potential $\phi \in \R^N$ to the function from $N$ to $\R$ given by
\[
n \longmapsto \frac{\partial P}{\partial \varphi(n)}\bigg|_{\varphi = \phi} \;.
\]
As we have seen, this function takes potentials to twice the pointwise currents that they induce. We have also seen that a potential $\phi$ is compatible with the governing laws of circuits if and only if
\begin{equation}
\nabla \phi \big|_{\R^{\partial N}} = 0 .\label{dirichlet}
\end{equation}
The operator $\nabla$ acts as a discrete analogue of the Laplacian for the graph $\Gamma$, so we call this operator the \define{Laplacian} of $\Gamma$, and say that  equation \eqref{dirichlet} is a version of Laplace's equation. We then say that the problem of finding an extension $\phi$ of some fixed boundary potential $\psi$ that solves this Laplace's equation---or, equivalently, the problem of finding a $\phi$ that obeys the principle of minimum power for $\psi$---is a discrete version of the \define{Dirichlet problem}. 

As we shall see, this version of the Dirichlet problem always has a solution.  However, the solution is not necessarily unique.  If we take a solution $\phi$ and some $\alpha \in \R^N$ that is constant on each connected component and vanishes on the boundary of $\Gamma$, it is clear that $\phi+\alpha$ is still an extension of $\psi$ and that 
\[
\left.\frac{\partial P(\varphi)}{\partial \varphi(n)}\right|_{\varphi = \phi} = 
\left.\frac{\partial P(\varphi)}{\partial \varphi(n)}\right|_{\varphi = \phi + \alpha},
\] 
so $\phi + \alpha$ is another solution. We say that a connected component of a circuit \define{touches the boundary} if it contains a vertex in $\partial N$. Note that such an $\alpha$ must vanish on all connected components touching the boundary.

With these preliminaries in hand, we can solve the Dirichlet problem:
\begin{proposition} \label{dirichlet_problem}
For any boundary potential $\psi \in \R^{\partial N}$ there exists a potential $\phi$ obeying the principle of minimum power for $\psi$.  If we also demand that $\phi$ vanish on every connected component of $\Gamma$ not touching the boundary, then $\phi$ is unique. 
\end{proposition}
\begin{proof}
For existence, observe that the power is a nonnegative quadratic form, the extensions of $\psi$ form an affine subspace of $\R^N$, and a nonnegative quadratic form restricted to an affine subspace of a real vector space must reach a minimum somewhere on this subspace. 

For uniqueness, suppose that both $\phi$ and $\phi'$ obey the principle of minimum power for $\psi$. Let 
\[
\alpha = \phi'-\phi.
\]
Then 
\[
\alpha\big|_{\partial N} = \phi'\big|_{\partial N}-\phi\big|_{\partial N} = \psi-\psi =0,
\] 
so $\phi+\lambda\alpha$ is an extension of $\psi$ for all $\lambda \in \R$. This implies that
\[
f(\lambda) := P(\phi+\lambda\alpha)
\]
is a smooth function attaining its minimum value at both $t =0$ and $t =1$. In particular, this implies that $f'(0)=0$. But this means that when writing $f$ as a quadratic, the coefficient of $\lambda$ must be $0$, so we can write
\begin{align*}
2f(\lambda) &= \sum_{e \in E} \frac1{r(e)}\big((\phi+\lambda\alpha)(t(e))-(\phi+\lambda\alpha)(s(e))\big)^2 \\
&= \sum_{e \in E} \frac1{r(e)}\Big(\big(\phi(t(e))-\phi(s(e))\big)+\lambda\big(\alpha(t(e))-\alpha(s(e))\big)\Big)^2 \\
&=  \sum_{e \in E} \frac1{r(e)}\big(\phi(t(e))-\phi(s(e))\big)^2 + \textrm{$\lambda$-term} +  \lambda^2 \sum_{e \in E} \frac1{r(e)}\big(\alpha(t(e))-\alpha(s(e))\big)^2 \\
&=  \sum_{e \in E} \frac1{r(e)}\big(\phi(t(e))-\phi(s(e))\big)^2 + \lambda^2 \sum_{e \in E} \frac1{r(e)}\big(\alpha(t(e))-\alpha(s(e))\big)^2.
\end{align*}
Then
\[
f(1) - f(0) 
= \frac{1}{2}\sum_{e \in E} \frac1{r(e)}\big(\alpha(t(e))-\alpha(s(e))\big)^2 =0,
\]
so $\alpha(t(e)) = \alpha(s(e))$ for every edge $e \in E$. This implies that $\a$ is constant on each connected component of the graph $\Gamma$ of our circuit. 

Note that as $\alpha|_{\partial N} = 0$, $\alpha$ vanishes on every connected component of $\Gamma$ touching the boundary. Thus, if we also require that $\phi$ and $\phi'$ vanish on every connected component of $\Gamma$ not touching the boundary, then $\alpha = \phi'-\phi$ vanishes on all connected components of $\Gamma$, and hence is identically zero. Thus $\phi' = \phi$, and this extra condition ensures a unique solution to the Dirichlet problem.
\end{proof}

We have also shown the following:

\begin{proposition}\label{dirichlet_problem_2}
Suppose $\psi \in \R^{\partial N}$ and $\phi$ is a potential obeying the principle of minimum power for $\psi$.  Then $\phi'$ obeys the principle of minimum power for $\psi$ if and only if the difference $\phi' - \phi$ is constant on every connected component of $\Gamma$ and vanishes on every connected component touching the boundary of $\Gamma$.
\end{proposition}

Furthermore, $\phi$ depends linearly on $\psi$:

\begin{proposition}\label{dirichlet_problem_3: linearity}
Fix $\psi \in \R^{\partial N}$, and suppose $\phi \in \R^N$ is the unique potential obeying the principle of minimum power for $\psi$ that vanishes on all connected components of $\Gamma$ not touching the boundary. Then $\phi$ depends linearly on $\psi$.
\end{proposition}
\begin{proof}
Fix $\psi, \psi' \in \R^{\partial N}$, and suppose $\phi, \phi' \in \R^N$ obey the principle of minimum power for $\psi,\psi'$ respectively, and that both $\phi$ and $\phi'$ vanish on all connected components of $\Gamma$ not touching the boundary. 

Then, for all $\lambda \in \R$,
\[
(\phi+\lambda\phi')\big|_{\R^{\partial N}} = \phi\big|_{\R^{\partial N}} +
\lambda\phi'\big|_{\R^{\partial N}}  = \psi + \lambda\psi'
\]
and
\[
(\nabla(\phi+\lambda\phi'))\big|_{\R^{\partial N}} = 
(\nabla\phi)\big|_{\R^{\partial N}} +
\lambda(\nabla\phi')\big|_{\R^{\partial N}}  = 0.
\]
Thus $\phi+\lambda\phi'$ solves the Dirichlet problem for $\psi+\lambda\psi'$, and thus $\phi$ depends linearly on $\psi$.
\end{proof}

Bamberg and Sternberg \cite{BS} describe another way to solve the Dirichlet problem, going back to Weyl \cite{Weyl}.

\subsection{Equivalent circuits}

We have seen that boundary potentials determine, essentially uniquely, the value of all the electric properties across the entire circuit. But from the perspective of control theory, this internal structure is irrelevant: we can only access the circuit at its terminals, and hence only need concern ourselves with the relationship between boundary potentials and boundary currents. In this section we streamline our investigations above to state the precise way in which boundary currents depend on boundary potentials. In particular, we shall see that the relationship is completely captured by the functional taking boundary potentials to the minimum power used by any extension of that boundary potential. Furthermore, each such power functional determines a different boundary potential--boundary current relationship, and so we can conclude that two circuits are equivalent if and only if they have the same power functional. 

An `external behavior', or \define{behavior} for short, is an equivalence class of circuits, where two are considered equivalent when the boundary current is the same function of the boundary potential. The idea is that the boundary current and boundary potential are all that can be observed `from outside', i.e. by making measurements at the terminals.  Restricting our attention to what can be observed by making measurements at the terminals amounts to treating a circuit as a `black box': that is, treating its interior as hidden from view.  So, two circuits give the same behavior when they behave the same as `black boxes'.

First let us check that the boundary current is a function of the boundary potential.  For this we introduce an important quadratic form on the space of boundary potentials:

\begin{definition}
The \define{power functional} $Q \maps \R^{\partial N} \to \R$ of a circuit with extended power functional $P$ is given by
\[
 Q(\psi) = \min_{\phi|_{\R^{\partial N}} = \psi } P(\phi).
\]
\end{definition}

Proposition \ref{dirichlet_problem} shows the minimum above exists, so the power functional is well-defined.  Thanks to the principle of minimum power, $Q(\psi)$ equals $\frac{1}{2}$ times the power dissipated by the circuit when the boundary voltage is $\psi$.  We will later see that in fact $Q(\psi)$ is a nonnegative quadratic form on $\R^{\partial N}$. 

Since $Q$ is a smooth real-valued function on $\R^{\partial N}$, its differential $d Q$ at any given point $\psi \in \R^{\partial N}$ defines an element of the dual space $(\R^{\partial N})^\ast$, which we denote by $d Q_\psi$.  In fact, this element is equal to the boundary current $\iota$ corresponding to the boundary voltage $\psi$:

\begin{proposition} \label{boundary_current_determines_boundary_voltage}
Suppose $\psi \in \R^{\partial N}$.  Suppose $\phi$ is any extension of $\psi$ minimizing the power. Then $dQ_\psi \in (\R^{\partial N})^\ast \cong \R^{\partial N}$ gives the boundary current of the current induced by the potential $\phi$.
\end{proposition}

\begin{proof}
Note first that while there may be several choices of $\phi$ minimizing the power subject to the constraint that $\phi|_{\R^{\partial N}} = \psi$, Proposition \ref{dirichlet_problem_2} says that the difference between any two choices vanishes on all components touching the boundary of $\Gamma$.  Thus, these two choices give the same value for the boundary current $\iota\maps \partial N \to \R$. So, with no loss of generality we may assume $\phi$ is the unique choice that vanishes on all components not touching the boundary. Write $\overline\iota\maps N \to \R$ for the extension of $\iota\maps \partial N \to \R$ to $N$ taking value $0$ on $N \setminus \partial N$. 

By Proposition \ref{dirichlet_problem_3: linearity}, there is a linear operator
\[
f\maps \R^{\partial N} \longrightarrow \R^N
\]
sending $\psi \in \R^{\partial N}$ to this choice of $\phi$, and then
\[
Q(\psi) = P(f\psi).
\]
Given any $\psi' \in \R^{\partial N}$, we thus have
\begin{align*}
dQ_\psi(\psi') &= \frac{d}{d\lambda}Q(\phi +\lambda\psi') \bigg|_{\lambda=0} \\
&= \frac{d}{d\lambda}P(f(\psi+\lambda\psi'))\bigg|_{\lambda=0} \\
&= \frac{1}{2} \frac{d}{d\lambda}\sum_{e \in E} \frac1{r(e)}\bigg(f(\psi+\lambda\psi'))(t(e))-(f(\psi+\lambda\psi'))(s(e))\bigg)^2 \bigg|_{\lambda=0} \\
&= \frac{1}{2} \frac{d}{d\lambda}\sum_{e \in E} \frac1{r(e)}\bigg((f\psi(t(e))-f\psi(s(e))) \;+\;\lambda (f\psi'(t(e))- f\psi'(s(e)))\bigg)^2 \bigg|_{\lambda=0} \\
&= \sum_{e \in E} \frac1{r(e)}(f\psi(t(e))-f\psi(s(e)))(f\psi'(t(e))- f\psi'(s(e))) \\
&= \sum_{e \in E} I(e)(f\psi'(t(e))- f\psi'(s(e))) \\
&= \sum_{n \in N}\left(\sum_{t(e) = n} I(e) - \sum_{s(e) = n} I(e)\right)f\psi'(n) \\
&= \sum_{n \in N}\overline \iota(n) f\psi'(n) \\
&= \sum_{n \in \partial N}\iota(n) \psi'(n).
\end{align*}
This shows that $dQ_\psi^\ast = \iota$, as claimed.  Note that this calculation explains why we inserted a factor of $\frac{1}{2}$ in the definition of $P$: it cancels the factor of $2$ obtained from differentating a square.
 \end{proof}

Note this only depends on $Q$, which makes no mention of the potentials at
nonterminals. This is amazing: the way power depends on boundary potentials
completely characterizes the way boundary currents depend on boundary
potentials. In particular, in Part  we shall see that this
allows us to define a composition rule for behaviors of circuits.

To demonstrate these notions, we give a basic example of equivalent circuits.

\begin{example}[Resistors in series] \label{resistors_in_series}
Resistors are said to be placed in \define{series} if they are placed end to end or, more
precisely, if they form a path with no self-intersections. It is well known that
resistors in series are equivalent to a single resistor with resistance equal to
the sum of their resistances. To prove this, consider the following circuit
comprising two resistors in series, with input $A$ and output $C$:
\[
  \begin{tikzpicture}[circuit ee IEC, set resistor graphic=var resistor IEC graphic]
    \node[contact] (I1) at (0,0) [label=left:$A$] {};
    \node[circle, minimum width = 3pt, inner sep = 0pt, fill=black] (int) at (3,0) [label=above:$B$] {};
    \node[contact] (O1) at (6,0) [label=right:$C$] {};
    \draw (I1) 	to [resistor] node [label={[label distance=3pt]90:{$r_{AB}$}}] {} (int)
    to [resistor] node [label={[label distance=3pt]90:{$r_{BC}$}}] {} (O1);
  \end{tikzpicture}
\]
Now, the extended power functional $P\maps \R^{\{A,B,C\}} \to \R$ for this circuit is
\[
P(\phi) = \frac12\left(\frac1{r_{AB}}\big(\phi(A)-\phi(B)\big)^2 +
\frac1{r_{BC}}\big(\phi(B)-\phi(C)\big)^2\right),
\]
while the power functional $Q\maps \R^{\{A,C\}} \to \R$ is given by minimization
over values of $\phi(B) = x$:
\[
Q(\psi) = \min_{x \in \R} \frac12 \left(\frac1{r_{AB}}\big(\psi(A)-x\big)^2 + \frac1{r_{BC}}\big(x-\psi(C)\big)^2 \right). 
\]
Differentiating with respect to $x$, we see that this minimum occurs when
\[
\frac1{r_{AB}}\big(x-\psi(A)\big) + \frac1{r_{BC}}\big(x-\psi(C)\big) = 0,
\]
and hence when $x$ is the $r$-weighted average of $\psi(A)$ and $\psi(C)$:
\[
x = \frac{r_{BC}\psi(A) + r_{AB}\psi(C)}{r_{BC}+ r_{AB}}.
\]
Substituting this value for $x$ into the expression for $Q$ above and simplifying gives
\[
Q(\psi) = \frac12\cdot\frac1{r_{AB}+r_{BC}}\big(\psi(A)-\psi(C)\big)^2. 
\]
This is also the power functional of the circuit
\[
\begin{tikzpicture}[circuit ee IEC, set resistor graphic=var resistor IEC graphic]
\node[contact] (I1) at (0,0) [label=left:$A$] {};
\node[contact] (O1) at (3,0) [label=right:$C$] {};
\draw (I1) 	to [resistor] node [label={[label distance=3pt]90:{$r_{AB}+r_{BC}$}}] {} (O1);
\end{tikzpicture}
\]
and so the circuits are equivalent.
\end{example}


\subsection{Dirichlet forms}

In the previous subsection we claimed that power functionals are quadratic forms
on the boundary of the circuit whose behavior they represent. They comprise, in
fact, precisely those quadratic forms known as Dirichlet forms.

\begin{definition}
Given a finite set $S$, a \define{Dirichlet form} on $S$ is a quadratic form $Q:
\mathbb{R}^S \to \mathbb{R}$ given by the formula
\[
  Q(\psi) = \sum_{i,j} c_{i j} (\psi_i - \psi_j)^2
\]
for some nonnegative real numbers $c_{i j}$.  
\end{definition}

Note that we may assume without loss of generality that $c_{i i} = 0$ and $c_{i
j} = c_{j i}$; we do this henceforth.  Any Dirichlet form is nonnegative:
$Q(\psi) \ge 0$ for all $\psi \in \mathbb{R}^S$.  However, not all nonnegative
quadratic forms are Dirichlet forms.  For example, if $S = \{1, 2\}$, the
nonnegative quadratic form $Q(\psi) = (\psi_1 + \psi_2)^2$ is not a Dirichlet
form. That said, the concept of Dirichlet form is vastly more general than the
above definition: such quadratic forms are studied not just on
finite-dimensional vector spaces $\mathbb{R}^S$ but on $L^2$ of any measure
space.  When this measure space is just a finite set, the concept of Dirichlet
form reduces to the definition above.  For a thorough introduction to Dirichlet
forms, see the text by Fukushima \cite{Fukushima}.  For a fun tour of the
underlying ideas, see the paper by Doyle and Snell \cite{DS}. 

The following characterizations of Dirichlet forms help illuminate the concept:

\begin{proposition} \label{dirichlet_characterizations}
  Given a finite set $S$ and a quadratic form $Q\maps \mathbb{R}^S \to \mathbb{R}$,
  the following are equivalent:
  \begin{enumerate}[(i)]
    \item $Q$ is a Dirichlet form.

    \item $Q(\phi) \le Q(\psi)$ whenever $|\phi_i - \phi_j| \le |\psi_i -
      \psi_j|$ for all $i, j$. 

    \item $Q(\phi) = 0$ whenever $\phi_i$ is independent of $i$, and $Q$ obeys
      the \define{Markov property}: $Q(\phi) \le Q(\psi)$ when $\phi_i = \min
      (\psi_i, 1) $.
  \end{enumerate}
\end{proposition}
\begin{proof}
See Fukushima \cite{Fukushima}.
\end{proof}

While the extended power functionals of circuits are evidently Dirichlet forms,
it is not immediate that all power functionals are. For this it is crucial that
the property of being a Dirichlet form is preserved under minimising over linear
subspaces of the domain that are generated by subsets of the given finite set.

\begin{proposition} \label{dirichlet_minimization}
  If $Q\maps \R^{S+T} \to \R$ is a Dirichlet form, then 
  \[
    \min_{\nu \in \R^T} Q(-,\nu)\maps \R^S \to \R 
  \]
  is Dirichlet.
\end{proposition}
\begin{proof}
  We first note that $\min_{\nu \in \R^S} Q(-,\nu)$ is a quadratic form. Again,
  $\min_{\nu \in \R^T} Q(-,\nu)$ is well-defined as a nonnegative quadratic form
  also attains its minimum on an affine subspace of its domain. Furthermore
  $\min_{\nu \in \R^T} Q(-,\nu)$ is itself a quadratic form, as the partial
  derivatives of $Q$ are linear, and hence the points at with these minima are
  attained depend linearly on the argument of $\min_{\nu \in \R^T} Q(-,\nu)$.

  Now by Proposition \ref{dirichlet_characterizations}, $Q(\phi) \le Q(\phi')$
  whenever $|\phi_i - \phi_j| \le |\phi'_i - \phi'_j|$ for all $i,j \in S+T$. In
  particular, this implies $\min_{\nu \in \R^T} Q(\psi,\nu) \le \min_{\nu \in
  \R^T} Q(\psi',\nu)$ whenever $|\psi_i - \psi_j| \le |\psi'_i - \psi'_j|$ for
  all $i,j \in S$. Using Proposition \ref{dirichlet_characterizations} again
  then implies that $\min_{\nu \in \R^T} Q(-,\nu)$ is a Dirichlet form.
\end{proof}


\begin{corollary}
  Let $Q \maps \R^{\partial N} \to \R$ be the power functional for some circuit. Then
  $Q$ is a Dirichlet form.
\end{corollary}
\begin{proof}
  The extended power functional $P$ is a Dirichlet form, and writing $\R^N=
  \R^{\partial N} \oplus \R^{N \setminus \partial N}$ allows us to write
  \[
    Q(-) =  \min_{\phi\in \R^{N \setminus \partial N}}
    P(-,\phi). \qedhere
  \]
\end{proof}

The converse is also true: simply construct the circuit with set of vertices
$\partial N$ and an edge of resistance $\frac{1}{2c_{ij}}$ between any $i,j \in
\partial N$ such that the term $c_{ij}(\psi_i - \psi_j)$ appears in the
Dirichlet form. This gives: 

\begin{proposition}
  A function $Q$ is the power functional for some circuit if and only if $Q$ is a
  Dirichlet form.
\end{proposition}

This is an expression of the `star-mesh transform', a well-known fact of
electrical engineering stating that every circuit of linear resistors is
equivalent to some complete graph of resistors between its terminals. For more
details see \cite{vLO}. We may interpret the proof of Proposition
\ref{dirichlet_minimization} as showing that intermediate potentials at minima
depend linearly on boundary potentials, in fact a weighted average, and that
substituting these into a quadratic form still gives quadratic form.

\bigskip

In summary, in this section we have shown the existence of a surjective function
\[
  \bigg\{\begin{array}{c} \mbox{circuits of linear resistors} \\ \mbox{ with
    boundary $\partial N$} \end{array} \bigg\} \longrightarrow \bigg\{
    \mbox{Dirichlet forms on $\partial N$}\bigg\}
\]
mapping two circuits to the same Dirichlet form if and only if they have the same
external behavior.  In the next section we extend this result to encompass
inductors and capacitors too.


\section{Inductors and capacitors} \label{sec:plcs}
%%fakesubsection
The intuition gleaned from the study of resistors carries over to inductors and
capacitors too, to provide a framework for studying what are known as passive
linear networks. To understand inductors and capacitors in this way, however, we
must introduce a notion of time dependency and subsequently the Laplace
transform, which allows us to work in the so-called frequency domain. Here, like
resistors, inductors and capacitors simply impose a relationship of
proportionality between the voltages and currents that run across them. The
constant of proportionality is known as the impedance of the component.

As for resistors, the interconnection of such components may be understood, at
least formally, as a minimization of some quantity, and we may represent the
behaviors of this class of circuits with a more general idea of Dirichlet form.
We conclude this section by noting an obstruction to building a composition rule
for Dirichlet forms, motivating our work in Part . 


\subsection{The frequency domain and Ohm's law revisited}

In broadening the class of electrical circuit components under examination, we
find ourselves dealing with components whose behaviors depend on the rates of
change of current and voltage with respect to time. We thus now consider
time-varying voltages $v \maps [0,\infty) \to \R$ and currents $i \maps
  [0,\infty) \to \R$, where $t \in [0,\infty)$ is a real variable representing
    time. For mathematical reasons, we
restrict these voltages and currents to only those with (i) zero initial
conditions (that is, $f(0) = 0$) and (ii) Laplace transform lying in the field
\[
  \R(s) = \left\{ Z(s) = \tfrac{P(s)}{Q(s)} \,\Big\vert\, P, Q \mbox{
  polynomials over $\R$ in $s$}, \, Q \ne 0 \right\}
\]
of real rational functions of one variable. 
%We don't need the currents and voltages to lie in this field!!!  They just
%need to lie in some vector space over this field!!!
While it is possible that
physical voltages and currents might vary with time in a more general way, we
restrict to these cases as the rational functions are, crucially, well-behaved
enough to form a field, and yet still general enough to provide arbitrarily
close approximations to currents and voltages found in standard applications.

An \define{inductor} is a two-terminal circuit component across which the voltage is
proportional to the rate of change of the current. By convention we draw this as
follows, with the inductance $L$ the constant of proportionality:\footnote{We
  follow the standard convention of denoting inductance by the letter $L$, after
  the work of Heinrich Lenz and to avoid confusion with the $I$ used for
current.}
\[
  \begin{tikzpicture}[circuit ee IEC]
    \node[contact] (I1) at (0,0) {};
    \node[contact] (I2) at (1.83,0) {};
    \draw (I1) 	to [inductor] node [label={[label distance=2pt]{$L$}}]
    {} (I2);
  \end{tikzpicture}
\]
Writing $v_L(t)$ and $i_L(t)$ for the voltage and current over time $t$ across
this component respectively, and using a dot to denote the derivative with
respect to time $t$, we thus have the relationship 
\[
  v_L(t) = L\, \dot{i}_L(t).
\]
Permuting the roles of current and voltage, a \define{capacitor} is a two-terminal
circuit component across which the current is proportional to the rate of change
of the voltage. We draw this as follows, with the capacitance $C$ the constant
of proportionality:
\[
  \begin{tikzpicture}[circuit ee IEC]
    \node[contact] (I1) at (0,0) {};
    \node[contact] (I2) at (1.83,0) {};
    \draw (I1) 	to [capacitor] node [label={[label distance=5pt]{$C$}}]
    {} (I2);
  \end{tikzpicture}
\]
Writing $v_C(t)$, $i_C(t)$ for the voltage and current across the capacitor,
this gives the equation
\[
  i_C(t) = C\, \dot{v}_C(t).
\]
We assume here that inductances $L$ and capacitances $C$ are positive real numbers.

Although inductors and capacitors impose a linear relationship if we involve the
derivatives of current and voltage, to mimic the above work on resistors we wish
to have a constant of proportionality between functions representing the current
and voltage themselves. Various integral transforms perform just this role; electrical
engineers typically use the Laplace transform. This lets us write a function of time $t$ instead as a function of frequencies $s$, and in doing so turns differentiation with respect to $t$ into multiplication by $s$, and integration with respect to $t$ into
division by $s$.  

In detail, given a function $f(t)\maps [0, \infty) \to \R$, we define the
\define{Laplace transform} of $f$
\[
  \mathfrak{L}\{f\}(s) = \int_{0}^\infty f(t) e^{-st} dt.
\]
We also use the notation $\mathfrak{L}\{f\}(s) = F(s)$, denoting the Laplace
transform of a function in upper case, and refer to the Laplace transforms as
lying in the \define{frequency domain} or \define{$s$-domain}. For us, the three
crucial properties of the Laplace transform are then: 
\begin{enumerate}[(i)]
  \item linearity: $\mathfrak{L}\{af+bg\}(s) = aF(s)+bG(s)$ for $a,b\in \R$;
  \item differentiation: $\mathfrak{L}\{\dot{f}\}(s) = s F(s) - f(0)$;
  \item integration: if $g(t) = \int_0^t f(\tau)d\tau$ then 
 $G(s) = \frac{1}{s} F(s)$.
\end{enumerate}
Writing $V(s)$ and $I(s)$ for the Laplace transform of the voltage $v(t)$ and
current $i(t)$ across a component respectively, and recalling that by assumption
$v(t) = i(t) = 0$ for $t \le 0$, the $s$-domain behaviors of components become,
for a resistor of resistance $R$:
\[
  V(s) = RI(s),
\]
for an inductor of inductance $L$:
\[
  V(s) = sLI(s),
\]
and for a capacitor of capacitance $C$:
\[
  V(s) = \frac1{sC} I(s). 
\]

Note that for each component the voltage equals the current times a rational function of
the real variable $s$, called the \define{impedance} and in general denoted by $Z$.
Note also that the impedance is a \define{positive real function}, meaning that it lies
in the set
\[         \R(s)^+ = \{ Z \in \R(s) : \forall s \in \C \;\; \mathrm{Re}(s) > 0 \implies 
\mathrm{Re}(Z(s)) > 0 \} . \]
While $Z$ is a quotient of polynomials with real cofficients, in this definition
we are applying it to complex values of $s$, and demanding that its real part be
positive in the open left half-plane.  Positive real functions were introduced by Otto 
Brune in 1931, and they play a basic role in circuit theory \cite{Brune}.  

Indeed, Brune convincingly argued that for any conceivable passive linear component we have this generalization of Ohm's law:
\[
  V(s)=Z(s)I(s)
\]
where $I \in \R(s)$ is the \define{current}, $V \in \R(s)$ is the \define{voltage}
and $Z \in \R(s)^+$ is the \define{impedance} of the component.   As we shall see, generalizing from circuits of linear resistors to arbitrary passive linear circuits is just a matter of formally replacing resistances by  impedances.  This amounts to replacing the field $\R$ by the larger field $\R(s)$, and replacing the set of positive reals, $\R^+ = (0,\infty)$, by the set of positive real functions, $\R(s)^+$.  From a mathematical perspective we might as well work with any field with a mildly well-behaved notion of `positive element', and we do this in the next section.

\subsection{The mechanical analogy} \label{sec:mechanical}

Now that we have introduced inductors and capacitors, it is worth taking 
another glance at the analogy chart in Section \ref{sec:intro}.  What are the
analogues of resistance, inductance and capacitance in mechanics?  If we restrict attention to systems with translational degrees of freedom, the answer is given in the following
chart.

\begin{small}
\begin{center}
\begin{tabular}{|c|c|}
\hline
Electronics & Mechanics (translation) \\
\hline\hline
charge $Q$ & position $q$ \\
\hline
current $i = \dot Q$ & velocity $v = \dot q$ \\
\hline
flux linkage $\lambda$ & momentum $p$ \\
\hline
voltage $v = \dot \lambda$ & force $F = \dot p$ \\
\hline
resistance $R$ & damping coefficient $c$ \\
\hline
inductance $L$ & mass $m$ \\
\hline
inverse capacitance $C^{-1}$ & spring constant $k$ \\
\hline
\end{tabular}
\end{center}
\end{small}

A famous example concerns an electric circuit with a resistor of resistance $R$, an inductor of inductance $L$, and a capacitor of capacitance $C$, all in series:
\[
  \begin{tikzpicture}[circuit ee IEC, set resistor graphic=var resistor IEC graphic]
    \node[contact] (I1) at (0,0) {};
    \node[contact] (I2) at (1.83,0) {};
    \node[contact] (I3) at (3.66,0) {};
    \node[contact] (I4) at (5.49,0) {};
    \draw (I1) 	to [resistor] node [label={[label distance=2pt]{$R$}}]
    {} (I2);
    \draw (I2) 	to [inductor] node [label={[label distance=5pt]{$L$}}]
    {} (I3);
     \draw (I3) 	to [capacitor] node [label={[label distance=5pt]{$C$}}]
    {} (I4);
  \end{tikzpicture}
\]
We saw in Example \ref{resistors_in_series} that for resistors in series, the
resistances add.  The same fact holds more generally for passive linear circuits,
so the impedance of this circuit is the sum
\[   Z = s L + R + (sC)^{-1}  .\]
Thus, the voltage across this circuit is related to the current through the
circuit by
\[  V(s) = (s L + R + (sC)^{-1}) I(s)  \]
If $v(t)$ and $i(t)$ are the voltage and current as functions of time, we conclude that
\[  v(t) = L \frac{d}{dt}i(t) + Ri(t) + C^{-1} \int_0^t i(s) \, ds  \]
It follows that 
\[   
L \ddot{Q} + R \dot{Q} + C^{-1} Q = v
\]
where $Q(t) = \int_0^t i(t) ds$ has units of charge.  As the chart above suggests,
this equation is analogous to that of a damped harmonic oscillator:
\[    
m \ddot{q} + c \dot{q} + k q = F 
\]
where $m$ is the mass of the oscillator, $c$ is the damping coefficient, $k$ is 
the spring constant and $F$ is a time-dependent external force.

For details, and many more analogies of this sort, see the book by Karnopp, 
Margolis and Rosenberg \cite{KRM} or Brown's enormous text \cite{Brown}.   While it would be a distraction to discuss them further here, these analogies mean that our
work applies to a wide class of networked systems, not just electrical circuits.

\subsection{Generalized Dirichlet forms} \label{sec:generalized}

To understand the behavior of passive linear circuits we need to
understand how the behaviors of individual components, governed by Ohm's law,
fit together give the behavior of an entire network.
Kirchhoff's laws still hold, and so does a version of the principle of minimum power.
To see this, we generalize our remarks so far to any field with a set of `positive
elements'.

\begin{definition} 
  Given a field $\F$, we define a \define{set of positive elements} for $\F$ to be     
  any subset $\F^+\subset \F$ containing $1$ but not $0$, and closed under 
  addition, multiplication and division.  
\end{definition}

Our first motivating example arises from circuits made of resistors.  Here $\F =
\R$ is the field of real numbers and we take $\F^+ = (0,\infty)$.   Our second
motivating example arises from general passive linear circuits.  Here $\F =
\R(s)$ is the field of rational functions in one real variable, and we take
$\F^+ = \R(s)^+$ to be the positive real functions, as defined in the last
section.  

In all that follows, we fix a field $\F$ equipped with a set of positive
elements $\F^+$.  By a `circuit', we shall henceforth mean a circuit over
$\F^+$, as explained in Definition \ref{def_circuit}.   To fix the notation:

\begin{definition} \label{def_circuit_2}
A \define{(passive linear) circuit} is a graph $s,t \maps E \to N$ with $E$ as its set of \define{edges} and $N$ as its set of \define{nodes}, equipped with function $Z \maps E \to \F^+$ assigning each edge an \define{impedance}, together with finite sets $X$, $Y$, and functions $i \maps X \to N$ and $o\maps Y \to  N$. We call the sets $i(X)$, $o(Y)$, and $\partial N = i(X) \cup o(Y)$ the \define{inputs}, \define{outputs}, and \define{terminals} or \define{boundary} of the circuit, respectively.
\end{definition}

Generalizing from circuits of resistors, we define the
\define{extended power functional} $P\maps \F^N \to \F$ of any circuit by
\[
  P(\varphi) = \frac{1}{2} \sum_{e \in E} \frac1{Z(e)}\big(\varphi(t(e))-\varphi(s(e))\big)^2.
\]
and we call $\varphi \in \F^N$ a \define{potential}.  Note that a field of
characteristic 2 cannot be given a set of positive elements, so dividing by 2 is
allowed.  Note also that the extended power functional is a Dirichlet form on
$N$.

Although it is not clear what it means to minimize over the field $\F$, we can
use formal derivatives to formulate an analogue of the principle of minimum 
power.   This will actually be a `variational principle', saying the derivative of
the power functional vanishes with respect to certain variations in the potential.
As before we shall see that given Ohm's law, this principle is equivalent to
Kirchhoff's current law.

Indeed, the extended power functional $P(\varphi)$ can be considered an element
of the polynomial ring $\F[\{\varphi(n)\}_{n \in N}]$ generated by formal
variables $\varphi(n)$ corresponding to potentials at the nodes $n \in N$. We
may thus take formal derivatives of the extended power functional with respect
to the $\varphi(n)$.  We then call $\phi \in \F^N$ a \define{realizable
potential} for the given circuit if for each nonterminal node, $n \in N\setminus
\partial N$, the formal partial derivative of the extended power functional with
respect to $\varphi(n)$ equals zero when evaluated at $\phi$:
\[
  \frac{\partial P}{\partial \varphi(n)}\bigg\vert_{\varphi = \phi} = 0
\]
This terminology arises from the following fact, a generalization of Proposition
\ref{minimum_power_implies_kirchhoff_current}:

\begin{theorem} \label{thm:realizablepotentials}
The potential $\phi \in \F^N$ is a realizable potential for a given
circuit if and only if the induced current 
\[  I(e) = \frac1{Z(e)}(\phi(t(e))-\phi(s(e))) \]
obeys Kirchhoff's current law:
\[ 
\sum_{s(e) = n} I(e) = \sum_{t(e) = n} I(e)
\]  
for all $n \in N\setminus \partial N$.
\end{theorem}
\begin{proof}
The proof of this statement is exactly that for Proposition
\ref{minimum_power_implies_kirchhoff_current}. 
\end{proof}

A corollary of Theorem \ref{thm:realizablepotentials} is that the set of
states---that is, potential--current pairs---that are compatible with the
governing laws of a circuit is given by the set of realizable potentials
together with their induced currents. 

We now generalize the theory of Dirichlet forms:

\begin{definition} Given a field $\F$ with a set of positive elements $\F^+$ and
  a finite set $S$, a \define{Dirichlet form over $\F$} on $S$ is a quadratic form   
  $Q\maps \F^S \to \F$ given by the formula 
  \[ Q(\psi) = \sum_{i,j \in S} c_{ij} 
  (\psi_i - \psi_j)^2 \] for some choice of $c_{i j} \in \F^+ \cup \{0\}$.  
\end{definition}

Note that the extended power functional of any circuit is a Dirichlet form,
since $Z(e) \in \F^+$ implies $\frac{1}{2}\frac{1}{Z(e)} \in \F^+$.

\subsection{A generalized minimizability result}

We begin to move from a discussion of the intrinsic behaviors of circuits to a
discussion of their behaviors under composition. The key fact for composition of
generalized Dirichlet forms is that, in analogy with Proposition
\ref{dirichlet_minimization}, we may speak of a formal version of minimization
of Dirichlet forms. We detail this here.  In what follows, fix a field $\F$ with a 
set of positive elements $\F^+$, and let $P$ be a Dirichlet form over $\F$ on 
some finite set $S$. 

Recall that given $R \subseteq S$, we
call $\tilde\psi \in \F^S$ an \define{extension} of $\psi \in \F^R$ if
$\tilde\psi$ restricted to $R$ equals $\psi$.   We call such an
extension \define{realizable} if 
\[
    \frac{\partial P}{\partial \varphi(s)}\bigg\vert_{\varphi = \tilde\psi} = 0
  \]
for all $s \in S \setminus R$.  Note that over the real numbers $\R$ this means
that among all the extensions of $\psi$, $\tilde\psi$ minimizes the function $P$.

\begin{theorem} \label{thm:dirichletminimization}
  Let $P$ be a Dirichlet form over $\F$ on $S$, and let $R \subseteq S$ be an
  inclusion of finite sets. Then we may uniquely define a Dirichlet form
  \[\min_{S \setminus R}P: \F^R \to \F\] 
   on $R$ by sending each $\psi \in \F^R$ to
  the value $P(\tilde\psi)$ of any realizable extension $\tilde\psi$ of $\psi$.
\end{theorem}


To prove this theorem, we must first show that $\min_{S \setminus R} P$ is well-defined as a function.

\begin{lemma} \label{lem:welldefineddirichletmin}
  Let $P$ be a Dirichlet form over $\F$ on $S$, let $R \subseteq S$ be an
  inclusion of finite sets, and let $\psi \in \F^R$. Then for all realizable
  extensions $\tilde\psi$, $\tilde\psi' \in \F^S$ of $\psi$ we have $P(\tilde\psi) =
  P(\tilde\psi')$. 
\end{lemma}
\begin{proof}
  This follows from the formal version of the multivariable Taylor theorem for
  polynomial rings over a field of characteristic zero. Let $\tilde\psi$,
  $\tilde\psi' \in \F^S$ be realizable extensions of $\psi$, and note that
  $dP_{\tilde\psi}(\tilde\psi-\tilde\psi')=0$, since for all $s \in R$ we have
  $\tilde\psi(s) -\tilde\psi'(s) =0$, and for all $s \in S \setminus R$ we have
  \[
    \frac{\partial P}{\partial \varphi(s)}\bigg\vert_{\varphi = \tilde\psi}=0. 
  \]
  We may take the Taylor expansion of $P$ around $\tilde\psi$ and evaluate at
  $\tilde\psi'$. As $P$ is a quadratic form, this gives
  \begin{align*}
    P(\tilde\psi') &=
    P(\tilde\psi)+dP_{\tilde\psi}(\tilde\psi'-\tilde\psi)+P(\tilde\psi'-\tilde\psi)
    \\
    & = P(\tilde\psi)+P(\tilde\psi'-\tilde\psi).
  \end{align*}
  Similarly, we arrive at  
  \[
    P(\tilde\psi)= P(\tilde\psi')+P(\tilde\psi-\tilde\psi').
  \]
  But again as $P$ is a quadratic form, we then see that 
  \[
    P(\tilde\psi')-P(\tilde\psi) = P(\tilde\psi'-\tilde\psi) =
    P(\tilde\psi-\tilde\psi') = P(\tilde\psi)-P(\tilde\psi').
  \]
  This implies that $P(\tilde\psi')-P(\tilde\psi) = 0$, as required.
\end{proof}

It remains to show that $\min P$ remains a Dirichlet form. We do this
inductively.

\begin{lemma} \label{lem:onestepdirichletmin}
  Let $P$ be a Dirichlet form over $\F$ on $S$, and let $s \in S$ be an element
  of $S$. Then the map $\min_{\{s\}} P:\F^{S \setminus\{s\}} \to \F$ sending
  $\psi$ to $P(\tilde\psi)$ is a Dirichlet form on $S \setminus \{s\}$.
\end{lemma}
\begin{proof}
  Write $P(\phi) = \sum_{i,j} c_{ij}(\phi_i -\phi_j)^2$, assuming without loss
  of generality that $c_{sk} =0$ for all $k$. We then have
  \[
    \frac{\partial P}{\partial \varphi(s)}\bigg\vert_{\varphi = \phi} = \sum_k
    2c_{ks}(\phi_s-\phi_k),
  \]
  and this is equal to zero when
  \[
    \phi_s = \frac{\sum_k c_{ks}\phi_k}{\sum_k c_{ks}}.
  \]
  Thus $\min_{\{s\}}P$ may be given explicitly by the expression
  \[
    \min_{\{s\}} P(\psi) = \sum_{i,j \in S \setminus \{s\}} c_{ij}(\psi_i -\psi_j)^2 +
    \sum_{\ell \in S \setminus \{s\}} c_{\ell s}\left(\psi_\ell - \tfrac{\sum_k
      c_{ks} \psi_k}{\sum_k c_{ks}}\right)^2.
  \]
  We must show this is a Dirichlet form on $S \setminus \{s\}$. 
  
  As the sum of Dirichlet forms is evidently Dirichlet, it suffices to check that the expression 
  \[
    \sum_\ell c_{\ell s}\left(\psi_\ell - \tfrac{\sum_k c_{ks} \psi_k}{\sum_k
      c_{ks}}\right)^2
  \]
  is Dirichlet on $S \setminus \{s\}$. Multiplying through by the constant
  $(\sum_k c_{ks})^2 \in \F^+$, it further suffices to check
  \begin{align*}
    \sum_\ell c_{\ell s}\left(\sum_k c_{ks} \psi_\ell - \sum_k c_{ks}
    \psi_k\right)^2 &= \sum_\ell c_{\ell s} \left(\sum_k c_{ks} (\psi_\ell -
    \psi_k)\right)^2 \\
    &= \sum_\ell c_{\ell s} \left(2 \sum_{\substack{k,m \\ k \ne m}} c_{k s} c_{ms}
    (\psi_\ell-\psi_k)(\psi_\ell - \psi_m) + \sum_{k} c_{k
    s}^2(\psi_\ell-\psi_k)^2\right) \\
    &= 2\sum_{\substack{k,\ell,m \\ k \ne m}} c_{\ell s} c_{k s} c_{ms}
    (\psi_\ell-\psi_k)(\psi_\ell - \psi_m) + \sum_{k, \ell} c_{\ell s}c_{k
    s}^2(\psi_\ell-\psi_k)^2
  \end{align*}
  is Dirichlet. But
  \begin{align*}
    &\quad (\psi_k - \psi_\ell)(\psi_k - \psi_m)+(\psi_\ell - \psi_k)(\psi_\ell -
    \psi_m) + (\psi_m-\psi_k)(\psi_m-\psi_\ell) \\ 
    &= \psi_k^2+\psi_\ell^2+\psi_m^2-\psi_k\psi_\ell- \psi_k\psi_m -
    \psi_\ell\psi_m \\
    &= \tfrac12\big( (\psi_k-\psi_\ell)^2 +(\psi_k-\psi_m)^2
    +(\psi_\ell-\psi_m)^2\big),
  \end{align*}
  so this expression is indeed Dirichlet. Indeed, pasting these computations
  together shows that
  \[
    \min_{\{s\}}P(\psi) = \sum_{i,j} \left(c_{ij}+\frac{c_{is}c_{js}}{{\textstyle \sum_k}
    c_{ks}}\right)(\psi_i-\psi_j)^2. \qedhere
  \]
\end{proof}

With these two lemmas, the proof of Theorem \ref{thm:dirichletminimization}
becomes straightforward.

\begin{proof}[Proof of Theorem \ref{thm:dirichletminimization}]
  Lemma \ref{lem:welldefineddirichletmin} shows that $\min_{S \setminus R}P$ is a well-defined
  function. As $R$ is a finite set, we may write it $R = \{s_1,\dots, s_n\}$ for
  some natural number $n$. Then we may define a sequence of functions $P_i =
  \min_{\{s_1, \dots,s_i\}} P_{i-1}$, $1 \le i\le n$. Define also $P_0 = P$, and note
  that $P_n = \min_{S \setminus R}P$. Then, by Lemma
  \ref{lem:onestepdirichletmin}, each $P_i$ is Dirichlet as $P_{i-1}$
  is. This proves the proposition.
\end{proof}

We can thus define the power functional of a circuit by analogy with circuits made
of resistors:

\begin{definition}
The \define{power functional} $Q \maps \R^{\partial N} \to \R$ of a circuit with extended power functional $P$ is given by
\[
 Q = \min_{N \setminus \partial N}  P .
\]
\end{definition}

As before, we define two circuits to be \define{equivalent} if they have the same
power functional, and define the \define{behavior} of a circuit to be its equivalence
class.  

\subsection{Composition of Dirichlet forms}

It would be nice to have a category in which circuits are morphisms, and a
category in which Dirichlet forms are morphisms, such that the map sending
a circuit to its behavior is a functor.  Here we present a na\"ive attempt to
constructed the category with Dirichlet forms as morphisms, using the principle
of minimum power to compose these morphisms.  Unfortunately the proposed category does not include identity morphisms.  However, it points in the right direction, and underlines the importance of the cospan formalism we then turn to develop.

We can define a composition rule for Dirichlet forms that reflects composition of circuits.
Given finite sets $S$ and $T$, let $S+T$ denote their disjoint union.  Let
$D(S,T)$ be the set of Dirichlet forms on $S+T$. There is a way
to compose these Dirichlet forms
\[ 
\circ \maps D(T,U) \times D(S,T) \to D(S,U) 
\]
defined as follows.  Given $P \in D(T,U)$ and $Q \in D(S,T)$, let
\[ 
  (P \circ Q)(\alpha, \gamma) = \min_{T} Q(\alpha, \beta) + P(\beta, \gamma),
\]
where $\alpha \in F^S, \gamma \in F^U$. This operation has a clear
interpretation in terms of electrical circuits: the power used by the entire
circuit is just the sum of the power used by its parts. 

It is immediate from Theorem \ref{thm:dirichletminimization} that this
composition rule is well-defined: the composite of two Dirichlet forms is again
a Dirichlet form. Moreover, this composition is associative. However, it fails
to provide the structure of a category, as there is typically no Dirichlet form
$1_S \in D(S,S)$ playing the role of the identity for this composition. For an
indication of why this is so, let $\{\bullet\}$ be a set with one element, and
suppose that some Dirichlet form $I(\beta,\gamma) = k(\beta-\gamma)^2 \in
D(\{\bullet\},\{\bullet\})$ acts as an identity on the right for this
composition. Then for all $Q(\alpha,\beta) = c(\alpha-\beta)^2 \in
D(\{\bullet\},\{\bullet\})$, we must have
\begin{align*}
  c\alpha^2 &= Q(\alpha,0) \\
  &= (I \circ Q)(\alpha,0) \\ 
  &= \min_{\beta \in \F} Q(\alpha, \beta) + I(\beta,0) \\
  &= \min_{\beta \in \F} k(\alpha-\beta)^2 + c\beta^2 \\
  &= \frac{kc}{k+c}\alpha^2,
\end{align*}
where we have noted that $\frac{kc}{k+c}\alpha^2$ minimizes $k(\alpha-\beta)^2 +
c\beta^2$ with respect to $\beta$. But for any choice of $k \in \F$ this
equality only holds when $c = 0$, so no such Dirichlet form exists. Note,
however, that for $k>> c$ we have $c\alpha^2 \approx \frac{kc}{k+c}\alpha^2$, so
Dirichlet forms with large values of $k$---corresponding to resistors with
resistance close to zero---act as `approximate identities'.

In this way we might interpret the identities we wish to introduce
into this category as the behaviors of idealized components with zero
resistance: perfectly conductive wires. Unfortunately, the power functional of a
purely conductive wire is undefined: the formula for it involves division by
zero.  In real life, coming close to this situation leads to the disaster that
electricians call a `short circuit': a huge amount of power dissipated for even
a small voltage.  This is why we have fuses and circuit breakers.

Nonetheless, we have most of the structure required for a category. A `category
without identity morphisms' is called a \define{semicategory}, so we see
\begin{proposition}
There is a semicategory where:
\begin{itemize}
\item the objects are finite sets,

\item a morphism from $T$ to $S$ is a Dirichlet form $Q \in D(S,T)$.  

\item composition of morphisms is given by 
\[
(R \circ Q)(\gamma, \alpha) = \min_{T} Q(\gamma, \beta) + R(\beta, \alpha).
\]

\end{itemize}
\end{proposition}

We would like to make this into a category. One easy way to do this is to
formally adjoin identity morphisms; this trick works for any semicategory.
However, we obtain a better category if we include \emph{more} morphisms: more
behaviors corresponding to circuits made of perfectly conductive wires. As the
expression for the extended power functional includes the reciprocals of
impedances, such circuits cannot be expressed within the framework we have
developed thus far. Indeed, for these idealized circuits there is no function
taking boundary potentials to boundary currents: the vanishing impedance would
imply that any difference in potentials at the boundary induces `infinite'
currents. To deal with this issue, we generalize Dirichlet forms to Lagrangian
relations.  First, however, we develop a category theoretic framework, based
around decorated cospans, to define the category of circuits itself and
understand its basic properties.

\section{The category of passive linear circuits} \label{sec:circdef}
In this part we move our focus from the semantics of circuit diagrams to the
syntax, addressing the question ``How do we interact with circuit diagrams?''.
Informally, the answer to this is that we interact with them by connecting them
to each other, perhaps after moving them into the right form by rotating or
reflecting them, or by crossing or bending some of the wires. To formalize this,
we adopt a category theoretic viewpoint, defining various dagger compact
categories with circuits and their behaviors as morphisms. We claim a formal
analysis of this structure, especially of the composition or connection of
circuits, has been overlooked in analysis of circuits thus far. This part
culminates in the definition of two important categories, the category $\Circ$
of circuit diagrams, and the category $\LagrRel$ containing all behaviors of
circuits. We also develop the technical material required to appreciate the
structure of these categories, and that aids understanding of the relationship
between the two, to be addressed in Part .

%%fakesubsection
In Part  we defined a circuit of linear resistors to be a labelled graph with marked input and output terminals, as in the example:
\[
\begin{tikzpicture}[circuit ee IEC, set resistor graphic=var resistor IEC graphic]
\node[contact] (I1) at (0,2) {};
\node[contact] (I2) at (0,0) {};
\coordinate (int1) at (2.83,1) {};
\coordinate (int2) at (5.83,1) {};
\node[contact] (O1) at (8.66,2) {};
\node[contact] (O2) at (8.66,0) {};
\node (input) at (-2,1) {\small{\textsf{inputs}}};
\node (output) at (10.66,1) {\small{\textsf{outputs}}};
\draw (I1) 	to [resistor] node [label={[label distance=2pt]85:{$1\Omega$}}] {} (int1);
\draw (I2)	to [resistor] node [label={[label distance=2pt]275:{$1\Omega$}}] {} (int1)
				to [resistor] node [label={[label distance=3pt]90:{$2\Omega$}}] {} (int2);
\draw (int2) 	to [resistor] node [label={[label distance=2pt]95:{$1\Omega$}}] {} (O1);
\draw (int2)		to [resistor] node [label={[label distance=2pt]265:{$3\Omega$}}] {} (O2);
\path[color=gray, very thick, shorten >=10pt, ->, >=stealth, bend left] (input) edge (I1);		\path[color=gray, very thick, shorten >=10pt, ->, >=stealth, bend right] (input) edge (I2);		
\path[color=gray, very thick, shorten >=10pt, ->, >=stealth, bend right] (output) edge (O1);
\path[color=gray, very thick, shorten >=10pt, ->, >=stealth, bend left] (output) edge (O2);
\end{tikzpicture}
\]
We then defined general passive linear circuits by replacing resistances with
impedances chosen from a set of positive elements $\F^+$ in any field $\F$.
In fact, these circuits are examples of decorated cospans.  This gives a dagger compact category $\Circ$ whose morphisms are circuits, with the
dagger compact structure expressing standard operations on circuits.

We actually give two constructions of this category, first arriving at a category of
cospans decorated by $\F^+$-graphs, and then showing that this is
a full subcategory of the decategorification of a bicategory of cospans of
$\F^+$-graphs.

\subsection{A decorated cospan construction}

We officially defined a `circuit' in Definition \ref{def_circuit_2}, but now we can give an equivalent definition using cospans:

\begin{lemma} A circuit is a cospan of finite sets $X \stackrel{i}{\longleftarrow} N
\stackrel{o}{\longrightarrow} Y$ together with an $\F^+$-graph whose set
of nodes is $N$.
\end{lemma}

\begin{proof}
This is just a matter of remembering the terminology: recall that an $\F^+$-graph
is a graph whose edges are labelled with elements of $\F^+$, our chosen set of positive
elements in the field $\F$.
\end{proof}

This suggests that circuits should be morphisms in a decorated  cospan category. Indeed, we can show that the map taking a finite set $N$ to the set of $\F^+$-graphs with
set $N$ of nodes in fact forms a lax symmetric monoidal functor. This allows us to apply
decorated cospans to construct a category of circuits.

To this end, define the functor
\[
  \mathrm{Circuit}\maps (\mathrm{FinSet},+) \longrightarrow (\mathrm{Set},\times)
\]
to take a finite set $N$, as an object of $\mathrm{FinSet}$, to the set
$\mathrm{Circuit}(N)$ of $\F^+$-graphs $(N,E,s,t,r)$ with $N$ as their
set of nodes. On
morphisms let it take a function $f\maps N \to M$ to the function that pushes
labelled graph structures on a set $N$ forward onto the set $M$:
\begin{align*}
  \mathrm{Circuit}(f)\maps \mathrm{Circuit}(N) &\longrightarrow
  \mathrm{Circuit}(M); \\
  (N,E,s,t,r) &\longmapsto (M,E,f \circ s, f \circ t, r).
\end{align*}
Note that as this map simply acts by post-composition, our map
$\mathrm{Circuit}$ is indeed functorial.

We then arrive at a lax symmetric monoidal functor by equipping this functor with the
natural transformation 
\begin{align*}
  \rho_{N,M}\maps \mathrm{Circuit}(N) \times \mathrm{Circuit}(M)
  &\longrightarrow \mathrm{Circuit}(N+M); \\
  \big( (N,E,s,t,r), (M,F,s',t',r') \big) &\longmapsto
  \big(N+M,E+F,s+s',t+t',[r,r']\big),
\end{align*}
together with the unit map
\begin{align*}
  \rho_1\maps 1 &\longrightarrow \mathrm{Circuit}(\varnothing); \\
  \bullet &\longmapsto
  (\varnothing,\varnothing,\varnothing,\varnothing,\varnothing),
\end{align*}
where we use $\varnothing$ to denote both the empty set and the unique function
of the appropriate codomain with domain the empty set. The naturality of this
collection of morphisms, as well as the coherence laws for lax symmetric
monoidal functors, follow from the universal property of the coproduct. 

\begin{definition}
  We define
  \[
    \mbox{\define{Circ}} = \mathrm{CircuitCospan} .
  \]
\end{definition}

\begin{corollary}
The category $\Circ$ is a dagger compact category. 
\end{corollary}

The different structures of this category capture different operations that can
be performed with circuits. The composition expresses the fact that we can
connect the outputs of one circuit to the inputs of the next, while the monoidal
composition models the placement of circuits side-by-side. The symmetric
monoidal structure allows us reorder input and output wires, and the compactness
captures the interchangeability between input and outputs of
circuits---that is, the fact that we can choose any input to our
circuit and consider it instead as an outputl, and vice versa.  Finally, the
dagger structure expresses the fact that we may reflect a whole circuit, switching
all inputs with all outputs.

\subsection{A bicategory of circuits}

As shown by B\'enabou \cite{Be}, cospans are most naturally thought of as
1-morphisms in a bicategory.  To obtain the category we are calling
$\mathrm{Cospan}(\mc C)$, we `decategorify' this bicategory by discarding
2-morphisms and identifying isomorphic 1-morphisms. Since we are studying
circuits using decorated cospans, this suggests that circuits, too, are most
naturally thought of as 1-morphisms in a bicategory.  Indeed this is the case.  

One route to the bicategory whose 1-morphisms are circuits would be to take
the theory of decorated cospans \cite{Fon} and show that it can be enhanced to
create not merely categories whose morphisms are isomorphism classes of
decorated cospans, but bicategories whose 1-morphisms are exactly decorated
cospans.  A less powerful but easier approach is as follows.

Recall that we define a graph to be a pair of functions $s,t \maps E \to
N$ where $E$ and $N$ are finite sets, and an $L$-graph to be a graph further
equipped with a function $r\maps E \to L$.  Thus, an $L$-graph looks like this:
\[
\xymatrix{
L & E \ar@<2.5pt>[r]^{s} \ar@<-2.5pt>[r]_{t} \ar[l]_{r} & N
}
\]
Given $L$-graphs $\Gamma = (E,N,s,t,r)$ and $\Gamma' = (E',N',s',t',r')$, a
morphism of $L$-graphs $\Gamma \to \Gamma'$ is a pair of functions $\eps\maps E \to
E'$, $v\maps N \to N'$ such that the following diagrams commute:
\[
\xymatrix @R=5.7pt{
& E \ar[dl]_{r} \ar[dd]^{\eps} \\
L \\
& E' \ar[ul]^{r'}
}
\qquad\qquad
\xymatrix{
E \ar[r]^{s} \ar[d]_{\eps} & N \ar[d]^{v}  \\
E' \ar[r]_{s'} & N'
}
\qquad\qquad
\xymatrix{
E \ar[r]^{t} \ar[d]_{\eps} & N \ar[d]^{v}  \\
E' \ar[r]_{t'} & N'.
}
\]
These $L$-graphs and their morphisms form a category \define{$L$-Graph}. Using
results about colimits in the category of sets, it is straightforward to check
that this category has finite colimits.

B\'enabou \cite{Be} gave a general construction of bicategories starting from any
category with pushouts.  Since $L$-Graph has pushouts, it follows that 
there is a bicategory \define{Cospan($L$-Graph)} with
\begin{itemize}
\item $L$-labelled graphs as objects,
\item cospans in $L$-Graphas morphisms, and 
\item maps of cospans as 2-morphisms.
\end{itemize}
Since $L$-Graph also has coproducts, this bicategory is symmetric monoidal and 
in fact compact, thanks to the work of Stay \cite{St}.   We make the following definition:

\begin{definition}
The bicategory \define{$2\mbox{-}\Circ$} is the full and 2-full sub-bicategory of
$\mathrm{Cospan(}\F^+\mbox{-}\mathrm{Graph})$ with objects those $\F^+$-graphs with no edges.
\end{definition}

It can be shown that every object in Cospan($\F^+$-Graph) is self-dual, and this
implies that $2\mbox{-}\Circ$ is again a compact closed bicategory.
Note that the category obtained by decategorifying this bicategory has
finite sets as objects, with morphisms being isomorphism classes of cospans in 
$\F^+$-Graph with feet such objects.   Thus, decategorifying $2\mbox{-}\Circ$
gives a category equivalent to our previously defined category $\Circ$.

\section{Circuits as Lagrangian relations} \label{sec:circlagr}
%%fakesubsection
In the first part of this paper, we explored the semantic content contained in
circuit diagrams, leading to an understanding of circuit diagrams as expressing
some relationship between the potentials and currents that can simultaneously be
imposed on some subset, the so-called terminals, of the nodes of the circuit. We
called this collection of possible relationships the behavior of the circuit.
While in that setting we used the concept of Dirichlet forms to describe this
relationship, we saw in the end that describing circuits as Dirichlet forms does
not allow for a straightforward notion of composition of circuits. 

In this section, inspired by the principle of least action of classical
mechanics in analogy with the principle of minimum power, we develop a setting
for describing behaviors that allows for easy discussion of composite
behaviors: Lagrangian subspaces of symplectic vector spaces. These Lagrangian
subspaces provide a more direct, invariant perspective, comprising precisely the
set of vectors describing the possible simultaneous potential and current
readings at all terminals of a given circuit. As we shall see, one immediate and
important advantage of this setting is that we may model wires of zero
resistance.

Recall that we write $\F$ for some field, which for our applications is
usually the field $\R$ of real numbers or the field $\R(s)$ of rational
functions of one real variable.

\subsection{Symplectic vector spaces}

A circuit made up of wires of positive resistance defines a function from
boundary potentials to boundary currents. A wire of zero resistance, however,
does not define a function: the principle of minimum power is obeyed as long as
the potentials at the two ends of the wire are equal. More generally, we may
thus think of circuits as specifying a set of allowed voltage-current pairs, or
as a relation between boundary potentials and boundary currents. This set forms
what is called a Lagrangian subspace, and is given by the graph of the
differential of the power functional. More generally, Lagrangian submanifolds
graph derivatives of smooth functions: they describe the point evaluated and the
tangent to that point within the same space.

The material in this section is all known, and follows without great difficulty
from the definitions. To keep this section brief we omit proofs. See any
introduction to symplectic vector spaces, such as Cimasoni and Turaev \cite{CT} or
Piccione and Tausk \cite{PT}, for details.

\begin{definition}
  Given a finite-dimensional vector space $V$ over a field $\F$, a 
  \define{symplectic form}
  $\omega\maps V \times V \to \F$ on $V$ is an alternating nondegenerate bilinear
  form.  That is, a symplectic form $\omega$ is a function $V \times V \to \F$
  that is
  \begin{enumerate}[(i)]
    \item bilinear: for all $\lambda \in \F$ and all $u,v \in V$ we have
      $\omega(\lambda u,v) = \omega(u,\lambda v) =  \lambda \omega(u,v)$;
    \item alternating: for all $v \in V$ we have $\omega(v,v) = 0$; and
    \item nondegenerate: given $v \in V$, $\omega(u,v) = 0$ for all $u \in V$ if
      and only if $u = 0$.
  \end{enumerate} 
  A \define{symplectic vector space} $(V,\omega)$ is a vector space $V$ equipped
  with a symplectic form $\omega$. 

  Given symplectic vector spaces $(V_1,\omega_1), (V_2, \omega_2)$, a
  \define{symplectic map} is a linear map 
  \[
    f\maps (V_1,\omega_1) \longrightarrow (V_2, \omega_2)
  \]
  such that $\omega_2(f(u),f(v)) = \omega_1(u,v)$ for all $u,v \in V_1$. A
  \define{symplectomorphism} is a symplectic map that is also an isomorphism. 
\end{definition}

An alternating form is always \define{antisymmetric}, meaning that $\omega(u,v) = 
-\omega(v,u)$ for all $u,v \in V$.  The converse is true except in characteristic 2.
A \define{symplectic basis} for a symplectic vector space $(V,\omega)$ is a
basis $\{p_1,\dots,p_n,q_1,\dots,q_n\}$ such that $\omega(p_i,p_j) =
\omega(q_i,q_j) = 0$ for all $1 \le i,j \le n$, and $\omega(p_i,q_j) =
\delta_{ij}$ for all $1 \le i,j\le n$, where $\delta_{ij}$ is the Kronecker delta,
equal to $1$ when $i =j$, and $0$ otherwise. A symplectomorphism maps symplectic
bases to symplectic bases, and conversely, any map that takes a symplectic basis
to another symplectic basis is a symplectomorphism.

\begin{example}[The symplectic vector space generated by a finite set]
  \label{ex:symplectic_space_generated_by_set}
  Given a finite set $N$, we consider the vector space $\vectf{N}$ a symplectic
  vector space $(\vectf{N},\omega)$, with symplectic form 
  \[
    \omega\big((\phi,i),(\phi',i')\big) = i'(\phi)-i(\phi').  
  \] 
  Let $\{\phi_n\}_{n \in N}$ be the basis of $\F^N$ consisting of the functions
  $N \to \F$ mapping $n$ to $1$ and all other elements of $n$ to $0$, and let
  $\{i_n\}_{n \in N} \subseteq {(\F^N)}^\ast$ be the dual basis. Then
  $\{(\phi_n,0),(0,i_n)\}_{n\in N}$ forms a symplectic basis for $\vectf{N}$.  
\end{example}

There are two common ways we will build symplectic spaces from other
symplectic spaces: conjugation and summation. Given a symplectic form $\omega$,
we may define its \define{conjugate} symplectic form $\overline\omega = -
\omega$, and write the conjugate symplectic space $(V,\overline\omega)$ as
$\overline V$. Given two symplectic vector spaces $(U, \nu),(V,\omega)$, we
consider their direct sum $U \oplus V$ a symplectic vector space with the
symplectic form $\nu+\omega$, and call this the \define{sum} of the two
symplectic vector spaces. Note that this is not a product in the category of
symplectic vector spaces and symplectic maps.

The symplectic form provides a notion of orthogonal complement.  Given a subspace $S$ of $V$, we define its \define{complement}
\[
  S^\circ = \{v \in V \mid \omega(v,s) = 0 \textrm{ for all } s \in S\}.
\]
Note that this construction obeys the following identities, where $S$ and $T$
are subspaces of $V$:
\begin{align*}
  \dim S+ \dim S^\circ &= \dim V \\
  (S^\circ)^\circ &= S \\
  (S + T)^\circ &= S^\circ \cap T^\circ \\
  (S \cap T)^\circ &= S^\circ + T^\circ.
\end{align*}

In the symplectic vector space $\vectf{N}$, the subspace $\F^N$ has the
property of being a maximal subspace such that the symplectic form restricts to
the zero form on this subspace. Subspaces with this property are known as Lagrangian
subspaces, and they may all be realized as the image of $\vectf{N}$ 
under symplectomorphisms from $\vectf{N}$ to itself.

\begin{definition} 
  Let $S$ be a linear subspace of a symplectic vector space $(V,\omega)$. We say
  that $S$ is \define{isotropic} if $\omega|_{S \times S} = 0$, and that $S$ is
  \define{coisotropic} if $S^\circ$ is isotropic. A subspace is
  \define{Lagrangian} if it is both isotropic and coisotropic, or equivalently, if it  
  is a maximal isotropic subspace.
\end{definition}

Lagrangian subspaces are also known as Lagrangian correspondences and canonical
relations. Note that a subspace $S$ is isotropic if and only if $S \subseteq
S^\circ$. This fact helps with the following characterizations of Lagrangian
subspaces.

\begin{proposition} \label{lagrangian_characterization} 
  Given a subspace $L \subset V$ of a symplectic vector space $(V,\omega)$, the
  following are equivalent: 
  \begin{enumerate}[(i)] 
    \item $L$ is Lagrangian.  
    \item $L$ is maximally isotropic.  
    \item $L$ is minimally coisotropic.  
    \item $L = L^\circ$.  
    \item $L$ is isotropic and $\dim L = \frac12 \dim V$.
  \end{enumerate} 
\end{proposition}

From this proposition it follows easily that the direct sum of two Lagrangian
subspaces in Lagrangian in the sum of their ambient spaces. We also observe that
an advantage of isotropy is that there is a good way to take a quotient of a
symplectic vector space by an isotropic subspace---that is, there is a way to
put a natural symplectic structure on the quotient space.

\begin{proposition}
  Let $S$ be an isotropic subspace of a symplectic vector space $(V,\omega)$.
  Then $S^\circ/S$ is a symplectic vector space with symplectic form
  $\omega'(v+S,u+S) = \omega(v,u)$.
\end{proposition}
\begin{proof} 
  The function $\omega'$ is a well-defined due to the isotropy of
  $S$---by definition adding any pair $(s,s')$ of elements of $S$ to a pair
  $(v,u)$ of elements of $S^\circ$ does not change the value of
  $\omega(v+s,u+s')$. As $\omega$ is a symplectic form, one can check that
  $\omega'$ is too.  
\end{proof}

\subsection{Lagrangian subspaces from quadratic forms}

Lagrangian subspaces are of relevance to us here as the behavior of any passive
linear circuit forms a Lagrangian subspace of the symplectic vector space
generated by the nodes of the circuit. We think of this vector space as
comprising two parts: a space $\F^N$ of potentials at each node, and a dual
space ${(\F^N)}^\ast$ of currents. To make clear how circuits can be interpreted
as Lagrangian subspaces, here we describe how Dirichlet forms on a finite set
$N$ give rise to Lagrangian subspaces of $\vectf{N}$. More generally, we show
that there is a one-to-one correspondence between Lagrangian subspaces and
quadratic forms.

\begin{proposition} \label{prop:qfls}
  Let $N$ be a finite set. Given a quadratic form $Q$ over $\F$ on $N$, the
  subspace 
  \[ 
    L_Q = \big\{(\phi,dQ_\phi) \mid \phi \in \F^N\big\} \subseteq \vectf{N},
  \] 
  where $dQ_\phi \in {(\F^N)}^\ast$ is the formal differential of $Q$ at $\phi
  \in \F^N$, is Lagrangian. Moreover, this construction gives a one-to-one correspondence 
  \[ 
    \bigg\{\mbox{Quadratic forms over $\F$ on $N$}\bigg\} \longleftrightarrow
    \bigg\{\begin{array}{c} \mbox{Lagrangian subspaces of $\vectf{N}$}\\
      \mbox{with trivial intersection with $\{0\} \oplus {(\F^N)}^\ast \subseteq
      \vectf{N}$} \end{array} \bigg\}.  
  \]
\end{proposition}
\begin{proof}
  The symplectic structure on $\vectf{N}$ and our notation for it is given in
  Example \ref{ex:symplectic_space_generated_by_set}. 

  Note that for all $n,m \in N$ the corresponding basis elements 
  \[
    \frac{\partial^2 Q}{\partial \phi_n \partial \phi_m} = dQ_{\phi_n}(\phi_m) =
    dQ_{\phi_m}(\phi_n),
  \]
  so $dQ_\phi(\psi) = dQ_\psi(\phi)$ for all $\phi,\psi \in \F^N$. Thus $L_Q$ is
  indeed Lagrangian: for all $\phi,\psi \in \F^N$ 
  \[
    \omega\big((\phi,dQ_\phi),(\psi,dQ_\psi)\big) = dQ_\psi(\phi) -
    dQ_\phi(\psi) = 0.
  \]

  Observe also that for all quadratic forms $Q$ we have $dQ_0 = 0$, so the only
  element of $L_Q$ of the form $(0,i)$, where $i \in {(\F^N)}^\ast$, is $(0,0)$.
  Thus $L_Q$ has trivial intersection with the subspace $\{0\} \oplus
  {(\F^N)}^\ast$ of $\vectf{N}$. This $L_Q$ construction forms the leftward
  direction of the above correspondence.

  For the rightward direction, suppose that $L$ is a Lagrangian subspace of
  $\vectf{N}$ such that $L \cap (\{0\} \oplus {(\F^N)}^\ast) = \{(0,0)\}$. Then
  for each $\phi \in \F^N$, there exists a unique $i_\phi \in {(\F^N)}^\ast$
  such that $(\phi,i_\phi) \in L$. Indeed, if $i_\phi$ and $i_\phi'$ were
  distinct elements of ${(\F^N)}^\ast$ with this property, then by linearity
  $(0,i_\phi-i_\phi')$ would be a nonzero element of $L \cap (\{0\} \oplus
  {(\F^N)}^\ast$, contradicting the hypothesis about trivial intersection. We
  thus can define a function, indeed a linear map, $\F^N \to {(\F^N)}^\ast; \phi
  \mapsto i_\phi$. This defines a bilinear form $Q(\phi,\psi) = i_\phi(\psi)$ on
  $\F^N \oplus \F^N$, and so $Q(\phi) = i_\phi(\phi)$ defines a quadratic form
  on $\F^N$. 

  Moreover, $L$ is Lagrangian, so
  \[
    \omega\big((\phi,i_\phi), (\psi,i_\psi)\big) = i_\psi(\phi) - i_\phi(\psi) =
    0,
  \]
  and so $Q(-,-)$ is a symmetric bilinear form. This gives a one-to-one
  correspondence between Lagrangian subspaces of specified type, symmetric
  bilinear forms, and quadratic forms, and so in particular gives the claimed
  one-to-one correspondence. 
\end{proof}

In particular, every Dirichlet form defines a Lagrangian subspace. 

\subsection{Lagrangian relations}

Recall that a relation between sets $X$ and $Y$ is a subset $R$ of their product
$X \times Y$. Furthermore, given relations $R \subseteq X \times Y$ and $S
\subseteq Y \times Z$, there is a composite relation $(S \circ R) \subseteq X
\times Z$ given by pairs $(x,z)$ such that there exists $y \in Y$ with $(x,y)
\in R$ and $(y,z) \in S$---a direct generalization of function composition. A
Lagrangian relation between symplectic vector spaces $V_1$ and $V_2$ is a
relation between $V_1$ and $V_2$ that forms a Lagrangian subspace of the
symplectic vector space $\overline{V_1} \oplus V_2$. This gives us a
way to think of certain Lagrangian subspaces, such as those arising from
circuits, as morphisms, giving a way to compose them.

\begin{definition}
  A \define{Lagrangian relation} $L\maps V_1 \to V_2$ is a Lagrangian subspace $L$
  of $\overline{V_1} \oplus V_2$. 
\end{definition}

This is a generalization of the notion of symplectomorphism: any symplectomorphism
$f\maps V_1 \to V_2$ forms a Lagrangian subspace when viewed as a
relation $f \subseteq \overline{V_1} \oplus V_2$. More generally, any symplectic
map $f\maps V_1 \to V_2$ forms an isotropic subspace when viewed as a relation in
$\overline{V_1} \oplus V_2$. 

Importantly for us, the composite of two Lagrangian relations is again a
Lagrangian relation.  This is well-known \cite{Weinstein}, but sufficiently 
easy and important to us that we provide a proof.

\begin{proposition} \label{prop:lagrangian_composition}
  Let $L\maps V_1 \to V_2$ and $L'\maps V_2 \to V_3$ be Lagrangian relations. Then their
  composite relation $L' \circ L$ is a Lagrangian relation $V_1 \to V_3$.
\end{proposition}

We prove this proposition by way of two lemmas detailing how the Lagrangian
property is preserved under various operations. The first lemma says that the
intersection of a Lagrangian space with a coisotropic space is in some sense
Lagrangian, once we account for the complement.

\begin{lemma} \label{restriction_of_lagrangians}
  Let $L \subseteq V$ be a Lagrangian subspace of a symplectic vector space $V$,
  and $S \subseteq V$ be an isotropic subspace of $V$. Then $(L\cap S^\circ) +S
  \subseteq V$ is Lagrangian in $V$.
\end{lemma}
\begin{proof}
  Recall from Proposition \ref{lagrangian_characterization} that a subspace is
  Lagrangian if and only if it is equal to its complement. The lemma is then
  immediate from the way taking the symplectic complement interacts with sums
  and intersections:
  \[
    ((L\cap S^\circ) +S)^\circ = (L\cap S^\circ)^\circ \cap S^\circ = (L^\circ +
    (S^\circ)^\circ) \cap S^\circ = (L+S) \cap S^\circ = (L \cap S^\circ)+(S
    \cap S^\circ) = (L\cap S^\circ) +S.
  \]
  Since $(L\cap S^\circ) +S$ is equal to its complement, it is Lagrangian.
\end{proof}

The second lemma says that if a subspace of a coisotropic space is Lagrangian,
taking quotients by the complementary isotropic space does not affect this.

\begin{lemma} \label{quotients_of_lagrangians}
  Let $L \subseteq V$ be a Lagrangian subspace of a symplectic vector space $V$,
  and $S \subseteq L$ an isotropic subspace of $V$ contained in $L$. Then $L/S
  \subseteq S^\circ/S$ is Lagrangian in the quotient symplectic space
  $S^\circ/S$.
\end{lemma}
\begin{proof}
  As $L$ is isotropic and the symplectic form on $S^\circ/S$ is given by
  $\omega'(v+S,u+S) = \omega(v,u)$, the quotient $L/S$ is immediately isotropic.
  Recall from Proposition \ref{lagrangian_characterization} that an isotropic
  subspace $S$ of a symplectic vector space $V$ is Lagrangian if and only if
  $\dim S = \frac12 \dim V$. Also recall that for any subspace $\dim S + \dim
  S^\circ = \dim V$. Thus
  \begin{multline*}
    \dim(L/S) = \dim L - \dim S = \tfrac12 \dim V - \dim S \\ = \tfrac12(\dim S
    + \dim S^\circ) - \dim S = \tfrac12(\dim S^\circ - \dim S) = \tfrac12
    \dim(S^\circ/S).
  \end{multline*}
  Thus $L/S$ is Lagrangian in $S^\circ/S$.
\end{proof}

Combining these two lemmas gives a proof that the composite of two Lagrangian
relations is again a Lagrangian relation.

\begin{proof}[Proof of Proposition \ref{prop:lagrangian_composition}]
  Let $\Delta$ be the diagonal subspace
  \[
    \Delta = \{(0,v_2,v_2,0) \mid v_2 \in V_2\} \subseteq \overline{V_1} \oplus
    V_2 \oplus \overline{V_2} \oplus V_3.
  \]
  Observe that $\Delta$ is isotropic, and has coisotropic complement
  \[
    \Delta^\circ = \{(v_1,v_2,v_2,v_3) \mid v_i \in V_i\} \subseteq
    \overline{V_1} \oplus V_2 \oplus \overline{V_2} \oplus V_3.
  \]
  As $\Delta$ is the kernel of the restriction of the projection map
  $\overline{V_1} \oplus V_2 \oplus \overline{V_2} \oplus V_3 \to \overline{V_1}
  \oplus V_3$ to $\Delta^\circ$, and after restriction this map is still
  surjective, the quotient space $\Delta^\circ/\Delta$ is isomorphic to
  $\overline{V_1} \oplus V_3$. 

  Now, by definition of composition of relations, 
  \[
    L' \circ L = \{(v_1,v_3) \mid \mbox{there exists } v_2 \in V_2 \mbox{ such
    that } (v_1,v_2) \in L, (v_2,v_3) \in L'\}.
  \]
  But note also that 
  \[
    L \oplus L'  = \{(v_1,v_2,v_2',v_3) \mid (v_1,v_2) \in L, (v_2',v_3) \in
    L'\},
  \]
  so 
  \[
    (L \oplus L')\cap \Delta^\circ = \{(v_1,v_2,v_2,v_3) \mid \mbox{there exists
    } v_2 \in V_2 \mbox{ such that } (v_1,v_2) \in L, (v_2,v_3) \in L'\}.
  \]
  Quotienting by $\Delta$ then gives
  \[
    L' \circ L = ((L \oplus L')\cap \Delta^\circ)+\Delta)/\Delta.
  \]
  As $L' \oplus L$ is Lagrangian in $\overline{V_1} \oplus V_2 \oplus
  \overline{V_2} \oplus V_3$, Lemma \ref{restriction_of_lagrangians} says that
  $(L' \oplus L)\cap \Delta^\circ)+\Delta$ is also Lagrangian in $\overline{V_1}
  \oplus V_2 \oplus \overline{V_2} \oplus V_3$. Lemma
  \ref{quotients_of_lagrangians} thus shows that $L' \circ L$ is Lagrangian in
  $\Delta^\circ/\Delta = \overline{V_1} \oplus V_3$, as required.
\end{proof}

Note that this composition is associative. We shall see later that this
composition agrees with composition of Dirichlet forms, and hence also
composition of circuits. 

\subsection{The dagger compact category of Lagrangian relations}

Lagrangian relations solve the identity problems we had with Dirichlet forms:
given a symplectic vector space $V$, the Lagrangian relation $\idn\maps V \to V$
specified by the Lagrangian subspace
\[
  \idn = \{(v,v) \mid v \in V\} \subseteq \overline{V} \oplus V,
\]
acts as an identity for composition of relations. We thus have a category.

\begin{definition}
  We write \define{$\LagrRel$} for the category with symplectic
  vector spaces as objects and Lagrangian relations as morphisms. 
\end{definition}

In fact the move to the setting of Lagrangian relations, rather than Dirichlet
forms, adds far richer structure than just identity morphisms. The category
$\LagrRel$ can be viewed as endowed with the structure of a dagger
compact category. We lay this out in steps.

\subsubsection*{Symmetric monoidal structure}

We define the tensor product of two objects of $\LagrRel$ to be their
direct sum. Similarly, we define the tensor product of two morphisms $L\maps U
\to V$, $L \subseteq \overline{U}\oplus V$ and $K\maps T \to W$, $K \subseteq
\overline{T} \oplus W$ to be their direct sum
\[
  L \oplus K \subseteq \overline{U}\oplus V \oplus\overline{T} \oplus W,
\]
\emph{but} considered as a subspace of the naturally isomorphic space
$\overline{U \oplus T} \oplus V \oplus W$.  Despite this subtlety, we abuse our
notation and write their tensor product $L \oplus K\maps U \oplus T \to V \oplus
W$, and move on having sounded this note of caution. 

Note that the direct sum of two Lagrangian subspaces is
again Lagrangian in the direct sum of their ambient spaces, and the zero
dimensional vector space $\{0\}$ acts as an identity for direct sum. Indeed,
defining for all objects $U,V,W$ in $\LagrRel$ unitors: 
\begin{align*}
  \lambda_V &= \{(0,v,v)\} \subseteq \overline{\{0\} \oplus V} \oplus V, \\
  \rho_V &= \{(v,0,v)\} \subseteq \overline{V \oplus \{0\}} \oplus V,
\end{align*}
associators:
\[
  \alpha_{U,V,W}= \{(u,v,w,u,v,w)\} \subseteq \overline{(U \oplus V)\oplus W}
  \oplus U \oplus (V \oplus W),
\]
and braidings:
\[
  \s_{U,V} = \{(u,v,v,u) \mid u \in U, v \in V\} \subseteq \overline{U \oplus V}
  \oplus V \oplus U,
\]
we have a symmetric monoidal category.  Note that all these structure
maps come from symplectomorphisms between the domain and codomain. From this
viewpoint it is immediate that all the necessary diagrams commute, so we
have a symmetric monoidal category. 

\subsubsection*{Duals for objects}

Each object $V$ of $\LagrRel$ is dual to its conjugate space $\overline
V$, with cup $\eta\maps \{0\} \to \overline{V} \oplus V$ given by 
\[
  \eta = \{(0,v,v) \mid v \in V\} \subseteq \overline{\{0\}} \oplus \overline{V}
  \oplus V
\]
and cap $\eps\maps V \oplus \overline{V} \to \{0\}$ given by
\[
  \eps = \{(v,v,0) \mid v \in V\} \subseteq \overline{V \oplus \overline{V}}
  \oplus \{0\}.
\]
It is straightforward to check these satisfy the zigzag identities.

\subsubsection*{Dagger structure}

Given symplectic vector spaces $U,V$, observe that the map
\begin{align*}
  (-)^\dagger\maps \overline{U} \oplus V &\longrightarrow \overline{V} \oplus U; \\
  (u,v) &\longmapsto (v,u)
\end{align*} 
takes Lagrangian subspaces of the domain to Lagrangian subspaces of the
codomain. Thus we can view it as a map $(-)^\dagger$ taking morphisms $L\maps U \to V$
of $\LagrRel$ to morphisms $L^\dagger\maps V \to U$. This defines a
dagger structure on $\LagrRel$, which makes this category into a
symmetric monoidal dagger category.

Moreover, every object in $\mathrm{LagrRel}$ has a dagger dual: it is clear that
$\eta^\dagger = \eps \circ \s$.   This category thus becomes a dagger compact
category.

\subsection{Names of Lagrangian relations} \label{subsec:names}

This brief subsection illustrates a guiding principle of this paper: \emph{duals for
objects allow us to blur the distinction between composition of morphisms and the tensor product of morphisms}. We will make use of this when we prove
the functoriality of the black box functor.

Observe that a Lagrangian relation $L\maps \{0\} \to V$ is the same as a Lagrangian
subspace of $V$. Moreover, given a Lagrangian relation $L\maps U \to V$, 
\[
  \begin{tikzpicture}
    \begin{pgfonlayer}{nodelayer}
      \node [style=none] (0) at (-0.25, 0.25) {};
      \node [style=none] (1) at (0.25, 0.25) {};
      \node [style=none] (2) at (-0.25, -0.25) {};
      \node [style=none] (3) at (0.25, -0.25) {};
      \node [style=none] (4) at (0, 0.75) {};
      \node [style=none] (5) at (0, -0.75) {};
      \node [style=none] (6) at (0, 0.25) {};
      \node [style=none] (7) at (0, -0.25) {};
      \node [style=none] (8) at (0, -0) {$L$};
      \node [style=none] (9) at (0, 1) {$U$};
      \node [style=none] (10) at (0, -1) {$V$};
    \end{pgfonlayer}
    \begin{pgfonlayer}{edgelayer}
      \draw (0.center) to (1.center);
      \draw (1.center) to (3.center);
      \draw (3.center) to (2.center);
      \draw (2.center) to (0.center);
      \draw (4.center) to (6.center);
      \draw (7.center) to (5.center);
    \end{pgfonlayer}
  \end{tikzpicture} 
\]
compactness allows us to view it as a Lagrangian relation $\{0\} \to
\overline{U} \oplus V$:
\[
  \begin{tikzpicture}
    \begin{pgfonlayer}{nodelayer}
      \node [style=none] (0) at (-0.25, 0.25) {};
      \node [style=none] (1) at (0.25, 0.25) {};
      \node [style=none] (2) at (-0.25, -0.25) {};
      \node [style=none] (3) at (0.25, -0.25) {};
      \node [style=none] (4) at (-0.5, 0.75) {};
      \node [style=none] (5) at (0, -0.75) {};
      \node [style=none] (6) at (0, 0.25) {};
      \node [style=none] (7) at (0, -0.25) {};
      \node [style=none] (8) at (0, -0) {$L$};
      \node [style=none] (9) at (-1, -1) {$\overline{U}$};
      \node [style=none] (10) at (0, -1) {$V$};
      \node [style=none] (11) at (-1, -0.75) {};
      \node [style=none] (12) at (-1, 0.25) {};
    \end{pgfonlayer}
    \begin{pgfonlayer}{edgelayer}
      \draw (0.center) to (1.center);
      \draw (1.center) to (3.center);
      \draw (3.center) to (2.center);
      \draw (2.center) to (0.center);
      \draw [bend left=45, looseness=1.00] (4.center) to (6.center);
      \draw (7.center) to (5.center);
      \draw (11.center) to (12.center);
      \draw [bend left=45, looseness=1.00] (12.center) to (4.center);
    \end{pgfonlayer}
  \end{tikzpicture}
\]
We call this subspace the \define{name} of the Lagrangian relation $L$; indeed,
we have used this one-to-one correspondence between morphisms and
their names to define Lagrangian relations.

By compactness, we have the equation
\[
  \begin{aligned}
    \begin{tikzpicture}
      \begin{pgfonlayer}{nodelayer}
	\node [style=none] (0) at (-0.75, 0.5) {};
	\node [style=none] (1) at (-0.25, 0.5) {};
	\node [style=none] (2) at (-0.75, -0) {};
	\node [style=none] (3) at (-0.25, -0) {};
	\node [style=none] (4) at (-1, 1) {};
	\node [style=none] (5) at (-0.5, 0.5) {};
	\node [style=none] (6) at (-0.5, -0) {};
	\node [style=none] (7) at (-0.5, 0.25) {$L$};
	\node [style=none] (8) at (-1.5, -1) {$\overline{U}$};
	\node [style=none] (9) at (-1.5, -0.75) {};
	\node [style=none] (10) at (-1.5, 0.5) {};
	\node [style=none] (11) at (1.5, -0) {};
	\node [style=none] (12) at (1.75, 0.5) {};
	\node [style=none] (13) at (1.5, -1) {$W$};
	\node [style=none] (14) at (0.5, -0) {};
	\node [style=none] (15) at (1.25, -0) {};
	\node [style=none] (16) at (1.75, -0) {};
	\node [style=none] (17) at (1.5, -0.75) {};
	\node [style=none] (18) at (1, 1) {};
	\node [style=none] (19) at (1.5, 0.25) {$M$};
	\node [style=none] (20) at (0.5, 0.5) {};
	\node [style=none] (21) at (1.5, 0.5) {};
	\node [style=none] (22) at (1.25, 0.5) {};
	\node [style=none] (23) at (0, -0.5) {};
      \end{pgfonlayer}
      \begin{pgfonlayer}{edgelayer}
	\draw (0.center) to (1.center);
	\draw (1.center) to (3.center);
	\draw (3.center) to (2.center);
	\draw (2.center) to (0.center);
	\draw [bend left=45, looseness=1.00] (4.center) to (5.center);
	\draw (9.center) to (10.center);
	\draw [bend left=45, looseness=1.00] (10.center) to (4.center);
	\draw (22.center) to (12.center);
	\draw (12.center) to (16.center);
	\draw (16.center) to (15.center);
	\draw (15.center) to (22.center);
	\draw [bend left=45, looseness=1.00] (18.center) to (21.center);
	\draw (11.center) to (17.center);
	\draw (14.center) to (20.center);
	\draw [bend left=45, looseness=1.00] (20.center) to (18.center);
	\draw [bend right=45, looseness=1.00] (6.center) to (23.center);
	\draw [bend right=45, looseness=1.00] (23.center) to (14.center);
      \end{pgfonlayer}
    \end{tikzpicture}
  \end{aligned}
  \qquad
  =
  \qquad
  \begin{aligned}
    \begin{tikzpicture}
      \begin{pgfonlayer}{nodelayer}
	\node [style=none] (0) at (-0.75, 0.5) {};
	\node [style=none] (1) at (-0.25, 0.5) {};
	\node [style=none] (2) at (-0.75, -0) {};
	\node [style=none] (3) at (-0.25, -0) {};
	\node [style=none] (4) at (-1, 1) {};
	\node [style=none] (5) at (-0.5, 0.5) {};
	\node [style=none] (6) at (-0.5, -0) {};
	\node [style=none] (7) at (-0.5, 0.25) {$L$};
	\node [style=none] (8) at (-1.5, -1.25) {$\overline{U}$};
	\node [style=none] (9) at (-1.5, -1) {};
	\node [style=none] (10) at (-1.5, 0.5) {};
	\node [style=none] (11) at (-0.5, -1) {};
	\node [style=none] (12) at (-0.25, -0.25) {};
	\node [style=none] (13) at (-0.5, -1.25) {$W$};
	\node [style=none] (14) at (-0.75, -0.75) {};
	\node [style=none] (15) at (-0.25, -0.75) {};
	\node [style=none] (16) at (-0.5, -0.75) {};
	\node [style=none] (17) at (-0.5, -0.5) {$M$};
	\node [style=none] (18) at (-0.5, -0.25) {};
	\node [style=none] (19) at (-0.75, -0.25) {};
      \end{pgfonlayer}
      \begin{pgfonlayer}{edgelayer}
	\draw (0.center) to (1.center);
	\draw (1.center) to (3.center);
	\draw (3.center) to (2.center);
	\draw (2.center) to (0.center);
	\draw [bend left=45, looseness=1.00] (4.center) to (5.center);
	\draw (9.center) to (10.center);
	\draw [bend left=45, looseness=1.00] (10.center) to (4.center);
	\draw (19.center) to (12.center);
	\draw (12.center) to (15.center);
	\draw (15.center) to (14.center);
	\draw (14.center) to (19.center);
	\draw (11.center) to (16.center);
	\draw (6.center) to (18.center);
      \end{pgfonlayer}
    \end{tikzpicture}
  \end{aligned}
\]
Here the right hand side is the name of the composite $M \circ L$ of Lagrangian
relations, while the left hand side is the direct sum of the names of $L$ and $M$
post-composed with the Lagrangian relation
\[
  \begin{tikzpicture}
    \begin{pgfonlayer}{nodelayer}
      \node [style=none] (0) at (-0.5, -0) {};
      \node [style=none] (1) at (-1.5, -1.25) {$\overline{U}$};
      \node [style=none] (2) at (-1.5, -1) {};
      \node [style=none] (3) at (-1.5, 0.25) {};
      \node [style=none] (4) at (1.5, 0.25) {};
      \node [style=none] (5) at (1.5, -1.25) {$W$};
      \node [style=none] (6) at (0.5, -0) {};
      \node [style=none] (7) at (1.5, -1) {};
      \node [style=none] (8) at (0, -0.5) {};
      \node [style=none] (9) at (-0.5, 0.25) {};
      \node [style=none] (10) at (0.5, 0.25) {};
      \node [style=none] (11) at (-1.5, 0.5) {$\overline{U}$};
      \node [style=none] (12) at (-0.5, 0.5) {$V$};
      \node [style=none] (13) at (0.5, 0.5) {$\overline{V}$};
      \node [style=none] (14) at (1.5, 0.5) {$W$};
    \end{pgfonlayer}
    \begin{pgfonlayer}{edgelayer}
      \draw (2.center) to (3.center);
      \draw (4.center) to (7.center);
      \draw [bend right=45, looseness=1.00] (0.center) to (8.center);
      \draw [bend right=45, looseness=1.00] (8.center) to (6.center);
      \draw (0.center) to (9.center);
      \draw (6.center) to (10.center);
    \end{pgfonlayer}
  \end{tikzpicture}
\]
Thus this relation above, the product of a cap and two identity maps, enacts
composition of Lagrangian relations. The proof of Proposition
\ref{prop:lagrangian_composition}, that the composite of two Lagrangian
relations is again a Lagrangian relation, makes use of this fact. We shall also
return to it when discussing our functor $\Circ \to \LagrRel$.
Note the similarity in form between our diagrams of cospans and of names of
Lagrangian relations. 

\section{Ideal wires and corelations} \label{sec:corel}
%%fakesubsection
In the previous section our exploration of the meaning of circuit diagrams 
culminated with our understanding of behaviors as Lagrangian subspaces.  We now 
turn our attention to how circuit components fit together, and the category of 
operations.  In this section we shall see that the algebra of connections is 
described by the concept of corelations, a generalization of the notion of 
function that forgets the directionality from the domain to the codomain. We
then observe that Kirchhoff's laws follow directly from interpreting these
structures in the category of linear relations.

\subsection{Ideal wires}

To motivate the definition of this category, let us start with a set of input
terminals $X$, and a set of output terminals $Y$.  We may connect these
terminals with ideal wires of zero impedance, whichever way we like---input to
input, output to output, input to output---producing something like:
\[
  \begin{tikzpicture}[circuit ee IEC]
	\begin{pgfonlayer}{nodelayer}
		\node [contact] (0) at (-2, 1) {};
		\node [contact] (1) at (-2, 0.5) {};
		\node [contact] (2) at (-2, -0) {};
		\node [contact] (3) at (-2, -0.5) {};
		\node [contact] (4) at (-2, -1) {};
		\node [contact] (5) at (1, 0.75) {};
		\node [contact] (6) at (1, 0.25) {};
		\node [contact] (7) at (1, -0.25) {};
		\node [contact] (8) at (1, -0.75) {};
		\node [style=none] (9) at (-2.75, -0) {$X$};
		\node [style=none] (10) at (1.75, -0) {$Y$};
	\end{pgfonlayer}
	\begin{pgfonlayer}{edgelayer}
	  \draw [thick] (0.center) to (5.center);
		\draw [thick] (5.center) to (1.center);
		\draw [thick] (6.center) to (1.center);
		\draw [thick] (3.center) to (2.center);
		\draw [thick] (4.center) to (8.center);
		\draw [thick] (5.center) to (6.center);
		\draw [thick] (6.center) to (0.center);
	\end{pgfonlayer}
\end{tikzpicture}
\]
In doing so, we introduce a notion of equivalence on our terminals, where two 
terminals are equivalent if we, or if electrons, can traverse from one to 
another via some sequence of wires.   Because of this, we consider our 
perfectly-conducting components to be equivalence relations on $X+Y$,
transforming the above picture into
\[
  \begin{tikzpicture}[circuit ee IEC]
	\begin{pgfonlayer}{nodelayer}
		\node [contact, outer sep=5pt] (0) at (-2, 1) {};
		\node [contact, outer sep=5pt] (1) at (-2, 0.5) {};
		\node [contact, outer sep=5pt] (2) at (-2, -0) {};
		\node [contact, outer sep=5pt] (3) at (-2, -0.5) {};
		\node [contact, outer sep=5pt] (4) at (-2, -1) {};
		\node [contact, outer sep=5pt] (5) at (1, 0.75) {};
		\node [contact, outer sep=5pt] (6) at (1, 0.25) {};
		\node [contact, outer sep=5pt] (7) at (1, -0.25) {};
		\node [contact, outer sep=5pt] (8) at (1, -0.75) {};
		\node [style=none] (9) at (-2.75, -0) {$X$};
		\node [style=none] (10) at (1.75, -0) {$Y$};
		\node [style=none] (11) at (-0.5, 0.625) {};
		\node [style=none] (12) at (-0.5, -0.25) {};
		\node [style=none] (13) at (-0.5, -0.875) {};
	\end{pgfonlayer}
	\begin{pgfonlayer}{edgelayer}
		\draw [color=gray] (0.center) to (11.center);
		\draw [color=gray] (1.center) to (11.center);
		\draw [color=gray] (5.center) to (11.center);
		\draw [color=gray] (6.center) to (11.center);
		\draw [color=gray] (2.center) to (12.center);
		\draw [color=gray] (12.center) to (3.center);
		\draw [color=gray] (4.center) to (13.center);
		\draw [color=gray] (13.center) to (8.center);
		\draw [rounded corners=5pt, dotted] 
   (node cs:name=0, anchor=north west) --
   (node cs:name=1, anchor=south west) --
   (node cs:name=6, anchor=south east) --
   (node cs:name=5, anchor=north east) --
   cycle;
		\draw [rounded corners=5pt, dotted] 
   (node cs:name=2, anchor=north west) --
   (node cs:name=3, anchor=south west) --
   (node cs:name=3, anchor=south east) --
   (node cs:name=2, anchor=north east) --
   cycle;
		\draw [rounded corners=5pt, dotted] 
   (node cs:name=4, anchor=north west) --
   (node cs:name=4, anchor=south west) --
   (node cs:name=8, anchor=south east) --
   (node cs:name=8, anchor=north east) --
   cycle;
		\draw [rounded corners=5pt, dotted] 
   (node cs:name=7, anchor=north west) --
   (node cs:name=7, anchor=south west) --
   (node cs:name=7, anchor=south east) --
   (node cs:name=7, anchor=north east) --
   cycle;
	\end{pgfonlayer}
\end{tikzpicture}
\]
The dotted lines indicate equivalence classes of points, while for reference the
grey lines indicate ideal wires connecting these points, running through a
central hub.

Given another circuit of this sort, say from sets $Y$ to $Z$,
\[
\begin{tikzpicture}[circuit ee IEC]
	\begin{pgfonlayer}{nodelayer}
		\node [style=none] (0) at (-2.75, -0) {$Y$};
		\node [style=none] (1) at (1.75, 0) {$Z$};
		\node [contact, outer sep=5pt] (2) at (-2, 0.75) {};
		\node [contact, outer sep=5pt] (3) at (-2, 0.25) {};
		\node [contact, outer sep=5pt] (4) at (-2, -0.25) {};
		\node [contact, outer sep=5pt] (5) at (-2, -0.75) {};
		\node [contact, outer sep=5pt] (6) at (1, 1) {};
		\node [contact, outer sep=5pt] (7) at (1, 0.5) {};
		\node [contact, outer sep=5pt] (8) at (1, -0) {};
		\node [contact, outer sep=5pt] (9) at (1, -0.5) {};
		\node [contact, outer sep=5pt] (10) at (1, -1) {};
		\node [style=none] (11) at (-0.5, 0.75) {};
		\node [style=none] (12) at (-0.5, -0.25) {};
		\node [style=none] (13) at (-0.5, -0.875) {};
	\end{pgfonlayer}
	\begin{pgfonlayer}{edgelayer}
	  \draw [color=gray] (2.center) to (11.center);
		\draw [color=gray] (11.center) to (6.center);
		\draw [color=gray] (3.center) to (11.center);
		\draw [color=gray] (8.center) to (12.center);
		\draw [color=gray] (12.center) to (4.center);
		\draw [color=gray] (9.center) to (12.center);
		\draw [color=gray] (5.center) to (13.center);
		\draw [color=gray] (13.center) to (10.center);
	\end{pgfonlayer}
		\draw [rounded corners=5pt, dotted] 
   (node cs:name=2, anchor=north west) --
   (node cs:name=3, anchor=south west) --
   (node cs:name=6, anchor=south east) --
   (node cs:name=6, anchor=north east) --
   cycle;
		\draw [rounded corners=5pt, dotted] 
   (node cs:name=4, anchor=north west) --
   (node cs:name=4, anchor=south west) --
   (node cs:name=9, anchor=south east) --
   (node cs:name=8, anchor=north east) --
   cycle;
		\draw [rounded corners=5pt, dotted] 
   (node cs:name=5, anchor=north west) --
   (node cs:name=5, anchor=south west) --
   (node cs:name=10, anchor=south east) --
   (node cs:name=10, anchor=north east) --
   cycle;
		\draw [rounded corners=5pt, dotted] 
   (node cs:name=7, anchor=north west) --
   (node cs:name=7, anchor=south west) --
   (node cs:name=7, anchor=south east) --
   (node cs:name=7, anchor=north east) --
   cycle;
\end{tikzpicture}
\]
we may combine these circuits in to a circuit $X$ to $Z$
\[
  \begin{aligned}
\begin{tikzpicture}[circuit ee IEC]
	\begin{pgfonlayer}{nodelayer}
		\node [contact, outer sep=5pt] (0) at (1, 0.75) {};
		\node [contact, outer sep=5pt] (1) at (1, 0.25) {};
		\node [contact, outer sep=5pt] (2) at (1, -0.25) {};
		\node [contact, outer sep=5pt] (3) at (1, -0.75) {};
		\node [style=none] (4) at (-2.75, -0) {$X$};
		\node [style=none] (5) at (4.75, -0) {$Z$};
		\node [contact, outer sep=5pt] (6) at (-2, 1) {};
		\node [contact, outer sep=5pt] (7) at (-2, -0.5) {};
		\node [contact, outer sep=5pt] (8) at (-2, 0.5) {};
		\node [contact, outer sep=5pt] (9) at (-2, -0) {};
		\node [contact, outer sep=5pt] (10) at (-2, -1) {};
		\node [contact, outer sep=5pt] (11) at (4, -0) {};
		\node [contact, outer sep=5pt] (12) at (4, -1) {};
		\node [contact, outer sep=5pt] (13) at (4, -0.5) {};
		\node [contact, outer sep=5pt] (14) at (4, 0.5) {};
		\node [style=none] (15) at (-0.5, 0.625) {};
		\node [style=none] (16) at (-0.5, -0.25) {};
		\node [style=none] (17) at (-0.5, -0.875) {};
		\node [style=none] (18) at (2.5, -0.875) {};
		\node [contact, outer sep=5pt] (19) at (4, 1) {};
		\node [style=none] (20) at (1, -1.25) {$Y$};
		\node [style=none] (21) at (2.5, 0.75) {};
		\node [style=none] (22) at (2.5, -0.25) {};
	\end{pgfonlayer}
	\begin{pgfonlayer}{edgelayer}
		\draw [color=gray] (6.center) to (15.center);
		\draw [color=gray] (8.center) to (15.center);
		\draw [color=gray] (0.center) to (15.center);
		\draw [color=gray] (1.center) to (15.center);
		\draw [color=gray] (9.center) to (16.center);
		\draw [color=gray] (7.center) to (16.center);
		\draw [color=gray] (10.center) to (17.center);
		\draw [color=gray] (17.center) to (3.center);
		\draw [color=gray] (3.center) to (18.center);
		\draw [color=gray] (18.center) to (12.center);
		\draw [color=gray] (0.center) to (21.center);
		\draw [color=gray] (1.center) to (21.center);
		\draw [color=gray] (21.center) to (19.center);
		\draw [color=gray] (2.center) to (22.center);
		\draw [color=gray] (22.center) to (11.center);
		\draw [color=gray] (22.center) to (13.center);
		\draw [rounded corners=5pt, dotted] 
   (node cs:name=6, anchor=north west) --
   (node cs:name=8, anchor=south west) --
   (node cs:name=1, anchor=south east) --
   (node cs:name=0, anchor=north east) --
   cycle;
		\draw [rounded corners=5pt, dotted] 
   (node cs:name=9, anchor=north west) --
   (node cs:name=7, anchor=south west) --
   (node cs:name=7, anchor=south east) --
   (node cs:name=9, anchor=north east) --
   cycle;
		\draw [rounded corners=5pt, dotted] 
   (node cs:name=10, anchor=north west) --
   (node cs:name=10, anchor=south west) --
   (node cs:name=3, anchor=south east) --
   (node cs:name=3, anchor=north east) --
   cycle;
		\draw [rounded corners=5pt, dotted] 
   (node cs:name=2, anchor=north west) --
   (node cs:name=2, anchor=south west) --
   (node cs:name=2, anchor=south east) --
   (node cs:name=2, anchor=north east) --
   cycle;
		\draw [rounded corners=5pt, dotted] 
   (node cs:name=0, anchor=north west) --
   (node cs:name=1, anchor=south west) --
   (node cs:name=19, anchor=south east) --
   (node cs:name=19, anchor=north east) --
   cycle;
		\draw [rounded corners=5pt, dotted] 
   (node cs:name=2, anchor=north west) --
   (node cs:name=2, anchor=south west) --
   (node cs:name=13, anchor=south east) --
   (node cs:name=11, anchor=north east) --
   cycle;
		\draw [rounded corners=5pt, dotted] 
   (node cs:name=3, anchor=north west) --
   (node cs:name=3, anchor=south west) --
   (node cs:name=12, anchor=south east) --
   (node cs:name=12, anchor=north east) --
   cycle;
		\draw [rounded corners=5pt, dotted] 
   (node cs:name=14, anchor=north west) --
   (node cs:name=14, anchor=south west) --
   (node cs:name=14, anchor=south east) --
   (node cs:name=14, anchor=north east) --
   cycle;
	\end{pgfonlayer}
\end{tikzpicture}
\end{aligned}
\:
  =
\:
\begin{aligned}
\begin{tikzpicture}[circuit ee IEC]
	\begin{pgfonlayer}{nodelayer}
		\node [style=none] (0) at (-2.75, -0) {$X$};
		\node [style=none] (1) at (1.75, -0) {$Z$};
		\node [contact, outer sep=5pt] (2) at (-2, 1) {};
		\node [contact, outer sep=5pt] (3) at (-2, -0.5) {};
		\node [contact, outer sep=5pt] (4) at (-2, 0.5) {};
		\node [contact, outer sep=5pt] (5) at (-2, -0) {};
		\node [contact, outer sep=5pt] (6) at (-2, -1) {};
		\node [contact, outer sep=5pt] (7) at (1, -0) {};
		\node [contact, outer sep=5pt] (8) at (1, -1) {};
		\node [contact, outer sep=5pt] (9) at (1, -0.5) {};
		\node [contact, outer sep=5pt] (10) at (1, 0.5) {};
		\node [style=none] (11) at (-0.5, 0.875) {};
		\node [style=none] (12) at (-1, -0.25) {};
		\node [contact, outer sep=5pt] (13) at (1, 1) {};
		\node [style=none] (14) at (0, -0.25) {};
	\end{pgfonlayer}
	\begin{pgfonlayer}{edgelayer}
		\draw [color=gray] (2.center) to (11.center);
		\draw [color=gray] (4.center) to (11.center);
		\draw [color=gray] (5.center) to (12.center);
		\draw [color=gray] (3.center) to (12.center);
		\draw [color=gray] (14.center) to (7.center);
		\draw [color=gray] (14.center) to (9.center);
		\draw [color=gray] (6.center) to (8.center);
		\draw [color=gray] (11.center) to (13.center);
		\draw [rounded corners=5pt, dotted] 
   (node cs:name=2, anchor=north west) --
   (node cs:name=4, anchor=south west) --
   (node cs:name=13, anchor=south east) --
   (node cs:name=13, anchor=north east) --
   cycle;
		\draw [rounded corners=5pt, dotted] 
   (node cs:name=5, anchor=north west) --
   (node cs:name=3, anchor=south west) --
   (node cs:name=3, anchor=south east) --
   (node cs:name=5, anchor=north east) --
   cycle;
		\draw [rounded corners=5pt, dotted] 
   (node cs:name=6, anchor=north west) --
   (node cs:name=6, anchor=south west) --
   (node cs:name=8, anchor=south east) --
   (node cs:name=8, anchor=north east) --
   cycle;
		\draw [rounded corners=5pt, dotted] 
   (node cs:name=10, anchor=north west) --
   (node cs:name=10, anchor=south west) --
   (node cs:name=10, anchor=south east) --
   (node cs:name=10, anchor=north east) --
   cycle;
		\draw [rounded corners=5pt, dotted] 
   (node cs:name=7, anchor=north west) --
   (node cs:name=9, anchor=south west) --
   (node cs:name=9, anchor=south east) --
   (node cs:name=7, anchor=north east) --
   cycle;
	\end{pgfonlayer}
\end{tikzpicture}
\end{aligned}
\]
by taking the transitive closure of the two equivalence relations, and then
restricting this to an equivalence relation on $X+Z$. This in fact defines a
dagger compact category of fundamental importance: the category of \emph{corelations}. 

Ellerman gives a detailed treatment of corelations from a logic viewpoint in
\cite{E}, while basic category theoretic aspects can be found in Lawvere and
Rosebrugh \cite{LR}.  However, to make this paper self-contained, we explain them
here.

\subsection{The category of corelations}

In the category of sets we hold the fundamental relationship between sets to be
that of functions. These encode the idea of a deterministic process that takes
each element of one set to a unique element of the other. For the study of
networks this is less appropriate, as the relationship between terminals is not
an input-output one, but rather one of interconnection. 

In particular, the direction of a function becomes irrelevant, and to describe
these interconnections via the category of sets we must develop an understanding
of how to compose functions head to head and tail to tail. We have so far used
cospans and pushouts to address this.  Cospans, however, come with an apex, which
represents extraneous structure beyond the two sets we wish to specify a
relationship between. Corelations arise from omitting this information.

\begin{definition}
  A \define{corelation} $\alpha\maps X \to Y$ between finite sets $X$ and $Y$ is a
partition $\alpha$ of the disjoint union $X+Y$.
\end{definition}

That is, for finite sets $X$ and $Y$, a corelation is a collection of nonempty subsets
$\alpha = \{A_1,A_2,\dots,A_n\}$ of $X+Y$ such that
\begin{enumerate}[(i)] 
  \item $\alpha$ does not contain the empty set.  
  \item $\bigcup_{i=1}^n A_i = X+Y$.
  \item $A_i \cap A_j = \varnothing$ whenever $i \ne j$.
\end{enumerate}

For example, we can take a circuit of ideal wires with $X$ as the set of inputs
and $Y$ as the set of outputs:
\[
  \begin{tikzpicture}[circuit ee IEC]
	\begin{pgfonlayer}{nodelayer}
		\node [contact] (0) at (-2, 1) {};
		\node [contact] (1) at (-2, 0.5) {};
		\node [contact] (2) at (-2, -0) {};
		\node [contact] (3) at (-2, -0.5) {};
		\node [contact] (4) at (-2, -1) {};
		\node [contact] (5) at (1, 0.75) {};
		\node [contact] (6) at (1, 0.25) {};
		\node [contact] (7) at (1, -0.25) {};
		\node [contact] (8) at (1, -0.75) {};
		\node [style=none] (9) at (-2.75, -0) {$X$};
		\node [style=none] (10) at (1.75, -0) {$Y$};
	\end{pgfonlayer}
	\begin{pgfonlayer}{edgelayer}
	  \draw [thick] (0.center) to (5.center);
		\draw [thick] (5.center) to (1.center);
		\draw [thick] (6.center) to (1.center);
		\draw [thick] (3.center) to (2.center);
		\draw [thick] (4.center) to (8.center);
		\draw [thick] (5.center) to (6.center);
		\draw [thick] (6.center) to (0.center);
	\end{pgfonlayer}
\end{tikzpicture}
\]
and define a corelation $\alpha \maps X \to Y$ for which terminals lie in 
the same set of the partition $\alpha = \{A_1,A_2,\dots,A_n\}$ when we can travel from one terminal to another following a path of wires:
\[
  \begin{tikzpicture}[circuit ee IEC]
	\begin{pgfonlayer}{nodelayer}
		\node [contact, outer sep=5pt] (0) at (-2, 1) {};
		\node [contact, outer sep=5pt] (1) at (-2, 0.5) {};
		\node [contact, outer sep=5pt] (2) at (-2, -0) {};
		\node [contact, outer sep=5pt] (3) at (-2, -0.5) {};
		\node [contact, outer sep=5pt] (4) at (-2, -1) {};
		\node [contact, outer sep=5pt] (5) at (1, 0.75) {};
		\node [contact, outer sep=5pt] (6) at (1, 0.25) {};
		\node [contact, outer sep=5pt] (7) at (1, -0.25) {};
		\node [contact, outer sep=5pt] (8) at (1, -0.75) {};
		\node [style=none] (9) at (-2.75, -0) {$X$};
		\node [style=none] (10) at (1.75, -0) {$Y$};
		\node [style=none] (11) at (-0.5, 0.625) {};
		\node [style=none] (12) at (-0.5, -0.25) {};
		\node [style=none] (13) at (-0.5, -0.875) {};
	\end{pgfonlayer}
	\begin{pgfonlayer}{edgelayer}
		\draw [color=gray] (0.center) to (11.center);
		\draw [color=gray] (1.center) to (11.center);
		\draw [color=gray] (5.center) to (11.center);
		\draw [color=gray] (6.center) to (11.center);
		\draw [color=gray] (2.center) to (12.center);
		\draw [color=gray] (12.center) to (3.center);
		\draw [color=gray] (4.center) to (13.center);
		\draw [color=gray] (13.center) to (8.center);
		\draw [rounded corners=5pt, dotted] 
   (node cs:name=0, anchor=north west) --
   (node cs:name=1, anchor=south west) --
   (node cs:name=6, anchor=south east) --
   (node cs:name=5, anchor=north east) --
   cycle;
		\draw [rounded corners=5pt, dotted] 
   (node cs:name=2, anchor=north west) --
   (node cs:name=3, anchor=south west) --
   (node cs:name=3, anchor=south east) --
   (node cs:name=2, anchor=north east) --
   cycle;
		\draw [rounded corners=5pt, dotted] 
   (node cs:name=4, anchor=north west) --
   (node cs:name=4, anchor=south west) --
   (node cs:name=8, anchor=south east) --
   (node cs:name=8, anchor=north east) --
   cycle;
		\draw [rounded corners=5pt, dotted] 
   (node cs:name=7, anchor=north west) --
   (node cs:name=7, anchor=south west) --
   (node cs:name=7, anchor=south east) --
   (node cs:name=7, anchor=north east) --
   cycle;
	\end{pgfonlayer}
\end{tikzpicture}
\]

The discussion in the previous section then motivates a rule for composing corelations. We compose a corelation $\alpha \maps X \to Y$ and a corelation $\beta \maps Y \to Z$ by finding the finest partition on $X+Y+Z$ that is coarser than both $\alpha$ and $\beta$ when restricted to $X+Y$ and $Y+Z$ respectively, and then restricting this to a
partition on $X+Z$. More explicitly, the corelation $\beta\circ\alpha = \{C_1,
C_2, \dots, C_m\}$ is the unique set of pairwise disjoint $C_i$ of the form 
\[
  C_i = \bigcup_j A_j \cap X \cup \bigcup_k B_k \cap Z
\]
for $j,k$ varying over indices such that 
\[
  \bigcup_j A_j \cap Y = \bigcup_k B_k \cap Y.
\]
This rule for composition is associative, as both pairwise methods of composing relations
$\alpha\maps X \to Y$, $\beta\maps Y \to Z$, and $\gamma\maps Z \to W$ amount to finding the finest partition on $X+Y+Z+W$ that is coarser than each of $\alpha$, $\beta$, and $\gamma$ when restricted to the relevant subset, and then restricting this
partition to a partition on $X+W$.  The pictures in the previous section make this clear.

The identity corelation $1_X\maps X \to X$ on a set $X$ is the map $[1_X,1_X]\maps X+X \to X$. Equivalently, it is the partition of $X+X$ such that each partition
comprises exactly two elements: an element $x \in X$ considered as an element of
both the first and then the second summand of $X+X$. We thus define a category:

\begin{definition}
  Let \define{$\mathrm{Corel}$} be the category with finite sets as objects and
  corelations between finite sets as morphisms. 
\end{definition}

This category becomes monoidal under disjoint union of finite sets, using the
fact that the union of partition of $X + Y$ and a partition of $X' + Y'$ is a
partition of $(X+X') + (Y+Y')$. Moreover, the commutativity and associativity of
the coproduct of finite sets also allows implies the category is symmetric
monoidal and indeed dagger compact: the braiding and cups and caps are
respectively given by the partitions of $(X+Y)+(Y+X)$ and $X+X$ where two
elements are in the same part of the partition if and only if they are equal as
elements of $X$ or $Y$, while a dagger is given by simply considering a
partition of $X+Y$ as a partition of $Y+X$..\footnote{In fact, if we take a
  skeleton of $\mathrm{Corel}$ we obtain the PROP for `extraspecial commutative
  Frobenius monoids', as defined by Baez and Erbele \cite{BE}.  We do not need
  this here, but it helps tie our current work to other work on categories in
  electrical engineering.  A proof can be assembled from Baez--Erbele together
  with the work of Lack \cite{La}.}

We can get a corelation $\alpha \maps X \to Y$ from a surjection $f \maps X + Y
\to S$ for some set $S$, by taking the sets in the partition $\alpha = (A_1,
\dots, A_n)$ of $X+Y$ to be the inverse images of the points of $S$.  Two
surjections $f \maps X + Y \to S$, $f \maps X + Y \to S'$ give the same
corelation if and only if there is an isomorphism $g \maps S \to S'$ making the
obvious triangle commute: $f' = g \circ f$.

By the universal property of the coproduct, surjections $X+Y \to A$ are in
one-to-one correspondence with jointly epic cospans $X \rightarrow A \leftarrow
Y$.   Thus, corelations can also be seen as isomorphism classes of jointly epic
cospans.  This lets us compose corelations by composing cospans and applying a
correction.  That is: given corelations $\alpha\maps X \to Y$, $\beta\maps Y \to
Z$, we may choose cospans representing them and compose these cospans.   The
composite cospan may not be jointly epic.  However, we can then replace the apex
of the composite cospan by the joint image of the feet.  The resulting jointly
epic cospan gives the composite corelation $\beta \circ \alpha\maps X \to Z$.
More precisely, we have the following proposition.

\begin{proposition}
  There is a strict symmetric monoidal dagger functor
  \[
    \xymatrix{
      \mathrm{Cospan(FinSet)}  \ar@<.5ex>[r] & \mathrm{Corel} 
    }
  \]
  mapping any cospan to the corelation it defines.
\end{proposition}
\begin{proof}
  This functor maps each finite sets to itself.  We thus need only discuss how
  it acts on morphisms.  A cospan in $\mathrm{FinSet}$ comprises a pair of
  functions $X \stackrel{f}\rightarrow N \stackrel{g}\leftarrow Y$. Restricting
  the apex $N$ down to the joint image $f(X) \cup g(Y)$ gives a jointly epic
  cospan $X \stackrel{f}\rightarrow f(X) \cup g(Y) \stackrel{g}\leftarrow Y$,
  and so a corelation $X \rightarrow Y$. As the elements of the apex not in the
  image of maps from the feet play only a trivial role in the pushout, and hence
  in composition of cospans, this map is functorial. It is now readily observed
  that the functor is symmetric monoidal.
\end{proof}

All this is dual to the more familiar connection between spans and relations,
where a relation is seen as an isomorphism class of jointly monic spans.  This
explains the name `corelation'. Note that neither relations nor corelations are
a generalization of the other.  The key property of corelations here is that it
forms a \emph{compact} category with disjoint union of finite sets as the
monoidal product.  This is not true of the category of relations between finite
sets.


Via the above proposition, the symmetric monoidal dagger functor embedding
$\mathrm{FinSet}$ into $\mathrm{Cospan(FinSet)}$ gives rise to a symmetric
monoidal dagger functor embedding $\mathrm{FinSet}$ into $\mathrm{Corel}$.
Composing this with the dagger structure on $\mathrm{Corel}$, we also obtain a
symmetric monoidal dagger functor embedding $\mathrm{FinSet}^{\mathrm{op}}$ into
$\mathrm{Corel}$.  Corelations thus give a method of composing functions
regardless of the direction of those functions.  Given some not necessarily
directed path of functions, for example
\[
  A \longrightarrow B \longleftarrow C \longleftarrow D \longrightarrow E
  \longrightarrow F,
\]
considering these functions as corelations gives a way to compose them.

In particular, while this mode of composition is simply composition of functions
for two functions head to tail, and turning a cospan into a corelation for two
functions head to head, for two tail to tail functions $C \leftarrow D
\rightarrow E$ we compute the composite of cospans
\[
  \xymatrix{
  & C && E \\
  C \ar@{=}[ur]^{1_C} && D \ar[ul] \ar[ur] && E \ar@{=}[ul]_{1_E}
  }
\]
before restricting the apex to arrive at a surjective function from $C+E$. This
implies the following lemma. 

\begin{lemma} \label{lem:pushoutcorelations}
  Let 
\[
  \xymatrix{
    & P \\
    A \ar[ur] && B \ar[ul] \\
    & N \ar[ur] \ar[ul]
  }
\]
be a pushout square in $\mathrm{FinSet}$. Then the composites of corelations $A
\rightarrow P \leftarrow B$ and $A \leftarrow N \rightarrow B$ are equal.
\end{lemma}

\subsection{Potentials on corelations} \label{ssec:potentialsoncorelations}

Chasing our interpretation of corelations as ideal wires, our aim for the
remainder of this section is to build a functor
\[
  S\maps \mathrm{Corel} \longrightarrow \LagrRel
\]
that expresses this interpretation. We break this functor down into the sum 
of two parts, according to the behaviors of potentials and currents
respectively. 

The consideration of potentials gives a functor $\Phi\maps \mathrm{Corel}
\to \mathrm{LinRel}$, where $\mathrm{LinRel}$ is the symmetric
monoidal dagger category of finite-dimensional $\F$-vector spaces, linear
relations, direct sum, and transpose. In particular, this functor expresses
Kirchhoff's voltage law: it requires that if two elements are in the same part
of the corelation partition---that is, if two nodes are connected by ideal
wires---then the potential at those two points must be the same.

This functor is a generalization of the
contravariant functor $\mathrm{FinSet} \to \mathrm{Vect}$ that maps a set to the
vector space of $\F$-valued functions on that set.

\begin{proposition}
  Define the functor 
  \[ 
    \Phi\maps \mathrm{Corel} \longrightarrow \mathrm{LinRel}, 
  \] 
  on objects by sending a finite set $X$ to the vector space $\F^X$, and on
  morphisms by sending a corelation $\alpha\maps X \to Y$ to the linear subspace
  $\Phi(\alpha)$ of $\F^X \oplus \F^Y$ comprising functions $\phi =
  [\phi_X,\phi_Y]\maps X+Y \to \F$ that are constant on each element of
  $\alpha$.  This is a symmetric monoidal dagger functor.
\end{proposition}

\begin{proof}
  For coherence maps we take the usual natural isomorphisms $\F^X \oplus \F^Y
  \cong \F^{X \times Y}$ and $\{0\} \cong \F^\varnothing$. We detail only the
  proof that $\Phi$ preserves composition; the other properties are
  straightforward to check.

  Let $\alpha\maps X \to Y$ and $\beta\maps Y \to Z$ be corelations. As $\Phi$
  maps corelations to relations, it is enough to check both inclusions
  $\Phi(\beta) \circ \Phi(\alpha) \subseteq \Phi(\beta\circ\alpha)$ and
  $\Phi(\beta\circ\alpha) \subseteq \Phi(\beta) \circ \Phi(\alpha)$. 

  \paragraph{$\Phi(\beta) \circ \Phi(\alpha) \subseteq \Phi(\beta\circ\alpha)$:}
  Let $\phi = [\phi_X,\phi_Z] \in \Phi(\beta) \circ \Phi(\alpha)$. We wish to show
  that for all $C_i \in \beta\circ\alpha$, for all $c,c' \in C_i$, we have
  $\phi(c) = \phi(c')$. To this end, note that there exists some $\phi_Y\maps Y \to \F$
  such that $\phi_{XY} := [\phi_X,\phi_Y] \in \Phi(\alpha)$ and $\phi_{YZ} :=
  [\phi_Y,\phi_Z] \in \Phi(\beta)$.  Furthermore, by definition this $\phi_Y$ has
  the property that for all $A_j \in \alpha$, for all $a,a' \in A_j$, we have
  $\phi_{XY}(a) = \phi_{XY}(a')$, and for all $B_k \in \beta$, for all $b,b' \in
  B_k$, we have $\phi_{YZ}(b) = \phi_{YZ}(b')$. Write $\phi_{XYZ} =
  [\phi_X,\phi_Y,\phi_Z]\maps X+Y+Z \to \F$.

  Our desired fact is thus true: for all $c,c' \in C_i$, there exists a sequence $c=c_0,
  c_1, \dots, c_n=c'$ in $X+Y+Z$ such that for all $\ell=0,1, \dots n-1$ we have
  either $c_\ell, c_{\ell+1} \in A_j$ for some $j$ or $c_\ell,c_{\ell+1} \in B_k$
  for some $k$, and hence that 
  \[
    \phi(c) = \phi_{X+Y+Z}(c_0)= \phi_{X+Y+Z}(c_1) = \dots = \phi_{X+Y+Z}(c_n) =
    \phi(c')
  \]
  as required.

  \paragraph{$\Phi(\beta\circ\alpha) \subseteq \Phi(\beta) \circ \Phi(\alpha)$:}
  Let $\phi = [\phi_X,\phi_Z] \in \Phi(\beta\circ\alpha)$. We must show that there
  exists $\phi_Y\maps Y \to \F$ such that $[\phi_X,\phi_Y] \in \Phi(\alpha)$ and
  $[\phi_Y,\phi_Z] \in \Phi(\beta)$. We claim
  \[
    \phi_Y(y):= \begin{cases}
      \phi(x) & \mbox{if } x \in X, \: x,y \in A_j \mbox{ for some }j;\\
      \phi(z) & \mbox{if } z \in Z, \: y,z \in B_k \mbox{ for some }k;\\
      0 & \mbox{if there exist no such }x \in X \mbox{ or }z \in Z
    \end{cases}
  \]
  satisfies this. This is well-defined as for all $A_j$, $A_j \cap X \subseteq
  C_i$ for some $i$, so $\phi$ is constant on $A_j$, and similarly for all $B_j$.
  Thus it does not matter if there are many such $x$ or $z$ with $x, y \in A_j$ or
  $y,z \in B_k$.  Furthermore, if there exists $y$ such that both $x,y \in A_j$
  for some $x,A_j$ and $y,z \in B_k$ for some $z,B_k$, then the $C_i$ with $A_j
  \cap X \subseteq C_i$ and $B_k \cap Z \subseteq C_i$ is unique, so the
  definitions of $\phi_Y(y)$ do not conflict.

  Moreover, by construction $[\phi_X,\phi_Y]$ is constant on all $A_j$, and
  $[\phi_Y,\phi_Z]$ is constant on all $B_k$, so $[\phi_X,\phi_Y] \in
  \Phi(\alpha)$ and $[\phi_Y,\phi_Z] \in \Phi(\beta)$ as required.
\end{proof}

To recap, we have now constructed a functor $\Phi\maps \mathrm{Corel} \to
\mathrm{LinRel}$ expressing the behavior of potentials on corelations interpreted
as ideal wires. We now do the same for currents.

\subsection{Currents on corelations}

Next we consider the case of currents, described by a functor $I\maps
\mathrm{Corel} \to \mathrm{LinRel}$. This functor now expresses Kirchhoff's
current law: it requires that the sum of currents flowing into each part of the
corelation partition must equal to the sum of currents flowing out.  It may also
be understood as a generalization of the covariant functor $\mathrm{Set} \to
\mathrm{Vect}$ that maps a set to the vector space of $\F$-linear combinations
of elements of that set.

\begin{proposition}
  Define the functor
  \[
    I\maps \mathrm{Corel} \longrightarrow \mathrm{LinRel}.
  \]
  as follows. On objects send a finite set $X$ to the vector space
  $(\F^{X})^\ast$. On morphisms send a corelation $\alpha\maps X \to Y$
  to the linear relation $I(\alpha)$ comprising precisely those 
  \[
    (i_X,i_Y) = \left(\sum_{x \in X} \lambda_xdx,\sum_{y \in Y}
    \lambda_ydy\right)  \in (\F^{X})^\ast \oplus (\F^{Y})^\ast
  \]
  such that for all $A_i \in \alpha$ the sum of the coefficients of the elements
  of $A_i \cap X$ is equal to that for $A_i \cap Y$:
  \[
    \sum_{x \in A_i \cap X} \lambda_x = \sum_{y \in A_i \cap Y} \lambda_y.
  \]
  This is a symmetric monoidal dagger functor.
\end{proposition}
\begin{proof}
The coherence maps are the natural isomorphisms $(\F^X)^\ast \oplus (\F^Y)^\ast
\to (\F^{X+Y})^\ast$ and $\{0\} \to (\F^\varnothing)^\ast$. Again the only
nontrivial task is to check this map $I$ preserves composition. Again we do this
by checking inclusions $I(\beta) \circ I(\alpha) \subseteq I(\beta\circ\alpha)$
and $I(\beta\circ\alpha) \subseteq I(\beta) \circ I(\alpha)$. 

\paragraph{$I(\beta) \circ I(\alpha) \subseteq I(\beta\circ\alpha)$:} Let
$(i_X,i_Z) \in I(\beta)\circ I(\alpha)$, with $i_X = \sum_{x \in X} \lambda_x dx$
and $i_Z = \sum_{z \in Z} \lambda_z dz$. Note that this implies that there is
some $i_Y = \sum_{y \in Y} \lambda_y dy$ such that $(i_X,i_Y) \in I(\alpha)$,
$(i_Y,i_Z) \in I(\beta)$. Then for each $C_i \in \beta\circ\alpha$ we have
\begin{align*}
  \sum_{x \in C_i \cap X} \lambda_x
  &= \sum_{\substack{x \in A_j \cap X \\ A_j \cap X \subseteq C_i}}
  \lambda_x \\
  &= \sum_{\substack{y \in A_j \cap Y \\ A_j \cap X \subseteq C_i}}
  \lambda_y \tag{By definition of $I(\alpha)$}\\
  &= \sum_{\substack{y \in B_k \cap Y \\ B_k \cap Z \subseteq C_i}}
  \lambda_y \tag{See composition of corelations}\\
  &= \sum_{\substack{z \in B_k \cap Z \\ B_k \cap Z \subseteq C_i}}
  \lambda_z \tag{By definition of $I(\beta)$} \\
  &= \sum_{z \in C_i \cap Z} \lambda_z. 
\end{align*}
Thus $(i_X,i_Z) \in I(\beta\circ\alpha)$.

\paragraph{$I(\beta\circ\alpha) \subseteq I(\beta) \circ I(\alpha)$:} The
reverse inclusion requires a bit more effort. Let $(i_X,i_Z) \in
I(\beta\circ\alpha)$. We wish to construct some $i_Y = \sum_{y \in Y} \lambda_y
dy \in (\F^{Y})^\ast$ such that $(i_X,i_Y) \in I(\alpha)$ and $(i_Y,i_Z) \in
I(\beta)$.  This means that we must find a vector $i_Y \in (\F^{Y})^\ast$
satisfying the $\#\alpha$ linear constraints
\[
  \sum_{y \in A_j \cap Y} \lambda_y = \sum_{x \in A_j \cap X} \lambda_x
\]
and the $\#\beta$ linear constraints
\[
  \sum_{y \in B_j \cap Y} \lambda_y = \sum_{z \in B_j \cap Z} \lambda_z.
\]
Note, however, that for each $C_i \in \beta\circ\alpha$, summing over the $A_j$
and $B_k$ that intersect $C_i$ shows a linear dependence between these constraints
themselves, as
\[
  \sum_{\substack{y \in A_j \cap Y \\ A_j \cap X \subseteq C_i}} \lambda_y =
  \sum_{x \in C_i \cap X} \lambda_x = \sum_{z \in C_i \cap Z}
  \lambda_z = \sum_{\substack{y \in B_k \cap Y \\ B_k \cap Z \subseteq C_i}}
  \lambda_y.
\]
Moreover, as each element of $Y$ lies in exactly one element of each $\alpha$
and $\beta$, we may view them as edges in a graph on $\alpha+\beta$, with
exactly $\#(\beta\circ\alpha)$ connected components. This implies that
\[
  \# Y > \#\alpha + \#\beta -\#(\beta\circ\alpha).
\]
Thus these constraints define an affine subspace of $(\F^{Y})^\ast$ of positive
dimension, and hence we can always find a vector $i_Y$ with the desired
property. This proves the claim. 
\end{proof}

Using elementary methods, an algorithm can also
be given to construct an explicit solution.

\subsection{The functor from Corel to LagrRel}

We have now defined functors that, when interpreting corelations as connections
of ideal wires, describe the behaviors of the currents and potentials at the
terminals of these wires. In this section, we combine these to define a single
functor $S\maps \mathrm{Corel} \to \LagrRel$ describing the behavior of both
currents and potentials as a Lagrangian subspace.

\begin{proposition} \label{prop:sympfunctor}
  We define the symplectification functor
  \[
    S\maps \mathrm{Corel} \longrightarrow \LagrRel
  \]
  sending a finite set $X$ to the symplectic vector space
\[
  S(X) = \F^X \oplus (\F^X)^\ast,
\]
 and a corelation $\alpha\maps X \to Y$ to the Lagrangian relation
\[
  S(\alpha) = \Phi(\alpha) \oplus I(\alpha) \subseteq \overline{\F^X \oplus
  (\F^X)^\ast}\oplus \F^Y \oplus (\F^Y)^\ast.
\]
  Then $S$ is a symmetric monoidal dagger functor, with coherence maps inherited
  from $\Phi$ and $I$.
\end{proposition}
\begin{proof}
  As $S$ is the tensor product in $\mathrm{LinRel}$ of the symmetric monoidal
  dagger functors $\Phi, I\maps \mathrm{Corel} \to \mathrm{LinRel}$, it is
  itself a symmetric monoidal dagger functor $\mathrm{Corel} \to
  \mathrm{LinRel}$. Thus it only remains to be checked that, with respect to the
  symplectic structure we put on the objects $S(X)$, the image of each
  corelation $S(\alpha)$ is Lagrangian. 

  This follows from condition (v) of Proposition
  \ref{lagrangian_characterization}: $S(\alpha)$ is (i) isotropic as, for all 
  $(\phi_X,i_X,\phi_Y,i_Y)$, $(\phi_X',i_X',\phi_Y',i_Y') \in S(\alpha)$ we have
  \begin{align*}
    &\phantom{=} \enskip
    \omega\big((\phi_X,i_X,\phi_Y,i_Y),(\phi_X',i_X',\phi_Y',i_Y')\big)
    \\
    &= -\big(i_X'(\phi_X)-i_X(\phi_X')\big) + i_Y'(\phi_Y)-i_Y(\phi_Y') \\
    &= i_X(\phi_X')-i_Y(\phi_Y') + i_X'(\phi_X)-i_Y'(\phi_Y) \\
    &= \sum_{x \in X} \lambda_x dx(\phi_X') - \sum_{y \in Y} \lambda_y dy(\phi_Y') +
    \sum_{y \in Y} \lambda_y' dy(\phi_Y) - \sum_{x \in X} \lambda_x' dx(\phi_X)\\
    &= \sum_{A_j \in \alpha}\left(\sum_{x \in A_j \cap X} \lambda_x dx(\phi_X') -
    \sum_{y \in A_j \cap Y} \lambda_y dy(\phi_Y')\right)+ \sum_{A_j \in
    \alpha}\left(\sum_{x \in A_j \cap X} \lambda_x'dx(\phi_X) - \sum_{y \in A_j
    \cap Y} \lambda_y' dy(\phi_Y)\right)\\
    &= \sum_{A_j \in \alpha}\left(\sum_{x \in A_j \cap X} \lambda_x - \sum_{y \in
    A_j \cap Y} \lambda_y \right)k'_{A_j} + \sum_{A_j \in \alpha}\left(\sum_{x \in
    A_j \cap X} \lambda_x'- \sum_{y \in A_j \cap Y} \lambda_y'\right)k_{A_j}\\
    &= 0,
  \end{align*}
  and (ii) has dimension equal to 
  \[
    \dim(\Phi(\alpha))+ \dim(I(\alpha) = \#\alpha+\#(X+Y) - \#\alpha= \#(X+Y).
  \]
  This proves the proposition.
\end{proof}

We have thus shown we do indeed have a functor $S\maps \mathrm{Corel} \to
\LagrRel$. In the next section we shall see that this functor provides
the engine of our black box functor, playing the key role in showing that we
may indeed treat circuit components as black boxes: that is, that circuits that
behave the same compose the same. Before we get there, we quickly make two
relevant observations.

\begin{example}[Symplectification of functions] \label{ex:sympfunction}
  Let $f: X \to Y$ be a function. In this example we show that $Sf$ has the form 
  \[
    Sf = \big\{(\phi_X,i_X,\phi_Y,i_Y) \, \big\vert \, \phi_X = f^\ast\phi_Y,
    i_Y = f_\ast i_X \big\} \subseteq \overline{\F^X \oplus (\F^X)^\ast} \oplus
    \F^Y \oplus (\F^Y)^\ast,
  \]
  where $f^\ast$ is the pullback map
  \begin{align*}
    f^\ast\maps \F^Y &\longrightarrow \F^X; \\
    \phi &\longmapsto \phi \circ f,
  \end{align*}
  and $f_\ast$ is the pushforward map
  \begin{align*}
    f_\ast\maps (\F^X)^\ast &\longrightarrow (\F^Y)^\ast; \\
    i(-) &\longmapsto i(-\circ f).
  \end{align*}
  (We shall also write $f_\ast$ for the more general map from functions on
  $\F^X$ to functions on $\F^Y$ that takes a function $i(-)$ on $\F^X$ to the
  function $i(-\circ f)$.) The claim is then that these pullback and pushforward
  constructions express Kirchhoff's laws.

  Recall that the corelation corresponding to $f$ partitions $X+Y$ into $\#Y$
  parts, each of the form $f^{-1}(y) \cup \{y\}$. The linear relation $\Phi(f)$
  requires that if $x \in X$ and $y \in Y$ lie in the same part of the
  partition, then they have the same potential: that is, $\phi_X(x) =
  \phi_Y(y)$. This is precisely the arrangement imposed by $f^\ast \phi_Y =
  \phi_X$: 
  \[
    \phi_X(x) = \phi_Y(f(x)) =\phi_Y(y).
  \] 
  On the other hand, the linear relation $I(f)$ requires that if $i_X = \sum_{x 
  \in X}\lambda_xdx$, $i_Y = \sum_{y \in Y}\lambda_y dy$, then for each $y \in Y$
  we have 
  \[
    \sum_{x \in f^{-1}(y)} \lambda_x = \lambda_y.
  \]
  This is precisely what is required by $f_\ast$: given any $\phi \in \F^Y$, we
  have
  \[
    f_\ast i_X(\phi) = f_\ast \sum_{x \in X}\lambda_xdx(\phi) = \sum_{x
    \in X}\lambda_xdx(\phi \circ f)= \sum_{y \in Y}\left( \sum_{x \in f^{-1}(y)}
    \lambda_x\right)dy.
  \]
  This gives us the above representation of $Sf$ when $f$ is a function.
\end{example}

%High road is probably better:
%Recall that given a Lagrangian subspace $L \subseteq V$, its image $f(L)$ 
%under a Lagrangian relation $f\maps V \nrightarrow W$ is a Lagrangian subspace of $W$.  

\begin{remark} \label{rmk:dualitydiagrams}
  We make a remark on our conventions for string diagrams representing the cups
  and caps of the dualities in $\mathrm{Corel}$ and $\LagrRel$. 
  
  Note that by the cap duality diagram
  \[
  \begin{tikzpicture}
	\begin{pgfonlayer}{nodelayer}
		\node [style=none] (0) at (-0.5, 0.5) {};
		\node [style=none] (1) at (0.5, 0.5) {};
		\node [style=none] (2) at (0, -0) {};
		\node [style=none] (3) at (-0.5, 0.75) {$X$};
		\node [style=none] (4) at (0.5, 0.75) {$X$};
	\end{pgfonlayer}
	\begin{pgfonlayer}{edgelayer}
		\draw [in=180, out=-90, looseness=1.00] (0.center) to (2.center);
		\draw [in=-90, out=0, looseness=1.00] (2.center) to (1.center);
	\end{pgfonlayer}
\end{tikzpicture}  
  \]
  in $\mathrm{Corel}$ we mean the corelation $\cup_X\maps (X+X \stackrel{[1,1]}{\rightarrow} X \stackrel{!}{\leftarrow} \varnothing)$, whereas by the cap
  \[
    \begin{tikzpicture}
	\begin{pgfonlayer}{nodelayer}
		\node [style=none] (0) at (-0.5, 0.5) {};
		\node [style=none] (1) at (0.5, 0.5) {};
		\node [style=none] (2) at (0, -0) {};
		\node [style=none] (3) at (-0.5, 0.75) {$V$};
		\node [style=none] (4) at (0.5, 0.75) {$\overline{V}$};
	\end{pgfonlayer}
	\begin{pgfonlayer}{edgelayer}
		\draw [in=180, out=-90, looseness=1.00] (0.center) to (2.center);
		\draw [in=-90, out=0, looseness=1.00] (2.center) to (1.center);
	\end{pgfonlayer}
\end{tikzpicture}
  \]
  in $\LagrRel$ we mean the Lagrangian relation $\cup_V\maps V\oplus \overline{V}
  \to 0$ given by the Lagrangian subspace $\{(v,v) \in \overline{V \oplus
  \overline{V}} \mid v \in V\}$. Although we represent them similarly, the
  functor $S$ does not map these directly onto each other: the image of $\cup_X$
  under $S$ is the Lagrangian relation $S(\cup_X)\maps \F^X \oplus (\F^X)^\ast \oplus
  \F^X \oplus (\F^X)^\ast \to 0$ given by the Lagrangian subspace
  \[
    \{(\phi,i,\phi, -i)\mid \phi \in \F^X, i \in (\F^X)^\ast\} \subseteq
    \overline{\F^X \oplus (\F^X)^\ast \oplus \F^X \oplus (\F^X)^\ast},
  \]
  and in particular a Lagrangian relation $\F^X \oplus (\F^X)^\ast \oplus \F^X
  \oplus (\F^X)^\ast \to 0$ and not $\F^X \oplus (\F^X)^\ast \oplus
  \overline{\F^X \oplus (\F^X)^\ast} \to 0$. As the cup diagrams in these
  categories simply denote the dagger image of these caps, the analogous
  statements apply to them too. 
  
  This is an expression of the fact that in cospan categories and
  $\mathrm{Corel}$ there is a canonical self-duality, while for Lagrangian
  relations a basis must be picked before there is an isomorphism between a
  symplectic vector space and its conjugate, the conjugate being a canonical
  dual object. Nonetheless as the functor $S$ uses the set $X$ to generate a
  symplectic vector space, each object in the image of $S$ has a canonical
  symplectomorphism with its dual $S^t\!X\maps \F^X \oplus (F^X)^\ast \to
  \overline{\F^X \oplus (F^X)^\ast}$, with name the Lagrangian subspace
  $\{(\phi,i,\phi, -i)\mid \phi \in \F^X, i \in (\F^X)^\ast\}$.\footnote{While we could
  treat $S^t\!X$ as an atomic notation, we write it here to evoke the idea of
the concept of `twisted' symplectification introduced in the introduction.} The image of the
  canonical self-duality $\mathrm{Corel}$ is thus given by the composite of this
  symplectomorphism with canonical duality in $\LagrRel$; that is
  \[
    S
    \begin{aligned}
      \begin{tikzpicture}
	\begin{pgfonlayer}{nodelayer}
		\node [style=none] (0) at (-0.5, 0.5) {};
		\node [style=none] (1) at (0.5, 0.5) {};
		\node [style=none] (2) at (0, -0) {};
		\node [style=none] (3) at (-0.5, 1.25) {};
		\node [style=none] (4) at (0.5, 1.25) {};
		\node [style=none] (5) at (-0.5, 1.5) {$X$};
		\node [style=none] (6) at (0.5, 1.5) {$X$};
	  \node[outer sep=0pt, left delimiter=(, right delimiter=),
	  align=center, fit = (0) (1) (2) (3) (4) (5) (6)] {};
	\end{pgfonlayer}
	\begin{pgfonlayer}{edgelayer}
		\draw [in=180, out=-90, looseness=1.00] (0.center) to (2.center);
		\draw [in=-90, out=0, looseness=1.00] (2.center) to (1.center);
		\draw (0.center) to (3.center);
		\draw (1.center) to (4.center);
	\end{pgfonlayer}
\end{tikzpicture}  
\end{aligned}
\qquad 
=
\qquad
\begin{aligned}
\begin{tikzpicture}
	\begin{pgfonlayer}{nodelayer}
		\node [style=none] (0) at (-0.5, 0.5) {};
		\node [style=none] (1) at (0.5, 0.5) {};
		\node [style=none] (2) at (0, 0) {};
		\node [style=none] (3) at (-0.5, 1.25) {};
		\node [style=none] (4) at (0.5, 1.25) {};
		\node [style=none] (5) at (0.25, 1) {};
		\node [style=none] (6) at (0.75, 1) {};
		\node [style=none] (7) at (0.25, 0.5) {};
		\node [style=none] (8) at (0.75, 0.5) {};
		\node [style=none] (9) at (0.5, 1) {};
		\node [style=none] (10) at (0.5, 0.75) {$S^t\!X$};
		\node [style=none] (11) at (-0.5, 1.5) {$S(X)$};
		\node [style=none] (12) at (0.5, 1.5) {$S(X)$};
	\end{pgfonlayer}
	\begin{pgfonlayer}{edgelayer}
		\draw [in=180, out=-90, looseness=1.00] (0.center) to (2.center);
		\draw [in=-90, out=0, looseness=1.00] (2.center) to (1.center);
		\draw (0.center) to (3.center);
		\draw (4.center) to (9.center);
		\draw (5.center) to (6.center);
		\draw (6.center) to (8.center);
		\draw (8.center) to (7.center);
		\draw (7.center) to (5.center);
	\end{pgfonlayer}
\end{tikzpicture}
\end{aligned}
  \]
\end{remark}

\section{Definition of the black box functor} \label{sec:blackbox}
%%fakesection
We have now developed enough machinery to prove Theorem \ref{main_theorem}:
there is a symmetric monoidal dagger functor, the black box functor
\[  
\blacksquare\maps \Circ \to \LagrRel 
\]
taking passive linear circuits to their behaviors. To recap, we have so far
developed two categories: $\Circ$, in which morphisms are passive linear
circuits, and $\LagrRel$, which captures the external behavior of such circuits.
We now define a functor that maps each circuit to its behavior, before proving
it is indeed a symmetric monoidal dagger functor. 

The role of the functor we construct here is to identify all circuits with the
same external behavior, making the internal structure of the circuit
inaccessible. Circuits treated this way are frequently referred to as
`black boxes', so we call this functor the \define{black box functor},
\[
\blacksquare\maps \Circ \to \LagrRel.
\] 
In this section we first provide the definition of this functor, and then check
that our definition really does map a circuit to its behavior.

\subsection{Definition}
It should be no surprise that the black box functor maps a finite set $X$ to the
symplectic vector space $\F^X \oplus (\F^X)^\ast$ of potentials and currents
that may be placed on that set. The challenge is to provide a succinct
statement of its action on the circuits themselves. To do this, we take
advantage of four processes we developed in Parts  and
.

Let $\Gamma\maps X \to Y$ be a circuit, represented by the decorated cospan
\[
  \big(X \stackrel{i}{\longrightarrow} N \stackrel{o}{\longleftarrow} Y,\:
  \Gamma\big).
\]
Recall that this means that $X$ and $Y$ are finite sets, $\Gamma$ is a
$\F$-graph $(N,E,s,t,r)$, and we have a cospan of finite sets
\[
  \xymatrix{
    & \Gamma \\
    X \ar[ur]^{i} && Y. \ar[ul]_{o}
  }
\]
To define the image of $\Gamma$ under our functor $\blacksquare$, by definition
a Lagrangian relation $\blacksquare(\Gamma): \blacksquare(X) \to
\blacksquare(Y)$, we must specify a Lagrangian subspace 
\[
  \blacksquare(\Gamma) \subseteq \overline{\F^X \oplus (\F^X)^\ast} \oplus \F^Y
  \oplus (\F^Y)^\ast.  
\]

Recall that to each $\F^+$-graph $\Gamma$ we can associate a Dirichlet form, 
the extended power functional 
\begin{align*}
  P_\Gamma\maps \F^N &\longrightarrow \F; \\
  \phi &\longmapsto \frac{1}{2} \sum_{e \in E} \frac{1}{r(e)} \big( \phi(t(e)) -
  \phi(s(e))  \big)^2,
\end{align*}
and to this Dirichlet form we associate a Lagrangian subspace 
\[
  \mathrm{Graph}(dP_\Gamma) = \{(\phi,d(P_\Gamma)_\phi) \mid \phi \in \F^N\}
  \subseteq \F^N \oplus (\F^N)^\ast.
\]
We consider this Lagrangian subspace as a Lagrangian relation $\{0\} \to \F^N
\oplus (\F^N)^\ast$.

From the legs of the cospan $\Gamma$, the symplectification functor $S$ gives the
Lagrangian relation
\[
  S[i,o]^\dagger\maps \F^N \oplus (\F^N)^\ast \longrightarrow \F^X \oplus
  (\F^X)^\ast \oplus \F^Y \oplus (\F^Y)^\ast.
\]
Writing $Y: Y \to Y$ for the identity morphism on the finite set $Y$, $S$ also
provides a way of writing the identity morphism $SY: \F^Y \oplus (\F^Y)^\ast \to
\F^Y \oplus (\F^Y)^\ast$. 

Lastly, we have the symplectomorphism
\begin{align*}
  S^t\!X\maps \F^X \oplus (\F^X)^\ast &\longrightarrow \overline{\F^X \oplus
  (\F^X)^\ast}; \\
  (\phi,i) &\longmapsto (\phi,-i).
\end{align*}

The black box functor maps a circuit $\Gamma$ to the Lagrangian relation
\[
  (S^t\!X\oplus SY) \circ S[i,o]^\dagger \circ \mathrm{Graph}(P_\Gamma).
\]

As isomorphisms of cospans of $\F^+$-graphs amount to no more than a 
relabelling of nodes and edges, this construction is independent of the cospan 
chosen as representative of the isomorphism class of cospans forming the 
circuit.

We picture this as
\[
  \blacksquare(\Gamma) =\qquad 
  \begin{aligned}
\begin{tikzpicture}
	\begin{pgfonlayer}{nodelayer}
		\node [style=none] (0) at (-1.25, 0.5) {};
		\node [style=none] (1) at (-0.75, 0.5) {};
		\node [style=none] (2) at (-1.25, -0) {};
		\node [style=none] (3) at (-0.75, -0) {};
		\node [style=none] (4) at (-1.25, 1.25) {};
		\node [style=none] (5) at (-1, -0.25) {};
		\node [style=none] (6) at (-1, 0.5) {};
		\node [style=none] (7) at (-1, -0) {};
		\node [style=none] (8) at (0, -0.25) {};
		\node [style=none] (9) at (0, 0.75) {};
		\node [style=none] (10) at (-0.5, 2) {};
		\node [style=none] (11) at (-1, 1.5) {};
		\node [style=none] (12) at (0, 1.5) {};
		\node [style=none] (13) at (-0.5, 1.5) {};
		\node [style=none] (14) at (-1.25, 0.75) {};
		\node [style=none] (15) at (0.25, 1.25) {};
		\node [style=none] (16) at (0.25, 0.75) {};
		\node [style=none] (17) at (-1, 0.75) {};
		\node [style=none] (18) at (-0.5, 1.25) {};
		\node [style=none] (19) at (-0.5, 1.75) {$L_\Gamma$};
		\node [style=none] (20) at (-0.5, 1) {$S[i,o]^\dagger$};
		\node [style=none] (21) at (-1, 0.25) {$S^t\! X$};
		\node [style=none] (22) at (-1.5, -0.75) {};
		\node [style=none] (23) at (-2, -0.25) {};
		\node [style=none] (24) at (-2, 1.5) {};
		\node [style=none] (25) at (-1, 2.5) {};
		\node [style=none] (26) at (-1, -1.25) {};
		\node [style=none] (27) at (-1, 2.75) {$\scriptstyle\blacksquare(X)$};
		\node [style=none] (28) at (-1, -1.5) {$\scriptstyle\blacksquare(Y)$};
	\end{pgfonlayer}
	\begin{pgfonlayer}{edgelayer}
		\draw (0.center) to (1.center);
		\draw (1.center) to (3.center);
		\draw (3.center) to (2.center);
		\draw (2.center) to (0.center);
		\draw (7.center) to (5.center);
		\draw (8.center) to (9.center);
		\draw (10.center) to (11.center);
		\draw (11.center) to (12.center);
		\draw (12.center) to (10.center);
		\draw (4.center) to (14.center);
		\draw (14.center) to (16.center);
		\draw (16.center) to (15.center);
		\draw (15.center) to (4.center);
		\draw (13.center) to (18.center);
		\draw (17.center) to (6.center);
		\draw [in=90, out=-90, looseness=1.00] (8.center) to (26.center);
		\draw [bend left=45, looseness=1.00] (5.center) to (22.center);
		\draw [bend left=45, looseness=1.00] (22.center) to (23.center);
		\draw (23.center) to (24.center);
		\draw [in=-90, out=90, looseness=1.00] (24.center) to (25.center);
	\end{pgfonlayer}
\end{tikzpicture}
\end{aligned}
\]

To summarize:

\begin{definition}
  We define the black box functor 
  \[
    \blacksquare\maps \Circ \to \LagrRel 
  \]
  on objects by mapping a finite set $X$ to the symplectic vector space 
  \[
    \blacksquare(X) = \F^X \oplus (\F^X)^\ast.
  \]
  and on morphisms by mapping a circuit $\Gamma\maps X \to Y$, represented by the
  decorated cospan
  \[
    \big(X \stackrel{i}{\longrightarrow} N \stackrel{o}{\longleftarrow} Y,\:
    \Gamma\big)
  \]
  to the Lagrangian relation
  \[
    \blacksquare(\Gamma) = (S^t\!X \oplus SY) \circ S[i,o]^\dagger \circ
    \mathrm{Graph}(dP_\Gamma).
  \]
\end{definition}

The coherence maps are given by the natural isomorphisms
\[
  \blacksquare(X) \oplus \blacksquare(Y) = \F^X \oplus (\F^X)^\ast \oplus \F^Y
  \oplus (\F^Y)^\ast \cong \F^{X+Y} \oplus (\F^{X+Y})^\ast = \blacksquare(X+Y)
\]
and
  \[
    \{0\} \cong \F^\varnothing \oplus (\F^\varnothing)^\ast =
    \blacksquare(\varnothing).
  \]


\begin{theorem} \label{thm:main}
  The black box functor is a well-defined symmetric monoidal dagger functor.
\end{theorem}

The next, and final, section is devoted to the proof of this theorem. Before we
get there, we first assure ourselves that we have indeed arrived at the theorem
we set out to prove.

\subsection{Minimization via composition of relations}

At this point the reader might voice two concerns: firstly, why does the
\emph{black box} functor refer to the \emph{extended} power functional $P$ and,
secondly, since it fails to talk about power minimization, how is it the same
functor as that defined in Theorem \ref{main_theorem}? These fears are allayed
by the remarkable trinity of minimization, the symplectification of functions,
and Kirchhoff's laws. 

We have seen that symplectification of functions views the cograph of the
function as a picture of ideal wires, governed by Kirchhoff's laws (Example
\ref{ex:sympfunction}). We have also seen that Kirchhoff's laws are closely
related to the principle of minimium power (Theorems
\ref{thm:realizablepotentials} and \ref{thm:dirichletminimization}). The final
aspect of this relationship is that we may use symplectification of functions to
enact minimization.

\begin{theorem} \label{thm:sympmin}
  Let $\iota: \partial N \to N$ be an injection, and let $P$ be a Dirichlet form on
  $N$. Write $Q = \min_{N \setminus \partial N} P$ for the Dirichlet form on
  $\partial N$ given by minimization over $N \setminus \partial N$. Then we have an
  equality of Lagrangian subspaces
  \[
    S\iota^\dagger \big( \mathrm{Graph}(dP)\big) = \mathrm{Graph}(dQ).
  \]
\end{theorem}
\begin{proof}
  Recall from Example \ref{ex:sympfunction} that $S\iota^\dagger$ is the Lagrangian relation
  \[
    S\iota^\dagger = \big\{(\phi, \iota_\ast i,\phi \circ \iota, i) \, \big\vert
      \, \phi \in \F^{N}, i \in (\F^{\partial N})^\ast \big\} \subseteq
      \overline{\F^N \oplus (\F^N)^\ast} \oplus \F^{\partial N} \oplus
      (\F^{\partial N})^\ast,
  \]
  where $\iota_\ast i(\phi) = i(\phi \circ \iota)$, and note that 
  \[
    \mathrm{Graph}(dP) = \big\{(\phi,dP_\phi) \,\big\vert\, \phi \in \F^N\big\}.
  \]
  This implies that their composite is given by the set
  \[
    S\iota^\dagger \circ \mathrm{Graph}(dP) = \big\{(\phi \circ \iota, i)
    \,\big\vert\, \phi \in \F^N, i \in (\F^{\partial N})^\ast, dP_\phi =
  \iota_\ast i \big\}.
  \]
  We must show this Lagrangian subspace is equal to $\mathrm{Graph}(dQ)$.
  
  Consider the constraint $dP_\phi = \iota_\ast i$. This states that for all
  $\varphi \in \F^N$ we have $dP_\phi(\varphi) = i(\varphi\circ \iota)$. Letting
  $\chi_n: N \to \F$ be the function sending $n \in N$ to $1$ and all other
  elements of $N$ to $0$, we see that when $n \in N \setminus \partial N$ we
  must have
  \[
    \frac{dP}{d\varphi(n)}\Bigg\vert_{\varphi = \phi}  = dP_\phi(\chi_n) = 
    i(\chi_n \circ \iota) = i(0) = 0.
  \]
  So $\phi$ must be a realizable extension of $\psi = \phi \circ \iota$. We
  henceforth write $\tilde\psi = \phi$. As $\iota$ is injective, $\psi = \phi
  \circ \iota$ gives no constraint on $\psi \in \F^{\partial N}$. 
 
  We next observe that we can write $S\iota^\dagger \circ \mathrm{Graph}(dP) =
  \mathrm{Graph}(dO)$ for some quadratic form $O$. Recall that Proposition
  \ref{prop:qfls} states that a Lagrangian subspace $L$ of $\F^{\partial N}
  \oplus (\F^{\partial N})^\ast$ is of the form $\mathrm{Graph}(dO)$ if and only
  if $L$ has trivial intersection with $\{0\} \oplus (\F^N)^\ast$. But indeed,
  if $\psi = 0$ then $0$ is a realizable extension of $\psi$, so $\iota_\ast i =
  dP_0 = 0$, and hence $i = 0$. 
  
  It remains to check that $O = Q$. This is a simple computation:
  \[
    O(\psi) = dO_\psi(\psi) = dO_\psi(\tilde\psi \circ \iota) = \iota_\ast
    dQ_\psi(\tilde\psi) = dP_{\tilde\psi}(\tilde\psi) = P(\tilde\psi) = Q(\psi),
  \]
  where $\tilde\psi$ is any realizable extension of $\psi \in \F^{\partial N}$.
\end{proof}

Write $\iota: \partial N \to N$ for the inclusion of the terminals into the set
of nodes of the circuit, and $i\rvert^{\partial N}: X \to \partial N$,
$o\rvert^{\partial N}: Y \to \partial N$ for the respective corestrictions of
the input and output map to $\partial N$. Note that $[i,o] = \iota \circ
[i\rvert^{\partial N}, o\rvert^{\partial N}]$.
Then we have the equalities of sets, and thus Lagrangian relations:
\begin{align*}
  (S^t\!X\oplus 1_Y) \circ S[i,o]^\dagger \circ \mathrm{Graph}(dP_\Gamma)
  &= (S^t\!X\oplus SY) \circ S[i\rvert^{\partial
  N},o\rvert^{\partial N}]^\dagger \circ S\iota^\dagger \circ \mathrm{Graph}(dP_\Gamma) \\
  &= (S^t\!X\oplus SY) \circ S[i\rvert^{\partial
  N},o\rvert^{\partial N}]^\dagger \circ \mathrm{Graph}(dQ_\Gamma) \\
  &= \bigcup_{v \in \mathrm{Graph}(dQ)} S^ti\rvert^{\partial N}(v) \times
  So\rvert^{\partial N}(v)
\end{align*}
We see now that Theorem \ref{thm:main} is a restatement of Theorem
\ref{main_theorem} in the introduction.

\section{Proof of functoriality} \label{sec:proof}
%%fakesubsection
To prove that the black box construction is indeed functorial, we
factor it into three functors. These functors are each symmetric monoidal dagger
functors, so the black box functor is too.

We first make use, twice, of our results on decorated cospans, showing the
existence of categories of cospans decorated by Dirichlet forms and then
Lagrangian subspaces, and the existence of functors from the category of
circuits to each of these. This proceeds by defining functors $\mathrm{Dirich}$
and $\mathrm{Lagr}$ to construction decorated cospan categories, and then by
defining the relevant natural transformations to construct the desired decorated
cospan functors.

The third functor takes cospans decorated by Lagrangian subspaces, and black
boxes them to give Lagrangian relations between the feet of such cospans. The
functoriality of this process relies on interpreting corelations as Lagrangian
relations.

This gives a factorization
\[
  \xymatrix{
    \blacksquare\maps \Circ \ar[r] & \mathrm{DirichCospan} \ar[r] &
    \mathrm{LagrCospan} \ar[r] & \LagrRel
  }
\]
The following subsections deal with these functors in sequence, defining and
proving the existence of each one using the techniques of Part
.


\subsection{From circuits to Dirichlet cospans}
To begin, we define a category of cospans of finite sets decorated by Dirichlet
forms. This might be seen as a solution to our sought after composition rule for
Dirichlet forms---but not quite, as it also requires the additional data of a
cospan. Nonetheless, it has the property that there is a functor from the
category of circuits to this so-called category of Dirichlet cospans.

\subsubsection*{The category of Dirichlet cospans} 

\begin{proposition}
Let
\[
  \mathrm{Dirich}\maps (\mathrm{FinSet},+) \longrightarrow (\mathrm{Set},\times)
\]
map a finite set $X$ to the set $\mathrm{Dirich}(X)$ of Dirichlet forms on $X$,
and map a function $f\maps X \to Y$ between finite sets to the pushforward function
\begin{align*}
  \mathrm{Dirich}(f)\maps \mathrm{Dirich}(X) &\longrightarrow \mathrm{Dirich}(Y); \\
  Q &\longmapsto f_{\ast}Q,
\end{align*}
where $f_\ast Q$ maps $\phi \in \F^Y$ to $Q(\phi \circ f))$. This defines a
functor.

Moreover, equipping this functor with the family of maps
\begin{align*}
  \delta_{N,M}\maps \mathrm{Dirich}(N) \times \mathrm{Dirich}(M) &\longrightarrow
  \mathrm{Dirich}(N+M);\\
  (Q_N,Q_M) &\longmapsto Q_N(-\circ \iota_N) +Q_M(-\circ \iota_M),
\end{align*}
where $\iota_N\maps N \to N+M$, $\iota_M\maps M \to N+M$ are the injections for
the coproduct, and with unit
\begin{align*}
  \delta_1\maps 1 &\longrightarrow \mathrm{Dirich}(\varnothing);\\
  \bullet &\longmapsto (\F^\varnothing \to \F; \varnothing \mapsto 0)
\end{align*}
defines a lax symmetric monoidal functor.
\end{proposition}

\begin{proof}
  Functoriality is just the associativity of composition of functions;
  lax symmetric monoidality is the associativity, unitality, and commutativity
  of addition of Dirichlet forms.
\end{proof}
  
By decorated cospans, we thus obtain a category $\mathrm{DirichCospan}$
where a morphism is a cospan of finite sets whose apex is equipped with a 
Dirichlet form.  In particular, note that we have overcome our inability to define a category whose morphisms are Dirichlet forms.   An identity morphism in $\mathrm{DirichCospan}$ is a circuit whose set of inputs is equal to its set of
outputs, with all of its nodes being terminals, and no edges:
\[
  \begin{tikzpicture}[circuit ee IEC, set resistor graphic=var resistor iec graphic]
    \node[contact] (c1) at (0,2) {};
    \node[contact] (c2) at (0,0) {};
    \node(input) at (-2,1) {\small{\textsf{inputs}}};
    \node(output) at (2,1) {\small{\textsf{outputs}}};
    \path[color=gray, very thick, shorten >=10pt, ->, >=stealth, bend left]
    (input) edge (c1);		
    \path[color=gray, very thick, shorten >=10pt, ->, >=stealth, bend right]
    (input) edge (c2);	
    \path[color=gray, very thick, shorten >=10pt, ->, >=stealth, bend right]
    (output) edge (c1);
    \path[color=gray, very thick, shorten >=10pt, ->, >=stealth, bend left]
    (output) edge (c2);
  \end{tikzpicture}
\]

\subsubsection*{The functor $\Circ \to \mathrm{DirichCospan}$}

We have now constructed two symmetric monoidal functors
\[
  (\mathrm{Circuit},\rho), (\mathrm{Dirich},\delta)\maps (\mathrm{FinSet},+) \longrightarrow (\mathrm{Set},\times)
\]
that describe the circuit structures and Dirichlet forms we may put on the set
respectively. We have also seen, motivating our discussion of Dirichlet forms,
that from any circuit we can obtain a Dirichlet form describing the power usage
of that circuit. Now we shall see that this process respects the tensor product.
More precisely, it specifies a monoidal natural transformation between
$\mathrm{Circuit}$ and $\mathrm{Dirich}$. By the decorated cospan construction,
this gives us a strict symmetric monoidal dagger
functor $\Circ \to \mathrm{DirichCospan}$.

\begin{proposition}
  Let
  \[
    \alpha\maps (\mathrm{Circuit},\rho) \Longrightarrow (\mathrm{Dirich},\delta)
  \]
  be the collection of functions
  \begin{align*}
    \alpha_N\maps \mathrm{Circuit}(N) &\longrightarrow \mathrm{Dirich}(N); \\
    (N,E,s,t,r) &\longmapsto \left(\phi \in \F^N \mapsto \frac{1}{2} \sum_{e \in E}
    \frac{1}{r(e)}\big(\phi(s(e))-\phi(t(e))\big)^2\right).
  \end{align*}
  Then $\alpha$ is a monoidal natural transformation.
\end{proposition}

\begin{proof}
Naturality requires that the square
\[
\xymatrix{
  \mathrm{Circuit}(N) \ar[r]^{\alpha_N} \ar[d]_{\mathrm{Circuit}(f)} &
  \mathrm{Dirich}(N) \ar[d]^{\mathrm{Dirich}(f)}  \\
  \mathrm{Circuit}(M) \ar[r]_{\alpha_M} & \mathrm{Dirich}(M)
}
\]
commutes. Let $(N,E,s,t,r)$ be an $\F^+$-graph on $N$ and $f\maps N \to M$ be a
function $N$ to $M$. Then both $\mathrm{Dirich}(f) \circ \alpha_N$ and $\alpha_M
\circ \mathrm{Circuit}(f)$ map $(N,E,s,t,r)$ to the Dirichlet form
\begin{align*}
  \F^M &\longrightarrow \F;\\
  \psi &\longmapsto \frac{1}{2} \sum_{e \in E}\frac{1}{r(e)}
  \big(\psi(f(s(e)))-\psi(f(t(e)))\big)^2.
\end{align*}
Thus both methods of constructing a power functional on a set of nodes $M$ from
a circuit on $N$ and a function $N \to M$ produce the same power functional.

To show that $\alpha$ is a monoidal natural transformation, we must check that
the square
\[
\xymatrix{
  \mathrm{Circuit}(N) \times \mathrm{Circuit}(M) \ar[r]^{\alpha_N \times
  \alpha_M} \ar[d]_{\rho_{N,M}} & \mathrm{Dirich}(N) \times \mathrm{Dirich}(M)
  \ar[d]^{\delta_{N,M}}  \\
  \mathrm{Circuit}(N+M) \ar[r]_{\alpha_{N+M}} & \mathrm{Dirich}(N+M)
}
\]
and the triangle
\[
\xymatrix{
  & 1 \ar[dl]_{\rho_\varnothing} \ar[dr]^{\delta_\varnothing}\\
\mathrm{Circuit}(\varnothing)  \ar[rr]_{\alpha_\varnothing} &&
\mathrm{Dirich}(\varnothing)
}
\]
commute. It is readily observed that both paths around the square lead to taking
two graphs and summing their corresponding Dirichlet forms, and that the
triangle commutes immediately as all objects in it are the one element set.
\end{proof}

From decorated cospans, we thus obtain a strict symmetric monoidal dagger functor
\[
  Q \maps \Circ = \mathrm{CircuitCospan} \longrightarrow \mathrm{DirichCospan}.
\]
Roughly, this says that the process of composition for circuit diagrams is the
same as that of composition for Dirichlet cospans. Note that although this
functor preserves much of the information in circuit diagrams, it is not a
faithful functor.  For example, applying $Q$ to a circuit with edges $e,e'$ from
the node $m$ to the node $n$ of resistance $r_e$ and $r_{e'}$ respectively, we
obtain the same result as for the circuit with just one edge $e''$ from $m$ to
$n$ whose resistance $r_{e''}$ is given by
\[
           \frac{1}{r_{e''}} = \frac{1}{r_e} + \frac{1}{r_{e'}}  .
\]

\subsection{From Dirichlet cospans to Lagrangian cospans}

The next step is to show that our process of turning a Dirichlet form into a Lagrangian subspace---by taking the graph of its differential---is functorial.

\subsubsection*{The category of Lagrangian cospans}

We begin by describing a category where morphisms are cospans decorated by
Lagrangian subspaces.

\begin{proposition}
Define 
\[
  \mathrm{Lagr}\maps (\mathrm{FinSet},+) \longrightarrow (\mathrm{Set},\times)
\]
as follows. For objects let $\mathrm{Lagr}$ map a finite set $X$ to the set
$\mathrm{Lagr}(X)$ of Lagrangian subspaces of the symplectic vector space
$\vectf{X}$.  For morphisms, recall that a function $f\maps X \to Y$ between
finite sets may be considered as a corelation, and the symplectification functor
$S$ thus maps this corelation to some Lagrangian relation $Sf\maps \vectf{X}
\to \vectf{Y}$. As Lagrangian relations map Lagrangian subspaces to Lagrangian
subspaces (Proposition \ref{prop:lagrangian_composition}), this gives a map: 
\begin{align*}
  \mathrm{Lagr}(f)\maps \mathrm{Lagr}(X) &\longrightarrow \mathrm{Lagr}(Y); \\
  L &\longmapsto Sf(L).
\end{align*}
Moreover, equipping this functor with the family of maps
\begin{align*}
  \lambda_{N,M}\maps \mathrm{Lagr}(N) \times \mathrm{Lagr}(M) &\longrightarrow
  \mathrm{Lagr}(N+M);\\
  (L_N,L_M) &\longmapsto L_N \oplus L_M,
\end{align*}
and unit
\begin{align*}
  \lambda_1\maps 1 &\longrightarrow \mathrm{Lagr}(\varnothing);\\
  \bullet &\longmapsto \{0\}
\end{align*}
defines a lax symmetric monoidal functor.
\end{proposition}
\begin{proof}
  The functoriality of this construction follows from the functoriality of $S$;
  the lax symmetric monoidality from the relevant properties of the direct sum
  of vector spaces.
\end{proof}
We thus obtain a dagger compact category $\mathrm{LagrCospan}$.

\subsubsection*{The functor $\mathrm{DirichCospan} \to \mathrm{LagrCospan}$}

We now wish to construct a strict symmetric monoidal dagger functor
$\mathrm{DirichCospan} \to \mathrm{LagrCospan}$.  For this we need a monoidal natural transformation 
\[
  (\mathrm{Dirich},\delta), (\mathrm{Lagr},\lambda)\maps (\mathrm{FinSet},+)
  \longrightarrow (\mathrm{Set},\times).
\]
\begin{proposition}
Let
\[
  \beta\maps (\mathrm{Dirich},\delta) \Longrightarrow (\mathrm{Lagr},\lambda)
\]
be the collection of functions
\begin{align*}
  \beta_N\maps \mathrm{Dirich}(N) &\longrightarrow \mathrm{Lagr}(N); \\
  Q &\longmapsto \{(\phi,dQ_\phi) \mid \phi \in \F^N\} \subseteq \vectf{N}.
\end{align*}
Then $\beta$ is a monoidal natural transformation.
\end{proposition}

\begin{proof}
Naturality requires that the square
\[
\xymatrix{
  \mathrm{Dirich}(N) \ar[r]^{\beta_N} \ar[d]_{\mathrm{Dirich}(f)} &
  \mathrm{Lagr}(N) \ar[d]^{\mathrm{Lagr}(f)}  \\
  \mathrm{Dirich}(M) \ar[r]_{\beta_M} & \mathrm{Lagr}(M)
}
\]
commutes for every function $f\maps N \to M$. This is primarily a consequence of the
fact that the differential commutes with pullbacks. As we did in Example
\ref{ex:sympfunction},
write $f^\ast$ for the pullback map and $f_\ast$ for the pushforward map.
Then $\mathrm{Dirich}(f)$ maps a Dirichlet form $Q$ on $N$ to the form $f_\ast Q$,
and $\beta_M$ in turn maps this to the Lagrangian subspace 
\[
  \big\{(\psi,d(f_\ast Q)_\psi) \, \big\vert \, \psi \in \F^M\big\} \subseteq
  \F^M \oplus (\F^M)^\ast.
\]
On the other hand, $\beta_N$ maps a Dirichlet form $Q$ on $N$ to the Lagrangian
subspace
\[
\big\{(\phi,dQ_\phi) \,\big\vert\, \phi \in \F^N\big\}\subseteq
  \F^N \oplus (\F^N)^\ast, 
\]
before $\mathrm{Lagr}(f)$ maps this to the Lagrangian subspace
\[
  \big\{(\psi, f_\ast dQ_\phi) \,\big\vert\, \psi \in \F^M, \phi =
  f^\ast(\psi)\big\} \subseteq \F^M\oplus (\F^M)^\ast.
\]
But 
\[
  f_\ast dQ_{f^\ast\psi} = d(f_\ast Q)_{\psi},
\]
so these two processes commute.

Monoidality requires that the diagrams 
\[
\begin{aligned}
\xymatrix{
  \mathrm{Dirich}(N) \times \mathrm{Dirich}(M) \ar[r]^(.52){\beta_N \times
  \beta_M} \ar[d]_{\delta_{N,M}} & \mathrm{Lagr}(N) \times \mathrm{Lagr}(M)
  \ar[d]^{\lambda_{N,M}}  \\
  \mathrm{Dirich}(N+M) \ar[r]_{\beta_{N+M}} & \mathrm{Lagr}(N+M)
}
\end{aligned}
\quad
\mbox{and}
\quad
\begin{aligned}
\xymatrix{
  & 1 \ar[dl]_{\delta_\varnothing} \ar[dr]^{\lambda_\varnothing}\\
\mathrm{Dirich}(\varnothing)  \ar[rr]_{\beta_\varnothing} &&
\mathrm{Lagr}(\varnothing)
}
\end{aligned}
\]
commute. These do: the Lagrangian subspace corresponding to the sum of Dirichlet
forms is equal to the sum of the Lagrangian subspaces that correspond to the
summand Dirichlet forms, while there is only a unique map $1 \to
\mathrm{Lagr}(\varnothing)$.
\end{proof}

We thus obtain a strict symmetric monoidal dagger functor 
\[   
\mathrm{DirichCospan} \to \mathrm{LagrCospan},
\]
which simply replaces the decoration on each cospan in $\mathrm{DirichCospan}$
with the corresponding Lagrangian subspace.

\subsection{From cospans to relations}

At this point we have checked that the process of reinterpreting a circuit as a
Lagrangian subspace of behaviors is functorial. Our task is now to integrate
this information as more than just a `decoration' on our morphisms. This process
constitutes a monoidal dagger functor
\[
  \mathrm{LagrCospan} \longrightarrow \LagrRel.
\]
This factor of the black box functor is the one that gives it its name;
through this functor we finally seal off the inner structure of our circuits,
leaving us access only to the behavior at the terminals. Its purpose is to take a 
Lagrangian cospan, which captures information about the behaviors of a 
circuit measured at every node, and restrict it down to a 
relation detailing the behaviors simply on the terminals. 

\begin{proposition}
We may define a symmetric monoidal dagger functor
\[
  \mathrm{LagrCospan} \longrightarrow \mathrm{LagrRel}
\]
as follows. On objects let this functor take a finite set $X$ to the symplectic
vector space $\F^X \oplus (\F^X)^\ast$. On morphisms let it take a Lagrangian
cospan
\[
  \big(X\stackrel{i}{\longrightarrow} N \stackrel{o}{\longleftarrow} Y; L
  \subseteq \F^N \oplus (\F^N)^\ast\big)
\]
to the Lagrangian relation
\[
  (S^t\!X\oplus SY) \circ S[i,o]^\dagger \circ L \subseteq
  \overline{\F^X \oplus (\F^X)^\ast} \oplus \F^Y \oplus (\F^Y)^\ast.  
\]  
\end{proposition}

\begin{proof}
The coherence maps here are the usual natural isomorphisms
\[
  \vectf{X} \oplus \vectf{Y} \stackrel{\sim}{\longrightarrow} \vectf{X+Y} 
\]
and
\[
  \{0\} \stackrel{\sim}{\longrightarrow} \vectf{\varnothing}.
\]
It is now routine to observe the symmetric monoidality and dagger-preserving
nature of this construction, as well as that it preserves identities. Finally,
we must check that composition is preserved.  

Using the concept of names introduced in Section \ref{subsec:names}, this comes
down to checking this equality of Lagrangian subspaces:
\[
\begin{aligned}
  \begin{tikzpicture}
	\begin{pgfonlayer}{nodelayer}
		\node [style=none] (0) at (-1.75, 0.5) {};
		\node [style=none] (1) at (-1.25, 0.5) {};
		\node [style=none] (2) at (-1.75, -0) {};
		\node [style=none] (3) at (-1.25, -0) {};
		\node [style=none] (4) at (-1.75, 1.25) {};
		\node [style=none] (5) at (-1.5, -0.75) {};
		\node [style=none] (6) at (-1.5, 0.5) {};
		\node [style=none] (7) at (-1.5, -0) {};
		\node [style=none] (8) at (-0.5, -0) {};
		\node [style=none] (9) at (-0.5, 0.75) {};
		\node [style=none] (10) at (-1, 2) {};
		\node [style=none] (11) at (-1.5, 1.5) {};
		\node [style=none] (12) at (-0.5, 1.5) {};
		\node [style=none] (13) at (-1, 1.5) {};
		\node [style=none] (14) at (-1.75, 0.75) {};
		\node [style=none] (15) at (-0.25, 1.25) {};
		\node [style=none] (16) at (-0.25, 0.75) {};
		\node [style=none] (17) at (-1.5, 0.75) {};
		\node [style=none] (18) at (-1, 1.25) {};
		\node [style=none] (19) at (0.25, 0.75) {};
		\node [style=none] (20) at (1.75, 1.25) {};
		\node [style=none] (21) at (1, 1.5) {};
		\node [style=none] (22) at (1.75, 0.75) {};
		\node [style=none] (23) at (0.5, -0) {};
		\node [style=none] (24) at (0.75, 0.5) {};
		\node [style=none] (25) at (1.5, 0.75) {};
		\node [style=none] (26) at (1, 1.25) {};
		\node [style=none] (27) at (1.5, -0.75) {};
		\node [style=none] (28) at (0.5, 0.75) {};
		\node [style=none] (29) at (0.25, 0.5) {};
		\node [style=none] (30) at (0.25, -0) {};
		\node [style=none] (31) at (0.5, 0.5) {};
		\node [style=none] (32) at (1, 2) {};
		\node [style=none] (33) at (1.5, 1.5) {};
		\node [style=none] (34) at (0.25, 1.25) {};
		\node [style=none] (35) at (0.5, 1.5) {};
		\node [style=none] (36) at (0.75, -0) {};
		\node [style=none] (37) at (0, -0.5) {};
		\node [style=none] (38) at (-1, 1.75) {$L$};
		\node [style=none] (39) at (1, 1.75) {$K$};
		\node [style=none] (40) at (-1, 1) {$S[i_X,o_Y]^\dagger$};
		\node [style=none] (41) at (1, 1) {$S[i_Y,o_Z]^\dagger$};
		\node [style=none] (42) at (-1.5, 0.25) {$S^t\!X$};
		\node [style=none] (43) at (0.5, 0.25) {$S^t\!Y$};
		\node [style=none] (44) at (-1.5, -1) {$\displaystyle\overline{\blacksquare(X)}$};
		\node [style=none] (45) at (1.5, -1) {$\displaystyle\blacksquare(Z)$};
	\end{pgfonlayer}
	\begin{pgfonlayer}{edgelayer}
		\draw (0.center) to (1.center);
		\draw (1.center) to (3.center);
		\draw (3.center) to (2.center);
		\draw (2.center) to (0.center);
		\draw (7.center) to (5.center);
		\draw (8.center) to (9.center);
		\draw (10.center) to (11.center);
		\draw (11.center) to (12.center);
		\draw (12.center) to (10.center);
		\draw (4.center) to (14.center);
		\draw (14.center) to (16.center);
		\draw (16.center) to (15.center);
		\draw (15.center) to (4.center);
		\draw (13.center) to (18.center);
		\draw (17.center) to (6.center);
		\draw (29.center) to (24.center);
		\draw (24.center) to (36.center);
		\draw (36.center) to (30.center);
		\draw (30.center) to (29.center);
		\draw (27.center) to (25.center);
		\draw (32.center) to (35.center);
		\draw (35.center) to (33.center);
		\draw (33.center) to (32.center);
		\draw (34.center) to (19.center);
		\draw (19.center) to (22.center);
		\draw (22.center) to (20.center);
		\draw (20.center) to (34.center);
		\draw (21.center) to (26.center);
		\draw (28.center) to (31.center);
		\draw [bend right=45, looseness=1.00] (8.center) to (37.center);
		\draw [bend right=45, looseness=1.00] (37.center) to (23.center);
	\end{pgfonlayer}
\end{tikzpicture}
  \end{aligned}
  \qquad
  =
  \qquad
  \begin{aligned}
    \begin{tikzpicture}
	\begin{pgfonlayer}{nodelayer}
		\node [style=none] (0) at (-0.75, -0.25) {};
		\node [style=none] (1) at (-0.25, -0.25) {};
		\node [style=none] (2) at (-0.75, -0.75) {};
		\node [style=none] (3) at (-0.25, -0.75) {};
		\node [style=none] (4) at (-0.75, 1.25) {};
		\node [style=none] (5) at (-0.5, -1) {};
		\node [style=none] (6) at (-0.5, -0.25) {};
		\node [style=none] (7) at (-0.5, -0.75) {};
		\node [style=none] (8) at (0, 0.5) {};
		\node [style=none] (9) at (0, 0.75) {};
		\node [style=none] (10) at (-0.5, 2) {};
		\node [style=none] (11) at (-1, 1.5) {};
		\node [style=none] (12) at (0, 1.5) {};
		\node [style=none] (13) at (-0.5, 1.5) {};
		\node [style=none] (14) at (-0.75, 0.75) {};
		\node [style=none] (15) at (0.75, 1.25) {};
		\node [style=none] (16) at (0.75, 0.75) {};
		\node [style=none] (17) at (-0.5, -0) {};
		\node [style=none] (18) at (-0.5, 1.25) {};
		\node [style=none] (19) at (-0.75, -0) {};
		\node [style=none] (20) at (0.75, 0.5) {};
		\node [style=none] (21) at (0.5, 1.25) {};
		\node [style=none] (22) at (0.75, -0) {};
		\node [style=none] (23) at (0.5, -0) {};
		\node [style=none] (24) at (0.5, 1.5) {};
		\node [style=none] (25) at (0.5, -1) {};
		\node [style=none] (26) at (0.5, 2) {};
		\node [style=none] (27) at (1, 1.5) {};
		\node [style=none] (28) at (-0.75, 0.5) {};
		\node [style=none] (29) at (0, 1.5) {};
		\node [style=none] (30) at (-0.5, 1.75) {$L$};
		\node [style=none] (31) at (0.5, 1.75) {$K$};
		\node [style=none] (32) at (0, 1) {$S[j_N,j_M]$};
		\node [style=none] (33) at (0, 0.25) {$\scriptscriptstyle
		  \stackrel{S[j_N \circ i_X,}{j_M \circ o_Z]^\dagger}$};
		\node [style=none] (34) at (-0.5, -0.5) {$S^t\!X$};
		\node [style=none] (35) at (-0.5, -1.25) {$\displaystyle\overline{\blacksquare(X)}$};
		\node [style=none] (36) at (0.5, -1.25) {$\displaystyle\blacksquare(Z)$};
	\end{pgfonlayer}
	\begin{pgfonlayer}{edgelayer}
		\draw (0.center) to (1.center);
		\draw (1.center) to (3.center);
		\draw (3.center) to (2.center);
		\draw (2.center) to (0.center);
		\draw (7.center) to (5.center);
		\draw (8.center) to (9.center);
		\draw (10.center) to (11.center);
		\draw (11.center) to (12.center);
		\draw (12.center) to (10.center);
		\draw (4.center) to (14.center);
		\draw (14.center) to (16.center);
		\draw (16.center) to (15.center);
		\draw (15.center) to (4.center);
		\draw (13.center) to (18.center);
		\draw (17.center) to (6.center);
		\draw (25.center) to (23.center);
		\draw (26.center) to (29.center);
		\draw (29.center) to (27.center);
		\draw (27.center) to (26.center);
		\draw (28.center) to (19.center);
		\draw (19.center) to (22.center);
		\draw (22.center) to (20.center);
		\draw (20.center) to (28.center);
		\draw (21.center) to (24.center);
	\end{pgfonlayer}
\end{tikzpicture}
  \end{aligned}
\]
where the left hand side is the composite in $\LagrRel$ of the images of
the Lagrangian cospans $(X\stackrel{i_X}{\longrightarrow} N
\stackrel{o_Y}{\longleftarrow} Y; L)$ and $(Y\stackrel{i_Y}{\longrightarrow} M
\stackrel{o_Z}{\longleftarrow} Z; K)$, and the right hand side is the image of
their composite in $\mathrm{LagrCospan}$. (Recall that we write $j_N: N \to
N+_YM$ and $j_M: M \to N+_YM$ for the maps given by the pushout.)

Recalling Remark \ref{rmk:dualitydiagrams} pertaining to the functor $S$ and
duals for objects, this implies that it is enough to check the equality of
corelations
\[
  \begin{aligned}
\begin{tikzpicture}
	\begin{pgfonlayer}{nodelayer}
		\node [style=none] (0) at (-1.75, 0.5) {};
		\node [style=none] (1) at (-1.75, 1.25) {};
		\node [style=none] (2) at (-1.5, -0) {};
		\node [style=none] (3) at (-0.5, 0.75) {};
		\node [style=none] (4) at (-1, 1.75) {};
		\node [style=none] (5) at (-1.75, 0.75) {};
		\node [style=none] (6) at (-0.25, 1.25) {};
		\node [style=none] (7) at (-0.25, 0.75) {};
		\node [style=none] (8) at (-1.5, 0.75) {};
		\node [style=none] (9) at (-1, 1.25) {};
		\node [style=none] (10) at (0.25, 0.75) {};
		\node [style=none] (11) at (1.75, 1.25) {};
		\node [style=none] (12) at (1, 1.75) {};
		\node [style=none] (13) at (1.75, 0.75) {};
		\node [style=none] (14) at (0.5, 0.75) {};
		\node [style=none] (15) at (1.5, 0.75) {};
		\node [style=none] (16) at (1, 1.25) {};
		\node [style=none] (17) at (1.5, -0) {};
		\node [style=none] (18) at (0.25, 1.25) {};
		\node [style=none] (19) at (0, 0.25) {};
		\node [style=none] (20) at (-1, 2) {$\displaystyle N$};
		\node [style=none] (21) at (1, 2) {$\displaystyle M$};
		\node [style=none] (22) at (-1, 1) {$[i_X,o_Y]^\dagger$};
		\node [style=none] (23) at (1, 1) {$[i_Y,o_Z]^\dagger$};
		\node [style=none] (24) at (-1.5, -0.25) {$\displaystyle X$};
		\node [style=none] (25) at (1.5, -0.25) {$\displaystyle Z$};
	\end{pgfonlayer}
	\begin{pgfonlayer}{edgelayer}
		\draw (1.center) to (5.center);
		\draw (5.center) to (7.center);
		\draw (7.center) to (6.center);
		\draw (6.center) to (1.center);
		\draw (4.center) to (9.center);
		\draw (8.center) to (2.center);
		\draw (17.center) to (15.center);
		\draw (18.center) to (10.center);
		\draw (10.center) to (13.center);
		\draw (13.center) to (11.center);
		\draw (11.center) to (18.center);
		\draw (12.center) to (16.center);
		\draw [bend right=45, looseness=1.00] (3.center) to (19.center);
		\draw [bend right=45, looseness=1.00] (19.center) to (14.center);
	\end{pgfonlayer}
\end{tikzpicture}
  \end{aligned}
  \qquad
  =
  \qquad
  \begin{aligned}
  \begin{tikzpicture}
	\begin{pgfonlayer}{nodelayer}
		\node [style=none] (0) at (-0.75, 1.25) {};
		\node [style=none] (1) at (-0.5, -0.25) {};
		\node [style=none] (2) at (0, 0.5) {};
		\node [style=none] (3) at (0, 0.75) {};
		\node [style=none] (4) at (-0.5, 1.5) {};
		\node [style=none] (5) at (-0.75, 0.75) {};
		\node [style=none] (6) at (0.75, 1.25) {};
		\node [style=none] (7) at (0.75, 0.75) {};
		\node [style=none] (8) at (-0.5, -0) {};
		\node [style=none] (9) at (-0.5, 1.25) {};
		\node [style=none] (10) at (-0.75, -0) {};
		\node [style=none] (11) at (0.75, 0.5) {};
		\node [style=none] (12) at (0.5, 1.25) {};
		\node [style=none] (13) at (0.75, -0) {};
		\node [style=none] (14) at (0.5, -0) {};
		\node [style=none] (15) at (0.5, 1.5) {};
		\node [style=none] (16) at (0.5, -0.25) {};
		\node [style=none] (17) at (-0.75, 0.5) {};
		\node [style=none] (18) at (-0.5, 1.75) {$\displaystyle N$};
		\node [style=none] (19) at (0.5, 1.75) {$\displaystyle M$};
		\node [style=none] (20) at (0, 1) {$[j_N,j_M]$};
		\node [style=none] (21) at (0, 0.25) {$\scriptscriptstyle \stackrel{[j_N
		\circ i_X,}{j_M \circ o_Z]^\dagger}$};
		\node [style=none] (22) at (-0.5, -0.5) {$\displaystyle X$};
		\node [style=none] (23) at (0.5, -0.5) {$\displaystyle Z$};
	\end{pgfonlayer}
	\begin{pgfonlayer}{edgelayer}
		\draw (2.center) to (3.center);
		\draw (0.center) to (5.center);
		\draw (5.center) to (7.center);
		\draw (7.center) to (6.center);
		\draw (6.center) to (0.center);
		\draw (4.center) to (9.center);
		\draw (8.center) to (1.center);
		\draw (16.center) to (14.center);
		\draw (17.center) to (10.center);
		\draw (10.center) to (13.center);
		\draw (13.center) to (11.center);
		\draw (11.center) to (17.center);
		\draw (12.center) to (15.center);
	\end{pgfonlayer}
\end{tikzpicture}
  \end{aligned}
\]
Writing this instead as a commutative diagram, we wish to prove the equality of
the two corelations $N+M \to X+Y$:
\[
  \xymatrix@C=30pt@R=5pt{
    &&& N+_YM \\
    N+M \ar^{[j_N,j_M]}[urrr] && && && X+Z
    \ar@{^{(}->}[dll]^(.48){\textrm{incl}_{X+Z}} 
    \ar_{[j_N\circ i_X, j_M \circ o_Z]}[ulll] \\
    && X+Y+Y+Z \ar[ull]^(.58){[i_X,o_Y]+[i_Y,o_Z]\quad}
    \ar[rr]_(.53){\idn_X+[\idn_Y,\idn_Y]+\idn_Z} && X+Y+Z
  }
\]
As the right-hand part admits the factorization 
\[
  \xymatrix@C=30pt@R=5pt{
    &&&N+_YM \\
    &&&& && X+Z
    \ar@{^{(}->}[dll]^(.48){\textrm{incl}_{X+Z}} 
    \ar_{[j_N\circ i_X, j_M \circ o_Z]}[ulll] \\
    &&&& X+Y+Z \ar[uul]^{[j_N\circ i_X,j_N\circ o_Y,j_M\circ o_Z]\qquad}
  }
\]
it is enough to check the following diagram commutes when interpreted as a pair
of morphisms from $N+M$ to $X+Y+Z$ in the category of corelations: 
\[
  \xymatrix{
    & N+_YM \\
    N+M \ar[ur]^{[j_N,j_M]} && X+Y+Z \ar[ul]_{\qquad\quad[j_N\circ i_X,j_N\circ o_Y,j_M\circ o_Z]} \\
    & X+Y+Y+Z \ar[ul]^{[i_X,o_Y]+[i_Y,o_Z]\qquad} \ar[ur]_{\qquad\idn_X+[\idn_Y,\idn_Y]+\idn_Z} 
  }
\]
By Lemma \ref{lem:pushoutcorelations}, this is equivalent to checking that it is
a pushout square in $\mathrm{FinSet}$.  This is so: the square commutes in
$\mathrm{FinSet}$ as it is the sum along the lower right edges of the three
commutative squares
\[
\xymatrix@=1.5em{
& N+_YM \\
N+M \ar[ur]^{[j_N,j_M]} && X \ar[ul]_{j_N\circ i_X} \\
& X \ar[ur]_{\idn_X} \ar[ul]^{i_X} 
}
\quad
\xymatrix@=1.5em{
& N+_YM \\
N+M \ar[ur]^{[j_N,j_M]} && Y \ar[ul]_{\quad j_N\circ o_Y=j_M\circ i_Y} \\
& Y+Y \ar[ul]^{o_Y+i_Y} \ar[ur]_{[\idn_Y,\idn_Y]} 
} 
\xymatrix@=1.5em{
& N+_YM \\
N+M \ar[ur]^{[j_N,j_M]} && Z \ar[ul]_{j_M\circ o_Z} \\
& Z \ar[ul]^{o_Z} \ar[ur]_{\idn_Z} 
} 
\]
and given any other object $T$ and maps $f,g$ such that
\[
  \xymatrix{
    & T \\
    N+M \ar[ur]^{f} && X+Y+Z \ar[ul]_{g} \\
    & X+Y+Y+Z \ar[ul]^{[i_X,o_Y]+[i_Y,o_Z]\quad} \ar[ur]_{\qquad \idn_X+[\idn_Y,\idn_Y]+\idn_Z} 
  }
\]
commutes, there is a unique map $N+_YM \to T$ defined by sending $a$ in $N+_YM$
to $f(\hat a)$, where $\hat a$ is a preimage of $a$ in $N+M$ under the coproduct
of pushout maps $[j_N,j_M]$. This is well-defined as the preimage of $a$ is
either unique or equal to $\{o_Y(y),i_Y(y)\}$ for some element $y \in Y$, and
the commutativity of the above square containing $T$ implies that $f(o_Y(y)) =
f(i_Y(y))$. This proves the functoriality of the map $\mathrm{LagrCospan} \to
\LagrRel$ defined above.
\end{proof}

The three functors of this section compose to give the black box functor
\[
\blacksquare\maps \Circ \to \LagrRel.
\] 
Since they are each separately symmetric monoidal dagger functors, the black box
functor is too.


%\section{Concluding remarks}



\chapter{Passive linear networks} \label{ch.circuits}
This chapter is about using decorated corelations to build semantic functors. We
first provide an introduction to passive linear networks and what they model. We
then use decorated cospans to define open networks, and use decorated
corelations to construct an appropriate hypergraph category of semantics. The
decorated corelations framework will show this semantics is functorial.

\section{Introduction}\label{sec:intro}
%%fakesubsection
In late 1940s, just as Feynman was developing his diagrams for processes in particle physics, Eilenberg and Mac Lane initiated their work on category theory.  Over the subsequent decades, and especially in the work of Joyal and Street in the 1980s \cite{JS1,JS2}, it became clear that these developments were profoundly linked: monoidal categories have a precise graphical representation in terms of string diagrams, and conversely monoidal categories provide an algebraic foundation for the intuitions behind Feynman diagrams.  The key insight is the use of categories where morphisms describe physical processes, rather than structure-preserving maps between mathematical objects \cite{BaezStay,CP}.

In work on fundamental physics, the cutting edge has moved from categories
to higher categories \cite{BL}.  But the same techniques have filtered into more
immediate applications, particularly in computation and quantum computation
\cite{AC,Ba1,Se}.  This chapter is part of a still nascent program of applying string diagrams to engineering, with the aim of giving diverse diagram languages a unified foundation based on category theory \cite{BE,BSZ,KSW,RSW,Sp}. 

Indeed, even before physicists began using Feynman diagrams, various branches of engineering were using diagrams that in retrospect are closely related.   Foremost among these are the ubiquitous electrical circuit diagrams. Although less well-known, similar diagrams are used to describe networks consisting of mechanical, hydraulic, thermodynamic and chemical systems.   Further work, pioneered in particular by 
Forrester \cite{Fo} and Odum \cite{Od}, applies similar diagrammatic methods to biology, ecology, and economics.

As discussed in detail by Olsen \cite{Ol}, Paynter \cite{Pa} and others, there are mathematically precise analogies between these different systems.  In each case, the system's state is described by variables that come in pairs, with one variable in each pair playing the role of  `displacement' and the other playing the role of `momentum'.  In engineering, the time derivatives of these variables are sometimes called `flow' and `effort'.    In classical mechanics, this pairing of variables is well understood using
symplectic geometry.  Thus, any mathematical formulation of the diagrams used to
describe networks in engineering needs to take symplectic geometry as well as category
theory into account. 

\vskip 1em
\begin{small}
\begin{center}
\begin{tabular}{|c||c|c|c|c|}
\hline
& displacement  &  flow & momentum & effort \\
& $q$ & $\dot{q}$ & $p$ & $\dot{p}$ \\
\hline\hline
Electronics & charge & current & flux linkage & voltage\\
\hline
Mechanics (translation) & position & velocity & momentum & force\\
\hline
Mechanics (rotation) & angle & angular velocity & angular momentum & torque\\
\hline
Hydraulics & volume & flow & pressure momentum & pressure\\
\hline
Thermodynamics & entropy & entropy flow & temperature momentum & temperature \\
\hline
Chemistry & moles & molar flow & chemical momentum & chemical potential \\
\hline
\end{tabular}
\end{center}
\end{small}

While diagrams of networks have been independently introduced in many disciplines, we do not expect formalizing these diagrams to immediately help the practitioners of these disciplines.  At first the flow of information will mainly go in the other direction: by translating ideas from these disciplines into the language of modern mathematics, we can provide mathematicians with food for thought and interesting new problems to solve.  We hope that in the long run mathematicians can return the favor by bringing new insights to the table.

Although we shall keep the broad applicability of network diagrams in the back
of our minds, we couch our discussion in terms of electrical circuits, for the
sake of familiarity. In this chapter our goal is somewhat limited.  We only study circuits built from `passive' components: that is, those that do not produce energy.  Thus, we exclude batteries and current sources.  We only consider components that respond linearly to an applied voltage.   Thus, we exclude components such as nonlinear resistors or diodes.  Finally, we only consider components with one input and one output, so that a circuit can be described as a graph with edges labeled by components.  Thus, we also exclude transformers.  The most familiar components our framework covers are linear resistors, capacitors and inductors.

While we hope to expand our scope in future work, the class of circuits made from these components has appealing mathematical properties, and is worthy of deep study.  Indeed, this class has been studied intensively for many decades by electrical engineers \cite{AV,Budak,Slepian}.  Even circuits made exclusively of resistors have inspired work by mathematicians of the caliber of Weyl \cite{Weyl} and Smale \cite{Smale}.  

Our work relies on this research.  All we are adding is an emphasis on symplectic geometry and an explicitly `compositional' framework, which clarifies the way a larger circuit can be built from smaller pieces.  This is where monoidal categories become important: the main operations for building circuits from pieces are composition and tensoring.
 
Our strategy is most easily illustrated for circuits made of linear resistors.  Such a resistor dissipates power, turning useful energy into heat at a rate determined by the voltage across the resistor.  However, a remarkable fact is that a circuit made of these resistors always acts to \emph{minimize} the power dissipated this way.  This `principle of minimum power' can be seen as the reason symplectic geometry becomes important in understanding circuits made of resistors, just as the principle of least action leads to the role of symplectic geometry in classical mechanics.  

Here is a circuit made of linear resistors:
\[
\begin{tikzpicture}[circuit ee IEC, set resistor graphic=var resistor IEC graphic]
\node[contact] (I1) at (0,2) {};
\node[contact] (I2) at (0,0) {};
\node[contact] (O1) at (5.83,1) {};
\node(input) at (-2,1) {\small{\textsf{inputs}}};
\node(output) at (7.83,1) {\small{\textsf{outputs}}};
\draw (I1) 	to [resistor] node [label={[label distance=2pt]85:{$3\Omega$}}] {} (2.83,1);
\draw (I2)	to [resistor] node [label={[label distance=2pt]275:{$1\Omega$}}] {} (2.83,1)
				to [resistor] node [label={[label distance=3pt]90:{$4\Omega$}}] {} (O1);
\path[color=gray, very thick, shorten >=10pt, ->, >=stealth, bend left] (input) edge (I1);		\path[color=gray, very thick, shorten >=10pt, ->, >=stealth, bend right] (input) edge (I2);		
\path[color=gray, very thick, shorten >=10pt, ->, >=stealth] (output) edge (O1);
\end{tikzpicture}
\]
The wiggly lines are resistors, and their resistances are written beside them: for example,
$3\Omega$ means 3 ohms, an `ohm' being a unit of resistance.  To formalize this, define a circuit of linear resistors to consist of:
\begin{itemize}
\item a set $N$ of nodes,
\item a set $E$ of edges, 
\item maps $s,t \maps E \to N$ sending each edge to its source and target node,
\item a map $r\maps E \to (0,\infty)$ specifying the resistance of the resistor 
labelling each edge, 
\item maps $i \maps X \to N$, $o \maps Y \to N$ specifying the
inputs and outputs of the circuit.
\end{itemize}

When we run electric current through such a circuit, each node $n \in N$ gets
a `potential' $\phi(n)$.  The `voltage' across an edge $e \in E$ is defined as the 
change in potential as we move from to the source of $e$ to its target, $\phi(t(e)) - 
\phi(s(e))$, and the power dissipated by the resistor on this edge equals
\[      
\frac{1}{r(e)}\big(\phi(t(e))-\phi(s(e))\big)^2. 
\]
The total power dissipated by the circuit is therefore twice
\[   
P(\phi) = \frac{1}{2}\sum_{e \in E} \frac{1}{r(e)}\big(\phi(t(e))-\phi(s(e))\big)^2.
\]
The factor of $\frac{1}{2}$ is convenient in some later calculations.  
Note that $P$ is a nonnegative quadratic form on the vector space $\R^N$.
However, not every nonnegative definite quadratic form on $\R^N$ arises in this way from some circuit of linear resistors with $N$ as its set of nodes.  The quadratic forms that do arise are called `Dirichlet forms'.  They have been extensively investigated \cite{Fukushima,MR,Sabot1997,Sabot2004}, and they play a major role in our work.

We write $\partial N = i(X) \cup o(Y)$ for the set of `terminals': that is,
nodes corresponding to inputs and outputs.    The principle of minimum
power says that if we fix the potential at the terminals, the circuit will choose
the potential at other nodes to minimize the total power dissipated.   
An element $\psi$ of the vector space $\R^{\partial N}$ assigns a potential 
to each terminal.   Thus, if we fix $\psi$, the total power dissipated will be twice
\[
  Q(\psi) = \min_{\substack{ \phi \in \R^N \\ \phi\vert_{\partial N} = \psi}} \; P(\phi)  
\]
The function $Q \maps \R^{\partial N} \to \R$ is again a Dirichlet form.  We call it the `power functional' of the circuit.  

Now, suppose we are unable to see the internal workings of a circuit, and can only observe its `external behaviour': that is, the potentials at its terminals and the currents flowing into or out of these terminals.   Remarkably, this behaviour is completely determined by the power functional $Q$.  The reason is that the current at any terminal can be obtained by differentiating $Q$ with respect to the potential at this terminal, and relations of this form are \emph{all} the relations that hold between potentials and currents at the terminals.

The Laplace transform allows us to generalize this immediately to circuits that
can also contain linear inductors and capacitors, simply by changing the field we work over, replacing $\R$ by the field $\R(s)$ of rational functions of a single real variable,
and talking of `impedance' where we previously talked of resistance.  We obtain
a category $\Circ$ where an object is a finite set, a morphism $f \maps X \to Y$ is a circuit with input set $X$ and output set $Y$, and composition is given by identifying the outputs of one circuit with the inputs of the next, and taking the resulting union of labelled graphs.  Each such circuit gives rise to a Dirichlet form, now defined over
$\R(s)$, and this Dirichlet form completely describes the externally observable
behaviour of the circuit.  

We can take equivalence classes of circuits, where two circuits count as the
same if they have the same Dirichlet form.  We wish for these equivalence classes of circuits to form a category. Although
there is a notion of composition for Dirichlet forms, we find that it lacks
identity morphisms or, equivalently, it lacks morphisms representing ideal wires
of zero impedance. To address this we turn to Lagrangian subspaces of
symplectic vector spaces.  These generalize quadratic forms via the map
\[
  \Big(Q\maps \F^{\partial N} \to \F\Big) \longmapsto \Big(\mathrm{Graph}(dQ) =
  \{(\psi, dQ_\psi) \mid \psi \in \F^{\partial N} \} \subseteq \F^{\partial
  N} \oplus (\F^{\partial N})^\ast\Big)
\]
taking a quadratic form $Q$ on the vector space $\F^{\partial N}$
over the field $\F$ to the graph
of its differential $dQ$. Here we think of the symplectic vector space
$\F^{\partial N} \oplus (\F^{\partial N})^\ast$ as the state space of the
circuit, and the subspace $\mathrm{Graph}(dQ)$ as the subspace of attainable
states, with $\psi \in \F^{\partial N}$ describing the potentials at the
terminals, and $dQ_\psi \in (\F^{\partial N})^\ast$ the currents. 

This construction is well-known in classical mechanics \cite{Weinstein}, where the principle of least action plays a role analogous to that of the principle of minimum power here.   The set of Lagrangian subspaces is actually an algebraic variety,
the `Lagrangian Grassmannian', which serves as a compactification of the
space of quadratic forms.  The Lagrangian Grassmannian has already played a
role in Sabot's work on circuits made of resistors \cite{Sabot1997,Sabot2004}.
For us, its importance it that we can find identity morphisms
for the composition of Dirichlet forms by taking circuits made of parallel resistors
and letting their resistances tend to zero: the limit is not a Dirichlet form, but
it exists in the Lagrangian Grassmannian.    Indeed, 
there exists a category $\LagrRel$ with finite dimensional
symplectic vector spaces as objects and `Lagrangian relations' as morphisms: 
that is, linear relations from $V$ to $W$ that are given by Lagrangian subspaces of $\overline{V} \oplus W$, where $\overline{V}$ is the symplectic vector space conjugate to $V$.   

To move from the Lagrangian subspace defined by the graph of the differential of
the power functional to a morphism in the category $\LagrRel$---that
is, to a Lagrangian relation---we must treat seriously the input and output
functions of the circuit. These express the circuit as built upon a cospan   
\[
  \xymatrix{
    & N \\
    X \ar[ur]^{i} && Y. \ar[ul]_o
  }
\]
Applicable far more broadly than this present formalization of circuits, cospans
model systems with two `ends', an input and output end, albeit without any
connotation of directionality: we might just as well exchange the role of the
inputs and outputs by taking the mirror image of the above diagram. The role of
the input and output functions, as we have discussed, is to mark the terminals
we may glue onto the terminals of another circuit, and the pushout of cospans
gives formal precision to this gluing construction.

One upshot of this cospan framework is that we may consider circuits with elements
of $N$ that are both inputs and outputs, such as this one:
\[
  \begin{tikzpicture}[circuit ee IEC, set resistor graphic=var resistor iec graphic]
    \node[contact] (c1) at (0,2) {};
    \node[contact] (c2) at (0,0) {};
    \node(input) at (-2,1) {\small{\textsf{inputs}}};
    \node(output) at (2,1) {\small{\textsf{outputs}}};
    \path[color=gray, very thick, shorten >=10pt, ->, >=stealth, bend left]
    (input) edge (c1);		
    \path[color=gray, very thick, shorten >=10pt, ->, >=stealth, bend right]
    (input) edge (c2);	
    \path[color=gray, very thick, shorten >=10pt, ->, >=stealth, bend right]
    (output) edge (c1);
    \path[color=gray, very thick, shorten >=10pt, ->, >=stealth, bend left]
    (output) edge (c2);
  \end{tikzpicture}
\]
This corresponds to the identity morphism on the finite set with two elements.
Another is that some points may be considered an input or output multiple
times; we draw this:
\[
  \begin{tikzpicture}[circuit ee IEC, set resistor graphic=var resistor IEC graphic]
    \node[contact] (I1) at (0,0) {};
    \node[contact] (O1) at (3,0) {};
    \node(input) at (-2,0) {\small{\textsf{inputs}}};
    \node(output) at (5,0) {\small{\textsf{outputs}}};
    \draw (I1) 	to [resistor] node [label={[label distance=3pt]90:{$1\Omega$}}]
    {} (O1);
    \path[color=gray, very thick, shorten >=10pt, ->, >=stealth, bend left] (input)
    edge (I1);		
    \path[color=gray, very thick, shorten >=10pt, ->,
    >=stealth, bend right] (input) edge (I1);		
    \path[color=gray, very thick, shorten >=10pt, ->, >=stealth] (output) edge (O1);
  \end{tikzpicture}
\]
This allows us to connect two distinct outputs to the above double
input.

Given a set $X$ of inputs or outputs, we understand the electrical behaviour on this set 
by considering the symplectic vector space $\vectf{X}$, the direct sum of the space
$\F^X$ of potentials and the space ${(\F^X)}^\ast$ of currents at these points.
A Lagrangian relation specifies which states of the output space $\vectf{Y}$ are
allowed for each state of the input space $\vectf{X}$.
Turning the Lagrangian subspace $\mathrm{Graph}(dQ)$ of a circuit into this
information requires that we understand the `symplectification' 
\[  Sf\maps \vectf{B} \to \vectf{A} \] 
and `twisted symplectification'
\[  S^tf\maps \vectf{B} \to \overline{\vectf{A}}\]
of a function $f\maps A \to B$ between finite sets.  In particular we need to understand how these apply to the input and output functions with codomain restricted to $\partial N$; abusing notation, we also write these $i\maps X \to \partial N$ and $o\maps Y \to \partial N$.

The symplectification is a Lagrangian relation, and the catch
phrase is that it `copies voltages' and `splits currents'.  More precisely,
for any given potential-current pair $(\psi,\iota)$ in $\vectf{B}$, its image
under $Sf$ comprises all elements of $(\psi', \iota') \in \vectf{A}$ such that
the potential at $a \in A$ is equal to the potential at $f(a) \in B$, and such
that, for each fixed $b \in B$, collectively the currents at the $a \in
f^{-1}(b)$ sum to the current at $b$.  We use the symplectification $So$ of the
output function to relate the state on $\partial N$ to that on the
outputs $Y$. As our current framework is set up to report the current \emph{out}
of each node, to describe input currents we define the twisted symplectification
$S^tf\maps \vectf{B} \to \overline{\vectf{A}}$ almost identically to the above, except that we flip the sign of the currents $\iota' \in (\F^A)^\ast$.  We use the twisted symplectification $S^ti$ of the input function to relate the state on $\partial N$
to that on the inputs.

The Lagrangian relation corresponding to a circuit is then the set of all
potential--current pairs that are possible at the inputs and outputs of that circuit. 
For instance, consider a resistor of resistance $r$, with one end considered as an
input and the other as an output:
\[
  \begin{tikzpicture}[circuit ee IEC, set resistor graphic=var resistor IEC graphic]
    \node[contact] (I1) at (0,0) {};
    \node[contact] (O1) at (3,0) {};
    \node(input) at (-2,0) {\small{\textsf{input}}};
    \node(output) at (5,0) {\small{\textsf{output}}};
    \draw (I1) 	to [resistor] node [label={[label distance=3pt]90:{$r$}}] {} (O1);
    \path[color=gray, very thick, shorten >=10pt, ->, >=stealth] (input)
    edge (I1);
    \path[color=gray, very thick, shorten >=10pt, ->, >=stealth] (output) edge (O1);
  \end{tikzpicture}
\]
To obtain the corresponding Lagrangian relation, we must first specify domain and
codomain symplectic vector spaces. In this case, as the input and output sets
each consist of a single point, these vector spaces are both $\F \oplus \F^\ast$,
where the first summand is understood as the space of potentials, and the second
the space of currents.

Now, the resistor has power functional $Q\maps \F^2 \to \F$ given by
\[   Q(\psi_1,\psi_2) = \frac1{2r}(\psi_2-\psi_1)^2, \]
and the graph of the differential of $Q$ is
\[
  \mathrm{Graph}(dQ) = \big\{\big(\psi_1,\psi_2,
  \tfrac1r(\psi_1-\psi_2),\tfrac1r(\psi_2-\psi_1)\big) \,\big|\, \psi_1,\psi_2 \in
  \F\big\} \subseteq \F^2 \oplus (\F^2)^\ast.
\]
In this example the input and output functions $i,o$ are simply the identity
functions on a one element set, so the symplectification of the output function
is simply the identity linear transformation, and the twisted symplectification
of the input function is the isomorphism  between conjugate
symplectic vector spaces $\F\oplus\F^\ast \to \overline{\F\oplus\F^\ast}$ mapping $(\phi,i)$ to $(\phi,-i)$ This implies that the behaviour associated to this
circuit is the Lagrangian relation
\[
  \big\{(\psi_1,i,\psi_2,i) \,\big|\, \psi_1,\psi_2 \in \F, i =
  \tfrac1r(\psi_2-\psi_1)\big\}\subseteq \overline{\F \oplus \F^\ast} \oplus \F
    \oplus \F^\ast.
\]
This is precisely the set of potential-current pairs that are allowed at the
input and output of a resistor of resistance $r$.  In particular, the relation
$i = \tfrac1r(\psi_2-\psi_1)$ is well-known in electrical engineering: it is
`Ohm's law'.

A crucial fact is that the process of mapping a circuit to its corresponding
Lagrangian relation identifies distinct circuits.  For example, a single 2-ohm resistor:
\[
  \begin{tikzpicture}[circuit ee IEC, set resistor graphic=var resistor IEC graphic]
    \node[contact] (I1) at (0,0) {};
    \node[contact] (O1) at (3,0) {};
    \node(input) at (-2,0) {\small{\textsf{input}}};
    \node(output) at (5,0) {\small{\textsf{output}}};
    \draw (I1) 	to [resistor] node [label={[label distance=3pt]90:{$2\Omega$}}] {} (O1);
    \path[color=gray, very thick, shorten >=10pt, ->, >=stealth] (input)
    edge (I1);
    \path[color=gray, very thick, shorten >=10pt, ->, >=stealth] (output) edge (O1);
  \end{tikzpicture}
\]
has the same Lagrangian relation as two 1-ohm resistors in series:
\[
  \begin{tikzpicture}[circuit ee IEC, set resistor graphic=var resistor IEC graphic]
    \node[contact] (I1) at (0,0) {};
    \node[circle, minimum width = 3pt, inner sep = 0pt, fill=black] (int) at
    (3,0) {};
    \node[contact] (O1) at (6,0) {};
    \node(input) at (-2,0) {\small{\textsf{input}}};
    \node(output) at (8,0) {\small{\textsf{output}}};
    \draw (I1) 	to [resistor] node [label={[label distance=3pt]90:{$1\Omega$}}] {} (int)
    to [resistor] node [label={[label distance=3pt]90:{$1\Omega$}}] {} (O1);
    \path[color=gray, very thick, shorten >=10pt, ->, >=stealth] (input)
    edge (I1);
    \path[color=gray, very thick, shorten >=10pt, ->, >=stealth] (output) edge (O1);
  \end{tikzpicture}
\]
The Lagrangian relation does not shed any light on the internal workings of a
circuit.  Thus, we call the process of computing this relation `black boxing':
it is like encasing the circuit in an opaque box, leaving only its terminals
accessible. Fortunately, the Lagrangian relation of a circuit is enough to
completely characterize its external behaviour, including how it interacts when
connected with other circuits. 

Put more precisely, the black boxing process is \emph{functorial}: we can 
compute the black boxed version of a circuit made of parts by computing the
black boxed versions of the parts and then composing them.   In fact we shall 
prove that $\Circ$ and $\LagrRel$ are dagger compact categories, and
the black box functor preserves all this extra structure:

\begin{theorem} \label{main_theorem}
  There exists a hypergraph functor, the \define{black box functor}   
  \[ \blacksquare\maps \Circ \to \LagrRel, \]
   mapping a finite set $X$ to the symplectic vector space
  $\vectf{X}$ it generates, and a circuit $\big((N,E,s,t,r),i,o\big)$ to the Lagrangian     
  relation 
  \[
    \bigcup_{v \in \mathrm{Graph}(dQ)} S^ti(v) \times So(v)
    \subseteq \overline{\F^X \oplus (\F^X)^\ast} \oplus \F^Y \oplus (\F^Y)^\ast,
  \]
  where $Q$ is the circuit's power functional.
\end{theorem}

The goal of this chapter is to prove and explain this result, demonstrating how
the mathematical machinery of Part I provides clarity. With these tools in hand,
the black box functor turns out to rely on a tight relationship between
Kirchhoff's laws, the minimization of Dirichlet forms, and the
`symplectification' of corelations. It is well-known that away from the
terminals, a circuit must obey two rules known as Kirchhoff's laws.  We have
already noted that the principle of minimum power states that a circuit will
`choose' potentials on its interior that minimize the power functional.  We
clarify the relation between these points in Theorems
\ref{thm:realizablepotentials} and \ref{thm:dirichletminimization}, which
together show that minimizing a Dirichlet form over some subset amounts to
assuming that the corresponding circuit obeys Kirchhoff's laws on that subset.

We have also mentioned the symplectification of functions above.  Extending this
to allow symplectification of epi-mono corelations in $\FinSet$, this process
gives a map sending corelations to Lagrangian relations that describe the
behaviour of ideal perfectly conductive wires.  We prove that these
symplectified corelations simultaneously impose Kirchhoff's laws (Proposition
\ref{prop:sympfunctor} and Example \ref{ex:sympfunction}) and accomplish the
minimization of Dirichlet forms (Theorem \ref{thm:sympmin}).  

Together, our results show that these three concepts---Kirchhoff's laws from
circuit theory, the analytic idea of minimizing power dissipation, and the
algebraic idea of symplectification of corelations---are merely different faces
of one law: the law of composition of circuits.

\paragraph{Outline.}
First we discuss the semantics of circuit diagrams. We begin with a discussion
of circuits of linear resistors, developing the intuition for the governing laws
of passive linear circuits---Ohm's law, Kirchhoff's voltage law, and Kirchhoff's
current law---in a time-independent setting (Section \ref{sec:resistors}), and
showing that Dirichlet forms represent circuit behaviours.  In Section
\ref{sec:plcs}, the Laplace transform then allows us to recapitulate these ideas
after introducing inductors and capacitors, speaking of impedance where we
formerly spoke of resistance, and generalizing Dirichlet forms from the field
$\R$ to the field $\R(s)$ of real rational functions. 

The goal of this applications chapter is to show how decorated corelations can
make these semantics compositional. As a network-style diagrammatic language, it
should thus be no surprise that we begin by constructing a hypergraph category
of circuits (Section \ref{sec:circdef}). At the end of this section, however, we
show that Dirichlet forms do not provide the flexibility to construct a category
of behaviours of circuits.  This motivates the development of more powerful
machinery.

In Section \ref{sec:circlagr} we review the basic theory of linear
Lagrangian relations, giving details to the correspondence we have defined
between Dirichlet forms, and hence passive linear circuits, and Lagrangian
relations. Section \ref{sec:corel} then takes immediate advantage of the added
flexibility of Lagrangian relations, discussing the `trivial' circuits
comprising only perfectly conductive wires, which mediate the notion of composition
of circuits.

With these tools, we prove the functoriality of black-boxing circuits in Section
\ref{sec:blackbox}.

\section{Circuits of linear resistors} \label{sec:resistors}
%%fakesubsection
Our first concern is for semantics: ``What do circuit diagrams mean?''. 

To elaborate, while circuit diagrams model electric circuits according to their
physical form, another, often more relevant, way to understand a circuit is by
its external behaviour. This means the following. To an electric circuit we
associate two quantities to each edge: voltage and current. We are not free,
however, to choose these quantities as we like; circuits are subject to
governing laws that imply voltages and currents must obey certain relationships.
From the perspective of control theory we are particularly interested in the
values these quantities take at the so-called terminals, and how altering one
value will affect the other values. We call two circuits equivalent when they
determine the same relationship. Our main task in this first part is to explore
when two circuits are equivalent.

In order to let physical intuition lead the way, we begin by specialising to the
case of linear resistors. In this section we describe how to find the function
of a circuit from its form, advocating in particular the perspective of the
principle of minimum power. This allows us to identify the external behaviour of a
circuit with a so-called Dirichlet form representing the dependence of its power
consumption on potentials at its terminals.

\subsection{Circuits as labelled graphs}

The concept of an abstract open electrical circuit made of linear resistors is
well-known in electrical engineering, but we shall need to formalize it with
more precision than usual.  The basic idea is that a circuit of linear resistors
is a graph whose edges are labelled by positive real numbers called
`resistances', and whose sets of vertices is equipped with two subsets: the
`inputs' and `outputs'. This unfolds as follows.

A (closed) circuit of resistors looks like this: 
\[
\begin{tikzpicture}[circuit ee IEC, set resistor graphic=var resistor IEC graphic]
\node (I1) at (0,0) {};
\node (I2) at (0,2) {};
\node (O1) at (5.83,1) {};
\draw (I1) 	to [resistor] node [label={[label distance=2pt]275:{$1\Omega$}}] {} (2.83,1);
\draw (I2)	to [resistor] node [label={[label distance=2pt]85:{$1\Omega$}}] {} (2.83,1)
				to [resistor] node [label={[label distance=3pt]90:{$2\Omega$}}] {} (O1);
\end{tikzpicture}
\]
We can consider this a labelled graph, with each resistor an edge of the graph,
its resistance its label, and the vertices of the graph the points at which
resistors are connected. 

A circuit is `open' if it can be connected to other circuits. To do this we
first mark points at which connections can be made by denoting some vertices as
input and output terminals:
\[
\begin{tikzpicture}[circuit ee IEC, set resistor graphic=var resistor IEC graphic]
\node[contact] (I1) at (0,2) {};
\node[contact] (I2) at (0,0) {};
\node[contact] (O1) at (5.83,1) {};
\node(input) at (-2,1) {\small{\textsf{inputs}}};
\node(output) at (7.83,1) {\small{\textsf{outputs}}};
\draw (I1) 	to [resistor] node [label={[label distance=2pt]85:{$1\Omega$}}] {} (2.83,1);
\draw (I2)	to [resistor] node [label={[label distance=2pt]275:{$1\Omega$}}] {} (2.83,1)
				to [resistor] node [label={[label distance=3pt]90:{$2\Omega$}}] {} (O1);
\path[color=gray, very thick, shorten >=10pt, ->, >=stealth, bend left] (input) edge (I1);		\path[color=gray, very thick, shorten >=10pt, ->, >=stealth, bend right] (input) edge (I2);		
\path[color=gray, very thick, shorten >=10pt, ->, >=stealth] (output) edge (O1);
\end{tikzpicture}
\]
Then, given a second circuit, we may choose a relation between the output set of
the first and the input set of this second circuit, such as the simple relation
of the single output vertex of the circuit above with the single input vertex of
the circuit below.
\[
\begin{tikzpicture}[circuit ee IEC, set resistor graphic=var resistor IEC graphic]
\node[contact] (I1) at (0,1) {};
\node[contact] (O1) at (2.83,2) {};
\node[contact] (O2) at (2.83,0) {};
\node (input) at (-2,1) {\small{\textsf{inputs}}};
\node (output) at (4.83,1) {\small{\textsf{outputs}}};
\draw (I1) 	to [resistor] node [label={[label distance=2pt]95:{$1\Omega$}}] {} (O1);
\draw (I1)		to [resistor] node [label={[label distance=2pt]265:{$3\Omega$}}] {} (O2);
\path[color=gray, very thick, shorten >=10pt, ->, >=stealth] (input) edge (I1);		\path[color=gray, very thick, shorten >=10pt, ->, >=stealth, bend right] (output) edge (O1);
\path[color=gray, very thick, shorten >=10pt, ->, >=stealth, bend left] (output) edge (O2);
\end{tikzpicture}
\]
We connect the two circuits by identifying output and input vertices according
to this relation, giving in this case the composite circuit:
\[
\begin{tikzpicture}[circuit ee IEC, set resistor graphic=var resistor IEC graphic]
\node[contact] (I1) at (0,2) {};
\node[contact] (I2) at (0,0) {};
\coordinate (int1) at (2.83,1) {};
\coordinate (int2) at (5.83,1) {};
\node[contact] (O1) at (8.66,2) {};
\node[contact] (O2) at (8.66,0) {};
\node (input) at (-2,1) {\small{\textsf{inputs}}};
\node (output) at (10.66,1) {\small{\textsf{outputs}}};
\draw (I1) 	to [resistor] node [label={[label distance=2pt]85:{$1\Omega$}}] {} (int1);
\draw (I2)	to [resistor] node [label={[label distance=2pt]275:{$1\Omega$}}] {} (int1)
				to [resistor] node [label={[label distance=3pt]90:{$2\Omega$}}] {} (int2);
\draw (int2) 	to [resistor] node [label={[label distance=2pt]95:{$1\Omega$}}] {} (O1);
\draw (int2)		to [resistor] node [label={[label distance=2pt]265:{$3\Omega$}}] {} (O2);
\path[color=gray, very thick, shorten >=10pt, ->, >=stealth, bend left] (input) edge (I1);		\path[color=gray, very thick, shorten >=10pt, ->, >=stealth, bend right] (input) edge (I2);		
\path[color=gray, very thick, shorten >=10pt, ->, >=stealth, bend right] (output) edge (O1);
\path[color=gray, very thick, shorten >=10pt, ->, >=stealth, bend left] (output) edge (O2);
\end{tikzpicture}
\]

\vskip 1em

More formally, we define a \define{graph} to be a pair of functions $s,t\maps E \to N$ where $E$ and $N$ are finite sets.  We call elements of $E$ \define{edges} and elements of $N$ \define{vertices} or
\define{nodes}.  We say that the edge $e \in E$ has \define{source} $s(e)$ and
\define{target} $t(e)$, and also say that $e$ is an edge \define{from} $s(e)$
\define{to} $t(e)$.

To study circuits we need graphs with labelled edges:

\begin{definition}
Given a set $L$ of \define{labels}, an \define{$L$-graph} is a graph equipped with a function $r\maps E \to L$:
\[
\xymatrix{
L & E \ar@<2.5pt>[r]^{s} \ar@<-2.5pt>[r]_{t} \ar[l]_{r} & N.
}
\]
\end{definition}

For circuits made of resistors we take $L = (0,\infty)$, but later we shall
take $L$ to be a set of positive elements in some more general field.  In either
case, a circuit will be an $L$-graph with some extra structure:

\begin{definition} \label{def_circuit}
Given a set $L$, a \define{circuit over $L$} is an $L$-graph $\xymatrix{
L & E \ar@<2.5pt>[r]^{s} \ar@<-2.5pt>[r]_{t} \ar[l]_{r} & N}$ together with finite sets $X$, $Y$, and functions $i \maps X \to N$ and $o\maps Y \to  N$. We call the sets $i(X)$, $o(Y)$, and $\partial N = i(X) \cup o(Y)$ the \define{inputs},  \define{outputs}, and \define{terminals} or \define{boundary} of the circuit, respectively.
\end{definition}

We will later make use of the notion of connectedness in graphs. Recall that
given two vertices $v, w \in N$ of a graph, a \define{path from $v$ to $w$} is a
finite sequence of vertices $v = v_0, v_1, \dots , v_n = w$ and edges $e_1,
\dots , e_n$ such that for each $1 \le i \le n$, either $e_i$ is an edge from
$v_i$ to $v_{i+1}$, or an edge from $v_{i+1}$ to $v_i$. A subset $S$ of the
vertices of a graph is \define{connected} if, for each pair of vertices in $S$,
there is a path from one to the other. A \define{connected component} of a graph
is a maximal connected subset of its vertices.\footnote{In the theory of
directed graphs the qualifier `weakly' is commonly used before the word
`connected' in these two definitions, in distinction from a stronger notion of
connectedness requiring paths to respect edge directions. As we never consider
any other sort of connectedness, we omit this qualifier.}

In the rest of this section we take $L = (0,\infty) \subseteq \R$ and fix a circuit over 
$(0,\infty)$.  The edges of this circuit should be thought of as `wires'.  The label 
$r_e \in (0,\infty)$ stands for the \define{resistance} of the resistor on the wire $e$.   
There will also be a voltage and current on each wire.  In this section, these will
be specified by functions $V \in \R^E$ and $I \in \R^E$.  Here, as customary in
engineering, we use $I$ for `intensity of current', following Amp\`ere.  

\subsection{Ohm's law, Kirchhoff's laws, and the principle of minimum power}

In 1827, Georg Ohm published a book which included a relation between the voltage
and current for circuits made of resistors \cite{O}.  At the time, the critical
reception was harsh: one contemporary called Ohm's work ``a web of naked
fancies, which can never find the semblance of support from even the most
superficial of observations'', and the German Minister of Education said that a
professor who preached such heresies was unworthy to teach science \cite{D,H}.
However, a simplified version of his relation is now widely used under the name
of `Ohm's law'. We say that \define{Ohm's law} holds if for all edges $e \in
E$ the voltage and current functions of a circuit obey:
\[ 
V(e) = r(e) I(e).  \label{ohm}  
\]

Kirchhoff's laws date to Gustav Kirchhoff in 1845, generalising Ohm's work. They
were in turn generalized into Maxwell's equations a few decades later. We say
\define{Kirchhoff's voltage law} holds if there exists $\phi \in \R^N$ such that
\[
V(e) = \phi(t(e)) - \phi(s(e)).
\]
We call the function $\phi$ a \define{potential}, and think of it as assigning
an electrical potential to each node in the circuit. The voltage then arises as
the differences in potentials between adjacent nodes. If Kirchhoff's voltage law
holds for some voltage $V$, the potential $\phi$ is unique only in the trivial
case of the empty circuit: when the set of nodes $N$ is empty. Indeed, two
potentials define the same voltage function if and only if their difference is
constant on each connected component of the graph $\Gamma$.

We say \define{Kirchhoff's current law} holds if for all nonterminal nodes $n
\in N\setminus \partial N$ we have
\[ 
\sum_{s(e) = n} I(e) = \sum_{t(e) = n} I(e).  \label{kcl}  
\]  
This is an expression of conservation of charge within the circuit; it says that
the total current flowing in or out of any nonterminal node is zero. Even when
Kirchhoff's current law is obeyed, terminals need not be sites of zero net
current; we call the function $\iota \in \R^{\partial N}$ that takes a terminal
to the difference between the outward and inward flowing currents,
\begin{align*}
\iota:\partial N &\longrightarrow \R \\
n &\longmapsto \sum_{t(e) = n} I(e) -\sum_{s(e) = n} I(e),
\end{align*}
the \define{boundary current} for $I$.

A \define{boundary potential} is also a function in $\R^{\partial N}$, but
instead thought of as specifying potentials on the terminals of a
circuit. As we think of our circuits as open circuits, with the terminals points
of interaction with the external world, we shall think of these potentials as
variables that are free for us to choose. Using the above three
principles---Ohm's law, Kirchhoff's voltage law, and Kirchhoff's current
law---it is possible to show that choosing a boundary potential determines
unique voltage and current functions on that circuit. 

The so-called `principle of minimum power' gives some insight into how this
occurs, by describing a way potentials on the terminals might determine
potentials at all nodes. From this, Kirchhoff's voltage law then gives rise to a
voltage function on the edges, and Ohm's law gives us a current function too. We
shall show, in fact, that a potential satisfies the principle of minimum power
for a given boundary potential if and only if this current obeys Kirchhoff's
current law.

A circuit with current $I$ and voltage $V$ dissipates energy at a rate
equal to
\[
 \sum_{e \in E} I(e)V(e).
\]  
Ohm's law allows us to rewrite $I$ as $V/r$, while Kirchhoff's voltage law gives
us a potential $\phi$ such that $V(e)$ can be written as
$\phi(t(e))-\phi(s(e))$, so for a circuit obeying these two laws the power can
also be expressed in terms of this potential. We thus arrive at a functional
mapping potentials $\phi$ to the power dissipated by the circuit when Ohm's law
and Kirchhoff's voltage law are obeyed for $\phi$. 

\begin{definition}
The \define{extended power functional} $P\maps \R^N \to \R$ of a circuit is
defined by
\[
P(\phi) =\frac{1}{2} \sum_{e \in E} \frac{1}{r(e)}\big(\phi(t(e))-\phi(s(e))\big)^2.
\]
\end{definition}

\noindent
The factor of $\frac{1}{2}$ is inserted to cancel the factor of 2 that appears when
we differentiate this expression.  We call $P$ the \emph{extended} power functional as we shall see that it is defined even on potentials that are not compatible with the three governing laws of electric circuits. We shall later restrict the domain of this functional so that it is defined precisely on those potentials that \emph{are} compatible with the
governing laws. Note that this functional does not depend on the directions
chosen for the edges of the circuit.

This expression lets us formulate the `principle of minimum power', which gives
us information about the potential $\phi$ given its restriction to the boundary
of $\Gamma$. Call a potential $\phi \in \R^N$ an \define{extension} of a
boundary potential $\psi \in \R^{\partial N}$ if $\phi$ is equal to $\psi$ when
restricted to $\R^{\partial N}$---that is, if $\phi|_{\partial N} = \psi$. 

\begin{definition}
We say a potential $\phi \in \R^{N}$ \define{obeys the principle of minimum
power} for a boundary potential $\psi \in \R^{\partial N}$ if $\phi$ minimizes
the extended power functional $P$ subject to the constraint that  $\phi$ is an
extension of $\psi$. 
\end{definition}

As promised, in the presence of Ohm's law and Kirchhoff's voltage law, the
principle of minimum power is equivalent to Kirchhoff's current law.

\begin{proposition} \label{minimum_power_implies_kirchhoff_current}
Let $\phi$ be a potential extending some boundary potential $\psi$. Then $\phi$
obeys the principle of minimum power for $\psi$ if and only if the 
current 
\[  I(e) = \frac1{r(e)}(\phi(t(e))-\phi(s(e))) \] 
obeys Kirchhoff's current law.
\end{proposition}

\begin{proof}
Fixing the potentials at the terminals to be those given by the boundary
potential $\psi$, the power is a nonnegative quadratic function of the
potentials at the nonterminals. This implies that an extension $\phi$ of $\psi$
minimizes $P$ precisely when 
\[ \left. \frac{\partial P(\varphi)}{\partial \varphi(n)}\right|_{\varphi = \phi} = 0 \]
for all nonterminals $n \in N \setminus \partial N$. Note that the
partial derivative of the power with respect to the potential at $n$ is given by 
\begin{align*}
  \frac{\partial P}{\partial \varphi(n)}\bigg|_{\varphi = \phi} 
  &= \sum_{t(e) = n} \frac1{r(e)}\big(\phi(t(e))-\phi(s(e))\big) - \sum_{s(e) =
  n} \frac1{r(e)}\big(\phi(t(e))-\phi(s(e))\big) \\
  &= \sum_{t(e) = n} I(e) - \sum_{s(e) = n} I(e).
\end{align*}
Thus $\phi$ obeys the principle of minimum power for $\psi$ if and only if
\[ \sum_{s(e) = n} I(e) = \sum_{t(e) = n} I(e)\] 
for all $n \in N \setminus \partial N$, and so if and only if Kirchhoff's current law holds.
\end{proof}

\subsection{A Dirichlet problem}

We remind ourselves that we are in the midst of understanding circuits as objects that define relationships between boundary potentials and boundary currents. This relationship is defined by the stipulation that voltage--current pairs on a circuit must obey Ohm's law and Kirchhoff's laws---or equivalently, Ohm's law, Kirchhoff's voltage law, and the principle of minimum power. In this subsection we show these conditions imply that for each boundary potential $\psi$ on the circuit there exists a potential $\phi$ on the circuit extending $\psi$, unique up to what may be interpreted as a choice of reference potential on each connected component of the circuit. From this potential $\phi$ we can then compute the unique voltage, current, and boundary current functions compatible with the given boundary potential.

Fix again a circuit with extended power functional $P\maps \R^N \to \R$. Let $\nabla\maps \R^{N} \to \R^{N}$ be the operator that maps a potential $\phi \in \R^N$ to the function from $N$ to $\R$ given by
\[
n \longmapsto \frac{\partial P}{\partial \varphi(n)}\bigg|_{\varphi = \phi} \;.
\]
As we have seen, this function takes potentials to twice the pointwise currents that they induce. We have also seen that a potential $\phi$ is compatible with the governing laws of circuits if and only if
\begin{equation}
\nabla \phi \big|_{\R^{\partial N}} = 0 .\label{dirichlet}
\end{equation}
The operator $\nabla$ acts as a discrete analogue of the Laplacian for the graph $\Gamma$, so we call this operator the \define{Laplacian} of $\Gamma$, and say that  equation \eqref{dirichlet} is a version of Laplace's equation. We then say that the problem of finding an extension $\phi$ of some fixed boundary potential $\psi$ that solves this Laplace's equation---or, equivalently, the problem of finding a $\phi$ that obeys the principle of minimum power for $\psi$---is a discrete version of the \define{Dirichlet problem}. 

As we shall see, this version of the Dirichlet problem always has a solution.  However, the solution is not necessarily unique.  If we take a solution $\phi$ and some $\alpha \in \R^N$ that is constant on each connected component and vanishes on the boundary of $\Gamma$, it is clear that $\phi+\alpha$ is still an extension of $\psi$ and that 
\[
\left.\frac{\partial P(\varphi)}{\partial \varphi(n)}\right|_{\varphi = \phi} = 
\left.\frac{\partial P(\varphi)}{\partial \varphi(n)}\right|_{\varphi = \phi + \alpha},
\] 
so $\phi + \alpha$ is another solution. We say that a connected component of a circuit \define{touches the boundary} if it contains a vertex in $\partial N$. Note that such an $\alpha$ must vanish on all connected components touching the boundary.

With these preliminaries in hand, we can solve the Dirichlet problem:
\begin{proposition} \label{dirichlet_problem}
For any boundary potential $\psi \in \R^{\partial N}$ there exists a potential $\phi$ obeying the principle of minimum power for $\psi$.  If we also demand that $\phi$ vanish on every connected component of $\Gamma$ not touching the boundary, then $\phi$ is unique. 
\end{proposition}
\begin{proof}
For existence, observe that the power is a nonnegative quadratic form, the extensions of $\psi$ form an affine subspace of $\R^N$, and a nonnegative quadratic form restricted to an affine subspace of a real vector space must reach a minimum somewhere on this subspace. 

For uniqueness, suppose that both $\phi$ and $\phi'$ obey the principle of minimum power for $\psi$. Let 
\[
\alpha = \phi'-\phi.
\]
Then 
\[
\alpha\big|_{\partial N} = \phi'\big|_{\partial N}-\phi\big|_{\partial N} = \psi-\psi =0,
\] 
so $\phi+\lambda\alpha$ is an extension of $\psi$ for all $\lambda \in \R$. This implies that
\[
f(\lambda) := P(\phi+\lambda\alpha)
\]
is a smooth function attaining its minimum value at both $t =0$ and $t =1$. In particular, this implies that $f'(0)=0$. But this means that when writing $f$ as a quadratic, the coefficient of $\lambda$ must be $0$, so we can write
\begin{align*}
2f(\lambda) &= \sum_{e \in E} \frac1{r(e)}\big((\phi+\lambda\alpha)(t(e))-(\phi+\lambda\alpha)(s(e))\big)^2 \\
&= \sum_{e \in E} \frac1{r(e)}\Big(\big(\phi(t(e))-\phi(s(e))\big)+\lambda\big(\alpha(t(e))-\alpha(s(e))\big)\Big)^2 \\
&=  \sum_{e \in E} \frac1{r(e)}\big(\phi(t(e))-\phi(s(e))\big)^2 + \textrm{$\lambda$-term} +  \lambda^2 \sum_{e \in E} \frac1{r(e)}\big(\alpha(t(e))-\alpha(s(e))\big)^2 \\
&=  \sum_{e \in E} \frac1{r(e)}\big(\phi(t(e))-\phi(s(e))\big)^2 + \lambda^2 \sum_{e \in E} \frac1{r(e)}\big(\alpha(t(e))-\alpha(s(e))\big)^2.
\end{align*}
Then
\[
f(1) - f(0) 
= \frac{1}{2}\sum_{e \in E} \frac1{r(e)}\big(\alpha(t(e))-\alpha(s(e))\big)^2 =0,
\]
so $\alpha(t(e)) = \alpha(s(e))$ for every edge $e \in E$. This implies that $\a$ is constant on each connected component of the graph $\Gamma$ of our circuit. 

Note that as $\alpha|_{\partial N} = 0$, $\alpha$ vanishes on every connected component of $\Gamma$ touching the boundary. Thus, if we also require that $\phi$ and $\phi'$ vanish on every connected component of $\Gamma$ not touching the boundary, then $\alpha = \phi'-\phi$ vanishes on all connected components of $\Gamma$, and hence is identically zero. Thus $\phi' = \phi$, and this extra condition ensures a unique solution to the Dirichlet problem.
\end{proof}

We have also shown the following:

\begin{proposition}\label{dirichlet_problem_2}
Suppose $\psi \in \R^{\partial N}$ and $\phi$ is a potential obeying the principle of minimum power for $\psi$.  Then $\phi'$ obeys the principle of minimum power for $\psi$ if and only if the difference $\phi' - \phi$ is constant on every connected component of $\Gamma$ and vanishes on every connected component touching the boundary of $\Gamma$.
\end{proposition}

Furthermore, $\phi$ depends linearly on $\psi$:

\begin{proposition}\label{dirichlet_problem_3: linearity}
Fix $\psi \in \R^{\partial N}$, and suppose $\phi \in \R^N$ is the unique potential obeying the principle of minimum power for $\psi$ that vanishes on all connected components of $\Gamma$ not touching the boundary. Then $\phi$ depends linearly on $\psi$.
\end{proposition}
\begin{proof}
Fix $\psi, \psi' \in \R^{\partial N}$, and suppose $\phi, \phi' \in \R^N$ obey the principle of minimum power for $\psi,\psi'$ respectively, and that both $\phi$ and $\phi'$ vanish on all connected components of $\Gamma$ not touching the boundary. 

Then, for all $\lambda \in \R$,
\[
(\phi+\lambda\phi')\big|_{\R^{\partial N}} = \phi\big|_{\R^{\partial N}} +
\lambda\phi'\big|_{\R^{\partial N}}  = \psi + \lambda\psi'
\]
and
\[
(\nabla(\phi+\lambda\phi'))\big|_{\R^{\partial N}} = 
(\nabla\phi)\big|_{\R^{\partial N}} +
\lambda(\nabla\phi')\big|_{\R^{\partial N}}  = 0.
\]
Thus $\phi+\lambda\phi'$ solves the Dirichlet problem for $\psi+\lambda\psi'$, and thus $\phi$ depends linearly on $\psi$.
\end{proof}

Bamberg and Sternberg \cite{BS} describe another way to solve the Dirichlet problem, going back to Weyl \cite{Weyl}.

\subsection{Equivalent circuits}

We have seen that boundary potentials determine, essentially uniquely, the value of all the electric properties across the entire circuit. But from the perspective of control theory, this internal structure is irrelevant: we can only access the circuit at its terminals, and hence only need concern ourselves with the relationship between boundary potentials and boundary currents. In this section we streamline our investigations above to state the precise way in which boundary currents depend on boundary potentials. In particular, we shall see that the relationship is completely captured by the functional taking boundary potentials to the minimum power used by any extension of that boundary potential. Furthermore, each such power functional determines a different boundary potential--boundary current relationship, and so we can conclude that two circuits are equivalent if and only if they have the same power functional. 

An `external behaviour', or \define{behaviour} for short, is an equivalence class of circuits, where two are considered equivalent when the boundary current is the same function of the boundary potential. The idea is that the boundary current and boundary potential are all that can be observed `from outside', i.e. by making measurements at the terminals.  Restricting our attention to what can be observed by making measurements at the terminals amounts to treating a circuit as a `black box': that is, treating its interior as hidden from view.  So, two circuits give the same behaviour when they behave the same as `black boxes'.

First let us check that the boundary current is a function of the boundary potential.  For this we introduce an important quadratic form on the space of boundary potentials:

\begin{definition}
The \define{power functional} $Q \maps \R^{\partial N} \to \R$ of a circuit with extended power functional $P$ is given by
\[
 Q(\psi) = \min_{\phi|_{\R^{\partial N}} = \psi } P(\phi).
\]
\end{definition}

Proposition \ref{dirichlet_problem} shows the minimum above exists, so the power functional is well-defined.  Thanks to the principle of minimum power, $Q(\psi)$ equals $\frac{1}{2}$ times the power dissipated by the circuit when the boundary voltage is $\psi$.  We will later see that in fact $Q(\psi)$ is a nonnegative quadratic form on $\R^{\partial N}$. 

Since $Q$ is a smooth real-valued function on $\R^{\partial N}$, its differential $d Q$ at any given point $\psi \in \R^{\partial N}$ defines an element of the dual space $(\R^{\partial N})^\ast$, which we denote by $d Q_\psi$.  In fact, this element is equal to the boundary current $\iota$ corresponding to the boundary voltage $\psi$:

\begin{proposition} \label{boundary_current_determines_boundary_voltage}
Suppose $\psi \in \R^{\partial N}$.  Suppose $\phi$ is any extension of $\psi$ minimizing the power. Then $dQ_\psi \in (\R^{\partial N})^\ast \cong \R^{\partial N}$ gives the boundary current of the current induced by the potential $\phi$.
\end{proposition}

\begin{proof}
Note first that while there may be several choices of $\phi$ minimizing the power subject to the constraint that $\phi|_{\R^{\partial N}} = \psi$, Proposition \ref{dirichlet_problem_2} says that the difference between any two choices vanishes on all components touching the boundary of $\Gamma$.  Thus, these two choices give the same value for the boundary current $\iota\maps \partial N \to \R$. So, with no loss of generality we may assume $\phi$ is the unique choice that vanishes on all components not touching the boundary. Write $\overline\iota\maps N \to \R$ for the extension of $\iota\maps \partial N \to \R$ to $N$ taking value $0$ on $N \setminus \partial N$. 

By Proposition \ref{dirichlet_problem_3: linearity}, there is a linear operator
\[
f\maps \R^{\partial N} \longrightarrow \R^N
\]
sending $\psi \in \R^{\partial N}$ to this choice of $\phi$, and then
\[
Q(\psi) = P(f\psi).
\]
Given any $\psi' \in \R^{\partial N}$, we thus have
\begin{align*}
dQ_\psi(\psi') &= \frac{d}{d\lambda}Q(\phi +\lambda\psi') \bigg|_{\lambda=0} \\
&= \frac{d}{d\lambda}P(f(\psi+\lambda\psi'))\bigg|_{\lambda=0} \\
&= \frac{1}{2} \frac{d}{d\lambda}\sum_{e \in E} \frac1{r(e)}\bigg(f(\psi+\lambda\psi'))(t(e))-(f(\psi+\lambda\psi'))(s(e))\bigg)^2 \bigg|_{\lambda=0} \\
&= \frac{1}{2} \frac{d}{d\lambda}\sum_{e \in E} \frac1{r(e)}\bigg((f\psi(t(e))-f\psi(s(e))) \;+\;\lambda (f\psi'(t(e))- f\psi'(s(e)))\bigg)^2 \bigg|_{\lambda=0} \\
&= \sum_{e \in E} \frac1{r(e)}(f\psi(t(e))-f\psi(s(e)))(f\psi'(t(e))- f\psi'(s(e))) \\
&= \sum_{e \in E} I(e)(f\psi'(t(e))- f\psi'(s(e))) \\
&= \sum_{n \in N}\left(\sum_{t(e) = n} I(e) - \sum_{s(e) = n} I(e)\right)f\psi'(n) \\
&= \sum_{n \in N}\overline \iota(n) f\psi'(n) \\
&= \sum_{n \in \partial N}\iota(n) \psi'(n).
\end{align*}
This shows that $dQ_\psi^\ast = \iota$, as claimed.  Note that this calculation explains why we inserted a factor of $\frac{1}{2}$ in the definition of $P$: it cancels the factor of $2$ obtained from differentating a square.
 \end{proof}

Note this only depends on $Q$, which makes no mention of the potentials at
nonterminals. This is amazing: the way power depends on boundary potentials
completely characterizes the way boundary currents depend on boundary
potentials. In particular, in Part  we shall see that this
allows us to define a composition rule for behaviours of circuits.

To demonstrate these notions, we give a basic example of equivalent circuits.

\begin{example}[Resistors in series] \label{resistors_in_series}
Resistors are said to be placed in \define{series} if they are placed end to end or, more
precisely, if they form a path with no self-intersections. It is well known that
resistors in series are equivalent to a single resistor with resistance equal to
the sum of their resistances. To prove this, consider the following circuit
comprising two resistors in series, with input $A$ and output $C$:
\[
  \begin{tikzpicture}[circuit ee IEC, set resistor graphic=var resistor IEC graphic]
    \node[contact] (I1) at (0,0) [label=left:$A$] {};
    \node[circle, minimum width = 3pt, inner sep = 0pt, fill=black] (int) at (3,0) [label=above:$B$] {};
    \node[contact] (O1) at (6,0) [label=right:$C$] {};
    \draw (I1) 	to [resistor] node [label={[label distance=3pt]90:{$r_{AB}$}}] {} (int)
    to [resistor] node [label={[label distance=3pt]90:{$r_{BC}$}}] {} (O1);
  \end{tikzpicture}
\]
Now, the extended power functional $P\maps \R^{\{A,B,C\}} \to \R$ for this circuit is
\[
P(\phi) = \frac12\left(\frac1{r_{AB}}\big(\phi(A)-\phi(B)\big)^2 +
\frac1{r_{BC}}\big(\phi(B)-\phi(C)\big)^2\right),
\]
while the power functional $Q\maps \R^{\{A,C\}} \to \R$ is given by minimization
over values of $\phi(B) = x$:
\[
Q(\psi) = \min_{x \in \R} \frac12 \left(\frac1{r_{AB}}\big(\psi(A)-x\big)^2 + \frac1{r_{BC}}\big(x-\psi(C)\big)^2 \right). 
\]
Differentiating with respect to $x$, we see that this minimum occurs when
\[
\frac1{r_{AB}}\big(x-\psi(A)\big) + \frac1{r_{BC}}\big(x-\psi(C)\big) = 0,
\]
and hence when $x$ is the $r$-weighted average of $\psi(A)$ and $\psi(C)$:
\[
x = \frac{r_{BC}\psi(A) + r_{AB}\psi(C)}{r_{BC}+ r_{AB}}.
\]
Substituting this value for $x$ into the expression for $Q$ above and simplifying gives
\[
Q(\psi) = \frac12\cdot\frac1{r_{AB}+r_{BC}}\big(\psi(A)-\psi(C)\big)^2. 
\]
This is also the power functional of the circuit
\[
\begin{tikzpicture}[circuit ee IEC, set resistor graphic=var resistor IEC graphic]
\node[contact] (I1) at (0,0) [label=left:$A$] {};
\node[contact] (O1) at (3,0) [label=right:$C$] {};
\draw (I1) 	to [resistor] node [label={[label distance=3pt]90:{$r_{AB}+r_{BC}$}}] {} (O1);
\end{tikzpicture}
\]
and so the circuits are equivalent.
\end{example}


\subsection{Dirichlet forms}

In the previous subsection we claimed that power functionals are quadratic forms
on the boundary of the circuit whose behaviour they represent. They comprise, in
fact, precisely those quadratic forms known as Dirichlet forms.

\begin{definition}
Given a finite set $S$, a \define{Dirichlet form} on $S$ is a quadratic form $Q:
\mathbb{R}^S \to \mathbb{R}$ given by the formula
\[
  Q(\psi) = \sum_{i,j} c_{i j} (\psi_i - \psi_j)^2
\]
for some nonnegative real numbers $c_{i j}$.  
\end{definition}

Note that we may assume without loss of generality that $c_{i i} = 0$ and $c_{i
j} = c_{j i}$; we do this henceforth.  Any Dirichlet form is nonnegative:
$Q(\psi) \ge 0$ for all $\psi \in \mathbb{R}^S$.  However, not all nonnegative
quadratic forms are Dirichlet forms.  For example, if $S = \{1, 2\}$, the
nonnegative quadratic form $Q(\psi) = (\psi_1 + \psi_2)^2$ is not a Dirichlet
form. That said, the concept of Dirichlet form is vastly more general than the
above definition: such quadratic forms are studied not just on
finite-dimensional vector spaces $\mathbb{R}^S$ but on $L^2$ of any measure
space.  When this measure space is just a finite set, the concept of Dirichlet
form reduces to the definition above.  For a thorough introduction to Dirichlet
forms, see the text by Fukushima \cite{Fukushima}.  For a fun tour of the
underlying ideas, see the paper by Doyle and Snell \cite{DS}. 

The following characterizations of Dirichlet forms help illuminate the concept:

\begin{proposition} \label{dirichlet_characterizations}
  Given a finite set $S$ and a quadratic form $Q\maps \mathbb{R}^S \to \mathbb{R}$,
  the following are equivalent:
  \begin{enumerate}[(i)]
    \item $Q$ is a Dirichlet form.

    \item $Q(\phi) \le Q(\psi)$ whenever $|\phi_i - \phi_j| \le |\psi_i -
      \psi_j|$ for all $i, j$. 

    \item $Q(\phi) = 0$ whenever $\phi_i$ is independent of $i$, and $Q$ obeys
      the \define{Markov property}: $Q(\phi) \le Q(\psi)$ when $\phi_i = \min
      (\psi_i, 1) $.
  \end{enumerate}
\end{proposition}
\begin{proof}
See Fukushima \cite{Fukushima}.
\end{proof}

While the extended power functionals of circuits are evidently Dirichlet forms,
it is not immediate that all power functionals are. For this it is crucial that
the property of being a Dirichlet form is preserved under minimising over linear
subspaces of the domain that are generated by subsets of the given finite set.

\begin{proposition} \label{dirichlet_minimization}
  If $Q\maps \R^{S+T} \to \R$ is a Dirichlet form, then 
  \[
    \min_{\nu \in \R^T} Q(-,\nu)\maps \R^S \to \R 
  \]
  is Dirichlet.
\end{proposition}
\begin{proof}
  We first note that $\min_{\nu \in \R^S} Q(-,\nu)$ is a quadratic form. Again,
  $\min_{\nu \in \R^T} Q(-,\nu)$ is well-defined as a nonnegative quadratic form
  also attains its minimum on an affine subspace of its domain. Furthermore
  $\min_{\nu \in \R^T} Q(-,\nu)$ is itself a quadratic form, as the partial
  derivatives of $Q$ are linear, and hence the points at with these minima are
  attained depend linearly on the argument of $\min_{\nu \in \R^T} Q(-,\nu)$.

  Now by Proposition \ref{dirichlet_characterizations}, $Q(\phi) \le Q(\phi')$
  whenever $|\phi_i - \phi_j| \le |\phi'_i - \phi'_j|$ for all $i,j \in S+T$. In
  particular, this implies $\min_{\nu \in \R^T} Q(\psi,\nu) \le \min_{\nu \in
  \R^T} Q(\psi',\nu)$ whenever $|\psi_i - \psi_j| \le |\psi'_i - \psi'_j|$ for
  all $i,j \in S$. Using Proposition \ref{dirichlet_characterizations} again
  then implies that $\min_{\nu \in \R^T} Q(-,\nu)$ is a Dirichlet form.
\end{proof}


\begin{corollary}
  Let $Q \maps \R^{\partial N} \to \R$ be the power functional for some circuit. Then
  $Q$ is a Dirichlet form.
\end{corollary}
\begin{proof}
  The extended power functional $P$ is a Dirichlet form, and writing $\R^N=
  \R^{\partial N} \oplus \R^{N \setminus \partial N}$ allows us to write
  \[
    Q(-) =  \min_{\phi\in \R^{N \setminus \partial N}}
    P(-,\phi). \qedhere
  \]
\end{proof}

The converse is also true: simply construct the circuit with set of vertices
$\partial N$ and an edge of resistance $\frac{1}{2c_{ij}}$ between any $i,j \in
\partial N$ such that the term $c_{ij}(\psi_i - \psi_j)$ appears in the
Dirichlet form. This gives: 

\begin{proposition}
  A function $Q$ is the power functional for some circuit if and only if $Q$ is a
  Dirichlet form.
\end{proposition}

This is an expression of the `star-mesh transform', a well-known fact of
electrical engineering stating that every circuit of linear resistors is
equivalent to some complete graph of resistors between its terminals. For more
details see \cite{vLO}. We may interpret the proof of Proposition
\ref{dirichlet_minimization} as showing that intermediate potentials at minima
depend linearly on boundary potentials, in fact a weighted average, and that
substituting these into a quadratic form still gives quadratic form.

\bigskip

In summary, in this section we have shown the existence of a surjective function
\[
  \bigg\{\begin{array}{c} \mbox{circuits of linear resistors} \\ \mbox{ with
    boundary $\partial N$} \end{array} \bigg\} \longrightarrow \bigg\{
    \mbox{Dirichlet forms on $\partial N$}\bigg\}
\]
mapping two circuits to the same Dirichlet form if and only if they have the same
external behaviour.  In the next section we extend this result to encompass
inductors and capacitors too.


\section{Inductors and capacitors} \label{sec:plcs}
%%fakesubsection
The intuition gleaned from the study of resistors carries over to inductors and
capacitors too, to provide a framework for studying what are known as passive
linear networks. To understand inductors and capacitors in this way, however, we
must introduce a notion of time dependency and subsequently the Laplace
transform, which allows us to work in the so-called frequency domain. Here, like
resistors, inductors and capacitors simply impose a relationship of
proportionality between the voltages and currents that run across them. The
constant of proportionality is known as the impedance of the component.

As for resistors, the interconnection of such components may be understood, at
least formally, as a minimization of some quantity, and we may represent the
behaviours of this class of circuits with a more general idea of Dirichlet form.

\subsection{The frequency domain and Ohm's law revisited}

In broadening the class of electrical circuit components under examination, we
find ourselves dealing with components whose behaviours depend on the rates of
change of current and voltage with respect to time. We thus now consider
time-varying voltages $v \maps [0,\infty) \to \R$ and currents $i \maps
  [0,\infty) \to \R$, where $t \in [0,\infty)$ is a real variable representing
    time. For mathematical reasons, we
restrict these voltages and currents to only those with (i) zero initial
conditions (that is, $f(0) = 0$) and (ii) Laplace transform lying in the field
\[
  \R(s) = \left\{ Z(s) = \tfrac{P(s)}{Q(s)} \,\Big\vert\, P, Q \mbox{
  polynomials over $\R$ in $s$}, \, Q \ne 0 \right\}
\]
of real rational functions of one variable. 
%We don't need the currents and voltages to lie in this field!!!  They just
%need to lie in some vector space over this field!!!
While it is possible that
physical voltages and currents might vary with time in a more general way, we
restrict to these cases as the rational functions are, crucially, well-behaved
enough to form a field, and yet still general enough to provide arbitrarily
close approximations to currents and voltages found in standard applications.

An \define{inductor} is a two-terminal circuit component across which the voltage is
proportional to the rate of change of the current. By convention we draw this as
follows, with the inductance $L$ the constant of proportionality:\footnote{We
  follow the standard convention of denoting inductance by the letter $L$, after
  the work of Heinrich Lenz and to avoid confusion with the $I$ used for
current.}
\[
  \begin{tikzpicture}[circuit ee IEC]
    \node[contact] (I1) at (0,0) {};
    \node[contact] (I2) at (1.83,0) {};
    \draw (I1) 	to [inductor] node [label={[label distance=2pt]{$L$}}]
    {} (I2);
  \end{tikzpicture}
\]
Writing $v_L(t)$ and $i_L(t)$ for the voltage and current over time $t$ across
this component respectively, and using a dot to denote the derivative with
respect to time $t$, we thus have the relationship 
\[
  v_L(t) = L\, \dot{i}_L(t).
\]
Permuting the roles of current and voltage, a \define{capacitor} is a two-terminal
circuit component across which the current is proportional to the rate of change
of the voltage. We draw this as follows, with the capacitance $C$ the constant
of proportionality:
\[
  \begin{tikzpicture}[circuit ee IEC]
    \node[contact] (I1) at (0,0) {};
    \node[contact] (I2) at (1.83,0) {};
    \draw (I1) 	to [capacitor] node [label={[label distance=5pt]{$C$}}]
    {} (I2);
  \end{tikzpicture}
\]
Writing $v_C(t)$, $i_C(t)$ for the voltage and current across the capacitor,
this gives the equation
\[
  i_C(t) = C\, \dot{v}_C(t).
\]
We assume here that inductances $L$ and capacitances $C$ are positive real numbers.

Although inductors and capacitors impose a linear relationship if we involve the
derivatives of current and voltage, to mimic the above work on resistors we wish
to have a constant of proportionality between functions representing the current
and voltage themselves. Various integral transforms perform just this role; electrical
engineers typically use the Laplace transform. This lets us write a function of time $t$ instead as a function of frequencies $s$, and in doing so turns differentiation with respect to $t$ into multiplication by $s$, and integration with respect to $t$ into
division by $s$.  

In detail, given a function $f(t)\maps [0, \infty) \to \R$, we define the
\define{Laplace transform} of $f$
\[
  \mathfrak{L}\{f\}(s) = \int_{0}^\infty f(t) e^{-st} dt.
\]
We also use the notation $\mathfrak{L}\{f\}(s) = F(s)$, denoting the Laplace
transform of a function in upper case, and refer to the Laplace transforms as
lying in the \define{frequency domain} or \define{$s$-domain}. For us, the three
crucial properties of the Laplace transform are then: 
\begin{enumerate}[(i)]
  \item linearity: $\mathfrak{L}\{af+bg\}(s) = aF(s)+bG(s)$ for $a,b\in \R$;
  \item differentiation: $\mathfrak{L}\{\dot{f}\}(s) = s F(s) - f(0)$;
  \item integration: if $g(t) = \int_0^t f(\tau)d\tau$ then 
 $G(s) = \frac{1}{s} F(s)$.
\end{enumerate}
Writing $V(s)$ and $I(s)$ for the Laplace transform of the voltage $v(t)$ and
current $i(t)$ across a component respectively, and recalling that by assumption
$v(t) = i(t) = 0$ for $t \le 0$, the $s$-domain behaviours of components become,
for a resistor of resistance $R$:
\[
  V(s) = RI(s),
\]
for an inductor of inductance $L$:
\[
  V(s) = sLI(s),
\]
and for a capacitor of capacitance $C$:
\[
  V(s) = \frac1{sC} I(s). 
\]

Note that for each component the voltage equals the current times a rational function of
the real variable $s$, called the \define{impedance} and in general denoted by $Z$.
Note also that the impedance is a \define{positive real function}, meaning that it lies
in the set
\[         \R(s)^+ = \{ Z \in \R(s) : \forall s \in \C \;\; \mathrm{Re}(s) > 0 \implies 
\mathrm{Re}(Z(s)) > 0 \} . \]
While $Z$ is a quotient of polynomials with real cofficients, in this definition
we are applying it to complex values of $s$, and demanding that its real part be
positive in the open left half-plane.  Positive real functions were introduced by Otto 
Brune in 1931, and they play a basic role in circuit theory \cite{Brune}.  

Indeed, Brune convincingly argued that for any conceivable passive linear
component with two terminals we have this generalization of Ohm's law:
\[
  V(s)=Z(s)I(s)
\]
where $I \in \R(s)$ is the \define{current}, $V \in \R(s)$ is the \define{voltage}
and $Z \in \R(s)^+$ is the \define{impedance} of the component.   As we shall
see, generalizing from circuits of linear resistors to arbitrary passive linear
circuits is just a matter of formally replacing resistances by  impedances.
This amounts to replacing the field $\R$ by the larger field $\R(s)$, and
replacing the set of positive reals, $\R^+ = (0,\infty)$, by the set of positive
real functions, $\R(s)^+$.  From a mathematical perspective we might as well
work with any field with a notion of `positive element'. 

\begin{definition} 
  Given a field $\F$, we define a \define{set of positive elements} for $\F$ to be  
  any subset $\F^+\subset \F$ not containing $0$.
\end{definition}

Our first motivating example arises from circuits made of resistors.  Here $\F =
\R$ is the field of real numbers and we take $\F^+ = (0,\infty)$.   Our second
motivating example arises from general passive linear circuits.  Here $\F =
\R(s)$ is the field of rational functions in one real variable, and we take
$\F^+ = \R(s)^+$ to be the positive real functions, as defined in the last
section.  

In all that follows, we fix a field $\F$ of characteristic zero equipped with a
set of positive elements $\F^+$.  By a `circuit', we shall henceforth mean a
circuit over $\F^+$, as explained in Definition \ref{def_circuit}.   To fix the
notation:

\begin{definition} \label{def_circuit_2}
A \define{(passive linear) circuit} is a graph $s,t \maps E \to N$ with $E$ as its set of \define{edges} and $N$ as its set of \define{nodes}, equipped with function $Z \maps E \to \F^+$ assigning each edge an \define{impedance}, together with finite sets $X$, $Y$, and functions $i \maps X \to N$ and $o\maps Y \to  N$. We call the sets $i(X)$, $o(Y)$, and $\partial N = i(X) \cup o(Y)$ the \define{inputs}, \define{outputs}, and \define{terminals} or \define{boundary} of the circuit, respectively.
\end{definition}

We next remark on the analogy between electronics and mechanics in light of
these new components.

\subsection{The mechanical analogy} \label{sec:mechanical}

Now that we have introduced inductors and capacitors, it is worth taking 
another glance at the analogy chart in Section \ref{sec:intro}.  What are the
analogues of resistance, inductance and capacitance in mechanics?  If we restrict attention to systems with translational degrees of freedom, the answer is given in the following
chart.

\begin{small}
\begin{center}
\begin{tabular}{|c|c|}
\hline
Electronics & Mechanics (translation) \\
\hline\hline
charge $Q$ & position $q$ \\
\hline
current $i = \dot Q$ & velocity $v = \dot q$ \\
\hline
flux linkage $\lambda$ & momentum $p$ \\
\hline
voltage $v = \dot \lambda$ & force $F = \dot p$ \\
\hline
resistance $R$ & damping coefficient $c$ \\
\hline
inductance $L$ & mass $m$ \\
\hline
inverse capacitance $C^{-1}$ & spring constant $k$ \\
\hline
\end{tabular}
\end{center}
\end{small}

A famous example concerns an electric circuit with a resistor of resistance $R$, an inductor of inductance $L$, and a capacitor of capacitance $C$, all in series:
\[
  \begin{tikzpicture}[circuit ee IEC, set resistor graphic=var resistor IEC graphic]
    \node[contact] (I1) at (0,0) {};
    \node[contact] (I2) at (1.83,0) {};
    \node[contact] (I3) at (3.66,0) {};
    \node[contact] (I4) at (5.49,0) {};
    \draw (I1) 	to [resistor] node [label={[label distance=2pt]{$R$}}]
    {} (I2);
    \draw (I2) 	to [inductor] node [label={[label distance=5pt]{$L$}}]
    {} (I3);
     \draw (I3) 	to [capacitor] node [label={[label distance=5pt]{$C$}}]
    {} (I4);
  \end{tikzpicture}
\]
We saw in Example \ref{resistors_in_series} that for resistors in series, the
resistances add.  The same fact holds more generally for passive linear circuits,
so the impedance of this circuit is the sum
\[   Z = s L + R + (sC)^{-1}  .\]
Thus, the voltage across this circuit is related to the current through the
circuit by
\[  V(s) = (s L + R + (sC)^{-1}) I(s)  \]
If $v(t)$ and $i(t)$ are the voltage and current as functions of time, we conclude that
\[  v(t) = L \frac{d}{dt}i(t) + Ri(t) + C^{-1} \int_0^t i(s) \, ds  \]
It follows that 
\[   
L \ddot{Q} + R \dot{Q} + C^{-1} Q = v
\]
where $Q(t) = \int_0^t i(t) ds$ has units of charge.  As the chart above suggests,
this equation is analogous to that of a damped harmonic oscillator:
\[    
m \ddot{q} + c \dot{q} + k q = F 
\]
where $m$ is the mass of the oscillator, $c$ is the damping coefficient, $k$ is 
the spring constant and $F$ is a time-dependent external force.

For details, and many more analogies of this sort, see the book by Karnopp, 
Margolis and Rosenberg \cite{KRM} or Brown's enormous text \cite{Brown}.   While it would be a distraction to discuss them further here, these analogies mean that our
work applies to a wide class of networked systems, not just electrical circuits.

\subsection{Generalized Dirichlet forms} \label{sec:generalized}

To understand the behaviour of passive linear circuits we need to
understand how the behaviours of individual components, governed by Ohm's law,
fit together give the behaviour of an entire network.
Kirchhoff's laws still hold, and so does a version of the principle of minimum power.

Like before, to each passive linear circuit we associate a generalised Dirichlet
form.

\begin{definition} Given a field $\F$ and
  a finite set $S$, a \define{Dirichlet form over $\F$} on $S$ is a quadratic form   
  $Q\maps \F^S \to \F$ given by the formula 
  \[ 
    Q(\psi) = \sum_{i,j \in S} c_{ij} (\psi_i - \psi_j)^2,
  \]
  where $c_{i j} \in \F$.  
\end{definition}

Generalizing from circuits of resistors, we define the
\define{extended power functional} $P\maps \F^N \to \F$ of any circuit by
\[
  P(\varphi) = \frac{1}{2} \sum_{e \in E}
  \frac1{Z(e)}\big(\varphi(t(e))-\varphi(s(e))\big)^2.
\]
and we call $\varphi \in \F^N$ a \define{potential}. Note that $\F$ has
characteristic not equal to 2, so dividing by 2 is allowed.  Note also that the
extended power functional is a Dirichlet form on $N$.

Although it is not clear what it means to minimize over the field $\F$, we can
use formal derivatives to formulate an analogue of the principle of minimum 
power.   This will actually be a `variational principle', saying the derivative of
the power functional vanishes with respect to certain variations in the potential.
As before we shall see that given Ohm's law, this principle is equivalent to
Kirchhoff's current law.

Indeed, the extended power functional $P(\varphi)$ can be considered an element
of the polynomial ring $\F[\{\varphi(n)\}_{n \in N}]$ generated by formal
variables $\varphi(n)$ corresponding to potentials at the nodes $n \in N$. We
may thus take formal derivatives of the extended power functional with respect
to the $\varphi(n)$.  We then call $\phi \in \F^N$ a \define{realizable
potential} for the given circuit if for each nonterminal node, $n \in N\setminus
\partial N$, the formal partial derivative of the extended power functional with
respect to $\varphi(n)$ equals zero when evaluated at $\phi$:
\[
  \frac{\partial P}{\partial \varphi(n)}\bigg\vert_{\varphi = \phi} = 0
\]
This terminology arises from the following fact, a generalization of Proposition
\ref{minimum_power_implies_kirchhoff_current}:

\begin{theorem} \label{thm:realizablepotentials}
The potential $\phi \in \F^N$ is a realizable potential for a given
circuit if and only if the induced current 
\[  I(e) = \frac1{Z(e)}(\phi(t(e))-\phi(s(e))) \]
obeys Kirchhoff's current law:
\[ 
\sum_{s(e) = n} I(e) = \sum_{t(e) = n} I(e)
\]  
for all $n \in N\setminus \partial N$.
\end{theorem}
\begin{proof}
The proof of this statement is exactly that for Proposition
\ref{minimum_power_implies_kirchhoff_current}. 
\end{proof}

A corollary of Theorem \ref{thm:realizablepotentials} is that the set of
states---that is, potential--current pairs---that are compatible with the
governing laws of a circuit is given by the set of realizable potentials
together with their induced currents. 

\subsection{A generalized minimizability result}

We begin to move from a discussion of the intrinsic behaviours of circuits to a
discussion of their behaviours under composition. The key fact for composition
of generalized Dirichlet forms is that, in analogy with Proposition
\ref{dirichlet_minimization}, we may speak of a formal version of minimization
of Dirichlet forms. We detail this here.  In what follows let $P$ be a Dirichlet
form over $\F$ on some finite set $S$. 

Recall that given $R \subseteq S$, we
call $\tilde\psi \in \F^S$ an \define{extension} of $\psi \in \F^R$ if
$\tilde\psi$ restricted to $R$ equals $\psi$.   We call such an
extension \define{realizable} if 
\[
    \frac{\partial P}{\partial \varphi(s)}\bigg\vert_{\varphi = \tilde\psi} = 0
  \]
for all $s \in S \setminus R$.  Note that over the real numbers $\R$ this means
that among all the extensions of $\psi$, $\tilde\psi$ minimizes the function $P$.

\begin{theorem} \label{thm:dirichletminimization}
  Let $P$ be a Dirichlet form over $\F$ on $S$, and let $R \subseteq S$ be an
  inclusion of finite sets. Then we may uniquely define a Dirichlet form
  \[\min_{S \setminus R}P: \F^R \to \F\] 
   on $R$ by sending each $\psi \in \F^R$ to
  the value $P(\tilde\psi)$ of any realizable extension $\tilde\psi$ of $\psi$.
\end{theorem}


To prove this theorem, we must first show that $\min_{S \setminus R} P$ is well-defined as a function.

\begin{lemma} \label{lem:welldefineddirichletmin}
  Let $P$ be a Dirichlet form over $\F$ on $S$, let $R \subseteq S$ be an
  inclusion of finite sets, and let $\psi \in \F^R$. Then for all realizable
  extensions $\tilde\psi$, $\tilde\psi' \in \F^S$ of $\psi$ we have $P(\tilde\psi) =
  P(\tilde\psi')$. 
\end{lemma}
\begin{proof}
  This follows from the formal version of the multivariable Taylor theorem for
  polynomial rings over a field of characteristic zero. Let $\tilde\psi$,
  $\tilde\psi' \in \F^S$ be realizable extensions of $\psi$, and note that
  $dP_{\tilde\psi}(\tilde\psi-\tilde\psi')=0$, since for all $s \in R$ we have
  $\tilde\psi(s) -\tilde\psi'(s) =0$, and for all $s \in S \setminus R$ we have
  \[
    \frac{\partial P}{\partial \varphi(s)}\bigg\vert_{\varphi = \tilde\psi}=0. 
  \]
  We may take the Taylor expansion of $P$ around $\tilde\psi$ and evaluate at
  $\tilde\psi'$. As $P$ is a quadratic form, this gives
  \begin{align*}
    P(\tilde\psi') &=
    P(\tilde\psi)+dP_{\tilde\psi}(\tilde\psi'-\tilde\psi)+P(\tilde\psi'-\tilde\psi)
    \\
    & = P(\tilde\psi)+P(\tilde\psi'-\tilde\psi).
  \end{align*}
  Similarly, we arrive at  
  \[
    P(\tilde\psi)= P(\tilde\psi')+P(\tilde\psi-\tilde\psi').
  \]
  But again as $P$ is a quadratic form, we then see that 
  \[
    P(\tilde\psi')-P(\tilde\psi) = P(\tilde\psi'-\tilde\psi) =
    P(\tilde\psi-\tilde\psi') = P(\tilde\psi)-P(\tilde\psi').
  \]
  This implies that $P(\tilde\psi')-P(\tilde\psi) = 0$, as required.
\end{proof}

It remains to show that $\min P$ remains a Dirichlet form. We do this
inductively.

\begin{lemma} \label{lem:onestepdirichletmin}
  Let $P$ be a Dirichlet form over $\F$ on $S$, and let $s \in S$ be an element
  of $S$. Then the map $\min_{\{s\}} P:\F^{S \setminus\{s\}} \to \F$ sending
  $\psi$ to $P(\tilde\psi)$ is a Dirichlet form on $S \setminus \{s\}$.
\end{lemma}
\begin{proof}
  Write $P(\phi) = \sum_{i,j} c_{ij}(\phi_i -\phi_j)^2$, assuming without loss
  of generality that $c_{sk} =0$ for all $k$. We then have
  \[
    \frac{\partial P}{\partial \varphi(s)}\bigg\vert_{\varphi = \phi} = \sum_k
    2c_{ks}(\phi_s-\phi_k),
  \]
  and this is equal to zero when
  \[
    \phi_s = \frac{\sum_k c_{ks}\phi_k}{\sum_k c_{ks}}.
  \]
  Thus $\min_{\{s\}}P$ may be given explicitly by the expression
  \[
    \min_{\{s\}} P(\psi) = \sum_{i,j \in S \setminus \{s\}} c_{ij}(\psi_i -\psi_j)^2 +
    \sum_{\ell \in S \setminus \{s\}} c_{\ell s}\left(\psi_\ell - \tfrac{\sum_k
      c_{ks} \psi_k}{\sum_k c_{ks}}\right)^2.
  \]
  We must show this is a Dirichlet form on $S \setminus \{s\}$. 
  
  As the sum of Dirichlet forms is evidently Dirichlet, it suffices to check that the expression 
  \[
    \sum_\ell c_{\ell s}\left(\psi_\ell - \tfrac{\sum_k c_{ks} \psi_k}{\sum_k
      c_{ks}}\right)^2
  \]
  is Dirichlet on $S \setminus \{s\}$. Multiplying through by the constant
  $(\sum_k c_{ks})^2 \in \F^+$, it further suffices to check
  \begin{align*}
    &\quad \sum_\ell c_{\ell s}\left(\sum_k c_{ks} \psi_\ell - \sum_k c_{ks}
    \psi_k\right)^2 \\
    &= \sum_\ell c_{\ell s} \left(\sum_k c_{ks} (\psi_\ell -
    \psi_k)\right)^2 \\
    &= \sum_\ell c_{\ell s} \left(2 \sum_{\substack{k,m \\ k \ne m}} c_{k s} c_{ms}
    (\psi_\ell-\psi_k)(\psi_\ell - \psi_m) + \sum_{k} c_{k
    s}^2(\psi_\ell-\psi_k)^2\right) \\
    &= 2\sum_{\substack{k,\ell,m \\ k \ne m}} c_{\ell s} c_{k s} c_{ms}
    (\psi_\ell-\psi_k)(\psi_\ell - \psi_m) + \sum_{k, \ell} c_{\ell s}c_{k
    s}^2(\psi_\ell-\psi_k)^2
  \end{align*}
  is Dirichlet. But
  \begin{align*}
    &\quad (\psi_k - \psi_\ell)(\psi_k - \psi_m)+(\psi_\ell - \psi_k)(\psi_\ell -
    \psi_m) + (\psi_m-\psi_k)(\psi_m-\psi_\ell) \\ 
    &= \psi_k^2+\psi_\ell^2+\psi_m^2-\psi_k\psi_\ell- \psi_k\psi_m -
    \psi_\ell\psi_m \\
    &= \tfrac12\big( (\psi_k-\psi_\ell)^2 +(\psi_k-\psi_m)^2
    +(\psi_\ell-\psi_m)^2\big),
  \end{align*}
  so this expression is indeed Dirichlet. Indeed, pasting these computations
  together shows that
  \[
    \min_{\{s\}}P(\psi) = \sum_{i,j} \left(c_{ij}+\frac{c_{is}c_{js}}{{\textstyle \sum_k}
    c_{ks}}\right)(\psi_i-\psi_j)^2. \qedhere
  \]
\end{proof}

With these two lemmas, the proof of Theorem \ref{thm:dirichletminimization}
becomes straightforward.

\begin{proof}[Proof of Theorem \ref{thm:dirichletminimization}]
  Lemma \ref{lem:welldefineddirichletmin} shows that $\min_{S \setminus R}P$ is a well-defined
  function. As $R$ is a finite set, we may write it $R = \{s_1,\dots, s_n\}$ for
  some natural number $n$. Then we may define a sequence of functions $P_i =
  \min_{\{s_1, \dots,s_i\}} P_{i-1}$, $1 \le i\le n$. Define also $P_0 = P$, and note
  that $P_n = \min_{S \setminus R}P$. Then, by Lemma
  \ref{lem:onestepdirichletmin}, each $P_i$ is Dirichlet as $P_{i-1}$
  is. This proves the proposition.
\end{proof}

We can thus define the power functional of a circuit by analogy with circuits made
of resistors:

\begin{definition}
The \define{power functional} $Q \maps \R^{\partial N} \to \R$ of a circuit with extended power functional $P$ is given by
\[
 Q = \min_{N \setminus \partial N}  P .
\]
\end{definition}

As before, we call two circuits equivalent if they have the same power
functional, and define the \define{behaviour} of a circuit to be its equivalence
class. 

We now have a complete description of the semantics of diagrams of passive
linear networks. The purpose of this chapter is to show how to use decorated
corelations to make this semantics compositional. We make a start on this in the
next section.


\section{A compositional perspective} \label{sec:circdef}
%%fakesubsection
Thus far our focus has been on the semantics of circuit diagrams, explaining how
labelled graphs represent Dirichlet forms. We now wish to work towards a
\emph{compositional} semantics. 

A prerequisite for this is compositional structure on our circuit diagrams and
Dirichlet forms. We desire a compositional structure that captures the syntax of
circuit diagrams, addressing the question: ``How do we interact with circuit
diagrams?'' Informally, the answer is that we interact with them by connecting them
to each other, perhaps after moving them into the right form by rotating or
reflecting them, or by crossing, bending, splitting, and combining some of the
wires. In line with the overarching philosophy of this thesis, it should come as
no surprise that we hence model circuits as morphisms in a hypergraph category. 

We begin this section by defining a hypergraph category of circuits. We then
want to define a hypergraph category of behaviours. We have three desiderata for
such a category. First, the category of behaviours should have a unique morphism
for each distinct circuit behaviour. Second, the map taking a circuit to its
behaviour should be structure preserving: that is, it should be a hypergraph
functor to a hypergraph category. The third desideratum is one of aesthetics:
the category of behaviours and its composition rule should be easy to define and
work with. 

It turns out, as we shall see, that Dirichlet forms are not quite general enough
to accommodate our needs. In Subsection \ref{ssec.noidentities} we discuss an
attempt to construct a `natural' category of Dirichlet forms, with composition
of Dirichlet forms given by taking their sum and minimising over the `interior'
terminals, but find that this operation does not have an identity. Despite this,
we give an ad hoc construction of a category of Dirichlet corelations that
at least fulfils the two technical desiderata above.

\subsection{The category of open circuits}

In Definition \ref{def_circuit_2}, we defined a circuit of linear resistors to
be a labelled graph with marked input and output terminals, as in the example:
\[
\begin{tikzpicture}[circuit ee IEC, set resistor graphic=var resistor IEC graphic]
\node[contact] (I1) at (0,2) {};
\node[contact] (I2) at (0,0) {};
\coordinate (int1) at (2.83,1) {};
\coordinate (int2) at (5.83,1) {};
\node[contact] (O1) at (8.66,2) {};
\node[contact] (O2) at (8.66,0) {};
\node (input) at (-2,1) {\small{\textsf{inputs}}};
\node (output) at (10.66,1) {\small{\textsf{outputs}}};
\draw (I1) 	to [resistor] node [label={[label distance=2pt]85:{$1\Omega$}}] {} (int1);
\draw (I2)	to [resistor] node [label={[label distance=2pt]275:{$1\Omega$}}] {} (int1)
				to [resistor] node [label={[label distance=3pt]90:{$2\Omega$}}] {} (int2);
\draw (int2) 	to [resistor] node [label={[label distance=2pt]95:{$1\Omega$}}] {} (O1);
\draw (int2)		to [resistor] node [label={[label distance=2pt]265:{$3\Omega$}}] {} (O2);
\path[color=gray, very thick, shorten >=10pt, ->, >=stealth, bend left] (input) edge (I1);		\path[color=gray, very thick, shorten >=10pt, ->, >=stealth, bend right] (input) edge (I2);		
\path[color=gray, very thick, shorten >=10pt, ->, >=stealth, bend right] (output) edge (O1);
\path[color=gray, very thick, shorten >=10pt, ->, >=stealth, bend left] (output) edge (O2);
\end{tikzpicture}
\]
We then defined general passive linear circuits by replacing resistances with
impedances chosen from a set of positive elements $\F^+$ in any field $\F$.
These circuits are examples of decorated cospans.  We now use decorated cospans
to construct a hypergraph category $\Circ$ whose morphisms are circuits, such
that the hypergraph structure expresses the syntactic operations on circuits
discussed above.

Indeed, observe that a circuit is just an $\F^+$-graph, as defined in the
introduction to Chapter \ref{ch.deccospans}.  Recall from Subsection
\ref{ssec.exlabelledgraphs} the lax symmetric monoidal functor
\[
  \mathrm{Graph}\maps (\mathrm{FinSet},+) \longrightarrow (\mathrm{Set},\times)
\]
mapping each finite set $N$ to the set $\mathrm{Graph}(N)$ of
$[0,\infty)$-graphs $(N,E,s,t,r)$ with $N$ as their set of nodes. Define the lax
symmetric monoidal functor $\mathrm{Circuit}$, generalising $\mathrm{Graph}$,
so that the set $N$ is mapped to the set of $\F^+$-graphs.
  
\begin{definition}
  We define the hypergraph category
  \[
    \mathrm{Circ} = \mathrm{CircuitCospan} .
  \]
\end{definition}

\begin{aside}
  As discussed in Subsection \ref{ssec.bicatdeccospan}, by the work of Courser
  \cite{Cou16} we in fact have a symmetric monoidal bicategory $2\mbox{-}\Circ$
  of circuits with 
  \begin{center}
    \begin{tabular}{ c | p{.65\textwidth} }
      \textbf{objects} & finite sets \\ 
      \textbf{morphisms} & cospans of finite sets decorated by $\F^+$-graphs \\ 
      \textbf{2-morphisms} & maps of decorated cospans \\
    \end{tabular}
  \end{center}
  Moreover, the work of Stay \cite{Sta16} can be used to show this bicategory is
  compact closed.  Decategorifying $2\mbox{-}\Circ$ gives a category equivalent
  to our previously defined category $\Circ$.
\end{aside}

The hypergraph structure, including the compactness and dagger, captures the
aforementioned syntactic operations that can be performed on circuits. The
composition expresses the fact that we can connect the outputs of one circuit to
the inputs of the next, like so:
\[
  \begin{array}{c}
    \begin{tikzpicture}[circuit ee IEC, set resistor graphic=var resistor IEC
      graphic,scale=.7]
      \node[circle,draw,inner sep=1pt,fill=gray,color=gray]         (x) at
      (-3,-1.3) {};
      \node at (-3,-3.2) {\footnotesize $X$};
      \node[circle,draw,inner sep=1pt,fill]         (A) at (0,0) {};
      \node[circle,draw,inner sep=1pt,fill]         (B) at (3,0) {};
      \node[circle,draw,inner sep=1pt,fill]         (C) at (1.5,-2.6) {};
      \node[circle,draw,inner sep=1pt,fill=gray,color=gray]         (y1) at
      (6,-.6) {};
      \node[circle,draw,inner sep=1pt,fill=gray,color=gray]         (y2) at
      (6,-2) {};
      \node at (6,-3.2) {\footnotesize $Y$};
      \coordinate         (ua) at (.5,.25) {};
      \coordinate         (ub) at (2.5,.25) {};
      \coordinate         (la) at (.5,-.25) {};
      \coordinate         (lb) at (2.5,-.25) {};
      \path (A) edge (ua);
      \path (A) edge (la);
      \path (B) edge (ub);
      \path (B) edge (lb);
      \path (ua) edge  [resistor, circuit symbol unit=5pt, circuit symbol size=width {5} height 1.5] node[above] {\footnotesize $2\Omega$} (ub);
      \path (la) edge  [resistor, circuit symbol unit=5pt, circuit symbol size=width {5} height 1.5] node[below] {\footnotesize $3\Omega$} (lb);
      \path (A) edge  [resistor, circuit symbol unit=5pt, circuit symbol size=width {5} height 1.5] node[left] {\footnotesize $1\Omega$} (C);
      \path (C) edge  [resistor, circuit symbol unit=5pt, circuit symbol size=width {5} height 1.5] node[right] {\footnotesize $1\Omega$} (B);
      \path[color=gray, very thick, shorten >=10pt, shorten <=5pt, ->, >=stealth] (x) edge (A);
      \path[color=gray, very thick, shorten >=10pt, shorten <=5pt, ->, >=stealth] (y1) edge (B);
      \path[color=gray, very thick, shorten >=10pt, shorten <=5pt, ->, >=stealth] (y2)
      edge (B);
      \node[circle,draw,inner sep=1pt,fill]         (A') at (9,0) {};
      \node[circle,draw,inner sep=1pt,fill]         (B') at (12,0) {};
      \node[circle,draw,inner sep=1pt,fill]         (C') at (10.5,-2.6) {};
      \node[circle,draw,inner sep=1pt,fill=gray,color=gray]         (z1) at
      (15,-.6) {};
      \node[circle,draw,inner sep=1pt,fill=gray,color=gray]         (z2) at (15,-2) {};
      \node at (15,-3.2) {\footnotesize $Z$};
      \path (A') edge  [resistor, circuit symbol unit=5pt, circuit symbol size=width {5} height 1.5] node[above] {\footnotesize $5\Omega$} (B');
      \path (C') edge  [resistor, circuit symbol unit=5pt, circuit symbol size=width {5} height 1.5] node[right] {\footnotesize $8\Omega$} (B');
      \path[color=gray, very thick, shorten >=10pt, shorten <=5pt, ->, >=stealth] (y1) edge (A');
      \path[color=gray, very thick, shorten >=10pt, shorten <=5pt, ->, >=stealth] (y2)
      edge (C');
      \path[color=gray, very thick, shorten >=10pt, shorten <=5pt, ->, >=stealth] (z1) edge (B');
      \path[color=gray, very thick, shorten >=10pt, shorten <=5pt, ->, >=stealth]
      (z2) edge (C');
    \end{tikzpicture} \\
    \Downarrow \\
    \begin{tikzpicture}[circuit ee IEC, set resistor graphic=var resistor IEC
      graphic,scale=0.7]
      \node[circle,draw,inner sep=1pt,fill=gray,color=gray]         (x) at (-4,-1.3) {};
      \node at (-4,-3.2) {\footnotesize $X$};
      \node[circle,draw,inner sep=1pt,fill]         (A) at (0,0) {};
      \node[circle,draw,inner sep=1pt,fill]         (B) at (3,0) {};
      \node[circle,draw,inner sep=1pt,fill]         (C) at (1.5,-2.6) {};
      \node[circle,draw,inner sep=1pt,fill]         (D) at (6,0) {};
      \coordinate         (ua) at (.5,.25) {};
      \coordinate         (ub) at (2.5,.25) {};
      \coordinate         (la) at (.5,-.25) {};
      \coordinate         (lb) at (2.5,-.25) {};
      \coordinate         (ub2) at (3.5,.25) {};
      \coordinate         (ud) at (5.5,.25) {};
      \coordinate         (lb2) at (3.5,-.25) {};
      \coordinate         (ld) at (5.5,-.25) {};
      \path (A) edge (ua);
      \path (A) edge (la);
      \path (B) edge (ub);
      \path (B) edge (lb);
      \path (B) edge (ub2);
      \path (B) edge (lb2);
      \path (D) edge (ud);
      \path (D) edge (ld);
      \node[circle,draw,inner sep=1pt,fill=gray,color=gray]         (z1) at
      (10,-.6) {};
      \node[circle,draw,inner sep=1pt,fill=gray,color=gray]         (z2) at (10,-2) {};
      \node at (10,-3.2) {\footnotesize $Z$};
      \path (ua) edge  [resistor, circuit symbol unit=5pt, circuit symbol size=width {5} height 1.5] node[above] {\footnotesize $2\Omega$} (ub);
      \path (la) edge  [resistor, circuit symbol unit=5pt, circuit symbol size=width {5} height 1.5] node[below] {\footnotesize $3\Omega$} (lb);
      \path (A) edge  [resistor, circuit symbol unit=5pt, circuit symbol size=width {5} height 1.5] node[left] {\footnotesize $1\Omega$} (C);
      \path (C) edge  [resistor, circuit symbol unit=5pt, circuit symbol size=width {5} height 1.5] node[right] {\footnotesize $1\Omega$} (B);
      \path (ub2) edge  [resistor, circuit symbol unit=5pt, circuit symbol size=width {5} height 1.5] node[above] {\footnotesize $5\Omega$} (ud);
      \path (lb2) edge  [resistor, circuit symbol unit=5pt, circuit symbol size=width {5} height 1.5] node[below] {\footnotesize $8\Omega$} (ld);
      \path[color=gray, very thick, shorten >=10pt, shorten <=5pt, ->, >=stealth] (x) edge (A);
      \path[color=gray, very thick, shorten >=10pt, shorten <=5pt, ->, >=stealth] (z1)
      edge (D);
      \path[bend left, color=gray, very thick, shorten >=10pt, shorten <=5pt, ->, >=stealth] (z2)
      edge (B);
    \end{tikzpicture}
  \end{array}
\]
The monoidal composition models the placement of
circuits side-by-side:
\[
  \begin{aligned}
    \begin{tikzpicture}[circuit ee IEC, set resistor graphic=var resistor IEC
      graphic,scale=.7]
      \node[circle,draw,inner sep=1pt,fill=gray,color=gray]         (x) at
      (-2.8,-1.3) {};
      \node at (-2.8,-3.2) {\footnotesize $X$};
      \node[circle,draw,inner sep=1pt,fill]         (A) at (0,0) {};
      \node[circle,draw,inner sep=1pt,fill]         (B) at (3,0) {};
      \node[circle,draw,inner sep=1pt,fill]         (C) at (1.5,-2.6) {};
      \node[circle,draw,inner sep=1pt,fill=gray,color=gray]         (y1) at
      (5.8,-.6) {};
      \node[circle,draw,inner sep=1pt,fill=gray,color=gray]         (y2) at
      (5.8,-2) {};
      \node at (5.8,-3.2) {\footnotesize $Y$};
      \coordinate         (ua) at (.5,.25) {};
      \coordinate         (ub) at (2.5,.25) {};
      \coordinate         (la) at (.5,-.25) {};
      \coordinate         (lb) at (2.5,-.25) {};
      \path (A) edge (ua);
      \path (A) edge (la);
      \path (B) edge (ub);
      \path (B) edge (lb);
      \path (ua) edge  [resistor, circuit symbol unit=5pt, circuit symbol size=width {5} height 1.5] node[above] {\footnotesize $2\Omega$} (ub);
      \path (la) edge  [resistor, circuit symbol unit=5pt, circuit symbol size=width {5} height 1.5] node[below] {\footnotesize $3\Omega$} (lb);
      \path (A) edge  [resistor, circuit symbol unit=5pt, circuit symbol size=width {5} height 1.5] node[left] {\footnotesize $1\Omega$} (C);
      \path (C) edge  [resistor, circuit symbol unit=5pt, circuit symbol size=width {5} height 1.5] node[right] {\footnotesize $1\Omega$} (B);
      \path[color=gray, very thick, shorten >=10pt, shorten <=5pt, ->, >=stealth] (x) edge (A);
      \path[color=gray, very thick, shorten >=10pt, shorten <=5pt, ->, >=stealth] (y1) edge (B);
      \path[color=gray, very thick, shorten >=10pt, shorten <=5pt, ->, >=stealth] (y2)
      edge (B);
      \node at (1.5,-3.5) {$\otimes$};
      \node[circle,draw,inner sep=1pt,fill=gray,color=gray]         (x) at
      (-2.8,-5) {};
      \node at (-2.8,-6.5) {\footnotesize $X'$};
      \node[circle,draw,inner sep=1pt,fill]         (A) at (0,-5) {};
      \node[circle,draw,inner sep=1pt,fill]         (B) at (3,-5) {};
      \node[circle,draw,inner sep=1pt,fill=gray,color=gray]         (y1) at
      (5.8,-5) {};
      \node at (5.8,-6.5) {\footnotesize $Y'$};
      \coordinate         (ua1) at (.5,-4.75) {};
      \coordinate         (ub1) at (2.5,-4.75) {};
      \coordinate         (la1) at (.5,-5.25) {};
      \coordinate         (lb1) at (2.5,-5.25) {};
      \path (A) edge (ua1);
      \path (A) edge (la1);
      \path (B) edge (ub1);
      \path (B) edge (lb1);
      \path (ua1) edge  [resistor, circuit symbol unit=5pt, circuit symbol
      size=width {5} height 1.5] node[above] {\footnotesize $5\Omega$} (ub1);
      \path (la1) edge  [resistor, circuit symbol unit=5pt, circuit symbol
      size=width {5} height 1.5] node[below] {\footnotesize $1\Omega$} (lb1);
      \path[color=gray, very thick, shorten >=10pt, shorten <=5pt, ->, >=stealth] (x) edge (A);
      \path[color=gray, very thick, shorten >=10pt, shorten <=5pt, ->, >=stealth] (y1) edge (B);
    \end{tikzpicture}
  \end{aligned}
  \Rightarrow
  \begin{aligned}
    \begin{tikzpicture}[circuit ee IEC, set resistor graphic=var resistor IEC
      graphic,scale=.7]
      \node[circle,draw,inner sep=1pt,fill=gray,color=gray]         (x) at
      (-2.8,-1.3) {};
      \node at (-2.8,-5.5) {\footnotesize $X+X'$};
      \node[circle,draw,inner sep=1pt,fill]         (A) at (0,0) {};
      \node[circle,draw,inner sep=1pt,fill]         (B) at (3,0) {};
      \node[circle,draw,inner sep=1pt,fill]         (C) at (1.5,-2.6) {};
      \node[circle,draw,inner sep=1pt,fill=gray,color=gray]         (y1) at
      (5.8,-.6) {};
      \node[circle,draw,inner sep=1pt,fill=gray,color=gray]         (y2) at
      (5.8,-2) {};
      \node at (5.8,-5.5) {\footnotesize $Y+Y'$};
      \coordinate         (ua) at (.5,.25) {};
      \coordinate         (ub) at (2.5,.25) {};
      \coordinate         (la) at (.5,-.25) {};
      \coordinate         (lb) at (2.5,-.25) {};
      \path (A) edge (ua);
      \path (A) edge (la);
      \path (B) edge (ub);
      \path (B) edge (lb);
      \path (ua) edge  [resistor, circuit symbol unit=5pt, circuit symbol size=width {5} height 1.5] node[above] {\footnotesize $2\Omega$} (ub);
      \path (la) edge  [resistor, circuit symbol unit=5pt, circuit symbol size=width {5} height 1.5] node[below] {\footnotesize $3\Omega$} (lb);
      \path (A) edge  [resistor, circuit symbol unit=5pt, circuit symbol size=width {5} height 1.5] node[left] {\footnotesize $1\Omega$} (C);
      \path (C) edge  [resistor, circuit symbol unit=5pt, circuit symbol size=width {5} height 1.5] node[right] {\footnotesize $1\Omega$} (B);
      \path[color=gray, very thick, shorten >=10pt, shorten <=5pt, ->, >=stealth] (x) edge (A);
      \path[color=gray, very thick, shorten >=10pt, shorten <=5pt, ->, >=stealth] (y1) edge (B);
      \path[color=gray, very thick, shorten >=10pt, shorten <=5pt, ->, >=stealth] (y2)
      edge (B);
      \node[circle,draw,inner sep=1pt,fill=gray,color=gray]         (x2) at
      (-2.8,-2.7) {};
      \node[circle,draw,inner sep=1pt,fill]         (A1) at (0,-4) {};
      \node[circle,draw,inner sep=1pt,fill]         (B1) at (3,-4) {};
      \node[circle,draw,inner sep=1pt,fill=gray,color=gray]         (y3) at
      (5.8,-3.4) {};
      \coordinate         (ua1) at (.5,-3.75) {};
      \coordinate         (ub1) at (2.5,-3.75) {};
      \coordinate         (la1) at (.5,-4.25) {};
      \coordinate         (lb1) at (2.5,-4.25) {};
      \path (A1) edge (ua1);
      \path (A1) edge (la1);
      \path (B1) edge (ub1);
      \path (B1) edge (lb1);
      \path (ua1) edge  [resistor, circuit symbol unit=5pt, circuit symbol
      size=width {5} height 1.5] node[above] {\footnotesize $5\Omega$} (ub1);
      \path (la1) edge  [resistor, circuit symbol unit=5pt, circuit symbol
      size=width {5} height 1.5] node[below] {\footnotesize $1\Omega$} (lb1);
      \path[color=gray, very thick, shorten >=10pt, shorten <=5pt, ->,
      >=stealth] (x2) edge (A1);
      \path[color=gray, very thick, shorten >=10pt, shorten <=5pt, ->,
      >=stealth] (y3) edge (B1);
    \end{tikzpicture}
  \end{aligned}
\]
Identities and Frobenius maps result from identifying points. For example, the
identity on a single point is the (empty) decorated cospan
\[
      \begin{tikzpicture}[circuit ee IEC, set resistor graphic=var resistor IEC
	graphic]
	\node[circle,draw,inner sep=1pt,fill=gray,color=gray]         (x) at
	(-2.8,0) {};
	\node at (-2.8,-.5) {\footnotesize $X$};
	\node[circle,draw,inner sep=1pt,fill]         (A) at (0,0) {};
	\node[circle,draw,inner sep=1pt,fill=gray,color=gray]         (y1) at
	(2.8,0) {};
	\node at (2.8,-.5) {\footnotesize $Y$};
	\path[color=gray, very thick, shorten >=10pt, shorten <=5pt, ->, >=stealth] (x) edge (A);
	\path[color=gray, very thick, shorten >=10pt, shorten <=5pt, ->,
	>=stealth] (y1) edge (A);
      \end{tikzpicture}
\]
while the multiplication on a point is the empty decorated cospan
\[
      \begin{tikzpicture}[circuit ee IEC, set resistor graphic=var resistor IEC
	graphic]
	\node[circle,draw,inner sep=1pt,fill=gray,color=gray]         (x) at
	(-2.8,0) {};
	\node at (-2.8,-1.3) {\footnotesize $X$};
	\node[circle,draw,inner sep=1pt,fill]         (A) at (0,0) {};
	\node[circle,draw,inner sep=1pt,fill=gray,color=gray]         (y1) at
	(2.8,.8) {};
	\node[circle,draw,inner sep=1pt,fill=gray,color=gray]         (y2) at
	(2.8,-.8) {};
	\node at (2.8,-1.3) {\footnotesize $Y$};
	\path[color=gray, very thick, shorten >=10pt, shorten <=5pt, ->, >=stealth] (x) edge (A);
	\path[color=gray, very thick, shorten >=10pt, shorten <=5pt, ->,
	>=stealth] (y1) edge (A);
	\path[color=gray, very thick, shorten >=10pt, shorten <=5pt, ->,
	>=stealth] (y2) edge (A);
      \end{tikzpicture}
\]
These identifications represent interconnection via ideal, perfectly conductive
wires: if two terminals are connected by such wires, then the electrical
behaviour at both terminals must be identical. The Frobenius maps allow us to
split, combine, and discard wires.  The symmetric monoidal structure allows
us reorder input and output wires. Derived from these, the compactness captures the
interchangeability between input and outputs of circuits---that is, the fact
that we can choose any input to our circuit and consider it instead as an
output, and vice versa---while the dagger structure expresses the fact that we
may reflect a whole circuit, switching all inputs with all outputs.

Recall that we consider two circuit diagrams equivalent if they have the same
behaviour, or power functional. Now that we have constructed a hypergraph
category where our morphisms are circuit diagrams---our `syntactic' category for
our diagrammatic language---we would also like to construct a hypergraph
category with morphisms behaviours of circuits, and show that the map from a
circuit to its behaviour is functorial.


\subsection{An obstruction to black boxing} \label{ssec.noidentities}

It would be nice to have a category in which Dirichlet forms are morphisms, such
that the map sending a circuit to its behaviour is a functor.  Here we present a
na\"ive attempt to constructed the category with Dirichlet forms as morphisms,
using the principle of minimum power to compose these morphisms.  Unfortunately
the proposed category does not include identity morphisms.  However, it points
in the right direction, and underlines the importance of the cospan formalism we
then turn to develop.

We can define a composition rule for Dirichlet forms that reflects composition
of circuits.  Given finite sets $X$ and $Y$, let $X+Y$ denote their disjoint
union.  Let $D(X,Y)$ be the set of Dirichlet forms on $X+Y$. There is a way to
compose these Dirichlet forms
\[ 
\circ \maps D(Y,Z) \times D(X,Y) \to D(X,Z) 
\]
defined as follows.  Given $P \in D(Y,Z)$ and $Q \in D(X,Y)$, let
\[ 
  (P \circ Q)(\alpha, \gamma) = \min_{Y} Q(\alpha, \beta) + P(\beta, \gamma),
\]
where $\alpha \in \F^X, \gamma \in \F^Z$. This operation has a clear
interpretation in terms of electrical circuits: the power used by the entire
circuit is just the sum of the power used by its parts. 

It is immediate from Theorem \ref{thm:dirichletminimization} that this
composition rule is well-defined: the composite of two Dirichlet forms is again
a Dirichlet form. Moreover, this composition is associative. However, it fails
to provide the structure of a category, as there is typically no Dirichlet form
$1_X \in D(X,X)$ playing the role of the identity for this composition. For an
indication of why this is so, let $\{\bullet\}$ be a set with one element, and
suppose that some Dirichlet form $I(\beta,\gamma) = k(\beta-\gamma)^2 \in
D(\{\bullet\},\{\bullet\})$ acts as an identity on the right for this
composition. Then for all $Q(\alpha,\beta) = c(\alpha-\beta)^2 \in
D(\{\bullet\},\{\bullet\})$, we must have
\begin{align*}
  c\alpha^2 &= Q(\alpha,0) \\
  &= (I \circ Q)(\alpha,0) \\ 
  &= \min_{\beta \in \F} Q(\alpha, \beta) + I(\beta,0) \\
  &= \min_{\beta \in \F} k(\alpha-\beta)^2 + c\beta^2 \\
  &= \frac{kc}{k+c}\alpha^2,
\end{align*}
where we have noted that $\frac{kc}{k+c}\alpha^2$ minimizes $k(\alpha-\beta)^2 +
c\beta^2$ with respect to $\beta$. But for any choice of $k \in \F$ this
equality only holds when $c = 0$, so no such Dirichlet form exists. Note,
however, that for $k>> c$ we have $c\alpha^2 \approx \frac{kc}{k+c}\alpha^2$, so
Dirichlet forms with large values of $k$---corresponding to resistors with
resistance close to zero---act as `approximate identities'.

In this way we might interpret the identities we wish to introduce
into this category as the behaviours of idealized components with zero
resistance: perfectly conductive wires. Unfortunately, the power functional of a
purely conductive wire is undefined: the formula for it involves division by
zero.  In real life, coming close to this situation leads to the disaster that
electricians call a `short circuit': a huge amount of power dissipated for even
a small voltage.  This is why we have fuses and circuit breakers.

Nonetheless, we have most of the structure required for a category. A `category
without identity morphisms' is called a \define{semicategory}, so we see
\begin{proposition}
There is a semicategory where:
\begin{itemize}
\item the objects are finite sets,

\item a morphism from $X$ to $Y$ is a Dirichlet form $Q \in D(X,Y)$.  

\item composition of morphisms is given by 
\[
(P \circ Q)(\gamma, \alpha) = \min_{Y} Q(\gamma, \beta) + P(\beta, \alpha).
\]

\end{itemize}
\end{proposition}

We would like to make this into a category. One easy way to do this is to
formally adjoin identity morphisms; this trick works for any semicategory.
However, we obtain a better category if we include \emph{more} morphisms: more
behaviours corresponding to circuits made of perfectly conductive wires.
The expression for the extended power functional includes the reciprocals of
impedances, such circuits cannot be expressed within the framework we have
developed thus far. Similarly, the Frobenius maps also have semantics as ideal
wires, and cannot be represented using Dirichlet forms.  Indeed, for these
idealized circuits there is no function taking boundary potentials to boundary
currents: the vanishing impedance would imply that any difference in potentials
at the boundary induces `infinite' currents. One way of dealing with this is to
use decorated cospans.


\subsection{The category of Dirichlet cospans}

Although we cannot have Dirichlet forms be the morphisms of a category
themselves, we have a standard trick for turning data into the morphisms of a
category: we decorate cospans with them. The cospans then handle the composition
for us. In this section we construct a category of cospans and a category of
corelations decorated by Dirichlet forms. We shall think of the former as having
extended power functionals as morphisms, and the latter as power functionals.

Consider a cospan of finite sets $X \to N \leftarrow Y$ together with a
Dirichlet form $Q_N$ on the apex $N$. We call this a \define{Dirichlet cospan}.
To compose such cospans, say when given another cospan $Y \to M \leftarrow Z$
decorated by Dirichlet form $Q_M$, we decorated the composite cospan $X \to
N+_YM \leftarrow Z$ with the Dirichlet form
\begin{align*}
  \Big({j_N}_\ast Q_N+ {j_M}_\ast Q_M\Big)\maps \F^{N+_YM} &\longrightarrow \F;\\
  \phi &\longmapsto Q_N(\phi\circ j_N)+Q_M(\phi\circ j_M),
\end{align*}
where the $j$ are the maps include $N$ and $M$ into the pushout as usual.
Interpreted in terms of extended power functionals, this simply says that the
power consumed by the interconnected circuit is just the power consumed by each
part. These form the morphisms of a decorated cospan category.

\begin{proposition}
The following defines a lax symmetric monoidal functor
$(\mathrm{Dirich},\delta)$: let
\[
  \mathrm{Dirich}\maps (\mathrm{FinSet},+) \longrightarrow (\mathrm{Set},\times)
\]
map a finite set $X$ to the set $\mathrm{Dirich}(X)$ of Dirichlet forms
$Q\maps \F^X \to \F$ on $X$, and map a function $f\maps X \to Y$ between finite
sets to the pushforward function
\begin{align*}
  \mathrm{Dirich}(f)\maps \mathrm{Dirich}(X) &\longrightarrow \mathrm{Dirich}(Y); \\
  Q &\longmapsto \Big(f_{\ast}Q\maps \phi \mapsto Q(\phi\circ f)\Big).
\end{align*}

For coherence maps, equip $\mathrm{Dirich}$ with the natural family of maps
\begin{align*}
  \delta_{N,M}\maps \mathrm{Dirich}(N) \times \mathrm{Dirich}(M) &\longrightarrow
  \mathrm{Dirich}(N+M) \\
  (Q_N,Q_M) &\longmapsto {\iota_N}_\ast Q_N+{\iota_M}_\ast Q_M
\end{align*}
and also with the unit
\begin{align*}
  \delta_1\maps 1 &\longrightarrow \mathrm{Dirich}(\varnothing);\\
  \bullet &\longmapsto (\F^\varnothing \to \F; ! \mapsto 0).
\end{align*}
Note that the sum of two Dirichlet forms is given pointwise by the addition in
$\F$.
\end{proposition}
\begin{proof}
  As composition of functions is associative and has an identity,
  $\mathrm{Dirich}$ is a functor.  The naturality of the $\delta_{N,M}$ follows
  from the universal property of the coproduct in $\FinSet$, while the symmetric
  monoidal coherence axioms follow from the associativity, unitality, and
  commutativity of addition in $\F$.
\end{proof}


Using decorated cospans, we thus obtain a hypergraph category
$\mathrm{DirichCospan}$ where a morphism is a cospan of finite sets whose apex
is equipped with a Dirichlet form.  Next, we use decorated cospans to construct
a strict hypergraph functor $\Circ \to \mathrm{DirichCospan}$ sending a circuit
to a cospan decorated by its extended power functional. For this we need a
monoidal natural transformation $\alpha\maps (\mathrm{Circuit},\rho) \Rightarrow
(\mathrm{Dirich},\delta)$.

\begin{proposition} 
  The collection of maps
\begin{align*}
  \alpha_N\maps \mathrm{Circuit}(N) &\longrightarrow \mathrm{Dirich}(N); \\
  (N,E,s,t,r) &\longmapsto \left(\phi \in \F^N \mapsto \frac{1}{2} \sum_{e \in E}
  \frac{1}{r(e)}\big(\phi(s(e))-\phi(t(e))\big)^2\right).
\end{align*}
defines a monoidal natural transformation
\[
  \alpha\maps (\mathrm{Circuit},\rho) \Longrightarrow
  (\mathrm{Dirich},\delta).
\]
\end{proposition}
\begin{proof}
Naturality requires that the square
\[
  \xymatrix{
    \mathrm{Circuit}(N) \ar[r]^{\alpha_N} \ar[d]_{\mathrm{Circuit}(f)} &
    \mathrm{Dirich}(N) \ar[d]^{\mathrm{Dirich}(f)}  \\
    \mathrm{Circuit}(M) \ar[r]_{\alpha_M} & \mathrm{Dirich}(M)
  }
\]
commutes. Let $(N,E,s,t,r)$ be an $\F^+$-graph on $N$ and $f\maps N \to M$ be a
function $N$ to $M$. Then both $\mathrm{Dirich}(f) \circ \alpha_N$ and $\alpha_M
\circ \mathrm{Circuit}(f)$ map $(N,E,s,t,r)$ to the Dirichlet form
\begin{align*}
  \F^M &\longrightarrow \F;\\
  \psi &\longmapsto \frac{1}{2} \sum_{e \in E}\frac{1}{r(e)}
  \big(\psi(f(s(e)))-\psi(f(t(e)))\big)^2.
\end{align*}
Thus both methods of constructing a power functional on a set of nodes $M$ from
a circuit on $N$ and a function $N \to M$ produce the same power functional.

To show that $\alpha$ is a monoidal natural transformation, we must check that
the square
\[
\xymatrix{
  \mathrm{Circuit}(N) \times \mathrm{Circuit}(M) \ar[r]^{\alpha_N \times
  \alpha_M} \ar[d]_{\rho_{N,M}} & \mathrm{Dirich}(N) \times \mathrm{Dirich}(M)
  \ar[d]^{\delta_{N,M}}  \\
  \mathrm{Circuit}(N+M) \ar[r]_{\alpha_{N+M}} & \mathrm{Dirich}(N+M)
}
\]
and the triangle
\[
\xymatrix{
  & 1 \ar[dl]_{\rho_\varnothing} \ar[dr]^{\delta_\varnothing}\\
\mathrm{Circuit}(\varnothing)  \ar[rr]_{\alpha_\varnothing} &&
\mathrm{Dirich}(\varnothing)
}
\]
commute. It is readily observed that both paths around the square lead to taking
two graphs and summing their corresponding Dirichlet forms, and that the
triangle commutes immediately as all objects in it are the one element set.
\end{proof}

From decorated cospans, we thus obtain a strict hypergraph functor
\[
  Q \maps \Circ = \mathrm{CircuitCospan} \longrightarrow \mathrm{DirichCospan}.
\]
Informally, this says that the process of composition for circuit diagrams is
the same as that of composition for Dirichlet cospans. Note that this is not a
faithful functor.  For example, applying $Q$ to a circuit 
\[
\begin{tikzpicture}[circuit ee IEC, set resistor graphic=var resistor IEC
	graphic,scale=.8]
	\node[circle,draw,inner sep=1pt,fill=gray,color=gray]         (x) at
	(-2.8,0) {};
	\node at (-2.8,-1) {\footnotesize $X$};
	\node[circle,draw,inner sep=1pt,fill]         (A) at (0,0) {};
	\node[circle,draw,inner sep=1pt,fill]         (B) at (3,0) {};
	\node[circle,draw,inner sep=1pt,fill=gray,color=gray]         (y1) at
	(5.8,0) {};
	\node at (5.8,-1) {\footnotesize $Y$};
	\coordinate         (ua) at (.5,.25) {};
	\coordinate         (ub) at (2.5,.25) {};
	\coordinate         (la) at (.5,-.25) {};
	\coordinate         (lb) at (2.5,-.25) {};
	\path (A) edge (ua);
	\path (A) edge (la);
	\path (B) edge (ub);
	\path (B) edge (lb);
	\path (ua) edge  [resistor, circuit symbol unit=5pt, circuit symbol
	size=width {5} height 1.5] node[label={[label
	distance=1pt]90:{\footnotesize $r$}}] {} (ub);
	\path (la) edge  [resistor, circuit symbol unit=5pt, circuit symbol
	size=width {5} height 1.5] node[label={[label
	distance=1pt]270:{\footnotesize $s$}}] {} (lb);
	\path[color=gray, very thick, shorten >=10pt, shorten <=5pt, ->, >=stealth] (x) edge (A);
	\path[color=gray, very thick, shorten >=10pt, shorten <=5pt, ->, >=stealth] (y1) edge (B);
      \end{tikzpicture}
    \]
with two parallel edges of resistance $r$ and $s$ respectively, we obtain
the same result as for the circuit
\[
\begin{tikzpicture}[circuit ee IEC, set resistor graphic=var resistor IEC
	graphic,scale=.8]
	\node[circle,draw,inner sep=1pt,fill=gray,color=gray]         (x) at
	(-2.8,0) {};
	\node at (-2.8,-1) {\footnotesize $X$};
	\node[circle,draw,inner sep=1pt,fill]         (A) at (0,0) {};
	\node[circle,draw,inner sep=1pt,fill]         (B) at (3,0) {};
	\node[circle,draw,inner sep=1pt,fill=gray,color=gray]         (y) at
	(5.8,0) {};
	\node at (5.8,-1) {\footnotesize $Y$};
	\path (A) edge  [resistor, circuit symbol unit=5pt, circuit symbol
	size=width {5} height 1.5] node[label={[label
	distance=1pt]90:{\footnotesize $t$}}] {} (B);
	\path[color=gray, very thick, shorten >=10pt, shorten <=5pt, ->, >=stealth] (x) edge (A);
	\path[color=gray, very thick, shorten >=10pt, shorten <=5pt, ->, >=stealth] (y) edge (B);
      \end{tikzpicture}
    \]
with just a single edge with resistance
\[
  t = \frac{1}{\tfrac{1}{r} + \tfrac{1}{s}}.
\]
That is, both map to the Dirichlet cospan $X=\{x\} \to \{x,y\} \leftarrow \{y\}
= Y$ with $\{x,y\}$ decorated by the Dirichlet form $P(\psi) =
\tfrac{1}{t}(\psi_x-\psi_y)^2$. Nonetheless, this functor is far from the
desired semantic functor: many different extended power functionals can restrict
to the same power functional on the boundary. 

\subsection{A category of behaviours}

For our functorial semantics, we wish to map a circuit to its power functional.
We take a moment to briefly sketch how one could use decorated corelations to
construct a codomain for such a functor. The result, however, is a little ad
hoc, and we shall not pursue it in depth. Instead, in the next section, we shall
generalize Dirichlet forms to Lagrangian relations.

Indeed, ideally we would have liked to use an isomorphism-morphism factorisation on
$\FinSet$, to construct isomorphism-morphism corelations decorated by Dirichlet
forms. Then for a map $X \to Y$ the Dirichlet form must decorate the set
$X+Y$, and we need not talk at all about cospans. We saw however, in the
previous section, that a category with morphisms Dirichlet forms on the disjoint
union of the domain and codomain is not posible.  As a consolation prize we may
use the epi-mono factorisation system $(\mathrm{Sur},\mathrm{Inj})$ on
$\FinSet$. 

The key idea is that given an injection $m\maps \overline N \to N$, we may
minimise a Dirichlet form on $N$ over the complement of the image of $\overline
N$ to obtain, in effect, a Dirichlet form on $\overline N$. Given a circuit $X
\to Y$, we may factor the function $X+Y \to N$ as $X+Y \stackrel{e}\to
\overline{N} \stackrel{m}\to N$, where $e$ is a surjection and $m$ is an
injection. Note the image $m(\overline{N})$ of $m$ in $N$ is equal to the
boundary $\partial N$ of the circuit. Thus if we map a circuit to its extended
power functional, and then minimise the Dirichlet form onto $\overline N$, we
obtain the power functional of the circuit. 

This motivates the following proposition. 

\begin{proposition}
  There is a lax symmetric monoidal functor
\[
  \mathrm{Dirich}\maps (\mathrm{FinSet};\mathrm{Inj}^\opp,+) \longrightarrow
  (\mathrm{Set},\times),
\]
extending the functor $\mathrm{Dirich} \maps (\FinSet,+) \rightarrow
(\Set,\times)$, such that the image of the cospan $N \stackrel{f}\to A
\stackrel{m}\hookleftarrow M$, where $f$ is a function and $m$ an injection, is the
map
\begin{align*}
  \mathrm{Dirich}(f;m^\opp)\maps \mathrm{Dirich}(N) &\longrightarrow
  \mathrm{Dirich}(M);\\
  Q &\longmapsto \min_{A\setminus M} f_\ast Q.
\end{align*}
Note that in writing $A \setminus M$ we are considering $M$ as a subset of $A$
by way of the injection $m$.

\end{proposition} 
By extending the previous functor $\mathrm{Dirich}$, we mean the triangle
\[
  \xymatrixcolsep{3pc}
  \xymatrixrowsep{1pc}
  \xymatrix{
    (\FinSet, +) \ar[dr]^{\mathrm{Dirich}} \ar@{^{(}->}[dd] \\
    & (\Set,\times) \\
 (\mathrm{FinSet};\mathrm{Inj}^\opp,+) \ar[ur]_{\mathrm{Dirich}}.
  }
\]
commutes, where the vertical map is the subcategory inclusion.

There are two aspects of this proposition that require detailed proof: that the
map $\mathrm{Dirich}$ preserves composition, and that the coherence maps
$\delta$ are still natural in this larger category. Both facts come down to
proving that the minimisation and pushforward commute in certain required circumstances. 

This extended functor $\mathrm{Dirich}$ then gives rise to a decorated
corelations category $\mathrm{DirichCorel}$ with morphisms jointly-epic cospans
decorated by Dirichlet forms; composition is given by summing the Dirichlet
forms on the factors, then minimising over the interior.

This category satisfies the two precise desiderata for a semantic category: each
behaviour is uniquely represented, and the map of a circuit to its behaviour
preserves all compositional structure. Nonetheless, we will not pursue this
construction in depth, feeling that this construction uses corelations to
shoehorn in the necessary compositional structure. Instead, we find that a
cleaner construction, based on the idea of a linear relation, can be given by
a slight generalisation of our decorations.

\section{Networks as Lagrangian relations} \label{sec:circlagr}
%%fakesubsection
In the first part of this chapter, we explored the semantic content contained in
circuit diagrams, leading to an understanding of circuit diagrams as expressing
some relationship between the potentials and currents that can simultaneously be
imposed on some subset, the so-called terminals, of the nodes of the circuit. We
called this collection of possible relationships the behaviour of the circuit.
While in that setting we used the concept of Dirichlet forms to describe this
relationship, we saw in the end that describing circuits as Dirichlet forms does
not allow for a straightforward notion of composition of circuits. 

In this section, inspired by the principle of least action of classical
mechanics in analogy with the principle of minimum power, we develop a setting
for describing behaviours that allows for easy discussion of composite
behaviours: Lagrangian subspaces of symplectic vector spaces. These Lagrangian
subspaces provide a more direct, invariant perspective, comprising precisely the
set of vectors describing the possible simultaneous potential and current
readings at all terminals of a given circuit. As we shall see, one immediate and
important advantage of this setting is that we may model wires of zero
resistance.

They also have a Willems-esque aesthetic to them, choosing to define physical
systems by listing their possible states.

Recall that we write $\F$ for some field, which for our applications is
usually the field $\R$ of real numbers or the field $\R(s)$ of rational
functions of one real variable.

\subsection{Symplectic vector spaces}

A circuit made up of wires of positive resistance defines a function from
boundary potentials to boundary currents. A wire of zero resistance, however,
does not define a function: the principle of minimum power is obeyed as long as
the potentials at the two ends of the wire are equal. More generally, we may
thus think of circuits as specifying a set of allowed voltage-current pairs, or
as a relation between boundary potentials and boundary currents. This set forms
what is called a Lagrangian subspace, and is given by the graph of the
differential of the power functional. More generally, Lagrangian submanifolds
graph derivatives of smooth functions: they describe the point evaluated and the
tangent to that point within the same space.

The material in this section is all known, and follows without great difficulty
from the definitions. To keep this section brief we omit proofs. See any
introduction to symplectic vector spaces, such as Cimasoni and Turaev \cite{CT} or
Piccione and Tausk \cite{PT}, for details.

\begin{definition}
  Given a finite-dimensional vector space $V$ over a field $\F$, a 
  \define{symplectic form}
  $\omega\maps V \times V \to \F$ on $V$ is an alternating nondegenerate bilinear
  form.  That is, a symplectic form $\omega$ is a function $V \times V \to \F$
  that is
  \begin{enumerate}[(i)]
    \item bilinear: for all $\lambda \in \F$ and all $u,v \in V$ we have
      $\omega(\lambda u,v) = \omega(u,\lambda v) =  \lambda \omega(u,v)$;
    \item alternating: for all $v \in V$ we have $\omega(v,v) = 0$; and
    \item nondegenerate: given $v \in V$, $\omega(u,v) = 0$ for all $u \in V$ if
      and only if $u = 0$.
  \end{enumerate} 
  A \define{symplectic vector space} $(V,\omega)$ is a vector space $V$ equipped
  with a symplectic form $\omega$. 

  Given symplectic vector spaces $(V_1,\omega_1), (V_2, \omega_2)$, a
  \define{symplectic map} is a linear map 
  \[
    f\maps (V_1,\omega_1) \longrightarrow (V_2, \omega_2)
  \]
  such that $\omega_2(f(u),f(v)) = \omega_1(u,v)$ for all $u,v \in V_1$. A
  \define{symplectomorphism} is a symplectic map that is also an isomorphism. 
\end{definition}

An alternating form is always \define{antisymmetric}, meaning that $\omega(u,v) = 
-\omega(v,u)$ for all $u,v \in V$.  The converse is true except in characteristic 2.
A \define{symplectic basis} for a symplectic vector space $(V,\omega)$ is a
basis $\{p_1,\dots,p_n,q_1,\dots,q_n\}$ such that $\omega(p_i,p_j) =
\omega(q_i,q_j) = 0$ for all $1 \le i,j \le n$, and $\omega(p_i,q_j) =
\delta_{ij}$ for all $1 \le i,j\le n$, where $\delta_{ij}$ is the Kronecker delta,
equal to $1$ when $i =j$, and $0$ otherwise. A symplectomorphism maps symplectic
bases to symplectic bases, and conversely, any map that takes a symplectic basis
to another symplectic basis is a symplectomorphism.

\begin{example}[The symplectic vector space generated by a finite set]
  \label{ex:symplectic_space_generated_by_set}
  Given a finite set $N$, we consider the vector space $\vectf{N}$ a symplectic
  vector space $(\vectf{N},\omega)$, with symplectic form 
  \[
    \omega\big((\phi,i),(\phi',i')\big) = i'(\phi)-i(\phi').  
  \] 
  Let $\{\phi_n\}_{n \in N}$ be the basis of $\F^N$ consisting of the functions
  $N \to \F$ mapping $n$ to $1$ and all other elements of $n$ to $0$, and let
  $\{i_n\}_{n \in N} \subseteq {(\F^N)}^\ast$ be the dual basis. Then
  $\{(\phi_n,0),(0,i_n)\}_{n\in N}$ forms a symplectic basis for $\vectf{N}$.  
\end{example}

There are two common ways we will build symplectic spaces from other
symplectic spaces: conjugation and summation. Given a symplectic form $\omega$,
we may define its \define{conjugate} symplectic form $\overline\omega = -
\omega$, and write the conjugate symplectic space $(V,\overline\omega)$ as
$\overline V$. Given two symplectic vector spaces $(U, \nu),(V,\omega)$, we
consider their direct sum $U \oplus V$ a symplectic vector space with the
symplectic form $\nu+\omega$, and call this the \define{sum} of the two
symplectic vector spaces. Note that this is not a product in the category of
symplectic vector spaces and symplectic maps.

The symplectic form provides a notion of orthogonal complement.  Given a subspace $S$ of $V$, we define its \define{complement}
\[
  S^\circ = \{v \in V \mid \omega(v,s) = 0 \textrm{ for all } s \in S\}.
\]
Note that this construction obeys the following identities, where $S$ and $T$
are subspaces of $V$:
\begin{align*}
  \dim S+ \dim S^\circ &= \dim V \\
  (S^\circ)^\circ &= S \\
  (S + T)^\circ &= S^\circ \cap T^\circ \\
  (S \cap T)^\circ &= S^\circ + T^\circ.
\end{align*}

In the symplectic vector space $\vectf{N}$, the subspace $\F^N$ has the
property of being a maximal subspace such that the symplectic form restricts to
the zero form on this subspace. Subspaces with this property are known as Lagrangian
subspaces, and they may all be realized as the image of $\vectf{N}$ 
under symplectomorphisms from $\vectf{N}$ to itself.

\begin{definition} 
  Let $S$ be a linear subspace of a symplectic vector space $(V,\omega)$. We say
  that $S$ is \define{isotropic} if $\omega|_{S \times S} = 0$, and that $S$ is
  \define{coisotropic} if $S^\circ$ is isotropic. A subspace is
  \define{Lagrangian} if it is both isotropic and coisotropic, or equivalently, if it  
  is a maximal isotropic subspace.
\end{definition}

Lagrangian subspaces are also known as Lagrangian correspondences and canonical
relations. Note that a subspace $S$ is isotropic if and only if $S \subseteq
S^\circ$. This fact helps with the following characterizations of Lagrangian
subspaces.

\begin{proposition} \label{lagrangian_characterization} 
  Given a subspace $L \subset V$ of a symplectic vector space $(V,\omega)$, the
  following are equivalent: 
  \begin{enumerate}[(i)] 
    \item $L$ is Lagrangian.  
    \item $L$ is maximally isotropic.  
    \item $L$ is minimally coisotropic.  
    \item $L = L^\circ$.  
    \item $L$ is isotropic and $\dim L = \frac12 \dim V$.
  \end{enumerate} 
\end{proposition}

From this proposition it follows easily that the direct sum of two Lagrangian
subspaces in Lagrangian in the sum of their ambient spaces. We also observe that
an advantage of isotropy is that there is a good way to take a quotient of a
symplectic vector space by an isotropic subspace---that is, there is a way to
put a natural symplectic structure on the quotient space.

\begin{proposition}
  Let $S$ be an isotropic subspace of a symplectic vector space $(V,\omega)$.
  Then $S^\circ/S$ is a symplectic vector space with symplectic form
  $\omega'(v+S,u+S) = \omega(v,u)$.
\end{proposition}
\begin{proof} 
  The function $\omega'$ is a well-defined due to the isotropy of
  $S$---by definition adding any pair $(s,s')$ of elements of $S$ to a pair
  $(v,u)$ of elements of $S^\circ$ does not change the value of
  $\omega(v+s,u+s')$. As $\omega$ is a symplectic form, one can check that
  $\omega'$ is too.  
\end{proof}

\subsection{Lagrangian subspaces from quadratic forms}

Lagrangian subspaces are of relevance to us here as the behaviour of any passive
linear circuit forms a Lagrangian subspace of the symplectic vector space
generated by the nodes of the circuit. We think of this vector space as
comprising two parts: a space $\F^N$ of potentials at each node, and a dual
space ${(\F^N)}^\ast$ of currents. To make clear how circuits can be interpreted
as Lagrangian subspaces, here we describe how Dirichlet forms on a finite set
$N$ give rise to Lagrangian subspaces of $\vectf{N}$. More generally, we show
that there is a one-to-one correspondence between Lagrangian subspaces and
quadratic forms.

\begin{proposition} \label{prop:qfls}
  Let $N$ be a finite set. Given a quadratic form $Q$ over $\F$ on $N$, the
  subspace 
  \[ 
    L_Q = \big\{(\phi,dQ_\phi) \mid \phi \in \F^N\big\} \subseteq \vectf{N},
  \] 
  where $dQ_\phi \in {(\F^N)}^\ast$ is the formal differential of $Q$ at $\phi
  \in \F^N$, is Lagrangian. Moreover, this construction gives a one-to-one correspondence 
  \[ 
    \left\{\begin{array}{c} 
      \\ \mbox{Quadratic forms over $\F$ on $N$} \\ \phantom{.}
    \end{array} \right\} 
    \longleftrightarrow
    \left\{\begin{array}{c} 
      \mbox{Lagrangian subspaces of $\vectf{N}$}\\
      \mbox{with trivial intersection with} \\ 
      \{0\} \oplus {(\F^N)}^\ast \subseteq \vectf{N} 
    \end{array} \right\}.  
  \]
\end{proposition}
\begin{proof}
  The symplectic structure on $\vectf{N}$ and our notation for it is given in
  Example \ref{ex:symplectic_space_generated_by_set}. 

  Note that for all $n,m \in N$ the corresponding basis elements 
  \[
    \frac{\partial^2 Q}{\partial \phi_n \partial \phi_m} = dQ_{\phi_n}(\phi_m) =
    dQ_{\phi_m}(\phi_n),
  \]
  so $dQ_\phi(\psi) = dQ_\psi(\phi)$ for all $\phi,\psi \in \F^N$. Thus $L_Q$ is
  indeed Lagrangian: for all $\phi,\psi \in \F^N$ 
  \[
    \omega\big((\phi,dQ_\phi),(\psi,dQ_\psi)\big) = dQ_\psi(\phi) -
    dQ_\phi(\psi) = 0.
  \]

  Observe also that for all quadratic forms $Q$ we have $dQ_0 = 0$, so the only
  element of $L_Q$ of the form $(0,i)$, where $i \in {(\F^N)}^\ast$, is $(0,0)$.
  Thus $L_Q$ has trivial intersection with the subspace $\{0\} \oplus
  {(\F^N)}^\ast$ of $\vectf{N}$. This $L_Q$ construction forms the leftward
  direction of the above correspondence.

  For the rightward direction, suppose that $L$ is a Lagrangian subspace of
  $\vectf{N}$ such that $L \cap (\{0\} \oplus {(\F^N)}^\ast) = \{(0,0)\}$. Then
  for each $\phi \in \F^N$, there exists a unique $i_\phi \in {(\F^N)}^\ast$
  such that $(\phi,i_\phi) \in L$. Indeed, if $i_\phi$ and $i_\phi'$ were
  distinct elements of ${(\F^N)}^\ast$ with this property, then by linearity
  $(0,i_\phi-i_\phi')$ would be a nonzero element of $L \cap (\{0\} \oplus
  {(\F^N)}^\ast)$, contradicting the hypothesis about trivial intersection. We
  thus can define a function, indeed a linear map, $\F^N \to {(\F^N)}^\ast; \phi
  \mapsto i_\phi$. This defines a bilinear form $Q(\phi,\psi) = i_\phi(\psi)$ on
  $\F^N \oplus \F^N$, and so $Q(\phi) = i_\phi(\phi)$ defines a quadratic form
  on $\F^N$. 

  Moreover, $L$ is Lagrangian, so
  \[
    \omega\big((\phi,i_\phi), (\psi,i_\psi)\big) = i_\psi(\phi) - i_\phi(\psi) =
    0,
  \]
  and so $Q(-,-)$ is a symmetric bilinear form. This gives a one-to-one
  correspondence between Lagrangian subspaces of specified type, symmetric
  bilinear forms, and quadratic forms, and so in particular gives the claimed
  one-to-one correspondence. 
\end{proof}

In particular, every Dirichlet form defines a Lagrangian subspace. 

\subsection{Lagrangian relations}

Recall that a relation between sets $X$ and $Y$ is a subset $R$ of their product
$X \times Y$. Furthermore, given relations $R \subseteq X \times Y$ and $S
\subseteq Y \times Z$, there is a composite relation $(S \circ R) \subseteq X
\times Z$ given by pairs $(x,z)$ such that there exists $y \in Y$ with $(x,y)
\in R$ and $(y,z) \in S$---a direct generalization of function composition. A
Lagrangian relation between symplectic vector spaces $V_1$ and $V_2$ is a
relation between $V_1$ and $V_2$ that forms a Lagrangian subspace of the
symplectic vector space $\overline{V_1} \oplus V_2$. This gives us a
way to think of certain Lagrangian subspaces, such as those arising from
circuits, as morphisms, giving a way to compose them.

Recall discussion of names/conames in Chapter refONE.
\begin{definition}
  A \define{Lagrangian relation} $L\maps V_1 \to V_2$ is a Lagrangian subspace $L$
  of $\overline{V_1} \oplus V_2$. 
\end{definition}

This is a generalization of the notion of symplectomorphism: any symplectomorphism
$f\maps V_1 \to V_2$ forms a Lagrangian subspace when viewed as a
relation $f \subseteq \overline{V_1} \oplus V_2$. More generally, any symplectic
map $f\maps V_1 \to V_2$ forms an isotropic subspace when viewed as a relation in
$\overline{V_1} \oplus V_2$. 

Importantly for us, the composite of two Lagrangian relations is again a
Lagrangian relation.  This is well-known \cite{Weinstein}, but sufficiently 
easy and important to us that we provide a proof.

\begin{proposition} \label{prop:lagrangian_composition}
  Let $L\maps V_1 \to V_2$ and $L'\maps V_2 \to V_3$ be Lagrangian relations. Then their
  composite relation $L' \circ L$ is a Lagrangian relation $V_1 \to V_3$.
\end{proposition}

We prove this proposition by way of two lemmas detailing how the Lagrangian
property is preserved under various operations. The first lemma says that the
intersection of a Lagrangian space with a coisotropic space is in some sense
Lagrangian, once we account for the complement.

\begin{lemma} \label{restriction_of_lagrangians}
  Let $L \subseteq V$ be a Lagrangian subspace of a symplectic vector space $V$,
  and $S \subseteq V$ be an isotropic subspace of $V$. Then $(L\cap S^\circ) +S
  \subseteq V$ is Lagrangian in $V$.
\end{lemma}
\begin{proof}
  Recall from Proposition \ref{lagrangian_characterization} that a subspace is
  Lagrangian if and only if it is equal to its complement. The lemma is then
  immediate from the way taking the symplectic complement interacts with sums
  and intersections:
  \begin{multline*}
    ((L\cap S^\circ) +S)^\circ = (L\cap S^\circ)^\circ \cap S^\circ = (L^\circ +
    (S^\circ)^\circ) \cap S^\circ \\
    = (L+S) \cap S^\circ = (L \cap S^\circ)+(S
    \cap S^\circ) = (L\cap S^\circ) +S.
  \end{multline*}
  Since $(L\cap S^\circ) +S$ is equal to its complement, it is Lagrangian.
\end{proof}

The second lemma says that if a subspace of a coisotropic space is Lagrangian,
taking quotients by the complementary isotropic space does not affect this.

\begin{lemma} \label{quotients_of_lagrangians}
  Let $L \subseteq V$ be a Lagrangian subspace of a symplectic vector space $V$,
  and $S \subseteq L$ an isotropic subspace of $V$ contained in $L$. Then $L/S
  \subseteq S^\circ/S$ is Lagrangian in the quotient symplectic space
  $S^\circ/S$.
\end{lemma}
\begin{proof}
  As $L$ is isotropic and the symplectic form on $S^\circ/S$ is given by
  $\omega'(v+S,u+S) = \omega(v,u)$, the quotient $L/S$ is immediately isotropic.
  Recall from Proposition \ref{lagrangian_characterization} that an isotropic
  subspace $S$ of a symplectic vector space $V$ is Lagrangian if and only if
  $\dim S = \frac12 \dim V$. Also recall that for any subspace $\dim S + \dim
  S^\circ = \dim V$. Thus
  \begin{multline*}
    \dim(L/S) = \dim L - \dim S = \tfrac12 \dim V - \dim S \\ = \tfrac12(\dim S
    + \dim S^\circ) - \dim S = \tfrac12(\dim S^\circ - \dim S) = \tfrac12
    \dim(S^\circ/S).
  \end{multline*}
  Thus $L/S$ is Lagrangian in $S^\circ/S$.
\end{proof}

Combining these two lemmas gives a proof that the composite of two Lagrangian
relations is again a Lagrangian relation.

\begin{proof}[Proof of Proposition \ref{prop:lagrangian_composition}]
  Let $\Delta$ be the diagonal subspace
  \[
    \Delta = \{(0,v_2,v_2,0) \mid v_2 \in V_2\} \subseteq \overline{V_1} \oplus
    V_2 \oplus \overline{V_2} \oplus V_3.
  \]
  Observe that $\Delta$ is isotropic, and has coisotropic complement
  \[
    \Delta^\circ = \{(v_1,v_2,v_2,v_3) \mid v_i \in V_i\} \subseteq
    \overline{V_1} \oplus V_2 \oplus \overline{V_2} \oplus V_3.
  \]
  As $\Delta$ is the kernel of the restriction of the projection map
  $\overline{V_1} \oplus V_2 \oplus \overline{V_2} \oplus V_3 \to \overline{V_1}
  \oplus V_3$ to $\Delta^\circ$, and after restriction this map is still
  surjective, the quotient space $\Delta^\circ/\Delta$ is isomorphic to
  $\overline{V_1} \oplus V_3$. 

  Now, by definition of composition of relations, 
  \[
    L' \circ L = \{(v_1,v_3) \mid \mbox{there exists } v_2 \in V_2 \mbox{ such
    that } (v_1,v_2) \in L, (v_2,v_3) \in L'\}.
  \]
  But note also that 
  \[
    L \oplus L'  = \{(v_1,v_2,v_2',v_3) \mid (v_1,v_2) \in L, (v_2',v_3) \in
    L'\},
  \]
  so 
  \[
    (L \oplus L')\cap \Delta^\circ = \{(v_1,v_2,v_2,v_3) \mid \mbox{there exists
    } v_2 \in V_2 \mbox{ such that } (v_1,v_2) \in L, (v_2,v_3) \in L'\}.
  \]
  Quotienting by $\Delta$ then gives
  \[
    L' \circ L = ((L \oplus L')\cap \Delta^\circ)+\Delta)/\Delta.
  \]
  As $L' \oplus L$ is Lagrangian in $\overline{V_1} \oplus V_2 \oplus
  \overline{V_2} \oplus V_3$, Lemma \ref{restriction_of_lagrangians} says that
  $(L' \oplus L)\cap \Delta^\circ)+\Delta$ is also Lagrangian in $\overline{V_1}
  \oplus V_2 \oplus \overline{V_2} \oplus V_3$. Lemma
  \ref{quotients_of_lagrangians} thus shows that $L' \circ L$ is Lagrangian in
  $\Delta^\circ/\Delta = \overline{V_1} \oplus V_3$, as required.
\end{proof}

Note that this composition is associative. We shall prove this composition
agrees with composition of Dirichlet forms, and hence also composition of
circuits. 

\subsection{The symmetric monoidal category of Lagrangian relations}

Lagrangian relations solve the identity problems we had with Dirichlet forms:
given a symplectic vector space $V$, the Lagrangian relation $\idn\maps V \to V$
specified by the Lagrangian subspace
\[
  \idn = \{(v,v) \mid v \in V\} \subseteq \overline{V} \oplus V,
\]
acts as an identity for composition of relations. We thus have a category.

\begin{definition}
  We write \define{$\LagrRel$} for the category with symplectic
  vector spaces as objects and Lagrangian relations as morphisms. 
\end{definition}

We define the tensor product of two objects of $\LagrRel$ to be their
direct sum. Similarly, we define the tensor product of two morphisms $L\maps U
\to V$, $L \subseteq \overline{U}\oplus V$ and $K\maps T \to W$, $K \subseteq
\overline{T} \oplus W$ to be their direct sum
\[
  L \oplus K \subseteq \overline{U}\oplus V \oplus\overline{T} \oplus W,
\]
\emph{but} considered as a subspace of the naturally isomorphic space
$\overline{U \oplus T} \oplus V \oplus W$.  Despite this subtlety, we abuse our
notation and write their tensor product $L \oplus K\maps U \oplus T \to V \oplus
W$, and move on having sounded this note of caution. 

Note that the direct sum of two Lagrangian subspaces is
again Lagrangian in the direct sum of their ambient spaces, and the zero
dimensional vector space $\{0\}$ acts as an identity for direct sum. Indeed,
defining for all objects $U,V,W$ in $\LagrRel$ unitors: 
\begin{align*}
  \lambda_V &= \{(0,v,v)\} \subseteq \overline{\{0\} \oplus V} \oplus V, \\
  \rho_V &= \{(v,0,v)\} \subseteq \overline{V \oplus \{0\}} \oplus V,
\end{align*}
associators:
\[
  \alpha_{U,V,W}= \{(u,v,w,u,v,w)\} \subseteq \overline{(U \oplus V)\oplus W}
  \oplus U \oplus (V \oplus W),
\]
and braidings:
\[
  \s_{U,V} = \{(u,v,v,u) \mid u \in U, v \in V\} \subseteq \overline{U \oplus V}
  \oplus V \oplus U,
\]
we have a symmetric monoidal category.  Note that all these structure
maps come from symplectomorphisms between the domain and codomain. From this
viewpoint it is immediate that all the necessary diagrams commute, so we
have a symmetric monoidal category. 

In fact the move to the setting of Lagrangian relations, rather than Dirichlet
forms, adds far richer structure than just identity morphisms. In the next
section we show $\LagrRel$ is a hypergraph category.

\section{Ideal wires and corelations} \label{sec:corel}
%%fakesubsection
We want to give a decorated corelations construction of $\LagrRel$ for two main
reasons: first, it endows $\LagrRel$ with a hypergraph structure, and second, 
Key point: examining the graphical elements that facilitate composition---here
ideal wires---shows us how to define hypergraph structure and get decorated
corelations construction. Decorated corelations says that to turn a semantic
function into a semantic functor, it's enough to look at the Frobenius maps. 

In the previous section our exploration of the meaning of circuit diagrams 
culminated with our understanding of behaviours as Lagrangian subspaces.  We now 
turn our attention to how circuit components fit together, and the category of 
operations.  In this section we shall see that the algebra of connections is 
described by the concept of corelations, a generalization of the notion of 
function that forgets the directionality from the domain to the codomain. We
then observe that Kirchhoff's laws follow directly from interpreting these
structures in the category of linear relations.

\subsection{Ideal wires as corelations}

The term corelation in this section we exclusively refer to epi-mono corelations
in $\FinSet$. As such, we write this hypergraph category $\mathrm{Corel}$. 

To motivate the definition of this category, let us start with a set of input
terminals $X$, and a set of output terminals $Y$.  We may connect these
terminals with ideal wires of zero impedance, whichever way we like---input to
input, output to output, input to output---producing something like:
\[
  \begin{tikzpicture}[circuit ee IEC]
	\begin{pgfonlayer}{nodelayer}
		\node [contact] (0) at (-2, 1) {};
		\node [contact] (1) at (-2, 0.5) {};
		\node [contact] (2) at (-2, -0) {};
		\node [contact] (3) at (-2, -0.5) {};
		\node [contact] (4) at (-2, -1) {};
		\node [contact] (5) at (1, 0.75) {};
		\node [contact] (6) at (1, 0.25) {};
		\node [contact] (7) at (1, -0.25) {};
		\node [contact] (8) at (1, -0.75) {};
		\node [style=none] (9) at (-2.75, -0) {$X$};
		\node [style=none] (10) at (1.75, -0) {$Y$};
	\end{pgfonlayer}
	\begin{pgfonlayer}{edgelayer}
	  \draw [thick] (0.center) to (5.center);
		\draw [thick] (5.center) to (1.center);
		\draw [thick] (6.center) to (1.center);
		\draw [thick] (3.center) to (2.center);
		\draw [thick] (4.center) to (8.center);
		\draw [thick] (5.center) to (6.center);
		\draw [thick] (6.center) to (0.center);
	\end{pgfonlayer}
\end{tikzpicture}
\]
In doing so, we introduce a notion of equivalence on our terminals, where two 
terminals are equivalent if we, or if electrons, can traverse from one to 
another via some sequence of wires.   Because of this, we consider our 
perfectly-conducting components to be equivalence relations on $X+Y$,
transforming the above picture into
\[
  \begin{tikzpicture}[circuit ee IEC]
	\begin{pgfonlayer}{nodelayer}
		\node [contact, outer sep=5pt] (0) at (-2, 1) {};
		\node [contact, outer sep=5pt] (1) at (-2, 0.5) {};
		\node [contact, outer sep=5pt] (2) at (-2, -0) {};
		\node [contact, outer sep=5pt] (3) at (-2, -0.5) {};
		\node [contact, outer sep=5pt] (4) at (-2, -1) {};
		\node [contact, outer sep=5pt] (5) at (1, 0.75) {};
		\node [contact, outer sep=5pt] (6) at (1, 0.25) {};
		\node [contact, outer sep=5pt] (7) at (1, -0.25) {};
		\node [contact, outer sep=5pt] (8) at (1, -0.75) {};
		\node [style=none] (9) at (-2.75, -0) {$X$};
		\node [style=none] (10) at (1.75, -0) {$Y$};
		\node [style=none] (11) at (-0.5, 0.625) {};
		\node [style=none] (12) at (-0.5, -0.25) {};
		\node [style=none] (13) at (-0.5, -0.875) {};
	\end{pgfonlayer}
	\begin{pgfonlayer}{edgelayer}
		\draw [color=gray] (0.center) to (11.center);
		\draw [color=gray] (1.center) to (11.center);
		\draw [color=gray] (5.center) to (11.center);
		\draw [color=gray] (6.center) to (11.center);
		\draw [color=gray] (2.center) to (12.center);
		\draw [color=gray] (12.center) to (3.center);
		\draw [color=gray] (4.center) to (13.center);
		\draw [color=gray] (13.center) to (8.center);
		\draw [rounded corners=5pt, dotted] 
   (node cs:name=0, anchor=north west) --
   (node cs:name=1, anchor=south west) --
   (node cs:name=6, anchor=south east) --
   (node cs:name=5, anchor=north east) --
   cycle;
		\draw [rounded corners=5pt, dotted] 
   (node cs:name=2, anchor=north west) --
   (node cs:name=3, anchor=south west) --
   (node cs:name=3, anchor=south east) --
   (node cs:name=2, anchor=north east) --
   cycle;
		\draw [rounded corners=5pt, dotted] 
   (node cs:name=4, anchor=north west) --
   (node cs:name=4, anchor=south west) --
   (node cs:name=8, anchor=south east) --
   (node cs:name=8, anchor=north east) --
   cycle;
		\draw [rounded corners=5pt, dotted] 
   (node cs:name=7, anchor=north west) --
   (node cs:name=7, anchor=south west) --
   (node cs:name=7, anchor=south east) --
   (node cs:name=7, anchor=north east) --
   cycle;
	\end{pgfonlayer}
\end{tikzpicture}
\]
The dotted lines indicate equivalence classes of points, while for reference the
grey lines indicate ideal wires connecting these points, running through a
central hub.

Given another circuit of this sort, say from sets $Y$ to $Z$,
\[
\begin{tikzpicture}[circuit ee IEC]
	\begin{pgfonlayer}{nodelayer}
		\node [style=none] (0) at (-2.75, -0) {$Y$};
		\node [style=none] (1) at (1.75, 0) {$Z$};
		\node [contact, outer sep=5pt] (2) at (-2, 0.75) {};
		\node [contact, outer sep=5pt] (3) at (-2, 0.25) {};
		\node [contact, outer sep=5pt] (4) at (-2, -0.25) {};
		\node [contact, outer sep=5pt] (5) at (-2, -0.75) {};
		\node [contact, outer sep=5pt] (6) at (1, 1) {};
		\node [contact, outer sep=5pt] (7) at (1, 0.5) {};
		\node [contact, outer sep=5pt] (8) at (1, -0) {};
		\node [contact, outer sep=5pt] (9) at (1, -0.5) {};
		\node [contact, outer sep=5pt] (10) at (1, -1) {};
		\node [style=none] (11) at (-0.5, 0.75) {};
		\node [style=none] (12) at (-0.5, -0.25) {};
		\node [style=none] (13) at (-0.5, -0.875) {};
	\end{pgfonlayer}
	\begin{pgfonlayer}{edgelayer}
	  \draw [color=gray] (2.center) to (11.center);
		\draw [color=gray] (11.center) to (6.center);
		\draw [color=gray] (3.center) to (11.center);
		\draw [color=gray] (8.center) to (12.center);
		\draw [color=gray] (12.center) to (4.center);
		\draw [color=gray] (9.center) to (12.center);
		\draw [color=gray] (5.center) to (13.center);
		\draw [color=gray] (13.center) to (10.center);
	\end{pgfonlayer}
		\draw [rounded corners=5pt, dotted] 
   (node cs:name=2, anchor=north west) --
   (node cs:name=3, anchor=south west) --
   (node cs:name=6, anchor=south east) --
   (node cs:name=6, anchor=north east) --
   cycle;
		\draw [rounded corners=5pt, dotted] 
   (node cs:name=4, anchor=north west) --
   (node cs:name=4, anchor=south west) --
   (node cs:name=9, anchor=south east) --
   (node cs:name=8, anchor=north east) --
   cycle;
		\draw [rounded corners=5pt, dotted] 
   (node cs:name=5, anchor=north west) --
   (node cs:name=5, anchor=south west) --
   (node cs:name=10, anchor=south east) --
   (node cs:name=10, anchor=north east) --
   cycle;
		\draw [rounded corners=5pt, dotted] 
   (node cs:name=7, anchor=north west) --
   (node cs:name=7, anchor=south west) --
   (node cs:name=7, anchor=south east) --
   (node cs:name=7, anchor=north east) --
   cycle;
\end{tikzpicture}
\]
we may combine these circuits in to a circuit $X$ to $Z$
\[
  \begin{aligned}
\begin{tikzpicture}[circuit ee IEC]
	\begin{pgfonlayer}{nodelayer}
		\node [contact, outer sep=5pt] (0) at (1, 0.75) {};
		\node [contact, outer sep=5pt] (1) at (1, 0.25) {};
		\node [contact, outer sep=5pt] (2) at (1, -0.25) {};
		\node [contact, outer sep=5pt] (3) at (1, -0.75) {};
		\node [style=none] (4) at (-2.75, -0) {$X$};
		\node [style=none] (5) at (4.75, -0) {$Z$};
		\node [contact, outer sep=5pt] (6) at (-2, 1) {};
		\node [contact, outer sep=5pt] (7) at (-2, -0.5) {};
		\node [contact, outer sep=5pt] (8) at (-2, 0.5) {};
		\node [contact, outer sep=5pt] (9) at (-2, -0) {};
		\node [contact, outer sep=5pt] (10) at (-2, -1) {};
		\node [contact, outer sep=5pt] (11) at (4, -0) {};
		\node [contact, outer sep=5pt] (12) at (4, -1) {};
		\node [contact, outer sep=5pt] (13) at (4, -0.5) {};
		\node [contact, outer sep=5pt] (14) at (4, 0.5) {};
		\node [style=none] (15) at (-0.5, 0.625) {};
		\node [style=none] (16) at (-0.5, -0.25) {};
		\node [style=none] (17) at (-0.5, -0.875) {};
		\node [style=none] (18) at (2.5, -0.875) {};
		\node [contact, outer sep=5pt] (19) at (4, 1) {};
		\node [style=none] (20) at (1, -1.25) {$Y$};
		\node [style=none] (21) at (2.5, 0.75) {};
		\node [style=none] (22) at (2.5, -0.25) {};
	\end{pgfonlayer}
	\begin{pgfonlayer}{edgelayer}
		\draw [color=gray] (6.center) to (15.center);
		\draw [color=gray] (8.center) to (15.center);
		\draw [color=gray] (0.center) to (15.center);
		\draw [color=gray] (1.center) to (15.center);
		\draw [color=gray] (9.center) to (16.center);
		\draw [color=gray] (7.center) to (16.center);
		\draw [color=gray] (10.center) to (17.center);
		\draw [color=gray] (17.center) to (3.center);
		\draw [color=gray] (3.center) to (18.center);
		\draw [color=gray] (18.center) to (12.center);
		\draw [color=gray] (0.center) to (21.center);
		\draw [color=gray] (1.center) to (21.center);
		\draw [color=gray] (21.center) to (19.center);
		\draw [color=gray] (2.center) to (22.center);
		\draw [color=gray] (22.center) to (11.center);
		\draw [color=gray] (22.center) to (13.center);
		\draw [rounded corners=5pt, dotted] 
   (node cs:name=6, anchor=north west) --
   (node cs:name=8, anchor=south west) --
   (node cs:name=1, anchor=south east) --
   (node cs:name=0, anchor=north east) --
   cycle;
		\draw [rounded corners=5pt, dotted] 
   (node cs:name=9, anchor=north west) --
   (node cs:name=7, anchor=south west) --
   (node cs:name=7, anchor=south east) --
   (node cs:name=9, anchor=north east) --
   cycle;
		\draw [rounded corners=5pt, dotted] 
   (node cs:name=10, anchor=north west) --
   (node cs:name=10, anchor=south west) --
   (node cs:name=3, anchor=south east) --
   (node cs:name=3, anchor=north east) --
   cycle;
		\draw [rounded corners=5pt, dotted] 
   (node cs:name=2, anchor=north west) --
   (node cs:name=2, anchor=south west) --
   (node cs:name=2, anchor=south east) --
   (node cs:name=2, anchor=north east) --
   cycle;
		\draw [rounded corners=5pt, dotted] 
   (node cs:name=0, anchor=north west) --
   (node cs:name=1, anchor=south west) --
   (node cs:name=19, anchor=south east) --
   (node cs:name=19, anchor=north east) --
   cycle;
		\draw [rounded corners=5pt, dotted] 
   (node cs:name=2, anchor=north west) --
   (node cs:name=2, anchor=south west) --
   (node cs:name=13, anchor=south east) --
   (node cs:name=11, anchor=north east) --
   cycle;
		\draw [rounded corners=5pt, dotted] 
   (node cs:name=3, anchor=north west) --
   (node cs:name=3, anchor=south west) --
   (node cs:name=12, anchor=south east) --
   (node cs:name=12, anchor=north east) --
   cycle;
		\draw [rounded corners=5pt, dotted] 
   (node cs:name=14, anchor=north west) --
   (node cs:name=14, anchor=south west) --
   (node cs:name=14, anchor=south east) --
   (node cs:name=14, anchor=north east) --
   cycle;
	\end{pgfonlayer}
\end{tikzpicture}
\end{aligned}
\:
  =
\:
\begin{aligned}
\begin{tikzpicture}[circuit ee IEC]
	\begin{pgfonlayer}{nodelayer}
		\node [style=none] (0) at (-2.75, -0) {$X$};
		\node [style=none] (1) at (1.75, -0) {$Z$};
		\node [contact, outer sep=5pt] (2) at (-2, 1) {};
		\node [contact, outer sep=5pt] (3) at (-2, -0.5) {};
		\node [contact, outer sep=5pt] (4) at (-2, 0.5) {};
		\node [contact, outer sep=5pt] (5) at (-2, -0) {};
		\node [contact, outer sep=5pt] (6) at (-2, -1) {};
		\node [contact, outer sep=5pt] (7) at (1, -0) {};
		\node [contact, outer sep=5pt] (8) at (1, -1) {};
		\node [contact, outer sep=5pt] (9) at (1, -0.5) {};
		\node [contact, outer sep=5pt] (10) at (1, 0.5) {};
		\node [style=none] (11) at (-0.5, 0.875) {};
		\node [style=none] (12) at (-1, -0.25) {};
		\node [contact, outer sep=5pt] (13) at (1, 1) {};
		\node [style=none] (14) at (0, -0.25) {};
	\end{pgfonlayer}
	\begin{pgfonlayer}{edgelayer}
		\draw [color=gray] (2.center) to (11.center);
		\draw [color=gray] (4.center) to (11.center);
		\draw [color=gray] (5.center) to (12.center);
		\draw [color=gray] (3.center) to (12.center);
		\draw [color=gray] (14.center) to (7.center);
		\draw [color=gray] (14.center) to (9.center);
		\draw [color=gray] (6.center) to (8.center);
		\draw [color=gray] (11.center) to (13.center);
		\draw [rounded corners=5pt, dotted] 
   (node cs:name=2, anchor=north west) --
   (node cs:name=4, anchor=south west) --
   (node cs:name=13, anchor=south east) --
   (node cs:name=13, anchor=north east) --
   cycle;
		\draw [rounded corners=5pt, dotted] 
   (node cs:name=5, anchor=north west) --
   (node cs:name=3, anchor=south west) --
   (node cs:name=3, anchor=south east) --
   (node cs:name=5, anchor=north east) --
   cycle;
		\draw [rounded corners=5pt, dotted] 
   (node cs:name=6, anchor=north west) --
   (node cs:name=6, anchor=south west) --
   (node cs:name=8, anchor=south east) --
   (node cs:name=8, anchor=north east) --
   cycle;
		\draw [rounded corners=5pt, dotted] 
   (node cs:name=10, anchor=north west) --
   (node cs:name=10, anchor=south west) --
   (node cs:name=10, anchor=south east) --
   (node cs:name=10, anchor=north east) --
   cycle;
		\draw [rounded corners=5pt, dotted] 
   (node cs:name=7, anchor=north west) --
   (node cs:name=9, anchor=south west) --
   (node cs:name=9, anchor=south east) --
   (node cs:name=7, anchor=north east) --
   cycle;
	\end{pgfonlayer}
\end{tikzpicture}
\end{aligned}
\]
by taking the transitive closure of the two equivalence relations, and then
restricting this to an equivalence relation on $X+Z$. 


In the category of sets we hold the fundamental relationship between sets to be
that of functions. These encode the idea of a deterministic process that takes
each element of one set to a unique element of the other. For the study of
networks this is less appropriate, as the relationship between terminals is not
an input-output one, but rather one of interconnection. 

In particular, the direction of a function becomes irrelevant, and to describe
these interconnections via the category of sets we must develop an understanding
of how to compose functions head to head and tail to tail. We have so far used
cospans and pushouts to address this.  Cospans, however, come with an apex, which
represents extraneous structure beyond the two sets we wish to specify a
relationship between. Corelations arise from omitting this information.

  A corelation $\alpha\maps X \to Y$ between finite sets $X$ and $Y$ is a
partition $\alpha$ of the disjoint union $X+Y$.

To introduce some notation, for finite sets $X$ and $Y$, a corelation is a collection of nonempty subsets
$\alpha = \{A_1,A_2,\dots,A_n\}$ of $X+Y$ such that
\begin{enumerate}[(i)] 
  \item $\alpha$ does not contain the empty set.  
  \item $\bigcup_{i=1}^n A_i = X+Y$.
  \item $A_i \cap A_j = \varnothing$ whenever $i \ne j$.
\end{enumerate}

For example, we can take a circuit of ideal wires with $X$ as the set of inputs
and $Y$ as the set of outputs:
\[
  \begin{tikzpicture}[circuit ee IEC]
	\begin{pgfonlayer}{nodelayer}
		\node [contact] (0) at (-2, 1) {};
		\node [contact] (1) at (-2, 0.5) {};
		\node [contact] (2) at (-2, -0) {};
		\node [contact] (3) at (-2, -0.5) {};
		\node [contact] (4) at (-2, -1) {};
		\node [contact] (5) at (1, 0.75) {};
		\node [contact] (6) at (1, 0.25) {};
		\node [contact] (7) at (1, -0.25) {};
		\node [contact] (8) at (1, -0.75) {};
		\node [style=none] (9) at (-2.75, -0) {$X$};
		\node [style=none] (10) at (1.75, -0) {$Y$};
	\end{pgfonlayer}
	\begin{pgfonlayer}{edgelayer}
	  \draw [thick] (0.center) to (5.center);
		\draw [thick] (5.center) to (1.center);
		\draw [thick] (6.center) to (1.center);
		\draw [thick] (3.center) to (2.center);
		\draw [thick] (4.center) to (8.center);
		\draw [thick] (5.center) to (6.center);
		\draw [thick] (6.center) to (0.center);
	\end{pgfonlayer}
\end{tikzpicture}
\]
and define a corelation $\alpha \maps X \to Y$ for which terminals lie in 
the same set of the partition $\alpha = \{A_1,A_2,\dots,A_n\}$ when we can travel from one terminal to another following a path of wires:
\[
  \begin{tikzpicture}[circuit ee IEC]
	\begin{pgfonlayer}{nodelayer}
		\node [contact, outer sep=5pt] (0) at (-2, 1) {};
		\node [contact, outer sep=5pt] (1) at (-2, 0.5) {};
		\node [contact, outer sep=5pt] (2) at (-2, -0) {};
		\node [contact, outer sep=5pt] (3) at (-2, -0.5) {};
		\node [contact, outer sep=5pt] (4) at (-2, -1) {};
		\node [contact, outer sep=5pt] (5) at (1, 0.75) {};
		\node [contact, outer sep=5pt] (6) at (1, 0.25) {};
		\node [contact, outer sep=5pt] (7) at (1, -0.25) {};
		\node [contact, outer sep=5pt] (8) at (1, -0.75) {};
		\node [style=none] (9) at (-2.75, -0) {$X$};
		\node [style=none] (10) at (1.75, -0) {$Y$};
		\node [style=none] (11) at (-0.5, 0.625) {};
		\node [style=none] (12) at (-0.5, -0.25) {};
		\node [style=none] (13) at (-0.5, -0.875) {};
	\end{pgfonlayer}
	\begin{pgfonlayer}{edgelayer}
		\draw [color=gray] (0.center) to (11.center);
		\draw [color=gray] (1.center) to (11.center);
		\draw [color=gray] (5.center) to (11.center);
		\draw [color=gray] (6.center) to (11.center);
		\draw [color=gray] (2.center) to (12.center);
		\draw [color=gray] (12.center) to (3.center);
		\draw [color=gray] (4.center) to (13.center);
		\draw [color=gray] (13.center) to (8.center);
		\draw [rounded corners=5pt, dotted] 
   (node cs:name=0, anchor=north west) --
   (node cs:name=1, anchor=south west) --
   (node cs:name=6, anchor=south east) --
   (node cs:name=5, anchor=north east) --
   cycle;
		\draw [rounded corners=5pt, dotted] 
   (node cs:name=2, anchor=north west) --
   (node cs:name=3, anchor=south west) --
   (node cs:name=3, anchor=south east) --
   (node cs:name=2, anchor=north east) --
   cycle;
		\draw [rounded corners=5pt, dotted] 
   (node cs:name=4, anchor=north west) --
   (node cs:name=4, anchor=south west) --
   (node cs:name=8, anchor=south east) --
   (node cs:name=8, anchor=north east) --
   cycle;
		\draw [rounded corners=5pt, dotted] 
   (node cs:name=7, anchor=north west) --
   (node cs:name=7, anchor=south west) --
   (node cs:name=7, anchor=south east) --
   (node cs:name=7, anchor=north east) --
   cycle;
	\end{pgfonlayer}
\end{tikzpicture}
\]

The discussion in the previous section then motivates a rule for composing corelations. We compose a corelation $\alpha \maps X \to Y$ and a corelation $\beta \maps Y \to Z$ by finding the finest partition on $X+Y+Z$ that is coarser than both $\alpha$ and $\beta$ when restricted to $X+Y$ and $Y+Z$ respectively, and then restricting this to a
partition on $X+Z$. More explicitly, the corelation $\beta\circ\alpha = \{C_1,
C_2, \dots, C_m\}$ is the unique set of pairwise disjoint $C_i$ of the form 
\[
  C_i = \bigcup_j A_j \cap X \cup \bigcup_k B_k \cap Z
\]
for $j,k$ varying over indices such that 
\[
  \bigcup_j A_j \cap Y = \bigcup_k B_k \cap Y.
\]
This rule for composition is associative, as both pairwise methods of composing relations
$\alpha\maps X \to Y$, $\beta\maps Y \to Z$, and $\gamma\maps Z \to W$ amount to finding the finest partition on $X+Y+Z+W$ that is coarser than each of $\alpha$, $\beta$, and $\gamma$ when restricted to the relevant subset, and then restricting this
partition to a partition on $X+W$.  The pictures in the previous section make this clear.

\subsection{Potentials on corelations} \label{ssec:potentialsoncorelations}

Chasing our interpretation of corelations as ideal wires, our aim for the
remainder of this section is to build a functor
\[
  S\maps \mathrm{Corel} \longrightarrow \LagrRel
\]
that expresses this interpretation. We break this functor down into the sum 
of two parts, according to the behaviours of potentials and currents
respectively. 

The consideration of potentials gives a functor $\Phi\maps \mathrm{Corel}
\to \mathrm{LinRel}$, where $\mathrm{LinRel}$ is the symmetric
monoidal dagger category of finite-dimensional $\F$-vector spaces, linear
relations, direct sum, and transpose. In particular, this functor expresses
Kirchhoff's voltage law: it requires that if two elements are in the same part
of the corelation partition---that is, if two nodes are connected by ideal
wires---then the potential at those two points must be the same.

This functor is a generalization of the
contravariant functor $\mathrm{FinSet} \to \mathrm{Vect}$ that maps a set to the
vector space of $\F$-valued functions on that set.

\begin{proposition}
  Define the functor 
  \[ 
    \Phi\maps \mathrm{Corel} \longrightarrow \mathrm{LinRel}, 
  \] 
  on objects by sending a finite set $X$ to the vector space $\F^X$, and on
  morphisms by sending a corelation $\alpha\maps X \to Y$ to the linear subspace
  $\Phi(\alpha)$ of $\F^X \oplus \F^Y$ comprising functions $\phi =
  [\phi_X,\phi_Y]\maps X+Y \to \F$ that are constant on each element of
  $\alpha$.  This is a hypergraph functor.
\end{proposition}

\begin{proof}
  For coherence maps we take the usual natural isomorphisms $\F^X \oplus \F^Y
  \cong \F^{X \times Y}$ and $\{0\} \cong \F^\varnothing$. We detail only the
  proof that $\Phi$ preserves composition; the other properties are
  straightforward to check.

  Let $\alpha\maps X \to Y$ and $\beta\maps Y \to Z$ be corelations. As $\Phi$
  maps corelations to relations, it is enough to check both inclusions
  $\Phi(\beta) \circ \Phi(\alpha) \subseteq \Phi(\beta\circ\alpha)$ and
  $\Phi(\beta\circ\alpha) \subseteq \Phi(\beta) \circ \Phi(\alpha)$. 

  \paragraph{$\Phi(\beta) \circ \Phi(\alpha) \subseteq \Phi(\beta\circ\alpha)$:}
  Let $\phi = [\phi_X,\phi_Z] \in \Phi(\beta) \circ \Phi(\alpha)$. We wish to show
  that for all $C_i \in \beta\circ\alpha$, for all $c,c' \in C_i$, we have
  $\phi(c) = \phi(c')$. To this end, note that there exists some $\phi_Y\maps Y \to \F$
  such that $\phi_{XY} := [\phi_X,\phi_Y] \in \Phi(\alpha)$ and $\phi_{YZ} :=
  [\phi_Y,\phi_Z] \in \Phi(\beta)$.  Furthermore, by definition this $\phi_Y$ has
  the property that for all $A_j \in \alpha$, for all $a,a' \in A_j$, we have
  $\phi_{XY}(a) = \phi_{XY}(a')$, and for all $B_k \in \beta$, for all $b,b' \in
  B_k$, we have $\phi_{YZ}(b) = \phi_{YZ}(b')$. Write $\phi_{XYZ} =
  [\phi_X,\phi_Y,\phi_Z]\maps X+Y+Z \to \F$.

  Our desired fact is thus true: for all $c,c' \in C_i$, there exists a sequence $c=c_0,
  c_1, \dots, c_n=c'$ in $X+Y+Z$ such that for all $\ell=0,1, \dots n-1$ we have
  either $c_\ell, c_{\ell+1} \in A_j$ for some $j$ or $c_\ell,c_{\ell+1} \in B_k$
  for some $k$, and hence that 
  \[
    \phi(c) = \phi_{X+Y+Z}(c_0)= \phi_{X+Y+Z}(c_1) = \dots = \phi_{X+Y+Z}(c_n) =
    \phi(c')
  \]
  as required.

  \paragraph{$\Phi(\beta\circ\alpha) \subseteq \Phi(\beta) \circ \Phi(\alpha)$:}
  Let $\phi = [\phi_X,\phi_Z] \in \Phi(\beta\circ\alpha)$. We must show that there
  exists $\phi_Y\maps Y \to \F$ such that $[\phi_X,\phi_Y] \in \Phi(\alpha)$ and
  $[\phi_Y,\phi_Z] \in \Phi(\beta)$. We claim
  \[
    \phi_Y(y):= \begin{cases}
      \phi(x) & \mbox{if } x \in X, \: x,y \in A_j \mbox{ for some }j;\\
      \phi(z) & \mbox{if } z \in Z, \: y,z \in B_k \mbox{ for some }k;\\
      0 & \mbox{if there exist no such }x \in X \mbox{ or }z \in Z
    \end{cases}
  \]
  satisfies this. This is well-defined as for all $A_j$, $A_j \cap X \subseteq
  C_i$ for some $i$, so $\phi$ is constant on $A_j$, and similarly for all $B_j$.
  Thus it does not matter if there are many such $x$ or $z$ with $x, y \in A_j$ or
  $y,z \in B_k$.  Furthermore, if there exists $y$ such that both $x,y \in A_j$
  for some $x,A_j$ and $y,z \in B_k$ for some $z,B_k$, then the $C_i$ with $A_j
  \cap X \subseteq C_i$ and $B_k \cap Z \subseteq C_i$ is unique, so the
  definitions of $\phi_Y(y)$ do not conflict.

  Moreover, by construction $[\phi_X,\phi_Y]$ is constant on all $A_j$, and
  $[\phi_Y,\phi_Z]$ is constant on all $B_k$, so $[\phi_X,\phi_Y] \in
  \Phi(\alpha)$ and $[\phi_Y,\phi_Z] \in \Phi(\beta)$ as required.
\end{proof}

To recap, we have now constructed a functor $\Phi\maps \mathrm{Corel} \to
\mathrm{LinRel}$ expressing the behaviour of potentials on corelations interpreted
as ideal wires. We now do the same for currents.

\subsection{Currents on corelations}

Next we consider the case of currents, described by a functor $I\maps
\mathrm{Corel} \to \mathrm{LinRel}$. This functor now expresses Kirchhoff's
current law: it requires that the sum of currents flowing into each part of the
corelation partition must equal to the sum of currents flowing out.  It may also
be understood as a generalization of the covariant functor $\mathrm{Set} \to
\mathrm{Vect}$ that maps a set to the vector space of $\F$-linear combinations
of elements of that set.

\begin{proposition}
  Define the functor
  \[
    I\maps \mathrm{Corel} \longrightarrow \mathrm{LinRel}.
  \]
  as follows. On objects send a finite set $X$ to the vector space
  $(\F^{X})^\ast$. On morphisms send a corelation $\alpha\maps X \to Y$
  to the linear relation $I(\alpha)$ comprising precisely those 
  \[
    (i_X,i_Y) = \left(\sum_{x \in X} \lambda_xdx,\sum_{y \in Y}
    \lambda_ydy\right)  \in (\F^{X})^\ast \oplus (\F^{Y})^\ast
  \]
  such that for all $A_i \in \alpha$ the sum of the coefficients of the elements
  of $A_i \cap X$ is equal to that for $A_i \cap Y$:
  \[
    \sum_{x \in A_i \cap X} \lambda_x = \sum_{y \in A_i \cap Y} \lambda_y.
  \]
  This is a hypergraph functor.
\end{proposition}
\begin{proof}
The coherence maps are the natural isomorphisms $(\F^X)^\ast \oplus (\F^Y)^\ast
\to (\F^{X+Y})^\ast$ and $\{0\} \to (\F^\varnothing)^\ast$. Again the only
nontrivial task is to check this map $I$ preserves composition. Again we do this
by checking inclusions $I(\beta) \circ I(\alpha) \subseteq I(\beta\circ\alpha)$
and $I(\beta\circ\alpha) \subseteq I(\beta) \circ I(\alpha)$. 

\paragraph{$I(\beta) \circ I(\alpha) \subseteq I(\beta\circ\alpha)$:} Let
$(i_X,i_Z) \in I(\beta)\circ I(\alpha)$, with $i_X = \sum_{x \in X} \lambda_x dx$
and $i_Z = \sum_{z \in Z} \lambda_z dz$. Note that this implies that there is
some $i_Y = \sum_{y \in Y} \lambda_y dy$ such that $(i_X,i_Y) \in I(\alpha)$,
$(i_Y,i_Z) \in I(\beta)$. Then for each $C_i \in \beta\circ\alpha$ we have
\begin{align*}
  \sum_{x \in C_i \cap X} \lambda_x
  &= \sum_{\substack{x \in A_j \cap X \\ A_j \cap X \subseteq C_i}}
  \lambda_x \\
  &= \sum_{\substack{y \in A_j \cap Y \\ A_j \cap X \subseteq C_i}}
  \lambda_y \tag{By definition of $I(\alpha)$}\\
  &= \sum_{\substack{y \in B_k \cap Y \\ B_k \cap Z \subseteq C_i}}
  \lambda_y \tag{See composition of corelations}\\
  &= \sum_{\substack{z \in B_k \cap Z \\ B_k \cap Z \subseteq C_i}}
  \lambda_z \tag{By definition of $I(\beta)$} \\
  &= \sum_{z \in C_i \cap Z} \lambda_z. 
\end{align*}
Thus $(i_X,i_Z) \in I(\beta\circ\alpha)$.

\paragraph{$I(\beta\circ\alpha) \subseteq I(\beta) \circ I(\alpha)$:} The
reverse inclusion requires a bit more effort. Let $(i_X,i_Z) \in
I(\beta\circ\alpha)$. We wish to construct some $i_Y = \sum_{y \in Y} \lambda_y
dy \in (\F^{Y})^\ast$ such that $(i_X,i_Y) \in I(\alpha)$ and $(i_Y,i_Z) \in
I(\beta)$.  This means that we must find a vector $i_Y \in (\F^{Y})^\ast$
satisfying the $\#\alpha$ linear constraints
\[
  \sum_{y \in A_j \cap Y} \lambda_y = \sum_{x \in A_j \cap X} \lambda_x
\]
and the $\#\beta$ linear constraints
\[
  \sum_{y \in B_j \cap Y} \lambda_y = \sum_{z \in B_j \cap Z} \lambda_z.
\]
Note, however, that for each $C_i \in \beta\circ\alpha$, summing over the $A_j$
and $B_k$ that intersect $C_i$ shows a linear dependence between these constraints
themselves, as
\[
  \sum_{\substack{y \in A_j \cap Y \\ A_j \cap X \subseteq C_i}} \lambda_y =
  \sum_{x \in C_i \cap X} \lambda_x = \sum_{z \in C_i \cap Z}
  \lambda_z = \sum_{\substack{y \in B_k \cap Y \\ B_k \cap Z \subseteq C_i}}
  \lambda_y.
\]
Moreover, as each element of $Y$ lies in exactly one element of each $\alpha$
and $\beta$, we may view them as edges in a graph on $\alpha+\beta$, with
exactly $\#(\beta\circ\alpha)$ connected components. This implies that
\[
  \# Y > \#\alpha + \#\beta -\#(\beta\circ\alpha).
\]
Thus these constraints define an affine subspace of $(\F^{Y})^\ast$ of positive
dimension, and hence we can always find a vector $i_Y$ with the desired
property. This proves the claim. 
\end{proof}

Using elementary methods, an algorithm can also
be given to construct an explicit solution.

\subsection{The symplectification functor}

We have now defined functors that, when interpreting corelations as connections
of ideal wires, describe the behaviours of the currents and potentials at the
terminals of these wires. In this section, we combine these to define a single
functor $S\maps \mathrm{Corel} \to \LagrRel$ describing the behaviour of both
currents and potentials as a Lagrangian subspace.

\begin{proposition} \label{prop:sympfunctor}
  We define the symplectification functor
  \[
    S\maps \mathrm{Corel} \longrightarrow \LagrRel
  \]
  sending a finite set $X$ to the symplectic vector space
\[
  S(X) = \F^X \oplus (\F^X)^\ast,
\]
 and a corelation $\alpha\maps X \to Y$ to the Lagrangian relation
\[
  S(\alpha) = \Phi(\alpha) \oplus I(\alpha) \subseteq \overline{\F^X \oplus
  (\F^X)^\ast}\oplus \F^Y \oplus (\F^Y)^\ast.
\]
  Then $S$ is a strong symmetric monoidal functor, with coherence maps inherited
  from $\Phi$ and $I$.
\end{proposition}
\begin{proof}
  As $S$ is the monoidal product in $\mathrm{LinRel}$ of the strong symmetric
  monoidal functors $\Phi, I\maps \mathrm{Corel} \to \mathrm{LinRel}$, it is
  itself a strong symmetric monoidal functor $\mathrm{Corel} \to
  \mathrm{LinRel}$. Thus it only remains to be checked that, with respect to the
  symplectic structure we put on the objects $S(X)$, the image of each
  corelation $S(\alpha)$ is Lagrangian. 

  This follows from condition (v) of Proposition
  \ref{lagrangian_characterization}: $S(\alpha)$ is (i) isotropic as, for all 
  $(\phi_X,i_X,\phi_Y,i_Y)$, $(\phi_X',i_X',\phi_Y',i_Y') \in S(\alpha)$ we have
  \begin{align*}
    &\phantom{=} \enskip
    \omega\big((\phi_X,i_X,\phi_Y,i_Y),(\phi_X',i_X',\phi_Y',i_Y')\big)
    \\
    &= -\big(i_X'(\phi_X)-i_X(\phi_X')\big) + i_Y'(\phi_Y)-i_Y(\phi_Y') \\
    &= i_X(\phi_X')-i_Y(\phi_Y') + i_X'(\phi_X)-i_Y'(\phi_Y) \\
    &= \sum_{x \in X} \lambda_x dx(\phi_X') - \sum_{y \in Y} \lambda_y dy(\phi_Y') +
    \sum_{y \in Y} \lambda_y' dy(\phi_Y) - \sum_{x \in X} \lambda_x' dx(\phi_X)\\
    &= \sum_{A_j \in \alpha}\left(\sum_{x \in A_j \cap X} \lambda_x dx(\phi_X') -
    \sum_{y \in A_j \cap Y} \lambda_y dy(\phi_Y')\right) \\
    &\qquad \qquad+ \sum_{A_j \in
    \alpha}\left(\sum_{x \in A_j \cap X} \lambda_x'dx(\phi_X) - \sum_{y \in A_j
    \cap Y} \lambda_y' dy(\phi_Y)\right)\\
    &= \sum_{A_j \in \alpha}\left(\sum_{x \in A_j \cap X} \lambda_x - \sum_{y \in
    A_j \cap Y} \lambda_y \right)k'_{A_j} + \sum_{A_j \in \alpha}\left(\sum_{x \in
    A_j \cap X} \lambda_x'- \sum_{y \in A_j \cap Y} \lambda_y'\right)k_{A_j}\\
    &= 0,
  \end{align*}
  and (ii) has dimension equal to 
  \[
    \dim(\Phi(\alpha))+ \dim(I(\alpha) = \#\alpha+\#(X+Y) - \#\alpha= \#(X+Y).
  \]
  This proves the proposition.
\end{proof}

We have thus shown we do indeed have a functor $S\maps \mathrm{Corel} \to
\LagrRel$. In the next section we shall see that this functor provides
the engine of our black box functor, playing the key role in showing that we
may indeed treat circuit components as black boxes: that is, that circuits that
behave the same compose the same. Before we get there, we quickly make two
relevant observations.

\begin{example}[Symplectification of functions] \label{ex:sympfunction}
  Let $f: X \to Y$ be a function. In this example we show that $Sf$ has the form 
  \[
    Sf = \big\{(\phi_X,i_X,\phi_Y,i_Y) \, \big\vert \, \phi_X = f^\ast\phi_Y,
    i_Y = f_\ast i_X \big\} \subseteq \overline{\F^X \oplus (\F^X)^\ast} \oplus
    \F^Y \oplus (\F^Y)^\ast,
  \]
  where $f^\ast$ is the pullback map
  \begin{align*}
    f^\ast\maps \F^Y &\longrightarrow \F^X; \\
    \phi &\longmapsto \phi \circ f,
  \end{align*}
  and $f_\ast$ is the pushforward map
  \begin{align*}
    f_\ast\maps (\F^X)^\ast &\longrightarrow (\F^Y)^\ast; \\
    i(-) &\longmapsto i(-\circ f).
  \end{align*}
  (We shall also write $f_\ast$ for the more general map from functions on
  $\F^X$ to functions on $\F^Y$ that takes a function $i(-)$ on $\F^X$ to the
  function $i(-\circ f)$.) The claim is then that these pullback and pushforward
  constructions express Kirchhoff's laws.

  Recall that the corelation corresponding to $f$ partitions $X+Y$ into $\#Y$
  parts, each of the form $f^{-1}(y) \cup \{y\}$. The linear relation $\Phi(f)$
  requires that if $x \in X$ and $y \in Y$ lie in the same part of the
  partition, then they have the same potential: that is, $\phi_X(x) =
  \phi_Y(y)$. This is precisely the arrangement imposed by $f^\ast \phi_Y =
  \phi_X$: 
  \[
    \phi_X(x) = \phi_Y(f(x)) =\phi_Y(y).
  \] 
  On the other hand, the linear relation $I(f)$ requires that if $i_X = \sum_{x 
  \in X}\lambda_xdx$, $i_Y = \sum_{y \in Y}\lambda_y dy$, then for each $y \in Y$
  we have 
  \[
    \sum_{x \in f^{-1}(y)} \lambda_x = \lambda_y.
  \]
  This is precisely what is required by $f_\ast$: given any $\phi \in \F^Y$, we
  have
  \[
    f_\ast i_X(\phi) = f_\ast \sum_{x \in X}\lambda_xdx(\phi) = \sum_{x
    \in X}\lambda_xdx(\phi \circ f)= \sum_{y \in Y}\left( \sum_{x \in f^{-1}(y)}
    \lambda_x\right)dy.
  \]
  This gives us the above representation of $Sf$ when $f$ is a function.
\end{example}


\subsection{Lagrangian relations as decorated corelations}
We have a wide functor $\cospan(\FinSet) \to \LagrRel$ (skeleton). Thus
$\LagrRel$ is a hypergraph category. This means we can construct it as decorated
corelations on its monoidal global sections functor.

We begin by describing a category where morphisms are cospans decorated by
Lagrangian subspaces.

\begin{proposition}
Define 
\[
  \mathrm{Lagr}\maps (\mathrm{FinSet},+) \longrightarrow (\mathrm{Set},\times)
\]
as follows. For objects let $\mathrm{Lagr}$ map a finite set $X$ to the set
$\mathrm{Lagr}(X)$ of Lagrangian subspaces of the symplectic vector space
$\vectf{X}$.  For morphisms, recall that a function $f\maps X \to Y$ between
finite sets may be considered as a corelation, and the symplectification functor
$S$ thus maps this corelation to some Lagrangian relation $Sf\maps \vectf{X}
\to \vectf{Y}$. As Lagrangian relations map Lagrangian subspaces to Lagrangian
subspaces (Proposition \ref{prop:lagrangian_composition}), this gives a map: 
\begin{align*}
  \mathrm{Lagr}(f)\maps \mathrm{Lagr}(X) &\longrightarrow \mathrm{Lagr}(Y); \\
  L &\longmapsto Sf(L).
\end{align*}
Moreover, equipping this functor with the family of maps
\begin{align*}
  \lambda_{N,M}\maps \mathrm{Lagr}(N) \times \mathrm{Lagr}(M) &\longrightarrow
  \mathrm{Lagr}(N+M);\\
  (L_N,L_M) &\longmapsto L_N \oplus L_M,
\end{align*}
and unit
\begin{align*}
  \lambda_1\maps 1 &\longrightarrow \mathrm{Lagr}(\varnothing);\\
  \bullet &\longmapsto \{0\}
\end{align*}
defines a lax symmetric monoidal functor.
\end{proposition}
\begin{proof}
  The functoriality of this construction follows from the functoriality of $S$;
  the lax symmetric monoidality from the relevant properties of the direct sum
  of vector spaces.
\end{proof}
We thus obtain a hypergraph category $\mathrm{LagrCospan}$.

At this point we have checked that the process of reinterpreting a circuit as a
Lagrangian subspace of behaviours is functorial. Our task is now to integrate
this information as more than just a `decoration' on our morphisms. This process
constitutes a monoidal dagger functor
\[
  \mathrm{LagrCospan} \longrightarrow \LagrRel.
\]
This factor of the black box functor is the one that gives it its name;
through this functor we finally seal off the inner structure of our circuits,
leaving us access only to the behaviour at the terminals. Its purpose is to take a 
Lagrangian cospan, which captures information about the behaviours of a 
circuit measured at every node, and restrict it down to a 
relation detailing the behaviours simply on the terminals. 

\begin{proposition}
We may define a hypergraph functor
\[
  \mathrm{LagrCospan} \longrightarrow \mathrm{LagrRel}
\]
as follows. On objects let this functor take a finite set $X$ to the symplectic
vector space $\F^X \oplus (\F^X)^\ast$. On morphisms let it take a Lagrangian
cospan
\[
  \big(X\stackrel{i}{\longrightarrow} N \stackrel{o}{\longleftarrow} Y; L
  \subseteq \F^N \oplus (\F^N)^\ast\big)
\]
to the Lagrangian relation
\[
  (S^t\!X\oplus SY) \circ S[i,o]^\dagger \circ L \subseteq
  \overline{\F^X \oplus (\F^X)^\ast} \oplus \F^Y \oplus (\F^Y)^\ast.  
\]  
\end{proposition}

The coherence maps here are the usual natural isomorphisms
\[
  \vectf{X} \oplus \vectf{Y} \stackrel{\sim}{\longrightarrow} \vectf{X+Y} 
\]
and
\[
  \{0\} \stackrel{\sim}{\longrightarrow} \vectf{\varnothing}.
\]

The three functors of this section compose to give the black box functor
\[
\blacksquare\maps \Circ \to \LagrRel.
\] 
Since they are each separately hypergraph functors, the black box
functor is too.


\subsubsection*{Duals for objects}

Each object $V$ of $\LagrRel$ is dual to its conjugate space $\overline
V$, with cup $\eta\maps \{0\} \to \overline{V} \oplus V$ given by 
\[
  \eta = \{(0,v,v) \mid v \in V\} \subseteq \overline{\{0\}} \oplus \overline{V}
  \oplus V
\]
and cap $\eps\maps V \oplus \overline{V} \to \{0\}$ given by
\[
  \eps = \{(v,v,0) \mid v \in V\} \subseteq \overline{V \oplus \overline{V}}
  \oplus \{0\}.
\]
It is straightforward to check these satisfy the zigzag identities.

\subsubsection*{Dagger structure}

Given symplectic vector spaces $U,V$, observe that the map
\begin{align*}
  (-)^\dagger\maps \overline{U} \oplus V &\longrightarrow \overline{V} \oplus U; \\
  (u,v) &\longmapsto (v,u)
\end{align*} 
takes Lagrangian subspaces of the domain to Lagrangian subspaces of the
codomain. Thus we can view it as a map $(-)^\dagger$ taking morphisms $L\maps U \to V$
of $\LagrRel$ to morphisms $L^\dagger\maps V \to U$. This defines a
dagger structure on $\LagrRel$, which makes this category into a
hypergraph category.

Moreover, every object in $\mathrm{LagrRel}$ has a dagger dual: it is clear that
$\eta^\dagger = \eps \circ \s$.   This category thus becomes a dagger compact
category.


\section{The black box functor} \label{sec:blackbox}
%%fakesubsection
We have now developed enough machinery to prove Theorem \ref{main_theorem}:
there is a hypergraph functor, the black box functor
\[  
\blacksquare\maps \Circ \to \LagrRel 
\]
taking passive linear circuits to their behaviours. To recap, we have so far
developed two categories: $\Circ$, in which morphisms are passive linear
circuits, and $\LagrRel$, which captures the external behaviour of such circuits.
We now define a functor that maps each circuit to its behaviour, before proving
it is indeed a hypergraph functor. 

The role of the functor we construct here is to identify all circuits with the
same external behaviour, making the internal structure of the circuit
inaccessible. Circuits treated this way are frequently referred to as
`black boxes', so we call this functor the \define{black box functor},
\[
\blacksquare\maps \Circ \to \LagrRel.
\] 
In this section we first provide the definition of this functor, and then check
that our definition really does map a circuit to its behaviour.

\subsection{Definition}
It should be no surprise that the black box functor maps a finite set $X$ to the
symplectic vector space $\F^X \oplus (\F^X)^\ast$ of potentials and currents
that may be placed on that set. The challenge is to provide a succinct
statement of its action on the circuits themselves. To do this, we take
advantage of four processes we developed in Parts  and
.

Let $\Gamma\maps X \to Y$ be a circuit, represented by the decorated cospan
\[
  \big(X \stackrel{i}{\longrightarrow} N \stackrel{o}{\longleftarrow} Y,\:
  \Gamma\big).
\]
Recall that this means that $X$ and $Y$ are finite sets, $\Gamma$ is a
$\F$-graph $(N,E,s,t,r)$, and we have a cospan of finite sets
\[
  \xymatrix{
    & \Gamma \\
    X \ar[ur]^{i} && Y. \ar[ul]_{o}
  }
\]
To define the image of $\Gamma$ under our functor $\blacksquare$, by definition
a Lagrangian relation $\blacksquare(\Gamma): \blacksquare(X) \to
\blacksquare(Y)$, we must specify a Lagrangian subspace 
\[
  \blacksquare(\Gamma) \subseteq \overline{\F^X \oplus (\F^X)^\ast} \oplus \F^Y
  \oplus (\F^Y)^\ast.  
\]

Recall that to each $\F^+$-graph $\Gamma$ we can associate a Dirichlet form, 
the extended power functional 
\begin{align*}
  P_\Gamma\maps \F^N &\longrightarrow \F; \\
  \phi &\longmapsto \frac{1}{2} \sum_{e \in E} \frac{1}{r(e)} \big( \phi(t(e)) -
  \phi(s(e))  \big)^2,
\end{align*}
and to this Dirichlet form we associate a Lagrangian subspace 
\[
  \mathrm{Graph}(dP_\Gamma) = \{(\phi,d(P_\Gamma)_\phi) \mid \phi \in \F^N\}
  \subseteq \F^N \oplus (\F^N)^\ast.
\]
We consider this Lagrangian subspace as a Lagrangian relation $\{0\} \to \F^N
\oplus (\F^N)^\ast$.

From the legs of the cospan $\Gamma$, the symplectification functor $S$ gives the
Lagrangian relation
\[
  S[i,o]^\dagger\maps \F^N \oplus (\F^N)^\ast \longrightarrow \F^X \oplus
  (\F^X)^\ast \oplus \F^Y \oplus (\F^Y)^\ast.
\]
Writing $Y: Y \to Y$ for the identity morphism on the finite set $Y$, $S$ also
provides a way of writing the identity morphism $SY: \F^Y \oplus (\F^Y)^\ast \to
\F^Y \oplus (\F^Y)^\ast$. 

Lastly, we have the symplectomorphism
\begin{align*}
  S^t\!X\maps \F^X \oplus (\F^X)^\ast &\longrightarrow \overline{\F^X \oplus
  (\F^X)^\ast}; \\
  (\phi,i) &\longmapsto (\phi,-i).
\end{align*}

The black box functor maps a circuit $\Gamma$ to the Lagrangian relation
\[
  (S^t\!X\oplus SY) \circ S[i,o]^\dagger \circ \mathrm{Graph}(P_\Gamma).
\]

As isomorphisms of cospans of $\F^+$-graphs amount to no more than a 
relabelling of nodes and edges, this construction is independent of the cospan 
chosen as representative of the isomorphism class of cospans forming the 
circuit.

We picture this as
\[
  \blacksquare(\Gamma) =\qquad 
  \begin{aligned}
\begin{tikzpicture}
	\begin{pgfonlayer}{nodelayer}
		\node [style=none] (0) at (-1.25, 0.5) {};
		\node [style=none] (1) at (-0.75, 0.5) {};
		\node [style=none] (2) at (-1.25, -0) {};
		\node [style=none] (3) at (-0.75, -0) {};
		\node [style=none] (4) at (-1.25, 1.25) {};
		\node [style=none] (5) at (-1, -0.25) {};
		\node [style=none] (6) at (-1, 0.5) {};
		\node [style=none] (7) at (-1, -0) {};
		\node [style=none] (8) at (0, -0.25) {};
		\node [style=none] (9) at (0, 0.75) {};
		\node [style=none] (10) at (-0.5, 2) {};
		\node [style=none] (11) at (-1, 1.5) {};
		\node [style=none] (12) at (0, 1.5) {};
		\node [style=none] (13) at (-0.5, 1.5) {};
		\node [style=none] (14) at (-1.25, 0.75) {};
		\node [style=none] (15) at (0.25, 1.25) {};
		\node [style=none] (16) at (0.25, 0.75) {};
		\node [style=none] (17) at (-1, 0.75) {};
		\node [style=none] (18) at (-0.5, 1.25) {};
		\node [style=none] (19) at (-0.5, 1.75) {$L_\Gamma$};
		\node [style=none] (20) at (-0.5, 1) {$S[i,o]^\dagger$};
		\node [style=none] (21) at (-1, 0.25) {$S^t\! X$};
		\node [style=none] (22) at (-1.5, -0.75) {};
		\node [style=none] (23) at (-2, -0.25) {};
		\node [style=none] (24) at (-2, 1.5) {};
		\node [style=none] (25) at (-1, 2.5) {};
		\node [style=none] (26) at (-1, -1.25) {};
		\node [style=none] (27) at (-1, 2.75) {$\scriptstyle\blacksquare(X)$};
		\node [style=none] (28) at (-1, -1.5) {$\scriptstyle\blacksquare(Y)$};
	\end{pgfonlayer}
	\begin{pgfonlayer}{edgelayer}
		\draw (0.center) to (1.center);
		\draw (1.center) to (3.center);
		\draw (3.center) to (2.center);
		\draw (2.center) to (0.center);
		\draw (7.center) to (5.center);
		\draw (8.center) to (9.center);
		\draw (10.center) to (11.center);
		\draw (11.center) to (12.center);
		\draw (12.center) to (10.center);
		\draw (4.center) to (14.center);
		\draw (14.center) to (16.center);
		\draw (16.center) to (15.center);
		\draw (15.center) to (4.center);
		\draw (13.center) to (18.center);
		\draw (17.center) to (6.center);
		\draw [in=90, out=-90, looseness=1.00] (8.center) to (26.center);
		\draw [bend left=45, looseness=1.00] (5.center) to (22.center);
		\draw [bend left=45, looseness=1.00] (22.center) to (23.center);
		\draw (23.center) to (24.center);
		\draw [in=-90, out=90, looseness=1.00] (24.center) to (25.center);
	\end{pgfonlayer}
\end{tikzpicture}
\end{aligned}
\]

To summarize:

\begin{definition}
  We define the black box functor 
  \[
    \blacksquare\maps \Circ \to \LagrRel 
  \]
  on objects by mapping a finite set $X$ to the symplectic vector space 
  \[
    \blacksquare(X) = \F^X \oplus (\F^X)^\ast.
  \]
  and on morphisms by mapping a circuit $\Gamma\maps X \to Y$, represented by the
  decorated cospan
  \[
    \big(X \stackrel{i}{\longrightarrow} N \stackrel{o}{\longleftarrow} Y,\:
    \Gamma\big)
  \]
  to the Lagrangian relation
  \[
    \blacksquare(\Gamma) = (S^t\!X \oplus SY) \circ S[i,o]^\dagger \circ
    \mathrm{Graph}(dP_\Gamma).
  \]
\end{definition}

The coherence maps are given by the natural isomorphisms
\[
  \blacksquare(X) \oplus \blacksquare(Y) = \F^X \oplus (\F^X)^\ast \oplus \F^Y
  \oplus (\F^Y)^\ast \cong \F^{X+Y} \oplus (\F^{X+Y})^\ast = \blacksquare(X+Y)
\]
and
  \[
    \{0\} \cong \F^\varnothing \oplus (\F^\varnothing)^\ast =
    \blacksquare(\varnothing).
  \]


\begin{theorem} \label{thm:main}
  The black box functor is a well-defined hypergraph functor.
\end{theorem}

The next, and final, section is devoted to the proof of this theorem. Before we
get there, we first assure ourselves that we have indeed arrived at the theorem
we set out to prove.

\subsection{Minimization via composition of relations}

At this point the reader might voice two concerns: firstly, why does the
\emph{black box} functor refer to the \emph{extended} power functional $P$ and,
secondly, since it fails to talk about power minimization, how is it the same
functor as that defined in Theorem \ref{main_theorem}? These fears are allayed
by the remarkable trinity of minimization, the symplectification of functions,
and Kirchhoff's laws. 

We have seen that symplectification of functions views the cograph of the
function as a picture of ideal wires, governed by Kirchhoff's laws (Example
\ref{ex:sympfunction}). We have also seen that Kirchhoff's laws are closely
related to the principle of minimium power (Theorems
\ref{thm:realizablepotentials} and \ref{thm:dirichletminimization}). The final
aspect of this relationship is that we may use symplectification of functions to
enact minimization.

\begin{theorem} \label{thm:sympmin}
  Let $\iota: \partial N \to N$ be an injection, and let $P$ be a Dirichlet form on
  $N$. Write $Q = \min_{N \setminus \partial N} P$ for the Dirichlet form on
  $\partial N$ given by minimization over $N \setminus \partial N$. Then we have an
  equality of Lagrangian subspaces
  \[
    S\iota^\dagger \big( \mathrm{Graph}(dP)\big) = \mathrm{Graph}(dQ).
  \]
\end{theorem}
\begin{proof}
  Recall from Example \ref{ex:sympfunction} that $S\iota^\dagger$ is the Lagrangian relation
  \[
    S\iota^\dagger = \big\{(\phi, \iota_\ast i,\phi \circ \iota, i) \, \big\vert
      \, \phi \in \F^{N}, i \in (\F^{\partial N})^\ast \big\} \subseteq
      \overline{\F^N \oplus (\F^N)^\ast} \oplus \F^{\partial N} \oplus
      (\F^{\partial N})^\ast,
  \]
  where $\iota_\ast i(\phi) = i(\phi \circ \iota)$, and note that 
  \[
    \mathrm{Graph}(dP) = \big\{(\phi,dP_\phi) \,\big\vert\, \phi \in \F^N\big\}.
  \]
  This implies that their composite is given by the set
  \[
    S\iota^\dagger \circ \mathrm{Graph}(dP) = \big\{(\phi \circ \iota, i)
    \,\big\vert\, \phi \in \F^N, i \in (\F^{\partial N})^\ast, dP_\phi =
  \iota_\ast i \big\}.
  \]
  We must show this Lagrangian subspace is equal to $\mathrm{Graph}(dQ)$.
  
  Consider the constraint $dP_\phi = \iota_\ast i$. This states that for all
  $\varphi \in \F^N$ we have $dP_\phi(\varphi) = i(\varphi\circ \iota)$. Letting
  $\chi_n: N \to \F$ be the function sending $n \in N$ to $1$ and all other
  elements of $N$ to $0$, we see that when $n \in N \setminus \partial N$ we
  must have
  \[
    \frac{dP}{d\varphi(n)}\Bigg\vert_{\varphi = \phi}  = dP_\phi(\chi_n) = 
    i(\chi_n \circ \iota) = i(0) = 0.
  \]
  So $\phi$ must be a realizable extension of $\psi = \phi \circ \iota$. We
  henceforth write $\tilde\psi = \phi$. As $\iota$ is injective, $\psi = \phi
  \circ \iota$ gives no constraint on $\psi \in \F^{\partial N}$. 
 
  We next observe that we can write $S\iota^\dagger \circ \mathrm{Graph}(dP) =
  \mathrm{Graph}(dO)$ for some quadratic form $O$. Recall that Proposition
  \ref{prop:qfls} states that a Lagrangian subspace $L$ of $\F^{\partial N}
  \oplus (\F^{\partial N})^\ast$ is of the form $\mathrm{Graph}(dO)$ if and only
  if $L$ has trivial intersection with $\{0\} \oplus (\F^N)^\ast$. But indeed,
  if $\psi = 0$ then $0$ is a realizable extension of $\psi$, so $\iota_\ast i =
  dP_0 = 0$, and hence $i = 0$. 
  
  It remains to check that $O = Q$. This is a simple computation:
  \[
    O(\psi) = dO_\psi(\psi) = dO_\psi(\tilde\psi \circ \iota) = \iota_\ast
    dQ_\psi(\tilde\psi) = dP_{\tilde\psi}(\tilde\psi) = P(\tilde\psi) = Q(\psi),
  \]
  where $\tilde\psi$ is any realizable extension of $\psi \in \F^{\partial N}$.
\end{proof}

Write $\iota: \partial N \to N$ for the inclusion of the terminals into the set
of nodes of the circuit, and $i\rvert^{\partial N}: X \to \partial N$,
$o\rvert^{\partial N}: Y \to \partial N$ for the respective corestrictions of
the input and output map to $\partial N$. Note that $[i,o] = \iota \circ
[i\rvert^{\partial N}, o\rvert^{\partial N}]$.
Then we have the equalities of sets, and thus Lagrangian relations:
\begin{align*}
  (S^t\!X\oplus 1_Y) \circ S[i,o]^\dagger \circ \mathrm{Graph}(dP_\Gamma)
  &= (S^t\!X\oplus SY) \circ S[i\rvert^{\partial
  N},o\rvert^{\partial N}]^\dagger \circ S\iota^\dagger \circ \mathrm{Graph}(dP_\Gamma) \\
  &= (S^t\!X\oplus SY) \circ S[i\rvert^{\partial
  N},o\rvert^{\partial N}]^\dagger \circ \mathrm{Graph}(dQ_\Gamma) \\
  &= \bigcup_{v \in \mathrm{Graph}(dQ)} S^ti\rvert^{\partial N}(v) \times
  So\rvert^{\partial N}(v)
\end{align*}
We see now that Theorem \ref{thm:main} is a restatement of Theorem
\ref{main_theorem} in the introduction.

\subsection{Proof of functoriality} \label{sec:proof}
%%fakesubsection
To prove that the black box construction is indeed functorial, we
factor it into three functors. These functors are each hypergraph
functors, so the black box functor is too.

This gives a factorization
\[
  \xymatrix{
    \blacksquare\maps \Circ \ar[r] & \mathrm{DirichCospan} \ar[r] &
    \mathrm{LagrCospan} \ar[r] & \LagrRel
  }
\]

  \[
    \xymatrixcolsep{4pc}
    \xymatrixrowsep{3pc}
    \xymatrix{
      (\FinSet,+) \ar^{\mathrm{Circuit}}[r] \ar@{=}[d] \drtwocell
      \omit{_\:\theta_1} & (\Set,\times) \ar@{=}[d]  \\
      (\FinSet,+) \ar^{\mathrm{Dirich}}[r] \ar[d] \drtwocell
      \omit{_\:\theta_2} & (\Set,\times) \ar@{=}[d]  \\
      (\cospan(\FinSet),+) \ar_{\mathrm{Lagr}}[r] & (\Set,\times).
    }
  \]

The next step is to show that our process of turning a Dirichlet form into a Lagrangian subspace---by taking the graph of its differential---is functorial.

\subsubsection*{The functor $\mathrm{DirichCospan} \to \mathrm{LagrCospan}$}

We now wish to construct a strict hypergraph functor
$\mathrm{DirichCospan} \to \mathrm{LagrCospan}$.  For this we need a monoidal natural transformation 
\[
  (\mathrm{Dirich},\delta), (\mathrm{Lagr},\lambda)\maps (\mathrm{FinSet},+)
  \longrightarrow (\mathrm{Set},\times).
\]
\begin{proposition}
Let
\[
  \beta\maps (\mathrm{Dirich},\delta) \Longrightarrow (\mathrm{Lagr},\lambda)
\]
be the collection of functions
\begin{align*}
  \beta_N\maps \mathrm{Dirich}(N) &\longrightarrow \mathrm{Lagr}(N); \\
  Q &\longmapsto \{(\phi,dQ_\phi) \mid \phi \in \F^N\} \subseteq \vectf{N}.
\end{align*}
Then $\beta$ is a monoidal natural transformation.
\end{proposition}

\begin{proof}
Naturality requires that the square
\[
\xymatrix{
  \mathrm{Dirich}(N) \ar[r]^{\beta_N} \ar[d]_{\mathrm{Dirich}(f)} &
  \mathrm{Lagr}(N) \ar[d]^{\mathrm{Lagr}(f)}  \\
  \mathrm{Dirich}(M) \ar[r]_{\beta_M} & \mathrm{Lagr}(M)
}
\]
commutes for every function $f\maps N \to M$. This is primarily a consequence of the
fact that the differential commutes with pullbacks. As we did in Example
\ref{ex:sympfunction},
write $f^\ast$ for the pullback map and $f_\ast$ for the pushforward map.
Then $\mathrm{Dirich}(f)$ maps a Dirichlet form $Q$ on $N$ to the form $f_\ast Q$,
and $\beta_M$ in turn maps this to the Lagrangian subspace 
\[
  \big\{(\psi,d(f_\ast Q)_\psi) \, \big\vert \, \psi \in \F^M\big\} \subseteq
  \F^M \oplus (\F^M)^\ast.
\]
On the other hand, $\beta_N$ maps a Dirichlet form $Q$ on $N$ to the Lagrangian
subspace
\[
\big\{(\phi,dQ_\phi) \,\big\vert\, \phi \in \F^N\big\}\subseteq
  \F^N \oplus (\F^N)^\ast, 
\]
before $\mathrm{Lagr}(f)$ maps this to the Lagrangian subspace
\[
  \big\{(\psi, f_\ast dQ_\phi) \,\big\vert\, \psi \in \F^M, \phi =
  f^\ast(\psi)\big\} \subseteq \F^M\oplus (\F^M)^\ast.
\]
But 
\[
  f_\ast dQ_{f^\ast\psi} = d(f_\ast Q)_{\psi},
\]
so these two processes commute.

Monoidality requires that the diagrams 
\[
\begin{aligned}
\xymatrix{
  \mathrm{Dirich}(N) \times \mathrm{Dirich}(M) \ar[r]^(.52){\beta_N \times
  \beta_M} \ar[d]_{\delta_{N,M}} & \mathrm{Lagr}(N) \times \mathrm{Lagr}(M)
  \ar[d]^{\lambda_{N,M}}  \\
  \mathrm{Dirich}(N+M) \ar[r]_{\beta_{N+M}} & \mathrm{Lagr}(N+M)
}
\end{aligned}
\quad
\mbox{and}
\quad
\begin{aligned}
\xymatrix{
  & 1 \ar[dl]_{\delta_\varnothing} \ar[dr]^{\lambda_\varnothing}\\
\mathrm{Dirich}(\varnothing)  \ar[rr]_{\beta_\varnothing} &&
\mathrm{Lagr}(\varnothing)
}
\end{aligned}
\]
commute. These do: the Lagrangian subspace corresponding to the sum of Dirichlet
forms is equal to the sum of the Lagrangian subspaces that correspond to the
summand Dirichlet forms, while there is only a unique map $1 \to
\mathrm{Lagr}(\varnothing)$.
\end{proof}

We thus obtain a strict hypergraph functor 
\[   
\mathrm{DirichCospan} \to \mathrm{LagrCospan},
\]
which simply replaces the decoration on each cospan in $\mathrm{DirichCospan}$
with the corresponding Lagrangian subspace.


%\section{Concluding remarks}



\chapter*{Notations and conventions}
\addcontentsline{toc}{chapter}{Notation and conventions}

Decorated cospans: 
We shall assume the following standard names for certain distinguished objects
and morphisms, only disambiguating the symbols with subscripts when we judge
that the extra clarity is worth the clutter. We write: 
\begin{itemize} 
  \item $1$ for both identity morphisms and monoidal units, leaving context to
    determine which one we mean.
  \item $\lambda$, $\rho$, $a$, and $\sigma$ for respectively the left unitor, right unitor,
    associator, and, if present, braiding, in a monoidal category.
  \item $\varnothing$ for the initial object in a category.
  \item $!$ for the unique map from the initial object to a given object.
  \item Given a category $\mc C$, write $\mc C^\opp$ for the opposite category.
    Given a morphism $f\maps A \to B \in \mc C$, write $f^\opp\maps B \to A \in
    \mc C^\opp$ for the corresponding morphism in the opposite category.
\end{itemize}

If a monoidal structure on $\Set$ is not specified, we assume it is the
cartesian product. 

If a monoidal structure on $\FinSet$ is not specified, we assume it is the
disjoint union. 



\phantomsection
\addcontentsline{toc}{chapter}{Bibliography}
%\renewcommand{\bibname}{References} %uncomment to change bibliography name to references
\bibliography{auxiliary/dphilrefs}
\bibliographystyle{alpha}  %use the plain bibliography style

\end{document}


% easy access templates

  \[
    \tikzset{every path/.style={line width=1.1pt}}
    \begin{aligned}
    \end{aligned}
    =
    \begin{aligned}
    \end{aligned}
  \]

  \begin{center}
    \begin{tabular}{| c | p{.65\textwidth} |}
      \hline
      \multicolumn{2}{|c|}{The hypergraph category $(,)$} \\
      \hline
      \textbf{objects} & \\ 
      \textbf{morphisms} & \\ 
      \textbf{composition} & \\
      \textbf{monoidal product} & \\
      \textbf{coherence maps} & \\
      \textbf{hypergraph maps} & \\
      \hline
    \end{tabular}
  \end{center}

