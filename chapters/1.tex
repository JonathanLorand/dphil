\chapter[Hypergraph categories: the algebra of interconnection]{Hypergraph
categories: the algebra of interconnection}

In this chapter we introduce hypergraph categories, giving a definition,
coherence theorem, and graphical language. We then explore a fundamental example
of hypergraph categories: categories of cospans.

We assume basic familiarity with category theory and symmetric monoidal
categories; although we give a sparse overview of the latter for reference. A
proper introduction to both can be found in Mac Lane \cite{Mac98}.

\section{The algebra of interconnection}

Our aim is to algebraicise network diagrams. A network diagram is built from
pieces like so:
\[
  \begin{tikzpicture}
    \node [thick, circle, draw] (0) at (0, -0) {};
    \node [style=none] (1) at (-0.75, 1.5) {};
    \node [style=none] (2) at (-1.75, -0) {};
    \node [style=none] (3) at (-0.75, -1.75) {};
    \node [style=none] (4) at (1.25, -1.5) {};
    \node [style=none] (5) at (1.75, 0.75) {};
    \draw (0) to (5);
    \draw [dashed] (0) to (4);
    \draw (0) to (3);
    \draw [line width=2pt, draw=gray] (0) to (2);
    \draw (0) to (1);
  \end{tikzpicture}
\]
These represent open systems, concrete or abstract; for example a resistor, a
chemical reaction, or a linear transformation. The essential feature, for
openness and for networking, is that the system may have terminals, perhaps of
different `types', each one depicted by a line radiating from the central body.
In the case of a resistor each terminal might represent a wire, for chemical
reactions a chemical species, for linear transformations a variable in the
domain or codomain.  Network diagrams are formed by connecting terminals of
systems to build larger systems.

A network-style diagrammatic language is a collection of network diagrams
together with the stipulation that if we take some of these network diagrams,
and connect terminals of the same type in any way we like, then we form
another diagram in the collection.  The point of this chapter is that hypergraph
categories provide a precise formalisation of network-style diagrammatic
languages.  

\begin{figure}
\[
\begin{aligned}
\begin{tikzpicture}
	\begin{pgfonlayer}{nodelayer}
		\node [thick, circle, draw] (0) at (0, -0) {};
		\node [style=none] (1) at (-0.75, 1.5) {};
		\node [style=none] (2) at (-1.75, -0) {};
		\node [style=none] (3) at (-0.75, -1.75) {};
		\node [style=none] (4) at (1.25, -1.5) {};
		\node [style=none] (5) at (1.75, 0.75) {};
		\node [thick, circle, draw=gray, fill=gray] (6) at (-1, -4.25) {};
		\node [style=none] (7) at (-1, -2.5) {};
		\node [style=none] (8) at (0.25, -5) {};
		\node [style=none] (9) at (-2.25, -4.5) {};
		\node [style=none] (10) at (0.5, -2.75) {};
		\node [thick, circle, draw=gray, fill=gray] (11) at (3.5, -3.75) {};
		\node [style=none] (12) at (3.5, -2.25) {};
		\node [style=none] (13) at (2, -4) {};
		\node [style=none] (14) at (4.25, -5) {};
		\node [thick, circle, draw, fill=black] (15) at (4, -0) {};
		\node [style=none] (16) at (2.5, 0.75) {};
		\node [style=none] (17) at (2.5, -0) {};
		\node [style=none] (18) at (3.75, -1.75) {};
		\node [style=none] (19) at (2, -2.25) {};
	\end{pgfonlayer}
	\begin{pgfonlayer}{edgelayer}
		\draw (0) to (5);
		\draw [dashed] (0) to (4);
		\draw (0) to (3);
		\draw [line width=2pt, draw=gray] (0) to (2);
		\draw (0) to (1);
		\draw (6) to (7);
		\draw [line width=2pt, draw=gray] (6) to (9);
		\draw [dashed] (6) to (10);
		\draw [dotted] (6) to (8);
		\draw [dotted] (11) to (13);
		\draw [line width=2pt, draw=gray] (11) to (12);
		\draw (11) to (14);
		\draw [line width=2pt, draw=gray] (15) to (18);
		\draw (15) to (17);
		\draw (15) to (16);
		\draw [dashed] (11) to (19);
	\end{pgfonlayer}
\end{tikzpicture}
\end{aligned}
\qquad
\Rightarrow
\qquad
\begin{aligned}
\begin{tikzpicture}
	\begin{pgfonlayer}{nodelayer}
		\node [draw, circle, thick] (0) at (0, -0) {};
		\node [style=none] (1) at (-0.75, 1.5) {};
		\node [style=none] (2) at (-1.75, -0) {};
		\node [style=none] (3) at (1.25, -0.25) {};
		\node [draw=gray, circle, thick, fill=gray] (4) at (-0.25, -3) {};
		\node [style=none] (5) at (1.25, -3.5) {};
		\node [style=none] (6) at (-1.5, -3.25) {};
		\node [draw=gray, circle, thick, fill=gray] (7) at (2.75, -2.25) {};
		\node [style=none] (8) at (3.5, -3.5) {};
		\node [draw, circle, thick, fill=black] (9) at (2.5, -0.5) {};
		\node [style=none] (10) at (1, -1.75) {};
	\end{pgfonlayer}
	\begin{pgfonlayer}{edgelayer}
		\draw (0) to (3);
		\draw [line width=2pt, draw=gray] (0) to (2);
		\draw (0) to (1);
		\draw [line width=2pt, draw=gray] (4) to (6);
		\draw [dotted] (4) to (7);
		\draw (7) to (8);
		\draw [dashed] (0) to (10);
		\draw [dashed] (4) to (10);
		\draw [dashed] (7) to (10);
		\draw [bend right, looseness=1.25] (9) to (3);
		\draw [bend left, looseness=1.25] (9) to (3);
		\draw [line width=2pt, draw=gray] (9) to (7);
		\draw (0) to (4);
	\end{pgfonlayer}
\end{tikzpicture}
\end{aligned}
\]
\caption{Interconnection of network diagrams. Note that we only connect
terminals of the same type, but we can connect as many as we like.}
\end{figure}

In jargon, a hypergraph category is a symmetric monoidal category in
which every object is equipped with a special commutative Frobenius monoid in a
way compatible with the monoidal product. We will walk through these terms in
detail, illustrating them with examples and a few theorems. 

The key data comprising a hypergraph category are its objects, morphisms,
composition rule, monoidal product, and Frobenius maps. Each of these model a
feature of network diagrams and their interconnection. The objects model the
terminal types, while the morphisms model the network diagrams themselves. The
composition, monoidal product, and Frobenius maps model different aspects of
interconnection: composition models the interconnection of two terminals of the
same type, the monoidal product models the network formed by taking two networks
without interconnecting any terminals, while the Frobenius maps model
multi-terminal interconnection.

These Frobenius maps are the distinguishing feature of hypergraph categories as
compared to other structured monoidal categories, and are crucial for
formalising the intuitive concept of network languages detailed above. In the
case of electric circuits the Frobenius maps model the `branching' of wires; in
the case when diagrams simply model an abstract system of equations and
terminals variables in these equations, the Frobenius maps allow variables to be
shared between many systems of equations.

Examining these correspondences, it is worthwhile to ask whether hypergraph
categories permit too much structure to be specified, given that the
interconnection rule is now divided into three different aspects, and features
such as domains and codomains of network diagrams, rather than just a collection
of terminals, exist. The answer is given by examining the additional coherence
laws that these data must obey. For example, in the case of the domain and
codomain, we shall see that hypergraph categories are all compact closed
categories, and so there is ultimately only a formal distinction between domain
and codomain objects. One way to think of these data is as scaffolding. We could
compare it to the use of matrices and bases to provide language for talking
about linear transformations and vector spaces.  They are not part of the target
structure, but nonetheless useful paraphenalia for constructing it.

\begin{figure}
  \begin{center}
  \begin{tabular}{c|c}
    Networks & Hypergraph categories \\
    \hline 
    list of terminal types & object \\
    network diagram & morphism \\
    series connection & composition \\
    juxtaposition & monoidal product \\
    branching & Frobenius maps
  \end{tabular}
  \end{center}
  \caption{Corresponding features of networks and hypergraph categories.}
\end{figure}

Network languages are not only syntactic entities: as befitting the descriptor
`language', they typically have some associated semantics. Circuits diagrams, for
instance, not only depict wire circuits that may be constructed, they also
represent the electrical behaviour of that circuit. Such semantics considers the
circuits 
\[
  \begin{aligned}
  \begin{tikzpicture}[circuit ee IEC, set resistor graphic=var resistor IEC graphic]
    \node[contact] (I1) at (0,0) {};
    \node[circle, minimum width = 3pt, inner sep = 0pt, fill=black] (int) at (3,0) {};
    \node[contact] (O1) at (6,0) {};
    \draw (I1) 	to [resistor] node [label={[label distance=3pt]90:{$1 \Omega$}}] {} (int)
    to [resistor] node [label={[label distance=3pt]90:{$1 \Omega$}}] {} (O1);
  \end{tikzpicture}
  \end{aligned}
  \qquad
  \mbox{and}
  \qquad
  \begin{aligned}
  \begin{tikzpicture}[circuit ee IEC, set resistor graphic=var resistor IEC graphic]
    \node[contact] (I1) at (0,0) {};
    \node[contact] (O1) at (3,0) {};
    \draw (I1) 	to [resistor] node [label={[label distance=3pt]90:{$2 \Omega$}}]
    {} (O1);
  \end{tikzpicture}
  \end{aligned}
\]
the same, even though as `syntactic' diagrams they are distinct. A cornerstone
of the utility of the hypergraph formalism is the ability to also realise the
semantics of these diagrams as morphisms of another hypergraph category. This
`semantic' hypergraph category, as a hypergraph category, still permits the rich
`networking' interconnection structure, and a so-called hypergraph functor
implies that the syntactic category provides a sound framework for depicting
these morphisms. Network languages syntactically are often `free' hypergraph
categories, and much of the interesting structure lies in their functors to
their semantic hypergraph categories.


\section{Symmetric monoidal categories}
Suppose we have some tiles with inputs and outputs of various types like so:
\[
    \tikzset{every path/.style={line width=1.1pt}}
  \begin{tikzpicture}
	\begin{pgfonlayer}{nodelayer}
		\node [style=none] (0) at (-0.25, 0.375) {};
		\node [style=none] (1) at (0.5, 0.375) {};
		\node [style=none] (2) at (-0.25, -1.375) {};
		\node [style=none] (3) at (0.5, -1.375) {};
		\node [style=none] (4) at (0.5, 0.25) {};
		\node [style=none] (5) at (0.5, -1.25) {};
		\node [style=none] (6) at (1.25, 0.25) {};
		\node [style=none] (7) at (1.25, -1.25) {};
		\node [style=none] (8) at (0.125, -0.5) {$f$};
		\node [style=none] (9) at (1.5, 0.25) {$Y_1$};
		\node [style=none] (10) at (1.5, -1.25) {$Y_m$};
		\node [style=none] (11) at (1.25, -0.25) {};
		\node [style=none] (12) at (1.5, -0.25) {$Y_2$};
		\node [style=none] (13) at (0.5, -0.25) {};
		\node [style=none] (14) at (1, -0.75) {$\vdots$};
		\node [style=none] (15) at (-1, -1.25) {};
		\node [style=none] (16) at (-0.25, -1.25) {};
		\node [style=none] (17) at (-0.75, -0.75) {$\vdots$};
		\node [style=none] (18) at (-1.25, -1.25) {$X_n$};
		\node [style=none] (19) at (-0.25, -0.25) {};
		\node [style=none] (20) at (-1.25, 0.25) {$X_1$};
		\node [style=none] (21) at (-1, 0.25) {};
		\node [style=none] (22) at (-0.25, 0.25) {};
		\node [style=none] (23) at (-1, -0.25) {};
		\node [style=none] (24) at (-1.25, -0.25) {$X_2$};
	\end{pgfonlayer}
	\begin{pgfonlayer}{edgelayer}
		\draw (0.center) to (1.center);
		\draw (1.center) to (3.center);
		\draw (3.center) to (2.center);
		\draw (2.center) to (0.center);
		\draw (4.center) to (6.center);
		\draw (5.center) to (7.center);
		\draw (13.center) to (11.center);
		\draw (22.center) to (21.center);
		\draw (16.center) to (15.center);
		\draw (19.center) to (23.center);
	\end{pgfonlayer}
\end{tikzpicture}
\]
These tiles may vary in height and width. We can place these tiles above and
below each other, and to the left and right, so long as the inputs on the right
tile match the outputs on the left. Suppose also that some arrangements of tiles
are equal to other arrangements of tiles. How do we formalise this structure
algebraically? The theory of monoidal categories provides an answer. 

Hypergraph categories are first monoidal categories, indeed symmetric monoidal
categories.  Monoidal categories are categories with two notions of composition:
ordinary categorical composition and monoidal composition, with the monoidal
composition only associative and unital up to natural isomorphism. They are the
algebra of processes that may occur simultaneously as well as sequentially.
First defined by B\'enabou and Mac Lane in the 1960s \cite{Ben63, Mac63}, their
theory and their links with graphical representation are well explored
\cite{JS91, Sel11}. We bootstrap on this, using monoidal categories to define
hypergraph categories, and so immediately arriving at an understanding of how
hypergraph categories formalise our network languages. 

Moreover, symmetric monoidal functors play a key role in our framework for
defining and working with hypergraph categories: decorated cospans and
corelations constructions. For this reason we provide, for quick reference, a
definition of symmetric monoidal categories.


\subsection{Monoidal categories}
A \define{monoidal category} $(\c, \ot)$ consists of a category $\c$, a
functor $\ot: \c \times \c \to \c$, a distinguished object $I$, and natural
isomorphisms $\a_{A,B,C}: (A \ot B) \ot C \to A \ot (B \ot C)$,
$\rho_A: A \ot I  \to A$, and $\lambda_A: I \ot A \to A$ such that for all
$A,B,C,D$ in $\mc C$ the following two diagrams commute: 
\[
  \xymatrixcolsep{3pc}
  \xymatrix{
    \big((A \ot B) \ot C\big) \ot D \ar[d]_{\a_{A,B,C}\ot\idn_D} \ar[rr]^{\a_{(A\ot B),C,D}} 
    &&(A \ot B) \ot (C \ot D) \ar[d]^{\a_{A,B,(C\ot D)}} \\
    \big(A \ot (B\ot C)\big) \ot D \ar[r]_{\a_{A,(B\ot C),D}} 
    & A\ot\big((B \ot C)\ot D\big)\ar[r]_{\idn_A \ot \a_{B,C,D}}
    &A \ot \big(B \ot (C \ot D)\big)
  }
\]
\[
  \xymatrix{
    (A\ot I)\ot B  \ar[rr]^{\a_{A,I,B}} \ar[dr]_{\rho_{A}\ot \idn_B} && A \ot (I \ot B) \ar[dl]^{\idn_A\ot \lambda_B}\\
    & A \ot B \\
  }
\]
We call $\ot$ the \define{monoidal product}, $I$ the \define{monoidal unit},
$\alpha$ the \define{associator}, $\rho$ and $\lambda$ the \define{right} and
\define{left unitor} respectively. The associator and unitors are known
collectively as the \define{coherence maps}.

By Mac Lane's coherence theorem, these two axioms are equivalent to requiring
that `all formal diagrams'---that is, all diagrams in which the morphism are
built from identity morphisms and the coherence maps using composition and the
monoidal product---commute. Consequently, between any two products of the same
ordered list of objects up to instances of the monoidal unit, such as $((A \ot
I) \ot B) \ot C$ and $A \ot ((B \ot C) \ot (I \ot I))$, there is a unique
so-called \define{canonical} map. See Mac Lane \cite[Corollary of Theorem
VII.2.1]{Mac98} for a precise statement and proof.

A \define{lax monoidal functor} $(F, \varphi): (\c,\otimes) \to (\c',\boxtimes)$
between monoidal categories consists of a functor $F: \c \to \c'$, and natural
transformations $\varphi_{A,B}: FA \boxtimes FB \to F(A \ot B)$ and $\varphi_1:
1_{\c'} \to F1_{\c}$, such that for all $A,B,C \in \c$ the three diagrams
\[
  \xymatrixcolsep{4pc}
  \xymatrix{
    (FA \ot FB) \ot FC \ar[d]_{\a_{FA,FB,FC}} \ar[r]^{\varphi_{A,B} \ot \idn_{FC}} &
    F(A \ot B) \ot FC \ar[r]^{\varphi_{A\ot B,C}} & F((A \ot B) \ot C) \ar[d]^{F\a_{A,B,C}}\\
    FA \ot (FB \ot FC) \ar[r]_{\idn_{FA} \ot \varphi_{B,C}} & FA \ot F(B \ot C)
    \ar[r]_{\varphi_{A,B\ot C}} & F(A \ot (B \ot C))
  }
\]
\[
  \xymatrixcolsep{3pc}
  \xymatrixrowsep{3pc}
  \xymatrix{
    F(A) \ot I' \ar[d]_{\idn \ot \varphi_1} \ar[r]^{\rho} & F(A) \\
    F(A) \ot F(I) \ar[r]_{\varphi_{A,I}} & F(A \ot I) \ar[u]_{F\rho} 
  }
  \qquad
  \xymatrix{
    I' \ot F(A) \ar[d]_{\varphi_1 \ot \idn} \ar[r]^{\lambda} & F(A) \\
    F(I) \ot F(A) \ar[r]_{\varphi_{I,A}} & F(I \ot A) \ar[u]_{F\lambda} 
  }
\]
commute. We further say a monoidal functor is a \define{strong monoidal functor}
if the $\varphi$ are isomorphisms, and a \define{strict monoidal functor} if the
$\varphi$ are identities. 

A \define{monoidal natural transformation} $\theta: (F,\varphi) \Rightarrow
(G,\gamma)$ between two monoidal functors $F$ and $G$ is a natural
transformation $\theta: F \Rightarrow G$ such that
\[
  \begin{aligned}
    \xymatrix{
      F1_{\c} \ar[rr]^{\theta_I}&& G1_{\c} \\
      & 1_{\c'} \ar[ul]^{\varphi_1} \ar[ur]_{\gamma_1}
    } 
  \end{aligned} 
  \qquad 
  \mbox{and}
  \qquad
  \begin{aligned}
    \xymatrixcolsep{3pc}
    \xymatrixrowsep{3pc}
    \xymatrix{
      FA \boxtimes FB \ar[r]^{\theta_A \ot \theta_B} \ar[d]_{\varphi_{A,B}} 
      & GA \boxtimes GB \ar[d]^{\gamma_{A,B}}\\
      F(A \ot B) \ar[r]_{\theta_{A\ot B}} & G(A \ot B)
    }
  \end{aligned} 
\]
commute for all objects $A,B$.

Two monoidal categories $\mc C, \mc D$ are \define{monoidally equivalent} if
there exist strong monoidal functors $F\maps \mc C \to \mc D$ and $G\maps \mc D
\to \mc C$ such that the composites $FG$ and $GF$ are monoidally naturally
isomorphic to the identity functors. (Note that identity functors are
immediately strict monoidal functors.)

\subsection{String diagrams}
A \define{strict monoidal category} category is a monoidal category in which the
associators and unitors are all identity maps. In this case then any two objects
that can be related by associators and unitors are equal, and so we may write
objects without parentheses and units without ambiguity. An equivalent statement
of Mac Lane's coherence theorem is that every monoidal category is monoidally
equivalent to strict monoidal category. 

Yet another equivalent statement of the coherence theorem is the existence of a
graphical calculus for monoidal categories. As discussed above, monoidal
categories figure strongly in our current investigations precisely because of
this. We leave the details to discussions elsewhere. The main point is that we
shall be free to assume our monoidal categories are strict, writing $X_1 \otimes
X_2 \otimes \dots \otimes X_n$ for objects in $(\mathcal C,\otimes)$ without a
care for parentheses. We then depict a morphism $f\maps X_1 \otimes X_2 \otimes
\dots \otimes X_n \to Y_1 \otimes Y_2 \otimes \dots \otimes Y_n$ with the
diagram:
\[
  f \quad = \quad
  \begin{aligned}
    \tikzset{every path/.style={line width=1.1pt}}
  \begin{tikzpicture}
	\begin{pgfonlayer}{nodelayer}
		\node [style=none] (0) at (-0.25, 0.375) {};
		\node [style=none] (1) at (0.5, 0.375) {};
		\node [style=none] (2) at (-0.25, -1.375) {};
		\node [style=none] (3) at (0.5, -1.375) {};
		\node [style=none] (4) at (0.5, 0.25) {};
		\node [style=none] (5) at (0.5, -1.25) {};
		\node [style=none] (6) at (1.25, 0.25) {};
		\node [style=none] (7) at (1.25, -1.25) {};
		\node [style=none] (8) at (0.125, -0.5) {$f$};
		\node [style=none] (9) at (1.5, 0.25) {$Y_1$};
		\node [style=none] (10) at (1.5, -1.25) {$Y_m$};
		\node [style=none] (11) at (1.25, -0.25) {};
		\node [style=none] (12) at (1.5, -0.25) {$Y_2$};
		\node [style=none] (13) at (0.5, -0.25) {};
		\node [style=none] (14) at (1, -0.75) {$\vdots$};
		\node [style=none] (15) at (-1, -1.25) {};
		\node [style=none] (16) at (-0.25, -1.25) {};
		\node [style=none] (17) at (-0.75, -0.75) {$\vdots$};
		\node [style=none] (18) at (-1.25, -1.25) {$X_n$};
		\node [style=none] (19) at (-0.25, -0.25) {};
		\node [style=none] (20) at (-1.25, 0.25) {$X_1$};
		\node [style=none] (21) at (-1, 0.25) {};
		\node [style=none] (22) at (-0.25, 0.25) {};
		\node [style=none] (23) at (-1, -0.25) {};
		\node [style=none] (24) at (-1.25, -0.25) {$X_2$};
	\end{pgfonlayer}
	\begin{pgfonlayer}{edgelayer}
		\draw (0.center) to (1.center);
		\draw (1.center) to (3.center);
		\draw (3.center) to (2.center);
		\draw (2.center) to (0.center);
		\draw (4.center) to (6.center);
		\draw (5.center) to (7.center);
		\draw (13.center) to (11.center);
		\draw (22.center) to (21.center);
		\draw (16.center) to (15.center);
		\draw (19.center) to (23.center);
	\end{pgfonlayer}
\end{tikzpicture}.
\end{aligned}
\]
Identity morphisms are depicted by `wires':
\[
  \idn_X \quad = \quad
  \begin{aligned}
    \tikzset{every path/.style={line width=1.1pt}}
\begin{tikzpicture}
	\begin{pgfonlayer}{nodelayer}
		\node [style=none] (0) at (1.25, 0.25) {};
		\node [style=none] (1) at (1.5, 0.25) {$X$};
		\node [style=none] (2) at (-1.25, 0.25) {$X$};
		\node [style=none] (3) at (-1, 0.25) {};
	\end{pgfonlayer}
	\begin{pgfonlayer}{edgelayer}
		\draw (3.center) to (0.center);
	\end{pgfonlayer}
\end{tikzpicture}
\end{aligned}
\]
and the monoidal unit is not depicted at all:
\[
\idn_I\quad = \quad
  \begin{aligned}
    \tikzset{every path/.style={line width=1.1pt}}
\begin{tikzpicture}
		\node [style=none] (1) at (1.5, 0.25) {};
		\node [style=none] (2) at (-1.25, 0.25) {};
\end{tikzpicture}
\end{aligned}
\]
Composition of morphisms is depicted by connecting the relevant `wires':
\[
    \tikzset{every path/.style={line width=1.1pt}}
  \begin{aligned}
    \begin{tikzpicture}
	\begin{pgfonlayer}{nodelayer}
		\node [style=none] (0) at (0.25, -0) {$Y_1$};
		\node [style=none] (1) at (0.5, -0) {};
		\node [style=none] (2) at (2.75, -0.75) {};
		\node [style=none] (3) at (3, -0.75) {$Y_2$};
		\node [style=none] (4) at (0.5, -0.75) {};
		\node [style=none] (5) at (2, -0.365) {};
		\node [style=none] (6) at (2.75, 0.25) {};
		\node [style=none] (7) at (3, 0.25) {$Z_1$};
		\node [style=none] (8) at (1.25, 0.375) {};
		\node [style=none] (9) at (1.25, -0) {};
		\node [style=none] (10) at (1.25, -0.365) {};
		\node [style=none] (11) at (2, 0.25) {};
		\node [style=none] (12) at (2, 0.375) {};
		\node [style=none] (13) at (1.625, -0) {$g$};
		\node [style=none] (14) at (2, -0.25) {};
		\node [style=none] (15) at (2.75, -0.25) {};
		\node [style=none] (16) at (3, -0.25) {$Z_2$};
		\node [style=none] (17) at (0.25, -0.75) {$Y_2$};
	\end{pgfonlayer}
	\begin{pgfonlayer}{edgelayer}
		\draw (4.center) to (2.center);
		\draw (8.center) to (12.center);
		\draw (5.center) to (10.center);
		\draw (11.center) to (6.center);
		\draw (12.center) to (5.center);
		\draw (10.center) to (8.center);
		\draw (14.center) to (15.center);
		\draw (1.center) to (9.center);
	\end{pgfonlayer}
\end{tikzpicture}
\end{aligned}
  \circ
  \begin{aligned}
  \begin{tikzpicture}
	\begin{pgfonlayer}{nodelayer}
		\node [style=none] (0) at (-0.25, 0.375) {};
		\node [style=none] (1) at (0.5, 0.375) {};
		\node [style=none] (2) at (-0.25, -.875) {};
		\node [style=none] (3) at (0.5, -.875) {};
		\node [style=none] (4) at (0.5, 0.125) {};
		\node [style=none] (5) at (1.25, 0.125) {};
		\node [style=none] (6) at (0.125, -0.25) {$f$};
		\node [style=none] (7) at (1.5, 0.125) {$Y_1$};
		\node [style=none] (8) at (1.25, -0.625) {};
		\node [style=none] (9) at (1.5, -0.625) {$Y_2$};
		\node [style=none] (10) at (0.5, -0.625) {};
		\node [style=none] (11) at (-0.25, -0.75) {};
		\node [style=none] (12) at (-1.25, -0.75) {$X_3$};
		\node [style=none] (13) at (-0.25, -0.25) {};
		\node [style=none] (14) at (-1.25, 0.25) {$X_1$};
		\node [style=none] (15) at (-1, 0.25) {};
		\node [style=none] (16) at (-0.25, 0.25) {};
		\node [style=none] (17) at (-1, -0.25) {};
		\node [style=none] (18) at (-1.25, -0.25) {$X_2$};
		\node [style=none] (19) at (-1, -0.75) {};
	\end{pgfonlayer}
	\begin{pgfonlayer}{edgelayer}
		\draw (0.center) to (1.center);
		\draw (1.center) to (3.center);
		\draw (3.center) to (2.center);
		\draw (2.center) to (0.center);
		\draw (4.center) to (5.center);
		\draw (10.center) to (8.center);
		\draw (16.center) to (15.center);
		\draw (13.center) to (17.center);
		\draw (11.center) to (19.center);
	\end{pgfonlayer}
\end{tikzpicture}
\end{aligned}
\quad = \quad
\begin{aligned}
\begin{tikzpicture}
	\begin{pgfonlayer}{nodelayer}
		\node [style=none] (0) at (-0.25, 0.375) {};
		\node [style=none] (1) at (0.5, 0.375) {};
		\node [style=none] (2) at (-0.25, -0.875) {};
		\node [style=none] (3) at (0.5, -0.875) {};
		\node [style=none] (4) at (0.5, 0.125) {};
		\node [style=none] (5) at (0.125, -0.25) {$f$};
		\node [style=none] (6) at (2.75, -0.625) {};
		\node [style=none] (7) at (3, -0.625) {$Y_2$};
		\node [style=none] (8) at (0.5, -0.625) {};
		\node [style=none] (9) at (-0.25, -0.75) {};
		\node [style=none] (10) at (-1.25, -0.75) {$X_3$};
		\node [style=none] (11) at (-0.25, -0.25) {};
		\node [style=none] (12) at (-1.25, 0.25) {$X_1$};
		\node [style=none] (13) at (-1, 0.25) {};
		\node [style=none] (14) at (-0.25, 0.25) {};
		\node [style=none] (15) at (-1, -0.25) {};
		\node [style=none] (16) at (-1.25, -0.25) {$X_2$};
		\node [style=none] (17) at (-1, -0.75) {};
		\node [style=none] (18) at (2, -0.25) {};
		\node [style=none] (19) at (2.75, 0.375) {};
		\node [style=none] (20) at (3, 0.375) {$Z_1$};
		\node [style=none] (21) at (1.25, 0.5) {};
		\node [style=none] (22) at (1.25, 0.125) {};
		\node [style=none] (23) at (1.25, -0.25) {};
		\node [style=none] (24) at (2, 0.375) {};
		\node [style=none] (25) at (2, 0.5) {};
		\node [style=none] (26) at (1.625, 0.125) {$g$};
		\node [style=none] (27) at (2, -0.125) {};
		\node [style=none] (28) at (2.75, -0.125) {};
		\node [style=none] (29) at (3, -0.125) {$Z_2$};
	\end{pgfonlayer}
	\begin{pgfonlayer}{edgelayer}
		\draw (0.center) to (1.center);
		\draw (1.center) to (3.center);
		\draw (3.center) to (2.center);
		\draw (2.center) to (0.center);
		\draw (8.center) to (6.center);
		\draw (14.center) to (13.center);
		\draw (11.center) to (15.center);
		\draw (9.center) to (17.center);
		\draw (21.center) to (25.center);
		\draw (18.center) to (23.center);
		\draw (24.center) to (19.center);
		\draw (25.center) to (18.center);
		\draw (23.center) to (21.center);
		\draw (27.center) to (28.center);
		\draw (4.center) to (22.center);
	\end{pgfonlayer}
\end{tikzpicture}
\end{aligned}
\]
while monoidal composition is just juxtaposition:
\[
    \tikzset{every path/.style={line width=1.1pt}}
  \begin{aligned}
    \begin{tikzpicture}
	\begin{pgfonlayer}{nodelayer}
		\node [style=none] (0) at (-0.25, 0.375) {};
		\node [style=none] (1) at (0.5, 0.375) {};
		\node [style=none] (2) at (-0.25, -0.375) {};
		\node [style=none] (3) at (0.5, -0.375) {};
		\node [style=none] (4) at (0.5, -0) {};
		\node [style=none] (5) at (1.25, -0) {};
		\node [style=none] (6) at (0.125, -0) {$h$};
		\node [style=none] (7) at (1.5, -0) {$Y_1$};
		\node [style=none] (8) at (-0.25, -0.25) {};
		\node [style=none] (9) at (-1.25, 0.25) {$X_1$};
		\node [style=none] (10) at (-1, 0.25) {};
		\node [style=none] (11) at (-0.25, 0.25) {};
		\node [style=none] (12) at (-1, -0.25) {};
		\node [style=none] (13) at (-1.25, -0.25) {$X_2$};
	\end{pgfonlayer}
	\begin{pgfonlayer}{edgelayer}
		\draw (0.center) to (1.center);
		\draw (3.center) to (2.center);
		\draw (2.center) to (0.center);
		\draw (4.center) to (5.center);
		\draw (11.center) to (10.center);
		\draw (8.center) to (12.center);
		\draw (1.center) to (3.center);
	\end{pgfonlayer}
\end{tikzpicture}
  \end{aligned}
  \ot \quad
  \begin{aligned}
    \begin{tikzpicture}
	\begin{pgfonlayer}{nodelayer}
		\node [style=none] (0) at (1.5, -0.75) {$Y_2$};
		\node [style=none] (1) at (-0.25, -1.375) {};
		\node [style=none] (2) at (1.25, -0.75) {};
		\node [style=none] (3) at (1.5, -1.25) {$Y_3$};
		\node [style=none] (4) at (1.25, -1.25) {};
		\node [style=none] (5) at (0.125, -1) {$k$};
		\node [style=none] (6) at (0.5, -0.75) {};
		\node [style=none] (7) at (0.5, -1.375) {};
		\node [style=none] (8) at (-0.25, -0.625) {};
		\node [style=none] (9) at (0.5, -1.25) {};
		\node [style=none] (10) at (0.5, -0.625) {};
	\end{pgfonlayer}
	\begin{pgfonlayer}{edgelayer}
		\draw (8.center) to (10.center);
		\draw (6.center) to (2.center);
		\draw (9.center) to (4.center);
		\draw (10.center) to (7.center);
		\draw (7.center) to (1.center);
		\draw (8.center) to (1.center);
	\end{pgfonlayer}
\end{tikzpicture}
  \end{aligned}
  \quad = \quad
  \begin{aligned}
    \begin{tikzpicture}
	\begin{pgfonlayer}{nodelayer}
		\node [style=none] (0) at (-0.25, 0.375) {};
		\node [style=none] (1) at (0.5, 0.375) {};
		\node [style=none] (2) at (-0.25, -0.375) {};
		\node [style=none] (3) at (0.5, -0.375) {};
		\node [style=none] (4) at (0.5, -0) {};
		\node [style=none] (5) at (1.25, -0) {};
		\node [style=none] (6) at (0.125, -0) {$h$};
		\node [style=none] (7) at (1.5, -0) {$Y_1$};
		\node [style=none] (8) at (-0.25, -0.25) {};
		\node [style=none] (9) at (-1.25, 0.25) {$X_1$};
		\node [style=none] (10) at (-1, 0.25) {};
		\node [style=none] (11) at (-0.25, 0.25) {};
		\node [style=none] (12) at (-1, -0.25) {};
		\node [style=none] (13) at (-1.25, -0.25) {$X_2$};
		\node [style=none] (14) at (0.5, -0.75) {};
		\node [style=none] (15) at (1.25, -0.75) {};
		\node [style=none] (16) at (0.125, -1) {$k$};
		\node [style=none] (17) at (0.5, -0.625) {};
		\node [style=none] (18) at (-0.25, -1.375) {};
		\node [style=none] (19) at (1.5, -1.25) {$Y_3$};
		\node [style=none] (20) at (0.5, -1.25) {};
		\node [style=none] (21) at (-0.25, -0.625) {};
		\node [style=none] (22) at (1.25, -1.25) {};
		\node [style=none] (23) at (1.5, -0.75) {$Y_2$};
		\node [style=none] (24) at (0.5, -1.375) {};
	\end{pgfonlayer}
	\begin{pgfonlayer}{edgelayer}
		\draw (0.center) to (1.center);
		\draw (3.center) to (2.center);
		\draw (2.center) to (0.center);
		\draw (4.center) to (5.center);
		\draw (11.center) to (10.center);
		\draw (8.center) to (12.center);
		\draw (21.center) to (17.center);
		\draw (14.center) to (15.center);
		\draw (20.center) to (22.center);
		\draw (1.center) to (3.center);
		\draw (21.center) to (18.center);
		\draw (17.center) to (24.center);
		\draw (24.center) to (18.center);
	\end{pgfonlayer}
\end{tikzpicture}
  \end{aligned}
\]

Only the `topology' of the diagrams matters: if two diagrams with the same
domain and codomain are equivalent up to isotopy, they represent the same
morphism.
On the other hand, two algebraic expressions might have the same diagrammatic
representation. For example, the equivalent diagrams
\[
    \tikzset{every path/.style={line width=1.1pt}}
\begin{aligned}
\begin{tikzpicture}
	\begin{pgfonlayer}{nodelayer}
		\node [style=none] (0) at (-0.25, 0.375) {};
		\node [style=none] (1) at (0.5, 0.375) {};
		\node [style=none] (2) at (-0.25, -0.875) {};
		\node [style=none] (3) at (0.5, -0.875) {};
		\node [style=none] (4) at (0.5, 0.125) {};
		\node [style=none] (5) at (0.125, -0.25) {$f$};
		\node [style=none] (6) at (2.75, -0.625) {};
		\node [style=none] (7) at (3, -0.625) {$Y_2$};
		\node [style=none] (8) at (0.5, -0.625) {};
		\node [style=none] (9) at (-0.25, -0.75) {};
		\node [style=none] (10) at (-1.25, -0.75) {$X_3$};
		\node [style=none] (11) at (-0.25, -0.25) {};
		\node [style=none] (12) at (-1.25, 0.25) {$X_1$};
		\node [style=none] (13) at (-1, 0.25) {};
		\node [style=none] (14) at (-0.25, 0.25) {};
		\node [style=none] (15) at (-1, -0.25) {};
		\node [style=none] (16) at (-1.25, -0.25) {$X_2$};
		\node [style=none] (17) at (-1, -0.75) {};
		\node [style=none] (18) at (2, -0.25) {};
		\node [style=none] (19) at (2.75, 0.375) {};
		\node [style=none] (20) at (3, 0.375) {$Z_1$};
		\node [style=none] (21) at (1.25, 0.5) {};
		\node [style=none] (22) at (1.25, 0.125) {};
		\node [style=none] (23) at (1.25, -0.25) {};
		\node [style=none] (24) at (2, 0.375) {};
		\node [style=none] (25) at (2, 0.5) {};
		\node [style=none] (26) at (1.625, 0.125) {$g$};
		\node [style=none] (27) at (2, -0.125) {};
		\node [style=none] (28) at (2.75, -0.125) {};
		\node [style=none] (29) at (3, -0.125) {$Z_2$};
		\node [style=none] (30) at (1.25, -1.75) {};
		\node [style=none] (31) at (2, -1.625) {};
		\node [style=none] (32) at (2.75, -1.125) {};
		\node [style=none] (33) at (3, -1.625) {$Y_3$};
		\node [style=none] (34) at (2.75, -1.625) {};
		\node [style=none] (35) at (1.625, -1.375) {$k$};
		\node [style=none] (36) at (3, -1.125) {$Y_2$};
		\node [style=none] (37) at (2, -1.125) {};
		\node [style=none] (38) at (2, -1.75) {};
		\node [style=none] (39) at (2, -1) {};
		\node [style=none] (40) at (1.25, -1) {};
	\end{pgfonlayer}
	\begin{pgfonlayer}{edgelayer}
		\draw (0.center) to (1.center);
		\draw (1.center) to (3.center);
		\draw (3.center) to (2.center);
		\draw (2.center) to (0.center);
		\draw (8.center) to (6.center);
		\draw (14.center) to (13.center);
		\draw (11.center) to (15.center);
		\draw (9.center) to (17.center);
		\draw (21.center) to (25.center);
		\draw (18.center) to (23.center);
		\draw (24.center) to (19.center);
		\draw (25.center) to (18.center);
		\draw (23.center) to (21.center);
		\draw (27.center) to (28.center);
		\draw (4.center) to (22.center);
		\draw (40.center) to (39.center);
		\draw (37.center) to (32.center);
		\draw (31.center) to (34.center);
		\draw (39.center) to (38.center);
		\draw (38.center) to (30.center);
		\draw (40.center) to (30.center);
	\end{pgfonlayer}
\end{tikzpicture}
\end{aligned}
=
\begin{aligned}
  \begin{tikzpicture}
	\begin{pgfonlayer}{nodelayer}
		\node [style=none] (0) at (0, 1) {};
		\node [style=none] (1) at (1, 1) {};
		\node [style=none] (2) at (0, -1.25) {};
		\node [style=none] (3) at (1, -1.25) {};
		\node [style=none] (4) at (1, -0.25) {};
		\node [style=none] (5) at (0.5, -0) {$f$};
		\node [style=none] (6) at (2.75, -0.5) {};
		\node [style=none] (7) at (3, -0.5) {$Y_2$};
		\node [style=none] (8) at (1, -0.75) {};
		\node [style=none] (9) at (0, -0.75) {};
		\node [style=none] (10) at (-1.25, -0.75) {$X_3$};
		\node [style=none] (11) at (0, -0.25) {};
		\node [style=none] (12) at (-1.25, 0.25) {$X_1$};
		\node [style=none] (13) at (-1, 0.25) {};
		\node [style=none] (14) at (0, 0.5) {};
		\node [style=none] (15) at (-1, -0.25) {};
		\node [style=none] (16) at (-1.25, -0.25) {$X_2$};
		\node [style=none] (17) at (-1, -0.75) {};
		\node [style=none] (18) at (2, -0.25) {};
		\node [style=none] (19) at (2.75, 0.375) {};
		\node [style=none] (20) at (3, 0.375) {$Z_1$};
		\node [style=none] (21) at (1.5, 0.25) {};
		\node [style=none] (22) at (1.375, -0) {};
		\node [style=none] (23) at (1.25, -0.25) {};
		\node [style=none] (24) at (2, 0.375) {};
		\node [style=none] (25) at (2, 0.5) {};
		\node [style=none] (26) at (1.75, -0) {$g$};
		\node [style=none] (27) at (2, -0.125) {};
		\node [style=none] (28) at (2.75, -0.125) {};
		\node [style=none] (29) at (3, -0.125) {$Z_2$};
		\node [style=none] (30) at (-0.5, -2.5) {};
		\node [style=none] (31) at (0.25, -1.75) {};
		\node [style=none] (32) at (2.75, -1.5) {};
		\node [style=none] (33) at (3, -1.75) {$Y_3$};
		\node [style=none] (34) at (2.75, -1.75) {};
		\node [style=none] (35) at (-0.125, -1.75) {$k$};
		\node [style=none] (36) at (3, -1.5) {$Y_2$};
		\node [style=none] (37) at (0.25, -1.5) {};
		\node [style=none] (38) at (0.25, -2.25) {};
		\node [style=none] (39) at (0.25, -1.375) {};
		\node [style=none] (40) at (-0.5, -1.375) {};
	\end{pgfonlayer}
	\begin{pgfonlayer}{edgelayer}
		\draw (0.center) to (1.center);
		\draw (1.center) to (3.center);
		\draw (3.center) to (2.center);
		\draw (2.center) to (0.center);
		\draw (8.center) to (6.center);
		\draw (14.center) to (13.center);
		\draw (11.center) to (15.center);
		\draw (9.center) to (17.center);
		\draw (21.center) to (25.center);
		\draw (18.center) to (23.center);
		\draw (24.center) to (19.center);
		\draw (25.center) to (18.center);
		\draw (23.center) to (21.center);
		\draw (27.center) to (28.center);
		\draw (4.center) to (22.center);
		\draw (40.center) to (39.center);
		\draw (37.center) to (32.center);
		\draw (31.center) to (34.center);
		\draw (39.center) to (38.center);
		\draw (38.center) to (30.center);
		\draw (40.center) to (30.center);
	\end{pgfonlayer}
\end{tikzpicture}
\end{aligned}
\]
read as all of the equivalent algebraic expressions 
\[
  ((g \ot \idn_{Y_2}) \ot k) \circ f = 
  (g \ot ((\idn_{Y_2} \ot k)) \circ \rho \circ (f \ot \idn_I) = 
  (g \ot \idn_{Y_2}) \circ f \circ (\idn_{X_1 \ot (X_2 \ot X_3)} \ot k) 
\]
and so on. The coherence theorem says that this does not matter: if two
algebraic expressions have the same diagrammatic representation, then the
algebraic expressions are equal. In more formal language, the graphical calculus
is sound and complete for the axioms of monoidal categories. See Joyal--Street
for details \cite{JS91}.


The coherence theorem thus implies that the graphical calculi goes beyond
visualisations of morphisms: it can provide provide bona-fide proofs of
equalities of morphisms. As a general principle, string diagrams are more
intuitive than the conventional algebraic language for understanding monoidal
categories.

\subsection{Symmetry}
A symmetric braiding in a monoidal category provides the ability to permute
objects or, equivalently, cross wires. We define symmetric monoidal categories
making use of the graphical notation outlined above, but introducing a new,
special symbol $\swap{.04\textwidth}$.

A \define{symmetric monoidal category} is a monoidal category $(\mc C,\ot)$
together with natural isomorphisms 
\[
    \tikzset{every path/.style={line width=1.1pt}}
    \xymatrixrowsep{0pt}
  \xymatrix{
\begin{tikzpicture}
	\begin{pgfonlayer}{nodelayer}
		\node [style=none] (0) at (1, 0.25) {$B$};
		\node [style=none] (1) at (0.5, -0.25) {};
		\node [style=none] (2) at (-1, 0.25) {$A$};
		\node [style=none] (3) at (0.5, 0.25) {};
		\node [style=none] (4) at (-0.5, -0.25) {};
		\node [style=none] (5) at (-1, -0.25) {$B$};
		\node [style=none] (6) at (1, -0.25) {$A$};
		\node [style=none] (7) at (-0.5, 0.25) {};
		\node [style=none] (8) at (-0.75, 0.25) {};
		\node [style=none] (9) at (-0.75, -0.25) {};
		\node [style=none] (10) at (0.75, 0.25) {};
		\node [style=none] (11) at (0.75, -0.25) {};
	\end{pgfonlayer}
	\begin{pgfonlayer}{edgelayer}
		\draw [in=180, out=0, looseness=1.00] (7.center) to (1.center);
		\draw [in=180, out=0, looseness=1.00] (4.center) to (3.center);
		\draw (7.center) to (8.center);
		\draw (4.center) to (9.center);
		\draw (3.center) to (10.center);
		\draw (1.center) to (11.center);
	\end{pgfonlayer}
\end{tikzpicture}
    \\
    \s_{A,B}: A \ot B \to B \ot A
  }
\]
such that
\[
    \tikzset{every path/.style={line width=1.1pt}}
  \begin{aligned}
    \begin{tikzpicture}
	\begin{pgfonlayer}{nodelayer}
		\node [style=none] (0) at (2, -0.25) {$B$};
		\node [style=none] (1) at (0.5, -0.25) {};
		\node [style=none] (2) at (-1, 0.25) {$A$};
		\node [style=none] (3) at (0.5, 0.25) {};
		\node [style=none] (4) at (-0.5, -0.25) {};
		\node [style=none] (5) at (-1, -0.25) {$B$};
		\node [style=none] (6) at (2, 0.25) {$A$};
		\node [style=none] (7) at (-0.5, 0.25) {};
		\node [style=none] (8) at (-0.75, 0.25) {};
		\node [style=none] (9) at (-0.75, -0.25) {};
		\node [style=none] (10) at (1.75, 0.25) {};
		\node [style=none] (11) at (1.75, -0.25) {};
		\node [style=none] (12) at (1.5, 0.25) {};
		\node [style=none] (13) at (1.5, -0.25) {};
	\end{pgfonlayer}
	\begin{pgfonlayer}{edgelayer}
		\draw [in=180, out=0, looseness=1.00] (7.center) to (1.center);
		\draw [in=180, out=0, looseness=1.00] (4.center) to (3.center);
		\draw (7.center) to (8.center);
		\draw (4.center) to (9.center);
		\draw [in=180, out=0, looseness=1.00] (3.center) to (13.center);
		\draw [in=180, out=-3, looseness=1.00] (1.center) to (12.center);
		\draw (13.center) to (11.center);
		\draw (12.center) to (10.center);
	\end{pgfonlayer}
\end{tikzpicture}
  \end{aligned}
\quad = \quad
  \begin{aligned}
\begin{tikzpicture}
	\begin{pgfonlayer}{nodelayer}
		\node [style=none] (0) at (1, -0.25) {$B$};
		\node [style=none] (1) at (-1, 0.25) {$A$};
		\node [style=none] (2) at (0.75, -0.25) {};
		\node [style=none] (3) at (-1, -0.25) {$B$};
		\node [style=none] (4) at (1, 0.25) {$A$};
		\node [style=none] (5) at (0.75, 0.25) {};
		\node [style=none] (6) at (-0.75, 0.25) {};
		\node [style=none] (7) at (-0.75, -0.25) {};
	\end{pgfonlayer}
	\begin{pgfonlayer}{edgelayer}
		\draw (5.center) to (6.center);
		\draw (2.center) to (7.center);
	\end{pgfonlayer}
\end{tikzpicture}
  \end{aligned}
\]
and
\[
    \tikzset{every path/.style={line width=1.1pt}}
  \begin{aligned}
    \begin{tikzpicture}
	\begin{pgfonlayer}{nodelayer}
		\node [style=none] (0) at (2, -0.25) {$C$};
		\node [style=none] (1) at (-1, 0.25) {$A$};
		\node [style=none] (2) at (-0.5, -0.25) {};
		\node [style=none] (3) at (-1, -0.25) {$B$};
		\node [style=none] (4) at (2, 0.25) {$B$};
		\node [style=none] (5) at (-0.5, 0.25) {};
		\node [style=none] (6) at (-0.75, 0.25) {};
		\node [style=none] (7) at (-0.75, -0.25) {};
		\node [style=none] (8) at (1.75, -0.25) {};
		\node [style=none] (9) at (1.75, 0.25) {};
		\node [style=none] (10) at (0.5, 0.25) {};
		\node [style=none] (11) at (1.5, -0.25) {};
		\node [style=none] (12) at (0.5, -0.25) {};
		\node [style=none] (13) at (-1, -0.75) {$C$};
		\node [style=none] (14) at (-0.75, -0.75) {};
		\node [style=none] (15) at (-0.5, -0.75) {};
		\node [style=none] (16) at (0.5, -0.75) {};
		\node [style=none] (17) at (1.5, -0.75) {};
		\node [style=none] (18) at (1.75, -0.75) {};
		\node [style=none] (19) at (2, -0.75) {$A$};
	\end{pgfonlayer}
	\begin{pgfonlayer}{edgelayer}
		\draw (11.center) to (8.center);
		\draw [in=180, out=0, looseness=1.00] (2.center) to (10.center);
		\draw [in=180, out=0, looseness=1.00] (5.center) to (12.center);
		\draw (10.center) to (9.center);
		\draw (14.center) to (15.center);
		\draw (15.center) to (16.center);
		\draw [in=180, out=0, looseness=1.00] (16.center) to (11.center);
		\draw [in=180, out=0, looseness=1.00] (12.center) to (17.center);
		\draw (17.center) to (18.center);
		\draw (6.center) to (5.center);
		\draw (2.center) to (7.center);
	\end{pgfonlayer}
\end{tikzpicture}
  \end{aligned}
\quad = \quad
  \begin{aligned}
\begin{tikzpicture}
	\begin{pgfonlayer}{nodelayer}
		\node [style=none] (0) at (1.35, 0.25) {$B\ot C$};
		\node [style=none] (1) at (0.5, -0.25) {};
		\node [style=none] (2) at (-1, 0.25) {$A$};
		\node [style=none] (3) at (0.5, 0.25) {};
		\node [style=none] (4) at (-0.5, -0.25) {};
		\node [style=none] (5) at (-1.35, -0.25) {$B\ot C$};
		\node [style=none] (6) at (1, -0.25) {$A$};
		\node [style=none] (7) at (-0.5, 0.25) {};
		\node [style=none] (8) at (-0.75, 0.25) {};
		\node [style=none] (9) at (-0.75, -0.25) {};
		\node [style=none] (10) at (0.75, 0.25) {};
		\node [style=none] (11) at (0.75, -0.25) {};
	\end{pgfonlayer}
	\begin{pgfonlayer}{edgelayer}
		\draw [in=180, out=0, looseness=1.00] (7.center) to (1.center);
		\draw [in=180, out=0, looseness=1.00] (4.center) to (3.center);
		\draw (7.center) to (8.center);
		\draw (4.center) to (9.center);
		\draw (3.center) to (10.center);
		\draw (1.center) to (11.center);
	\end{pgfonlayer}
\end{tikzpicture}
  \end{aligned}
\]
for all $A,B,C$ in $\mc C$.  We call $\s$ the \define{braiding}. We will also
talk, somewhat incidentally, of braided monoidal categories in the next
chapter; a \define{braided monoidal category} is a monoidal category with a
braiding that only obeys the latter axiom.

A \define{(lax/strong) symmetric monoidal functor} is a (lax/strong) monoidal
functor that further obeys
\[
\xymatrixcolsep{3pc}
\xymatrixrowsep{3pc}
\xymatrix{
FA \ot FB \ar[r]^{\varphi_{A,B}} \ar[d]_{\s'_{FA,FB}} & F(A \ot B)\ar[d]^{F\s_{A,B}}\\
FB \ot FA \ar[r]_{\varphi_{B,A}} & F(B \ot A)
}
\]
Morphisms between symmetric monoidal functors are simply monoidal natural
transformations. Thus two symmetric monoidal categories are \define{symmetric
monoidally equivalent} if they are monoidally equivalent by strong
\emph{symmetric} monoidal functors. If our categories are merely braided, we
refer to these functors as \define{braided monoidal functors}.

The coherence theorem for symmetric monoidal categories, with respect to string
diagrams, states that two morphisms in a symmetric monoidal category are equal
according to the axioms of symmetric monoidal categories if and only if their
diagrams are equal up to homotopy equivalence and applications of the defining
graphical identities above. See Joyal--Street \cite[Theorem 2.3]{JoyStr} for
more precision and details.

\section{Hypergraph categories}
Just as symmetric monoidal categories equip monoidal categories with precisely
enough extra structure to model crossing of strings in the graphical calculus,
hypergraph categories equip symmetric monoidal categories with precisely enough
extra structure to model multi-input multi-output interconnections of strings of
the same type. For this, we require each object to be equipped with a so-called
special commutative Frobenius monoid, which provides chosen maps to model this
interaction. These have a coherence result, known as the `spider theorem', that
says exactly how we use the maps to describe the connection of strings does not
matter: all that matters is that the strings are connected. 

\subsection{Frobenius monoids}
A Frobenius monoid comprises a monoid and comonoid on the same object that
interact according to the so-called Frobenius law.
\begin{definition}
  A \define{special commutative Frobenius monoid} $(X,\mu,\eta,\delta,\epsilon)$
  in a symmetric monoidal category $(\mathcal C, \otimes)$ is an object $X$ of
  $\mathcal C$ together with maps 
\[
  \xymatrixrowsep{1pt}
  \xymatrixcolsep{20pt}
  \xymatrix{
    \mult{.075\textwidth} & & \unit{.075\textwidth} & & 
    \comult{.075\textwidth} & & \counit{.075\textwidth} \\
    \mu\maps X\otimes X \to X & & \eta\maps I \to X & & 
    \delta\maps X\to X \otimes X & & \epsilon\maps X \to I
  }
\]
obeying the commutative monoid axioms
\[
  \xymatrixrowsep{1pt}
  \xymatrixcolsep{25pt}
  \xymatrix{
    \assocl{.1\textwidth} = \assocr{.1\textwidth} & \unitl{.1\textwidth} =
    \idone{.1\textwidth} & \commute{.1\textwidth} = \mult{.07\textwidth} \\
    \textrm{(associativity)} & \textrm{(unitality)} & \textrm{(commutativity)}
  }
\]
the cocommutative comonoid axioms
\[
  \xymatrixrowsep{1pt}
  \xymatrixcolsep{25pt}
  \xymatrix{
    \coassocl{.1\textwidth} = \coassocr{.1\textwidth} & \counitl{.1\textwidth} =
    \idone{.1\textwidth} & \cocommute{.1\textwidth} = \comult{.07\textwidth} \\
    \textrm{(coassociativity)} & \textrm{(counitality)} &
    \textrm{(cocommutativity)}
  }
\]
and the Frobenius and special axioms
  \[
  \xymatrixrowsep{1pt}
  \xymatrixcolsep{25pt}
  \xymatrix{
    \frobs{.1\textwidth} = \frobx{.1\textwidth} = \frobz{.1\textwidth} & \spec{.1\textwidth} =
    \idone{.1\textwidth} \\
    \textrm{(Frobenius)} & \textrm{(special)} 
  }
  \]
  We call $\mu$ the \define{multiplication}, $\eta$ the \define{unit}, $\delta$
  the \define{comultiplication}, and $\epsilon$ the \define{counit}.
\end{definition}

Special commutative Frobenius monoids were first formulated by Carboni and
Walters, under the name commutative separable algebras. The Frobenius law and
the special law were termed the S=X law and the diamond=1 law respectively
\cite{CarWal,RosSabWal}.

Alternate axiomatisations are possible. In addition to the `upper' unitality law
above, the mirror image `lower' unitality law also holds, due to commutativity
and the naturality of the braiding. While we write two equations for the
Frobenius law, this is redundant: given the other axioms, the equality of any
two of the diagrams implies the equality of all three.  Further, note that a
monoid and comonoid obeying the Frobenius law is commutative if and only if it
is cocommutative. Thus while a commutative and cocommutative Frobenius monoid
might more properly be called a bicommutative Frobenius monoid, there is no
ambiguity if we only say commutative.

The common feature to these equations is that each side describes a different
way of using the generators to connect some chosen set of inputs to some chosen
set of outputs. This observation provides a `coherence' type result for special
commutative Frobenius monoids, known as the `spider theorem'.
\begin{theorem}
  Let $(X,\mu,\eta,\delta,\epsilon)$ be a special commutative Frobenius monoid,
  and let $f,g\maps X^{\ot n} \to X^{\ot m}$ be map constructed, using
  composition and the monoidal product, from $\mu$, $\eta$, $\delta$, $\epsilon$, the
  coherence maps and braiding, and the identity map on $X$. Then $f$ and $g$ are
  equal if and only if given their string diagrams in the above notation, there
  exists a bijection between the connected components of the two diagrams such
  that corresponding connected components connect the exact same sets of inputs
  and outputs.
\end{theorem}%
See \cite{Kis16,FonCoy16} \cite{CK,CPP} for further details.

\subsection{Hypergraph categories}

\begin{definition}
  A \define{hypergraph category} is a symmetric monoidal category in which each
  object $X$ is equipped with a special commutative Frobenius structure
  $(X,\mu_X,\delta_X,\eta_X,\epsilon_X)$ such that 
  \[
    \tikzset{every path/.style={line width=1.1pt}}
    \xymatrixcolsep{1.8ex}
    \xymatrixrowsep{1ex}
    \xymatrix{
    \begin{aligned}
      \begin{tikzpicture}[scale=.65]
	\begin{pgfonlayer}{nodelayer}
		\node [style=none] (0) at (-0.25, 0.5) {};
		\node [style=dot] (1) at (0.5, -0) {};
		\node [style=none] (2) at (-0.25, -0.5) {};
		\node [style=none] (3) at (1.25, -0) {};
		\node [style=none] (4) at (-1.5, 0.5) {$X \otimes Y$};
		\node [style=none] (5) at (-1.5, -0.5) {$X \otimes Y$};
		\node [style=none] (6) at (2, -0) {$X \otimes Y$};
		\node [style=none] (7) at (-0.75, 0.5) {};
		\node [style=none] (8) at (-0.75, -0.5) {};
	\end{pgfonlayer}
	\begin{pgfonlayer}{edgelayer}
		\draw [in=90, out=0, looseness=0.90] (0.center) to (1.center);
		\draw [in=-90, out=0, looseness=0.90] (2.center) to (1.center);
		\draw (1.center) to (3.center);
		\draw (7.center) to (0.center);
		\draw (8.center) to (2.center);
	\end{pgfonlayer}
\end{tikzpicture}
\end{aligned}
=
\begin{aligned}
  \begin{tikzpicture}[scale=.65]
	\begin{pgfonlayer}{nodelayer}
		\node [style=none] (0) at (-0.25, 0.5) {};
		\node [style=dot] (1) at (0.5, -0) {};
		\node [style=none] (2) at (-0.25, -0.5) {};
		\node [style=none] (3) at (1.25, -0) {};
		\node [style=none] (4) at (-1.75, 0.5) {$X$};
		\node [style=none] (5) at (-1.75, -1.25) {$X$};
		\node [style=none] (6) at (1.5, -0) {$X$};
		\node [style=none] (7) at (-1.5, 0.5) {};
		\node [style=none] (8) at (-1.5, -1.25) {};
		\node [style=none] (9) at (1.25, -1.75) {};
		\node [style=none] (10) at (-1.5, -2.25) {};
		\node [style=none] (11) at (1.5, -1.75) {$Y$};
		\node [style=none] (12) at (-1.5, -0.5) {};
		\node [style=none] (13) at (-0.25, -2.25) {};
		\node [style=dot] (14) at (0.5, -1.75) {};
		\node [style=none] (15) at (-0.25, -1.25) {};
		\node [style=none] (16) at (-1.75, -0.5) {$Y$};
		\node [style=none] (17) at (-1.75, -2.25) {$Y$};
	\end{pgfonlayer}
	\begin{pgfonlayer}{edgelayer}
		\draw [in=90, out=0, looseness=0.90] (0.center) to (1.center);
		\draw [in=-90, out=0, looseness=0.90] (2.center) to (1.center);
		\draw (1.center) to (3.center);
		\draw (7.center) to (0.center);
		\draw [in=180, out=0, looseness=1.00] (8.center) to (2.center);
		\draw [in=90, out=0, looseness=0.90] (15.center) to (14);
		\draw [in=-90, out=0, looseness=0.90] (13.center) to (14);
		\draw (14) to (9.center);
		\draw [in=180, out=0, looseness=1.00] (12.center) to (15.center);
		\draw (10.center) to (13.center);
	\end{pgfonlayer}
\end{tikzpicture}
\end{aligned}
& &   
\qquad
  \begin{aligned}
    \begin{tikzpicture}[scale=.65]
	\begin{pgfonlayer}{nodelayer}
		\node [style=dot] (0) at (-0.5, 0.5) {};
		\node [style=none] (1) at (1.5, 0.5) {$X\otimes Y$};
		\node [style=none] (2) at (0.75, 0.5) {};
	\end{pgfonlayer}
	\begin{pgfonlayer}{edgelayer}
		\draw (2.center) to (0.center);
	\end{pgfonlayer}
\end{tikzpicture}
  \end{aligned}
  = \qquad
  \begin{aligned}
    \begin{tikzpicture}[scale=.65]
	\begin{pgfonlayer}{nodelayer}
		\node [style=dot] (0) at (0, 0.5) {};
		\node [style=none] (1) at (1.5, 0.5) {$X$};
		\node [style=none] (2) at (1.25, 0.5) {};
		\node [style=none] (3) at (1.25, -0.25) {};
		\node [style=dot] (4) at (0, -0.25) {};
		\node [style=none] (5) at (1.5, -0.25) {$Y$};
	\end{pgfonlayer}
	\begin{pgfonlayer}{edgelayer}
		\draw (2.center) to (0.center);
		\draw (3.center) to (4.center);
	\end{pgfonlayer}
\end{tikzpicture}
  \end{aligned}
\\
    \begin{aligned}
\begin{tikzpicture}[scale=.65]
	\begin{pgfonlayer}{nodelayer}
		\node [style=none] (0) at (0.75, 0.5) {};
		\node [style=dot] (1) at (0, -0) {};
		\node [style=none] (2) at (0.75, -0.5) {};
		\node [style=none] (3) at (-0.75, -0) {};
		\node [style=none] (4) at (2, 0.5) {$X \otimes Y$};
		\node [style=none] (5) at (2, -0.5) {$X \otimes Y$};
		\node [style=none] (6) at (-1.5, -0) {$X \otimes Y$};
		\node [style=none] (7) at (1.25, 0.5) {};
		\node [style=none] (8) at (1.25, -0.5) {};
	\end{pgfonlayer}
	\begin{pgfonlayer}{edgelayer}
		\draw [in=90, out=180, looseness=0.90] (0.center) to (1.center);
		\draw [in=-90, out=180, looseness=0.90] (2.center) to (1.center);
		\draw (1.center) to (3.center);
		\draw (7.center) to (0.center);
		\draw (8.center) to (2.center);
	\end{pgfonlayer}
\end{tikzpicture}
\end{aligned}
=
\begin{aligned}
\begin{tikzpicture}[scale=.65]
	\begin{pgfonlayer}{nodelayer}
		\node [style=none] (0) at (0, 0.5) {};
		\node [style=dot] (1) at (-0.75, -0) {};
		\node [style=none] (2) at (0, -0.5) {};
		\node [style=none] (3) at (-1.5, -0) {};
		\node [style=none] (4) at (1.5, 0.5) {$X$};
		\node [style=none] (5) at (1.5, -1.25) {$X$};
		\node [style=none] (6) at (-1.75, -0) {$X$};
		\node [style=none] (7) at (1.25, 0.5) {};
		\node [style=none] (8) at (1.25, -1.25) {};
		\node [style=none] (9) at (-1.5, -1.75) {};
		\node [style=none] (10) at (1.25, -2.25) {};
		\node [style=none] (11) at (-1.75, -1.75) {$Y$};
		\node [style=none] (12) at (1.25, -0.5) {};
		\node [style=none] (13) at (0, -2.25) {};
		\node [style=dot] (14) at (-0.75, -1.75) {};
		\node [style=none] (15) at (0, -1.25) {};
		\node [style=none] (16) at (1.5, -0.5) {$Y$};
		\node [style=none] (17) at (1.5, -2.25) {$Y$};
	\end{pgfonlayer}
	\begin{pgfonlayer}{edgelayer}
		\draw [in=90, out=180, looseness=0.90] (0.center) to (1.center);
		\draw [in=-90, out=180, looseness=0.90] (2.center) to (1.center);
		\draw (1.center) to (3.center);
		\draw (7.center) to (0.center);
		\draw [in=0, out=180, looseness=1.00] (8.center) to (2.center);
		\draw [in=90, out=180, looseness=0.90] (15.center) to (14);
		\draw [in=-90, out=180, looseness=0.90] (13.center) to (14);
		\draw (14) to (9.center);
		\draw [in=0, out=180, looseness=1.00] (12.center) to (15.center);
		\draw (10.center) to (13.center);
	\end{pgfonlayer}
\end{tikzpicture}
\end{aligned}
& &
  \begin{aligned}
    \begin{tikzpicture}[scale=.65]
	\begin{pgfonlayer}{nodelayer}
		\node [style=dot] (0) at (1.5, 0.5) {};
		\node [style=none] (1) at (-0.5, 0.5) {$X\otimes Y$};
		\node [style=none] (2) at (0.25, 0.5) {};
	\end{pgfonlayer}
	\begin{pgfonlayer}{edgelayer}
		\draw (2.center) to (0.center);
	\end{pgfonlayer}
\end{tikzpicture}
  \end{aligned}
  \qquad \quad =
  \begin{aligned}
    \begin{tikzpicture}[scale=.65]
	\begin{pgfonlayer}{nodelayer}
		\node [style=dot] (0) at (1.5, 0.5) {};
		\node [style=none] (1) at (0, 0.5) {$X$};
		\node [style=none] (2) at (0.25, 0.5) {};
		\node [style=none] (3) at (0.25, -0.25) {};
		\node [style=dot] (4) at (1.5, -0.25) {};
		\node [style=none] (5) at (0, -0.25) {$Y$};
	\end{pgfonlayer}
	\begin{pgfonlayer}{edgelayer}
		\draw (2.center) to (0.center);
		\draw (3.center) to (4.center);
	\end{pgfonlayer}
\end{tikzpicture}
\qquad \quad
  \end{aligned}
}
\]
\end{definition}

Note that we do \emph{not} require these Frobenius morphisms to be natural in
$X$. While morphisms in a hypergraph category need not interact with the Frobenius
structure in any particular way, we do require functors between hypergraph
categories to preserve it.

\begin{definition}
A functor $(F,\varphi)$ of hypergraph categories, or \define{hypergraph
functor}, is a strong symmetric monoidal functor $(F,\varphi)$ such that for
each object $X$ the following diagrams commute:
\[
  \xymatrix{
    FX\boxtimes FX \ar[rr]^{\mu_{FX}} \ar[dr]_{\varphi} && FX \\
    & F(X \ot X) \ar[ur]_{F\mu_X}
  }
  \qquad
  \xymatrix{
    1_{\mc D} \ar[rr]^{\eta_{FX}} \ar[dr]_{\varphi_1} && FX \\
    & F1_{\mc C} \ar[ur]_{F\eta_X}
  }
\]
\[
  \xymatrix{
    FX \ar[rr]^{\delta_{FX}} \ar[dr]_{F\delta_X} && FX \boxtimes FX\\
    & F(X \ot X) \ar[ur]_{\varphi^{-1}}
  }
  \qquad
  \xymatrix{
    FX \ar[rr]^{\epsilon_{FX}} \ar[dr]_{F\epsilon_X} && 1_{\mc D} \\
    & F1_{\mc C} \ar[ur]_{\varphi^{-1}}
  }
\]
\end{definition}

Equivalently, a strong symmetric monoidal functor $F$ is a hypergraph functor if
for every $X$ the special commutative Frobenius structure on $FX$ is
\[
  (FX,\enspace F\mu_X \circ \varphi_{X,X},\enspace  \varphi^{-1}_{X,X} \circ F\delta_X,\enspace  F\eta_X \circ
  \varphi_1,\enspace  \varphi_1^{-1} \circ F\epsilon_X).
\]

Just as monoidal natural transformations themselves are enough as morphisms
between symmetric monoidal functors, so too they suffice as morphisms between
hypergraph functors. Two hypergraph categories are \define{hypergraph
equivalent} if there exist hypergraph functors with monoidal natural
transformations to the identity functors. 
  
The term hypergraph category was introduced recently \cite{Fon15,Kis16}, in
reference to the fact that these special commutative Frobenius monoids provide
precisely the structure required to draw graphs with `hyperedges': edges
connecting any number of inputs to any number of outputs. Again first defined by
Walters and Carboni \cite{Car91}, under the name well-supported compact closed
categories, in recent years hypergraph categories have been rediscovered a
number of times, also appearing under the names dungeon categories \cite{Mor12}
and dgs-monoidal categories \cite{Gad}. 

%Nonetheless, this flexibility is a strength of hypergraph categories. In the
%category of vector spaces, for example, a special commutative Frobenius monoid
%is a basis \cite{CoePavVic}. This allows us to talk about the hypergraph category of vector spaces with bases .



\subsection{Hypergraph categories are self-dual compact closed}

Note that if an object $X$ is equipped with a Frobenius monoid structure then
the maps 
\[
    \xymatrixrowsep{0pt}
    \xymatrix{
  \begin{aligned}
      \resizebox{.09\textwidth}{!}{
	\begin{tikzpicture}
	  \begin{pgfonlayer}{nodelayer}
	    \node [style=circ] (0) at (0.75, -0) {};
	    \node [style=circ] (1) at (0.125, -0) {};
	    \node [style=none] (2) at (-1, 0.5) {};
	    \node [style=none] (3) at (-1, -0.5) {};
	  \end{pgfonlayer}
	  \begin{pgfonlayer}{edgelayer}
	    \draw [line width=2pt] (0.center) to (1.center);
	    \draw [line width=2pt, in=0, out=120, looseness=1.20] (1.center) to (2.center);
	    \draw [line width=2pt, in=0, out=-120, looseness=1.20] (1.center) to (3.center);
	  \end{pgfonlayer}
	\end{tikzpicture} 
    }
  \end{aligned}
  & \quad \mbox{and} \quad&
  \begin{aligned}
      \resizebox{.09\textwidth}{!}{
	\begin{tikzpicture}
	  \begin{pgfonlayer}{nodelayer}
	    \node [style=circ] (0) at (-1, 0) {};
	    \node [style=circ] (1) at (-0.375, 0) {};
	    \node [style=none] (2) at (0.75, -0.5) {};
	    \node [style=none] (3) at (0.75, 0.5) {};
	  \end{pgfonlayer}
	  \begin{pgfonlayer}{edgelayer}
	    \draw [line width=2pt] (0.center) to (1.center);
	    \draw [line width=2pt, in=180, out=-60, looseness=1.20] (1.center) to (2.center);
	    \draw [line width=2pt, in=180, out=60, looseness=1.20] (1.center) to (3.center);
	  \end{pgfonlayer}
	\end{tikzpicture}
      } 
  \end{aligned} \\
      \epsilon \circ \mu\maps X \ot X \to 1 & &
      \delta \circ \eta\maps 1 \to X \ot X
    }
\]
obey both
\[
  \begin{aligned}
    \resizebox{3cm}{!}{
      \begin{tikzpicture}
	\begin{pgfonlayer}{nodelayer}
	  \node [style=circ] (0) at (-1.5, 0.5) {};
	  \node [style=circ] (1) at (-0.75, 0.5) {};
	  \node [style=none] (2) at (0.25, -0) {};
	  \node [style=none] (3) at (0.25, 1) {};
	  \node [style=circ] (4) at (1, -0.5) {};
	  \node [style=none] (5) at (0, -0) {};
	  \node [style=circ] (6) at (1.75, -0.5) {};
	  \node [style=none] (7) at (0, -1) {};
	  \node [style=none] (8) at (2.5, 1) {};
	  \node [style=none] (9) at (-2.5, -1) {};
	\end{pgfonlayer}
	\begin{pgfonlayer}{edgelayer}
	  \draw [line width=2pt, in=180, out=-60, looseness=1.20] (1) to (2.center);
	  \draw [line width=2pt, in=180, out=60, looseness=1.20] (1) to (3.center);
	  \draw [line width=2pt] (0.center) to (1);
	  \draw [line width=2pt] (6.center) to (4);
	  \draw [line width=2pt, in=0, out=120, looseness=1.20] (4) to (5.center);
	  \draw [line width=2pt, in=0, out=-120, looseness=1.20] (4) to (7.center);
	  \draw [line width=2pt] (3.center) to (8.center);
	  \draw [line width=2pt] (7.center) to (9.center);
	\end{pgfonlayer}
      \end{tikzpicture}
    }
  \end{aligned}
  \quad = \quad
  \begin{aligned}
    \resizebox{3cm}{!}{
      \begin{tikzpicture}
	\begin{pgfonlayer}{nodelayer}
	  \node [style=circ] (0) at (-0.5, -0) {};
	  \node [style=none] (1) at (-1.5, -0.5) {};
	  \node [style=circ] (2) at (-1.5, 0.5) {};
	  \node [style=circ] (3) at (0.5, -0) {};
	  \node [style=circ] (4) at (1.5, -0.5) {};
	  \node [style=none] (5) at (1.5, 0.5) {};
	  \node [style=none] (6) at (2.5, 0.5) {};
	  \node [style=none] (7) at (-2.5, -0.5) {};
	\end{pgfonlayer}
	\begin{pgfonlayer}{edgelayer}
	  \draw [line width=2pt, in=0, out=-120, looseness=1.20] (0.center) to (1.center);
	  \draw [line width=2pt, in=0, out=120, looseness=1.20] (0.center) to (2.center);
	  \draw [line width=2pt, in=180, out=-60, looseness=1.20] (3) to (4.center);
	  \draw [line width=2pt, in=180, out=60, looseness=1.20] (3) to (5.center);
	  \draw [line width=2pt] (0) to (3);
	  \draw [line width=2pt] (7.center) to (1.center);
	  \draw [line width=2pt] (5.center) to (6.center);
	\end{pgfonlayer}
      \end{tikzpicture}
    }
  \end{aligned}
  \quad = \quad
  \begin{aligned}
    \resizebox{2cm}{!}{
      \begin{tikzpicture}
	\begin{pgfonlayer}{nodelayer}
	  \node [style=none] (0) at (2, -0) {};
	  \node [style=none] (1) at (-2, -0) {};
	  \node [style=none] (2) at (0, -0.5) {};
	  \node [style=none] (3) at (0, 0.5) {};
	\end{pgfonlayer}
	\begin{pgfonlayer}{edgelayer}
	  \draw [line width=2pt](1.center) to (0.center);
	\end{pgfonlayer}
      \end{tikzpicture}
    }
  \end{aligned}
\]
and the reflected equations. Thus if an object carries a Frobenius monoid it is
also self-dual, and any hypergraph category is a fortiori self-dual compact
closed. 

We introduce the notation
  \[
    \tikzset{every path/.style={line width=1.1pt}}
    \begin{aligned}
      \begin{tikzpicture}[scale=.65]
	\begin{pgfonlayer}{nodelayer}
		\node [style=none] (0) at (0.75, 0.5) {};
		\node [style=none] (1) at (0, -0) {};
		\node [style=none] (2) at (0.75, -0.5) {};
		\node [style=none] (3) at (1.25, 0.5) {};
		\node [style=none] (4) at (1.25, -0.5) {};
	\end{pgfonlayer}
	\begin{pgfonlayer}{edgelayer}
		\draw [in=90, out=180, looseness=0.90] (0.center) to (1.center);
		\draw [in=-90, out=180, looseness=0.90] (2.center) to (1.center);
		\draw (3.center) to (0.center);
		\draw (4.center) to (2.center);
	\end{pgfonlayer}
\end{tikzpicture}
    \end{aligned}
    :=
    \begin{aligned}
      \begin{tikzpicture}[scale=.65]
	\begin{pgfonlayer}{nodelayer}
		\node [style=none] (0) at (0.75, 0.5) {};
		\node [style=dot] (1) at (0, -0) {};
		\node [style=none] (2) at (0.75, -0.5) {};
		\node [style=none] (3) at (1.25, 0.5) {};
		\node [style=none] (4) at (1.25, -0.5) {};
		\node [style=dot] (5) at (-0.5, -0) {};
	\end{pgfonlayer}
	\begin{pgfonlayer}{edgelayer}
		\draw [in=90, out=180, looseness=0.90] (0.center) to (1.center);
		\draw [in=-90, out=180, looseness=0.90] (2.center) to (1.center);
		\draw (3.center) to (0.center);
		\draw (4.center) to (2.center);
		\draw (5.center) to (1.center);
	\end{pgfonlayer}
\end{tikzpicture}
    \end{aligned}
    \qquad
    \qquad
    \begin{aligned}
      \begin{tikzpicture}[scale=.65]
	\begin{pgfonlayer}{nodelayer}
		\node [style=none] (0) at (0, 0.5) {};
		\node [style=none] (1) at (0.75, -0) {};
		\node [style=none] (2) at (0, -0.5) {};
		\node [style=none] (3) at (-0.5, 0.5) {};
		\node [style=none] (4) at (-0.5, -0.5) {};
	\end{pgfonlayer}
	\begin{pgfonlayer}{edgelayer}
		\draw [in=90, out=0, looseness=0.90] (0.center) to (1.center);
		\draw [in=-90, out=0, looseness=0.90] (2.center) to (1.center);
		\draw (3.center) to (0.center);
		\draw (4.center) to (2.center);
	\end{pgfonlayer}
\end{tikzpicture}
    \end{aligned}
    :=
    \begin{aligned}
      \begin{tikzpicture}[scale=.65]
	\begin{pgfonlayer}{nodelayer}
		\node [style=none] (0) at (0, 0.5) {};
		\node [style=dot] (1) at (0.75, -0) {};
		\node [style=none] (2) at (0, -0.5) {};
		\node [style=none] (3) at (-0.5, 0.5) {};
		\node [style=none] (4) at (-0.5, -0.5) {};
		\node [style=dot] (5) at (1.25, -0) {};
	\end{pgfonlayer}
	\begin{pgfonlayer}{edgelayer}
		\draw [in=90, out=0, looseness=0.90] (0.center) to (1.center);
		\draw [in=-90, out=0, looseness=0.90] (2.center) to (1.center);
		\draw (3.center) to (0.center);
		\draw (4.center) to (2.center);
		\draw (5.center) to (1.center);
	\end{pgfonlayer}
\end{tikzpicture}
    \end{aligned}
  \]
As in any self-dual compact closed category, mapping each morphism 
$
    \tikzset{every path/.style={line width=1.1pt}}
    \begin{aligned}
  \begin{tikzpicture}[scale=.65]
	\begin{pgfonlayer}{nodelayer}
		\node [style=none] (0) at (0.5, -0.5) {};
		\node [style=none] (1) at (-0.25, -0.5) {};
		\node [style=none] (2) at (-1, -0.5) {};
		\node [style=none] (3) at (-1.75, -0.5) {};
		\node [style=none] (4) at (-1, -0.125) {};
		\node [style=none] (5) at (-1, -0.875) {};
		\node [style=none] (6) at (-0.25, -0.875) {};
		\node [style=none] (7) at (-0.25, -0.125) {};
		\node [style=none] (8) at (-0.625, -0.5) {$f$};
		\node [style=none] (9) at (-2, -0.5) {$X$};
		\node [style=none] (10) at (0.75, -0.5) {$Y$};
	\end{pgfonlayer}
	\begin{pgfonlayer}{edgelayer}
		\draw (1.center) to (0.center);
		\draw (2.center) to (3.center);
		\draw (4.center) to (5.center);
		\draw (5.center) to (6.center);
		\draw (6.center) to (7.center);
		\draw (7.center) to (4.center);
	\end{pgfonlayer}
\end{tikzpicture}
    \end{aligned}
$
to its dual morphism
\[
    \tikzset{every path/.style={line width=1.1pt}}
\begin{tikzpicture}[scale=.65]
	\begin{pgfonlayer}{nodelayer}
		\node [style=none] (0) at (0, 0.5) {};
		\node [style=none] (1) at (0.75, -0) {};
		\node [style=none] (2) at (0, -0.5) {};
		\node [style=none] (3) at (-2.5, 0.5) {};
		\node [style=none] (4) at (-0.25, -0.5) {};
		\node [style=none] (5) at (-2, -1) {};
		\node [style=none] (6) at (-1.25, -1.5) {};
		\node [style=none] (7) at (-1, -0.5) {};
		\node [style=none] (8) at (1.25, -1.5) {};
		\node [style=none] (9) at (-1.25, -0.5) {};
		\node [style=none] (10) at (-1, -0.125) {};
		\node [style=none] (11) at (-1, -0.875) {};
		\node [style=none] (12) at (-0.25, -0.875) {};
		\node [style=none] (13) at (-0.25, -0.125) {};
		\node [style=none] (14) at (-0.625, -0.5) {$f$};
		\node [style=none] (15) at (1.5, -1.5) {$X$};
		\node [style=none] (16) at (-2.75, 0.5) {$Y$};
	\end{pgfonlayer}
	\begin{pgfonlayer}{edgelayer}
		\draw [in=90, out=0, looseness=0.90] (0.center) to (1.center);
		\draw [in=-90, out=0, looseness=0.90] (2.center) to (1.center);
		\draw (3.center) to (0.center);
		\draw (4.center) to (2.center);
		\draw [in=90, out=180, looseness=0.90] (9.center) to (5.center);
		\draw [in=-90, out=180, looseness=0.90] (6.center) to (5.center);
		\draw (7.center) to (9.center);
		\draw (8.center) to (6.center);
		\draw (10.center) to (11.center);
		\draw (11.center) to (12.center);
		\draw (12.center) to (13.center);
		\draw (13.center) to (10.center);
	\end{pgfonlayer}
\end{tikzpicture}
\]
further equips each hypergraph category with a so-called dagger functor---an
involutive contravariant endofunctor that is the identity on objects---such that
the category is a dagger compact category. Dagger compact categories were first
introduced in the context of categorical quantum mechanics \cite{AC}, under the
name strongly compact closed category, and have been demonstrated to be a key
structure in diagrammatic reasoning and the logic of quantum mechanics.

Compactness allows us to blur the distinction between composition and the
monoidal product of morphisms. Firstly, there is a one-to-one correspondence
between morphisms $X \to Y$ and morphisms $1 \to X\ot Y$ given by taking 
$
    \tikzset{every path/.style={line width=1.1pt}}
    \begin{aligned}
  \begin{tikzpicture}[scale=.65]
	\begin{pgfonlayer}{nodelayer}
		\node [style=none] (0) at (0.5, -0.5) {};
		\node [style=none] (1) at (-0.25, -0.5) {};
		\node [style=none] (2) at (-1, -0.5) {};
		\node [style=none] (3) at (-1.75, -0.5) {};
		\node [style=none] (4) at (-1, -0.125) {};
		\node [style=none] (5) at (-1, -0.875) {};
		\node [style=none] (6) at (-0.25, -0.875) {};
		\node [style=none] (7) at (-0.25, -0.125) {};
		\node [style=none] (8) at (-0.625, -0.5) {$f$};
		\node [style=none] (9) at (-2, -0.5) {$X$};
		\node [style=none] (10) at (0.75, -0.5) {$Y$};
	\end{pgfonlayer}
	\begin{pgfonlayer}{edgelayer}
		\draw (1.center) to (0.center);
		\draw (2.center) to (3.center);
		\draw (4.center) to (5.center);
		\draw (5.center) to (6.center);
		\draw (6.center) to (7.center);
		\draw (7.center) to (4.center);
	\end{pgfonlayer}
\end{tikzpicture}
    \end{aligned}
$
to its so-called \define{name}
\[
    \tikzset{every path/.style={line width=1.1pt}}
  \begin{tikzpicture}[scale=.65]
	\begin{pgfonlayer}{nodelayer}
		\node [style=none] (0) at (0.5, -0.5) {};
		\node [style=none] (1) at (-0.25, -0.5) {};
		\node [style=none] (2) at (-2, -0) {};
		\node [style=none] (3) at (-1.25, -0.5) {};
		\node [style=none] (4) at (0.5, 0.5) {};
		\node [style=none] (5) at (-1, -0.5) {};
		\node [style=none] (6) at (-1.25, 0.5) {};
		\node [style=none] (7) at (-1, -0.125) {};
		\node [style=none] (8) at (-1, -0.875) {};
		\node [style=none] (9) at (-0.25, -0.875) {};
		\node [style=none] (10) at (-0.25, -0.125) {};
		\node [style=none] (11) at (-0.625, -0.5) {$f$};
		\node [style=none] (12) at (0.75, 0.5) {$X$};
		\node [style=none] (13) at (0.75, -0.5) {$Y$};
	\end{pgfonlayer}
	\begin{pgfonlayer}{edgelayer}
		\draw (1.center) to (0.center);
		\draw [in=90, out=180, looseness=0.90] (6.center) to (2.center);
		\draw [in=-90, out=180, looseness=0.90] (3.center) to (2.center);
		\draw (4.center) to (6.center);
		\draw (5.center) to (3.center);
		\draw (7.center) to (8.center);
		\draw (8.center) to (9.center);
		\draw (9.center) to (10.center);
		\draw (10.center) to (7.center);
	\end{pgfonlayer}
\end{tikzpicture}
\]
By compactness, we have the equation
\[
    \tikzset{every path/.style={line width=1.1pt}}
  \begin{aligned}
\begin{tikzpicture}[scale=.65]
	\begin{pgfonlayer}{nodelayer}
		\node [style=none] (0) at (0, -0.5) {};
		\node [style=none] (1) at (-0.25, -0.5) {};
		\node [style=none] (2) at (-2, -0) {};
		\node [style=none] (3) at (-1.25, -0.5) {};
		\node [style=none] (4) at (1.5, 0.5) {};
		\node [style=none] (5) at (-1, -0.5) {};
		\node [style=none] (6) at (-1.25, 0.5) {};
		\node [style=none] (7) at (-1, -0.125) {};
		\node [style=none] (8) at (-1, -0.875) {};
		\node [style=none] (9) at (-0.25, -0.875) {};
		\node [style=none] (10) at (-0.25, -0.125) {};
		\node [style=none] (11) at (-0.625, -0.5) {$f$};
		\node [style=none] (12) at (1.75, 0.5) {$X$};
		\node [style=none] (13) at (-2, -2) {};
		\node [style=none] (14) at (-0.25, -2.125) {};
		\node [style=none] (15) at (-0.25, -2.5) {};
		\node [style=none] (16) at (-1, -2.5) {};
		\node [style=none] (17) at (1.5, -2.5) {};
		\node [style=none] (18) at (-0.25, -2.875) {};
		\node [style=none] (19) at (-1.25, -2.5) {};
		\node [style=none] (20) at (0, -1.5) {};
		\node [style=none] (21) at (-0.625, -2.5) {$g$};
		\node [style=none] (22) at (-1, -2.125) {};
		\node [style=none] (23) at (-1.25, -1.5) {};
		\node [style=none] (24) at (-1, -2.875) {};
		\node [style=none] (25) at (1.75, -2.5) {$Z$};
		\node [style=none] (26) at (1, -1) {};
	\end{pgfonlayer}
	\begin{pgfonlayer}{edgelayer}
		\draw (1.center) to (0.center);
		\draw [in=90, out=180, looseness=0.90] (6.center) to (2.center);
		\draw [in=-90, out=180, looseness=0.90] (3.center) to (2.center);
		\draw (4.center) to (6.center);
		\draw (5.center) to (3.center);
		\draw (7.center) to (8.center);
		\draw (8.center) to (9.center);
		\draw (9.center) to (10.center);
		\draw (10.center) to (7.center);
		\draw (15.center) to (17.center);
		\draw [in=90, out=180, looseness=0.90] (23.center) to (13.center);
		\draw [in=-90, out=180, looseness=0.90] (19.center) to (13.center);
		\draw (20.center) to (23.center);
		\draw (16.center) to (19.center);
		\draw (22.center) to (24.center);
		\draw (24.center) to (18.center);
		\draw (18.center) to (14.center);
		\draw (14.center) to (22.center);
		\draw [in=90, out=0, looseness=0.90] (0.center) to (26.center);
		\draw [in=0, out=-90, looseness=0.90] (26.center) to (20.center);
	\end{pgfonlayer}
\end{tikzpicture}
  \end{aligned}
  \qquad
  =
  \qquad
  \begin{aligned}
\begin{tikzpicture}[scale=.65]
	\begin{pgfonlayer}{nodelayer}
		\node [style=none] (0) at (-0.25, -0.5) {};
		\node [style=none] (1) at (-2, -0) {};
		\node [style=none] (2) at (-1.25, -0.5) {};
		\node [style=none] (3) at (1.5, 0.5) {};
		\node [style=none] (4) at (-1, -0.5) {};
		\node [style=none] (5) at (-1.25, 0.5) {};
		\node [style=none] (6) at (-1, -0.125) {};
		\node [style=none] (7) at (-1, -0.875) {};
		\node [style=none] (8) at (-0.25, -0.875) {};
		\node [style=none] (9) at (-0.25, -0.125) {};
		\node [style=none] (10) at (-0.625, -0.5) {$f$};
		\node [style=none] (11) at (1.75, 0.5) {$X$};
		\node [style=none] (12) at (1, -0.125) {};
		\node [style=none] (13) at (1, -0.5) {};
		\node [style=none] (14) at (0.25, -0.5) {};
		\node [style=none] (15) at (1.5, -0.5) {};
		\node [style=none] (16) at (1, -.875) {};
		\node [style=none] (17) at (0.625, -0.5) {$g$};
		\node [style=none] (18) at (0.25, -0.125) {};
		\node [style=none] (19) at (0.25, -.875) {};
		\node [style=none] (20) at (1.75, -0.5) {$Z$};
	\end{pgfonlayer}
	\begin{pgfonlayer}{edgelayer}
		\draw [in=90, out=180, looseness=0.90] (5.center) to (1.center);
		\draw [in=-90, out=180, looseness=0.90] (2.center) to (1.center);
		\draw (3.center) to (5.center);
		\draw (4.center) to (2.center);
		\draw (6.center) to (7.center);
		\draw (7.center) to (8.center);
		\draw (8.center) to (9.center);
		\draw (9.center) to (6.center);
		\draw (13.center) to (15.center);
		\draw (18.center) to (19.center);
		\draw (19.center) to (16.center);
		\draw (16.center) to (12.center);
		\draw (12.center) to (18.center);
		\draw (0.center) to (14.center);
	\end{pgfonlayer}
\end{tikzpicture}
  \end{aligned}
\]
Here the right hand side is the name of the composite $f \circ g$, while the
left hand side is the monoidal product post-composed with the map
\[
    \tikzset{every path/.style={line width=1.1pt}}
\begin{tikzpicture}[scale=.65]
	\begin{pgfonlayer}{nodelayer}
		\node [style=none] (0) at (0, -0.5) {};
		\node [style=none] (1) at (1.5, 0.5) {};
		\node [style=none] (2) at (1.75, 0.5) {$X$};
		\node [style=none] (3) at (-0.5, -0.5) {$Y$};
		\node [style=none] (4) at (-0.25, -2.5) {};
		\node [style=none] (5) at (1.5, -2.5) {};
		\node [style=none] (6) at (0, -1.5) {};
		\node [style=none] (7) at (1.75, -2.5) {$Z$};
		\node [style=none] (8) at (1, -1) {};
		\node [style=none] (9) at (-0.25, 0.5) {};
		\node [style=none] (10) at (-0.5, 0.5) {$X$};
		\node [style=none] (11) at (-0.5, -2.5) {$Z$};
		\node [style=none] (12) at (-0.5, -1.5) {$Y$};
		\node [style=none] (13) at (-0.25, -1.5) {};
		\node [style=none] (14) at (-0.25, -0.5) {};
	\end{pgfonlayer}
	\begin{pgfonlayer}{edgelayer}
		\draw (4.center) to (5.center);
		\draw [in=90, out=0, looseness=0.90] (0.center) to (8.center);
		\draw [in=0, out=-90, looseness=0.90] (8.center) to (6.center);
		\draw (1.center) to (9.center);
		\draw (6.center) to (13.center);
		\draw (14.center) to (0.center);
	\end{pgfonlayer}
\end{tikzpicture}
\]
Thus this morphism, the product of a cap and two identity maps, enacts the
categorical composition on monoidal products of names. We will make liberal use
of this fact.

\subsection{Coherence}

The lack of naturality of the Frobenius maps in hypergraph categories affects
some common properties of structured categories. For example, it is not always
possible to construct a skeletal hypergraph category hypergraph equivalent to a
given hypergraph category: isomorphic objects may be equipped with `different'
Frobenius monoids.  Similarly, a fully faithful, essentially surjective
hypergraph functor does not necessarily define a hypergraph equivalence of
categories. 

Nonetheless, in this section we prove that every hypergraph category is
hypergraph equivalent to a strict hypergraph category. This coherence result
will be important in proving that every hypergraph category can be constructed
using decorated corelations.

\begin{theorem} \label{thm.stricthypergraphs}
  Every hypergraph category is hypergraph equivalent to a strict hypergraph
  category. Moreover, the objects of this strict hypergraph category form a free
  monoid.
\end{theorem}
\begin{proof}
  Let $(\mc H,\ot)$ be a hypergraph category. As $\mc H$ is a fortiori a
  symmetric monoidal category, a standard construction (see Mac Lane
  \cite[Theorem]{Mac98}) gives an equivalent strict symmetric monoidal category
  $(\mc H_{\mathrm{str}}, \cdot)$ with objects finite lists $[x_1,\dots,x_n]$ of
  objects of $\mc H$ and morphisms $[x_1,\dots,x_n] \to [y_1,\dots,y_m]$ those
  morphisms from $((x_1 \ot x_2) \ot \dots) \ot x_n \to ((y_1 \ot y_2) \ot
  \dots) \ot y_m$ in $\mc H$.  Composition is given by composition in $\mc H$.
  
  The monoidal structure is given as follows. Given a list $X$ of objects in
  $\mc H$, write $PX$ for the corresponding monoidal product in $\mc H$ with all
  open parenthesis at the front.  The monoidal product of objects in $\mc
  H_{\mathrm{str}}$ is given by concatenation $\cdot$ of lists; the monoidal
  unit is the empty list. The monoidal product of two morphisms is given by
  their monoidal product in $\mc H$ pre- and post-composed with the necessary
  canonical maps: given $f\maps X \to Y$ and $g\maps Z \to W$, their product
  $f\cdot g\maps X\cdot Y \to Z \cdot
  W$ is \[ P(X \cdot Y) \longrightarrow PX \ot PY \stackrel{f \ot
  g}{\longrightarrow} PZ \ot PW \longrightarrow P(Z \cdot W).  \] By design, the
  associators and unitors are simply identity maps. The braiding $X \cdot Y \to
  Y \cdot X$ is given by the braiding $PX \ot PY \to PY \ot PX$ in $\mc H$,
  similarly pre- and post-composed with the necessary canonical maps. This
  defines a strict symmetric monoidal category \cite{Mac98}.

  To make $\mc H_{\mathrm{str}}$ into a hypergraph category, we make each
  object $[x_1,\dots,x_n]$ into a special commutative Frobenius monoid in a
  similar way. For example, the multiplication on $[x_1,\dots,x_n]$ is given by 
  \[
    \xymatrixcolsep{-4pt}
    \xymatrix{
      P([x_1,\dots,x_n]\cdot[x_1,&\dots,x_n])=
      ((((((x_1 \ot x_2) \ot \dots) \ot x_n) \ot x_1) \ot x_2) \ot \dots) \ot
      x_n \ar[d] \\
      &(((x_1 \ot x_1) \ot (x_2 \ot x_2)) \ot \dots) \ot (x_n \ot x_n)
      \ar[d]^{((\mu_{x_1} \ot \mu_{x_2}) \ot \dots ) \ot \mu_{x_n}} \\
      &P([x_1,\dots,x_n])=((x_1 \ot x_2) \ot \dots ) \ot x_n 
      \phantom{P([x_1,\dots,x_n])=}
    }
  \]
  where the first map is the canonical map such that each pair of $x_i$'s remains
  in the same order. It is straightforward to check that this defines a
  hypergraph category.
  \begin{center}
    \begin{tabular}{| c | p{.65\textwidth} |}
      \hline
      \multicolumn{2}{|c|}{The strict hypergraph category $(\mc H_{\mathrm{str}},
      \cdot)$} \\
      \hline
      \textbf{objects} & finite lists $[x_1, \dots, x_n]$ of objects of
      $\mathcal H$ \\ 
      \textbf{morphisms} & $\mathrm{hom}_{\mc H_{\mathrm{str}}}\big([x_1, \dots,
      x_n],[y_1, \dots, y_m]\big)$ \newline $= \mathrm{hom}_{\mc H}\big(((x_1 \ot x_2) \ot
      \dots) \ot x_n, ((y_1 \ot y_2) \ot \dots) \ot y_m\big)$\\ 
      \textbf{composition} & composition of corresponding maps in $\mc H$ \\
      \textbf{monoidal product} & concatenation of lists \\
      \textbf{coherence maps} & associators and unitors are strict; braiding is
      inherited from $\mc H$  \\
      \textbf{hypergraph maps} & lists of hypergraph maps in $\mc H$ \\
      \hline
    \end{tabular}
  \end{center}

  Our standard construction further gives strong symmetric monoidal functors
  $P\maps \mc H_{\mathrm{str}} \to \mc H$ extending the map $P$ above, and
  $S\maps \mc H \to \mc H_{\mathrm{str}}$ sending $x \in \mc H$ to the string
  $[x]$ of length 1 in $\mc H_{\mathrm{str}}$. These extend to hypergraph
  functors.

  In detail, the functor $P$ is given on morphisms by taking a map in
  $\mathrm{hom}_{\mc H_{\mathrm{str}}}(X,Y)$ to the same map considered now as a
  map in $\mathrm{hom}_{\mc H}(PX,PY)$; its coherence maps are given by the
  canonical maps $PX \ot PY \to P(X\cdot Y)$. The functor $S$ is even easier to
  define: a morphism $x \to y$ in $\mc H$ is by definition a morphism $[x] \to
  [y]$ in $\mc H_{\mathrm{str}}$, so $S$ is a monoidal embedding of $\mc H$ into
  $\mc H_{\mathrm{str}}$. 
  
  By Mac Lane's proof of the coherence theorem for monoidal categories these are
  both strong monoidal functors; by inspection they also preserve hypergraph
  structure, and so are hypergraph functors.  As they already witness an
  equivalence of symmetric monoidal categories, thus $\mc H$ and $\mc
  H_{\mathrm{str}}$ are equivalent as hypergraph categories.
\end{proof}

\section{Example: cospan categories}

A central example of a hypergraph category is the category
$\mathrm{Cospan(\mathcal C)}$ of cospans in any category $\mathcal C$ with
finite colimits. We will later see that decorated cospan categories are a
generalisation of such categories, and each inherits a hypergraph structure
from such. 

We first recall the basic definitions. Let $\mc C$ be a category with finite
colimits, writing the coproduct $+$. A \define{cospan}
\[
  \xymatrix{
    & N \\
    X \ar[ur]^{i} && Y \ar[ul]_{o}
  }
\]
from $X$ to $Y$ in $\mathcal C$ is a pair of morphisms with common codomain. We
refer to $X$ and $Y$ as the \define{feet}, and $N$ as the \define{apex}.  Given
two cospans $X \stackrel{i}{\longrightarrow} N \stackrel{o}{\longleftarrow} Y$
and $X \stackrel{i'}{\longrightarrow} N' \stackrel{o'}{\longleftarrow} Y$ with
the same feet, a \define{map of cospans} is a morphism $n\colon  N \to N'$ in
$\mathcal C$ between the apices such that
\[
  \xymatrix{
    & N \ar[dd]^n  \\
    X \ar[ur]^{i} \ar[dr]_{i'} && Y \ar[ul]_{o} \ar[dl]^{o'}\\
    & N'
  }
\]
commutes.

Cospans may be composed using the pushout from the common
foot: given cospans $X \stackrel{i_X}{\longrightarrow} N
\stackrel{o_Y}{\longleftarrow} Y$ and $Y \stackrel{i_Y}{\longrightarrow} M
\stackrel{o_Z}{\longleftarrow} Z$, their composite cospan is $X \stackrel{j_N
\circ i_X}{\longrightarrow} N+_YM \stackrel{j_M\circ i_Z}{\longleftarrow} Z$,
where the top part of
\[
  \xymatrix{
    && N+_YM \\
    & N \ar[ur]^{j_N} && M \ar[ul]_{j_M} \\
    \quad X \quad \ar[ur]^{i_X} && Y \ar[ul]_{o_Y} \ar[ur]^{i_Y} && \quad Z \quad \ar[ul]_{o_Z}
  }
\]
is a pushout square. This composition rule is associative up to isomorphism, and
so we may define a category, in fact a symmetric monoidal bicategory,
$\mathrm{Cospan}(\mathcal C)$ with objects the objects of $\mathcal C$ and
morphisms isomorphism classes of cospans \cite{Ben}.

The symmetric monoidal structure is `inherited' from $\mc C$. Indeed, we shall
consider any category $\mc C$ with finite colimits a symmetric monoidal category
as follows. Given maps $f \maps A \to C$, $g \maps B \to C$ with common
codomain, the universal property of the coproduct gives a unique map $A+B \to
C$. We call this the \define{copairing} of $f$ and $g$, and write it $[f,g]$. The
monoidal product on $\mc C$ is then given by the coproduct $+$, with monoidal
unit the initial object $\varnothing$ and coherence maps given by copairing the
appropriate identity, inclusion, and initial object maps. For example, the
braiding is given by $[\iota_X,\iota_Y]\maps X+Y \to Y+X$ where $\iota_X\maps X
\to Y+X$ and $\iota_Y\maps Y \to Y+X$ are the inclusion maps into the coproduct
$Y+X$.

The category $\mathrm{Cospan(\mathcal C)}$ inherits this symmetric monoidal
structure from $\mathcal C$ as follows. Call a subcategory $\mathcal C$ of a
category $\mathcal D$ \define{wide} if $\mathcal C$ contains all objects of
$\mathcal D$, and call a functor that is faithful and bijective-on-objects a
\define{wide embedding}.  Note then that we have a wide embedding
\[
  \mathcal C \hooklongrightarrow \mathrm{Cospan(\mathcal C)}
\]
that takes each object of $\mathcal C$ to itself as an object of
$\mathrm{Cospan(\mathcal C)}$, and each morphism $f\colon  X \to Y$ in $\mathcal
C$ to the cospan
\[
  \xymatrix{
    & Y \\
    X \ar[ur]^{f} && Y, \ar@{=}[ul]
  }
\]
where the extended `equals' sign denotes an identity morphism. Now since the
monoidal product $+\colon \mathcal C \times \mathcal C \to \mathcal C$ is left
adjoint to the diagram functor, it preserves colimits, and so extends to a
functor $+\colon \mathrm{Cospan(\mathcal C)} \times \mathrm{Cospan(\mathcal C)}
\to \mathrm{Cospan(\mathcal C)}$. The coherence maps are just the images of the
coherence maps in $\mathcal C$ under this wide embedding; checking naturality is
routine, and clearly they still obey the required axioms.

Write $\FinSet$ for the category of finite sets and functions. Replacing
$\FinSet$ with its equivalent strict skeleton, the following proposition is
well-known. 
\begin{proposition} \label{prop.cospanscfm}
  Special commutative Frobenius monoids in a strict symmetric monoidal category
$\mc C$ are in one-to-one correspondence with strict symmetric monoidal functors
$\cospan(\FinSet) \to \mc C$.
\end{proposition}
\begin{proof}
  The special commutative Frobenius monoid is given by the image of the one
  element set. See Lack \cite{Lac04}.
\end{proof}
Over the next few chapters we will further explore this deep link between
cospans and special commutative Frobenius monoids and hypergraph categories. To
begin, we detail a natural hypergraph structure on $\cospan(\mc C)$. 

This hypergraph structure also comes from copairings of identity morphisms.
Call cospans 
\[
  \xymatrix{
    & N \\
    X \ar[ur]^{i} && Y \ar[ul]_{o}
  }
  \qquad \xymatrix@R=8pt{\\\textrm{and}} \qquad 
  \xymatrix{
    & N \\
    Y \ar[ur]^{o} && X \ar[ul]_{i}
  }
\]
that are reflections of each other \define{opposite} cospans. Given any object
$X$ in $\mathcal C$, the copairing $[1_X,1_X]\colon  X + X \to X$ of two identity
maps on $X$, together with the unique map $!\colon  \varnothing \to X$ from the
initial object to $X$, define a monoid structure on $X$. Considering these
maps as morphisms in $\mathrm{Cospan(\mathcal C)}$, we may take them together
with their opposites to give a special commutative Frobenius structure on $X$.
It is easily verified that this gives a hypergraph category.

Given $f \maps X \to Y$ in $\mc C$, abuse notation by writing $f \in
\mathrm{Cospan}(\mc C)$ for the cospan $X \stackrel{f}\to Y
\stackrel{1_Y}\leftarrow Y$, and $f^\opp$ for the cospan $Y \stackrel{1_Y}\to Y
\stackrel{f}\leftarrow X$. To summarise:
  \begin{center}
  \begin{tabular}{ |c| p{.65\textwidth}|}
      \hline
      \multicolumn{2}{|c|}{The hypergraph category $(\mathrm{Cospan(\mc C)},+)$} \\
    \hline
    \textbf{objects} & the objects of $\mathcal C$ \\ 
    \textbf{morphisms} & isomorphism classes of cospans in
    $\mathcal C$\\ 
  \textbf{composition} & given by pushout \\
  \textbf{monoidal product} & the coproduct in $\mathcal C$. \\
  \textbf{coherence maps} & inherited from $(\mc C,+)$\\
  \textbf{hypergraph maps} & $\mu = [1,1]$, $\eta = !$,
  $\delta = [1,1]^\opp$, $\epsilon = !^\opp$. \\
      \hline
  \end{tabular}
\end{center}
   
We will often abuse our terminology and refer to cospans themselves as
morphisms in some cospan category $\mathrm{Cospan}(\mathcal C)$; we of course
refer instead to the isomorphism class of the said cospan.
 
It is not difficult to show that we in fact have a functor from the category
having categories with finite colimits as objects and colimit preserving
functors as morphisms to the category of hypergraph categories and hypergraph
functors. In the next chapter we show that this extends to a functor from the
category of symmetric lax monoidal presheaves over categories with finite
colimits. This is known as the decorated cospans construction.

%Walters: cospan graph is the generic special commutative Frobenius monoid.  The
%free hypergraph category on a single object in the category of cospans in the
%category of finite sets.

The decorated cospans theorem then leads to the decorated corelations
construction, which gives a way to build every hypergraph category using
cospans. Write $\mathrm{Mon}_C$ for the category with objects functions $[n]=\{1,2,
\dots,n \} \to C$, where $C$ is some set, and morphisms functions $[n] \to [m]$
that commute over $C$. We will prove the following.
\begin{theorem}
  The category of hypergraph categories is equivalent to the category of
  lax symmetric monoidal functors
  \[
    (\mathrm{Cospan}(\FinSet_{\mc O}),+) \to (\Set,\times).
  \]
  varying over sets $\mc O$.
\end{theorem}
This theorem, couched in the language of algebras for operads, is also the
subject of a forthcoming paper by Vagner, Spivak, and Schultz \cite{VagSpiSch}.


%\begin{proposition}
%  The monoidal product in a hypergraph category is a coproduct for something if
%  and only if every morphism is a Frobenius monoid homomorphism. (ie take
%  subcategory of all objects, monoid maps, and monoid homs. This is cocomplete.)
%\end{proposition}
%Clearly not true for circuits for example.

