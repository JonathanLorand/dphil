\chapter[Hypergraph categories]{Hypergraph categories: the algebra of
interconnection}
\section{The algebra of interconnection}
Motivation.

Setting has algebraic operations: juxtaposition, interconnection.

\section{Frobenius monoids}

We introduce special commutative Frobenius monoids, writing
our axioms using the string calculus for monoidal categories introduced by Joyal
and Street \cite{JS}. Diagrams will be read left to right, and we shall suppress
the labels as we deal with a unique generating object and a unique generator of
each type. 

Recall that a \define{commutative monoid} $(X,\mu,\eta)$ in symmetric monoidal
category $(\mathcal C,\otimes)$ is an object $X$ of $\mathcal C$ together with
maps 
\[
  \xymatrixrowsep{1pt}
  \xymatrix{
    \mult{.075\textwidth} & & \unit{.075\textwidth} \\
    \mu\maps X\otimes X \to X & & \eta\maps I \to X
  }
\]
obeying
\[
  \xymatrixrowsep{1pt}
  \xymatrixcolsep{25pt}
  \xymatrix{
    \assocl{.1\textwidth} = \assocr{.1\textwidth} & \unitl{.1\textwidth} =
    \idone{.1\textwidth} & \commute{.1\textwidth} = \mult{.07\textwidth} \\
    \textrm{(associativity)} & \textrm{(unitality)} & \textrm{(commutativity)}
  }
\]
where $\swap{1em}$ is the braiding on $X \otimes X$. In addition to the
`upper' unitality law above, the mirror image `lower' unitality law also holds,
due to commutativity and the naturality of the braiding.  

Dually, a \define{cocommutative comonoid} $(X,\mu,\eta)$ in $\mathcal C$ is an
object $X$ together with maps 
\[
  \xymatrixrowsep{1pt}
  \xymatrix{
    \comult{.075\textwidth} & & \counit{.075\textwidth} \\
    \delta\maps X\to X \otimes X & & \epsilon\maps X \to I
  }
\]
obeying
\[
  \xymatrixrowsep{1pt}
  \xymatrixcolsep{25pt}
  \xymatrix{
    \coassocl{.1\textwidth} = \coassocr{.1\textwidth} & \counitl{.1\textwidth} =
    \idone{.1\textwidth} & \cocommute{.1\textwidth} = \comult{.07\textwidth} \\
    \textrm{(coassociativity)} & \textrm{(counitality)} &
    \textrm{(cocommutativity)}
  }
\]
Given a monoid and comonoid on the same object, there are two well-known ways
for them to interact: the bimonoid laws and the Frobenius law. We shall discuss
both in this paper, but for now we restrict our attention to Frobenius structure. 

\begin{definition}
  An \define{extraspecial commutative Frobenius monoid}
  $(X,\mu,\eta,\delta,\epsilon)$ in a monoidal category $(\mathcal C, \otimes)$
  comprises a commutative monoid $(X,\mu,\eta)$ and a cocommutative comonoid
  $(X,\delta,\epsilon)$ that further obey
  \[
  \xymatrixrowsep{1pt}
  \xymatrixcolsep{25pt}
  \xymatrix{
    \frobs{.1\textwidth} = \frobx{.1\textwidth} = \frobz{.1\textwidth} & \spec{.1\textwidth} =
    \idone{.1\textwidth} & \extral{.1\textwidth} = \extrar{.1\textwidth} \\
    \textrm{(the Frobenius law)} & \textrm{(the special law)} &
    \textrm{(the extra law)}
  }
  \]
\end{definition}
While we write two equations for the Frobenius law, this is redundant: the
equality of any two of the expressions implies the equality of all three.  Note
that a monoid and comonoid obeying the Frobenius law is commutative if and only
if it is cocommutative.  Thus while a commutative and cocommutative Frobenius
monoid might more properly be called a bicommutative Frobenius monoid, there is
no ambiguity if we only say commutative.

The Frobenius law and the special law go back to Carboni and Walters, under the
names S=X law and the diamond=1 law respectively \cite{CW}. 


\begin{definition}
  A \define{special commutative Frobenius monoid} $(X,\mu,\eta,\delta,\epsilon)$
  in a monoidal category $(\mathcal C, \otimes)$ is an object $X$ of $\mathcal
  C$ together with maps 
\[
  \xymatrixrowsep{1pt}
  \xymatrix{
    \mult{.075\textwidth} & & \unit{.075\textwidth} & & 
    \comult{.075\textwidth} & & \counit{.075\textwidth} \\
    \mu\maps X\otimes X \to X & & \eta\maps I \to X & & 
    \delta\maps X\to X \otimes X & & \epsilon\maps X \to I
  }
\]
obeying the commutative monoid axioms
\[
  \xymatrixrowsep{1pt}
  \xymatrixcolsep{25pt}
  \xymatrix{
    \assocl{.1\textwidth} = \assocr{.1\textwidth} & \unitl{.1\textwidth} =
    \idone{.1\textwidth} & \commute{.1\textwidth} = \mult{.07\textwidth} \\
    \textrm{(associativity)} & \textrm{(unitality)} & \textrm{(commutativity)}
  }
\]
the cocommutative comonoid axioms
\[
  \xymatrixrowsep{1pt}
  \xymatrixcolsep{25pt}
  \xymatrix{
    \coassocl{.1\textwidth} = \coassocr{.1\textwidth} & \counitl{.1\textwidth} =
    \idone{.1\textwidth} & \cocommute{.1\textwidth} = \comult{.07\textwidth} \\
    \textrm{(coassociativity)} & \textrm{(counitality)} &
    \textrm{(cocommutativity)}
  }
\]
and the Frobenius and special axioms
  \[
  \xymatrixrowsep{1pt}
  \xymatrixcolsep{25pt}
  \xymatrix{
    \frobs{.1\textwidth} = \frobx{.1\textwidth} = \frobz{.1\textwidth} & \spec{.1\textwidth} =
    \idone{.1\textwidth} \\
    \textrm{(Frobenius)} & \textrm{(special)} 
  }
  \]
where $\swap{1em}$ is the braiding on $X \otimes X$.   
\end{definition}

\section{Hypergraph categories}

\subsection{Definition}
\begin{definition}
  A \define{hypergraph category} is a symmetric monoidal category in which each
  object $X$ is equipped with a special commutative Frobenius structure
  $(X,\mu_X,\delta_X,\eta_X,\epsilon_X)$ such that 
\[
  \begin{array}{cc}
    \mu_{X\otimes Y} = (\mu_X \otimes \mu_Y)\circ(1_X \otimes \sigma_{YX}\otimes
    1_Y) \qquad&
    \eta_{X\otimes Y} = \eta_X \otimes \eta_Y \\
    \delta_{X\otimes Y} = (1_X \otimes \sigma_{XY}\otimes 1_Y)\circ(\delta_X
    \otimes \delta_Y) \qquad&
    \epsilon_{X\otimes Y} = \epsilon_X \otimes \epsilon_Y.
  \end{array}
\]
A functor $(F,\varphi)$ of hypergraph categories, or \define{hypergraph
functor}, is a strong symmetric monoidal functor $(F,\varphi)$ that preserves
the hypergraph structure. More precisely, the latter condition means that given
an object $X$, the special commutative Frobenius structure on $FX$ must be 
\[
  (FX,\enspace F\mu_X \circ \varphi_{X,X},\enspace  \varphi^{-1} \circ F\delta_X,\enspace  F\eta_X \circ
\varphi_1,\enspace  \varphi_1 \circ \epsilon_X).
\]
\end{definition}

Write $\mathrm{HypCat}$ for the category of hypergraph categories. Hypergraph
natural transformations are just monoidal natural transformations?

Hypergraph categories have also been called well-supported compact closed
categories \cite{Ca}, dungeon categories \cite{Mo}, dgs-monoidal categories. 

Hypergraph categories are self-dual compact closed. We draw cups and caps for
shorthand.

\begin{proposition}
  Let $\mc H \to \mc H'$ be a fully faithful, essentially surjective, hypergraph
  functor. Then $\mc H$ and $\mc H'$ are equivalent as hypergraph categories.
\end{proposition}

We say that a monoidal category is objectwise-free if the object set is freely
generated by a some set under tensor products.

\begin{proposition}
  Coherence theorem: Every hypergraph category is equivalent as a hypergraph category to a
  objectwise-free strict hypergraph category.
\end{proposition}
\begin{proof}
  Mac Lane's celebrated coherence theorem states that every monoidal category is
  monoidally equivalent to a objectwise-free strict monoidal category. On each
  object of this strictification define the obvious Frobenius structure. Then
  we have a hypergraph category, and the strong monoidal equivalence is a
  hypergraph functor.
\end{proof}


\subsection{Graphical calculus}
\begin{proposition}
  Graphical calculus for hypergraph categories: two morphisms are equal if and
  only if their string diagrams are equivalent via Frobenius laws and
  topological deformation.
\end{proposition}
\begin{proof}
  Use coherence theorem and spider theorem.
\end{proof}

\subsection{History}
This terminology was introduced recently \cite{Ki}, in reference to the fact
that these special commutative Frobenius monoids provide precisely the structure
required to draw graphs with `hyperedges': wires connecting any number of
inputs to any number of outputs. Commutative special Frobenius monoids are also
known as commutative separable algebras \cite{RSW}, and hypergraph categories as
well-supported compact closed categories \cite{Ca}.


\subsection{Hypergraph categories are dagger compact categories} \label{subsec:dagger}

In a dagger category, the distinction between the source and target of a morphism is arbitrary, and can be reversed. This is true of electrical circuits: if we like we may treat the set of inputs as the set of outputs instead, and the set of outputs as the set of inputs.

Note that if an object $X$ is equipped with a Frobenius monoid structure then
the maps $\epsilon \circ \mu\colon  X \otimes X \longrightarrow 1$ and $\delta \circ
\eta\colon  1 \longrightarrow X \otimes X$ obey 
\[
  \big(1 \otimes (\epsilon \circ \mu)\big) \circ \big((\delta \circ \eta)
  \otimes 1\big) = 1_X = \big((\epsilon \circ \mu) \otimes 1\big) \circ \big(1
  \otimes (\delta \circ \eta)\big)\colon X \longrightarrow X.
\]
Thus if an object carries a Frobenius monoid it is also self-dual, and any
hypergraph category is a fortiori self-dual compact closed. Mapping each
morphism $f\colon  X \to Y$ to its dual morphism
\[
  \big((\epsilon_Y \circ \mu_Y) \otimes 1_X\big) \circ \big( 1_Y \otimes f
  \otimes 1_X \big) \circ \big(1_Y \otimes (\delta_X \circ \eta_X)\big)\colon  Y
  \longrightarrow X
\]
further equips each hypergraph category with a so-called dagger functor---an
involutive contravariant endofunctor that is the identity on objects---such that
the category is a dagger compact category. Dagger compact categories were first
introduced in the context of categorical quantum mechanics \cite{AC}, under the
name strongly compact closed category, and have been demonstrated to be a key
structure in diagrammatic reasoning and the logic of quantum mechanics.

Recall that a \define{dagger category} is a category equipped with an
involutive, contravariant endofunctor that is the identity on objects
\cite{AC,Se}. In other words, a \define{dagger structure} on a category $\mc C$
is a contravariant functor $(-)^\dagger\maps \mc C \to \mc C$ such that
$A^\dagger = A$ for all objects $A \in \ob \mc C$ and $(f^\dagger)^\dagger = f$
for all morphisms $f$ in $\mc C$.  A dagger category is then a category equipped
with a dagger structure.

When other structure is present, we prefer this dagger to be compatible with it.
We say that a morphism $f$ is \define{unitary} if its dagger provides it with an
inverse morphism. A \define{symmetric monoidal dagger category} is a symmetric
monoidal category equipped with a \define{symmetric monoidal dagger
structure}---that is, a dagger structure that coherently preserves the symmetric
monoidal structure. Concretely, this requires that the functor $(-)^\dagger\maps
\mc C \to \mc C^{\mathrm op}$ be symmetric monoidal, and that the associator,
unitors, and braiding of $\mc C$ be unitary. Furthermore, letting $L$ and $R$ be
dual objects of a symmetric monoidal dagger category, with monoidal unit $I$,
braiding $\s_{L,R}\maps L \ot R \to R \ot L$, and unit $\eta\maps I \to R
\otimes L$ and counit $\eps\maps L \otimes R \to I$, we say that $L$ and $R$ are
\define{dagger dual} if $\eta = \s \circ \eps^\dagger$.  A \define{dagger
compact category} is a symmetric monoidal dagger category in which every object
has a dagger dual, while a \define{dagger functor} is a functor $F$ between
dagger compact categories that preserves the dagger structures: $F((-)^\dagger)
= (F(-))^\dagger$.

Importantly for our applications, dagger compact categories come with a
graphical calculus, where each morphism is represented by a diagram such that
two diagrams are considered equal by the rules of this calculus if and only if
they are equal according to the defining laws of dagger compact categories
\cite{Se}. In brief, to set up our conventions, we represent a morphism $f\maps X_1
\otimes \dots \otimes X_n \to Y_1 \otimes \dots \otimes Y_m$ as a `downwards
flow chart':
\[
  \begin{tikzpicture}
	  \begin{pgfonlayer}{nodelayer}
		  \node [style=none] (0) at (-0.75, 0.25) {};
		  \node [style=none] (1) at (0.75, 0.25) {};
		  \node [style=none] (2) at (-0.75, -0.25) {};
		  \node [style=none] (3) at (0.75, -0.25) {};
		  \node [style=none] (4) at (0, 0.25) {};
		  \node [style=none] (5) at (0, -0.25) {};
		  \node [style=none] (6) at (0, -0) {$f$};
		  \node [style=none] (7) at (-0.5, 0.25) {};
		  \node [style=none] (8) at (0.5, 0.25) {};
		  \node [style=none] (9) at (-0.5, -0.25) {};
		  \node [style=none] (10) at (0.5, -0.25) {};
		  \node [style=none] (11) at (-0.5, 0.75) {};
		  \node [style=none] (12) at (0.5, 0.75) {};
		  \node [style=none] (13) at (-0.5, -0.75) {};
		  \node [style=none] (14) at (0.5, -0.75) {};
		  \node [style=none] (15) at (0, 0.5) {$\dots$};
		  \node [style=none] (16) at (0, -0.5) {$\dots$};
		  \node [style=none] (17) at (-0.5, 1) {$X_1$};
		  \node [style=none] (18) at (0.5, 1) {$X_n$};
		  \node [style=none] (19) at (-0.5, -1) {$Y_1$};
		  \node [style=none] (20) at (0.5, -1) {$Y_m$};
	  \end{pgfonlayer}
	  \begin{pgfonlayer}{edgelayer}
		  \draw (0.center) to (1.center);
		  \draw (1.center) to (3.center);
		  \draw (3.center) to (2.center);
		  \draw (2.center) to (0.center);
		  \draw (7.center) to (11.center);
		  \draw (8.center) to (12.center);
		  \draw (9.center) to (13.center);
		  \draw (10.center) to (14.center);
	  \end{pgfonlayer}
  \end{tikzpicture}
\]
Composition is then represented by connecting the lines (or wires) representing
the codomain of one morphism with the domain of the another placed below it,
the monoidal product of two morphisms is represented by their side-by-side
juxtaposition, the swap map by crossing two wires, the compact structure by
bending a wire 180 degrees, and the dagger functor by flipping a diagram in the
horizontal axis. We believe these operations on diagrams---placing diagrams on
the same page, rearranging their wires/flipping them, and then connecting their
wires to form a larger diagram---represents the collection of operations used
for reasoning with circuit diagrams, and hence that dagger compact categories
are an appropriate structure for the formalization of such.
\section{Cospan categories}

Recall that a \define{cospan} from $X$ to $Y$ in a category $\mathcal C$ is an
object $N$ in $\mathcal C$ with a pair of morphisms $(i\colon  X \to N$, $o\colon  Y \to
N$):
\[
  \xymatrix{
    & N \\
    X \ar[ur]^{i} && Y. \ar[ul]_{o}
  }
\]
We shall refer to $X$ and $Y$ as the \define{feet}, and $N$ as the
\define{apex} of the cospan.  Cospans may be composed using the pushout from
the common foot, when such a pushout exists: given cospans $X
\stackrel{i_X}{\longrightarrow} N \stackrel{o_Y}{\longleftarrow} Y$ from $X$ to
$Y$ and $Y \stackrel{i_Y}{\longrightarrow} M \stackrel{o_Z}{\longleftarrow} Z$
from $Y$ to $Z$, their composite cospan is $X \stackrel{j \circ
i_X}{\longrightarrow} P \stackrel{j'\circ i_Z}{\longleftarrow} Z$, where $P$,
$(j\colon  N \to P)$, and $(j'\colon  M \to P)$ form the top half of the pushout square
\[
  \xymatrix{
    && P \\
    & N \ar[ur]^j && M \ar[ul]_{j'} \\
    X \ar[ur]^{i_X} && Y \ar[ul]_{o_Y} \ar[ur]^{i_Y} && Z. \ar[ul]_{o_Z}
  }
\]
A \define{map of cospans} is a morphism $n\colon  N \to N'$ in $\mathcal C$ between
the apices of two cospans $X \stackrel{i}{\longrightarrow} N
\stackrel{o}{\longleftarrow} Y$ and $X \stackrel{i'}{\longrightarrow} N'
\stackrel{o'}{\longleftarrow} Y$ with the same feet, such that both triangles 
\[
  \xymatrix{
    & N \ar[dd]^n  \\
    X \ar[ur]^{i} \ar[dr]_{i'} && Y \ar[ul]_{o} \ar[dl]^{o'}\\
    & N'
  }
\]
commute. Given a category $\mathcal C$ with pushouts, we may define a category
$\mathrm{Cospan}(\mathcal C)$ with objects the objects of $\mathcal C$ and
morphisms isomorphism classes of cospans \cite{Be}. We will often abuse our
terminology and refer to cospans themselves as morphisms in some cospan
category $\mathrm{Cospan}(\mathcal C)$; we of course refer instead to the
isomorphism class of the said cospan.

\begin{example}
  A central example of a hypergraph category is the category
  $\mathrm{Cospan(\mathcal C)}$ of cospans in any category $\mathcal C$ with
  finite colimits. We will later see that decorated cospan categories are a
  generalisation of such categories, and each inherits a hypergraph structure
  from such. 

  First, $\mathrm{Cospan(\mathcal C)}$ inherits a symmetric monoidal structure
  from $\mathcal C$. We call a subcategory $\mathcal C$ of a category $\mathcal
  D$ \define{wide} if $\mathcal C$ contains all objects of $\mathcal D$, and
  call a functor that is faithful and bijective-on-objects a \define{wide
  embedding}. Note then that we have a wide embedding
  \[
    \mathcal C \hooklongrightarrow \mathrm{Cospan(\mathcal C)}
  \]
  that takes each object of $\mathcal C$ to itself as an object of
  $\mathrm{Cospan(\mathcal C)}$, and each morphism $f\colon  X \to Y$ in $\mathcal C$
  to the cospan
  \[
    \xymatrix{
      & Y \\
      X \ar[ur]^{f} && Y, \ar@{=}[ul]
    }
  \]
  where the extended `equals' sign denotes an identity morphism. This allows us
  to view $\mathcal C$ as a wide subcategory of $\mathrm{Cospan(\mathcal C)}$.

  Now as $\mathcal C$ has finite colimits, it can be given a symmetric monoidal
  structure with the coproduct the monoidal product; we write this monoidal
  category $(\mathcal C,+)$, and write $\varnothing$ for the initial object, the
  monoidal unit of this category. Then $\mathrm{Cospan(\mathcal C)}$ inherits
  the same symmetric monoidal structure: since the monoidal product $+\colon \mathcal
  C \times \mathcal C \to \mathcal C$ is left adjoint to the diagram functor, it
  preserves colimits, and so extends to a functor $+\colon
  \mathrm{Cospan(\mathcal C)} \times \mathrm{Cospan(\mathcal C)} \to
  \mathrm{Cospan(\mathcal C)}$. The remainder of the monoidal structure is
  inherited because $\mathcal C$ is a wide subcategory of
  $\mathrm{Cospan(\mathcal C)}$.

  Next, the Frobenius structure comes from copairings of identity morphisms. We
  call cospans 
  \[
    \xymatrix{
      & N \\
      X \ar[ur]^{i} && Y \ar[ul]_{o}
    }
    \qquad \xymatrix@R=8pt{\\\textrm{and}} \qquad 
    \xymatrix{
      & N \\
      Y \ar[ur]^{o} && X \ar[ul]_{i}
    }
  \]
  that are reflections of each other \define{opposite} cospans. Given any object
  $X$ in $\mathcal C$, the copairing $[1_X,1_X]\colon  X + X \to X$ of two identity
  maps on $X$, together with the unique map $!\colon  \varnothing \to X$ from the
  initial object to $X$, define a monoid structure on $X$. Considering these
  maps as morphisms in $\mathrm{Cospan(\mathcal C)}$, we may take them together
  with their opposites to give a special commutative Frobenius structure on $X$.
  In this way we consider each category $\mathrm{Cospan(\mathcal C)}$ a
  hypergraph category.

  It is a simple computation to check that the resulting dagger functor simply
  takes a cospan $X \stackrel{i}{\longrightarrow} N \stackrel{o}{\longleftarrow}
  Y$ to its opposite cospan $Y \stackrel{o}{\longrightarrow} N
  \stackrel{i}{\longleftarrow} X$.
\end{example}

\begin{proposition}
  The monoidal product in a hypergraph category is a coproduct for something if and only if
  every morphism is a Frobenius monoid homomorphism. Perhaps, hypergraph
  category is a corelation category iff Frob monoid homs.
\end{proposition}

Clearly not true for circuits for example.


We give a fundamental example of hypergraph categories.

  Hypergraph categories are closely related to cospans. The free hypergraph
  category on a single object in the category of cospans in the category of
  finite sets. SpivakVagner?
  
  Let $\mc C$ be a category with finite colimits.
Recall that a \define{cospan} $X \stackrel{i}{\longrightarrow} N
\stackrel{o}{\longleftarrow} Y$  from $X$ to $Y$ in $\mathcal C$ is a pair of
morphisms with common codomain. We refer to $X$ and $Y$ as the \define{feet},
and $N$ as the \define{apex}.  Given two cospans $X
\stackrel{i}{\longrightarrow} N \stackrel{o}{\longleftarrow} Y$ and $X
\stackrel{i'}{\longrightarrow} N' \stackrel{o'}{\longleftarrow} Y$ with the same
feet, a \define{map of cospans} is a morphism $n\colon  N \to N'$ in $\mathcal
C$ between the apices such that
\[
  \xymatrix{
    & N \ar[dd]^n  \\
    X \ar[ur]^{i} \ar[dr]_{i'} && Y \ar[ul]_{o} \ar[dl]^{o'}\\
    & N'
  }
\]
commutes.

Cospans may be composed, up to isomorphism, using the pushout from the common
foot: given cospans $X \stackrel{i_X}{\longrightarrow} N
\stackrel{o_Y}{\longleftarrow} Y$ and $Y \stackrel{i_Y}{\longrightarrow} M
\stackrel{o_Z}{\longleftarrow} Z$, their composite cospan is $X \stackrel{j_N
  \circ i_X}{\longrightarrow} N+_YM \stackrel{j_M\circ i_Z}{\longleftarrow} Z$,
  where 
\[
  \xymatrix{
    && N+_YM \\
    & N \ar[ur]^{j_N} && M \ar[ul]_{j_M} \\
    \quad X \quad \ar[ur]^{i_X} && Y \ar[ul]_{o_Y} \ar[ur]^{i_Y} && \quad Z \quad \ar[ul]_{o_Z}
  }
\]
is a pushout square. 

We consider any category $\mathcal C$ as a symmetric monoidal category with
monoidal product given by the coproduct, written $+$, and braiding given by the
maps $A+B \to B+A$ by copairing identity maps. 

Given maps $f \maps A \to C$, $g \maps B \to C$ with common codomain, the
universal property of the coproduct gives a unique map $\maps A+B \to C$. We
call this the \define{copairing} of $f$ and $g$, and write it $[f,g]$. 

\begin{definition} \label{thm.cospanwelldef}
  Let $\mc C$ be a category with finite colimits. We define the hypergraph
  category $\mathrm{Cospan}(\mathcal C)$ to comprise:
  
  \begin{tabular}{ r p{.6\textwidth}}
    \textbf{objects} & the objects of $\mathcal C$ \\ 
    \textbf{morphisms} & isomorphism classes of cospans in
    $\mathcal C$\\ 
  \textbf{composition} & given by pushout \\
  \textbf{tensor product} & the coproduct in $\mathcal C$. \\
  \textbf{coherence maps} & inherited from $(\mc C,+)$ $\sigma_{X,Y} = [\iota_Y,\iota_X] \maps X+Y
      \to Y+X$\\
  \textbf{hypergraph maps} & $\mu_X = [1_X,1_X]$, $\eta_X = !$,
      $\delta_X = \mu_X^\opp$, $\epsilon_X = \eta_X^\opp$.
  \end{tabular}
\end{definition}

\begin{notation}
Note that we have committed the following common abuses of notation in the above
definition: by referring to cospans themselves as
  morphisms in some cospan category $\mathrm{Cospan}(\mathcal C)$; this of
  course refers instead to the isomorphism class of the said cospan.

  Given $f \maps X \to Y$ in $\mc C$, we also abuse notation by writing $f \in
  \mathrm{Cospan}(\mc C)$ for the cospan $X \stackrel{f}\to Y
  \stackrel{1_Y}\leftarrow Y$, and $f^\opp$ for the cospan $Y \stackrel{1_Y}\to
  Y \stackrel{f}\leftarrow X$.
\end{notation}

FinCocompleteCat faithfully embeds into HyperCat. ie any monoidal category of
cospans has a hypergraph structure inherited from the identity morphisms.


Walters: cospan graph is the generic special commutative Frobenius monoid.

Later, also Vagner Spivak Schultz: hypergraph categories are algebras of cospan.

