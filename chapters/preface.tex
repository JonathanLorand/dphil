\phantomsection
\addcontentsline{toc}{chapter}{Preface}
\chapter*{Preface}

This is a thesis in the mathematical sciences, with emphasis on the mathematics.
But before we get to the category theory, I want to say a few words about the
scientific tradition in which this thesis is situated.

Mathematics is the language of science. Twinned so intimately with physics, over
the past centuries mathematics has become superb---indeed, unreasonably
effective---language for understanding planets moving in space, particles in a
vacuum, the structure of spacetime, and so on.  Yet, while Wigner speaks of the
unreasonble effectiveness of mathematics in the natural sciences \cite{Wig60},
equally eminent mathematicians, not least Gelfand, speak of the unreasonable
\emph{ineffectiveness} of mathematics in biology and related fields. Why such a
difference?

A contrast between physics and biology is that while physical systems can often
be studied in isolation---the particle in a vacuum---, biological systems are
necessarily situated in their environment. A heart belongs in a body, an ant in
a colony. One of the first to draw attention to this contrast was Ludwig on
Bertalanffy, biologist and founder of general systems theory, who articulated
the difference as one between closed and open systems: 
\begin{quote}
  Conventional physics deals only with closed systems, i.e. systems which are
  considered to be isolated from their environment. \dots\ However, we find
  systems which by their very nature and definition are not closed systems.
  Every living organism is essentially an open system. It maintains itself in a
  continuous inflow and outflow, a building up and breaking down of components,
  never being, so long as it is alive, in a state of chemical and thermodynamic
  equilibrium but maintained in a so-called steady state which is distinct from
  the latter \cite{Ber68}.
\end{quote}
While von Bertalanffy's general systems theory is still approaching its
generalist ambition, his philosophy has had great impact in his home field of
biology, leading to the modern field of systems biology. Half a century later,
Dennis Noble, another great pioneer of systems biology and the originator of the
first mathematical model of a working heart, describes the shift as one from
reduction to integration.
\begin{quote}
  Systems biology\dots\ is about putting together rather than taking apart,
  integration rather than reduction. It requires that we develop ways of
  thinking about integration that are as rigorous as our reductionist
  programmes, but different \cite{Nob06}.
%\dots\ It means changing our philosophy, in the full sense of the term 
\end{quote}
In this thesis we develop rigorous ways of thinking about integration or, as we
refer to it, interconnection.

Interconnection and openness are tightly related. Indeed, openness implies that
a system may be interconnected with its environment. But what is an environment
but comprised of other systems? Thus the study of open systems becomes the study
of how a system changes under interconnection with other systems.

To model this, we must begin by creating language to describe the
interconnection of systems. While reductionism hopes that phenomena can be
explained by reducing them to ``elementary units investigable independently of
each other'' \cite{Ber68}, this philosophy of integration introduces as an
additional and equal priority the investigation of the way these units are
interconnected. As such, this thesis is predicated on the hope that the meaning
of an expression in our new language is determined by the meanings of its
constituent expressions together with the syntactic rules combining them. This
is known as the principle of compositionality. 

Also commonly known as Frege's principle, the principle of compositionality and
its precise meaning and applicability both dates back to Ancient Greek and Vedic
philosophy, and is still the subject of active research today
\cite{Jan86,Sza13}. More recently, through the work of Montague \cite{Mon70} in
natural language semantics and Strachey and Scott \cite{SS71} in programming
languages semantics, the principle of compositionality has found formal
expression as the dictum that the interpretation of a language should be given
by a homomorphism between an algebra of syntactic representations and an algebra
of semantic objects. We too shall follow this route.

The question then arises: what do we mean by algebra? This mathematical question
leads us back to our scientific objectives: what do we mean by system? Here we
must narrow, or at least define, our scope. We give some examples. The
investigations of thesis began with electrical circuits and their diagrams, and
we will devote significant time to exploring their compositional formulation. We
discussed biological systems above, and our notion of system includes these,
modelled say in the form of chemical reaction networks or Markov processes, or
the compartmental models of epidemiology, population biology, and ecology. From
computer science, we consider Petri nets, automata, logic circuits, and similar.
More abstractly, our notion of system encompasses matrices and systems of
differential equations. 

Drawing together these notions of system are well-developed diagrammatic
representations based on network diagrams---that is, topological graphs. We call
these network-style diagrammatic languages. In the abstract, by system we shall
simply mean that which can be represented by box with a collection of terminals,
perhaps of different types, through which it interfaces with the surroundings.
Concretely, one might envision a circuit diagram with terminals, such as
\[
\begin{tikzpicture}[circuit ee IEC, set resistor graphic=var resistor IEC graphic]
\node[contact] (I1) at (0,0) {};
\node[contact] (O1) at (3,0) {};
\node (input) at (-2,0) {\small{\textsf{terminal}}};
\node (output) at (5,0) {\small{\textsf{terminal}}};
\draw (I1) 	to [resistor] node [above]
{$1\Omega$} (O1);
\path[color=gray, very thick, shorten >=10pt, ->, >=stealth] (input) edge (I1);	
\path[color=gray, very thick, shorten >=10pt, ->, >=stealth] (output) edge (O1);
\end{tikzpicture}
\]
or
\[
\begin{tikzpicture}[circuit ee IEC, set resistor graphic=var resistor IEC graphic]
\node[contact] (I1) at (0,2) {};
\node[contact] (I2) at (0,0) {};
\coordinate (int1) at (2.83,1) {};
\coordinate (int2) at (5.83,1) {};
\node[contact] (O1) at (8.66,2) {};
\node[contact] (O2) at (8.66,0) {};
\node (input) at (-2,1) {\small{\textsf{terminals}}};
\node (output) at (10.66,1) {\small{\textsf{terminals}}};
\draw (I1) 	to [resistor] node [label={[label distance=2pt]85:{$1\Omega$}}] {} (int1);
\draw (I2)	to [resistor] node [label={[label distance=2pt]275:{$1\Omega$}}] {} (int1)
				to [resistor] node [label={[label distance=3pt]90:{$2\Omega$}}] {} (int2);
\draw (int2) 	to [resistor] node [label={[label distance=2pt]95:{$1\Omega$}}] {} (O1);
\draw (int2)		to [resistor] node [label={[label distance=2pt]265:{$3\Omega$}}] {} (O2);
\path[color=gray, very thick, shorten >=10pt, ->, >=stealth, bend left] (input) edge (I1);		\path[color=gray, very thick, shorten >=10pt, ->, >=stealth, bend right] (input) edge (I2);		
\path[color=gray, very thick, shorten >=10pt, ->, >=stealth, bend right] (output) edge (O1);
\path[color=gray, very thick, shorten >=10pt, ->, >=stealth, bend left] (output) edge (O2);
\end{tikzpicture}
\]
The algebraic structure of interconnection is then simply the structure that
results from the ability to connect terminals of one system with terminals of
another. This graphical approach motivates our language of interconnection:
indeed, these diagrams will be the expressions of our language.

We claim that the existence of a network-style diagrammatic language to
represent a system implies that interconnection is inherently important in
understanding the system. Yet, while each of these example notions of system are
well-studied in and of themselves, their compositional, or algebraic, structure
has received scant attention. In this thesis, we study an algebraic structure
called a hypergraph category, and argue that this is the relevant algebraic
structure for modelling interconnection of open systems. 

Given these pre-existing diagrammatic formalisms and our visual intuition,
constructing algebras of syntactic representations is thus rather
straightforward. The semantics and their algebraic structure are more subtle. 

In some sense our semantics is already given to us too: in studying these
systems as closed systems, scientists have already formalised the meaning of
these diagrams. But we have shifted from a closed perspective to an open one,
and we need our semantics to also account for points of interconnection.

Taking inspiration from Willems' behavioural approach \cite{Wi} and Deutsch's
constructor theory \cite{Deu}, in this thesis I advocate the following position.
First, at each terminal of an open system we may make measurements appropriate
to the type of terminal. Given a collection of terminals, the \emph{universum}
is then the set of all possible measurement outcomes. Each open system has a
collection of terminals, and hence a universum. The semantics of an open system
is the subset of measurement outcomes on the terminals that are permitted by the
system. This is known as the \emph{behaviour} of the system.

For example, consider a resistor of resistance $r$. This has two terminals---the
two ends of the resistor---and at each terminal, we may measure the potential
and the current. Thus the universum of this system is the set
$\mathbb{R}\oplus\mathbb{R}\oplus\mathbb{R}\oplus\mathbb{R}$, where the summands
represent respectively the potentials and currents at each of the two terminals.
The resistor is governed by Kirchhoff's current law, or conservation of charge,
and Ohm's law.  Conservation of charge states that the current flowing into one
terminal must equal the current flowing out of the other terminal, while Ohm's
law states that this current will be proportional to the potential difference,
with constant of proportionality $1/r$. Thus the behaviour of the resistor is
the set 
\[
  \big\{\big(\phi_1,\phi_2,
    -\tfrac1r(\phi_2-\phi_1),\tfrac1r(\phi_2-\phi_1)\big)\,\big\vert\,
    \phi_1,\phi_2 \in \mathbb{R}\big\}.
\]
Note that in this perspective a law such as Ohm's law is a mechanism for
partitioning \emph{behaviours} into possible and impossible
behaviours.\footnote{That is, given a universum $\mathcal U$ of trajectories, a
  behaviour of a system is an element of the power set $\mathcal P(\mathcal
U)$ representing all possible measurements of this system, and a law or principle
is an element of $\mathcal P(\mathcal P(\mathcal U))$ representing all possible
behaviours of a class of systems.}

Interconnection of terminals then asserts the identification of the variables at
the identified terminals. Fixing some notion of open system and subsequently an
algebra of syntactic representations for these systems, our approach, based on
the principle of compositionality, requires this to define an algebra of
semantic objects and a homomorphism from syntax to semantics. The first part of
this thesis develops the mathematical tools necessary to pursue this vision for
modelling open systems and their interconnection. 

The next goal is to demonstrate the efficacy of this philosophy in applications.
At core, this work is done in the faith that the right language allows deeper
insight into the underlying structure. Indeed, after setting up such a language
for open systems there are many questions to be asked: Can we find a sound and
complete logic for determining when two syntactic expressions have the same
semantics? Suppose we have systems have some property, like being controllable.
In what ways can we interconnect controllable systems so that the combined
system is also controllable? Can we compute the semantics of a large system
quicker by computing the semantics of subsystems and then composing them?  If I
want a given system to acheive a specified trajectory, can we interconnect
another system to make it do so? How do two different notions of system, such as
circuit diagrams and signal flow graphs, relate to each other? Can we find
homomorphisms between their syntactic and semantic algebras? In the second part
of this thesis we explore some applications in depth, providing answers to
questions of the above sort.


\subsection*{Organisation of this thesis}
%\addcontentsline{toc}{section}{Organisation of this thesis}

This thesis is divided into two parts. Part \ref{part.maths}, comprising
Chapters \ref{ch.hypcats} to \ref{ch.deccorels}, focusses on mathematical
foundations. In it we develop the theory of hypergraph categories and a powerful
tool for constructing and manipulating them: decorated corelations. Part
\ref{part.apps}, comprising Chapters \ref{ch.sigflow} to \ref{ch.further}, then
discusses applications of this theory to examples of open systems.

The central refrain of this thesis is that the syntax and semantics of
network-style diagrammatic languages can be modelled by hypergraph categories.
These are introduced in Chapter \ref{ch.hypcats}. Hypergraph categories are
symmetric monoidal categories in which every object is equipped with the
structure of a special commutative Frobenius monoid in a way compatible with the
monoidal product. As we will rely heavily on properties of monoidal categories,
their functors, and their graphical calculus, we begin with a whirlwind review
of these ideas. We then provide a definition of hypergraph categories and their
functors, a strictification theorem, and an important example: the category of
cospans in a category with finite colimits.

Cospans are pairs of morphisms $X \to N \leftarrow Y$ with a common codomain.
In Chapter \ref{ch.deccospans} we introduce the idea of a decorated cospan, which
equips the apex $N$ with extra structure. Our motivating example is cospans of
finite sets decorated by graphs, as in the picture
\begin{center}
  \begin{tikzpicture}[auto,scale=2.15]
    \node[circle,draw,inner sep=1pt,fill=gray,color=gray]         (x) at (-1.4,-.43) {};
    \node at (-1.4,-.9) {$X$};
    \node[circle,draw,inner sep=1pt,fill]         (A) at (0,0) {};
    \node[circle,draw,inner sep=1pt,fill]         (B) at (1,0) {};
    \node[circle,draw,inner sep=1pt,fill]         (C) at (0.5,-.86) {};
    \node[circle,draw,inner sep=1pt,fill=gray,color=gray]         (y1) at (2.4,-.25) {};
    \node[circle,draw,inner sep=1pt,fill=gray,color=gray]         (y2) at (2.4,-.61) {};
    \node at (2.4,-.9) {$Y$};
    \path (B) edge  [bend right,->-] node[above] {0.2} (A);
    \path (A) edge  [bend right,->-] node[below] {1.3} (B);
    \path (A) edge  [->-] node[left] {0.8} (C);
    \path (C) edge  [->-] node[right] {2.0} (B);
    \path[color=gray, very thick, shorten >=10pt, shorten <=5pt, ->, >=stealth] (x) edge (A);
    \path[color=gray, very thick, shorten >=10pt, shorten <=5pt, ->, >=stealth] (y1) edge (B);
    \path[color=gray, very thick, shorten >=10pt, shorten <=5pt, ->, >=stealth] (y2) edge (B);
  \end{tikzpicture}
\end{center}
Here graphs are a proxy for expressions in a network-style diagrammatic
language. To give a bit more formal detail, let $\mathcal C$ be a category with
finite colimits, writing its coproduct $+$, and let $(\mathcal D, \otimes)$ be a
braided monoidal category. Decorated cospans provide a method of producing a
hypergraph category from a lax braided monoidal functor $F\colon (\mathcal C,+)
\to (\mathcal D, \otimes)$. The objects of these categories are simply the
objects of $\mathcal C$, while the morphisms are pairs comprising a cospan $X
\rightarrow N \leftarrow Y$ in $\mathcal C$ together with an element $I \to FN$
in $\mathcal D$---the so-called decoration. We will also describe how to
construct hypergraph functors between decorated cospan categories. In
particular, this provides a useful tool for constructing a hypergraph category
that captures the syntax of a network-style diagrammatic language.

Having developed a method to construct a category where the morphisms are
expressions in a diagrammatic language, we turn our attention to categories of
semantics. This leads us to the notion of a corelation, to which we devote
Chapter \ref{ch.corelations}. Given a factorisation system $(\mc E,\mc M)$ on a
category $\mc C$, we define a corelation to be a cospan $X \to N \leftarrow Y$
such that the copairing of the two maps, a map $X+Y \to N$, is a morphism in
$\mc E$. Factorising maps $X+Y \to N$ using the factorisation system leads to a
notion of equivalence on cospans, and this helps us describe when two diagrams
are equivalent. Like cospans, corelations form hypergraph categories.

In Chapter \ref{ch.deccorels} we decorate corelations. Like decorated cospans,
decorated corelations are corelations together with some additional structure on
the apex. We again use a lax braided monoidal functor to specify the sorts of
extra structure allowed. Moreover, decorated corelations too form the morphisms
of a hypergraph category. The culmination of our theoretical work is to show
that every hypergraph category and every hypergraph functor can be constructed
using decorated corelations. This implies that we can use decorated corelations
to construct a semantic hypergraph category for any network-style diagrammatic
language, as well as a hypergraph functor from its syntactic category that
interprets each diagram. We also discuss how the intuitions behind decorated
corelations guide construction of these categories and functors.

%The key of decorated corelations is that hypergraph structure
%requires some sort of uniformity in the composition rule, and it is easier to
%work by acknowledging this structure, defining them as algebras over some
%theory. 
%
%Hypergraph categories are really just $\Set$-valued lax symmetric monoidal
%functors: and these, being simply data structures, are often simpler to work
%with.


Having developed these theoretical tools, in the second part we turn to
demonstrating that the have useful application. Chapter \ref{ch.sigflow}
uses corelations to formalise signal flow diagrams representing linear
time-invariant discrete dynamical systems as morphisms in a category.
Our main result gives an intuitive sound and fully complete equational theory
for reasoning about these linear time-invariant systems. Using this framework,
we derive a novel structural characterisation of controllability, and
consequently provide a methodology for analysing controllability of networked
and interconnected systems.

Chapter \ref{ch.circuits} studies passive linear networks. Passive linear
networks are used in a wide variety of engineering applications, but the best
studied are electrical circuits made of resistors, inductors and capacitors. The
goal is to construct what we call the black box functor, a hypergraph functor
from a category of open circuit diagrams to a category of behaviours of
circuits. We construct the former as a decorated cospan category, with each morphism
a cospan of finite sets decorated by a circuit diagram on the apex. In this
category, composition describes the process of attaching the outputs of one
circuit to the inputs of another. The behaviour of a circuit is the relation it
imposes between currents and potentials at their terminals. The space of these
currents and potentials naturally has the structure of a symplectic vector
space, and the relation imposed by a circuit is a Lagrangian linear relation.
Thus, the black box functor goes from our category of circuits to the category
of symplectic vector spaces and Lagrangian linear relations. Decorated
corelations provide a critical tool for constructing these hypergraph categories
and the black box functor.

Finally, in Chapter \ref{ch.further} we mention two further research directions.
The first is the idea of a bound colimit, which aims to describe why epi-mono
factorisation systems are useful for constructing corelation categories of
semantics for open systems. The second research direction pertains to
applications of the black box functor for passive linear networks, discussing
the work of Jekel on the inverse problem for electric circuits \cite{Jek} and
the work of Baez, Fong, and Pollard on open Markov processes \cite{BFP,
Pol16}.

\subsection*{Related work}
%\addcontentsline{toc}{section}{Related work}

The work here is underpinned not just by philosophical precedent, but by a rich
tradition in mathematics, physics, and computer science. Here we make some
remarks on work on our broad theme of categorical network theory; more specific
references are included in each chapter. 

With its emphasis on composition, category theory is an attractive framework for
modelling open systems, and the present work is not the first attempt at develop
this idea. Notably from as early as the 1960s Goguen and Rosen both led efforts,
Goguen from a computer science perspective \cite{Go}, while Rosen from
biology \cite{Ros12}. Goguen in particular promoted the idea, which we take up,
that composition of systems should be modelled by colimits \cite{Gog91}. This
manifests in our emphasis on cospans.

Indeed, cospans are well known as a formalism for making entities with an
arbitrarily designated `input end' and `output end' into the morphisms of a
category. This has roots in topological quantum field theory, where a particular
type of cospan known as a `cobordism' is used to describe pieces of spacetime
\cite{BL,BaezStay}. The general idea of using functors to associate algebraic
semantics to topological diagrams has also long been a technique associated with
topological quantum field theory, dating back to \cite{At}.  

Developed around the same time, the work of Joyal and Street showing the tight
connection between string diagrams and monoidal categories \cite{JS91,JS93} is
also a critical aspect of the work here. This rigorous link inspires our use of
hypergraph categories---a type of monoidal category---in modelling network-style
diagrammatic languages.

The work of Walters, Sabadini, and collaborators also makes use of cospans to
model interconnection of systems, including electric circuits, as well as
automata, Petri nets, transition systems, and Markov processes \cite{KSW,KSW2,
RSW05,RSW08,ASW}. Indeed, the definition of a hypergraph category is due to
Walters and Carboni \cite{Car91}. A difference and original contribution is our
present use of decorations. One paper the deserves particular mention is that of
Rosebrugh, Sabadini, and Walters on calculating colimits compositionally
\cite{RSW08}, which develops the beginnings of some of the present ideas on
corelations. Their discussion of automata and the proof of Kleene's theorem is
reminscent of our proof of the functoriality of the black box functor, and the
precise relationship deserves to be pinned down.

At the present time, the work of Bonchi, Soboc\'inski, and Zanasi on signal flow
diagrams and graphical linear algebra \cite{BSZ,BSZ2,BSZ3,Za} has heavily
influenced the work here. Indeed, Chapter \ref{ch.sigflow} is built upon the
foundation they created.

Spivak and collaborators use operads to study networked systems
diagrammatically. Although a more general framework, capable of studying
diagrammatic languages with more flexible syntax than that of hypergraph
categories, Spivak studies an operad of so-called wiring diagrams built from
cospans of labelled finite sets in a number of papers, including \cite{VSL,Sp,SSR}.

Finally, this thesis fits into the categorical network theory programme led by
Baez, together with Erbele, Pollard, Courser and others. In particular, the
papers \cite{BE,Erb16,BFP,Pol16,Cou16} on signal flow diagrams, open Markov
processes, and decorated cospans have been developed alongside and influenced
the work here.

\subsection*{Statement of work}
%\addcontentsline{toc}{section}{Statement of work}

The Examination Schools make the following request:
\begin{quote}
\emph{Where some part of the thesis is not solely the work of the candidate or
has been carried out in collaboration with one or more persons, the candidate
shall submit a clear statement of the extent of his or her own contribution.}
\end{quote}
I address this now. 

The first four chapters, on hypergraph categories and decorated corelations, are
my own work. (Chapter \ref{ch.deccospans} has been previously published as
\cite{Fon15}.) The applications chapters were developed with collaborators. 

Chapter \ref{ch.sigflow} arises from a weekly seminar with Paolo Rapisarda and
Pawe\l\ Soboc\'inski at Southampton in the Spring of 2015. The text is a minor
adaptation of that in the paper \cite{FRS16}. For that paper I developed the
corelation formalism, providing a first draft. Pawe\l\ provided much expertise
in signal flow graphs, significantly revising the text and contributing the
section on operational semantics. Paolo contributed comparisons to classical
methods in control theory.  A number of anonymous referees contributed helpful
and detailed comments.

Chapter \ref{ch.circuits} is joint work with John Baez; the majority of the text
is taken from our paper \cite{BF}. For that paper John supplied writing on
Dirichlet forms and the principle of minimum power that became the second
section of Chapter \ref{ch.circuits}, as well as parts of the next two sections.
I produced a first draft of the rest of the paper. We collaboratively revised
the text for publication.

\subsection*{Acknowledgements}
%\addcontentsline{toc}{section}{Acknowledgements}
This work could not have been completed alone, and I am grateful for the
assistance of so many people, indeed far more than I list here. Foremost
deserving of mention are my supervisors John Baez, Bob Coecke, and Rob Ghrist.
John's influence on this work cannot be understated; he has been tireless in
patiently explaining new mathematics to me, in sharing his vision for this great
project, and in supporting, reading, promoting, and encouraging my work. Bob is
the reason I started this work in the first place, and with Rob they have both
provided fantastic research environments, a generous amount of freedom to pursue
my passions and, most importantly, encouragement and faith.

Jamie Vicary, too, has always been available for advice and encouragement. My
coauthors, including John, but also Brandon Coya, Hugo Nava-Kopp, Blake Pollard,
Paolo Rapisarda, Pawe\l\ Soboc\'inski, have been teachers in the process.
And valuable comments and conversations on the present work have come, not only
from all of the above, but also from Rashmi Kumar, Omar Camarena, Dan Marsden,
Bernhard Reinke, Jason Erbele, Stefano Gogioso.

I am grateful for the generous support of a number of institutions.  This work
was supported by the Clarendon Fund, Hertford College, and the Queen Elizabeth
Scholarships, Oxford, and completed at the University of Oxford, the Centre for
Quantum Technologies, Singapore, and the University of Pennsylvania.

Finally, I thank my partner, Stephanie, and my family---Mum, Dad, Justin,
Calvin, Tania---for their love and patience.

