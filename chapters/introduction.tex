\phantomsection
\addcontentsline{toc}{chapter}{Introduction}
\chapter*{Introduction}

This is a thesis in the mathematical sciences, with emphasis on the mathematics.
But before we get to the category theory, I want to say a few words about the
scientific tradition in which this thesis is situated.

Mathematics is the language of science. Twinned so intimately with physics, over
the past centuries mathematics has become superb---indeed, unreasonably
effective---language for understanding planets moving in space, particles in a
vacuum, the structure of spacetime, and so on.  Yet, while Wigner speaks of the
unreasonble effectiveness of mathematics in the natural sciences, equally
eminent mathematicians, not least Gelfand, speak of the unreasonable
\emph{ineffectiveness} of mathematics in biology and related fields. Why such a
difference?

A contrast between physics and biology is that while physical systems can often
be studied in isolation---the particle in a vacuum---, biological systems are
necessarily situated in their environment. A heart belongs in a body, an ant in
a colony. One of the first to draw attention to this contrast was Ludwig on
Bertalanffy, biologist and founder of general systems theory, who articulated
the difference as one between closed and open systems: 
\begin{quote}
  Conventional physics deals only with closed systems, i.e. systems which are
  considered to be isolated from their environment. \dots\ However, we find
  systems which by their very nature and definition are not closed systems.
  Every living organism is essentially an open system. It maintains itself in a
  continuous inflow and outflow, a building up and breaking down of components,
  never being, so long as it is alive, in a state of chemical and thermodynamic
  equilibrium but maintained in a so-called steady state which is distinct from
  the latter \cite{Ber68}.
\end{quote}
While von Bertalanffy's general systems theory is still approaching its
generalist ambition, his philosophy has had great impact in his home field of
biology, leading to the modern field of systems biology. Half a century later,
Dennis Noble, another great pioneer of systems biology and the originator of the
first mathematical model of a working heart, describes the shift as one from
reduction to integration.
\begin{quotation}
  Systems biology\dots\ is about putting together rather than taking apart,
  integration rather than reduction. It requires that we develop ways of
  thinking about integration that are as rigorous as our reductionist
  programmes, but different. \dots\ It means changing our philosophy, in the full
  sense of the term \cite{Nob06}.
\end{quotation}
In this thesis we refer to this new philosophy as one of interconnection.

Interconnection and openness are tightly related. Indeed, openness implies that
a system may be interconnected with its environment. But what is an environment
but comprised of other systems? Thus the study of open systems becomes the study
of how a system changes under interconnection with other systems.

The hope for formalising this is predicated on some principle of
compositionality. The principle of compositionality goes back to Frege, although
its precise meaning and applicability is still the subject of active research
today \cite{Sza13}. Loosely stated, it requires
that the meaning of the composite must be the composite of the meanings.

Christopher Strachley and Dana Scott. In the semantics of programming languages
community, this the interpretation of a language is given by a homomorphism
between an algebra of syntactic representations and an algebra of semantic
objects.



We then have language to tackle concrete questions such as
\begin{itemize}
  \item Suppose we have systems have some property, like being controllable. In
    what ways can we interconnect controllable systems so that the combined
    system is also controllable?

  \item Is it faster to compute using factorised systems? Bayesian networks.

  \item Can we identify and classify synonyms: how do we know when two diagrams
    represent the same system?

  \item How does this system behave when connected with another system? Can we
    describe control like that? 

\end{itemize}


\begin{quotation}
  While in the past, science tried to explain observable phenomena by reducing
  them to an interplay of elementary units investigable independently of each
  other, conceptions appear in contemporary science that are concerned with what
  is somewhat vaguely termed `wholeness', i.e. problems of organization,
  phenomena not resolvable into local events, dynamic interactions manifest in
  difference of behaviour of parts when isolated or in a higher configuration,
  etc.; in short, `systems' of various order not understandable by investigation
  of their respective parts in isolation. 
\end{quotation}

%http://vhpark.hyperbody.nl/images/a/aa/Bertalanffy-The_Theory_of_Open_Systems_in_Physics_and_Biology.pdf

This dissertation represents the beginnings of an attempt at a category
theoretic framework for general systems theory. A loosely organised body of
research dating back to biologist Ludwig von Bertalanffy in the mid-20th
century \cite{Ber}, general systems theory represents the study of systems built
from rich interconnections of simple component systems. Our central example in
this dissertation will be passive linear circuits: systems built from linear
resistors, inductors, and capacitors.  While each component here is simple,
networks built from such components are complex enough to form the foundation of
modern electronics.

Feedback; Wiener's cybernetics was often viewed as identical in agenda, 

More technical progress has been realised: cybernetics, catastrophe theory,
chaos theory, complex systems, network analysis

The goal here is to investigate these ideas from an algebraic perspective,
creating structural and compositional techniques for modelling
interconnection, open systems, and formal analogies, or isomorphisms, between
different fields.



\subsection*{What do we want a theory of open systems to look like?}

Now that we have explored the algebra and interconnection, what do we mean by
system? We might be tempted to simply develop structures for interconnection,
and say these model whatever they end up modelling. I hope I have made a case
that the application might be broad. But it falls on me to also show that, in
practice, that the applications do exist. For this we take the following,
definition of system, inspired especially by Willems.

Syntactically, by system we mean a `box' with finitely many `ports' through which it
interfaces with the external world. These ports may be of different types. These
systems may be connected together, along ports of the same type, to form larger
systems. Examples of such systems abound; a motivating source of them is
network-style diagrammatic languages, such as the aforementioned electrical
circuit diagrams, but also including chemical reaction networks, Petri nets,
automata, and Markov processes.

Broad appeal. It has always had a unification bent to it \cite{Ber50}

\begin{quotation}
  Not only are general aspects and viewpoints alike in different sciences;
  frequently we find formally identical or isomorphic laws in different fields.
  In many cases, isomorphic laws hold for certain classes or subclasses of
  `systems', irrespective of the nature of the entities involved. There appear
  to exist general system laws which apply to any system of a certain type,
  irrespective if the particular properties of the system and of the elements
  involved.
\end{quotation}
We will return to these ideas about isomorphic laws in Chapter 5.

This vision extended into the social sciences and humanities, influencing
economics, urban planning, sociology, psychology, management, philosophy. Eg
Forrester. These aspects of system theory are beyond the scope of the present
work.


We capture this syntactic conception of a system is formally through the notion
of a hypergraph category---a symmetric monoidal category in which every object
is equipped with a special commutative Frobenius monoid in a way compatible with
the monoidal product. This framework provides precise language for describing
how we manipulate and interact with systems.

Behaviours.

But what \emph{is} a system, and what do these manipulations and
interconnections mean? Taking inspiration from Willems \cite{Wi} and Deutsch
\cite{Deu} among others, we consider a system to be merely the set of all possible
different observations one might make by measuring all relevant variables at all
ports of the `box' we use to represent it.  We call this set the
\emph{behaviour} of the system. This arises from a view of physical laws as a
mechanism for simply partitioning the set of all trajectories of a system, the
so-called \emph{universum}, into possible and impossible trajectories.

For example, consider a resistor of resistance $r$. This has two ports---the two
ends of the resistor---and at each port we may measure the potential, and the
current flowing into the port. Now the resistor is governed by Kirchhoff's
current law, or conservation of charge, and Ohm's law. Conservation of charge
states that the current flowing into one port must equal the current flowing out
of the other port, while Ohm's law states that this current will be proportional
to the potential difference, with constant of proportionality $1/r$. Thus the
behaviour of the resistor is the set 
\[
  \big\{\big(\phi_1,\phi_2,
    -\tfrac1r(\phi_2-\phi_1),\tfrac1r(\phi_2-\phi_1)\big)\,\big\vert\,
    \phi_1,\phi_2 \in \mathbb{R}\big\}.
\]
Here the universum is the set
$\mathbb{R}\oplus\mathbb{R}\oplus\mathbb{R}\oplus\mathbb{R}$, where the
summands represent respectively the potentials and currents at each of the two
terminals.

% universum $\mathcal U$ of trajectories, behaviour $\mathcal P(\mathcal U)$,
% principle $\mathcal P(\mathcal P(\mathcal U))$,

The variables associated to each port are determined by the type of the port.
Interconnection of ports then, simply asserts the identification of the
variables at the connected ports. On the level of behaviours, this becomes a
generalised version of composition of relations.

In this thesis I argue that this general framework has wide applicability to
applied science and engineering, and with appropriate additional mathematical
tools allows one to formalise certain diagrammatic languages and their
relationships.

Relation

Example: trijunction

Interconnection language

Diagrammatic

In this program Odum, energese. diagrams. universal language SBGN, UML

examples examples.


signal flow

circuits

automata, markov processes, flow networks.

Accessible

Kindergarten quantum mechanics

Inspired by Coecke Abramsky CQM


\subsection*{More specifically?}

Categorical

In late 1940s, just as Feynman was developing his diagrams for processes in
particle physics, Eilenberg and Mac Lane initiated their work on category
theory.  Over the subsequent decades, and especially in the work of Joyal and
Street in the 1980s \cite{JS91,JS93}, it became clear that these developments were profoundly linked: monoidal categories have a precise graphical representation in terms of string diagrams, and conversely monoidal categories provide an algebraic foundation for the intuitions behind Feynman diagrams.  The key insight is the use of categories where morphisms describe physical processes, rather than structure-preserving maps between mathematical objects \cite{BaezStay,CP}.

In work on fundamental physics, the cutting edge has moved from categories
to higher categories \cite{BL}.  But the same techniques have filtered into more
immediate applications, particularly in computation and quantum computation
\cite{AC04,Ba1,Sel07}.  This chapter is part of a still nascent program of
applying string diagrams to engineering, with the aim of giving diverse diagram
languages a unified foundation based on category theory \cite{BE,BSZ,KSW,RSW05,Sp}. 

Category-theoretic frameworks for general systems theory have been developed
before. Notably Goguen and Rosen led efforts. Goguen from a more computer
science perspective, Rosen more in biology.


Computer science-y

Our work lies in the intersection of computer science, mathematics, and systems
theory.  From computer science, we use concepts  of formal semantics of
programming languages, with an emphasis on compositionality and a firm
denotational foundation for operational definitions.  From a mathematical
perspective, we obtain presentations of several relevant domains, and identify
the rich underlying algebraic structures.  For systems theory, our insight is
that mere matrices are not optimised for discussing behaviour; instead it is
profitable to use signal flow graphs, which treat linear subspaces rather than
linear transformations as the primitive concept and are thus closer to the idea
of system as a set of trajectories. At the core is the maxim---perhaps best
understood by computer scientists---that the right language allows deeper
insights into the underlying structure.

Influence of computer science. syntax semantics. rewrite rules/local rather than
global analysis.

\subsection*{What are we adding here?}


what is the algebra of network languages?

hypergraph categories.

graph example

black-boxing, information compression.

The Yoneda lemma and representability.

The central message of decorated corelations is that hypergraph structure
requires some sort of uniformity in the composition rule, and it is easier to
work by acknowledging this structure, defining them as algebras over some
theory. But we can weaken this notion of theory.

Beyond category theoretic interest, the motivation for such a method lies in
developing compositional accounts of semantics associated to topological
diagrams. While this has long been a technique associated with topological
quantum field theory, dating back to \cite{At}, it has most recently had
significant influence in the nascent field of categorical network theory, with
application to automata and computation \cite{KSW2, Sp}, electrical circuits
\cite{BF}, signal flow diagrams \cite{BSZ, BE}, Markov processes \cite{BFP,
ASW}, and dynamical systems \cite{VSL}, among others. 

Cospans are already familiar as a formalism for making entities with an
arbitrarily designated `input end' and `output end' into the morphisms of a
category.  For example, in topological quantum field theory we use special
cospans called `cobordisms' to describe pieces of spacetime \cite{BL,BaezStay}.

An advantage of the decorated cospan framework is that the resulting categories
are hypergraph categories, and the resulting functors respect this structure.
As dagger compact categories, hypergraph categories themselves have a rich
diagrammatic nature \cite{Sel11}, and in cases when our decorated cospan categories
are inspired by diagrammatic applications, the hypergraph structure provides
language to describe natural operations on our diagrams, such as juxtaposing,
rotating, and reflecting them.



Hypergraph categories are really just $\Set$-valued lax symmetric monoidal
functors: and these, being simply data structures, are often simpler to work
with.



\subsection*{How does this relate to what's come before?}


Work on categorical approaches to control systems goes back at least to Goguen
\cite{Go} and Arbib and Manes \cite{AM}. In recent years, there has been a
resurgence of interest in the topic, including work by Baez and Erbele
\cite{BE}, Vagner, Spivak, and Lerman \cite{VSL}, as well as Bonchi, Zanasi, and
the second author \cite{BSZ,BSZ2,BSZ3,Za}. 

Spivak operad approach

Zanasi, Soboc\'inski, Bonchi

Most similar is the work of Walters, together with Sabadini and Rosebrugh. In
particular, automata.









\section{Outline}
Chapter 1 introduces hypergraph categories. Then decorated cospans and decorated
corelations.

Then applications. First to control theory from this behavioural perspective,
demonstrating the ideas of black boxing and corelations, open systems and
control by interconnection. We give a new characterisation of controllability.
Second to passive linear networks, which speaks to the idea that we can find
isomorphisms across disciplines, and explore the insights in new ways.


\section{Contributions of collaborators}

\begin{quotation}
  \emph{
Where some part of the thesis is not solely the work of the candidate or has
been carried out in collaboration with one or more persons, the candidate shall
submit a clear statement of the extent of his or her own contribution.
}
\end{quotation}

The first four chapters are my own work. The applications chapters were
developed with collaborators. Chapter \ref{ch.sigflow} arises from a weekly
seminar with Paolo Rapisarda and Pawe\l\ Soboc\'inski. I developed the
corelation formalism and provided a first draft of the paper. Pawe\l\ provided
much expertise in signal flow graphs, significantly revising the text and
contributing the section on operational semantics. Paolo contributed comparisons
to classical methods in control theory.  Much of the text is taken from the
paper \cite{FRS16}.

Chapter \ref{ch.circuits} is joint work with John Baez. John proposed the topic
of research, and supplied writing on Ohm's law and the principle of minimum
power that became the second section of . I was responsible for producing the
first draft of the chapter from this start.  John guided and assisted revisions
of this text for publication \cite{BF}.

I thank my collaborators.
