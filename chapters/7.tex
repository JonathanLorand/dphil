\chapter{Further directions} \label{ch.further}
In summary, we have set up a framework for formalising network-style
diagrammatic languages, and illustrated it in detail with applications to two
types of such languages.

We close with a few words regarding ongoing research extending these ideas. For
balance, the first is theoretical, regarding the use of epi-mono factorisation
systems for black boxing systems, and the second applied, regarding consequences
of our work on passive linear networks in Chapter \ref{ch.circuits}. In
particular, in the second section we demonstrate that the existence of the black
box functor is not just a pretty theoretical result, but creates a language in
which genuinely applied, novel results can be formulated and proved.

\section{Bound colimits}

A theme of this thesis is that while cospans in some category $\mc C$ with
finite colimits are adequate for describing the interconnection of open systems,
their composition often retains more information than we can `see' from the
boundary. In the cases of ideal wires (\textsection\ref{ssec.equivrels}), linear
relations (\textsection\ref{ssec.linrel}), and signal flow diagrams (Chapter
\ref{ch.sigflow}), corelations with respect to an epi-mono factorisation system
instead provide the right notion of black boxed open system. Why? What do
different factorisation systems model, and how to choose the right one?

In \cite{RSW08}, Rosebrugh, Sabadini, and Walters describe how composition of
cospans can be used to calculate colimits. Here we argue that the idea of an
open system or, equivalently, a system with boundary, motivates a variant of
the colimit, which we term the bound colimit.

Given a category, an open system is a morphism $s\maps B \to T$. Call $B$ the
boundary, and $T$ the (total) system. For example, in $\FinSet$ a system is just
a set of `connected components', and the boundary describes how we may connect
other systems to these components.

In the above, have worked with with the boundary as two objects, so an open
system is a cospan $B_1 \to T \leftarrow B_2$. To return to the above picture,
we just observe our boundary consists of the discrete two object category
$\{B_1,B_2\}$, and take a colimit, yielding the open system $B_1+B_2 \to T$.

In general we can think of specifying an open system in this way: not just as a
single morphism $B \to T$, but as a diagram in our category, where some objects
$B_i$ are denoted `boundary' objects, and the remainder $T_j$ `system' objects. It
helps intuition to add the condition that our diagram cannot contain maps from a
`system' object to a `boundary' object.

We can compute a more efficient representation of our diagrammed open system by
first taking the colimit of the full subcategory with objects the boundary
objects $B_i$, to arrive at a single boundary object $B$. We can also take a
colimit over the entire diagram to arrive at a single total system object $T$.
Since a cocone over the entire diagram is a fortiori a cocone over the boundary
subdiagram, we get a map $B \to T$. This is our open system.

Suppose now we are interested in the most efficient representation of this
system. In the examples mentioned above this has meant $s\maps B \to T$ is an
epimorphism. Why? Consider the following intuition. Varying over all `viewing
objects' $V$, a system $T$ is completely described by $\hom(T,V)$ (in the sense
of the Yoneda lemma).  But from at the boundary $B$, not all of these maps can
be witnessed. Instead the best we can do is study the image of $\hom(T,V)$ in
$\hom(B,V)$, where the image is given by precomposition with $s\maps B \to T$.
Call two systems with boundary $B$ equivalent if for all objects $V$ they give
the same image in $\hom(B,V)$.

Now, many different systems will be equivalent. For example, if we take any open
system $s\maps B \to T$ and split mono $m\maps T \to T'$ (with retraction
$m'\maps T' \to T$), then any map $f\maps B \xrightarrow{s} T \xrightarrow{r} V$
can be written $f\maps B \xrightarrow{s;m} T' \xrightarrow{m';r} V$.  And any
map $g\maps B \xrightarrow{s;m} T' \xrightarrow{r'}V$ can be written $g\maps B
\xrightarrow{s} T \xrightarrow{m;r'} V$. So $s\maps B \to T$ and $s;m\maps B \to
T'$ give the same image in $\hom(B,V)$.  (These images being the images of the
maps $\hom(T,V) \to \hom(B,V)$ and $\hom(T',V) \to \hom(B,V)$ they respectively
induce.)

Consequently, if we are interested in efficient representations, then we should
`subtract' the largest split monic we can from T. This `retracts' the redundant
information. Also note that if we have an epimorphism $e\maps B \to T$ then the
map $\hom(T,V) \to \hom(B,V)$ is one-to-one.

Thus, if our category has an epi-split mono factorisation system, then
equivalence classes of open systems can be indexed by epis $e\maps B \to T$. The
epi $e$ is the `minimal representation', or the `black boxed' system.

Suppose now that we want to compute the interconnection---or colimit---and minimal
representation---the epi part---simultaneously. This allows us to consider a
setting where there are only minimal representations: where everything is always
`black boxed'. This is where bound colimits come in.

To compute the minimal representation we were interested in the image of system
homset $\hom(T,V)$ in the boundary homset $\hom(B,V)$. What is the universal
property of the epi part $e\maps B \to M$ of an open system $s\maps B \to T$?

First, assuming we have an epi-split mono factorisation $s = e;m$, with some
retraction $m'$, then $e\maps B \to M$ is a cocone over $B$ that extends
(nonuniquely) to a cocone
\[
  \xymatrix{
    & M \\
    B \ar[ur]^e \ar[rr]^s &&  T \ar[ul]_{m'}
  }
\]
over $s\maps B \to T$. Second, if there is another cocone $c\maps B \to N$ over $B$ that extends to
some cocone
\[
  \xymatrix{
    & N \\
    B \ar[ur]^c \ar[rr]^s &&  T \ar[ul]_{x}
  }
\]
over $s\maps B \to T$, then there exists a unique map $m;x\maps M \to N$ such that
\[
  \xymatrix{
    M \ar[rr]^{m;x} && N\\
    & B \ar[ul]^e \ar[ur]_c
  }
\]
commutes. Uniqueness is by the epi property of $e$.

Thus the `minimal representation' $M$ is initial in the cocones over $B$ that
extend to cocones over $s: B \to T$.  The intuition is that $M$ finds the
balance between removing useful information---if we remove too much it won't
extend to a cocone over $B \to T$---and removing redundant information---if we
don't remove enough then it won't be initial.

To generalise this diagrams of open systems, rather than just an arrow $B \to
T$, we instead want the apex of the cocone that is initial out of all cocones
over the boundary diagram that extend to a cocone over the system diagram. Call
this the bound colimit. Then, instead of describing composition of corelations
as `taking the jointly epic part of the pushout', we might simply say we
`compute the bound pushout'. Further work will explore the relationship between
epi-mono corelations and bound colimits in analogy with the relationship between
cospans and colimits.


\section{Open circuits and other systems}

We conclude with brief mentions of two applications of the category of passive
linear networks, illustrating the further utility of formalising network-style
diagrammatic languages as hypergraph categories.

The first concerns making explicit use of the compositional structure to study
networks of linear resistors. In \cite{Jek}, Jekel uses the compositional
structure to study the inverse problem for these networks. The inverse problem
seeks to determine the resistances on the edges of a circuit knowing only the
underlying graph and the behaviour. Jekel defines a notion of elementary
factorisation of circuits in terms of the compositional structure in $\Circ$,
and shows that these factorisations can be used to state a sufficient condition
for the size of connections through the graph to be detectable from the
linear-algebraic properties of the behaviour.

The second demonstrates how formalisation of network-style diagrammatic
languages allows formal exploration of the relationships between them. Work with
Baez and Pollard \cite{BFP,Pol16} defines a decorated cospan category
$\mathrm{DetBalMark}$ in which the morphisms are open detailed balanced
continuous-time Markov chains, or open detailed balanced Markov processes for
short.  An open Markov process is one where probability can flow in or out of
certain states called `inputs' and `outputs'. A detailed balanced Markov process
is one equipped with a chosen equilibrium distribution such that the flow of
probability from any state $i$ to any state $j$ is equal to the flow of
probability from $j$ to $i$. 

The behaviour of a detailed balanced open Markov process is determined by a
principle of minimum dissipation, closely related to Prigogine's principle of
minimum entropy production. Moreover, the semantics of these Markov processes are
given by a functor $\square\maps \mathrm{DetBalMark} \to \mathrm{LagrRel}$. This has a strong analogy with the principle of
minimum power and the black box functor, and we show there is a functor $K\maps
\mathrm{DetBalMark} \to \Circ$ and a natural transformation $\alpha$
\[
  \xymatrixrowsep{2ex}
  \xymatrix{
    \mathrm{DetBalMark} \ar[dd]_{K} \ar[drr]^{\square}  \\
    &\twocell \omit{_\:\alpha}& \mathrm{LagrRel} \\
    \Circ. \ar[urr]_{\blacksquare} 
  }
\]
This natural transformation makes precise these analogies, and shows that we can
model detailed balanced Markov processes with circuits of linear resistors in a
compositional way.

%Finally, note that one straightforward formula for extending the work of this
%thesis is to take a network-style diagrammatic language and formulate it and its
%semantics as a hypergraph category. Further ongoing work investigates
%applications of these ideas to bond graphs, chemical reaction networks,
%stochastic Petri nets, and flow networks. 


%Open systems. 
%
%\section{Theoretical}
%\subsection{Bounded colimits}
%Why are epis so important? Extended theory
%
%Grothendieck construction (Kissinger comment)
%
%\subsection{Classifying hypergraph categories}
%
%
%\subsection{Other categorical structures}
%Not all diagrammatic structures have the same hypergraph network style. For
%example, some have substantive notions of input and output. Spivak's work on
%operads of wiring diagrams 
%
%
%\section{Applied}
%
%\subsection{Flow networks}
%cf linear algebra. diagrams+rewrite rules, algebraic semantics, summary arrays.
%
%
%
%\subsection{Chemical reaction networks}
%
%
%\subsection{Automata}
