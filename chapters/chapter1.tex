\chapter{Hypergraph categories}


\subsection{Hypergraph categories.}

A \define{Frobenius monoid} $(X,\mu,\delta,\eta,\epsilon)$ in a monoidal
category $(\mathcal C, \otimes)$ is an object $X$ together with monoid $(X,\mu,
\eta)$ and comonoid $(X,\delta,\epsilon)$ structures such that
\[
  (1 \otimes \mu) \circ (\delta \otimes 1) = \delta \circ \mu = (\mu \otimes 1)
  \circ (1 \otimes \delta)\colon  X \otimes X \longrightarrow X \otimes X.
\]
A Frobenius monoid is further called \define{special} if 
\[
  \mu \circ \delta = 1\colon  X \longrightarrow X,
\]
and further called \define{commutative} if the ambient monoidal category is symmetric
and the monoid and comonoid structures that comprise the Frobenius monoid are
commutative and cocommutative respectively. Note that for Frobenius monoids
commutativity of the monoid structure implies cocommutativity of the comoniod
structure, and vice versa, so the use of the term `commutativity' for both the
Frobenius monoid and the constituent monoid is not ambiguous.

A \define{hypergraph category} is a
symmetric monoidal category in which each object is equipped with a special
commutative Frobenius structure $(X,\mu_X,\delta_X,\eta_X,\epsilon_X)$ such that 
\[
  \begin{array}{cc}
    \mu_{X\otimes Y} = (\mu_X \otimes \mu_Y)\circ(1_X \otimes \sigma_{YX}\otimes
    1_Y) \qquad&
    \eta_{X\otimes Y} = \eta_X \otimes \eta_Y \\
    \delta_{X\otimes Y} = (1_X \otimes \sigma_{XY}\otimes 1_Y)\circ(\delta_X
    \otimes \delta_Y) \qquad&
    \epsilon_{X\otimes Y} = \epsilon_X \otimes \epsilon_Y.
  \end{array}
\]
A functor $(F,\varphi)$ of hypergraph categories, or \define{hypergraph
functor}, is a strong symmetric monoidal functor $(F,\varphi)$ that preserves
the hypergraph structure. More precisely, the latter condition means that given
an object $X$, the special commutative Frobenius structure on $FX$ must be 
\[
  (FX,\enspace F\mu_X \circ \varphi_{X,X},\enspace  \varphi^{-1} \circ F\delta_X,\enspace  F\eta_X \circ
\varphi_1,\enspace  \varphi_1 \circ \epsilon_X).
\]

This terminology was introduced recently \cite{Ki}, in reference to the fact
that these special commutative Frobenius monoids provide precisely the structure
required to draw graphs with `hyperedges': wires connecting any number of
inputs to any number of outputs. Commutative special Frobenius monoids are also
known as commutative separable algebras \cite{RSW}, and hypergraph categories as
well-supported compact closed categories \cite{Ca}.

Note that if an object $X$ is equipped with a Frobenius monoid structure then
the maps $\epsilon \circ \mu\colon  X \otimes X \longrightarrow 1$ and $\delta \circ
\eta\colon  1 \longrightarrow X \otimes X$ obey 
\[
  \big(1 \otimes (\epsilon \circ \mu)\big) \circ \big((\delta \circ \eta)
  \otimes 1\big) = 1_X = \big((\epsilon \circ \mu) \otimes 1\big) \circ \big(1
  \otimes (\delta \circ \eta)\big)\colon X \longrightarrow X.
\]
Thus if an object carries a Frobenius monoid it is also self-dual, and any
hypergraph category is a fortiori self-dual compact closed. Mapping each
morphism $f\colon  X \to Y$ to its dual morphism
\[
  \big((\epsilon_Y \circ \mu_Y) \otimes 1_X\big) \circ \big( 1_Y \otimes f
  \otimes 1_X \big) \circ \big(1_Y \otimes (\delta_X \circ \eta_X)\big)\colon  Y
  \longrightarrow X
\]
further equips each hypergraph category with a so-called dagger functor---an
involutive contravariant endofunctor that is the identity on objects---such that
the category is a dagger compact category. Dagger compact categories were first
introduced in the context of categorical quantum mechanics \cite{AC}, under the
name strongly compact closed category, and have been demonstrated to be a key
structure in diagrammatic reasoning and the logic of quantum mechanics.

We shall see that every decorated cospan category is a hypergraph category, and
hence also a dagger compact category.

\begin{example}
  A central example of a hypergraph category is the category
  $\mathrm{Cospan(\mathcal C)}$ of cospans in any category $\mathcal C$ with
  finite colimits. We will later see that decorated cospan categories are a
  generalisation of such categories, and each inherits a hypergraph structure
  from such. 

  First, $\mathrm{Cospan(\mathcal C)}$ inherits a symmetric monoidal structure
  from $\mathcal C$. We call a subcategory $\mathcal C$ of a category $\mathcal
  D$ \define{wide} if $\mathcal C$ contains all objects of $\mathcal D$, and
  call a functor that is faithful and bijective-on-objects a \define{wide
  embedding}. Note then that we have a wide embedding
  \[
    \mathcal C \hooklongrightarrow \mathrm{Cospan(\mathcal C)}
  \]
  that takes each object of $\mathcal C$ to itself as an object of
  $\mathrm{Cospan(\mathcal C)}$, and each morphism $f\colon  X \to Y$ in $\mathcal C$
  to the cospan
  \[
    \xymatrix{
      & Y \\
      X \ar[ur]^{f} && Y, \ar@{=}[ul]
    }
  \]
  where the extended `equals' sign denotes an identity morphism. This allows us
  to view $\mathcal C$ as a wide subcategory of $\mathrm{Cospan(\mathcal C)}$.

  Now as $\mathcal C$ has finite colimits, it can be given a symmetric monoidal
  structure with the coproduct the monoidal product; we write this monoidal
  category $(\mathcal C,+)$, and write $\varnothing$ for the initial object, the
  monoidal unit of this category. Then $\mathrm{Cospan(\mathcal C)}$ inherits
  the same symmetric monoidal structure: since the monoidal product $+\colon \mathcal
  C \times \mathcal C \to \mathcal C$ is left adjoint to the diagram functor, it
  preserves colimits, and so extends to a functor $+\colon
  \mathrm{Cospan(\mathcal C)} \times \mathrm{Cospan(\mathcal C)} \to
  \mathrm{Cospan(\mathcal C)}$. The remainder of the monoidal structure is
  inherited because $\mathcal C$ is a wide subcategory of
  $\mathrm{Cospan(\mathcal C)}$.

  Next, the Frobenius structure comes from copairings of identity morphisms. We
  call cospans 
  \[
    \xymatrix{
      & N \\
      X \ar[ur]^{i} && Y \ar[ul]_{o}
    }
    \qquad \xymatrix@R=8pt{\\\textrm{and}} \qquad 
    \xymatrix{
      & N \\
      Y \ar[ur]^{o} && X \ar[ul]_{i}
    }
  \]
  that are reflections of each other \define{opposite} cospans. Given any object
  $X$ in $\mathcal C$, the copairing $[1_X,1_X]\colon  X + X \to X$ of two identity
  maps on $X$, together with the unique map $!\colon  \varnothing \to X$ from the
  initial object to $X$, define a monoid structure on $X$. Considering these
  maps as morphisms in $\mathrm{Cospan(\mathcal C)}$, we may take them together
  with their opposites to give a special commutative Frobenius structure on $X$.
  In this way we consider each category $\mathrm{Cospan(\mathcal C)}$ a
  hypergraph category.

  It is a simple computation to check that the resulting dagger functor simply
  takes a cospan $X \stackrel{i}{\longrightarrow} N \stackrel{o}{\longleftarrow}
  Y$ to its opposite cospan $Y \stackrel{o}{\longrightarrow} N
  \stackrel{i}{\longleftarrow} X$.
\end{example}

\subsection{Dagger compact categories} \label{subsec:dagger}

In a dagger category, the distinction between the source and target of a morphism is arbitrary, and can be reversed. This is true of electrical circuits: if we like we may treat the set of inputs as the set of outputs instead, and the set of outputs as the set of inputs.

Recall that a \define{dagger category} is a category equipped with an
involutive, contravariant endofunctor that is the identity on objects
\cite{AC,Se}. In other words, a \define{dagger structure} on a category $\mc C$
is a contravariant functor $(-)^\dagger\maps \mc C \to \mc C$ such that
$A^\dagger = A$ for all objects $A \in \ob \mc C$ and $(f^\dagger)^\dagger = f$
for all morphisms $f$ in $\mc C$.  A dagger category is then a category equipped
with a dagger structure.

When other structure is present, we prefer this dagger to be compatible with it.
We say that a morphism $f$ is \define{unitary} if its dagger provides it with an
inverse morphism. A \define{symmetric monoidal dagger category} is a symmetric
monoidal category equipped with a \define{symmetric monoidal dagger
structure}---that is, a dagger structure that coherently preserves the symmetric
monoidal structure. Concretely, this requires that the functor $(-)^\dagger\maps
\mc C \to \mc C^{\mathrm op}$ be symmetric monoidal, and that the associator,
unitors, and braiding of $\mc C$ be unitary. Furthermore, letting $L$ and $R$ be
dual objects of a symmetric monoidal dagger category, with monoidal unit $I$,
braiding $\s_{L,R}\maps L \ot R \to R \ot L$, and unit $\eta\maps I \to R
\otimes L$ and counit $\eps\maps L \otimes R \to I$, we say that $L$ and $R$ are
\define{dagger dual} if $\eta = \s \circ \eps^\dagger$.  A \define{dagger
compact category} is a symmetric monoidal dagger category in which every object
has a dagger dual, while a \define{dagger functor} is a functor $F$ between
dagger compact categories that preserves the dagger structures: $F((-)^\dagger)
= (F(-))^\dagger$.

Importantly for our applications, dagger compact categories come with a
graphical calculus, where each morphism is represented by a diagram such that
two diagrams are considered equal by the rules of this calculus if and only if
they are equal according to the defining laws of dagger compact categories
\cite{Se}. In brief, to set up our conventions, we represent a morphism $f\maps X_1
\otimes \dots \otimes X_n \to Y_1 \otimes \dots \otimes Y_m$ as a `downwards
flow chart':
\[
  \begin{tikzpicture}
	  \begin{pgfonlayer}{nodelayer}
		  \node [style=none] (0) at (-0.75, 0.25) {};
		  \node [style=none] (1) at (0.75, 0.25) {};
		  \node [style=none] (2) at (-0.75, -0.25) {};
		  \node [style=none] (3) at (0.75, -0.25) {};
		  \node [style=none] (4) at (0, 0.25) {};
		  \node [style=none] (5) at (0, -0.25) {};
		  \node [style=none] (6) at (0, -0) {$f$};
		  \node [style=none] (7) at (-0.5, 0.25) {};
		  \node [style=none] (8) at (0.5, 0.25) {};
		  \node [style=none] (9) at (-0.5, -0.25) {};
		  \node [style=none] (10) at (0.5, -0.25) {};
		  \node [style=none] (11) at (-0.5, 0.75) {};
		  \node [style=none] (12) at (0.5, 0.75) {};
		  \node [style=none] (13) at (-0.5, -0.75) {};
		  \node [style=none] (14) at (0.5, -0.75) {};
		  \node [style=none] (15) at (0, 0.5) {$\dots$};
		  \node [style=none] (16) at (0, -0.5) {$\dots$};
		  \node [style=none] (17) at (-0.5, 1) {$X_1$};
		  \node [style=none] (18) at (0.5, 1) {$X_n$};
		  \node [style=none] (19) at (-0.5, -1) {$Y_1$};
		  \node [style=none] (20) at (0.5, -1) {$Y_m$};
	  \end{pgfonlayer}
	  \begin{pgfonlayer}{edgelayer}
		  \draw (0.center) to (1.center);
		  \draw (1.center) to (3.center);
		  \draw (3.center) to (2.center);
		  \draw (2.center) to (0.center);
		  \draw (7.center) to (11.center);
		  \draw (8.center) to (12.center);
		  \draw (9.center) to (13.center);
		  \draw (10.center) to (14.center);
	  \end{pgfonlayer}
  \end{tikzpicture}
\]
Composition is then represented by connecting the lines (or wires) representing
the codomain of one morphism with the domain of the another placed below it,
the monoidal product of two morphisms is represented by their side-by-side
juxtaposition, the swap map by crossing two wires, the compact structure by
bending a wire 180 degrees, and the dagger functor by flipping a diagram in the
horizontal axis. We believe these operations on diagrams---placing diagrams on
the same page, rearranging their wires/flipping them, and then connecting their
wires to form a larger diagram---represents the collection of operations used
for reasoning with circuit diagrams, and hence that dagger compact categories
are an appropriate structure for the formalization of such.
