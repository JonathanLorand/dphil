\chapter{Passive linear networks} \label{ch.circuits}


\section{Introduction}\label{sec:intro}
%%fakesubsection
In late 1940s, just as Feynman was developing his diagrams for processes in particle physics, Eilenberg and Mac Lane initiated their work on category theory.  Over the subsequent decades, and especially in the work of Joyal and Street in the 1980s \cite{JS1,JS2}, it became clear that these developments were profoundly linked: monoidal categories have a precise graphical representation in terms of string diagrams, and conversely monoidal categories provide an algebraic foundation for the intuitions behind Feynman diagrams.  The key insight is the use of categories where morphisms describe physical processes, rather than structure-preserving maps between mathematical objects \cite{BaezStay,CP}.

In work on fundamental physics, the cutting edge has moved from categories
to higher categories \cite{BL}.  But the same techniques have filtered into more immediate applications, particularly in computation and quantum computation \cite{AC,Ba1,Se}.  This paper is part of a still nascent program of applying string diagrams to engineering, with the aim of giving diverse diagram languages a unified foundation based on category theory \cite{BE,BSZ,KSW,RSW,Sp}. 

Indeed, even before physicists began using Feynman diagrams, various branches of engineering were using diagrams that in retrospect are closely related.   Foremost among these are the ubiquitous electrical circuit diagrams. Although less well-known, similar diagrams are used to describe networks consisting of mechanical, hydraulic, thermodynamic and chemical systems.   Further work, pioneered in particular by 
Forrester \cite{Fo} and Odum \cite{Od}, applies similar diagrammatic methods to biology, ecology, and economics.

As discussed in detail by Olsen \cite{Ol}, Paynter \cite{Pa} and others, there are mathematically precise analogies between these different systems.  In each case, the system's state is described by variables that come in pairs, with one variable in each pair playing the role of  `displacement' and the other playing the role of `momentum'.  In engineering, the time derivatives of these variables are sometimes called `flow' and `effort'.    In classical mechanics, this pairing of variables is well understood using
symplectic geometry.  Thus, any mathematical formulation of the diagrams used to
describe networks in engineering needs to take symplectic geometry as well as category
theory into account. 

\vskip 1em
\begin{small}
\begin{center}
\begin{tabular}{|c||c|c|c|c|}
\hline
& displacement  &  flow & momentum & effort \\
& $q$ & $\dot{q}$ & $p$ & $\dot{p}$ \\
\hline\hline
Electronics & charge & current & flux linkage & voltage\\
\hline
Mechanics (translation) & position & velocity & momentum & force\\
\hline
Mechanics (rotation) & angle & angular velocity & angular momentum & torque\\
\hline
Hydraulics & volume & flow & pressure momentum & pressure\\
\hline
Thermodynamics & entropy & entropy flow & temperature momentum & temperature \\
\hline
Chemistry & moles & molar flow & chemical momentum & chemical potential \\
\hline
\end{tabular}
\end{center}
\end{small}

While diagrams of networks have been independently introduced in many disciplines, we do not expect formalizing these diagrams to immediately help the practitioners of these disciplines.  At first the flow of information will mainly go in the other direction: by translating ideas from these disciplines into the language of modern mathematics, we can provide mathematicians with food for thought and interesting new problems to solve.  We hope that in the long run mathematicians can return the favor by bringing new insights to the table.

Although we shall keep the broad applicability of network diagrams in the back of our minds, we couch our discussion in terms of electrical circuits, for the sake of familiarity. In this paper our goal is somewhat limited.  We only study circuits built from `passive' components: that is, those that do not produce energy.  Thus, we exclude batteries and current sources.  We only consider components that respond linearly to an applied voltage.   Thus, we exclude components such as nonlinear resistors or diodes.  Finally, we only consider components with one input and one output, so that a circuit can be described as a graph with edges labeled by components.  Thus, we also exclude transformers.  The most familiar components our framework covers are linear resistors, capacitors and inductors.

While we hope to expand our scope in future work, the class of circuits made from these components has appealing mathematical properties, and is worthy of deep study.  Indeed, this class has been studied intensively for many decades by electrical engineers \cite{AV,Budak,Slepian}.  Even circuits made exclusively of resistors have inspired work by mathematicians of the caliber of Weyl \cite{Weyl} and Smale \cite{Smale}.  

Our work relies on this research.  All we are adding is an emphasis on symplectic geometry and an explicitly `compositional' framework, which clarifies the way a larger circuit can be built from smaller pieces.  This is where monoidal categories become important: the main operations for building circuits from pieces are composition and tensoring.
 
Our strategy is most easily illustrated for circuits made of linear resistors.  Such a resistor dissipates power, turning useful energy into heat at a rate determined by the voltage across the resistor.  However, a remarkable fact is that a circuit made of these resistors always acts to \emph{minimize} the power dissipated this way.  This `principle of minimum power' can be seen as the reason symplectic geometry becomes important in understanding circuits made of resistors, just as the principle of least action leads to the role of symplectic geometry in classical mechanics.  

Here is a circuit made of linear resistors:
\[
\begin{tikzpicture}[circuit ee IEC, set resistor graphic=var resistor IEC graphic]
\node[contact] (I1) at (0,2) {};
\node[contact] (I2) at (0,0) {};
\node[contact] (O1) at (5.83,1) {};
\node(input) at (-2,1) {\small{\textsf{inputs}}};
\node(output) at (7.83,1) {\small{\textsf{outputs}}};
\draw (I1) 	to [resistor] node [label={[label distance=2pt]85:{$3\Omega$}}] {} (2.83,1);
\draw (I2)	to [resistor] node [label={[label distance=2pt]275:{$1\Omega$}}] {} (2.83,1)
				to [resistor] node [label={[label distance=3pt]90:{$4\Omega$}}] {} (O1);
\path[color=gray, very thick, shorten >=10pt, ->, >=stealth, bend left] (input) edge (I1);		\path[color=gray, very thick, shorten >=10pt, ->, >=stealth, bend right] (input) edge (I2);		
\path[color=gray, very thick, shorten >=10pt, ->, >=stealth] (output) edge (O1);
\end{tikzpicture}
\]
The wiggly lines are resistors, and their resistances are written beside them: for example,
$3\Omega$ means 3 ohms, an `ohm' being a unit of resistance.  To formalize this, define a circuit of linear resistors to consist of:
\begin{itemize}
\item a set $N$ of nodes,
\item a set $E$ of edges, 
\item maps $s,t \maps E \to N$ sending each edge to its source and target node,
\item a map $r\maps E \to (0,\infty)$ specifying the resistance of the resistor 
labelling each edge, 
\item maps $i \maps X \to N$, $o \maps Y \to N$ specifying the
inputs and outputs of the circuit.
\end{itemize}

When we run electric current through such a circuit, each node $n \in N$ gets
a `potential' $\phi(n)$.  The `voltage' across an edge $e \in E$ is defined as the 
change in potential as we move from to the source of $e$ to its target, $\phi(t(e)) - 
\phi(s(e))$, and the power dissipated by the resistor on this edge equals
\[      
\frac{1}{r(e)}\big(\phi(t(e))-\phi(s(e))\big)^2. 
\]
The total power dissipated by the circuit is therefore twice
\[   
P(\phi) = \frac{1}{2}\sum_{e \in E} \frac{1}{r(e)}\big(\phi(t(e))-\phi(s(e))\big)^2.
\]
The factor of $\frac{1}{2}$ is convenient in some later calculations.  
Note that $P$ is a nonnegative quadratic form on the vector space $\R^N$.
However, not every nonnegative definite quadratic form on $\R^N$ arises in this way from some circuit of linear resistors with $N$ as its set of nodes.  The quadratic forms that do arise are called `Dirichlet forms'.  They have been extensively investigated \cite{Fukushima,MR,Sabot1997,Sabot2004}, and they play a major role in our work.

We write $\partial N = i(X) \cup o(Y)$ for the set of `terminals': that is,
nodes corresponding to inputs and outputs.    The principle of minimum
power says that if we fix the potential at the terminals, the circuit will choose
the potential at other nodes to minimize the total power dissipated.   
An element $\psi$ of the vector space $\R^{\partial N}$ assigns a potential 
to each terminal.   Thus, if we fix $\psi$, the total power dissipated will be twice
\[
  Q(\psi) = \min_{\substack{ \phi \in \R^N \\ \phi\vert_{\partial N} = \psi}} \; P(\phi)  
\]
The function $Q \maps \R^{\partial N} \to \R$ is again a Dirichlet form.  We call it the `power functional' of the circuit.  

Now, suppose we are unable to see the internal workings of a circuit, and can only observe its `external behavior': that is, the potentials at its terminals and the currents flowing into or out of these terminals.   Remarkably, this behavior is completely determined by the power functional $Q$.  The reason is that the current at any terminal can be obtained by differentiating $Q$ with respect to the potential at this terminal, and relations of this form are \emph{all} the relations that hold between potentials and currents at the terminals.

The Laplace transform allows us to generalize this immediately to circuits that
can also contain linear inductors and capacitors, simply by changing the field we work over, replacing $\R$ by the field $\R(s)$ of rational functions of a single real variable,
and talking of `impedance' where we previously talked of resistance.  We obtain
a category $\Circ$ where an object is a finite set, a morphism $f \maps X \to Y$ is a circuit with input set $X$ and output set $Y$, and composition is given by identifying the outputs of one circuit with the inputs of the next, and taking the resulting union of labelled graphs.  Each such circuit gives rise to a Dirichlet form, now defined over
$\R(s)$, and this Dirichlet form completely describes the externally observable
behavior of the circuit.  

We can take equivalence classes of circuits, where two circuits count as the
same if they have the same Dirichlet form.  We wish for these equivalence classes of circuits to form a category. Although
there is a notion of composition for Dirichlet forms, we find that it lacks
identity morphisms or, equivalently, it lacks morphisms representing ideal wires
of zero impedance. To address this we turn to Lagrangian subspaces of
symplectic vector spaces.  These generalize quadratic forms via the map
\[
  \Big(Q\maps \F^{\partial N} \to \F\Big) \longmapsto \Big(\mathrm{Graph}(dQ) =
  \{(\psi, dQ_\psi) \mid \psi \in \F^{\partial N} \} \subseteq \F^{\partial
  N} \oplus (\F^{\partial N})^\ast\Big)
\]
taking a quadratic form $Q$ on the vector space $\F^{\partial N}$
over the field $\F$ to the graph
of its differential $dQ$. Here we think of the symplectic vector space
$\F^{\partial N} \oplus (\F^{\partial N})^\ast$ as the state space of the
circuit, and the subspace $\mathrm{Graph}(dQ)$ as the subspace of attainable
states, with $\psi \in \F^{\partial N}$ describing the potentials at the
terminals, and $dQ_\psi \in (\F^{\partial N})^\ast$ the currents. 

This construction is well-known in classical mechanics \cite{Weinstein}, where the principle of least action plays a role analogous to that of the principle of minimum power here.   The set of Lagrangian subspaces is actually an algebraic variety,
the `Lagrangian Grassmannian', which serves as a compactification of the
space of quadratic forms.  The Lagrangian Grassmannian has already played a
role in Sabot's work on circuits made of resistors \cite{Sabot1997,Sabot2004}.
For us, its importance it that we can find identity morphisms
for the composition of Dirichlet forms by taking circuits made of parallel resistors
and letting their resistances tend to zero: the limit is not a Dirichlet form, but
it exists in the Lagrangian Grassmannian.    Indeed, 
there exists a category $\LagrRel$ with finite dimensional
symplectic vector spaces as objects and `Lagrangian relations' as morphisms: 
that is, linear relations from $V$ to $W$ that are given by Lagrangian subspaces of $\overline{V} \oplus W$, where $\overline{V}$ is the symplectic vector space conjugate to $V$.   

To move from the Lagrangian subspace defined by the graph of the differential of
the power functional to a morphism in the category $\LagrRel$---that
is, to a Lagrangian relation---we must treat seriously the input and output
functions of the circuit. These express the circuit as built upon a cospan   
\[
  \xymatrix{
    & N \\
    X \ar[ur]^{i} && Y. \ar[ul]_o
  }
\]
Applicable far more broadly than this present formalization of circuits, cospans
model systems with two `ends', an input and output end, albeit without any
connotation of directionality: we might just as well exchange the role of the
inputs and outputs by taking the mirror image of the above diagram. The role of
the input and output functions, as we have discussed, is to mark the terminals
we may glue onto the terminals of another circuit, and the pushout of cospans
gives formal precision to this gluing construction.

One upshot of this cospan framework is that we may consider circuits with elements
of $N$ that are both inputs and outputs, such as this one:
\[
  \begin{tikzpicture}[circuit ee IEC, set resistor graphic=var resistor iec graphic]
    \node[contact] (c1) at (0,2) {};
    \node[contact] (c2) at (0,0) {};
    \node(input) at (-2,1) {\small{\textsf{inputs}}};
    \node(output) at (2,1) {\small{\textsf{outputs}}};
    \path[color=gray, very thick, shorten >=10pt, ->, >=stealth, bend left]
    (input) edge (c1);		
    \path[color=gray, very thick, shorten >=10pt, ->, >=stealth, bend right]
    (input) edge (c2);	
    \path[color=gray, very thick, shorten >=10pt, ->, >=stealth, bend right]
    (output) edge (c1);
    \path[color=gray, very thick, shorten >=10pt, ->, >=stealth, bend left]
    (output) edge (c2);
  \end{tikzpicture}
\]
This corresponds to the identity morphism on the finite set with two elements.
Another is that some points may be considered an input or output multiple
times; we draw this:
\[
  \begin{tikzpicture}[circuit ee IEC, set resistor graphic=var resistor IEC graphic]
    \node[contact] (I1) at (0,0) {};
    \node[contact] (O1) at (3,0) {};
    \node(input) at (-2,0) {\small{\textsf{inputs}}};
    \node(output) at (5,0) {\small{\textsf{outputs}}};
    \draw (I1) 	to [resistor] node [label={[label distance=3pt]90:{$1\Omega$}}]
    {} (O1);
    \path[color=gray, very thick, shorten >=10pt, ->, >=stealth, bend left] (input)
    edge (I1);		
    \path[color=gray, very thick, shorten >=10pt, ->,
    >=stealth, bend right] (input) edge (I1);		
    \path[color=gray, very thick, shorten >=10pt, ->, >=stealth] (output) edge (O1);
  \end{tikzpicture}
\]
This allows us to connect two distinct outputs to the above double
input.

Given a set $X$ of inputs or outputs, we understand the electrical behavior on this set 
by considering the symplectic vector space $\vectf{X}$, the direct sum of the space
$\F^X$ of potentials and the space ${(\F^X)}^\ast$ of currents at these points.
A Lagrangian relation specifies which states of the output space $\vectf{Y}$ are
allowed for each state of the input space $\vectf{X}$.
Turning the Lagrangian subspace $\mathrm{Graph}(dQ)$ of a circuit into this
information requires that we understand the `symplectification' 
\[  Sf\maps \vectf{B} \to \vectf{A} \] 
and `twisted symplectification'
\[  S^tf\maps \vectf{B} \to \overline{\vectf{A}}\]
of a function $f\maps A \to B$ between finite sets.  In particular we need to understand how these apply to the input and output functions with codomain restricted to $\partial N$; abusing notation, we also write these $i\maps X \to \partial N$ and $o\maps Y \to \partial N$.

The symplectification is a Lagrangian relation, and the catch
phrase is that it `copies voltages' and `splits currents'.  More precisely,
for any given potential-current pair $(\psi,\iota)$ in $\vectf{B}$, its image
under $Sf$ comprises all elements of $(\psi', \iota') \in \vectf{A}$ such that
the potential at $a \in A$ is equal to the potential at $f(a) \in B$, and such
that, for each fixed $b \in B$, collectively the currents at the $a \in
f^{-1}(b)$ sum to the current at $b$.  We use the symplectification $So$ of the
output function to relate the state on $\partial N$ to that on the
outputs $Y$. As our current framework is set up to report the current \emph{out}
of each node, to describe input currents we define the twisted symplectification
$S^tf\maps \vectf{B} \to \overline{\vectf{A}}$ almost identically to the above, except that we flip the sign of the currents $\iota' \in (\F^A)^\ast$.  We use the twisted symplectification $S^ti$ of the input function to relate the state on $\partial N$
to that on the inputs.

The Lagrangian relation corresponding to a circuit is then the set of all
potential--current pairs that are possible at the inputs and outputs of that circuit. 
For instance, consider a resistor of resistance $r$, with one end considered as an
input and the other as an output:
\[
  \begin{tikzpicture}[circuit ee IEC, set resistor graphic=var resistor IEC graphic]
    \node[contact] (I1) at (0,0) {};
    \node[contact] (O1) at (3,0) {};
    \node(input) at (-2,0) {\small{\textsf{input}}};
    \node(output) at (5,0) {\small{\textsf{output}}};
    \draw (I1) 	to [resistor] node [label={[label distance=3pt]90:{$r$}}] {} (O1);
    \path[color=gray, very thick, shorten >=10pt, ->, >=stealth] (input)
    edge (I1);
    \path[color=gray, very thick, shorten >=10pt, ->, >=stealth] (output) edge (O1);
  \end{tikzpicture}
\]
To obtain the corresponding Lagrangian relation, we must first specify domain and
codomain symplectic vector spaces. In this case, as the input and output sets
each consist of a single point, these vector spaces are both $\F \oplus \F^\ast$,
where the first summand is understood as the space of potentials, and the second
the space of currents.

Now, the resistor has power functional $Q\maps \F^2 \to \F$ given by
\[   Q(\psi_1,\psi_2) = \frac1{2r}(\psi_2-\psi_1)^2, \]
and the graph of the differential of $Q$ is
\[
  \mathrm{Graph}(dQ) = \big\{\big(\psi_1,\psi_2,
  \tfrac1r(\psi_1-\psi_2),\tfrac1r(\psi_2-\psi_1)\big) \,\big|\, \psi_1,\psi_2 \in
  \F\big\} \subseteq \F^2 \oplus (\F^2)^\ast.
\]
In this example the input and output functions $i,o$ are simply the identity
functions on a one element set, so the symplectification of the output function
is simply the identity linear transformation, and the twisted symplectification
of the input function is the isomorphism  between conjugate
symplectic vector spaces $\F\oplus\F^\ast \to \overline{\F\oplus\F^\ast}$ mapping $(\phi,i)$ to $(\phi,-i)$ This implies that the behavior associated to this
circuit is the Lagrangian relation
\[
  \big\{(\psi_1,i,\psi_2,i) \,\big|\, \psi_1,\psi_2 \in \F, i =
  \tfrac1r(\psi_2-\psi_1)\big\}\subseteq \overline{\F \oplus \F^\ast} \oplus \F
    \oplus \F^\ast.
\]
This is precisely the set of potential-current pairs that are allowed at the
input and output of a resistor of resistance $r$.  In particular, the relation
$i = \tfrac1r(\psi_2-\psi_1)$ is well-known in electrical engineering: it is
`Ohm's law'.

A crucial fact is that the process of mapping a circuit to its corresponding
Lagrangian relation identifies distinct circuits.  For example, a single 2-ohm resistor:
\[
  \begin{tikzpicture}[circuit ee IEC, set resistor graphic=var resistor IEC graphic]
    \node[contact] (I1) at (0,0) {};
    \node[contact] (O1) at (3,0) {};
    \node(input) at (-2,0) {\small{\textsf{input}}};
    \node(output) at (5,0) {\small{\textsf{output}}};
    \draw (I1) 	to [resistor] node [label={[label distance=3pt]90:{$2\Omega$}}] {} (O1);
    \path[color=gray, very thick, shorten >=10pt, ->, >=stealth] (input)
    edge (I1);
    \path[color=gray, very thick, shorten >=10pt, ->, >=stealth] (output) edge (O1);
  \end{tikzpicture}
\]
has the same Lagrangian relation as two 1-ohm resistors in series:
\[
  \begin{tikzpicture}[circuit ee IEC, set resistor graphic=var resistor IEC graphic]
    \node[contact] (I1) at (0,0) {};
    \node[circle, minimum width = 3pt, inner sep = 0pt, fill=black] (int) at
    (3,0) {};
    \node[contact] (O1) at (6,0) {};
    \node(input) at (-2,0) {\small{\textsf{input}}};
    \node(output) at (8,0) {\small{\textsf{output}}};
    \draw (I1) 	to [resistor] node [label={[label distance=3pt]90:{$1\Omega$}}] {} (int)
    to [resistor] node [label={[label distance=3pt]90:{$1\Omega$}}] {} (O1);
    \path[color=gray, very thick, shorten >=10pt, ->, >=stealth] (input)
    edge (I1);
    \path[color=gray, very thick, shorten >=10pt, ->, >=stealth] (output) edge (O1);
  \end{tikzpicture}
\]
The Lagrangian relation does not shed any light on the internal workings of a
circuit.  Thus, we call the process of computing this relation `black boxing':
it is like encasing the circuit in an opaque box, leaving only its terminals
accessible. Fortunately, the Lagrangian relation of a circuit is enough to
completely characterize its external behavior, including how it interacts when
connected with other circuits. 

Put more precisely, the black boxing process is \emph{functorial}: we can 
compute the black boxed version of a circuit made of parts by computing the
black boxed versions of the parts and then composing them.   In fact we shall 
prove that $\Circ$ and $\LagrRel$ are dagger compact categories, and
the black box functor preserves all this extra structure:

\begin{theorem} \label{main_theorem}
  There exists a symmetric monoidal dagger functor, the \define{black box functor}   
  \[ \blacksquare\maps \Circ \to \LagrRel, \]
   mapping a finite set $X$ to the symplectic vector space
  $\vectf{X}$ it generates, and a circuit $\big((N,E,s,t,r),i,o\big)$ to the Lagrangian     
  relation 
  \[
    \bigcup_{v \in \mathrm{Graph}(dQ)} S^ti(v) \times So(v)
    \subseteq \overline{\F^X \oplus (\F^X)^\ast} \oplus \F^Y \oplus (\F^Y)^\ast,
  \]
  where $Q$ is the circuit's power functional.
\end{theorem}

The goal of this paper is to prove and explain this result. The proof itself is
more tricky than one might first expect, but our approach introduces various
concepts that should be useful throughout the study of networks, such as
`decorated cospans' and `corelations'.  These provide a general framework for
discussing open networked systems---and not only the passive linear systems
discussed here, but also others, such as Markov processes \cite{BFP}.

Cospans are already familiar as a formalism for making entities with an
arbitrarily designated `input end' and `output end' into the morphisms of a
category.  For example, in topological quantum field theory we use special
cospans called `cobordisms' to describe pieces of spacetime \cite{BL,BaezStay}.
Earlier we introduced `decorated cospans' to describe
circuits with specified inputs and outputs.  Later, with the help of machinery
developed in a companion paper \cite{Fon}, we use decorated cospans to set up
several functors that appear as factors of the black box functor, as steps in 
proving its functoriality.

Just as a relation is an isomorphism class of spans of a special sort, namely
jointly monic spans, a `corelation' is an isomorphism class of jointly epic
cospans.  While corelations are less widely used than relations, they turn out
to be perfectly suited for describing circuits of ideal perfectly conductive
wires---precisely the class of circuits that cannot be modeled simply as
Dirichlet forms, since the power functional becomes infinite when it includes
terms where the resistance is zero.  We introduce corelations in Section
\ref{sec:corel}, and they too play a key role in constructing the black box
functor.

With these tools in hand, the black box functor turns out to rely on a tight
relationship between Kirchhoff's laws, the minimization of Dirichlet forms, and
the `symplectification' of corelations. It is well-known that away from the terminals,
a circuit must obey two rules known as Kirchhoff's laws.  We have already noted that 
the principle of minimum power states that a circuit will `choose' potentials on its interior
that minimize the power functional.  We clarify the relation between these points
in Theorems \ref{thm:realizablepotentials} and \ref{thm:dirichletminimization}, 
which together show that minimizing a Dirichlet form over some subset amounts to
assuming that the corresponding circuit obeys Kirchhoff's laws on that subset.

We have also mentioned the symplectification of functions above.  Extending this to
allow symplectification of corelations, this process gives a map sending corelations
to Lagrangian relations that describe the behavior of ideal perfectly conductive wires. 
We prove that these symplectified corelations simultaneously impose Kirchhoff's 
laws (Proposition \ref{prop:sympfunctor} and Example \ref{ex:sympfunction}) and 
accomplish the minimization of Dirichlet forms (Theorem \ref{thm:sympmin}).  

Together, our results show that these three concepts---Kirchhoff's laws from circuit theory, 
the analytic idea of minimizing power dissipation, and the algebraic idea of symplectification of corelations---are merely different faces of one law: the law of composition of circuits.

\subsection{Finding your way through this paper}
This paper is split into three parts, addressing in turn the questions:
\begin{enumerate}[I.]
  \item What do circuit diagrams mean?
  \item How do we interact with circuit diagrams?
  \item How is meaning preserved under these interactions?
\end{enumerate}

We begin Part , on the semantics of circuit diagrams, with
a discussion of circuits of linear resistors, developing the intuition for the
governing laws of passive linear circuits---Ohm's law, Kirchhoff's voltage law,
and Kirchhoff's current law---in a time-independent setting (Section
\ref{sec:resistors}). This allows us to develop the concept of
Dirichlet form as a representation of power consumption, and understand their
composition as minimizing power, an expression of the current law. In Section
\ref{sec:plcs}, the Laplace transform then allows us to recapitulate these ideas
after introducing inductors and capacitors, speaking of impedance where we
formerly spoke of resistance, and generalizing Dirichlet forms from the field
$\R$ to the field $\R(s)$ of real rational functions. While in this setting the
principle of minimum power is replaced by a variational principle,
the intuitions gained from circuits of resistors still remain useful. 

At the end of this part, we show that Dirichlet forms alone do not provide the 
flexibility to construct a category representing the semantics of circuit diagrams. 
The fundamental reason is that they do not allow us to describe circuits involving 
ideal perfectly conductive wires.  This motivates the development of more powerful machinery.

Part , on the syntax of circuit diagrams, contains the main
technical contributions of the paper. It begins with Section,
which develops machinery to construct what we call categories of decorated
cospans. These are categories where the objects are finite sets and the morphisms
are cospans in the category of finite sets together with some extra structure on
the apex. Circuits, as defined above, are naturally an example of such a
construction, and Section \ref{sec:circdef} lays out the details of this.
In Section \ref{sec:circlagr} we then review the basic theory of linear
Lagrangian relations, giving details to the correspondence we have defined
between Dirichlet forms, and hence passive linear circuits, and Lagrangian
relations. Section \ref{sec:corel} then takes immediate advantage of the added
flexibility of Lagrangian relations, discussing the `trivial' circuits
comprising only perfectly conductive wires, which mediate the notion of composition
of circuits.

Having introduced these prerequisites, we get to the point in Part .
In Section \ref{sec:blackbox} we introduce the black box functor,  and  in Section \ref{sec:proof} we prove our main result.

\section{Circuits of linear resistors} \label{sec:resistors}
%%fakesubsection
In this part we review the meaning of circuit diagrams comprising resistors,
inductors, and capacitors, giving an answer to the question ``What do circuit
diagrams mean?''. 

To elaborate, while circuit diagrams model electric circuits according to their
physical form, another, often more relevant, way to understand a circuit is by
its external behavior. This means the following. To an electric circuit we
associate two quantities to each edge: voltage and current. We are not free,
however, to choose these quantities as we like; circuits are subject to
governing laws that imply voltages and currents must obey certain relationships.
From the perspective of control theory we are particularly interested in the
values these quantities take at the so-called terminals, and how altering one
value will affect the other values. We call two circuits equivalent when they
determine the same relationship. Our main task in this first part is to explore
when two circuits are equivalent.

In order to let physical intuition lead the way, we begin by specialising to the
case of linear resistors. In this section we describe how to find the function
of a circuit from its form, advocating in particular the perspective of the
principle of minimum power. This allows us to identify the external behavior of a
circuit with a so-called Dirichlet form representing the dependence of its power
consumption on potentials at its terminals.

\subsection{Circuits as labelled graphs}

The concept of an abstract open electrical circuit made of linear resistors is
well-known in electrical engineering, but we shall need to formalize it with
more precision than usual.  The basic idea is that a circuit of linear resistors
is a graph whose edges are labelled by positive real numbers called
`resistances', and whose sets of vertices is equipped with two subsets: the
`inputs' and `outputs'. This unfolds as follows.

A (closed) circuit of resistors looks like this: 
\[
\begin{tikzpicture}[circuit ee IEC, set resistor graphic=var resistor IEC graphic]
\node (I1) at (0,0) {};
\node (I2) at (0,2) {};
\node (O1) at (5.83,1) {};
\draw (I1) 	to [resistor] node [label={[label distance=2pt]275:{$1\Omega$}}] {} (2.83,1);
\draw (I2)	to [resistor] node [label={[label distance=2pt]85:{$1\Omega$}}] {} (2.83,1)
				to [resistor] node [label={[label distance=3pt]90:{$2\Omega$}}] {} (O1);
\end{tikzpicture}
\]
We can consider this a labelled graph, with each resistor an edge of the graph,
its resistance its label, and the vertices of the graph the points at which
resistors are connected. 

A circuit is `open' if it can be connected to other circuits. To do this we
first mark points at which connections can be made by denoting some vertices as
input and output terminals:
\[
\begin{tikzpicture}[circuit ee IEC, set resistor graphic=var resistor IEC graphic]
\node[contact] (I1) at (0,2) {};
\node[contact] (I2) at (0,0) {};
\node[contact] (O1) at (5.83,1) {};
\node(input) at (-2,1) {\small{\textsf{inputs}}};
\node(output) at (7.83,1) {\small{\textsf{outputs}}};
\draw (I1) 	to [resistor] node [label={[label distance=2pt]85:{$1\Omega$}}] {} (2.83,1);
\draw (I2)	to [resistor] node [label={[label distance=2pt]275:{$1\Omega$}}] {} (2.83,1)
				to [resistor] node [label={[label distance=3pt]90:{$2\Omega$}}] {} (O1);
\path[color=gray, very thick, shorten >=10pt, ->, >=stealth, bend left] (input) edge (I1);		\path[color=gray, very thick, shorten >=10pt, ->, >=stealth, bend right] (input) edge (I2);		
\path[color=gray, very thick, shorten >=10pt, ->, >=stealth] (output) edge (O1);
\end{tikzpicture}
\]
Then, given a second circuit, we may choose a relation between the output set of
the first and the input set of this second circuit, such as the simple relation
of the single output vertex of the circuit above with the single input vertex of
the circuit below.
\[
\begin{tikzpicture}[circuit ee IEC, set resistor graphic=var resistor IEC graphic]
\node[contact] (I1) at (0,1) {};
\node[contact] (O1) at (2.83,2) {};
\node[contact] (O2) at (2.83,0) {};
\node (input) at (-2,1) {\small{\textsf{inputs}}};
\node (output) at (4.83,1) {\small{\textsf{outputs}}};
\draw (I1) 	to [resistor] node [label={[label distance=2pt]95:{$1\Omega$}}] {} (O1);
\draw (I1)		to [resistor] node [label={[label distance=2pt]265:{$3\Omega$}}] {} (O2);
\path[color=gray, very thick, shorten >=10pt, ->, >=stealth] (input) edge (I1);		\path[color=gray, very thick, shorten >=10pt, ->, >=stealth, bend right] (output) edge (O1);
\path[color=gray, very thick, shorten >=10pt, ->, >=stealth, bend left] (output) edge (O2);
\end{tikzpicture}
\]
We connect the two circuits by identifying output and input vertices according
to this relation, giving in this case the composite circuit:
\[
\begin{tikzpicture}[circuit ee IEC, set resistor graphic=var resistor IEC graphic]
\node[contact] (I1) at (0,2) {};
\node[contact] (I2) at (0,0) {};
\coordinate (int1) at (2.83,1) {};
\coordinate (int2) at (5.83,1) {};
\node[contact] (O1) at (8.66,2) {};
\node[contact] (O2) at (8.66,0) {};
\node (input) at (-2,1) {\small{\textsf{inputs}}};
\node (output) at (10.66,1) {\small{\textsf{outputs}}};
\draw (I1) 	to [resistor] node [label={[label distance=2pt]85:{$1\Omega$}}] {} (int1);
\draw (I2)	to [resistor] node [label={[label distance=2pt]275:{$1\Omega$}}] {} (int1)
				to [resistor] node [label={[label distance=3pt]90:{$2\Omega$}}] {} (int2);
\draw (int2) 	to [resistor] node [label={[label distance=2pt]95:{$1\Omega$}}] {} (O1);
\draw (int2)		to [resistor] node [label={[label distance=2pt]265:{$3\Omega$}}] {} (O2);
\path[color=gray, very thick, shorten >=10pt, ->, >=stealth, bend left] (input) edge (I1);		\path[color=gray, very thick, shorten >=10pt, ->, >=stealth, bend right] (input) edge (I2);		
\path[color=gray, very thick, shorten >=10pt, ->, >=stealth, bend right] (output) edge (O1);
\path[color=gray, very thick, shorten >=10pt, ->, >=stealth, bend left] (output) edge (O2);
\end{tikzpicture}
\]

\vskip 1em

More formally, we define a \define{graph}\footnote{In this paper we refer
to directed multigraphs simply as graphs.} to be a pair of functions $s,t\maps E \to N$ where $E$ and $N$ are finite sets.  We call elements of $E$ \define{edges} and elements of $N$ \define{vertices} or
\define{nodes}.  We say that the edge $e \in E$ has \define{source} $s(e)$ and
\define{target} $t(e)$, and also say that $e$ is an edge \define{from} $s(e)$
\define{to} $t(e)$.

To study circuits we need graphs with labelled edges:

\begin{definition}
Given a set $L$ of \define{labels}, an \define{$L$-graph} is a graph equipped with a function $r\maps E \to L$:
\[
\xymatrix{
L & E \ar@<2.5pt>[r]^{s} \ar@<-2.5pt>[r]_{t} \ar[l]_{r} & N.
}
\]
\end{definition}

For circuits made of resistors we take $L = (0,\infty)$, but later we shall
take $L$ to be a set of positive elements in some more general field.  In either
case, a circuit will be an $L$-graph with some extra structure:

\begin{definition} \label{def_circuit}
Given a set $L$, a \define{circuit over $L$} is an $L$-graph $\xymatrix{
L & E \ar@<2.5pt>[r]^{s} \ar@<-2.5pt>[r]_{t} \ar[l]_{r} & N}$ together with finite sets $X$, $Y$, and functions $i \maps X \to N$ and $o\maps Y \to  N$. We call the sets $i(X)$, $o(Y)$, and $\partial N = i(X) \cup o(Y)$ the \define{inputs},  \define{outputs}, and \define{terminals} or \define{boundary} of the circuit, respectively.
\end{definition}

We will later make use of the notion of connectedness in graphs. Recall that
given two vertices $v, w \in N$ of a graph, a \define{path from $v$ to $w$} is a
finite sequence of vertices $v = v_0, v_1, \dots , v_n = w$ and edges $e_1,
\dots , e_n$ such that for each $1 \le i \le n$, either $e_i$ is an edge from
$v_i$ to $v_{i+1}$, or an edge from $v_{i+1}$ to $v_i$. A subset $S$ of the
vertices of a graph is \define{connected} if, for each pair of vertices in $S$,
there is a path from one to the other. A \define{connected component} of a graph
is a maximal connected subset of its vertices.\footnote{In the theory of
directed graphs the qualifier `weakly' is commonly used before the word
`connected' in these two definitions, in distinction from a stronger notion of
connectedness requiring paths to respect edge directions. As we never consider
any other sort of connectedness, we omit this qualifier.}

In the rest of this section we take $L = (0,\infty) \subseteq \R$ and fix a circuit over 
$(0,\infty)$.  The edges of this circuit should be thought of as `wires'.  The label 
$r_e \in (0,\infty)$ stands for the \define{resistance} of the resistor on the wire $e$.   
There will also be a voltage and current on each wire.  In this section, these will
be specified by functions $V \in \R^E$ and $I \in \R^E$.  Here, as customary in
engineering, we use $I$ for `intensity of current', following Amp\`ere.  

\subsection{Ohm's law, Kirchhoff's laws, and the principle of minimum power}

In 1827, Georg Ohm published a book which included a relation between the voltage
and current for circuits made of resistors \cite{O}.  At the time, the critical
reception was harsh: one contemporary called Ohm's work ``a web of naked
fancies, which can never find the semblance of support from even the most
superficial of observations'', and the German Minister of Education said that a
professor who preached such heresies was unworthy to teach science \cite{D,H}.
However, a simplified version of his relation is now widely used under the name
of `Ohm's law'. We say that \define{Ohm's law} holds if for all edges $e \in
E$ the voltage and current functions of a circuit obey:
\[ 
V(e) = r(e) I(e).  \label{ohm}  
\]

Kirchhoff's laws date to Gustav Kirchhoff in 1845, generalising Ohm's work. They
were in turn generalized into Maxwell's equations a few decades later. We say
\define{Kirchhoff's voltage law} holds if there exists $\phi \in \R^N$ such that
\[
V(e) = \phi(t(e)) - \phi(s(e)).
\]
We call the function $\phi$ a \define{potential}, and think of it as assigning
an electrical potential to each node in the circuit. The voltage then arises as
the differences in potentials between adjacent nodes. If Kirchhoff's voltage law
holds for some voltage $V$, the potential $\phi$ is unique only in the trivial
case of the empty circuit: when the set of nodes $N$ is empty. Indeed, two
potentials define the same voltage function if and only if their difference is
constant on each connected component of the graph $\Gamma$.

We say \define{Kirchhoff's current law} holds if for all nonterminal nodes $n
\in N\setminus \partial N$ we have
\[ 
\sum_{s(e) = n} I(e) = \sum_{t(e) = n} I(e).  \label{kcl}  
\]  
This is an expression of conservation of charge within the circuit; it says that
the total current flowing in or out of any nonterminal node is zero. Even when
Kirchhoff's current law is obeyed, terminals need not be sites of zero net
current; we call the function $\iota \in \R^{\partial N}$ that takes a terminal
to the difference between the outward and inward flowing currents,
\begin{align*}
\iota:\partial N &\longrightarrow \R \\
n &\longmapsto \sum_{t(e) = n} I(e) -\sum_{s(e) = n} I(e),
\end{align*}
the \define{boundary current} for $I$.

A \define{boundary potential} is also a function in $\R^{\partial N}$, but
instead thought of as specifying potentials on the terminals of a
circuit. As we think of our circuits as open circuits, with the terminals points
of interaction with the external world, we shall think of these potentials as
variables that are free for us to choose. Using the above three
principles---Ohm's law, Kirchhoff's voltage law, and Kirchhoff's current
law---it is possible to show that choosing a boundary potential determines
unique voltage and current functions on that circuit. 

The so-called `principle of minimum power' gives some insight into how this
occurs, by describing a way potentials on the terminals might determine
potentials at all nodes. From this, Kirchhoff's voltage law then gives rise to a
voltage function on the edges, and Ohm's law gives us a current function too. We
shall show, in fact, that a potential satisfies the principle of minimum power
for a given boundary potential if and only if this current obeys Kirchhoff's
current law.

A circuit with current $I$ and voltage $V$ dissipates energy at a rate
equal to
\[
 \sum_{e \in E} I(e)V(e).
\]  
Ohm's law allows us to rewrite $I$ as $V/r$, while Kirchhoff's voltage law gives
us a potential $\phi$ such that $V(e)$ can be written as
$\phi(t(e))-\phi(s(e))$, so for a circuit obeying these two laws the power can
also be expressed in terms of this potential. We thus arrive at a functional
mapping potentials $\phi$ to the power dissipated by the circuit when Ohm's law
and Kirchhoff's voltage law are obeyed for $\phi$. 

\begin{definition}
The \define{extended power functional} $P\maps \R^N \to \R$ of a circuit is
defined by
\[
P(\phi) =\frac{1}{2} \sum_{e \in E} \frac{1}{r(e)}\big(\phi(t(e))-\phi(s(e))\big)^2.
\]
\end{definition}

\noindent
The factor of $\frac{1}{2}$ is inserted to cancel the factor of 2 that appears when
we differentiate this expression.  We call $P$ the \emph{extended} power functional as we shall see that it is defined even on potentials that are not compatible with the three governing laws of electric circuits. We shall later restrict the domain of this functional so that it is defined precisely on those potentials that \emph{are} compatible with the
governing laws. Note that this functional does not depend on the directions
chosen for the edges of the circuit.

This expression lets us formulate the `principle of minimum power', which gives
us information about the potential $\phi$ given its restriction to the boundary
of $\Gamma$. Call a potential $\phi \in \R^N$ an \define{extension} of a
boundary potential $\psi \in \R^{\partial N}$ if $\phi$ is equal to $\psi$ when
restricted to $\R^{\partial N}$---that is, if $\phi|_{\partial N} = \psi$. 

\begin{definition}
We say a potential $\phi \in \R^{N}$ \define{obeys the principle of minimum
power} for a boundary potential $\psi \in \R^{\partial N}$ if $\phi$ minimizes
the extended power functional $P$ subject to the constraint that  $\phi$ is an
extension of $\psi$. 
\end{definition}

As promised, in the presence of Ohm's law and Kirchhoff's voltage law, the
principle of minimum power is equivalent to Kirchhoff's current law.

\begin{proposition} \label{minimum_power_implies_kirchhoff_current}
Let $\phi$ be a potential extending some boundary potential $\psi$. Then $\phi$
obeys the principle of minimum power for $\psi$ if and only if the 
current 
\[  I(e) = \frac1{r(e)}(\phi(t(e))-\phi(s(e))) \] 
obeys Kirchhoff's current law.
\end{proposition}

\begin{proof}
Fixing the potentials at the terminals to be those given by the boundary
potential $\psi$, the power is a nonnegative quadratic function of the
potentials at the nonterminals. This implies that an extension $\phi$ of $\psi$
minimizes $P$ precisely when 
\[ \left. \frac{\partial P(\varphi)}{\partial \varphi(n)}\right|_{\varphi = \phi} = 0 \]
for all nonterminals $n \in N \setminus \partial N$. Note that the
partial derivative of the power with respect to the potential at $n$ is given by 
\begin{align*}
  \frac{\partial P}{\partial \varphi(n)}\bigg|_{\varphi = \phi} 
  &= \sum_{t(e) = n} \frac1{r(e)}\big(\phi(t(e))-\phi(s(e))\big) - \sum_{s(e) =
  n} \frac1{r(e)}\big(\phi(t(e))-\phi(s(e))\big) \\
  &= \sum_{t(e) = n} I(e) - \sum_{s(e) = n} I(e).
\end{align*}
Thus $\phi$ obeys the principle of minimum power for $\psi$ if and only if
\[ \sum_{s(e) = n} I(e) = \sum_{t(e) = n} I(e)\] 
for all $n \in N \setminus \partial N$, and so if and only if Kirchhoff's current law holds.
\end{proof}

\subsection{A Dirichlet problem}

We remind ourselves that we are in the midst of understanding circuits as objects that define relationships between boundary potentials and boundary currents. This relationship is defined by the stipulation that voltage--current pairs on a circuit must obey Ohm's law and Kirchhoff's laws---or equivalently, Ohm's law, Kirchhoff's voltage law, and the principle of minimum power. In this subsection we show these conditions imply that for each boundary potential $\psi$ on the circuit there exists a potential $\phi$ on the circuit extending $\psi$, unique up to what may be interpreted as a choice of reference potential on each connected component of the circuit. From this potential $\phi$ we can then compute the unique voltage, current, and boundary current functions compatible with the given boundary potential.

Fix again a circuit with extended power functional $P\maps \R^N \to \R$. Let $\nabla\maps \R^{N} \to \R^{N}$ be the operator that maps a potential $\phi \in \R^N$ to the function from $N$ to $\R$ given by
\[
n \longmapsto \frac{\partial P}{\partial \varphi(n)}\bigg|_{\varphi = \phi} \;.
\]
As we have seen, this function takes potentials to twice the pointwise currents that they induce. We have also seen that a potential $\phi$ is compatible with the governing laws of circuits if and only if
\begin{equation}
\nabla \phi \big|_{\R^{\partial N}} = 0 .\label{dirichlet}
\end{equation}
The operator $\nabla$ acts as a discrete analogue of the Laplacian for the graph $\Gamma$, so we call this operator the \define{Laplacian} of $\Gamma$, and say that  equation \eqref{dirichlet} is a version of Laplace's equation. We then say that the problem of finding an extension $\phi$ of some fixed boundary potential $\psi$ that solves this Laplace's equation---or, equivalently, the problem of finding a $\phi$ that obeys the principle of minimum power for $\psi$---is a discrete version of the \define{Dirichlet problem}. 

As we shall see, this version of the Dirichlet problem always has a solution.  However, the solution is not necessarily unique.  If we take a solution $\phi$ and some $\alpha \in \R^N$ that is constant on each connected component and vanishes on the boundary of $\Gamma$, it is clear that $\phi+\alpha$ is still an extension of $\psi$ and that 
\[
\left.\frac{\partial P(\varphi)}{\partial \varphi(n)}\right|_{\varphi = \phi} = 
\left.\frac{\partial P(\varphi)}{\partial \varphi(n)}\right|_{\varphi = \phi + \alpha},
\] 
so $\phi + \alpha$ is another solution. We say that a connected component of a circuit \define{touches the boundary} if it contains a vertex in $\partial N$. Note that such an $\alpha$ must vanish on all connected components touching the boundary.

With these preliminaries in hand, we can solve the Dirichlet problem:
\begin{proposition} \label{dirichlet_problem}
For any boundary potential $\psi \in \R^{\partial N}$ there exists a potential $\phi$ obeying the principle of minimum power for $\psi$.  If we also demand that $\phi$ vanish on every connected component of $\Gamma$ not touching the boundary, then $\phi$ is unique. 
\end{proposition}
\begin{proof}
For existence, observe that the power is a nonnegative quadratic form, the extensions of $\psi$ form an affine subspace of $\R^N$, and a nonnegative quadratic form restricted to an affine subspace of a real vector space must reach a minimum somewhere on this subspace. 

For uniqueness, suppose that both $\phi$ and $\phi'$ obey the principle of minimum power for $\psi$. Let 
\[
\alpha = \phi'-\phi.
\]
Then 
\[
\alpha\big|_{\partial N} = \phi'\big|_{\partial N}-\phi\big|_{\partial N} = \psi-\psi =0,
\] 
so $\phi+\lambda\alpha$ is an extension of $\psi$ for all $\lambda \in \R$. This implies that
\[
f(\lambda) := P(\phi+\lambda\alpha)
\]
is a smooth function attaining its minimum value at both $t =0$ and $t =1$. In particular, this implies that $f'(0)=0$. But this means that when writing $f$ as a quadratic, the coefficient of $\lambda$ must be $0$, so we can write
\begin{align*}
2f(\lambda) &= \sum_{e \in E} \frac1{r(e)}\big((\phi+\lambda\alpha)(t(e))-(\phi+\lambda\alpha)(s(e))\big)^2 \\
&= \sum_{e \in E} \frac1{r(e)}\Big(\big(\phi(t(e))-\phi(s(e))\big)+\lambda\big(\alpha(t(e))-\alpha(s(e))\big)\Big)^2 \\
&=  \sum_{e \in E} \frac1{r(e)}\big(\phi(t(e))-\phi(s(e))\big)^2 + \textrm{$\lambda$-term} +  \lambda^2 \sum_{e \in E} \frac1{r(e)}\big(\alpha(t(e))-\alpha(s(e))\big)^2 \\
&=  \sum_{e \in E} \frac1{r(e)}\big(\phi(t(e))-\phi(s(e))\big)^2 + \lambda^2 \sum_{e \in E} \frac1{r(e)}\big(\alpha(t(e))-\alpha(s(e))\big)^2.
\end{align*}
Then
\[
f(1) - f(0) 
= \frac{1}{2}\sum_{e \in E} \frac1{r(e)}\big(\alpha(t(e))-\alpha(s(e))\big)^2 =0,
\]
so $\alpha(t(e)) = \alpha(s(e))$ for every edge $e \in E$. This implies that $\a$ is constant on each connected component of the graph $\Gamma$ of our circuit. 

Note that as $\alpha|_{\partial N} = 0$, $\alpha$ vanishes on every connected component of $\Gamma$ touching the boundary. Thus, if we also require that $\phi$ and $\phi'$ vanish on every connected component of $\Gamma$ not touching the boundary, then $\alpha = \phi'-\phi$ vanishes on all connected components of $\Gamma$, and hence is identically zero. Thus $\phi' = \phi$, and this extra condition ensures a unique solution to the Dirichlet problem.
\end{proof}

We have also shown the following:

\begin{proposition}\label{dirichlet_problem_2}
Suppose $\psi \in \R^{\partial N}$ and $\phi$ is a potential obeying the principle of minimum power for $\psi$.  Then $\phi'$ obeys the principle of minimum power for $\psi$ if and only if the difference $\phi' - \phi$ is constant on every connected component of $\Gamma$ and vanishes on every connected component touching the boundary of $\Gamma$.
\end{proposition}

Furthermore, $\phi$ depends linearly on $\psi$:

\begin{proposition}\label{dirichlet_problem_3: linearity}
Fix $\psi \in \R^{\partial N}$, and suppose $\phi \in \R^N$ is the unique potential obeying the principle of minimum power for $\psi$ that vanishes on all connected components of $\Gamma$ not touching the boundary. Then $\phi$ depends linearly on $\psi$.
\end{proposition}
\begin{proof}
Fix $\psi, \psi' \in \R^{\partial N}$, and suppose $\phi, \phi' \in \R^N$ obey the principle of minimum power for $\psi,\psi'$ respectively, and that both $\phi$ and $\phi'$ vanish on all connected components of $\Gamma$ not touching the boundary. 

Then, for all $\lambda \in \R$,
\[
(\phi+\lambda\phi')\big|_{\R^{\partial N}} = \phi\big|_{\R^{\partial N}} +
\lambda\phi'\big|_{\R^{\partial N}}  = \psi + \lambda\psi'
\]
and
\[
(\nabla(\phi+\lambda\phi'))\big|_{\R^{\partial N}} = 
(\nabla\phi)\big|_{\R^{\partial N}} +
\lambda(\nabla\phi')\big|_{\R^{\partial N}}  = 0.
\]
Thus $\phi+\lambda\phi'$ solves the Dirichlet problem for $\psi+\lambda\psi'$, and thus $\phi$ depends linearly on $\psi$.
\end{proof}

Bamberg and Sternberg \cite{BS} describe another way to solve the Dirichlet problem, going back to Weyl \cite{Weyl}.

\subsection{Equivalent circuits}

We have seen that boundary potentials determine, essentially uniquely, the value of all the electric properties across the entire circuit. But from the perspective of control theory, this internal structure is irrelevant: we can only access the circuit at its terminals, and hence only need concern ourselves with the relationship between boundary potentials and boundary currents. In this section we streamline our investigations above to state the precise way in which boundary currents depend on boundary potentials. In particular, we shall see that the relationship is completely captured by the functional taking boundary potentials to the minimum power used by any extension of that boundary potential. Furthermore, each such power functional determines a different boundary potential--boundary current relationship, and so we can conclude that two circuits are equivalent if and only if they have the same power functional. 

An `external behavior', or \define{behavior} for short, is an equivalence class of circuits, where two are considered equivalent when the boundary current is the same function of the boundary potential. The idea is that the boundary current and boundary potential are all that can be observed `from outside', i.e. by making measurements at the terminals.  Restricting our attention to what can be observed by making measurements at the terminals amounts to treating a circuit as a `black box': that is, treating its interior as hidden from view.  So, two circuits give the same behavior when they behave the same as `black boxes'.

First let us check that the boundary current is a function of the boundary potential.  For this we introduce an important quadratic form on the space of boundary potentials:

\begin{definition}
The \define{power functional} $Q \maps \R^{\partial N} \to \R$ of a circuit with extended power functional $P$ is given by
\[
 Q(\psi) = \min_{\phi|_{\R^{\partial N}} = \psi } P(\phi).
\]
\end{definition}

Proposition \ref{dirichlet_problem} shows the minimum above exists, so the power functional is well-defined.  Thanks to the principle of minimum power, $Q(\psi)$ equals $\frac{1}{2}$ times the power dissipated by the circuit when the boundary voltage is $\psi$.  We will later see that in fact $Q(\psi)$ is a nonnegative quadratic form on $\R^{\partial N}$. 

Since $Q$ is a smooth real-valued function on $\R^{\partial N}$, its differential $d Q$ at any given point $\psi \in \R^{\partial N}$ defines an element of the dual space $(\R^{\partial N})^\ast$, which we denote by $d Q_\psi$.  In fact, this element is equal to the boundary current $\iota$ corresponding to the boundary voltage $\psi$:

\begin{proposition} \label{boundary_current_determines_boundary_voltage}
Suppose $\psi \in \R^{\partial N}$.  Suppose $\phi$ is any extension of $\psi$ minimizing the power. Then $dQ_\psi \in (\R^{\partial N})^\ast \cong \R^{\partial N}$ gives the boundary current of the current induced by the potential $\phi$.
\end{proposition}

\begin{proof}
Note first that while there may be several choices of $\phi$ minimizing the power subject to the constraint that $\phi|_{\R^{\partial N}} = \psi$, Proposition \ref{dirichlet_problem_2} says that the difference between any two choices vanishes on all components touching the boundary of $\Gamma$.  Thus, these two choices give the same value for the boundary current $\iota\maps \partial N \to \R$. So, with no loss of generality we may assume $\phi$ is the unique choice that vanishes on all components not touching the boundary. Write $\overline\iota\maps N \to \R$ for the extension of $\iota\maps \partial N \to \R$ to $N$ taking value $0$ on $N \setminus \partial N$. 

By Proposition \ref{dirichlet_problem_3: linearity}, there is a linear operator
\[
f\maps \R^{\partial N} \longrightarrow \R^N
\]
sending $\psi \in \R^{\partial N}$ to this choice of $\phi$, and then
\[
Q(\psi) = P(f\psi).
\]
Given any $\psi' \in \R^{\partial N}$, we thus have
\begin{align*}
dQ_\psi(\psi') &= \frac{d}{d\lambda}Q(\phi +\lambda\psi') \bigg|_{\lambda=0} \\
&= \frac{d}{d\lambda}P(f(\psi+\lambda\psi'))\bigg|_{\lambda=0} \\
&= \frac{1}{2} \frac{d}{d\lambda}\sum_{e \in E} \frac1{r(e)}\bigg(f(\psi+\lambda\psi'))(t(e))-(f(\psi+\lambda\psi'))(s(e))\bigg)^2 \bigg|_{\lambda=0} \\
&= \frac{1}{2} \frac{d}{d\lambda}\sum_{e \in E} \frac1{r(e)}\bigg((f\psi(t(e))-f\psi(s(e))) \;+\;\lambda (f\psi'(t(e))- f\psi'(s(e)))\bigg)^2 \bigg|_{\lambda=0} \\
&= \sum_{e \in E} \frac1{r(e)}(f\psi(t(e))-f\psi(s(e)))(f\psi'(t(e))- f\psi'(s(e))) \\
&= \sum_{e \in E} I(e)(f\psi'(t(e))- f\psi'(s(e))) \\
&= \sum_{n \in N}\left(\sum_{t(e) = n} I(e) - \sum_{s(e) = n} I(e)\right)f\psi'(n) \\
&= \sum_{n \in N}\overline \iota(n) f\psi'(n) \\
&= \sum_{n \in \partial N}\iota(n) \psi'(n).
\end{align*}
This shows that $dQ_\psi^\ast = \iota$, as claimed.  Note that this calculation explains why we inserted a factor of $\frac{1}{2}$ in the definition of $P$: it cancels the factor of $2$ obtained from differentating a square.
 \end{proof}

Note this only depends on $Q$, which makes no mention of the potentials at
nonterminals. This is amazing: the way power depends on boundary potentials
completely characterizes the way boundary currents depend on boundary
potentials. In particular, in Part  we shall see that this
allows us to define a composition rule for behaviors of circuits.

To demonstrate these notions, we give a basic example of equivalent circuits.

\begin{example}[Resistors in series] \label{resistors_in_series}
Resistors are said to be placed in \define{series} if they are placed end to end or, more
precisely, if they form a path with no self-intersections. It is well known that
resistors in series are equivalent to a single resistor with resistance equal to
the sum of their resistances. To prove this, consider the following circuit
comprising two resistors in series, with input $A$ and output $C$:
\[
  \begin{tikzpicture}[circuit ee IEC, set resistor graphic=var resistor IEC graphic]
    \node[contact] (I1) at (0,0) [label=left:$A$] {};
    \node[circle, minimum width = 3pt, inner sep = 0pt, fill=black] (int) at (3,0) [label=above:$B$] {};
    \node[contact] (O1) at (6,0) [label=right:$C$] {};
    \draw (I1) 	to [resistor] node [label={[label distance=3pt]90:{$r_{AB}$}}] {} (int)
    to [resistor] node [label={[label distance=3pt]90:{$r_{BC}$}}] {} (O1);
  \end{tikzpicture}
\]
Now, the extended power functional $P\maps \R^{\{A,B,C\}} \to \R$ for this circuit is
\[
P(\phi) = \frac12\left(\frac1{r_{AB}}\big(\phi(A)-\phi(B)\big)^2 +
\frac1{r_{BC}}\big(\phi(B)-\phi(C)\big)^2\right),
\]
while the power functional $Q\maps \R^{\{A,C\}} \to \R$ is given by minimization
over values of $\phi(B) = x$:
\[
Q(\psi) = \min_{x \in \R} \frac12 \left(\frac1{r_{AB}}\big(\psi(A)-x\big)^2 + \frac1{r_{BC}}\big(x-\psi(C)\big)^2 \right). 
\]
Differentiating with respect to $x$, we see that this minimum occurs when
\[
\frac1{r_{AB}}\big(x-\psi(A)\big) + \frac1{r_{BC}}\big(x-\psi(C)\big) = 0,
\]
and hence when $x$ is the $r$-weighted average of $\psi(A)$ and $\psi(C)$:
\[
x = \frac{r_{BC}\psi(A) + r_{AB}\psi(C)}{r_{BC}+ r_{AB}}.
\]
Substituting this value for $x$ into the expression for $Q$ above and simplifying gives
\[
Q(\psi) = \frac12\cdot\frac1{r_{AB}+r_{BC}}\big(\psi(A)-\psi(C)\big)^2. 
\]
This is also the power functional of the circuit
\[
\begin{tikzpicture}[circuit ee IEC, set resistor graphic=var resistor IEC graphic]
\node[contact] (I1) at (0,0) [label=left:$A$] {};
\node[contact] (O1) at (3,0) [label=right:$C$] {};
\draw (I1) 	to [resistor] node [label={[label distance=3pt]90:{$r_{AB}+r_{BC}$}}] {} (O1);
\end{tikzpicture}
\]
and so the circuits are equivalent.
\end{example}


\subsection{Dirichlet forms}

In the previous subsection we claimed that power functionals are quadratic forms
on the boundary of the circuit whose behavior they represent. They comprise, in
fact, precisely those quadratic forms known as Dirichlet forms.

\begin{definition}
Given a finite set $S$, a \define{Dirichlet form} on $S$ is a quadratic form $Q:
\mathbb{R}^S \to \mathbb{R}$ given by the formula
\[
  Q(\psi) = \sum_{i,j} c_{i j} (\psi_i - \psi_j)^2
\]
for some nonnegative real numbers $c_{i j}$.  
\end{definition}

Note that we may assume without loss of generality that $c_{i i} = 0$ and $c_{i
j} = c_{j i}$; we do this henceforth.  Any Dirichlet form is nonnegative:
$Q(\psi) \ge 0$ for all $\psi \in \mathbb{R}^S$.  However, not all nonnegative
quadratic forms are Dirichlet forms.  For example, if $S = \{1, 2\}$, the
nonnegative quadratic form $Q(\psi) = (\psi_1 + \psi_2)^2$ is not a Dirichlet
form. That said, the concept of Dirichlet form is vastly more general than the
above definition: such quadratic forms are studied not just on
finite-dimensional vector spaces $\mathbb{R}^S$ but on $L^2$ of any measure
space.  When this measure space is just a finite set, the concept of Dirichlet
form reduces to the definition above.  For a thorough introduction to Dirichlet
forms, see the text by Fukushima \cite{Fukushima}.  For a fun tour of the
underlying ideas, see the paper by Doyle and Snell \cite{DS}. 

The following characterizations of Dirichlet forms help illuminate the concept:

\begin{proposition} \label{dirichlet_characterizations}
  Given a finite set $S$ and a quadratic form $Q\maps \mathbb{R}^S \to \mathbb{R}$,
  the following are equivalent:
  \begin{enumerate}[(i)]
    \item $Q$ is a Dirichlet form.

    \item $Q(\phi) \le Q(\psi)$ whenever $|\phi_i - \phi_j| \le |\psi_i -
      \psi_j|$ for all $i, j$. 

    \item $Q(\phi) = 0$ whenever $\phi_i$ is independent of $i$, and $Q$ obeys
      the \define{Markov property}: $Q(\phi) \le Q(\psi)$ when $\phi_i = \min
      (\psi_i, 1) $.
  \end{enumerate}
\end{proposition}
\begin{proof}
See Fukushima \cite{Fukushima}.
\end{proof}

While the extended power functionals of circuits are evidently Dirichlet forms,
it is not immediate that all power functionals are. For this it is crucial that
the property of being a Dirichlet form is preserved under minimising over linear
subspaces of the domain that are generated by subsets of the given finite set.

\begin{proposition} \label{dirichlet_minimization}
  If $Q\maps \R^{S+T} \to \R$ is a Dirichlet form, then 
  \[
    \min_{\nu \in \R^T} Q(-,\nu)\maps \R^S \to \R 
  \]
  is Dirichlet.
\end{proposition}
\begin{proof}
  We first note that $\min_{\nu \in \R^S} Q(-,\nu)$ is a quadratic form. Again,
  $\min_{\nu \in \R^T} Q(-,\nu)$ is well-defined as a nonnegative quadratic form
  also attains its minimum on an affine subspace of its domain. Furthermore
  $\min_{\nu \in \R^T} Q(-,\nu)$ is itself a quadratic form, as the partial
  derivatives of $Q$ are linear, and hence the points at with these minima are
  attained depend linearly on the argument of $\min_{\nu \in \R^T} Q(-,\nu)$.

  Now by Proposition \ref{dirichlet_characterizations}, $Q(\phi) \le Q(\phi')$
  whenever $|\phi_i - \phi_j| \le |\phi'_i - \phi'_j|$ for all $i,j \in S+T$. In
  particular, this implies $\min_{\nu \in \R^T} Q(\psi,\nu) \le \min_{\nu \in
  \R^T} Q(\psi',\nu)$ whenever $|\psi_i - \psi_j| \le |\psi'_i - \psi'_j|$ for
  all $i,j \in S$. Using Proposition \ref{dirichlet_characterizations} again
  then implies that $\min_{\nu \in \R^T} Q(-,\nu)$ is a Dirichlet form.
\end{proof}


\begin{corollary}
  Let $Q \maps \R^{\partial N} \to \R$ be the power functional for some circuit. Then
  $Q$ is a Dirichlet form.
\end{corollary}
\begin{proof}
  The extended power functional $P$ is a Dirichlet form, and writing $\R^N=
  \R^{\partial N} \oplus \R^{N \setminus \partial N}$ allows us to write
  \[
    Q(-) =  \min_{\phi\in \R^{N \setminus \partial N}}
    P(-,\phi). \qedhere
  \]
\end{proof}

The converse is also true: simply construct the circuit with set of vertices
$\partial N$ and an edge of resistance $\frac{1}{2c_{ij}}$ between any $i,j \in
\partial N$ such that the term $c_{ij}(\psi_i - \psi_j)$ appears in the
Dirichlet form. This gives: 

\begin{proposition}
  A function $Q$ is the power functional for some circuit if and only if $Q$ is a
  Dirichlet form.
\end{proposition}

This is an expression of the `star-mesh transform', a well-known fact of
electrical engineering stating that every circuit of linear resistors is
equivalent to some complete graph of resistors between its terminals. For more
details see \cite{vLO}. We may interpret the proof of Proposition
\ref{dirichlet_minimization} as showing that intermediate potentials at minima
depend linearly on boundary potentials, in fact a weighted average, and that
substituting these into a quadratic form still gives quadratic form.

\bigskip

In summary, in this section we have shown the existence of a surjective function
\[
  \bigg\{\begin{array}{c} \mbox{circuits of linear resistors} \\ \mbox{ with
    boundary $\partial N$} \end{array} \bigg\} \longrightarrow \bigg\{
    \mbox{Dirichlet forms on $\partial N$}\bigg\}
\]
mapping two circuits to the same Dirichlet form if and only if they have the same
external behavior.  In the next section we extend this result to encompass
inductors and capacitors too.


\section{Inductors and capacitors} \label{sec:plcs}
%%fakesubsection
The intuition gleaned from the study of resistors carries over to inductors and
capacitors too, to provide a framework for studying what are known as passive
linear networks. To understand inductors and capacitors in this way, however, we
must introduce a notion of time dependency and subsequently the Laplace
transform, which allows us to work in the so-called frequency domain. Here, like
resistors, inductors and capacitors simply impose a relationship of
proportionality between the voltages and currents that run across them. The
constant of proportionality is known as the impedance of the component.

As for resistors, the interconnection of such components may be understood, at
least formally, as a minimization of some quantity, and we may represent the
behaviors of this class of circuits with a more general idea of Dirichlet form.
We conclude this section by noting an obstruction to building a composition rule
for Dirichlet forms, motivating our work in Part . 


\subsection{The frequency domain and Ohm's law revisited}

In broadening the class of electrical circuit components under examination, we
find ourselves dealing with components whose behaviors depend on the rates of
change of current and voltage with respect to time. We thus now consider
time-varying voltages $v \maps [0,\infty) \to \R$ and currents $i \maps
  [0,\infty) \to \R$, where $t \in [0,\infty)$ is a real variable representing
    time. For mathematical reasons, we
restrict these voltages and currents to only those with (i) zero initial
conditions (that is, $f(0) = 0$) and (ii) Laplace transform lying in the field
\[
  \R(s) = \left\{ Z(s) = \tfrac{P(s)}{Q(s)} \,\Big\vert\, P, Q \mbox{
  polynomials over $\R$ in $s$}, \, Q \ne 0 \right\}
\]
of real rational functions of one variable. 
%We don't need the currents and voltages to lie in this field!!!  They just
%need to lie in some vector space over this field!!!
While it is possible that
physical voltages and currents might vary with time in a more general way, we
restrict to these cases as the rational functions are, crucially, well-behaved
enough to form a field, and yet still general enough to provide arbitrarily
close approximations to currents and voltages found in standard applications.

An \define{inductor} is a two-terminal circuit component across which the voltage is
proportional to the rate of change of the current. By convention we draw this as
follows, with the inductance $L$ the constant of proportionality:\footnote{We
  follow the standard convention of denoting inductance by the letter $L$, after
  the work of Heinrich Lenz and to avoid confusion with the $I$ used for
current.}
\[
  \begin{tikzpicture}[circuit ee IEC]
    \node[contact] (I1) at (0,0) {};
    \node[contact] (I2) at (1.83,0) {};
    \draw (I1) 	to [inductor] node [label={[label distance=2pt]{$L$}}]
    {} (I2);
  \end{tikzpicture}
\]
Writing $v_L(t)$ and $i_L(t)$ for the voltage and current over time $t$ across
this component respectively, and using a dot to denote the derivative with
respect to time $t$, we thus have the relationship 
\[
  v_L(t) = L\, \dot{i}_L(t).
\]
Permuting the roles of current and voltage, a \define{capacitor} is a two-terminal
circuit component across which the current is proportional to the rate of change
of the voltage. We draw this as follows, with the capacitance $C$ the constant
of proportionality:
\[
  \begin{tikzpicture}[circuit ee IEC]
    \node[contact] (I1) at (0,0) {};
    \node[contact] (I2) at (1.83,0) {};
    \draw (I1) 	to [capacitor] node [label={[label distance=5pt]{$C$}}]
    {} (I2);
  \end{tikzpicture}
\]
Writing $v_C(t)$, $i_C(t)$ for the voltage and current across the capacitor,
this gives the equation
\[
  i_C(t) = C\, \dot{v}_C(t).
\]
We assume here that inductances $L$ and capacitances $C$ are positive real numbers.

Although inductors and capacitors impose a linear relationship if we involve the
derivatives of current and voltage, to mimic the above work on resistors we wish
to have a constant of proportionality between functions representing the current
and voltage themselves. Various integral transforms perform just this role; electrical
engineers typically use the Laplace transform. This lets us write a function of time $t$ instead as a function of frequencies $s$, and in doing so turns differentiation with respect to $t$ into multiplication by $s$, and integration with respect to $t$ into
division by $s$.  

In detail, given a function $f(t)\maps [0, \infty) \to \R$, we define the
\define{Laplace transform} of $f$
\[
  \mathfrak{L}\{f\}(s) = \int_{0}^\infty f(t) e^{-st} dt.
\]
We also use the notation $\mathfrak{L}\{f\}(s) = F(s)$, denoting the Laplace
transform of a function in upper case, and refer to the Laplace transforms as
lying in the \define{frequency domain} or \define{$s$-domain}. For us, the three
crucial properties of the Laplace transform are then: 
\begin{enumerate}[(i)]
  \item linearity: $\mathfrak{L}\{af+bg\}(s) = aF(s)+bG(s)$ for $a,b\in \R$;
  \item differentiation: $\mathfrak{L}\{\dot{f}\}(s) = s F(s) - f(0)$;
  \item integration: if $g(t) = \int_0^t f(\tau)d\tau$ then 
 $G(s) = \frac{1}{s} F(s)$.
\end{enumerate}
Writing $V(s)$ and $I(s)$ for the Laplace transform of the voltage $v(t)$ and
current $i(t)$ across a component respectively, and recalling that by assumption
$v(t) = i(t) = 0$ for $t \le 0$, the $s$-domain behaviors of components become,
for a resistor of resistance $R$:
\[
  V(s) = RI(s),
\]
for an inductor of inductance $L$:
\[
  V(s) = sLI(s),
\]
and for a capacitor of capacitance $C$:
\[
  V(s) = \frac1{sC} I(s). 
\]

Note that for each component the voltage equals the current times a rational function of
the real variable $s$, called the \define{impedance} and in general denoted by $Z$.
Note also that the impedance is a \define{positive real function}, meaning that it lies
in the set
\[         \R(s)^+ = \{ Z \in \R(s) : \forall s \in \C \;\; \mathrm{Re}(s) > 0 \implies 
\mathrm{Re}(Z(s)) > 0 \} . \]
While $Z$ is a quotient of polynomials with real cofficients, in this definition
we are applying it to complex values of $s$, and demanding that its real part be
positive in the open left half-plane.  Positive real functions were introduced by Otto 
Brune in 1931, and they play a basic role in circuit theory \cite{Brune}.  

Indeed, Brune convincingly argued that for any conceivable passive linear component we have this generalization of Ohm's law:
\[
  V(s)=Z(s)I(s)
\]
where $I \in \R(s)$ is the \define{current}, $V \in \R(s)$ is the \define{voltage}
and $Z \in \R(s)^+$ is the \define{impedance} of the component.   As we shall see, generalizing from circuits of linear resistors to arbitrary passive linear circuits is just a matter of formally replacing resistances by  impedances.  This amounts to replacing the field $\R$ by the larger field $\R(s)$, and replacing the set of positive reals, $\R^+ = (0,\infty)$, by the set of positive real functions, $\R(s)^+$.  From a mathematical perspective we might as well work with any field with a mildly well-behaved notion of `positive element', and we do this in the next section.

\subsection{The mechanical analogy} \label{sec:mechanical}

Now that we have introduced inductors and capacitors, it is worth taking 
another glance at the analogy chart in Section \ref{sec:intro}.  What are the
analogues of resistance, inductance and capacitance in mechanics?  If we restrict attention to systems with translational degrees of freedom, the answer is given in the following
chart.

\begin{small}
\begin{center}
\begin{tabular}{|c|c|}
\hline
Electronics & Mechanics (translation) \\
\hline\hline
charge $Q$ & position $q$ \\
\hline
current $i = \dot Q$ & velocity $v = \dot q$ \\
\hline
flux linkage $\lambda$ & momentum $p$ \\
\hline
voltage $v = \dot \lambda$ & force $F = \dot p$ \\
\hline
resistance $R$ & damping coefficient $c$ \\
\hline
inductance $L$ & mass $m$ \\
\hline
inverse capacitance $C^{-1}$ & spring constant $k$ \\
\hline
\end{tabular}
\end{center}
\end{small}

A famous example concerns an electric circuit with a resistor of resistance $R$, an inductor of inductance $L$, and a capacitor of capacitance $C$, all in series:
\[
  \begin{tikzpicture}[circuit ee IEC, set resistor graphic=var resistor IEC graphic]
    \node[contact] (I1) at (0,0) {};
    \node[contact] (I2) at (1.83,0) {};
    \node[contact] (I3) at (3.66,0) {};
    \node[contact] (I4) at (5.49,0) {};
    \draw (I1) 	to [resistor] node [label={[label distance=2pt]{$R$}}]
    {} (I2);
    \draw (I2) 	to [inductor] node [label={[label distance=5pt]{$L$}}]
    {} (I3);
     \draw (I3) 	to [capacitor] node [label={[label distance=5pt]{$C$}}]
    {} (I4);
  \end{tikzpicture}
\]
We saw in Example \ref{resistors_in_series} that for resistors in series, the
resistances add.  The same fact holds more generally for passive linear circuits,
so the impedance of this circuit is the sum
\[   Z = s L + R + (sC)^{-1}  .\]
Thus, the voltage across this circuit is related to the current through the
circuit by
\[  V(s) = (s L + R + (sC)^{-1}) I(s)  \]
If $v(t)$ and $i(t)$ are the voltage and current as functions of time, we conclude that
\[  v(t) = L \frac{d}{dt}i(t) + Ri(t) + C^{-1} \int_0^t i(s) \, ds  \]
It follows that 
\[   
L \ddot{Q} + R \dot{Q} + C^{-1} Q = v
\]
where $Q(t) = \int_0^t i(t) ds$ has units of charge.  As the chart above suggests,
this equation is analogous to that of a damped harmonic oscillator:
\[    
m \ddot{q} + c \dot{q} + k q = F 
\]
where $m$ is the mass of the oscillator, $c$ is the damping coefficient, $k$ is 
the spring constant and $F$ is a time-dependent external force.

For details, and many more analogies of this sort, see the book by Karnopp, 
Margolis and Rosenberg \cite{KRM} or Brown's enormous text \cite{Brown}.   While it would be a distraction to discuss them further here, these analogies mean that our
work applies to a wide class of networked systems, not just electrical circuits.

\subsection{Generalized Dirichlet forms} \label{sec:generalized}

To understand the behavior of passive linear circuits we need to
understand how the behaviors of individual components, governed by Ohm's law,
fit together give the behavior of an entire network.
Kirchhoff's laws still hold, and so does a version of the principle of minimum power.
To see this, we generalize our remarks so far to any field with a set of `positive
elements'.

\begin{definition} 
  Given a field $\F$, we define a \define{set of positive elements} for $\F$ to be     
  any subset $\F^+\subset \F$ containing $1$ but not $0$, and closed under 
  addition, multiplication and division.  
\end{definition}

Our first motivating example arises from circuits made of resistors.  Here $\F =
\R$ is the field of real numbers and we take $\F^+ = (0,\infty)$.   Our second
motivating example arises from general passive linear circuits.  Here $\F =
\R(s)$ is the field of rational functions in one real variable, and we take
$\F^+ = \R(s)^+$ to be the positive real functions, as defined in the last
section.  

In all that follows, we fix a field $\F$ equipped with a set of positive
elements $\F^+$.  By a `circuit', we shall henceforth mean a circuit over
$\F^+$, as explained in Definition \ref{def_circuit}.   To fix the notation:

\begin{definition} \label{def_circuit_2}
A \define{(passive linear) circuit} is a graph $s,t \maps E \to N$ with $E$ as its set of \define{edges} and $N$ as its set of \define{nodes}, equipped with function $Z \maps E \to \F^+$ assigning each edge an \define{impedance}, together with finite sets $X$, $Y$, and functions $i \maps X \to N$ and $o\maps Y \to  N$. We call the sets $i(X)$, $o(Y)$, and $\partial N = i(X) \cup o(Y)$ the \define{inputs}, \define{outputs}, and \define{terminals} or \define{boundary} of the circuit, respectively.
\end{definition}

Generalizing from circuits of resistors, we define the
\define{extended power functional} $P\maps \F^N \to \F$ of any circuit by
\[
  P(\varphi) = \frac{1}{2} \sum_{e \in E} \frac1{Z(e)}\big(\varphi(t(e))-\varphi(s(e))\big)^2.
\]
and we call $\varphi \in \F^N$ a \define{potential}.  Note that a field of
characteristic 2 cannot be given a set of positive elements, so dividing by 2 is
allowed.  Note also that the extended power functional is a Dirichlet form on
$N$.

Although it is not clear what it means to minimize over the field $\F$, we can
use formal derivatives to formulate an analogue of the principle of minimum 
power.   This will actually be a `variational principle', saying the derivative of
the power functional vanishes with respect to certain variations in the potential.
As before we shall see that given Ohm's law, this principle is equivalent to
Kirchhoff's current law.

Indeed, the extended power functional $P(\varphi)$ can be considered an element
of the polynomial ring $\F[\{\varphi(n)\}_{n \in N}]$ generated by formal
variables $\varphi(n)$ corresponding to potentials at the nodes $n \in N$. We
may thus take formal derivatives of the extended power functional with respect
to the $\varphi(n)$.  We then call $\phi \in \F^N$ a \define{realizable
potential} for the given circuit if for each nonterminal node, $n \in N\setminus
\partial N$, the formal partial derivative of the extended power functional with
respect to $\varphi(n)$ equals zero when evaluated at $\phi$:
\[
  \frac{\partial P}{\partial \varphi(n)}\bigg\vert_{\varphi = \phi} = 0
\]
This terminology arises from the following fact, a generalization of Proposition
\ref{minimum_power_implies_kirchhoff_current}:

\begin{theorem} \label{thm:realizablepotentials}
The potential $\phi \in \F^N$ is a realizable potential for a given
circuit if and only if the induced current 
\[  I(e) = \frac1{Z(e)}(\phi(t(e))-\phi(s(e))) \]
obeys Kirchhoff's current law:
\[ 
\sum_{s(e) = n} I(e) = \sum_{t(e) = n} I(e)
\]  
for all $n \in N\setminus \partial N$.
\end{theorem}
\begin{proof}
The proof of this statement is exactly that for Proposition
\ref{minimum_power_implies_kirchhoff_current}. 
\end{proof}

A corollary of Theorem \ref{thm:realizablepotentials} is that the set of
states---that is, potential--current pairs---that are compatible with the
governing laws of a circuit is given by the set of realizable potentials
together with their induced currents. 

We now generalize the theory of Dirichlet forms:

\begin{definition} Given a field $\F$ with a set of positive elements $\F^+$ and
  a finite set $S$, a \define{Dirichlet form over $\F$} on $S$ is a quadratic form   
  $Q\maps \F^S \to \F$ given by the formula 
  \[ Q(\psi) = \sum_{i,j \in S} c_{ij} 
  (\psi_i - \psi_j)^2 \] for some choice of $c_{i j} \in \F^+ \cup \{0\}$.  
\end{definition}

Note that the extended power functional of any circuit is a Dirichlet form,
since $Z(e) \in \F^+$ implies $\frac{1}{2}\frac{1}{Z(e)} \in \F^+$.

\subsection{A generalized minimizability result}

We begin to move from a discussion of the intrinsic behaviors of circuits to a
discussion of their behaviors under composition. The key fact for composition of
generalized Dirichlet forms is that, in analogy with Proposition
\ref{dirichlet_minimization}, we may speak of a formal version of minimization
of Dirichlet forms. We detail this here.  In what follows, fix a field $\F$ with a 
set of positive elements $\F^+$, and let $P$ be a Dirichlet form over $\F$ on 
some finite set $S$. 

Recall that given $R \subseteq S$, we
call $\tilde\psi \in \F^S$ an \define{extension} of $\psi \in \F^R$ if
$\tilde\psi$ restricted to $R$ equals $\psi$.   We call such an
extension \define{realizable} if 
\[
    \frac{\partial P}{\partial \varphi(s)}\bigg\vert_{\varphi = \tilde\psi} = 0
  \]
for all $s \in S \setminus R$.  Note that over the real numbers $\R$ this means
that among all the extensions of $\psi$, $\tilde\psi$ minimizes the function $P$.

\begin{theorem} \label{thm:dirichletminimization}
  Let $P$ be a Dirichlet form over $\F$ on $S$, and let $R \subseteq S$ be an
  inclusion of finite sets. Then we may uniquely define a Dirichlet form
  \[\min_{S \setminus R}P: \F^R \to \F\] 
   on $R$ by sending each $\psi \in \F^R$ to
  the value $P(\tilde\psi)$ of any realizable extension $\tilde\psi$ of $\psi$.
\end{theorem}


To prove this theorem, we must first show that $\min_{S \setminus R} P$ is well-defined as a function.

\begin{lemma} \label{lem:welldefineddirichletmin}
  Let $P$ be a Dirichlet form over $\F$ on $S$, let $R \subseteq S$ be an
  inclusion of finite sets, and let $\psi \in \F^R$. Then for all realizable
  extensions $\tilde\psi$, $\tilde\psi' \in \F^S$ of $\psi$ we have $P(\tilde\psi) =
  P(\tilde\psi')$. 
\end{lemma}
\begin{proof}
  This follows from the formal version of the multivariable Taylor theorem for
  polynomial rings over a field of characteristic zero. Let $\tilde\psi$,
  $\tilde\psi' \in \F^S$ be realizable extensions of $\psi$, and note that
  $dP_{\tilde\psi}(\tilde\psi-\tilde\psi')=0$, since for all $s \in R$ we have
  $\tilde\psi(s) -\tilde\psi'(s) =0$, and for all $s \in S \setminus R$ we have
  \[
    \frac{\partial P}{\partial \varphi(s)}\bigg\vert_{\varphi = \tilde\psi}=0. 
  \]
  We may take the Taylor expansion of $P$ around $\tilde\psi$ and evaluate at
  $\tilde\psi'$. As $P$ is a quadratic form, this gives
  \begin{align*}
    P(\tilde\psi') &=
    P(\tilde\psi)+dP_{\tilde\psi}(\tilde\psi'-\tilde\psi)+P(\tilde\psi'-\tilde\psi)
    \\
    & = P(\tilde\psi)+P(\tilde\psi'-\tilde\psi).
  \end{align*}
  Similarly, we arrive at  
  \[
    P(\tilde\psi)= P(\tilde\psi')+P(\tilde\psi-\tilde\psi').
  \]
  But again as $P$ is a quadratic form, we then see that 
  \[
    P(\tilde\psi')-P(\tilde\psi) = P(\tilde\psi'-\tilde\psi) =
    P(\tilde\psi-\tilde\psi') = P(\tilde\psi)-P(\tilde\psi').
  \]
  This implies that $P(\tilde\psi')-P(\tilde\psi) = 0$, as required.
\end{proof}

It remains to show that $\min P$ remains a Dirichlet form. We do this
inductively.

\begin{lemma} \label{lem:onestepdirichletmin}
  Let $P$ be a Dirichlet form over $\F$ on $S$, and let $s \in S$ be an element
  of $S$. Then the map $\min_{\{s\}} P:\F^{S \setminus\{s\}} \to \F$ sending
  $\psi$ to $P(\tilde\psi)$ is a Dirichlet form on $S \setminus \{s\}$.
\end{lemma}
\begin{proof}
  Write $P(\phi) = \sum_{i,j} c_{ij}(\phi_i -\phi_j)^2$, assuming without loss
  of generality that $c_{sk} =0$ for all $k$. We then have
  \[
    \frac{\partial P}{\partial \varphi(s)}\bigg\vert_{\varphi = \phi} = \sum_k
    2c_{ks}(\phi_s-\phi_k),
  \]
  and this is equal to zero when
  \[
    \phi_s = \frac{\sum_k c_{ks}\phi_k}{\sum_k c_{ks}}.
  \]
  Thus $\min_{\{s\}}P$ may be given explicitly by the expression
  \[
    \min_{\{s\}} P(\psi) = \sum_{i,j \in S \setminus \{s\}} c_{ij}(\psi_i -\psi_j)^2 +
    \sum_{\ell \in S \setminus \{s\}} c_{\ell s}\left(\psi_\ell - \tfrac{\sum_k
      c_{ks} \psi_k}{\sum_k c_{ks}}\right)^2.
  \]
  We must show this is a Dirichlet form on $S \setminus \{s\}$. 
  
  As the sum of Dirichlet forms is evidently Dirichlet, it suffices to check that the expression 
  \[
    \sum_\ell c_{\ell s}\left(\psi_\ell - \tfrac{\sum_k c_{ks} \psi_k}{\sum_k
      c_{ks}}\right)^2
  \]
  is Dirichlet on $S \setminus \{s\}$. Multiplying through by the constant
  $(\sum_k c_{ks})^2 \in \F^+$, it further suffices to check
  \begin{align*}
    \sum_\ell c_{\ell s}\left(\sum_k c_{ks} \psi_\ell - \sum_k c_{ks}
    \psi_k\right)^2 &= \sum_\ell c_{\ell s} \left(\sum_k c_{ks} (\psi_\ell -
    \psi_k)\right)^2 \\
    &= \sum_\ell c_{\ell s} \left(2 \sum_{\substack{k,m \\ k \ne m}} c_{k s} c_{ms}
    (\psi_\ell-\psi_k)(\psi_\ell - \psi_m) + \sum_{k} c_{k
    s}^2(\psi_\ell-\psi_k)^2\right) \\
    &= 2\sum_{\substack{k,\ell,m \\ k \ne m}} c_{\ell s} c_{k s} c_{ms}
    (\psi_\ell-\psi_k)(\psi_\ell - \psi_m) + \sum_{k, \ell} c_{\ell s}c_{k
    s}^2(\psi_\ell-\psi_k)^2
  \end{align*}
  is Dirichlet. But
  \begin{align*}
    &\quad (\psi_k - \psi_\ell)(\psi_k - \psi_m)+(\psi_\ell - \psi_k)(\psi_\ell -
    \psi_m) + (\psi_m-\psi_k)(\psi_m-\psi_\ell) \\ 
    &= \psi_k^2+\psi_\ell^2+\psi_m^2-\psi_k\psi_\ell- \psi_k\psi_m -
    \psi_\ell\psi_m \\
    &= \tfrac12\big( (\psi_k-\psi_\ell)^2 +(\psi_k-\psi_m)^2
    +(\psi_\ell-\psi_m)^2\big),
  \end{align*}
  so this expression is indeed Dirichlet. Indeed, pasting these computations
  together shows that
  \[
    \min_{\{s\}}P(\psi) = \sum_{i,j} \left(c_{ij}+\frac{c_{is}c_{js}}{{\textstyle \sum_k}
    c_{ks}}\right)(\psi_i-\psi_j)^2. \qedhere
  \]
\end{proof}

With these two lemmas, the proof of Theorem \ref{thm:dirichletminimization}
becomes straightforward.

\begin{proof}[Proof of Theorem \ref{thm:dirichletminimization}]
  Lemma \ref{lem:welldefineddirichletmin} shows that $\min_{S \setminus R}P$ is a well-defined
  function. As $R$ is a finite set, we may write it $R = \{s_1,\dots, s_n\}$ for
  some natural number $n$. Then we may define a sequence of functions $P_i =
  \min_{\{s_1, \dots,s_i\}} P_{i-1}$, $1 \le i\le n$. Define also $P_0 = P$, and note
  that $P_n = \min_{S \setminus R}P$. Then, by Lemma
  \ref{lem:onestepdirichletmin}, each $P_i$ is Dirichlet as $P_{i-1}$
  is. This proves the proposition.
\end{proof}

We can thus define the power functional of a circuit by analogy with circuits made
of resistors:

\begin{definition}
The \define{power functional} $Q \maps \R^{\partial N} \to \R$ of a circuit with extended power functional $P$ is given by
\[
 Q = \min_{N \setminus \partial N}  P .
\]
\end{definition}

As before, we define two circuits to be \define{equivalent} if they have the same
power functional, and define the \define{behavior} of a circuit to be its equivalence
class.  

\subsection{Composition of Dirichlet forms}

It would be nice to have a category in which circuits are morphisms, and a
category in which Dirichlet forms are morphisms, such that the map sending
a circuit to its behavior is a functor.  Here we present a na\"ive attempt to
constructed the category with Dirichlet forms as morphisms, using the principle
of minimum power to compose these morphisms.  Unfortunately the proposed category does not include identity morphisms.  However, it points in the right direction, and underlines the importance of the cospan formalism we then turn to develop.

We can define a composition rule for Dirichlet forms that reflects composition of circuits.
Given finite sets $S$ and $T$, let $S+T$ denote their disjoint union.  Let
$D(S,T)$ be the set of Dirichlet forms on $S+T$. There is a way
to compose these Dirichlet forms
\[ 
\circ \maps D(T,U) \times D(S,T) \to D(S,U) 
\]
defined as follows.  Given $P \in D(T,U)$ and $Q \in D(S,T)$, let
\[ 
  (P \circ Q)(\alpha, \gamma) = \min_{T} Q(\alpha, \beta) + P(\beta, \gamma),
\]
where $\alpha \in F^S, \gamma \in F^U$. This operation has a clear
interpretation in terms of electrical circuits: the power used by the entire
circuit is just the sum of the power used by its parts. 

It is immediate from Theorem \ref{thm:dirichletminimization} that this
composition rule is well-defined: the composite of two Dirichlet forms is again
a Dirichlet form. Moreover, this composition is associative. However, it fails
to provide the structure of a category, as there is typically no Dirichlet form
$1_S \in D(S,S)$ playing the role of the identity for this composition. For an
indication of why this is so, let $\{\bullet\}$ be a set with one element, and
suppose that some Dirichlet form $I(\beta,\gamma) = k(\beta-\gamma)^2 \in
D(\{\bullet\},\{\bullet\})$ acts as an identity on the right for this
composition. Then for all $Q(\alpha,\beta) = c(\alpha-\beta)^2 \in
D(\{\bullet\},\{\bullet\})$, we must have
\begin{align*}
  c\alpha^2 &= Q(\alpha,0) \\
  &= (I \circ Q)(\alpha,0) \\ 
  &= \min_{\beta \in \F} Q(\alpha, \beta) + I(\beta,0) \\
  &= \min_{\beta \in \F} k(\alpha-\beta)^2 + c\beta^2 \\
  &= \frac{kc}{k+c}\alpha^2,
\end{align*}
where we have noted that $\frac{kc}{k+c}\alpha^2$ minimizes $k(\alpha-\beta)^2 +
c\beta^2$ with respect to $\beta$. But for any choice of $k \in \F$ this
equality only holds when $c = 0$, so no such Dirichlet form exists. Note,
however, that for $k>> c$ we have $c\alpha^2 \approx \frac{kc}{k+c}\alpha^2$, so
Dirichlet forms with large values of $k$---corresponding to resistors with
resistance close to zero---act as `approximate identities'.

In this way we might interpret the identities we wish to introduce
into this category as the behaviors of idealized components with zero
resistance: perfectly conductive wires. Unfortunately, the power functional of a
purely conductive wire is undefined: the formula for it involves division by
zero.  In real life, coming close to this situation leads to the disaster that
electricians call a `short circuit': a huge amount of power dissipated for even
a small voltage.  This is why we have fuses and circuit breakers.

Nonetheless, we have most of the structure required for a category. A `category
without identity morphisms' is called a \define{semicategory}, so we see
\begin{proposition}
There is a semicategory where:
\begin{itemize}
\item the objects are finite sets,

\item a morphism from $T$ to $S$ is a Dirichlet form $Q \in D(S,T)$.  

\item composition of morphisms is given by 
\[
(R \circ Q)(\gamma, \alpha) = \min_{T} Q(\gamma, \beta) + R(\beta, \alpha).
\]

\end{itemize}
\end{proposition}

We would like to make this into a category. One easy way to do this is to
formally adjoin identity morphisms; this trick works for any semicategory.
However, we obtain a better category if we include \emph{more} morphisms: more
behaviors corresponding to circuits made of perfectly conductive wires. As the
expression for the extended power functional includes the reciprocals of
impedances, such circuits cannot be expressed within the framework we have
developed thus far. Indeed, for these idealized circuits there is no function
taking boundary potentials to boundary currents: the vanishing impedance would
imply that any difference in potentials at the boundary induces `infinite'
currents. To deal with this issue, we generalize Dirichlet forms to Lagrangian
relations.  First, however, we develop a category theoretic framework, based
around decorated cospans, to define the category of circuits itself and
understand its basic properties.

\section{The category of passive linear circuits} \label{sec:circdef}
In this part we move our focus from the semantics of circuit diagrams to the
syntax, addressing the question ``How do we interact with circuit diagrams?''.
Informally, the answer to this is that we interact with them by connecting them
to each other, perhaps after moving them into the right form by rotating or
reflecting them, or by crossing or bending some of the wires. To formalize this,
we adopt a category theoretic viewpoint, defining various dagger compact
categories with circuits and their behaviors as morphisms. We claim a formal
analysis of this structure, especially of the composition or connection of
circuits, has been overlooked in analysis of circuits thus far. This part
culminates in the definition of two important categories, the category $\Circ$
of circuit diagrams, and the category $\LagrRel$ containing all behaviors of
circuits. We also develop the technical material required to appreciate the
structure of these categories, and that aids understanding of the relationship
between the two, to be addressed in Part .

%%fakesubsection
In Part  we defined a circuit of linear resistors to be a labelled graph with marked input and output terminals, as in the example:
\[
\begin{tikzpicture}[circuit ee IEC, set resistor graphic=var resistor IEC graphic]
\node[contact] (I1) at (0,2) {};
\node[contact] (I2) at (0,0) {};
\coordinate (int1) at (2.83,1) {};
\coordinate (int2) at (5.83,1) {};
\node[contact] (O1) at (8.66,2) {};
\node[contact] (O2) at (8.66,0) {};
\node (input) at (-2,1) {\small{\textsf{inputs}}};
\node (output) at (10.66,1) {\small{\textsf{outputs}}};
\draw (I1) 	to [resistor] node [label={[label distance=2pt]85:{$1\Omega$}}] {} (int1);
\draw (I2)	to [resistor] node [label={[label distance=2pt]275:{$1\Omega$}}] {} (int1)
				to [resistor] node [label={[label distance=3pt]90:{$2\Omega$}}] {} (int2);
\draw (int2) 	to [resistor] node [label={[label distance=2pt]95:{$1\Omega$}}] {} (O1);
\draw (int2)		to [resistor] node [label={[label distance=2pt]265:{$3\Omega$}}] {} (O2);
\path[color=gray, very thick, shorten >=10pt, ->, >=stealth, bend left] (input) edge (I1);		\path[color=gray, very thick, shorten >=10pt, ->, >=stealth, bend right] (input) edge (I2);		
\path[color=gray, very thick, shorten >=10pt, ->, >=stealth, bend right] (output) edge (O1);
\path[color=gray, very thick, shorten >=10pt, ->, >=stealth, bend left] (output) edge (O2);
\end{tikzpicture}
\]
We then defined general passive linear circuits by replacing resistances with
impedances chosen from a set of positive elements $\F^+$ in any field $\F$.
In fact, these circuits are examples of decorated cospans.  This gives a dagger compact category $\Circ$ whose morphisms are circuits, with the
dagger compact structure expressing standard operations on circuits.

We actually give two constructions of this category, first arriving at a category of
cospans decorated by $\F^+$-graphs, and then showing that this is
a full subcategory of the decategorification of a bicategory of cospans of
$\F^+$-graphs.

\subsection{A decorated cospan construction}

We officially defined a `circuit' in Definition \ref{def_circuit_2}, but now we can give an equivalent definition using cospans:

\begin{lemma} A circuit is a cospan of finite sets $X \stackrel{i}{\longrightarrow} N
\stackrel{o}{\longleftarrow} Y$ together with an $\F^+$-graph whose set
of nodes is $N$.
\end{lemma}

\begin{proof}
This is just a matter of remembering the terminology: recall that an $\F^+$-graph
is a graph whose edges are labelled with elements of $\F^+$, our chosen set of positive
elements in the field $\F$.
\end{proof}

This suggests that circuits should be morphisms in a decorated  cospan category. Indeed, we can show that the map taking a finite set $N$ to the set of $\F^+$-graphs with
set $N$ of nodes in fact forms a lax symmetric monoidal functor. This allows us to apply
decorated cospans to construct a category of circuits.

To this end, define the functor
\[
  \mathrm{Circuit}\maps (\mathrm{FinSet},+) \longrightarrow (\mathrm{Set},\times)
\]
to take a finite set $N$, as an object of $\mathrm{FinSet}$, to the set
$\mathrm{Circuit}(N)$ of $\F^+$-graphs $(N,E,s,t,r)$ with $N$ as their
set of nodes. On
morphisms let it take a function $f\maps N \to M$ to the function that pushes
labelled graph structures on a set $N$ forward onto the set $M$:
\begin{align*}
  \mathrm{Circuit}(f)\maps \mathrm{Circuit}(N) &\longrightarrow
  \mathrm{Circuit}(M); \\
  (N,E,s,t,r) &\longmapsto (M,E,f \circ s, f \circ t, r).
\end{align*}
Note that as this map simply acts by post-composition, our map
$\mathrm{Circuit}$ is indeed functorial.

We then arrive at a lax symmetric monoidal functor by equipping this functor with the
natural transformation 
\begin{align*}
  \rho_{N,M}\maps \mathrm{Circuit}(N) \times \mathrm{Circuit}(M)
  &\longrightarrow \mathrm{Circuit}(N+M); \\
  \big( (N,E,s,t,r), (M,F,s',t',r') \big) &\longmapsto
  \big(N+M,E+F,s+s',t+t',[r,r']\big),
\end{align*}
together with the unit map
\begin{align*}
  \rho_1\maps 1 &\longrightarrow \mathrm{Circuit}(\varnothing); \\
  \bullet &\longmapsto
  (\varnothing,\varnothing,\varnothing,\varnothing,\varnothing),
\end{align*}
where we use $\varnothing$ to denote both the empty set and the unique function
of the appropriate codomain with domain the empty set. The naturality of this
collection of morphisms, as well as the coherence laws for lax symmetric
monoidal functors, follow from the universal property of the coproduct. 

\begin{definition}
  We define
  \[
    \mbox{\define{Circ}} = \mathrm{CircuitCospan} .
  \]
\end{definition}

\begin{corollary}
The category $\Circ$ is a dagger compact category. 
\end{corollary}

The different structures of this category capture different operations that can
be performed with circuits. The composition expresses the fact that we can
connect the outputs of one circuit to the inputs of the next, while the monoidal
composition models the placement of circuits side-by-side. The symmetric
monoidal structure allows us reorder input and output wires, and the compactness
captures the interchangeability between input and outputs of
circuits---that is, the fact that we can choose any input to our
circuit and consider it instead as an output, and vice versa.  Finally, the
dagger structure expresses the fact that we may reflect a whole circuit, switching
all inputs with all outputs.

\subsection{A bicategory of circuits}

As shown by B\'enabou \cite{Be}, cospans are most naturally thought of as
1-morphisms in a bicategory.  To obtain the category we are calling
$\mathrm{Cospan}(\mc C)$, we `decategorify' this bicategory by discarding
2-morphisms and identifying isomorphic 1-morphisms. Since we are studying
circuits using decorated cospans, this suggests that circuits, too, are most
naturally thought of as 1-morphisms in a bicategory.  Indeed this is the case.  

One route to the bicategory whose 1-morphisms are circuits would be to take
the theory of decorated cospans \cite{Fon} and show that it can be enhanced to
create not merely categories whose morphisms are isomorphism classes of
decorated cospans, but bicategories whose 1-morphisms are exactly decorated
cospans.  A less powerful but easier approach is as follows.

Recall that we define a graph to be a pair of functions $s,t \maps E \to
N$ where $E$ and $N$ are finite sets, and an $L$-graph to be a graph further
equipped with a function $r\maps E \to L$.  Thus, an $L$-graph looks like this:
\[
\xymatrix{
L & E \ar@<2.5pt>[r]^{s} \ar@<-2.5pt>[r]_{t} \ar[l]_{r} & N
}
\]
Given $L$-graphs $\Gamma = (E,N,s,t,r)$ and $\Gamma' = (E',N',s',t',r')$, a
morphism of $L$-graphs $\Gamma \to \Gamma'$ is a pair of functions $\eps\maps E \to
E'$, $v\maps N \to N'$ such that the following diagrams commute:
\[
\xymatrix @R=5.7pt{
& E \ar[dl]_{r} \ar[dd]^{\eps} \\
L \\
& E' \ar[ul]^{r'}
}
\qquad\qquad
\xymatrix{
E \ar[r]^{s} \ar[d]_{\eps} & N \ar[d]^{v}  \\
E' \ar[r]_{s'} & N'
}
\qquad\qquad
\xymatrix{
E \ar[r]^{t} \ar[d]_{\eps} & N \ar[d]^{v}  \\
E' \ar[r]_{t'} & N'.
}
\]
These $L$-graphs and their morphisms form a category \define{$L$-Graph}. Using
results about colimits in the category of sets, it is straightforward to check
that this category has finite colimits.

B\'enabou \cite{Be} gave a general construction of bicategories starting from any
category with pushouts.  Since $L$-Graph has pushouts, it follows that 
there is a bicategory \define{Cospan($L$-Graph)} with
\begin{itemize}
\item $L$-labelled graphs as objects,
\item cospans in $L$-Graphas morphisms, and 
\item maps of cospans as 2-morphisms.
\end{itemize}
Since $L$-Graph also has coproducts, this bicategory is symmetric monoidal and 
in fact compact, thanks to the work of Stay \cite{St}.   We make the following definition:

\begin{definition}
The bicategory \define{$2\mbox{-}\Circ$} is the full and 2-full sub-bicategory of
$\mathrm{Cospan(}\F^+\mbox{-}\mathrm{Graph})$ with objects those $\F^+$-graphs with no edges.
\end{definition}

It can be shown that every object in Cospan($\F^+$-Graph) is self-dual, and this
implies that $2\mbox{-}\Circ$ is again a compact closed bicategory.
Note that the category obtained by decategorifying this bicategory has
finite sets as objects, with morphisms being isomorphism classes of cospans in 
$\F^+$-Graph with feet such objects.   Thus, decategorifying $2\mbox{-}\Circ$
gives a category equivalent to our previously defined category $\Circ$.

\section{Circuits as Lagrangian relations} \label{sec:circlagr}
%%fakesubsection
In the first part of this paper, we explored the semantic content contained in
circuit diagrams, leading to an understanding of circuit diagrams as expressing
some relationship between the potentials and currents that can simultaneously be
imposed on some subset, the so-called terminals, of the nodes of the circuit. We
called this collection of possible relationships the behavior of the circuit.
While in that setting we used the concept of Dirichlet forms to describe this
relationship, we saw in the end that describing circuits as Dirichlet forms does
not allow for a straightforward notion of composition of circuits. 

In this section, inspired by the principle of least action of classical
mechanics in analogy with the principle of minimum power, we develop a setting
for describing behaviors that allows for easy discussion of composite
behaviors: Lagrangian subspaces of symplectic vector spaces. These Lagrangian
subspaces provide a more direct, invariant perspective, comprising precisely the
set of vectors describing the possible simultaneous potential and current
readings at all terminals of a given circuit. As we shall see, one immediate and
important advantage of this setting is that we may model wires of zero
resistance.

Recall that we write $\F$ for some field, which for our applications is
usually the field $\R$ of real numbers or the field $\R(s)$ of rational
functions of one real variable.

\subsection{Symplectic vector spaces}

A circuit made up of wires of positive resistance defines a function from
boundary potentials to boundary currents. A wire of zero resistance, however,
does not define a function: the principle of minimum power is obeyed as long as
the potentials at the two ends of the wire are equal. More generally, we may
thus think of circuits as specifying a set of allowed voltage-current pairs, or
as a relation between boundary potentials and boundary currents. This set forms
what is called a Lagrangian subspace, and is given by the graph of the
differential of the power functional. More generally, Lagrangian submanifolds
graph derivatives of smooth functions: they describe the point evaluated and the
tangent to that point within the same space.

The material in this section is all known, and follows without great difficulty
from the definitions. To keep this section brief we omit proofs. See any
introduction to symplectic vector spaces, such as Cimasoni and Turaev \cite{CT} or
Piccione and Tausk \cite{PT}, for details.

\begin{definition}
  Given a finite-dimensional vector space $V$ over a field $\F$, a 
  \define{symplectic form}
  $\omega\maps V \times V \to \F$ on $V$ is an alternating nondegenerate bilinear
  form.  That is, a symplectic form $\omega$ is a function $V \times V \to \F$
  that is
  \begin{enumerate}[(i)]
    \item bilinear: for all $\lambda \in \F$ and all $u,v \in V$ we have
      $\omega(\lambda u,v) = \omega(u,\lambda v) =  \lambda \omega(u,v)$;
    \item alternating: for all $v \in V$ we have $\omega(v,v) = 0$; and
    \item nondegenerate: given $v \in V$, $\omega(u,v) = 0$ for all $u \in V$ if
      and only if $u = 0$.
  \end{enumerate} 
  A \define{symplectic vector space} $(V,\omega)$ is a vector space $V$ equipped
  with a symplectic form $\omega$. 

  Given symplectic vector spaces $(V_1,\omega_1), (V_2, \omega_2)$, a
  \define{symplectic map} is a linear map 
  \[
    f\maps (V_1,\omega_1) \longrightarrow (V_2, \omega_2)
  \]
  such that $\omega_2(f(u),f(v)) = \omega_1(u,v)$ for all $u,v \in V_1$. A
  \define{symplectomorphism} is a symplectic map that is also an isomorphism. 
\end{definition}

An alternating form is always \define{antisymmetric}, meaning that $\omega(u,v) = 
-\omega(v,u)$ for all $u,v \in V$.  The converse is true except in characteristic 2.
A \define{symplectic basis} for a symplectic vector space $(V,\omega)$ is a
basis $\{p_1,\dots,p_n,q_1,\dots,q_n\}$ such that $\omega(p_i,p_j) =
\omega(q_i,q_j) = 0$ for all $1 \le i,j \le n$, and $\omega(p_i,q_j) =
\delta_{ij}$ for all $1 \le i,j\le n$, where $\delta_{ij}$ is the Kronecker delta,
equal to $1$ when $i =j$, and $0$ otherwise. A symplectomorphism maps symplectic
bases to symplectic bases, and conversely, any map that takes a symplectic basis
to another symplectic basis is a symplectomorphism.

\begin{example}[The symplectic vector space generated by a finite set]
  \label{ex:symplectic_space_generated_by_set}
  Given a finite set $N$, we consider the vector space $\vectf{N}$ a symplectic
  vector space $(\vectf{N},\omega)$, with symplectic form 
  \[
    \omega\big((\phi,i),(\phi',i')\big) = i'(\phi)-i(\phi').  
  \] 
  Let $\{\phi_n\}_{n \in N}$ be the basis of $\F^N$ consisting of the functions
  $N \to \F$ mapping $n$ to $1$ and all other elements of $n$ to $0$, and let
  $\{i_n\}_{n \in N} \subseteq {(\F^N)}^\ast$ be the dual basis. Then
  $\{(\phi_n,0),(0,i_n)\}_{n\in N}$ forms a symplectic basis for $\vectf{N}$.  
\end{example}

There are two common ways we will build symplectic spaces from other
symplectic spaces: conjugation and summation. Given a symplectic form $\omega$,
we may define its \define{conjugate} symplectic form $\overline\omega = -
\omega$, and write the conjugate symplectic space $(V,\overline\omega)$ as
$\overline V$. Given two symplectic vector spaces $(U, \nu),(V,\omega)$, we
consider their direct sum $U \oplus V$ a symplectic vector space with the
symplectic form $\nu+\omega$, and call this the \define{sum} of the two
symplectic vector spaces. Note that this is not a product in the category of
symplectic vector spaces and symplectic maps.

The symplectic form provides a notion of orthogonal complement.  Given a subspace $S$ of $V$, we define its \define{complement}
\[
  S^\circ = \{v \in V \mid \omega(v,s) = 0 \textrm{ for all } s \in S\}.
\]
Note that this construction obeys the following identities, where $S$ and $T$
are subspaces of $V$:
\begin{align*}
  \dim S+ \dim S^\circ &= \dim V \\
  (S^\circ)^\circ &= S \\
  (S + T)^\circ &= S^\circ \cap T^\circ \\
  (S \cap T)^\circ &= S^\circ + T^\circ.
\end{align*}

In the symplectic vector space $\vectf{N}$, the subspace $\F^N$ has the
property of being a maximal subspace such that the symplectic form restricts to
the zero form on this subspace. Subspaces with this property are known as Lagrangian
subspaces, and they may all be realized as the image of $\vectf{N}$ 
under symplectomorphisms from $\vectf{N}$ to itself.

\begin{definition} 
  Let $S$ be a linear subspace of a symplectic vector space $(V,\omega)$. We say
  that $S$ is \define{isotropic} if $\omega|_{S \times S} = 0$, and that $S$ is
  \define{coisotropic} if $S^\circ$ is isotropic. A subspace is
  \define{Lagrangian} if it is both isotropic and coisotropic, or equivalently, if it  
  is a maximal isotropic subspace.
\end{definition}

Lagrangian subspaces are also known as Lagrangian correspondences and canonical
relations. Note that a subspace $S$ is isotropic if and only if $S \subseteq
S^\circ$. This fact helps with the following characterizations of Lagrangian
subspaces.

\begin{proposition} \label{lagrangian_characterization} 
  Given a subspace $L \subset V$ of a symplectic vector space $(V,\omega)$, the
  following are equivalent: 
  \begin{enumerate}[(i)] 
    \item $L$ is Lagrangian.  
    \item $L$ is maximally isotropic.  
    \item $L$ is minimally coisotropic.  
    \item $L = L^\circ$.  
    \item $L$ is isotropic and $\dim L = \frac12 \dim V$.
  \end{enumerate} 
\end{proposition}

From this proposition it follows easily that the direct sum of two Lagrangian
subspaces in Lagrangian in the sum of their ambient spaces. We also observe that
an advantage of isotropy is that there is a good way to take a quotient of a
symplectic vector space by an isotropic subspace---that is, there is a way to
put a natural symplectic structure on the quotient space.

\begin{proposition}
  Let $S$ be an isotropic subspace of a symplectic vector space $(V,\omega)$.
  Then $S^\circ/S$ is a symplectic vector space with symplectic form
  $\omega'(v+S,u+S) = \omega(v,u)$.
\end{proposition}
\begin{proof} 
  The function $\omega'$ is a well-defined due to the isotropy of
  $S$---by definition adding any pair $(s,s')$ of elements of $S$ to a pair
  $(v,u)$ of elements of $S^\circ$ does not change the value of
  $\omega(v+s,u+s')$. As $\omega$ is a symplectic form, one can check that
  $\omega'$ is too.  
\end{proof}

\subsection{Lagrangian subspaces from quadratic forms}

Lagrangian subspaces are of relevance to us here as the behavior of any passive
linear circuit forms a Lagrangian subspace of the symplectic vector space
generated by the nodes of the circuit. We think of this vector space as
comprising two parts: a space $\F^N$ of potentials at each node, and a dual
space ${(\F^N)}^\ast$ of currents. To make clear how circuits can be interpreted
as Lagrangian subspaces, here we describe how Dirichlet forms on a finite set
$N$ give rise to Lagrangian subspaces of $\vectf{N}$. More generally, we show
that there is a one-to-one correspondence between Lagrangian subspaces and
quadratic forms.

\begin{proposition} \label{prop:qfls}
  Let $N$ be a finite set. Given a quadratic form $Q$ over $\F$ on $N$, the
  subspace 
  \[ 
    L_Q = \big\{(\phi,dQ_\phi) \mid \phi \in \F^N\big\} \subseteq \vectf{N},
  \] 
  where $dQ_\phi \in {(\F^N)}^\ast$ is the formal differential of $Q$ at $\phi
  \in \F^N$, is Lagrangian. Moreover, this construction gives a one-to-one correspondence 
  \[ 
    \bigg\{\mbox{Quadratic forms over $\F$ on $N$}\bigg\} \longleftrightarrow
    \bigg\{\begin{array}{c} \mbox{Lagrangian subspaces of $\vectf{N}$}\\
      \mbox{with trivial intersection with $\{0\} \oplus {(\F^N)}^\ast \subseteq
      \vectf{N}$} \end{array} \bigg\}.  
  \]
\end{proposition}
\begin{proof}
  The symplectic structure on $\vectf{N}$ and our notation for it is given in
  Example \ref{ex:symplectic_space_generated_by_set}. 

  Note that for all $n,m \in N$ the corresponding basis elements 
  \[
    \frac{\partial^2 Q}{\partial \phi_n \partial \phi_m} = dQ_{\phi_n}(\phi_m) =
    dQ_{\phi_m}(\phi_n),
  \]
  so $dQ_\phi(\psi) = dQ_\psi(\phi)$ for all $\phi,\psi \in \F^N$. Thus $L_Q$ is
  indeed Lagrangian: for all $\phi,\psi \in \F^N$ 
  \[
    \omega\big((\phi,dQ_\phi),(\psi,dQ_\psi)\big) = dQ_\psi(\phi) -
    dQ_\phi(\psi) = 0.
  \]

  Observe also that for all quadratic forms $Q$ we have $dQ_0 = 0$, so the only
  element of $L_Q$ of the form $(0,i)$, where $i \in {(\F^N)}^\ast$, is $(0,0)$.
  Thus $L_Q$ has trivial intersection with the subspace $\{0\} \oplus
  {(\F^N)}^\ast$ of $\vectf{N}$. This $L_Q$ construction forms the leftward
  direction of the above correspondence.

  For the rightward direction, suppose that $L$ is a Lagrangian subspace of
  $\vectf{N}$ such that $L \cap (\{0\} \oplus {(\F^N)}^\ast) = \{(0,0)\}$. Then
  for each $\phi \in \F^N$, there exists a unique $i_\phi \in {(\F^N)}^\ast$
  such that $(\phi,i_\phi) \in L$. Indeed, if $i_\phi$ and $i_\phi'$ were
  distinct elements of ${(\F^N)}^\ast$ with this property, then by linearity
  $(0,i_\phi-i_\phi')$ would be a nonzero element of $L \cap (\{0\} \oplus
  {(\F^N)}^\ast$, contradicting the hypothesis about trivial intersection. We
  thus can define a function, indeed a linear map, $\F^N \to {(\F^N)}^\ast; \phi
  \mapsto i_\phi$. This defines a bilinear form $Q(\phi,\psi) = i_\phi(\psi)$ on
  $\F^N \oplus \F^N$, and so $Q(\phi) = i_\phi(\phi)$ defines a quadratic form
  on $\F^N$. 

  Moreover, $L$ is Lagrangian, so
  \[
    \omega\big((\phi,i_\phi), (\psi,i_\psi)\big) = i_\psi(\phi) - i_\phi(\psi) =
    0,
  \]
  and so $Q(-,-)$ is a symmetric bilinear form. This gives a one-to-one
  correspondence between Lagrangian subspaces of specified type, symmetric
  bilinear forms, and quadratic forms, and so in particular gives the claimed
  one-to-one correspondence. 
\end{proof}

In particular, every Dirichlet form defines a Lagrangian subspace. 

\subsection{Lagrangian relations}

Recall that a relation between sets $X$ and $Y$ is a subset $R$ of their product
$X \times Y$. Furthermore, given relations $R \subseteq X \times Y$ and $S
\subseteq Y \times Z$, there is a composite relation $(S \circ R) \subseteq X
\times Z$ given by pairs $(x,z)$ such that there exists $y \in Y$ with $(x,y)
\in R$ and $(y,z) \in S$---a direct generalization of function composition. A
Lagrangian relation between symplectic vector spaces $V_1$ and $V_2$ is a
relation between $V_1$ and $V_2$ that forms a Lagrangian subspace of the
symplectic vector space $\overline{V_1} \oplus V_2$. This gives us a
way to think of certain Lagrangian subspaces, such as those arising from
circuits, as morphisms, giving a way to compose them.

\begin{definition}
  A \define{Lagrangian relation} $L\maps V_1 \to V_2$ is a Lagrangian subspace $L$
  of $\overline{V_1} \oplus V_2$. 
\end{definition}

This is a generalization of the notion of symplectomorphism: any symplectomorphism
$f\maps V_1 \to V_2$ forms a Lagrangian subspace when viewed as a
relation $f \subseteq \overline{V_1} \oplus V_2$. More generally, any symplectic
map $f\maps V_1 \to V_2$ forms an isotropic subspace when viewed as a relation in
$\overline{V_1} \oplus V_2$. 

Importantly for us, the composite of two Lagrangian relations is again a
Lagrangian relation.  This is well-known \cite{Weinstein}, but sufficiently 
easy and important to us that we provide a proof.

\begin{proposition} \label{prop:lagrangian_composition}
  Let $L\maps V_1 \to V_2$ and $L'\maps V_2 \to V_3$ be Lagrangian relations. Then their
  composite relation $L' \circ L$ is a Lagrangian relation $V_1 \to V_3$.
\end{proposition}

We prove this proposition by way of two lemmas detailing how the Lagrangian
property is preserved under various operations. The first lemma says that the
intersection of a Lagrangian space with a coisotropic space is in some sense
Lagrangian, once we account for the complement.

\begin{lemma} \label{restriction_of_lagrangians}
  Let $L \subseteq V$ be a Lagrangian subspace of a symplectic vector space $V$,
  and $S \subseteq V$ be an isotropic subspace of $V$. Then $(L\cap S^\circ) +S
  \subseteq V$ is Lagrangian in $V$.
\end{lemma}
\begin{proof}
  Recall from Proposition \ref{lagrangian_characterization} that a subspace is
  Lagrangian if and only if it is equal to its complement. The lemma is then
  immediate from the way taking the symplectic complement interacts with sums
  and intersections:
  \[
    ((L\cap S^\circ) +S)^\circ = (L\cap S^\circ)^\circ \cap S^\circ = (L^\circ +
    (S^\circ)^\circ) \cap S^\circ = (L+S) \cap S^\circ = (L \cap S^\circ)+(S
    \cap S^\circ) = (L\cap S^\circ) +S.
  \]
  Since $(L\cap S^\circ) +S$ is equal to its complement, it is Lagrangian.
\end{proof}

The second lemma says that if a subspace of a coisotropic space is Lagrangian,
taking quotients by the complementary isotropic space does not affect this.

\begin{lemma} \label{quotients_of_lagrangians}
  Let $L \subseteq V$ be a Lagrangian subspace of a symplectic vector space $V$,
  and $S \subseteq L$ an isotropic subspace of $V$ contained in $L$. Then $L/S
  \subseteq S^\circ/S$ is Lagrangian in the quotient symplectic space
  $S^\circ/S$.
\end{lemma}
\begin{proof}
  As $L$ is isotropic and the symplectic form on $S^\circ/S$ is given by
  $\omega'(v+S,u+S) = \omega(v,u)$, the quotient $L/S$ is immediately isotropic.
  Recall from Proposition \ref{lagrangian_characterization} that an isotropic
  subspace $S$ of a symplectic vector space $V$ is Lagrangian if and only if
  $\dim S = \frac12 \dim V$. Also recall that for any subspace $\dim S + \dim
  S^\circ = \dim V$. Thus
  \begin{multline*}
    \dim(L/S) = \dim L - \dim S = \tfrac12 \dim V - \dim S \\ = \tfrac12(\dim S
    + \dim S^\circ) - \dim S = \tfrac12(\dim S^\circ - \dim S) = \tfrac12
    \dim(S^\circ/S).
  \end{multline*}
  Thus $L/S$ is Lagrangian in $S^\circ/S$.
\end{proof}

Combining these two lemmas gives a proof that the composite of two Lagrangian
relations is again a Lagrangian relation.

\begin{proof}[Proof of Proposition \ref{prop:lagrangian_composition}]
  Let $\Delta$ be the diagonal subspace
  \[
    \Delta = \{(0,v_2,v_2,0) \mid v_2 \in V_2\} \subseteq \overline{V_1} \oplus
    V_2 \oplus \overline{V_2} \oplus V_3.
  \]
  Observe that $\Delta$ is isotropic, and has coisotropic complement
  \[
    \Delta^\circ = \{(v_1,v_2,v_2,v_3) \mid v_i \in V_i\} \subseteq
    \overline{V_1} \oplus V_2 \oplus \overline{V_2} \oplus V_3.
  \]
  As $\Delta$ is the kernel of the restriction of the projection map
  $\overline{V_1} \oplus V_2 \oplus \overline{V_2} \oplus V_3 \to \overline{V_1}
  \oplus V_3$ to $\Delta^\circ$, and after restriction this map is still
  surjective, the quotient space $\Delta^\circ/\Delta$ is isomorphic to
  $\overline{V_1} \oplus V_3$. 

  Now, by definition of composition of relations, 
  \[
    L' \circ L = \{(v_1,v_3) \mid \mbox{there exists } v_2 \in V_2 \mbox{ such
    that } (v_1,v_2) \in L, (v_2,v_3) \in L'\}.
  \]
  But note also that 
  \[
    L \oplus L'  = \{(v_1,v_2,v_2',v_3) \mid (v_1,v_2) \in L, (v_2',v_3) \in
    L'\},
  \]
  so 
  \[
    (L \oplus L')\cap \Delta^\circ = \{(v_1,v_2,v_2,v_3) \mid \mbox{there exists
    } v_2 \in V_2 \mbox{ such that } (v_1,v_2) \in L, (v_2,v_3) \in L'\}.
  \]
  Quotienting by $\Delta$ then gives
  \[
    L' \circ L = ((L \oplus L')\cap \Delta^\circ)+\Delta)/\Delta.
  \]
  As $L' \oplus L$ is Lagrangian in $\overline{V_1} \oplus V_2 \oplus
  \overline{V_2} \oplus V_3$, Lemma \ref{restriction_of_lagrangians} says that
  $(L' \oplus L)\cap \Delta^\circ)+\Delta$ is also Lagrangian in $\overline{V_1}
  \oplus V_2 \oplus \overline{V_2} \oplus V_3$. Lemma
  \ref{quotients_of_lagrangians} thus shows that $L' \circ L$ is Lagrangian in
  $\Delta^\circ/\Delta = \overline{V_1} \oplus V_3$, as required.
\end{proof}

Note that this composition is associative. We shall see later that this
composition agrees with composition of Dirichlet forms, and hence also
composition of circuits. 

\subsection{The dagger compact category of Lagrangian relations}

Lagrangian relations solve the identity problems we had with Dirichlet forms:
given a symplectic vector space $V$, the Lagrangian relation $\idn\maps V \to V$
specified by the Lagrangian subspace
\[
  \idn = \{(v,v) \mid v \in V\} \subseteq \overline{V} \oplus V,
\]
acts as an identity for composition of relations. We thus have a category.

\begin{definition}
  We write \define{$\LagrRel$} for the category with symplectic
  vector spaces as objects and Lagrangian relations as morphisms. 
\end{definition}

In fact the move to the setting of Lagrangian relations, rather than Dirichlet
forms, adds far richer structure than just identity morphisms. The category
$\LagrRel$ can be viewed as endowed with the structure of a dagger
compact category. We lay this out in steps.

\subsubsection*{Symmetric monoidal structure}

We define the tensor product of two objects of $\LagrRel$ to be their
direct sum. Similarly, we define the tensor product of two morphisms $L\maps U
\to V$, $L \subseteq \overline{U}\oplus V$ and $K\maps T \to W$, $K \subseteq
\overline{T} \oplus W$ to be their direct sum
\[
  L \oplus K \subseteq \overline{U}\oplus V \oplus\overline{T} \oplus W,
\]
\emph{but} considered as a subspace of the naturally isomorphic space
$\overline{U \oplus T} \oplus V \oplus W$.  Despite this subtlety, we abuse our
notation and write their tensor product $L \oplus K\maps U \oplus T \to V \oplus
W$, and move on having sounded this note of caution. 

Note that the direct sum of two Lagrangian subspaces is
again Lagrangian in the direct sum of their ambient spaces, and the zero
dimensional vector space $\{0\}$ acts as an identity for direct sum. Indeed,
defining for all objects $U,V,W$ in $\LagrRel$ unitors: 
\begin{align*}
  \lambda_V &= \{(0,v,v)\} \subseteq \overline{\{0\} \oplus V} \oplus V, \\
  \rho_V &= \{(v,0,v)\} \subseteq \overline{V \oplus \{0\}} \oplus V,
\end{align*}
associators:
\[
  \alpha_{U,V,W}= \{(u,v,w,u,v,w)\} \subseteq \overline{(U \oplus V)\oplus W}
  \oplus U \oplus (V \oplus W),
\]
and braidings:
\[
  \s_{U,V} = \{(u,v,v,u) \mid u \in U, v \in V\} \subseteq \overline{U \oplus V}
  \oplus V \oplus U,
\]
we have a symmetric monoidal category.  Note that all these structure
maps come from symplectomorphisms between the domain and codomain. From this
viewpoint it is immediate that all the necessary diagrams commute, so we
have a symmetric monoidal category. 

\subsubsection*{Duals for objects}

Each object $V$ of $\LagrRel$ is dual to its conjugate space $\overline
V$, with cup $\eta\maps \{0\} \to \overline{V} \oplus V$ given by 
\[
  \eta = \{(0,v,v) \mid v \in V\} \subseteq \overline{\{0\}} \oplus \overline{V}
  \oplus V
\]
and cap $\eps\maps V \oplus \overline{V} \to \{0\}$ given by
\[
  \eps = \{(v,v,0) \mid v \in V\} \subseteq \overline{V \oplus \overline{V}}
  \oplus \{0\}.
\]
It is straightforward to check these satisfy the zigzag identities.

\subsubsection*{Dagger structure}

Given symplectic vector spaces $U,V$, observe that the map
\begin{align*}
  (-)^\dagger\maps \overline{U} \oplus V &\longrightarrow \overline{V} \oplus U; \\
  (u,v) &\longmapsto (v,u)
\end{align*} 
takes Lagrangian subspaces of the domain to Lagrangian subspaces of the
codomain. Thus we can view it as a map $(-)^\dagger$ taking morphisms $L\maps U \to V$
of $\LagrRel$ to morphisms $L^\dagger\maps V \to U$. This defines a
dagger structure on $\LagrRel$, which makes this category into a
symmetric monoidal dagger category.

Moreover, every object in $\mathrm{LagrRel}$ has a dagger dual: it is clear that
$\eta^\dagger = \eps \circ \s$.   This category thus becomes a dagger compact
category.

\subsection{Names of Lagrangian relations} \label{subsec:names}

This brief subsection illustrates a guiding principle of this paper: \emph{duals for
objects allow us to blur the distinction between composition of morphisms and the tensor product of morphisms}. We will make use of this when we prove
the functoriality of the black box functor.

Observe that a Lagrangian relation $L\maps \{0\} \to V$ is the same as a Lagrangian
subspace of $V$. Moreover, given a Lagrangian relation $L\maps U \to V$, 
\[
  \begin{tikzpicture}
    \begin{pgfonlayer}{nodelayer}
      \node [style=none] (0) at (-0.25, 0.25) {};
      \node [style=none] (1) at (0.25, 0.25) {};
      \node [style=none] (2) at (-0.25, -0.25) {};
      \node [style=none] (3) at (0.25, -0.25) {};
      \node [style=none] (4) at (0, 0.75) {};
      \node [style=none] (5) at (0, -0.75) {};
      \node [style=none] (6) at (0, 0.25) {};
      \node [style=none] (7) at (0, -0.25) {};
      \node [style=none] (8) at (0, -0) {$L$};
      \node [style=none] (9) at (0, 1) {$U$};
      \node [style=none] (10) at (0, -1) {$V$};
    \end{pgfonlayer}
    \begin{pgfonlayer}{edgelayer}
      \draw (0.center) to (1.center);
      \draw (1.center) to (3.center);
      \draw (3.center) to (2.center);
      \draw (2.center) to (0.center);
      \draw (4.center) to (6.center);
      \draw (7.center) to (5.center);
    \end{pgfonlayer}
  \end{tikzpicture} 
\]
compactness allows us to view it as a Lagrangian relation $\{0\} \to
\overline{U} \oplus V$:
\[
  \begin{tikzpicture}
    \begin{pgfonlayer}{nodelayer}
      \node [style=none] (0) at (-0.25, 0.25) {};
      \node [style=none] (1) at (0.25, 0.25) {};
      \node [style=none] (2) at (-0.25, -0.25) {};
      \node [style=none] (3) at (0.25, -0.25) {};
      \node [style=none] (4) at (-0.5, 0.75) {};
      \node [style=none] (5) at (0, -0.75) {};
      \node [style=none] (6) at (0, 0.25) {};
      \node [style=none] (7) at (0, -0.25) {};
      \node [style=none] (8) at (0, -0) {$L$};
      \node [style=none] (9) at (-1, -1) {$\overline{U}$};
      \node [style=none] (10) at (0, -1) {$V$};
      \node [style=none] (11) at (-1, -0.75) {};
      \node [style=none] (12) at (-1, 0.25) {};
    \end{pgfonlayer}
    \begin{pgfonlayer}{edgelayer}
      \draw (0.center) to (1.center);
      \draw (1.center) to (3.center);
      \draw (3.center) to (2.center);
      \draw (2.center) to (0.center);
      \draw [bend left=45, looseness=1.00] (4.center) to (6.center);
      \draw (7.center) to (5.center);
      \draw (11.center) to (12.center);
      \draw [bend left=45, looseness=1.00] (12.center) to (4.center);
    \end{pgfonlayer}
  \end{tikzpicture}
\]
We call this subspace the \define{name} of the Lagrangian relation $L$; indeed,
we have used this one-to-one correspondence between morphisms and
their names to define Lagrangian relations.

By compactness, we have the equation
\[
  \begin{aligned}
    \begin{tikzpicture}
      \begin{pgfonlayer}{nodelayer}
	\node [style=none] (0) at (-0.75, 0.5) {};
	\node [style=none] (1) at (-0.25, 0.5) {};
	\node [style=none] (2) at (-0.75, -0) {};
	\node [style=none] (3) at (-0.25, -0) {};
	\node [style=none] (4) at (-1, 1) {};
	\node [style=none] (5) at (-0.5, 0.5) {};
	\node [style=none] (6) at (-0.5, -0) {};
	\node [style=none] (7) at (-0.5, 0.25) {$L$};
	\node [style=none] (8) at (-1.5, -1) {$\overline{U}$};
	\node [style=none] (9) at (-1.5, -0.75) {};
	\node [style=none] (10) at (-1.5, 0.5) {};
	\node [style=none] (11) at (1.5, -0) {};
	\node [style=none] (12) at (1.75, 0.5) {};
	\node [style=none] (13) at (1.5, -1) {$W$};
	\node [style=none] (14) at (0.5, -0) {};
	\node [style=none] (15) at (1.25, -0) {};
	\node [style=none] (16) at (1.75, -0) {};
	\node [style=none] (17) at (1.5, -0.75) {};
	\node [style=none] (18) at (1, 1) {};
	\node [style=none] (19) at (1.5, 0.25) {$M$};
	\node [style=none] (20) at (0.5, 0.5) {};
	\node [style=none] (21) at (1.5, 0.5) {};
	\node [style=none] (22) at (1.25, 0.5) {};
	\node [style=none] (23) at (0, -0.5) {};
      \end{pgfonlayer}
      \begin{pgfonlayer}{edgelayer}
	\draw (0.center) to (1.center);
	\draw (1.center) to (3.center);
	\draw (3.center) to (2.center);
	\draw (2.center) to (0.center);
	\draw [bend left=45, looseness=1.00] (4.center) to (5.center);
	\draw (9.center) to (10.center);
	\draw [bend left=45, looseness=1.00] (10.center) to (4.center);
	\draw (22.center) to (12.center);
	\draw (12.center) to (16.center);
	\draw (16.center) to (15.center);
	\draw (15.center) to (22.center);
	\draw [bend left=45, looseness=1.00] (18.center) to (21.center);
	\draw (11.center) to (17.center);
	\draw (14.center) to (20.center);
	\draw [bend left=45, looseness=1.00] (20.center) to (18.center);
	\draw [bend right=45, looseness=1.00] (6.center) to (23.center);
	\draw [bend right=45, looseness=1.00] (23.center) to (14.center);
      \end{pgfonlayer}
    \end{tikzpicture}
  \end{aligned}
  \qquad
  =
  \qquad
  \begin{aligned}
    \begin{tikzpicture}
      \begin{pgfonlayer}{nodelayer}
	\node [style=none] (0) at (-0.75, 0.5) {};
	\node [style=none] (1) at (-0.25, 0.5) {};
	\node [style=none] (2) at (-0.75, -0) {};
	\node [style=none] (3) at (-0.25, -0) {};
	\node [style=none] (4) at (-1, 1) {};
	\node [style=none] (5) at (-0.5, 0.5) {};
	\node [style=none] (6) at (-0.5, -0) {};
	\node [style=none] (7) at (-0.5, 0.25) {$L$};
	\node [style=none] (8) at (-1.5, -1.25) {$\overline{U}$};
	\node [style=none] (9) at (-1.5, -1) {};
	\node [style=none] (10) at (-1.5, 0.5) {};
	\node [style=none] (11) at (-0.5, -1) {};
	\node [style=none] (12) at (-0.25, -0.25) {};
	\node [style=none] (13) at (-0.5, -1.25) {$W$};
	\node [style=none] (14) at (-0.75, -0.75) {};
	\node [style=none] (15) at (-0.25, -0.75) {};
	\node [style=none] (16) at (-0.5, -0.75) {};
	\node [style=none] (17) at (-0.5, -0.5) {$M$};
	\node [style=none] (18) at (-0.5, -0.25) {};
	\node [style=none] (19) at (-0.75, -0.25) {};
      \end{pgfonlayer}
      \begin{pgfonlayer}{edgelayer}
	\draw (0.center) to (1.center);
	\draw (1.center) to (3.center);
	\draw (3.center) to (2.center);
	\draw (2.center) to (0.center);
	\draw [bend left=45, looseness=1.00] (4.center) to (5.center);
	\draw (9.center) to (10.center);
	\draw [bend left=45, looseness=1.00] (10.center) to (4.center);
	\draw (19.center) to (12.center);
	\draw (12.center) to (15.center);
	\draw (15.center) to (14.center);
	\draw (14.center) to (19.center);
	\draw (11.center) to (16.center);
	\draw (6.center) to (18.center);
      \end{pgfonlayer}
    \end{tikzpicture}
  \end{aligned}
\]
Here the right hand side is the name of the composite $M \circ L$ of Lagrangian
relations, while the left hand side is the direct sum of the names of $L$ and $M$
post-composed with the Lagrangian relation
\[
  \begin{tikzpicture}
    \begin{pgfonlayer}{nodelayer}
      \node [style=none] (0) at (-0.5, -0) {};
      \node [style=none] (1) at (-1.5, -1.25) {$\overline{U}$};
      \node [style=none] (2) at (-1.5, -1) {};
      \node [style=none] (3) at (-1.5, 0.25) {};
      \node [style=none] (4) at (1.5, 0.25) {};
      \node [style=none] (5) at (1.5, -1.25) {$W$};
      \node [style=none] (6) at (0.5, -0) {};
      \node [style=none] (7) at (1.5, -1) {};
      \node [style=none] (8) at (0, -0.5) {};
      \node [style=none] (9) at (-0.5, 0.25) {};
      \node [style=none] (10) at (0.5, 0.25) {};
      \node [style=none] (11) at (-1.5, 0.5) {$\overline{U}$};
      \node [style=none] (12) at (-0.5, 0.5) {$V$};
      \node [style=none] (13) at (0.5, 0.5) {$\overline{V}$};
      \node [style=none] (14) at (1.5, 0.5) {$W$};
    \end{pgfonlayer}
    \begin{pgfonlayer}{edgelayer}
      \draw (2.center) to (3.center);
      \draw (4.center) to (7.center);
      \draw [bend right=45, looseness=1.00] (0.center) to (8.center);
      \draw [bend right=45, looseness=1.00] (8.center) to (6.center);
      \draw (0.center) to (9.center);
      \draw (6.center) to (10.center);
    \end{pgfonlayer}
  \end{tikzpicture}
\]
Thus this relation above, the product of a cap and two identity maps, enacts
composition of Lagrangian relations. The proof of Proposition
\ref{prop:lagrangian_composition}, that the composite of two Lagrangian
relations is again a Lagrangian relation, makes use of this fact. We shall also
return to it when discussing our functor $\Circ \to \LagrRel$.
Note the similarity in form between our diagrams of cospans and of names of
Lagrangian relations. 

\section{Ideal wires and corelations} \label{sec:corel}
%%fakesubsection
In the previous section our exploration of the meaning of circuit diagrams 
culminated with our understanding of behaviors as Lagrangian subspaces.  We now 
turn our attention to how circuit components fit together, and the category of 
operations.  In this section we shall see that the algebra of connections is 
described by the concept of corelations, a generalization of the notion of 
function that forgets the directionality from the domain to the codomain. We
then observe that Kirchhoff's laws follow directly from interpreting these
structures in the category of linear relations.

\subsection{Ideal wires}

To motivate the definition of this category, let us start with a set of input
terminals $X$, and a set of output terminals $Y$.  We may connect these
terminals with ideal wires of zero impedance, whichever way we like---input to
input, output to output, input to output---producing something like:
\[
  \begin{tikzpicture}[circuit ee IEC]
	\begin{pgfonlayer}{nodelayer}
		\node [contact] (0) at (-2, 1) {};
		\node [contact] (1) at (-2, 0.5) {};
		\node [contact] (2) at (-2, -0) {};
		\node [contact] (3) at (-2, -0.5) {};
		\node [contact] (4) at (-2, -1) {};
		\node [contact] (5) at (1, 0.75) {};
		\node [contact] (6) at (1, 0.25) {};
		\node [contact] (7) at (1, -0.25) {};
		\node [contact] (8) at (1, -0.75) {};
		\node [style=none] (9) at (-2.75, -0) {$X$};
		\node [style=none] (10) at (1.75, -0) {$Y$};
	\end{pgfonlayer}
	\begin{pgfonlayer}{edgelayer}
	  \draw [thick] (0.center) to (5.center);
		\draw [thick] (5.center) to (1.center);
		\draw [thick] (6.center) to (1.center);
		\draw [thick] (3.center) to (2.center);
		\draw [thick] (4.center) to (8.center);
		\draw [thick] (5.center) to (6.center);
		\draw [thick] (6.center) to (0.center);
	\end{pgfonlayer}
\end{tikzpicture}
\]
In doing so, we introduce a notion of equivalence on our terminals, where two 
terminals are equivalent if we, or if electrons, can traverse from one to 
another via some sequence of wires.   Because of this, we consider our 
perfectly-conducting components to be equivalence relations on $X+Y$,
transforming the above picture into
\[
  \begin{tikzpicture}[circuit ee IEC]
	\begin{pgfonlayer}{nodelayer}
		\node [contact, outer sep=5pt] (0) at (-2, 1) {};
		\node [contact, outer sep=5pt] (1) at (-2, 0.5) {};
		\node [contact, outer sep=5pt] (2) at (-2, -0) {};
		\node [contact, outer sep=5pt] (3) at (-2, -0.5) {};
		\node [contact, outer sep=5pt] (4) at (-2, -1) {};
		\node [contact, outer sep=5pt] (5) at (1, 0.75) {};
		\node [contact, outer sep=5pt] (6) at (1, 0.25) {};
		\node [contact, outer sep=5pt] (7) at (1, -0.25) {};
		\node [contact, outer sep=5pt] (8) at (1, -0.75) {};
		\node [style=none] (9) at (-2.75, -0) {$X$};
		\node [style=none] (10) at (1.75, -0) {$Y$};
		\node [style=none] (11) at (-0.5, 0.625) {};
		\node [style=none] (12) at (-0.5, -0.25) {};
		\node [style=none] (13) at (-0.5, -0.875) {};
	\end{pgfonlayer}
	\begin{pgfonlayer}{edgelayer}
		\draw [color=gray] (0.center) to (11.center);
		\draw [color=gray] (1.center) to (11.center);
		\draw [color=gray] (5.center) to (11.center);
		\draw [color=gray] (6.center) to (11.center);
		\draw [color=gray] (2.center) to (12.center);
		\draw [color=gray] (12.center) to (3.center);
		\draw [color=gray] (4.center) to (13.center);
		\draw [color=gray] (13.center) to (8.center);
		\draw [rounded corners=5pt, dotted] 
   (node cs:name=0, anchor=north west) --
   (node cs:name=1, anchor=south west) --
   (node cs:name=6, anchor=south east) --
   (node cs:name=5, anchor=north east) --
   cycle;
		\draw [rounded corners=5pt, dotted] 
   (node cs:name=2, anchor=north west) --
   (node cs:name=3, anchor=south west) --
   (node cs:name=3, anchor=south east) --
   (node cs:name=2, anchor=north east) --
   cycle;
		\draw [rounded corners=5pt, dotted] 
   (node cs:name=4, anchor=north west) --
   (node cs:name=4, anchor=south west) --
   (node cs:name=8, anchor=south east) --
   (node cs:name=8, anchor=north east) --
   cycle;
		\draw [rounded corners=5pt, dotted] 
   (node cs:name=7, anchor=north west) --
   (node cs:name=7, anchor=south west) --
   (node cs:name=7, anchor=south east) --
   (node cs:name=7, anchor=north east) --
   cycle;
	\end{pgfonlayer}
\end{tikzpicture}
\]
The dotted lines indicate equivalence classes of points, while for reference the
grey lines indicate ideal wires connecting these points, running through a
central hub.

Given another circuit of this sort, say from sets $Y$ to $Z$,
\[
\begin{tikzpicture}[circuit ee IEC]
	\begin{pgfonlayer}{nodelayer}
		\node [style=none] (0) at (-2.75, -0) {$Y$};
		\node [style=none] (1) at (1.75, 0) {$Z$};
		\node [contact, outer sep=5pt] (2) at (-2, 0.75) {};
		\node [contact, outer sep=5pt] (3) at (-2, 0.25) {};
		\node [contact, outer sep=5pt] (4) at (-2, -0.25) {};
		\node [contact, outer sep=5pt] (5) at (-2, -0.75) {};
		\node [contact, outer sep=5pt] (6) at (1, 1) {};
		\node [contact, outer sep=5pt] (7) at (1, 0.5) {};
		\node [contact, outer sep=5pt] (8) at (1, -0) {};
		\node [contact, outer sep=5pt] (9) at (1, -0.5) {};
		\node [contact, outer sep=5pt] (10) at (1, -1) {};
		\node [style=none] (11) at (-0.5, 0.75) {};
		\node [style=none] (12) at (-0.5, -0.25) {};
		\node [style=none] (13) at (-0.5, -0.875) {};
	\end{pgfonlayer}
	\begin{pgfonlayer}{edgelayer}
	  \draw [color=gray] (2.center) to (11.center);
		\draw [color=gray] (11.center) to (6.center);
		\draw [color=gray] (3.center) to (11.center);
		\draw [color=gray] (8.center) to (12.center);
		\draw [color=gray] (12.center) to (4.center);
		\draw [color=gray] (9.center) to (12.center);
		\draw [color=gray] (5.center) to (13.center);
		\draw [color=gray] (13.center) to (10.center);
	\end{pgfonlayer}
		\draw [rounded corners=5pt, dotted] 
   (node cs:name=2, anchor=north west) --
   (node cs:name=3, anchor=south west) --
   (node cs:name=6, anchor=south east) --
   (node cs:name=6, anchor=north east) --
   cycle;
		\draw [rounded corners=5pt, dotted] 
   (node cs:name=4, anchor=north west) --
   (node cs:name=4, anchor=south west) --
   (node cs:name=9, anchor=south east) --
   (node cs:name=8, anchor=north east) --
   cycle;
		\draw [rounded corners=5pt, dotted] 
   (node cs:name=5, anchor=north west) --
   (node cs:name=5, anchor=south west) --
   (node cs:name=10, anchor=south east) --
   (node cs:name=10, anchor=north east) --
   cycle;
		\draw [rounded corners=5pt, dotted] 
   (node cs:name=7, anchor=north west) --
   (node cs:name=7, anchor=south west) --
   (node cs:name=7, anchor=south east) --
   (node cs:name=7, anchor=north east) --
   cycle;
\end{tikzpicture}
\]
we may combine these circuits in to a circuit $X$ to $Z$
\[
  \begin{aligned}
\begin{tikzpicture}[circuit ee IEC]
	\begin{pgfonlayer}{nodelayer}
		\node [contact, outer sep=5pt] (0) at (1, 0.75) {};
		\node [contact, outer sep=5pt] (1) at (1, 0.25) {};
		\node [contact, outer sep=5pt] (2) at (1, -0.25) {};
		\node [contact, outer sep=5pt] (3) at (1, -0.75) {};
		\node [style=none] (4) at (-2.75, -0) {$X$};
		\node [style=none] (5) at (4.75, -0) {$Z$};
		\node [contact, outer sep=5pt] (6) at (-2, 1) {};
		\node [contact, outer sep=5pt] (7) at (-2, -0.5) {};
		\node [contact, outer sep=5pt] (8) at (-2, 0.5) {};
		\node [contact, outer sep=5pt] (9) at (-2, -0) {};
		\node [contact, outer sep=5pt] (10) at (-2, -1) {};
		\node [contact, outer sep=5pt] (11) at (4, -0) {};
		\node [contact, outer sep=5pt] (12) at (4, -1) {};
		\node [contact, outer sep=5pt] (13) at (4, -0.5) {};
		\node [contact, outer sep=5pt] (14) at (4, 0.5) {};
		\node [style=none] (15) at (-0.5, 0.625) {};
		\node [style=none] (16) at (-0.5, -0.25) {};
		\node [style=none] (17) at (-0.5, -0.875) {};
		\node [style=none] (18) at (2.5, -0.875) {};
		\node [contact, outer sep=5pt] (19) at (4, 1) {};
		\node [style=none] (20) at (1, -1.25) {$Y$};
		\node [style=none] (21) at (2.5, 0.75) {};
		\node [style=none] (22) at (2.5, -0.25) {};
	\end{pgfonlayer}
	\begin{pgfonlayer}{edgelayer}
		\draw [color=gray] (6.center) to (15.center);
		\draw [color=gray] (8.center) to (15.center);
		\draw [color=gray] (0.center) to (15.center);
		\draw [color=gray] (1.center) to (15.center);
		\draw [color=gray] (9.center) to (16.center);
		\draw [color=gray] (7.center) to (16.center);
		\draw [color=gray] (10.center) to (17.center);
		\draw [color=gray] (17.center) to (3.center);
		\draw [color=gray] (3.center) to (18.center);
		\draw [color=gray] (18.center) to (12.center);
		\draw [color=gray] (0.center) to (21.center);
		\draw [color=gray] (1.center) to (21.center);
		\draw [color=gray] (21.center) to (19.center);
		\draw [color=gray] (2.center) to (22.center);
		\draw [color=gray] (22.center) to (11.center);
		\draw [color=gray] (22.center) to (13.center);
		\draw [rounded corners=5pt, dotted] 
   (node cs:name=6, anchor=north west) --
   (node cs:name=8, anchor=south west) --
   (node cs:name=1, anchor=south east) --
   (node cs:name=0, anchor=north east) --
   cycle;
		\draw [rounded corners=5pt, dotted] 
   (node cs:name=9, anchor=north west) --
   (node cs:name=7, anchor=south west) --
   (node cs:name=7, anchor=south east) --
   (node cs:name=9, anchor=north east) --
   cycle;
		\draw [rounded corners=5pt, dotted] 
   (node cs:name=10, anchor=north west) --
   (node cs:name=10, anchor=south west) --
   (node cs:name=3, anchor=south east) --
   (node cs:name=3, anchor=north east) --
   cycle;
		\draw [rounded corners=5pt, dotted] 
   (node cs:name=2, anchor=north west) --
   (node cs:name=2, anchor=south west) --
   (node cs:name=2, anchor=south east) --
   (node cs:name=2, anchor=north east) --
   cycle;
		\draw [rounded corners=5pt, dotted] 
   (node cs:name=0, anchor=north west) --
   (node cs:name=1, anchor=south west) --
   (node cs:name=19, anchor=south east) --
   (node cs:name=19, anchor=north east) --
   cycle;
		\draw [rounded corners=5pt, dotted] 
   (node cs:name=2, anchor=north west) --
   (node cs:name=2, anchor=south west) --
   (node cs:name=13, anchor=south east) --
   (node cs:name=11, anchor=north east) --
   cycle;
		\draw [rounded corners=5pt, dotted] 
   (node cs:name=3, anchor=north west) --
   (node cs:name=3, anchor=south west) --
   (node cs:name=12, anchor=south east) --
   (node cs:name=12, anchor=north east) --
   cycle;
		\draw [rounded corners=5pt, dotted] 
   (node cs:name=14, anchor=north west) --
   (node cs:name=14, anchor=south west) --
   (node cs:name=14, anchor=south east) --
   (node cs:name=14, anchor=north east) --
   cycle;
	\end{pgfonlayer}
\end{tikzpicture}
\end{aligned}
\:
  =
\:
\begin{aligned}
\begin{tikzpicture}[circuit ee IEC]
	\begin{pgfonlayer}{nodelayer}
		\node [style=none] (0) at (-2.75, -0) {$X$};
		\node [style=none] (1) at (1.75, -0) {$Z$};
		\node [contact, outer sep=5pt] (2) at (-2, 1) {};
		\node [contact, outer sep=5pt] (3) at (-2, -0.5) {};
		\node [contact, outer sep=5pt] (4) at (-2, 0.5) {};
		\node [contact, outer sep=5pt] (5) at (-2, -0) {};
		\node [contact, outer sep=5pt] (6) at (-2, -1) {};
		\node [contact, outer sep=5pt] (7) at (1, -0) {};
		\node [contact, outer sep=5pt] (8) at (1, -1) {};
		\node [contact, outer sep=5pt] (9) at (1, -0.5) {};
		\node [contact, outer sep=5pt] (10) at (1, 0.5) {};
		\node [style=none] (11) at (-0.5, 0.875) {};
		\node [style=none] (12) at (-1, -0.25) {};
		\node [contact, outer sep=5pt] (13) at (1, 1) {};
		\node [style=none] (14) at (0, -0.25) {};
	\end{pgfonlayer}
	\begin{pgfonlayer}{edgelayer}
		\draw [color=gray] (2.center) to (11.center);
		\draw [color=gray] (4.center) to (11.center);
		\draw [color=gray] (5.center) to (12.center);
		\draw [color=gray] (3.center) to (12.center);
		\draw [color=gray] (14.center) to (7.center);
		\draw [color=gray] (14.center) to (9.center);
		\draw [color=gray] (6.center) to (8.center);
		\draw [color=gray] (11.center) to (13.center);
		\draw [rounded corners=5pt, dotted] 
   (node cs:name=2, anchor=north west) --
   (node cs:name=4, anchor=south west) --
   (node cs:name=13, anchor=south east) --
   (node cs:name=13, anchor=north east) --
   cycle;
		\draw [rounded corners=5pt, dotted] 
   (node cs:name=5, anchor=north west) --
   (node cs:name=3, anchor=south west) --
   (node cs:name=3, anchor=south east) --
   (node cs:name=5, anchor=north east) --
   cycle;
		\draw [rounded corners=5pt, dotted] 
   (node cs:name=6, anchor=north west) --
   (node cs:name=6, anchor=south west) --
   (node cs:name=8, anchor=south east) --
   (node cs:name=8, anchor=north east) --
   cycle;
		\draw [rounded corners=5pt, dotted] 
   (node cs:name=10, anchor=north west) --
   (node cs:name=10, anchor=south west) --
   (node cs:name=10, anchor=south east) --
   (node cs:name=10, anchor=north east) --
   cycle;
		\draw [rounded corners=5pt, dotted] 
   (node cs:name=7, anchor=north west) --
   (node cs:name=9, anchor=south west) --
   (node cs:name=9, anchor=south east) --
   (node cs:name=7, anchor=north east) --
   cycle;
	\end{pgfonlayer}
\end{tikzpicture}
\end{aligned}
\]
by taking the transitive closure of the two equivalence relations, and then
restricting this to an equivalence relation on $X+Z$. This in fact defines a
dagger compact category of fundamental importance: the category of \emph{corelations}. 

Ellerman gives a detailed treatment of corelations from a logic viewpoint in
\cite{E}, while basic category theoretic aspects can be found in Lawvere and
Rosebrugh \cite{LR}.  However, to make this paper self-contained, we explain them
here.

\subsection{The category of corelations}

In the category of sets we hold the fundamental relationship between sets to be
that of functions. These encode the idea of a deterministic process that takes
each element of one set to a unique element of the other. For the study of
networks this is less appropriate, as the relationship between terminals is not
an input-output one, but rather one of interconnection. 

In particular, the direction of a function becomes irrelevant, and to describe
these interconnections via the category of sets we must develop an understanding
of how to compose functions head to head and tail to tail. We have so far used
cospans and pushouts to address this.  Cospans, however, come with an apex, which
represents extraneous structure beyond the two sets we wish to specify a
relationship between. Corelations arise from omitting this information.

\begin{definition}
  A \define{corelation} $\alpha\maps X \to Y$ between finite sets $X$ and $Y$ is a
partition $\alpha$ of the disjoint union $X+Y$.
\end{definition}

That is, for finite sets $X$ and $Y$, a corelation is a collection of nonempty subsets
$\alpha = \{A_1,A_2,\dots,A_n\}$ of $X+Y$ such that
\begin{enumerate}[(i)] 
  \item $\alpha$ does not contain the empty set.  
  \item $\bigcup_{i=1}^n A_i = X+Y$.
  \item $A_i \cap A_j = \varnothing$ whenever $i \ne j$.
\end{enumerate}

For example, we can take a circuit of ideal wires with $X$ as the set of inputs
and $Y$ as the set of outputs:
\[
  \begin{tikzpicture}[circuit ee IEC]
	\begin{pgfonlayer}{nodelayer}
		\node [contact] (0) at (-2, 1) {};
		\node [contact] (1) at (-2, 0.5) {};
		\node [contact] (2) at (-2, -0) {};
		\node [contact] (3) at (-2, -0.5) {};
		\node [contact] (4) at (-2, -1) {};
		\node [contact] (5) at (1, 0.75) {};
		\node [contact] (6) at (1, 0.25) {};
		\node [contact] (7) at (1, -0.25) {};
		\node [contact] (8) at (1, -0.75) {};
		\node [style=none] (9) at (-2.75, -0) {$X$};
		\node [style=none] (10) at (1.75, -0) {$Y$};
	\end{pgfonlayer}
	\begin{pgfonlayer}{edgelayer}
	  \draw [thick] (0.center) to (5.center);
		\draw [thick] (5.center) to (1.center);
		\draw [thick] (6.center) to (1.center);
		\draw [thick] (3.center) to (2.center);
		\draw [thick] (4.center) to (8.center);
		\draw [thick] (5.center) to (6.center);
		\draw [thick] (6.center) to (0.center);
	\end{pgfonlayer}
\end{tikzpicture}
\]
and define a corelation $\alpha \maps X \to Y$ for which terminals lie in 
the same set of the partition $\alpha = \{A_1,A_2,\dots,A_n\}$ when we can travel from one terminal to another following a path of wires:
\[
  \begin{tikzpicture}[circuit ee IEC]
	\begin{pgfonlayer}{nodelayer}
		\node [contact, outer sep=5pt] (0) at (-2, 1) {};
		\node [contact, outer sep=5pt] (1) at (-2, 0.5) {};
		\node [contact, outer sep=5pt] (2) at (-2, -0) {};
		\node [contact, outer sep=5pt] (3) at (-2, -0.5) {};
		\node [contact, outer sep=5pt] (4) at (-2, -1) {};
		\node [contact, outer sep=5pt] (5) at (1, 0.75) {};
		\node [contact, outer sep=5pt] (6) at (1, 0.25) {};
		\node [contact, outer sep=5pt] (7) at (1, -0.25) {};
		\node [contact, outer sep=5pt] (8) at (1, -0.75) {};
		\node [style=none] (9) at (-2.75, -0) {$X$};
		\node [style=none] (10) at (1.75, -0) {$Y$};
		\node [style=none] (11) at (-0.5, 0.625) {};
		\node [style=none] (12) at (-0.5, -0.25) {};
		\node [style=none] (13) at (-0.5, -0.875) {};
	\end{pgfonlayer}
	\begin{pgfonlayer}{edgelayer}
		\draw [color=gray] (0.center) to (11.center);
		\draw [color=gray] (1.center) to (11.center);
		\draw [color=gray] (5.center) to (11.center);
		\draw [color=gray] (6.center) to (11.center);
		\draw [color=gray] (2.center) to (12.center);
		\draw [color=gray] (12.center) to (3.center);
		\draw [color=gray] (4.center) to (13.center);
		\draw [color=gray] (13.center) to (8.center);
		\draw [rounded corners=5pt, dotted] 
   (node cs:name=0, anchor=north west) --
   (node cs:name=1, anchor=south west) --
   (node cs:name=6, anchor=south east) --
   (node cs:name=5, anchor=north east) --
   cycle;
		\draw [rounded corners=5pt, dotted] 
   (node cs:name=2, anchor=north west) --
   (node cs:name=3, anchor=south west) --
   (node cs:name=3, anchor=south east) --
   (node cs:name=2, anchor=north east) --
   cycle;
		\draw [rounded corners=5pt, dotted] 
   (node cs:name=4, anchor=north west) --
   (node cs:name=4, anchor=south west) --
   (node cs:name=8, anchor=south east) --
   (node cs:name=8, anchor=north east) --
   cycle;
		\draw [rounded corners=5pt, dotted] 
   (node cs:name=7, anchor=north west) --
   (node cs:name=7, anchor=south west) --
   (node cs:name=7, anchor=south east) --
   (node cs:name=7, anchor=north east) --
   cycle;
	\end{pgfonlayer}
\end{tikzpicture}
\]

The discussion in the previous section then motivates a rule for composing corelations. We compose a corelation $\alpha \maps X \to Y$ and a corelation $\beta \maps Y \to Z$ by finding the finest partition on $X+Y+Z$ that is coarser than both $\alpha$ and $\beta$ when restricted to $X+Y$ and $Y+Z$ respectively, and then restricting this to a
partition on $X+Z$. More explicitly, the corelation $\beta\circ\alpha = \{C_1,
C_2, \dots, C_m\}$ is the unique set of pairwise disjoint $C_i$ of the form 
\[
  C_i = \bigcup_j A_j \cap X \cup \bigcup_k B_k \cap Z
\]
for $j,k$ varying over indices such that 
\[
  \bigcup_j A_j \cap Y = \bigcup_k B_k \cap Y.
\]
This rule for composition is associative, as both pairwise methods of composing relations
$\alpha\maps X \to Y$, $\beta\maps Y \to Z$, and $\gamma\maps Z \to W$ amount to finding the finest partition on $X+Y+Z+W$ that is coarser than each of $\alpha$, $\beta$, and $\gamma$ when restricted to the relevant subset, and then restricting this
partition to a partition on $X+W$.  The pictures in the previous section make this clear.

The identity corelation $1_X\maps X \to X$ on a set $X$ is the map $[1_X,1_X]\maps X+X \to X$. Equivalently, it is the partition of $X+X$ such that each partition
comprises exactly two elements: an element $x \in X$ considered as an element of
both the first and then the second summand of $X+X$. We thus define a category:

\begin{definition}
  Let \define{$\mathrm{Corel}$} be the category with finite sets as objects and
  corelations between finite sets as morphisms. 
\end{definition}

This category becomes monoidal under disjoint union of finite sets, using the
fact that the union of partition of $X + Y$ and a partition of $X' + Y'$ is a
partition of $(X+X') + (Y+Y')$. Moreover, the commutativity and associativity of
the coproduct of finite sets also allows implies the category is symmetric
monoidal and indeed dagger compact: the braiding and cups and caps are
respectively given by the partitions of $(X+Y)+(Y+X)$ and $X+X$ where two
elements are in the same part of the partition if and only if they are equal as
elements of $X$ or $Y$, while a dagger is given by simply considering a
partition of $X+Y$ as a partition of $Y+X$..\footnote{In fact, if we take a
  skeleton of $\mathrm{Corel}$ we obtain the PROP for `extraspecial commutative
  Frobenius monoids', as defined by Baez and Erbele \cite{BE}.  We do not need
  this here, but it helps tie our current work to other work on categories in
  electrical engineering.  A proof can be assembled from Baez--Erbele together
  with the work of Lack \cite{La}.}

We can get a corelation $\alpha \maps X \to Y$ from a surjection $f \maps X + Y
\to S$ for some set $S$, by taking the sets in the partition $\alpha = (A_1,
\dots, A_n)$ of $X+Y$ to be the inverse images of the points of $S$.  Two
surjections $f \maps X + Y \to S$, $f \maps X + Y \to S'$ give the same
corelation if and only if there is an isomorphism $g \maps S \to S'$ making the
obvious triangle commute: $f' = g \circ f$.

By the universal property of the coproduct, surjections $X+Y \to A$ are in
one-to-one correspondence with jointly epic cospans $X \rightarrow A \leftarrow
Y$.   Thus, corelations can also be seen as isomorphism classes of jointly epic
cospans.  This lets us compose corelations by composing cospans and applying a
correction.  That is: given corelations $\alpha\maps X \to Y$, $\beta\maps Y \to
Z$, we may choose cospans representing them and compose these cospans.   The
composite cospan may not be jointly epic.  However, we can then replace the apex
of the composite cospan by the joint image of the feet.  The resulting jointly
epic cospan gives the composite corelation $\beta \circ \alpha\maps X \to Z$.
More precisely, we have the following proposition.

\begin{proposition}
  There is a strict symmetric monoidal dagger functor
  \[
    \xymatrix{
      \mathrm{Cospan(FinSet)}  \ar@<.5ex>[r] & \mathrm{Corel} 
    }
  \]
  mapping any cospan to the corelation it defines.
\end{proposition}
\begin{proof}
  This functor maps each finite sets to itself.  We thus need only discuss how
  it acts on morphisms.  A cospan in $\mathrm{FinSet}$ comprises a pair of
  functions $X \stackrel{f}\rightarrow N \stackrel{g}\leftarrow Y$. Restricting
  the apex $N$ down to the joint image $f(X) \cup g(Y)$ gives a jointly epic
  cospan $X \stackrel{f}\rightarrow f(X) \cup g(Y) \stackrel{g}\leftarrow Y$,
  and so a corelation $X \rightarrow Y$. As the elements of the apex not in the
  image of maps from the feet play only a trivial role in the pushout, and hence
  in composition of cospans, this map is functorial. It is now readily observed
  that the functor is symmetric monoidal.
\end{proof}

All this is dual to the more familiar connection between spans and relations,
where a relation is seen as an isomorphism class of jointly monic spans.  This
explains the name `corelation'. Note that neither relations nor corelations are
a generalization of the other.  The key property of corelations here is that it
forms a \emph{compact} category with disjoint union of finite sets as the
monoidal product.  This is not true of the category of relations between finite
sets.


Via the above proposition, the symmetric monoidal dagger functor embedding
$\mathrm{FinSet}$ into $\mathrm{Cospan(FinSet)}$ gives rise to a symmetric
monoidal dagger functor embedding $\mathrm{FinSet}$ into $\mathrm{Corel}$.
Composing this with the dagger structure on $\mathrm{Corel}$, we also obtain a
symmetric monoidal dagger functor embedding $\mathrm{FinSet}^{\mathrm{op}}$ into
$\mathrm{Corel}$.  Corelations thus give a method of composing functions
regardless of the direction of those functions.  Given some not necessarily
directed path of functions, for example
\[
  A \longrightarrow B \longleftarrow C \longleftarrow D \longrightarrow E
  \longrightarrow F,
\]
considering these functions as corelations gives a way to compose them.

In particular, while this mode of composition is simply composition of functions
for two functions head to tail, and turning a cospan into a corelation for two
functions head to head, for two tail to tail functions $C \leftarrow D
\rightarrow E$ we compute the composite of cospans
\[
  \xymatrix{
  & C && E \\
  C \ar@{=}[ur]^{1_C} && D \ar[ul] \ar[ur] && E \ar@{=}[ul]_{1_E}
  }
\]
before restricting the apex to arrive at a surjective function from $C+E$. This
implies the following lemma. 

\begin{lemma} \label{lem:pushoutcorelations}
  Let 
\[
  \xymatrix{
    & P \\
    A \ar[ur] && B \ar[ul] \\
    & N \ar[ur] \ar[ul]
  }
\]
be a pushout square in $\mathrm{FinSet}$. Then the composites of corelations $A
\rightarrow P \leftarrow B$ and $A \leftarrow N \rightarrow B$ are equal.
\end{lemma}

\subsection{Potentials on corelations} \label{ssec:potentialsoncorelations}

Chasing our interpretation of corelations as ideal wires, our aim for the
remainder of this section is to build a functor
\[
  S\maps \mathrm{Corel} \longrightarrow \LagrRel
\]
that expresses this interpretation. We break this functor down into the sum 
of two parts, according to the behaviors of potentials and currents
respectively. 

The consideration of potentials gives a functor $\Phi\maps \mathrm{Corel}
\to \mathrm{LinRel}$, where $\mathrm{LinRel}$ is the symmetric
monoidal dagger category of finite-dimensional $\F$-vector spaces, linear
relations, direct sum, and transpose. In particular, this functor expresses
Kirchhoff's voltage law: it requires that if two elements are in the same part
of the corelation partition---that is, if two nodes are connected by ideal
wires---then the potential at those two points must be the same.

This functor is a generalization of the
contravariant functor $\mathrm{FinSet} \to \mathrm{Vect}$ that maps a set to the
vector space of $\F$-valued functions on that set.

\begin{proposition}
  Define the functor 
  \[ 
    \Phi\maps \mathrm{Corel} \longrightarrow \mathrm{LinRel}, 
  \] 
  on objects by sending a finite set $X$ to the vector space $\F^X$, and on
  morphisms by sending a corelation $\alpha\maps X \to Y$ to the linear subspace
  $\Phi(\alpha)$ of $\F^X \oplus \F^Y$ comprising functions $\phi =
  [\phi_X,\phi_Y]\maps X+Y \to \F$ that are constant on each element of
  $\alpha$.  This is a symmetric monoidal dagger functor.
\end{proposition}

\begin{proof}
  For coherence maps we take the usual natural isomorphisms $\F^X \oplus \F^Y
  \cong \F^{X \times Y}$ and $\{0\} \cong \F^\varnothing$. We detail only the
  proof that $\Phi$ preserves composition; the other properties are
  straightforward to check.

  Let $\alpha\maps X \to Y$ and $\beta\maps Y \to Z$ be corelations. As $\Phi$
  maps corelations to relations, it is enough to check both inclusions
  $\Phi(\beta) \circ \Phi(\alpha) \subseteq \Phi(\beta\circ\alpha)$ and
  $\Phi(\beta\circ\alpha) \subseteq \Phi(\beta) \circ \Phi(\alpha)$. 

  \paragraph{$\Phi(\beta) \circ \Phi(\alpha) \subseteq \Phi(\beta\circ\alpha)$:}
  Let $\phi = [\phi_X,\phi_Z] \in \Phi(\beta) \circ \Phi(\alpha)$. We wish to show
  that for all $C_i \in \beta\circ\alpha$, for all $c,c' \in C_i$, we have
  $\phi(c) = \phi(c')$. To this end, note that there exists some $\phi_Y\maps Y \to \F$
  such that $\phi_{XY} := [\phi_X,\phi_Y] \in \Phi(\alpha)$ and $\phi_{YZ} :=
  [\phi_Y,\phi_Z] \in \Phi(\beta)$.  Furthermore, by definition this $\phi_Y$ has
  the property that for all $A_j \in \alpha$, for all $a,a' \in A_j$, we have
  $\phi_{XY}(a) = \phi_{XY}(a')$, and for all $B_k \in \beta$, for all $b,b' \in
  B_k$, we have $\phi_{YZ}(b) = \phi_{YZ}(b')$. Write $\phi_{XYZ} =
  [\phi_X,\phi_Y,\phi_Z]\maps X+Y+Z \to \F$.

  Our desired fact is thus true: for all $c,c' \in C_i$, there exists a sequence $c=c_0,
  c_1, \dots, c_n=c'$ in $X+Y+Z$ such that for all $\ell=0,1, \dots n-1$ we have
  either $c_\ell, c_{\ell+1} \in A_j$ for some $j$ or $c_\ell,c_{\ell+1} \in B_k$
  for some $k$, and hence that 
  \[
    \phi(c) = \phi_{X+Y+Z}(c_0)= \phi_{X+Y+Z}(c_1) = \dots = \phi_{X+Y+Z}(c_n) =
    \phi(c')
  \]
  as required.

  \paragraph{$\Phi(\beta\circ\alpha) \subseteq \Phi(\beta) \circ \Phi(\alpha)$:}
  Let $\phi = [\phi_X,\phi_Z] \in \Phi(\beta\circ\alpha)$. We must show that there
  exists $\phi_Y\maps Y \to \F$ such that $[\phi_X,\phi_Y] \in \Phi(\alpha)$ and
  $[\phi_Y,\phi_Z] \in \Phi(\beta)$. We claim
  \[
    \phi_Y(y):= \begin{cases}
      \phi(x) & \mbox{if } x \in X, \: x,y \in A_j \mbox{ for some }j;\\
      \phi(z) & \mbox{if } z \in Z, \: y,z \in B_k \mbox{ for some }k;\\
      0 & \mbox{if there exist no such }x \in X \mbox{ or }z \in Z
    \end{cases}
  \]
  satisfies this. This is well-defined as for all $A_j$, $A_j \cap X \subseteq
  C_i$ for some $i$, so $\phi$ is constant on $A_j$, and similarly for all $B_j$.
  Thus it does not matter if there are many such $x$ or $z$ with $x, y \in A_j$ or
  $y,z \in B_k$.  Furthermore, if there exists $y$ such that both $x,y \in A_j$
  for some $x,A_j$ and $y,z \in B_k$ for some $z,B_k$, then the $C_i$ with $A_j
  \cap X \subseteq C_i$ and $B_k \cap Z \subseteq C_i$ is unique, so the
  definitions of $\phi_Y(y)$ do not conflict.

  Moreover, by construction $[\phi_X,\phi_Y]$ is constant on all $A_j$, and
  $[\phi_Y,\phi_Z]$ is constant on all $B_k$, so $[\phi_X,\phi_Y] \in
  \Phi(\alpha)$ and $[\phi_Y,\phi_Z] \in \Phi(\beta)$ as required.
\end{proof}

To recap, we have now constructed a functor $\Phi\maps \mathrm{Corel} \to
\mathrm{LinRel}$ expressing the behavior of potentials on corelations interpreted
as ideal wires. We now do the same for currents.

\subsection{Currents on corelations}

Next we consider the case of currents, described by a functor $I\maps
\mathrm{Corel} \to \mathrm{LinRel}$. This functor now expresses Kirchhoff's
current law: it requires that the sum of currents flowing into each part of the
corelation partition must equal to the sum of currents flowing out.  It may also
be understood as a generalization of the covariant functor $\mathrm{Set} \to
\mathrm{Vect}$ that maps a set to the vector space of $\F$-linear combinations
of elements of that set.

\begin{proposition}
  Define the functor
  \[
    I\maps \mathrm{Corel} \longrightarrow \mathrm{LinRel}.
  \]
  as follows. On objects send a finite set $X$ to the vector space
  $(\F^{X})^\ast$. On morphisms send a corelation $\alpha\maps X \to Y$
  to the linear relation $I(\alpha)$ comprising precisely those 
  \[
    (i_X,i_Y) = \left(\sum_{x \in X} \lambda_xdx,\sum_{y \in Y}
    \lambda_ydy\right)  \in (\F^{X})^\ast \oplus (\F^{Y})^\ast
  \]
  such that for all $A_i \in \alpha$ the sum of the coefficients of the elements
  of $A_i \cap X$ is equal to that for $A_i \cap Y$:
  \[
    \sum_{x \in A_i \cap X} \lambda_x = \sum_{y \in A_i \cap Y} \lambda_y.
  \]
  This is a symmetric monoidal dagger functor.
\end{proposition}
\begin{proof}
The coherence maps are the natural isomorphisms $(\F^X)^\ast \oplus (\F^Y)^\ast
\to (\F^{X+Y})^\ast$ and $\{0\} \to (\F^\varnothing)^\ast$. Again the only
nontrivial task is to check this map $I$ preserves composition. Again we do this
by checking inclusions $I(\beta) \circ I(\alpha) \subseteq I(\beta\circ\alpha)$
and $I(\beta\circ\alpha) \subseteq I(\beta) \circ I(\alpha)$. 

\paragraph{$I(\beta) \circ I(\alpha) \subseteq I(\beta\circ\alpha)$:} Let
$(i_X,i_Z) \in I(\beta)\circ I(\alpha)$, with $i_X = \sum_{x \in X} \lambda_x dx$
and $i_Z = \sum_{z \in Z} \lambda_z dz$. Note that this implies that there is
some $i_Y = \sum_{y \in Y} \lambda_y dy$ such that $(i_X,i_Y) \in I(\alpha)$,
$(i_Y,i_Z) \in I(\beta)$. Then for each $C_i \in \beta\circ\alpha$ we have
\begin{align*}
  \sum_{x \in C_i \cap X} \lambda_x
  &= \sum_{\substack{x \in A_j \cap X \\ A_j \cap X \subseteq C_i}}
  \lambda_x \\
  &= \sum_{\substack{y \in A_j \cap Y \\ A_j \cap X \subseteq C_i}}
  \lambda_y \tag{By definition of $I(\alpha)$}\\
  &= \sum_{\substack{y \in B_k \cap Y \\ B_k \cap Z \subseteq C_i}}
  \lambda_y \tag{See composition of corelations}\\
  &= \sum_{\substack{z \in B_k \cap Z \\ B_k \cap Z \subseteq C_i}}
  \lambda_z \tag{By definition of $I(\beta)$} \\
  &= \sum_{z \in C_i \cap Z} \lambda_z. 
\end{align*}
Thus $(i_X,i_Z) \in I(\beta\circ\alpha)$.

\paragraph{$I(\beta\circ\alpha) \subseteq I(\beta) \circ I(\alpha)$:} The
reverse inclusion requires a bit more effort. Let $(i_X,i_Z) \in
I(\beta\circ\alpha)$. We wish to construct some $i_Y = \sum_{y \in Y} \lambda_y
dy \in (\F^{Y})^\ast$ such that $(i_X,i_Y) \in I(\alpha)$ and $(i_Y,i_Z) \in
I(\beta)$.  This means that we must find a vector $i_Y \in (\F^{Y})^\ast$
satisfying the $\#\alpha$ linear constraints
\[
  \sum_{y \in A_j \cap Y} \lambda_y = \sum_{x \in A_j \cap X} \lambda_x
\]
and the $\#\beta$ linear constraints
\[
  \sum_{y \in B_j \cap Y} \lambda_y = \sum_{z \in B_j \cap Z} \lambda_z.
\]
Note, however, that for each $C_i \in \beta\circ\alpha$, summing over the $A_j$
and $B_k$ that intersect $C_i$ shows a linear dependence between these constraints
themselves, as
\[
  \sum_{\substack{y \in A_j \cap Y \\ A_j \cap X \subseteq C_i}} \lambda_y =
  \sum_{x \in C_i \cap X} \lambda_x = \sum_{z \in C_i \cap Z}
  \lambda_z = \sum_{\substack{y \in B_k \cap Y \\ B_k \cap Z \subseteq C_i}}
  \lambda_y.
\]
Moreover, as each element of $Y$ lies in exactly one element of each $\alpha$
and $\beta$, we may view them as edges in a graph on $\alpha+\beta$, with
exactly $\#(\beta\circ\alpha)$ connected components. This implies that
\[
  \# Y > \#\alpha + \#\beta -\#(\beta\circ\alpha).
\]
Thus these constraints define an affine subspace of $(\F^{Y})^\ast$ of positive
dimension, and hence we can always find a vector $i_Y$ with the desired
property. This proves the claim. 
\end{proof}

Using elementary methods, an algorithm can also
be given to construct an explicit solution.

\subsection{The functor from Corel to LagrRel}

We have now defined functors that, when interpreting corelations as connections
of ideal wires, describe the behaviors of the currents and potentials at the
terminals of these wires. In this section, we combine these to define a single
functor $S\maps \mathrm{Corel} \to \LagrRel$ describing the behavior of both
currents and potentials as a Lagrangian subspace.

\begin{proposition} \label{prop:sympfunctor}
  We define the symplectification functor
  \[
    S\maps \mathrm{Corel} \longrightarrow \LagrRel
  \]
  sending a finite set $X$ to the symplectic vector space
\[
  S(X) = \F^X \oplus (\F^X)^\ast,
\]
 and a corelation $\alpha\maps X \to Y$ to the Lagrangian relation
\[
  S(\alpha) = \Phi(\alpha) \oplus I(\alpha) \subseteq \overline{\F^X \oplus
  (\F^X)^\ast}\oplus \F^Y \oplus (\F^Y)^\ast.
\]
  Then $S$ is a symmetric monoidal dagger functor, with coherence maps inherited
  from $\Phi$ and $I$.
\end{proposition}
\begin{proof}
  As $S$ is the tensor product in $\mathrm{LinRel}$ of the symmetric monoidal
  dagger functors $\Phi, I\maps \mathrm{Corel} \to \mathrm{LinRel}$, it is
  itself a symmetric monoidal dagger functor $\mathrm{Corel} \to
  \mathrm{LinRel}$. Thus it only remains to be checked that, with respect to the
  symplectic structure we put on the objects $S(X)$, the image of each
  corelation $S(\alpha)$ is Lagrangian. 

  This follows from condition (v) of Proposition
  \ref{lagrangian_characterization}: $S(\alpha)$ is (i) isotropic as, for all 
  $(\phi_X,i_X,\phi_Y,i_Y)$, $(\phi_X',i_X',\phi_Y',i_Y') \in S(\alpha)$ we have
  \begin{align*}
    &\phantom{=} \enskip
    \omega\big((\phi_X,i_X,\phi_Y,i_Y),(\phi_X',i_X',\phi_Y',i_Y')\big)
    \\
    &= -\big(i_X'(\phi_X)-i_X(\phi_X')\big) + i_Y'(\phi_Y)-i_Y(\phi_Y') \\
    &= i_X(\phi_X')-i_Y(\phi_Y') + i_X'(\phi_X)-i_Y'(\phi_Y) \\
    &= \sum_{x \in X} \lambda_x dx(\phi_X') - \sum_{y \in Y} \lambda_y dy(\phi_Y') +
    \sum_{y \in Y} \lambda_y' dy(\phi_Y) - \sum_{x \in X} \lambda_x' dx(\phi_X)\\
    &= \sum_{A_j \in \alpha}\left(\sum_{x \in A_j \cap X} \lambda_x dx(\phi_X') -
    \sum_{y \in A_j \cap Y} \lambda_y dy(\phi_Y')\right)+ \sum_{A_j \in
    \alpha}\left(\sum_{x \in A_j \cap X} \lambda_x'dx(\phi_X) - \sum_{y \in A_j
    \cap Y} \lambda_y' dy(\phi_Y)\right)\\
    &= \sum_{A_j \in \alpha}\left(\sum_{x \in A_j \cap X} \lambda_x - \sum_{y \in
    A_j \cap Y} \lambda_y \right)k'_{A_j} + \sum_{A_j \in \alpha}\left(\sum_{x \in
    A_j \cap X} \lambda_x'- \sum_{y \in A_j \cap Y} \lambda_y'\right)k_{A_j}\\
    &= 0,
  \end{align*}
  and (ii) has dimension equal to 
  \[
    \dim(\Phi(\alpha))+ \dim(I(\alpha) = \#\alpha+\#(X+Y) - \#\alpha= \#(X+Y).
  \]
  This proves the proposition.
\end{proof}

We have thus shown we do indeed have a functor $S\maps \mathrm{Corel} \to
\LagrRel$. In the next section we shall see that this functor provides
the engine of our black box functor, playing the key role in showing that we
may indeed treat circuit components as black boxes: that is, that circuits that
behave the same compose the same. Before we get there, we quickly make two
relevant observations.

\begin{example}[Symplectification of functions] \label{ex:sympfunction}
  Let $f: X \to Y$ be a function. In this example we show that $Sf$ has the form 
  \[
    Sf = \big\{(\phi_X,i_X,\phi_Y,i_Y) \, \big\vert \, \phi_X = f^\ast\phi_Y,
    i_Y = f_\ast i_X \big\} \subseteq \overline{\F^X \oplus (\F^X)^\ast} \oplus
    \F^Y \oplus (\F^Y)^\ast,
  \]
  where $f^\ast$ is the pullback map
  \begin{align*}
    f^\ast\maps \F^Y &\longrightarrow \F^X; \\
    \phi &\longmapsto \phi \circ f,
  \end{align*}
  and $f_\ast$ is the pushforward map
  \begin{align*}
    f_\ast\maps (\F^X)^\ast &\longrightarrow (\F^Y)^\ast; \\
    i(-) &\longmapsto i(-\circ f).
  \end{align*}
  (We shall also write $f_\ast$ for the more general map from functions on
  $\F^X$ to functions on $\F^Y$ that takes a function $i(-)$ on $\F^X$ to the
  function $i(-\circ f)$.) The claim is then that these pullback and pushforward
  constructions express Kirchhoff's laws.

  Recall that the corelation corresponding to $f$ partitions $X+Y$ into $\#Y$
  parts, each of the form $f^{-1}(y) \cup \{y\}$. The linear relation $\Phi(f)$
  requires that if $x \in X$ and $y \in Y$ lie in the same part of the
  partition, then they have the same potential: that is, $\phi_X(x) =
  \phi_Y(y)$. This is precisely the arrangement imposed by $f^\ast \phi_Y =
  \phi_X$: 
  \[
    \phi_X(x) = \phi_Y(f(x)) =\phi_Y(y).
  \] 
  On the other hand, the linear relation $I(f)$ requires that if $i_X = \sum_{x 
  \in X}\lambda_xdx$, $i_Y = \sum_{y \in Y}\lambda_y dy$, then for each $y \in Y$
  we have 
  \[
    \sum_{x \in f^{-1}(y)} \lambda_x = \lambda_y.
  \]
  This is precisely what is required by $f_\ast$: given any $\phi \in \F^Y$, we
  have
  \[
    f_\ast i_X(\phi) = f_\ast \sum_{x \in X}\lambda_xdx(\phi) = \sum_{x
    \in X}\lambda_xdx(\phi \circ f)= \sum_{y \in Y}\left( \sum_{x \in f^{-1}(y)}
    \lambda_x\right)dy.
  \]
  This gives us the above representation of $Sf$ when $f$ is a function.
\end{example}

%High road is probably better:
%Recall that given a Lagrangian subspace $L \subseteq V$, its image $f(L)$ 
%under a Lagrangian relation $f\maps V \nrightarrow W$ is a Lagrangian subspace of $W$.  

\begin{remark} \label{rmk:dualitydiagrams}
  We make a remark on our conventions for string diagrams representing the cups
  and caps of the dualities in $\mathrm{Corel}$ and $\LagrRel$. 
  
  Note that by the cap duality diagram
  \[
  \begin{tikzpicture}
	\begin{pgfonlayer}{nodelayer}
		\node [style=none] (0) at (-0.5, 0.5) {};
		\node [style=none] (1) at (0.5, 0.5) {};
		\node [style=none] (2) at (0, -0) {};
		\node [style=none] (3) at (-0.5, 0.75) {$X$};
		\node [style=none] (4) at (0.5, 0.75) {$X$};
	\end{pgfonlayer}
	\begin{pgfonlayer}{edgelayer}
		\draw [in=180, out=-90, looseness=1.00] (0.center) to (2.center);
		\draw [in=-90, out=0, looseness=1.00] (2.center) to (1.center);
	\end{pgfonlayer}
\end{tikzpicture}  
  \]
  in $\mathrm{Corel}$ we mean the corelation $\cup_X\maps (X+X \stackrel{[1,1]}{\rightarrow} X \stackrel{!}{\leftarrow} \varnothing)$, whereas by the cap
  \[
    \begin{tikzpicture}
	\begin{pgfonlayer}{nodelayer}
		\node [style=none] (0) at (-0.5, 0.5) {};
		\node [style=none] (1) at (0.5, 0.5) {};
		\node [style=none] (2) at (0, -0) {};
		\node [style=none] (3) at (-0.5, 0.75) {$V$};
		\node [style=none] (4) at (0.5, 0.75) {$\overline{V}$};
	\end{pgfonlayer}
	\begin{pgfonlayer}{edgelayer}
		\draw [in=180, out=-90, looseness=1.00] (0.center) to (2.center);
		\draw [in=-90, out=0, looseness=1.00] (2.center) to (1.center);
	\end{pgfonlayer}
\end{tikzpicture}
  \]
  in $\LagrRel$ we mean the Lagrangian relation $\cup_V\maps V\oplus \overline{V}
  \to 0$ given by the Lagrangian subspace $\{(v,v) \in \overline{V \oplus
  \overline{V}} \mid v \in V\}$. Although we represent them similarly, the
  functor $S$ does not map these directly onto each other: the image of $\cup_X$
  under $S$ is the Lagrangian relation $S(\cup_X)\maps \F^X \oplus (\F^X)^\ast \oplus
  \F^X \oplus (\F^X)^\ast \to 0$ given by the Lagrangian subspace
  \[
    \{(\phi,i,\phi, -i)\mid \phi \in \F^X, i \in (\F^X)^\ast\} \subseteq
    \overline{\F^X \oplus (\F^X)^\ast \oplus \F^X \oplus (\F^X)^\ast},
  \]
  and in particular a Lagrangian relation $\F^X \oplus (\F^X)^\ast \oplus \F^X
  \oplus (\F^X)^\ast \to 0$ and not $\F^X \oplus (\F^X)^\ast \oplus
  \overline{\F^X \oplus (\F^X)^\ast} \to 0$. As the cup diagrams in these
  categories simply denote the dagger image of these caps, the analogous
  statements apply to them too. 
  
  This is an expression of the fact that in cospan categories and
  $\mathrm{Corel}$ there is a canonical self-duality, while for Lagrangian
  relations a basis must be picked before there is an isomorphism between a
  symplectic vector space and its conjugate, the conjugate being a canonical
  dual object. Nonetheless as the functor $S$ uses the set $X$ to generate a
  symplectic vector space, each object in the image of $S$ has a canonical
  symplectomorphism with its dual $S^t\!X\maps \F^X \oplus (F^X)^\ast \to
  \overline{\F^X \oplus (F^X)^\ast}$, with name the Lagrangian subspace
  $\{(\phi,i,\phi, -i)\mid \phi \in \F^X, i \in (\F^X)^\ast\}$.\footnote{While we could
  treat $S^t\!X$ as an atomic notation, we write it here to evoke the idea of
the concept of `twisted' symplectification introduced in the introduction.} The image of the
  canonical self-duality $\mathrm{Corel}$ is thus given by the composite of this
  symplectomorphism with canonical duality in $\LagrRel$; that is
  \[
    S
    \begin{aligned}
      \begin{tikzpicture}
	\begin{pgfonlayer}{nodelayer}
		\node [style=none] (0) at (-0.5, 0.5) {};
		\node [style=none] (1) at (0.5, 0.5) {};
		\node [style=none] (2) at (0, -0) {};
		\node [style=none] (3) at (-0.5, 1.25) {};
		\node [style=none] (4) at (0.5, 1.25) {};
		\node [style=none] (5) at (-0.5, 1.5) {$X$};
		\node [style=none] (6) at (0.5, 1.5) {$X$};
	  \node[outer sep=0pt, left delimiter=(, right delimiter=),
	  align=center, fit = (0) (1) (2) (3) (4) (5) (6)] {};
	\end{pgfonlayer}
	\begin{pgfonlayer}{edgelayer}
		\draw [in=180, out=-90, looseness=1.00] (0.center) to (2.center);
		\draw [in=-90, out=0, looseness=1.00] (2.center) to (1.center);
		\draw (0.center) to (3.center);
		\draw (1.center) to (4.center);
	\end{pgfonlayer}
\end{tikzpicture}  
\end{aligned}
\qquad 
=
\qquad
\begin{aligned}
\begin{tikzpicture}
	\begin{pgfonlayer}{nodelayer}
		\node [style=none] (0) at (-0.5, 0.5) {};
		\node [style=none] (1) at (0.5, 0.5) {};
		\node [style=none] (2) at (0, 0) {};
		\node [style=none] (3) at (-0.5, 1.25) {};
		\node [style=none] (4) at (0.5, 1.25) {};
		\node [style=none] (5) at (0.25, 1) {};
		\node [style=none] (6) at (0.75, 1) {};
		\node [style=none] (7) at (0.25, 0.5) {};
		\node [style=none] (8) at (0.75, 0.5) {};
		\node [style=none] (9) at (0.5, 1) {};
		\node [style=none] (10) at (0.5, 0.75) {$S^t\!X$};
		\node [style=none] (11) at (-0.5, 1.5) {$S(X)$};
		\node [style=none] (12) at (0.5, 1.5) {$S(X)$};
	\end{pgfonlayer}
	\begin{pgfonlayer}{edgelayer}
		\draw [in=180, out=-90, looseness=1.00] (0.center) to (2.center);
		\draw [in=-90, out=0, looseness=1.00] (2.center) to (1.center);
		\draw (0.center) to (3.center);
		\draw (4.center) to (9.center);
		\draw (5.center) to (6.center);
		\draw (6.center) to (8.center);
		\draw (8.center) to (7.center);
		\draw (7.center) to (5.center);
	\end{pgfonlayer}
\end{tikzpicture}
\end{aligned}
  \]
\end{remark}

\section{Defining the black box functor} \label{sec:blackbox}
%%fakesection
We have now developed enough machinery to prove Theorem \ref{main_theorem}:
there is a symmetric monoidal dagger functor, the black box functor
\[  
\blacksquare\maps \Circ \to \LagrRel 
\]
taking passive linear circuits to their behaviors. To recap, we have so far
developed two categories: $\Circ$, in which morphisms are passive linear
circuits, and $\LagrRel$, which captures the external behavior of such circuits.
We now define a functor that maps each circuit to its behavior, before proving
it is indeed a symmetric monoidal dagger functor. 

The role of the functor we construct here is to identify all circuits with the
same external behavior, making the internal structure of the circuit
inaccessible. Circuits treated this way are frequently referred to as
`black boxes', so we call this functor the \define{black box functor},
\[
\blacksquare\maps \Circ \to \LagrRel.
\] 
In this section we first provide the definition of this functor, and then check
that our definition really does map a circuit to its behavior.

\subsection{Definition}
It should be no surprise that the black box functor maps a finite set $X$ to the
symplectic vector space $\F^X \oplus (\F^X)^\ast$ of potentials and currents
that may be placed on that set. The challenge is to provide a succinct
statement of its action on the circuits themselves. To do this, we take
advantage of four processes we developed in Parts  and
.

Let $\Gamma\maps X \to Y$ be a circuit, represented by the decorated cospan
\[
  \big(X \stackrel{i}{\longrightarrow} N \stackrel{o}{\longleftarrow} Y,\:
  \Gamma\big).
\]
Recall that this means that $X$ and $Y$ are finite sets, $\Gamma$ is a
$\F$-graph $(N,E,s,t,r)$, and we have a cospan of finite sets
\[
  \xymatrix{
    & \Gamma \\
    X \ar[ur]^{i} && Y. \ar[ul]_{o}
  }
\]
To define the image of $\Gamma$ under our functor $\blacksquare$, by definition
a Lagrangian relation $\blacksquare(\Gamma): \blacksquare(X) \to
\blacksquare(Y)$, we must specify a Lagrangian subspace 
\[
  \blacksquare(\Gamma) \subseteq \overline{\F^X \oplus (\F^X)^\ast} \oplus \F^Y
  \oplus (\F^Y)^\ast.  
\]

Recall that to each $\F^+$-graph $\Gamma$ we can associate a Dirichlet form, 
the extended power functional 
\begin{align*}
  P_\Gamma\maps \F^N &\longrightarrow \F; \\
  \phi &\longmapsto \frac{1}{2} \sum_{e \in E} \frac{1}{r(e)} \big( \phi(t(e)) -
  \phi(s(e))  \big)^2,
\end{align*}
and to this Dirichlet form we associate a Lagrangian subspace 
\[
  \mathrm{Graph}(dP_\Gamma) = \{(\phi,d(P_\Gamma)_\phi) \mid \phi \in \F^N\}
  \subseteq \F^N \oplus (\F^N)^\ast.
\]
We consider this Lagrangian subspace as a Lagrangian relation $\{0\} \to \F^N
\oplus (\F^N)^\ast$.

From the legs of the cospan $\Gamma$, the symplectification functor $S$ gives the
Lagrangian relation
\[
  S[i,o]^\dagger\maps \F^N \oplus (\F^N)^\ast \longrightarrow \F^X \oplus
  (\F^X)^\ast \oplus \F^Y \oplus (\F^Y)^\ast.
\]
Writing $Y: Y \to Y$ for the identity morphism on the finite set $Y$, $S$ also
provides a way of writing the identity morphism $SY: \F^Y \oplus (\F^Y)^\ast \to
\F^Y \oplus (\F^Y)^\ast$. 

Lastly, we have the symplectomorphism
\begin{align*}
  S^t\!X\maps \F^X \oplus (\F^X)^\ast &\longrightarrow \overline{\F^X \oplus
  (\F^X)^\ast}; \\
  (\phi,i) &\longmapsto (\phi,-i).
\end{align*}

The black box functor maps a circuit $\Gamma$ to the Lagrangian relation
\[
  (S^t\!X\oplus SY) \circ S[i,o]^\dagger \circ \mathrm{Graph}(P_\Gamma).
\]

As isomorphisms of cospans of $\F^+$-graphs amount to no more than a 
relabelling of nodes and edges, this construction is independent of the cospan 
chosen as representative of the isomorphism class of cospans forming the 
circuit.

We picture this as
\[
  \blacksquare(\Gamma) =\qquad 
  \begin{aligned}
\begin{tikzpicture}
	\begin{pgfonlayer}{nodelayer}
		\node [style=none] (0) at (-1.25, 0.5) {};
		\node [style=none] (1) at (-0.75, 0.5) {};
		\node [style=none] (2) at (-1.25, -0) {};
		\node [style=none] (3) at (-0.75, -0) {};
		\node [style=none] (4) at (-1.25, 1.25) {};
		\node [style=none] (5) at (-1, -0.25) {};
		\node [style=none] (6) at (-1, 0.5) {};
		\node [style=none] (7) at (-1, -0) {};
		\node [style=none] (8) at (0, -0.25) {};
		\node [style=none] (9) at (0, 0.75) {};
		\node [style=none] (10) at (-0.5, 2) {};
		\node [style=none] (11) at (-1, 1.5) {};
		\node [style=none] (12) at (0, 1.5) {};
		\node [style=none] (13) at (-0.5, 1.5) {};
		\node [style=none] (14) at (-1.25, 0.75) {};
		\node [style=none] (15) at (0.25, 1.25) {};
		\node [style=none] (16) at (0.25, 0.75) {};
		\node [style=none] (17) at (-1, 0.75) {};
		\node [style=none] (18) at (-0.5, 1.25) {};
		\node [style=none] (19) at (-0.5, 1.75) {$L_\Gamma$};
		\node [style=none] (20) at (-0.5, 1) {$S[i,o]^\dagger$};
		\node [style=none] (21) at (-1, 0.25) {$S^t\! X$};
		\node [style=none] (22) at (-1.5, -0.75) {};
		\node [style=none] (23) at (-2, -0.25) {};
		\node [style=none] (24) at (-2, 1.5) {};
		\node [style=none] (25) at (-1, 2.5) {};
		\node [style=none] (26) at (-1, -1.25) {};
		\node [style=none] (27) at (-1, 2.75) {$\scriptstyle\blacksquare(X)$};
		\node [style=none] (28) at (-1, -1.5) {$\scriptstyle\blacksquare(Y)$};
	\end{pgfonlayer}
	\begin{pgfonlayer}{edgelayer}
		\draw (0.center) to (1.center);
		\draw (1.center) to (3.center);
		\draw (3.center) to (2.center);
		\draw (2.center) to (0.center);
		\draw (7.center) to (5.center);
		\draw (8.center) to (9.center);
		\draw (10.center) to (11.center);
		\draw (11.center) to (12.center);
		\draw (12.center) to (10.center);
		\draw (4.center) to (14.center);
		\draw (14.center) to (16.center);
		\draw (16.center) to (15.center);
		\draw (15.center) to (4.center);
		\draw (13.center) to (18.center);
		\draw (17.center) to (6.center);
		\draw [in=90, out=-90, looseness=1.00] (8.center) to (26.center);
		\draw [bend left=45, looseness=1.00] (5.center) to (22.center);
		\draw [bend left=45, looseness=1.00] (22.center) to (23.center);
		\draw (23.center) to (24.center);
		\draw [in=-90, out=90, looseness=1.00] (24.center) to (25.center);
	\end{pgfonlayer}
\end{tikzpicture}
\end{aligned}
\]

To summarize:

\begin{definition}
  We define the black box functor 
  \[
    \blacksquare\maps \Circ \to \LagrRel 
  \]
  on objects by mapping a finite set $X$ to the symplectic vector space 
  \[
    \blacksquare(X) = \F^X \oplus (\F^X)^\ast.
  \]
  and on morphisms by mapping a circuit $\Gamma\maps X \to Y$, represented by the
  decorated cospan
  \[
    \big(X \stackrel{i}{\longrightarrow} N \stackrel{o}{\longleftarrow} Y,\:
    \Gamma\big)
  \]
  to the Lagrangian relation
  \[
    \blacksquare(\Gamma) = (S^t\!X \oplus SY) \circ S[i,o]^\dagger \circ
    \mathrm{Graph}(dP_\Gamma).
  \]
\end{definition}

The coherence maps are given by the natural isomorphisms
\[
  \blacksquare(X) \oplus \blacksquare(Y) = \F^X \oplus (\F^X)^\ast \oplus \F^Y
  \oplus (\F^Y)^\ast \cong \F^{X+Y} \oplus (\F^{X+Y})^\ast = \blacksquare(X+Y)
\]
and
  \[
    \{0\} \cong \F^\varnothing \oplus (\F^\varnothing)^\ast =
    \blacksquare(\varnothing).
  \]


\begin{theorem} \label{thm:main}
  The black box functor is a well-defined symmetric monoidal dagger functor.
\end{theorem}

The next, and final, section is devoted to the proof of this theorem. Before we
get there, we first assure ourselves that we have indeed arrived at the theorem
we set out to prove.

\subsection{Minimization via composition of relations}

At this point the reader might voice two concerns: firstly, why does the
\emph{black box} functor refer to the \emph{extended} power functional $P$ and,
secondly, since it fails to talk about power minimization, how is it the same
functor as that defined in Theorem \ref{main_theorem}? These fears are allayed
by the remarkable trinity of minimization, the symplectification of functions,
and Kirchhoff's laws. 

We have seen that symplectification of functions views the cograph of the
function as a picture of ideal wires, governed by Kirchhoff's laws (Example
\ref{ex:sympfunction}). We have also seen that Kirchhoff's laws are closely
related to the principle of minimium power (Theorems
\ref{thm:realizablepotentials} and \ref{thm:dirichletminimization}). The final
aspect of this relationship is that we may use symplectification of functions to
enact minimization.

\begin{theorem} \label{thm:sympmin}
  Let $\iota: \partial N \to N$ be an injection, and let $P$ be a Dirichlet form on
  $N$. Write $Q = \min_{N \setminus \partial N} P$ for the Dirichlet form on
  $\partial N$ given by minimization over $N \setminus \partial N$. Then we have an
  equality of Lagrangian subspaces
  \[
    S\iota^\dagger \big( \mathrm{Graph}(dP)\big) = \mathrm{Graph}(dQ).
  \]
\end{theorem}
\begin{proof}
  Recall from Example \ref{ex:sympfunction} that $S\iota^\dagger$ is the Lagrangian relation
  \[
    S\iota^\dagger = \big\{(\phi, \iota_\ast i,\phi \circ \iota, i) \, \big\vert
      \, \phi \in \F^{N}, i \in (\F^{\partial N})^\ast \big\} \subseteq
      \overline{\F^N \oplus (\F^N)^\ast} \oplus \F^{\partial N} \oplus
      (\F^{\partial N})^\ast,
  \]
  where $\iota_\ast i(\phi) = i(\phi \circ \iota)$, and note that 
  \[
    \mathrm{Graph}(dP) = \big\{(\phi,dP_\phi) \,\big\vert\, \phi \in \F^N\big\}.
  \]
  This implies that their composite is given by the set
  \[
    S\iota^\dagger \circ \mathrm{Graph}(dP) = \big\{(\phi \circ \iota, i)
    \,\big\vert\, \phi \in \F^N, i \in (\F^{\partial N})^\ast, dP_\phi =
  \iota_\ast i \big\}.
  \]
  We must show this Lagrangian subspace is equal to $\mathrm{Graph}(dQ)$.
  
  Consider the constraint $dP_\phi = \iota_\ast i$. This states that for all
  $\varphi \in \F^N$ we have $dP_\phi(\varphi) = i(\varphi\circ \iota)$. Letting
  $\chi_n: N \to \F$ be the function sending $n \in N$ to $1$ and all other
  elements of $N$ to $0$, we see that when $n \in N \setminus \partial N$ we
  must have
  \[
    \frac{dP}{d\varphi(n)}\Bigg\vert_{\varphi = \phi}  = dP_\phi(\chi_n) = 
    i(\chi_n \circ \iota) = i(0) = 0.
  \]
  So $\phi$ must be a realizable extension of $\psi = \phi \circ \iota$. We
  henceforth write $\tilde\psi = \phi$. As $\iota$ is injective, $\psi = \phi
  \circ \iota$ gives no constraint on $\psi \in \F^{\partial N}$. 
 
  We next observe that we can write $S\iota^\dagger \circ \mathrm{Graph}(dP) =
  \mathrm{Graph}(dO)$ for some quadratic form $O$. Recall that Proposition
  \ref{prop:qfls} states that a Lagrangian subspace $L$ of $\F^{\partial N}
  \oplus (\F^{\partial N})^\ast$ is of the form $\mathrm{Graph}(dO)$ if and only
  if $L$ has trivial intersection with $\{0\} \oplus (\F^N)^\ast$. But indeed,
  if $\psi = 0$ then $0$ is a realizable extension of $\psi$, so $\iota_\ast i =
  dP_0 = 0$, and hence $i = 0$. 
  
  It remains to check that $O = Q$. This is a simple computation:
  \[
    O(\psi) = dO_\psi(\psi) = dO_\psi(\tilde\psi \circ \iota) = \iota_\ast
    dQ_\psi(\tilde\psi) = dP_{\tilde\psi}(\tilde\psi) = P(\tilde\psi) = Q(\psi),
  \]
  where $\tilde\psi$ is any realizable extension of $\psi \in \F^{\partial N}$.
\end{proof}

Write $\iota: \partial N \to N$ for the inclusion of the terminals into the set
of nodes of the circuit, and $i\rvert^{\partial N}: X \to \partial N$,
$o\rvert^{\partial N}: Y \to \partial N$ for the respective corestrictions of
the input and output map to $\partial N$. Note that $[i,o] = \iota \circ
[i\rvert^{\partial N}, o\rvert^{\partial N}]$.
Then we have the equalities of sets, and thus Lagrangian relations:
\begin{align*}
  (S^t\!X\oplus 1_Y) \circ S[i,o]^\dagger \circ \mathrm{Graph}(dP_\Gamma)
  &= (S^t\!X\oplus SY) \circ S[i\rvert^{\partial
  N},o\rvert^{\partial N}]^\dagger \circ S\iota^\dagger \circ \mathrm{Graph}(dP_\Gamma) \\
  &= (S^t\!X\oplus SY) \circ S[i\rvert^{\partial
  N},o\rvert^{\partial N}]^\dagger \circ \mathrm{Graph}(dQ_\Gamma) \\
  &= \bigcup_{v \in \mathrm{Graph}(dQ)} S^ti\rvert^{\partial N}(v) \times
  So\rvert^{\partial N}(v)
\end{align*}
We see now that Theorem \ref{thm:main} is a restatement of Theorem
\ref{main_theorem} in the introduction.

\section{Proof of functoriality} \label{sec:proof}
%%fakesubsection
To prove that the black box construction is indeed functorial, we
factor it into three functors. These functors are each symmetric monoidal dagger
functors, so the black box functor is too.

We first make use, twice, of our results on decorated cospans, showing the
existence of categories of cospans decorated by Dirichlet forms and then
Lagrangian subspaces, and the existence of functors from the category of
circuits to each of these. This proceeds by defining functors $\mathrm{Dirich}$
and $\mathrm{Lagr}$ to construction decorated cospan categories, and then by
defining the relevant natural transformations to construct the desired decorated
cospan functors.

The third functor takes cospans decorated by Lagrangian subspaces, and black
boxes them to give Lagrangian relations between the feet of such cospans. The
functoriality of this process relies on interpreting corelations as Lagrangian
relations.

This gives a factorization
\[
  \xymatrix{
    \blacksquare\maps \Circ \ar[r] & \mathrm{DirichCospan} \ar[r] &
    \mathrm{LagrCospan} \ar[r] & \LagrRel
  }
\]
The following subsections deal with these functors in sequence, defining and
proving the existence of each one using the techniques of Part
.


\subsection{From circuits to Dirichlet cospans}
To begin, we define a category of cospans of finite sets decorated by Dirichlet
forms. This might be seen as a solution to our sought after composition rule for
Dirichlet forms---but not quite, as it also requires the additional data of a
cospan. Nonetheless, it has the property that there is a functor from the
category of circuits to this so-called category of Dirichlet cospans.

\subsubsection*{The category of Dirichlet cospans} 

\begin{proposition}
Let
\[
  \mathrm{Dirich}\maps (\mathrm{FinSet},+) \longrightarrow (\mathrm{Set},\times)
\]
map a finite set $X$ to the set $\mathrm{Dirich}(X)$ of Dirichlet forms on $X$,
and map a function $f\maps X \to Y$ between finite sets to the pushforward function
\begin{align*}
  \mathrm{Dirich}(f)\maps \mathrm{Dirich}(X) &\longrightarrow \mathrm{Dirich}(Y); \\
  Q &\longmapsto f_{\ast}Q,
\end{align*}
where $f_\ast Q$ maps $\phi \in \F^Y$ to $Q(\phi \circ f))$. This defines a
functor.

Moreover, equipping this functor with the family of maps
\begin{align*}
  \delta_{N,M}\maps \mathrm{Dirich}(N) \times \mathrm{Dirich}(M) &\longrightarrow
  \mathrm{Dirich}(N+M);\\
  (Q_N,Q_M) &\longmapsto Q_N(-\circ \iota_N) +Q_M(-\circ \iota_M),
\end{align*}
where $\iota_N\maps N \to N+M$, $\iota_M\maps M \to N+M$ are the injections for
the coproduct, and with unit
\begin{align*}
  \delta_1\maps 1 &\longrightarrow \mathrm{Dirich}(\varnothing);\\
  \bullet &\longmapsto (\F^\varnothing \to \F; \varnothing \mapsto 0)
\end{align*}
defines a lax symmetric monoidal functor.
\end{proposition}

\begin{proof}
  Functoriality is just the associativity of composition of functions;
  lax symmetric monoidality is the associativity, unitality, and commutativity
  of addition of Dirichlet forms.
\end{proof}
  
By decorated cospans, we thus obtain a category $\mathrm{DirichCospan}$
where a morphism is a cospan of finite sets whose apex is equipped with a 
Dirichlet form.  In particular, note that we have overcome our inability to define a category whose morphisms are Dirichlet forms.   An identity morphism in $\mathrm{DirichCospan}$ is a circuit whose set of inputs is equal to its set of
outputs, with all of its nodes being terminals, and no edges:
\[
  \begin{tikzpicture}[circuit ee IEC, set resistor graphic=var resistor iec graphic]
    \node[contact] (c1) at (0,2) {};
    \node[contact] (c2) at (0,0) {};
    \node(input) at (-2,1) {\small{\textsf{inputs}}};
    \node(output) at (2,1) {\small{\textsf{outputs}}};
    \path[color=gray, very thick, shorten >=10pt, ->, >=stealth, bend left]
    (input) edge (c1);		
    \path[color=gray, very thick, shorten >=10pt, ->, >=stealth, bend right]
    (input) edge (c2);	
    \path[color=gray, very thick, shorten >=10pt, ->, >=stealth, bend right]
    (output) edge (c1);
    \path[color=gray, very thick, shorten >=10pt, ->, >=stealth, bend left]
    (output) edge (c2);
  \end{tikzpicture}
\]

\subsubsection*{The functor $\Circ \to \mathrm{DirichCospan}$}

We have now constructed two symmetric monoidal functors
\[
  (\mathrm{Circuit},\rho), (\mathrm{Dirich},\delta)\maps (\mathrm{FinSet},+) \longrightarrow (\mathrm{Set},\times)
\]
that describe the circuit structures and Dirichlet forms we may put on the set
respectively. We have also seen, motivating our discussion of Dirichlet forms,
that from any circuit we can obtain a Dirichlet form describing the power usage
of that circuit. Now we shall see that this process respects the tensor product.
More precisely, it specifies a monoidal natural transformation between
$\mathrm{Circuit}$ and $\mathrm{Dirich}$. By the decorated cospan construction,
this gives us a strict symmetric monoidal dagger
functor $\Circ \to \mathrm{DirichCospan}$.

\begin{proposition}
  Let
  \[
    \alpha\maps (\mathrm{Circuit},\rho) \Longrightarrow (\mathrm{Dirich},\delta)
  \]
  be the collection of functions
  \begin{align*}
    \alpha_N\maps \mathrm{Circuit}(N) &\longrightarrow \mathrm{Dirich}(N); \\
    (N,E,s,t,r) &\longmapsto \left(\phi \in \F^N \mapsto \frac{1}{2} \sum_{e \in E}
    \frac{1}{r(e)}\big(\phi(s(e))-\phi(t(e))\big)^2\right).
  \end{align*}
  Then $\alpha$ is a monoidal natural transformation.
\end{proposition}

\begin{proof}
Naturality requires that the square
\[
\xymatrix{
  \mathrm{Circuit}(N) \ar[r]^{\alpha_N} \ar[d]_{\mathrm{Circuit}(f)} &
  \mathrm{Dirich}(N) \ar[d]^{\mathrm{Dirich}(f)}  \\
  \mathrm{Circuit}(M) \ar[r]_{\alpha_M} & \mathrm{Dirich}(M)
}
\]
commutes. Let $(N,E,s,t,r)$ be an $\F^+$-graph on $N$ and $f\maps N \to M$ be a
function $N$ to $M$. Then both $\mathrm{Dirich}(f) \circ \alpha_N$ and $\alpha_M
\circ \mathrm{Circuit}(f)$ map $(N,E,s,t,r)$ to the Dirichlet form
\begin{align*}
  \F^M &\longrightarrow \F;\\
  \psi &\longmapsto \frac{1}{2} \sum_{e \in E}\frac{1}{r(e)}
  \big(\psi(f(s(e)))-\psi(f(t(e)))\big)^2.
\end{align*}
Thus both methods of constructing a power functional on a set of nodes $M$ from
a circuit on $N$ and a function $N \to M$ produce the same power functional.

To show that $\alpha$ is a monoidal natural transformation, we must check that
the square
\[
\xymatrix{
  \mathrm{Circuit}(N) \times \mathrm{Circuit}(M) \ar[r]^{\alpha_N \times
  \alpha_M} \ar[d]_{\rho_{N,M}} & \mathrm{Dirich}(N) \times \mathrm{Dirich}(M)
  \ar[d]^{\delta_{N,M}}  \\
  \mathrm{Circuit}(N+M) \ar[r]_{\alpha_{N+M}} & \mathrm{Dirich}(N+M)
}
\]
and the triangle
\[
\xymatrix{
  & 1 \ar[dl]_{\rho_\varnothing} \ar[dr]^{\delta_\varnothing}\\
\mathrm{Circuit}(\varnothing)  \ar[rr]_{\alpha_\varnothing} &&
\mathrm{Dirich}(\varnothing)
}
\]
commute. It is readily observed that both paths around the square lead to taking
two graphs and summing their corresponding Dirichlet forms, and that the
triangle commutes immediately as all objects in it are the one element set.
\end{proof}

From decorated cospans, we thus obtain a strict symmetric monoidal dagger functor
\[
  Q \maps \Circ = \mathrm{CircuitCospan} \longrightarrow \mathrm{DirichCospan}.
\]
Roughly, this says that the process of composition for circuit diagrams is the
same as that of composition for Dirichlet cospans. Note that although this
functor preserves much of the information in circuit diagrams, it is not a
faithful functor.  For example, applying $Q$ to a circuit with edges $e,e'$ from
the node $m$ to the node $n$ of resistance $r_e$ and $r_{e'}$ respectively, we
obtain the same result as for the circuit with just one edge $e''$ from $m$ to
$n$ whose resistance $r_{e''}$ is given by
\[
           \frac{1}{r_{e''}} = \frac{1}{r_e} + \frac{1}{r_{e'}}  .
\]

\subsection{From Dirichlet cospans to Lagrangian cospans}

The next step is to show that our process of turning a Dirichlet form into a Lagrangian subspace---by taking the graph of its differential---is functorial.

\subsubsection*{The category of Lagrangian cospans}

We begin by describing a category where morphisms are cospans decorated by
Lagrangian subspaces.

\begin{proposition}
Define 
\[
  \mathrm{Lagr}\maps (\mathrm{FinSet},+) \longrightarrow (\mathrm{Set},\times)
\]
as follows. For objects let $\mathrm{Lagr}$ map a finite set $X$ to the set
$\mathrm{Lagr}(X)$ of Lagrangian subspaces of the symplectic vector space
$\vectf{X}$.  For morphisms, recall that a function $f\maps X \to Y$ between
finite sets may be considered as a corelation, and the symplectification functor
$S$ thus maps this corelation to some Lagrangian relation $Sf\maps \vectf{X}
\to \vectf{Y}$. As Lagrangian relations map Lagrangian subspaces to Lagrangian
subspaces (Proposition \ref{prop:lagrangian_composition}), this gives a map: 
\begin{align*}
  \mathrm{Lagr}(f)\maps \mathrm{Lagr}(X) &\longrightarrow \mathrm{Lagr}(Y); \\
  L &\longmapsto Sf(L).
\end{align*}
Moreover, equipping this functor with the family of maps
\begin{align*}
  \lambda_{N,M}\maps \mathrm{Lagr}(N) \times \mathrm{Lagr}(M) &\longrightarrow
  \mathrm{Lagr}(N+M);\\
  (L_N,L_M) &\longmapsto L_N \oplus L_M,
\end{align*}
and unit
\begin{align*}
  \lambda_1\maps 1 &\longrightarrow \mathrm{Lagr}(\varnothing);\\
  \bullet &\longmapsto \{0\}
\end{align*}
defines a lax symmetric monoidal functor.
\end{proposition}
\begin{proof}
  The functoriality of this construction follows from the functoriality of $S$;
  the lax symmetric monoidality from the relevant properties of the direct sum
  of vector spaces.
\end{proof}
We thus obtain a dagger compact category $\mathrm{LagrCospan}$.

\subsubsection*{The functor $\mathrm{DirichCospan} \to \mathrm{LagrCospan}$}

We now wish to construct a strict symmetric monoidal dagger functor
$\mathrm{DirichCospan} \to \mathrm{LagrCospan}$.  For this we need a monoidal natural transformation 
\[
  (\mathrm{Dirich},\delta), (\mathrm{Lagr},\lambda)\maps (\mathrm{FinSet},+)
  \longrightarrow (\mathrm{Set},\times).
\]
\begin{proposition}
Let
\[
  \beta\maps (\mathrm{Dirich},\delta) \Longrightarrow (\mathrm{Lagr},\lambda)
\]
be the collection of functions
\begin{align*}
  \beta_N\maps \mathrm{Dirich}(N) &\longrightarrow \mathrm{Lagr}(N); \\
  Q &\longmapsto \{(\phi,dQ_\phi) \mid \phi \in \F^N\} \subseteq \vectf{N}.
\end{align*}
Then $\beta$ is a monoidal natural transformation.
\end{proposition}

\begin{proof}
Naturality requires that the square
\[
\xymatrix{
  \mathrm{Dirich}(N) \ar[r]^{\beta_N} \ar[d]_{\mathrm{Dirich}(f)} &
  \mathrm{Lagr}(N) \ar[d]^{\mathrm{Lagr}(f)}  \\
  \mathrm{Dirich}(M) \ar[r]_{\beta_M} & \mathrm{Lagr}(M)
}
\]
commutes for every function $f\maps N \to M$. This is primarily a consequence of the
fact that the differential commutes with pullbacks. As we did in Example
\ref{ex:sympfunction},
write $f^\ast$ for the pullback map and $f_\ast$ for the pushforward map.
Then $\mathrm{Dirich}(f)$ maps a Dirichlet form $Q$ on $N$ to the form $f_\ast Q$,
and $\beta_M$ in turn maps this to the Lagrangian subspace 
\[
  \big\{(\psi,d(f_\ast Q)_\psi) \, \big\vert \, \psi \in \F^M\big\} \subseteq
  \F^M \oplus (\F^M)^\ast.
\]
On the other hand, $\beta_N$ maps a Dirichlet form $Q$ on $N$ to the Lagrangian
subspace
\[
\big\{(\phi,dQ_\phi) \,\big\vert\, \phi \in \F^N\big\}\subseteq
  \F^N \oplus (\F^N)^\ast, 
\]
before $\mathrm{Lagr}(f)$ maps this to the Lagrangian subspace
\[
  \big\{(\psi, f_\ast dQ_\phi) \,\big\vert\, \psi \in \F^M, \phi =
  f^\ast(\psi)\big\} \subseteq \F^M\oplus (\F^M)^\ast.
\]
But 
\[
  f_\ast dQ_{f^\ast\psi} = d(f_\ast Q)_{\psi},
\]
so these two processes commute.

Monoidality requires that the diagrams 
\[
\begin{aligned}
\xymatrix{
  \mathrm{Dirich}(N) \times \mathrm{Dirich}(M) \ar[r]^(.52){\beta_N \times
  \beta_M} \ar[d]_{\delta_{N,M}} & \mathrm{Lagr}(N) \times \mathrm{Lagr}(M)
  \ar[d]^{\lambda_{N,M}}  \\
  \mathrm{Dirich}(N+M) \ar[r]_{\beta_{N+M}} & \mathrm{Lagr}(N+M)
}
\end{aligned}
\quad
\mbox{and}
\quad
\begin{aligned}
\xymatrix{
  & 1 \ar[dl]_{\delta_\varnothing} \ar[dr]^{\lambda_\varnothing}\\
\mathrm{Dirich}(\varnothing)  \ar[rr]_{\beta_\varnothing} &&
\mathrm{Lagr}(\varnothing)
}
\end{aligned}
\]
commute. These do: the Lagrangian subspace corresponding to the sum of Dirichlet
forms is equal to the sum of the Lagrangian subspaces that correspond to the
summand Dirichlet forms, while there is only a unique map $1 \to
\mathrm{Lagr}(\varnothing)$.
\end{proof}

We thus obtain a strict symmetric monoidal dagger functor 
\[   
\mathrm{DirichCospan} \to \mathrm{LagrCospan},
\]
which simply replaces the decoration on each cospan in $\mathrm{DirichCospan}$
with the corresponding Lagrangian subspace.

\subsection{From cospans to relations}

At this point we have checked that the process of reinterpreting a circuit as a
Lagrangian subspace of behaviors is functorial. Our task is now to integrate
this information as more than just a `decoration' on our morphisms. This process
constitutes a monoidal dagger functor
\[
  \mathrm{LagrCospan} \longrightarrow \LagrRel.
\]
This factor of the black box functor is the one that gives it its name;
through this functor we finally seal off the inner structure of our circuits,
leaving us access only to the behavior at the terminals. Its purpose is to take a 
Lagrangian cospan, which captures information about the behaviors of a 
circuit measured at every node, and restrict it down to a 
relation detailing the behaviors simply on the terminals. 

\begin{proposition}
We may define a symmetric monoidal dagger functor
\[
  \mathrm{LagrCospan} \longrightarrow \mathrm{LagrRel}
\]
as follows. On objects let this functor take a finite set $X$ to the symplectic
vector space $\F^X \oplus (\F^X)^\ast$. On morphisms let it take a Lagrangian
cospan
\[
  \big(X\stackrel{i}{\longrightarrow} N \stackrel{o}{\longleftarrow} Y; L
  \subseteq \F^N \oplus (\F^N)^\ast\big)
\]
to the Lagrangian relation
\[
  (S^t\!X\oplus SY) \circ S[i,o]^\dagger \circ L \subseteq
  \overline{\F^X \oplus (\F^X)^\ast} \oplus \F^Y \oplus (\F^Y)^\ast.  
\]  
\end{proposition}

\begin{proof}
The coherence maps here are the usual natural isomorphisms
\[
  \vectf{X} \oplus \vectf{Y} \stackrel{\sim}{\longrightarrow} \vectf{X+Y} 
\]
and
\[
  \{0\} \stackrel{\sim}{\longrightarrow} \vectf{\varnothing}.
\]
It is now routine to observe the symmetric monoidality and dagger-preserving
nature of this construction, as well as that it preserves identities. Finally,
we must check that composition is preserved.  

Using the concept of names introduced in Section \ref{subsec:names}, this comes
down to checking this equality of Lagrangian subspaces:
\[
\begin{aligned}
  \begin{tikzpicture}
	\begin{pgfonlayer}{nodelayer}
		\node [style=none] (0) at (-1.75, 0.5) {};
		\node [style=none] (1) at (-1.25, 0.5) {};
		\node [style=none] (2) at (-1.75, -0) {};
		\node [style=none] (3) at (-1.25, -0) {};
		\node [style=none] (4) at (-1.75, 1.25) {};
		\node [style=none] (5) at (-1.5, -0.75) {};
		\node [style=none] (6) at (-1.5, 0.5) {};
		\node [style=none] (7) at (-1.5, -0) {};
		\node [style=none] (8) at (-0.5, -0) {};
		\node [style=none] (9) at (-0.5, 0.75) {};
		\node [style=none] (10) at (-1, 2) {};
		\node [style=none] (11) at (-1.5, 1.5) {};
		\node [style=none] (12) at (-0.5, 1.5) {};
		\node [style=none] (13) at (-1, 1.5) {};
		\node [style=none] (14) at (-1.75, 0.75) {};
		\node [style=none] (15) at (-0.25, 1.25) {};
		\node [style=none] (16) at (-0.25, 0.75) {};
		\node [style=none] (17) at (-1.5, 0.75) {};
		\node [style=none] (18) at (-1, 1.25) {};
		\node [style=none] (19) at (0.25, 0.75) {};
		\node [style=none] (20) at (1.75, 1.25) {};
		\node [style=none] (21) at (1, 1.5) {};
		\node [style=none] (22) at (1.75, 0.75) {};
		\node [style=none] (23) at (0.5, -0) {};
		\node [style=none] (24) at (0.75, 0.5) {};
		\node [style=none] (25) at (1.5, 0.75) {};
		\node [style=none] (26) at (1, 1.25) {};
		\node [style=none] (27) at (1.5, -0.75) {};
		\node [style=none] (28) at (0.5, 0.75) {};
		\node [style=none] (29) at (0.25, 0.5) {};
		\node [style=none] (30) at (0.25, -0) {};
		\node [style=none] (31) at (0.5, 0.5) {};
		\node [style=none] (32) at (1, 2) {};
		\node [style=none] (33) at (1.5, 1.5) {};
		\node [style=none] (34) at (0.25, 1.25) {};
		\node [style=none] (35) at (0.5, 1.5) {};
		\node [style=none] (36) at (0.75, -0) {};
		\node [style=none] (37) at (0, -0.5) {};
		\node [style=none] (38) at (-1, 1.75) {$L$};
		\node [style=none] (39) at (1, 1.75) {$K$};
		\node [style=none] (40) at (-1, 1) {$S[i_X,o_Y]^\dagger$};
		\node [style=none] (41) at (1, 1) {$S[i_Y,o_Z]^\dagger$};
		\node [style=none] (42) at (-1.5, 0.25) {$S^t\!X$};
		\node [style=none] (43) at (0.5, 0.25) {$S^t\!Y$};
		\node [style=none] (44) at (-1.5, -1) {$\displaystyle\overline{\blacksquare(X)}$};
		\node [style=none] (45) at (1.5, -1) {$\displaystyle\blacksquare(Z)$};
	\end{pgfonlayer}
	\begin{pgfonlayer}{edgelayer}
		\draw (0.center) to (1.center);
		\draw (1.center) to (3.center);
		\draw (3.center) to (2.center);
		\draw (2.center) to (0.center);
		\draw (7.center) to (5.center);
		\draw (8.center) to (9.center);
		\draw (10.center) to (11.center);
		\draw (11.center) to (12.center);
		\draw (12.center) to (10.center);
		\draw (4.center) to (14.center);
		\draw (14.center) to (16.center);
		\draw (16.center) to (15.center);
		\draw (15.center) to (4.center);
		\draw (13.center) to (18.center);
		\draw (17.center) to (6.center);
		\draw (29.center) to (24.center);
		\draw (24.center) to (36.center);
		\draw (36.center) to (30.center);
		\draw (30.center) to (29.center);
		\draw (27.center) to (25.center);
		\draw (32.center) to (35.center);
		\draw (35.center) to (33.center);
		\draw (33.center) to (32.center);
		\draw (34.center) to (19.center);
		\draw (19.center) to (22.center);
		\draw (22.center) to (20.center);
		\draw (20.center) to (34.center);
		\draw (21.center) to (26.center);
		\draw (28.center) to (31.center);
		\draw [bend right=45, looseness=1.00] (8.center) to (37.center);
		\draw [bend right=45, looseness=1.00] (37.center) to (23.center);
	\end{pgfonlayer}
\end{tikzpicture}
  \end{aligned}
  \qquad
  =
  \qquad
  \begin{aligned}
    \begin{tikzpicture}
	\begin{pgfonlayer}{nodelayer}
		\node [style=none] (0) at (-0.75, -0.25) {};
		\node [style=none] (1) at (-0.25, -0.25) {};
		\node [style=none] (2) at (-0.75, -0.75) {};
		\node [style=none] (3) at (-0.25, -0.75) {};
		\node [style=none] (4) at (-0.75, 1.25) {};
		\node [style=none] (5) at (-0.5, -1) {};
		\node [style=none] (6) at (-0.5, -0.25) {};
		\node [style=none] (7) at (-0.5, -0.75) {};
		\node [style=none] (8) at (0, 0.5) {};
		\node [style=none] (9) at (0, 0.75) {};
		\node [style=none] (10) at (-0.5, 2) {};
		\node [style=none] (11) at (-1, 1.5) {};
		\node [style=none] (12) at (0, 1.5) {};
		\node [style=none] (13) at (-0.5, 1.5) {};
		\node [style=none] (14) at (-0.75, 0.75) {};
		\node [style=none] (15) at (0.75, 1.25) {};
		\node [style=none] (16) at (0.75, 0.75) {};
		\node [style=none] (17) at (-0.5, -0) {};
		\node [style=none] (18) at (-0.5, 1.25) {};
		\node [style=none] (19) at (-0.75, -0) {};
		\node [style=none] (20) at (0.75, 0.5) {};
		\node [style=none] (21) at (0.5, 1.25) {};
		\node [style=none] (22) at (0.75, -0) {};
		\node [style=none] (23) at (0.5, -0) {};
		\node [style=none] (24) at (0.5, 1.5) {};
		\node [style=none] (25) at (0.5, -1) {};
		\node [style=none] (26) at (0.5, 2) {};
		\node [style=none] (27) at (1, 1.5) {};
		\node [style=none] (28) at (-0.75, 0.5) {};
		\node [style=none] (29) at (0, 1.5) {};
		\node [style=none] (30) at (-0.5, 1.75) {$L$};
		\node [style=none] (31) at (0.5, 1.75) {$K$};
		\node [style=none] (32) at (0, 1) {$S[j_N,j_M]$};
		\node [style=none] (33) at (0, 0.25) {$\scriptscriptstyle
		  \stackrel{S[j_N \circ i_X,}{j_M \circ o_Z]^\dagger}$};
		\node [style=none] (34) at (-0.5, -0.5) {$S^t\!X$};
		\node [style=none] (35) at (-0.5, -1.25) {$\displaystyle\overline{\blacksquare(X)}$};
		\node [style=none] (36) at (0.5, -1.25) {$\displaystyle\blacksquare(Z)$};
	\end{pgfonlayer}
	\begin{pgfonlayer}{edgelayer}
		\draw (0.center) to (1.center);
		\draw (1.center) to (3.center);
		\draw (3.center) to (2.center);
		\draw (2.center) to (0.center);
		\draw (7.center) to (5.center);
		\draw (8.center) to (9.center);
		\draw (10.center) to (11.center);
		\draw (11.center) to (12.center);
		\draw (12.center) to (10.center);
		\draw (4.center) to (14.center);
		\draw (14.center) to (16.center);
		\draw (16.center) to (15.center);
		\draw (15.center) to (4.center);
		\draw (13.center) to (18.center);
		\draw (17.center) to (6.center);
		\draw (25.center) to (23.center);
		\draw (26.center) to (29.center);
		\draw (29.center) to (27.center);
		\draw (27.center) to (26.center);
		\draw (28.center) to (19.center);
		\draw (19.center) to (22.center);
		\draw (22.center) to (20.center);
		\draw (20.center) to (28.center);
		\draw (21.center) to (24.center);
	\end{pgfonlayer}
\end{tikzpicture}
  \end{aligned}
\]
where the left hand side is the composite in $\LagrRel$ of the images of
the Lagrangian cospans $(X\stackrel{i_X}{\longrightarrow} N
\stackrel{o_Y}{\longleftarrow} Y; L)$ and $(Y\stackrel{i_Y}{\longrightarrow} M
\stackrel{o_Z}{\longleftarrow} Z; K)$, and the right hand side is the image of
their composite in $\mathrm{LagrCospan}$. (Recall that we write $j_N: N \to
N+_YM$ and $j_M: M \to N+_YM$ for the maps given by the pushout.)

Recalling Remark \ref{rmk:dualitydiagrams} pertaining to the functor $S$ and
duals for objects, this implies that it is enough to check the equality of
corelations
\[
  \begin{aligned}
\begin{tikzpicture}
	\begin{pgfonlayer}{nodelayer}
		\node [style=none] (0) at (-1.75, 0.5) {};
		\node [style=none] (1) at (-1.75, 1.25) {};
		\node [style=none] (2) at (-1.5, -0) {};
		\node [style=none] (3) at (-0.5, 0.75) {};
		\node [style=none] (4) at (-1, 1.75) {};
		\node [style=none] (5) at (-1.75, 0.75) {};
		\node [style=none] (6) at (-0.25, 1.25) {};
		\node [style=none] (7) at (-0.25, 0.75) {};
		\node [style=none] (8) at (-1.5, 0.75) {};
		\node [style=none] (9) at (-1, 1.25) {};
		\node [style=none] (10) at (0.25, 0.75) {};
		\node [style=none] (11) at (1.75, 1.25) {};
		\node [style=none] (12) at (1, 1.75) {};
		\node [style=none] (13) at (1.75, 0.75) {};
		\node [style=none] (14) at (0.5, 0.75) {};
		\node [style=none] (15) at (1.5, 0.75) {};
		\node [style=none] (16) at (1, 1.25) {};
		\node [style=none] (17) at (1.5, -0) {};
		\node [style=none] (18) at (0.25, 1.25) {};
		\node [style=none] (19) at (0, 0.25) {};
		\node [style=none] (20) at (-1, 2) {$\displaystyle N$};
		\node [style=none] (21) at (1, 2) {$\displaystyle M$};
		\node [style=none] (22) at (-1, 1) {$[i_X,o_Y]^\dagger$};
		\node [style=none] (23) at (1, 1) {$[i_Y,o_Z]^\dagger$};
		\node [style=none] (24) at (-1.5, -0.25) {$\displaystyle X$};
		\node [style=none] (25) at (1.5, -0.25) {$\displaystyle Z$};
	\end{pgfonlayer}
	\begin{pgfonlayer}{edgelayer}
		\draw (1.center) to (5.center);
		\draw (5.center) to (7.center);
		\draw (7.center) to (6.center);
		\draw (6.center) to (1.center);
		\draw (4.center) to (9.center);
		\draw (8.center) to (2.center);
		\draw (17.center) to (15.center);
		\draw (18.center) to (10.center);
		\draw (10.center) to (13.center);
		\draw (13.center) to (11.center);
		\draw (11.center) to (18.center);
		\draw (12.center) to (16.center);
		\draw [bend right=45, looseness=1.00] (3.center) to (19.center);
		\draw [bend right=45, looseness=1.00] (19.center) to (14.center);
	\end{pgfonlayer}
\end{tikzpicture}
  \end{aligned}
  \qquad
  =
  \qquad
  \begin{aligned}
  \begin{tikzpicture}
	\begin{pgfonlayer}{nodelayer}
		\node [style=none] (0) at (-0.75, 1.25) {};
		\node [style=none] (1) at (-0.5, -0.25) {};
		\node [style=none] (2) at (0, 0.5) {};
		\node [style=none] (3) at (0, 0.75) {};
		\node [style=none] (4) at (-0.5, 1.5) {};
		\node [style=none] (5) at (-0.75, 0.75) {};
		\node [style=none] (6) at (0.75, 1.25) {};
		\node [style=none] (7) at (0.75, 0.75) {};
		\node [style=none] (8) at (-0.5, -0) {};
		\node [style=none] (9) at (-0.5, 1.25) {};
		\node [style=none] (10) at (-0.75, -0) {};
		\node [style=none] (11) at (0.75, 0.5) {};
		\node [style=none] (12) at (0.5, 1.25) {};
		\node [style=none] (13) at (0.75, -0) {};
		\node [style=none] (14) at (0.5, -0) {};
		\node [style=none] (15) at (0.5, 1.5) {};
		\node [style=none] (16) at (0.5, -0.25) {};
		\node [style=none] (17) at (-0.75, 0.5) {};
		\node [style=none] (18) at (-0.5, 1.75) {$\displaystyle N$};
		\node [style=none] (19) at (0.5, 1.75) {$\displaystyle M$};
		\node [style=none] (20) at (0, 1) {$[j_N,j_M]$};
		\node [style=none] (21) at (0, 0.25) {$\scriptscriptstyle \stackrel{[j_N
		\circ i_X,}{j_M \circ o_Z]^\dagger}$};
		\node [style=none] (22) at (-0.5, -0.5) {$\displaystyle X$};
		\node [style=none] (23) at (0.5, -0.5) {$\displaystyle Z$};
	\end{pgfonlayer}
	\begin{pgfonlayer}{edgelayer}
		\draw (2.center) to (3.center);
		\draw (0.center) to (5.center);
		\draw (5.center) to (7.center);
		\draw (7.center) to (6.center);
		\draw (6.center) to (0.center);
		\draw (4.center) to (9.center);
		\draw (8.center) to (1.center);
		\draw (16.center) to (14.center);
		\draw (17.center) to (10.center);
		\draw (10.center) to (13.center);
		\draw (13.center) to (11.center);
		\draw (11.center) to (17.center);
		\draw (12.center) to (15.center);
	\end{pgfonlayer}
\end{tikzpicture}
  \end{aligned}
\]
Writing this instead as a commutative diagram, we wish to prove the equality of
the two corelations $N+M \to X+Y$:
\[
  \xymatrix@C=30pt@R=5pt{
    &&& N+_YM \\
    N+M \ar^{[j_N,j_M]}[urrr] && && && X+Z
    \ar@{^{(}->}[dll]^(.48){\textrm{incl}_{X+Z}} 
    \ar_{[j_N\circ i_X, j_M \circ o_Z]}[ulll] \\
    && X+Y+Y+Z \ar[ull]^(.58){[i_X,o_Y]+[i_Y,o_Z]\quad}
    \ar[rr]_(.53){\idn_X+[\idn_Y,\idn_Y]+\idn_Z} && X+Y+Z
  }
\]
As the right-hand part admits the factorization 
\[
  \xymatrix@C=30pt@R=5pt{
    &&&N+_YM \\
    &&&& && X+Z
    \ar@{^{(}->}[dll]^(.48){\textrm{incl}_{X+Z}} 
    \ar_{[j_N\circ i_X, j_M \circ o_Z]}[ulll] \\
    &&&& X+Y+Z \ar[uul]^{[j_N\circ i_X,j_N\circ o_Y,j_M\circ o_Z]\qquad}
  }
\]
it is enough to check the following diagram commutes when interpreted as a pair
of morphisms from $N+M$ to $X+Y+Z$ in the category of corelations: 
\[
  \xymatrix{
    & N+_YM \\
    N+M \ar[ur]^{[j_N,j_M]} && X+Y+Z \ar[ul]_{\qquad\quad[j_N\circ i_X,j_N\circ o_Y,j_M\circ o_Z]} \\
    & X+Y+Y+Z \ar[ul]^{[i_X,o_Y]+[i_Y,o_Z]\qquad} \ar[ur]_{\qquad\idn_X+[\idn_Y,\idn_Y]+\idn_Z} 
  }
\]
By Lemma \ref{lem:pushoutcorelations}, this is equivalent to checking that it is
a pushout square in $\mathrm{FinSet}$.  This is so: the square commutes in
$\mathrm{FinSet}$ as it is the sum along the lower right edges of the three
commutative squares
\[
\xymatrix@=1.5em{
& N+_YM \\
N+M \ar[ur]^{[j_N,j_M]} && X \ar[ul]_{j_N\circ i_X} \\
& X \ar[ur]_{\idn_X} \ar[ul]^{i_X} 
}
\quad
\xymatrix@=1.5em{
& N+_YM \\
N+M \ar[ur]^{[j_N,j_M]} && Y \ar[ul]_{\quad j_N\circ o_Y=j_M\circ i_Y} \\
& Y+Y \ar[ul]^{o_Y+i_Y} \ar[ur]_{[\idn_Y,\idn_Y]} 
} 
\xymatrix@=1.5em{
& N+_YM \\
N+M \ar[ur]^{[j_N,j_M]} && Z \ar[ul]_{j_M\circ o_Z} \\
& Z \ar[ul]^{o_Z} \ar[ur]_{\idn_Z} 
} 
\]
and given any other object $T$ and maps $f,g$ such that
\[
  \xymatrix{
    & T \\
    N+M \ar[ur]^{f} && X+Y+Z \ar[ul]_{g} \\
    & X+Y+Y+Z \ar[ul]^{[i_X,o_Y]+[i_Y,o_Z]\quad} \ar[ur]_{\qquad \idn_X+[\idn_Y,\idn_Y]+\idn_Z} 
  }
\]
commutes, there is a unique map $N+_YM \to T$ defined by sending $a$ in $N+_YM$
to $f(\hat a)$, where $\hat a$ is a preimage of $a$ in $N+M$ under the coproduct
of pushout maps $[j_N,j_M]$. This is well-defined as the preimage of $a$ is
either unique or equal to $\{o_Y(y),i_Y(y)\}$ for some element $y \in Y$, and
the commutativity of the above square containing $T$ implies that $f(o_Y(y)) =
f(i_Y(y))$. This proves the functoriality of the map $\mathrm{LagrCospan} \to
\LagrRel$ defined above.
\end{proof}

The three functors of this section compose to give the black box functor
\[
\blacksquare\maps \Circ \to \LagrRel.
\] 
Since they are each separately symmetric monoidal dagger functors, the black box
functor is too.


%\section{Concluding remarks}


