\chapter{Corelations: a tool for black boxing} \label{ch.corelations}

Black boxing!

\section{Introduction}

Consider a circuit diagram.
\[
  \begin{tikzpicture}
    \draw (0,0) -- (0,4.5) -- (8,4.5) -- (8,0) -- (0,0);
  \end{tikzpicture}
\]
We often view such diagrams atomically, representing the complete physical
system built as specified. Yet the very process of building such a system
involves assembling it from its parts, each of which we might diagram in the
same way. The goal of this paper is to develop formal category-theoretic tools
for describing and interpreting this process of assembly.

Our interest will be in diagrammatic languages that can be formalised as
hypergraph categories, with the diagrams themselves morphisms in such a
category. A hypergraph category is a symmetric monoidal category in which every
object is equipped with a special Frobenius commutative monoid in a coherent
way. In this diagrammatic picture, the objects are types of interconnection. In
the case of the electrical circuit above, The Frobenius structure

In decorated cospans we saw a good way to construct these categories.

Circuits are constructed to get at some behaviour, with two circuits equivalent
if they specify the same behaviour. Here we focus on this `black boxing' aspect.



In this paper, we introduce a new framework for working with hypergraph
categories: decorated corelations. This extends the decorated cospans framework
introduced in a previous paper \cite{F}. Decorated cospans have been
used \cite{BF, BFP, Po}. Bicategory \cite{Co}.

Semantics live on boundaries only.

Corelations \cite{CF}.

Decorated corelations adds compositional operations to network-diagram
representations, and handles composition of semantics too. 

Two main theorems: 
\begin{theorem}
Given a category $\mc C$ with finite colimits, factorisation system $(\mc E,\mc
M)$ such that $\mc M$ is stable under pushouts, and a symmetric lax monoidal functor 
\[
  \mc C; \mc M^\opp \longrightarrow \Set,
\]
we may define a hypergraph category of with morphisms decorated corelations.
\end{theorem}
\begin{theorem}
Every hypergraph category can be constructed in this way.
\end{theorem}

They apply to functors too.



\section{Corelations}

Given sets $X$, $Y$, a relation $X \leadsto Y$ is a subset of the product $X
\times Y$. More abstractly, we might say a relation is an isomorphism class of
jointly-monic spans in the category of sets. We generalise the dual concept.

\subsection{Factorisation systems.}
The relevant properties of jointly-monic spans come from the fact that
monomorphisms form one half of a factorisation system. A factorisation system
allows any morphism in a category to be factored into the composite of two
morphisms in a coherent way.

\begin{definition}
  A \define{factorisation system} $(\mathcal E,\mathcal M)$ in a category $\mathcal C$
  comprises subcategories $\mathcal E$, $\mathcal M$ of $\mathcal C$ such that
  \begin{enumerate}[(i)]
    \item $\mathcal E$ and $\mathcal M$ contain all isomorphisms of $\mathcal
      C$.
    \item  every morphism $f \in \mathcal C$ admits a factorisation $f=m \circ e$, $e \in
  \mathcal E$, $m \in \mathcal M$
\item given morphisms $f,f'$, with factorisations $f = m \circ e$, $f' = m' \circ
  e'$ of the above sort, for every $u$, $v$ such that the square
  \[
    \xymatrixcolsep{3pc}
    \xymatrixrowsep{3pc}
    \xymatrix{
       \ar[r]^f \ar[d]_u &  \ar[d]^v \\
       \ar[r]_{f'} & 
    }
  \]
  commutes, there exists a unique morphism $s$ such that
  \[
    \xymatrixcolsep{3pc}
    \xymatrixrowsep{3pc}
    \xymatrix{
      \ar[r]^e \ar[d]_u & \ar[r]^m \ar@{.>}[d]^{\exists! s} &  \ar[d]^v \\
       \ar[r]_{e'}& \ar[r]_{m'} & 
    }
  \]
  commutes.
  \end{enumerate}
\end{definition}

\begin{examples}
  We introduce some factorisation systems we will make heavy use of in what
  follows.
  \begin{itemize}
    \item Firstly, write $\mathcal I_{\mathcal C}$ for the wide subcategory of
      $\mathcal C$ containing exactly the isomorphisms of $\mathcal C$. Then
      $(\mathcal I_{\mathcal C}, \mathcal C)$ and $(\mathcal C, \mathcal
      I_{\mathcal C})$ are both factorisation systems in $\mathcal C$. 
    
    \item The prototypical example of a factorisation system is the epi-mono
      factorisation system in $\Set$. This follows from a more general fact,
      true in any category, that if every arrow can be factorised as an epi
      followed by a split mono, then this results in a factorisation system.
      The only non-trivial part to check is the uniqueness condition: given epi
      $e_1,e_2$, split mono $m_1,m_2$ and commutative diagram
      \[
	\xymatrixcolsep{3pc}
	\xymatrixrowsep{3pc}
	\xymatrix{
	  \ar[r]^{e_1} \ar[d]_u & \ar[r]^{m_1} \ar@{.>}[d]^{\exists! t} &  \ar[d]^v \\
	  \ar[r]_{e_2}& \ar[r]_{m_2} & 
	}
      \]
      we must show that there is a unique $t$ that makes the diagram commute.
      Indeed let $t= m_2'vm_1$ where $m_2'$ satisfies $m_2'm_2=id$. 
      To see that the right square commutes, first
      \[
	m_2 t e_1 =  m_2 m_2' v m_1 e_1 = m_2 m_2' m_2 e_2 u = m_2 e_2 u = v m_1 e_1
      \]
      and since $e_1$ is epi we have $m_2 t = v m_1$. For the left square,
      $t e_1 = m_2' v m_1 e_1 = m_2' m_2 e_2 u = e_2 u$. Uniqueness is immediate,
      since, e.g.\ $m_2$ is mono. 
\end{itemize}
\end{examples}

See \cite[\textsection 14]{AHS} for more details.


\subsection{Corelations}
Relations, then may be generalised as spans such that the span maps `jointly'
belong to some class $\mc M$ of an $(\mc E,\mc M)$-factorisation system. We
define corelations in the dual manner.

\begin{definition}
  Let $\mathcal C$ be a category with finite colimits, and let $(\mathcal E,
  \mathcal M)$ be a factorisation system on $\mathcal C$. An \define{$(\mathcal
  E,\mathcal M)$-corelation} $X \leadsto Y$, is an isomorphism class of cospans
  such that for each representative $X \stackrel{i}\longrightarrow N
  \stackrel{o}\longleftarrow Y$ of the corelation, the copairing $[i,o]\maps X+Y
  \to N$ lies in $\mathcal E$.
\end{definition}

Note that if any cospan $X \stackrel{i}\longrightarrow N
\stackrel{o}\longleftarrow Y$ has the property that the copairing $[i,o]\maps
X+Y \to N$ lies in $\mathcal E$, then all isomorphic cospans have the same
property. Thus to specify a corelation one may simply choose any cospan with
this property. We say that the cospan is \define{jointly $\mathcal E$-like}.

When the factorisation system is clear from context, we simply call $(\mathcal
E,\mathcal M)$-corelations: `corelations'. We shall also take the freedom to be
lazy with our language, and often refer to jointly $\mathcal E$-like cospans
themselves as corelations---of course we mean the corelation given by its
isomorphism class.

If $f\maps A \to N$ is a morphism with factorisation $f = m \circ e$, write
$\widetilde N$ for the object such that $e\maps A \to \widetilde N$ and $m\maps
\widetilde N \to N$. Now, given a cospan $X \stackrel{i_X}{\longrightarrow} N
\stackrel{o_Y}{\longleftarrow} Y$, we may use the factorisation system to write
the copairing $[i_X,o_Y]\maps X+Y \to N$ as
\[
  X+Y \stackrel{e}{\longrightarrow} \widetilde{N} \stackrel{m}{\longrightarrow}
  N.
\]
From the universal property of the coproduct, we also have maps $\iota_X\maps X
\to X+Y$ and $\iota_Y\maps Y \to X+Y$. We then call the jointly $\mathcal
E$-like cospan, or corelation,
\[
  X \stackrel{e\circ \iota_X}{\longrightarrow} \widetilde{N} \stackrel{e \circ
  \iota_Y}{\longleftarrow} Y
\]
the \define{$\mathcal E$-part} of the above cospan. On occasion we will also
write $e\maps X+Y \to \widetilde N$ for the same corelation.

\begin{examples}
  \begin{itemize}
    \item $\Set$ with morphism-isomorphism factorisation. Corelations are just cospans.
    \item $\Set$ with isomorphism-morphism factorisation. Objects are sets. There exists
      a unique morphism between any two objects. Note hypergraph structure is very
      trivial.
    \item Set has colimits. Set also has epi-mono factorisations. This is stable under
  pushout. Corelations in $\mathrm{Set}$ are equivalence relations. Also the free
  extraspecial commutative Frobenius monoid on one object \cite{CF}.
  \end{itemize}
\end{examples}

\subsection{Categories of corelations.}

We compose two corelations by choosing representative jointly $\mathcal E$-like
cospans and taking the isomorphism class of the $\mathcal E$-part of their
composite cospan.  This is well-defined.

\begin{proposition}
  Composition of corelations is well-defined.
\end{proposition}
\begin{proof}
  Let
    $(X \stackrel{i_X}{\longrightarrow} N \stackrel{o_Y}{\longleftarrow} Y)$ and 
    $(X \stackrel{i_X'}{\longrightarrow} N' \stackrel{o_Y'}{\longleftarrow} Y)$,
%  \[
%    X \stackrel{i_X}{\longrightarrow} N \stackrel{o_Y}{\longleftarrow} Y 
%    \qquad \mbox{and} \qquad 
%    X \stackrel{i_X'}{\longrightarrow} N' \stackrel{o_Y'}{\longleftarrow} Y
%  \]
  and
    $(Y \stackrel{i_Y}{\longrightarrow} M \stackrel{o_Z}{\longleftarrow} Z)$ and 
    $(Y \stackrel{i_Y'}{\longrightarrow} M' \stackrel{o_Z'}{\longleftarrow} Z)$
%  \[
%    Y \stackrel{i_Y}{\longrightarrow} M \stackrel{o_Z}{\longleftarrow} Z 
%    \qquad \mbox{and} \qquad 
%    Y \stackrel{i_Y'}{\longrightarrow} M' \stackrel{o_Z'}{\longleftarrow} Z
%  \]
  be pairs of isomorphic jointly $\mathcal E$-like cospans. As composites
  of isomorphic cospans are isomorphic, their composites \emph{as cospans}
    $(X \longrightarrow N+_YM \longleftarrow Z)$ and 
    $(X \longrightarrow N'+_YM' \longleftarrow Z)$
%  \[
%    X \longrightarrow N+_YM \longleftarrow Z 
%    \qquad \mbox{and} \qquad 
%    X \longrightarrow N'+_YM' \longleftarrow Z
%  \]
  are isomorphic. Thus, by the properties of factorisation systems, there exists
  an isomorphism $s$ such that the diagram
  \[
    \xymatrixcolsep{3pc}
    \xymatrixrowsep{3pc}
    \xymatrix{
      X+Z \ar[r]^e \ar@{=}[d] & \widetilde{N+_YM} \ar[r]^m \ar@{.>}[d]^{s}_\sim & N+_YM
      \ar[d]^\sim \\
      X+Z \ar[r]_{e'}&\widetilde{N'+_YM'} \ar[r]_{m'} & N'+_YM'
    }
  \]
  commutes, where we write $m\circ e\maps X+Z \to N+_YM$ and $m'\circ e'\maps
  X+Z \to N'+_YM'$ for the maps obtained from the above composite cospans. This
  is an isomorphism of the composite corelations, as required.
\end{proof}

This composition is not associative in general. It is, however, associative when
$\mathcal M$ is \define{stable under pushout}: that is, when for any pushout square
  \[
    \xymatrixcolsep{3pc}
    \xymatrixrowsep{3pc}
    \xymatrix{
      & \ar[l]^j \\
      \ar[u] &  \ar[u] \ar[l]_m 
    }
  \]
such that $m \in \mathcal M$, we also have that $j \in \mathcal M$. 

The fact that it forms a category is well-known. Moreover, corelations form a
hypergraph category. For us it will be the consequence of a general fact
regarding functors between corelation categories.

\begin{lemma}
  Let $\mc C$ be a category with finite coproducts, and let $(\mc E, \mc M)$ be a
  factorisation system on $\mc C$. Write $+$ for the coproduct in $\mc C$. Then
  $(\mc E,+)$ is a symmetric monoidal category.
\end{lemma}
\begin{proof}
  The only thing to check is that $\mc E$ is closed under $+$. That is, given
  $f\maps A \to B$ and $g\maps C \to D$ in $\mc E$, we wish to show that
  $f+g\maps A+C \to B+D$, defined in $\mc C$, is also a morphism in $\mc E$. 

  Let $f+g$ have factorisation $A+C \stackrel{e}\longrightarrow \widetilde{B+D}
  \stackrel{m}\longrightarrow B+D$, where $e \in \mc E$ and $m \in \mc
  M$. We will prove that $m$ is an isomorphism. To construct an inverse, recall
  that by definition, as $f$ and $g$ lie in $\mc E$, there exist morphisms
  $x\maps B \to \widetilde{B+D}$ and $y\maps D \to \widetilde{B+D}$ such that
  \[ \label{eq.coreltensor}
    \xymatrixcolsep{2pc}
    \xymatrixrowsep{2pc}
    \xymatrix{
      A \ar[r]^f \ar[d] & B \ar@{=}[r] \ar@{.>}[d]^x & B
      \ar[d] \\
      A+C \ar[r]_{e}&\widetilde{B+D} \ar[r]_{m} & B+D
    }
    \qquad \mbox{and} \qquad
    \xymatrixcolsep{2pc}
    \xymatrixrowsep{2pc}
    \xymatrix{
      C \ar[r]^g \ar[d] & D \ar@{=}[r] \ar@{.>}[d]^y & D
      \ar[d] \\
      A+C \ar[r]_{e}&\widetilde{B+D} \ar[r]_{m} & B+D
    }
    \tag{1}
  \]
  The copairing $[x,y]$ is an inverse to $m$. 
  
  Indeed, taking the coproduct of the top rows of the two diagrams above and the
  copairings of the vertical maps gives the commutative diagram
  \[
    \xymatrix{
      A+C \ar[r]^{f+g} \ar@{=}[d] & B+D \ar@{=}[r] \ar[d]_{[x,y]} & B+D \ar@{=}[d] \\
      A+C \ar[r]^{e} & \widetilde{B+D} \ar[r]^{m} & B+D
    }
  \]
  Reading the right-hand square immediately gives $m \circ [x,y] =1$.
  
  Conversely, to see that $[x,y] \circ m = 1$, remember that by definition $f+g
  = m \circ e$. So the left-hand square above implies that
  \[
    \xymatrixcolsep{2pc}
    \xymatrixrowsep{2pc}
    \xymatrix{
      A+C \ar[r]^e \ar@{=}[d] & \widetilde{B+D} \ar[d]^{[x,y] \circ m} \\
      A+C \ar[r]_{e}&\widetilde{B+D} 
    }
  \]
  commutes. But by the universal property of factorisation systems, there is a
  unique map $\widetilde{B+D} \to \widetilde{B+D}$ such that this diagram
  commutes, and clearly the identity map also suffices. Thus $[x,y] \circ m =
  1$.
\end{proof}

This implies that every factorisation system on a monoidal category in which the
monoidal product is the coproduct is a so-called monoidal factorisation system,
as defined by Ambler \cite{Am}.
\begin{corollary}
  Monoidal composition of corelations is well-defined.
\end{corollary}

%We sketch an
%elementary argument. A bicategorical version of the dual argument, for spans and
%relations, can be found in \cite{JW}.



  Recall that $\cospan(\mathcal C)$ is a hypergraph category, with unit
  $\epsilon_X$ on each object $X$ given by the map $0 \to X$ from the initial
  object $0$ in $\mathcal C$, the multiplication $\mu_X$ given by the map $X+X
  \to X$ in $\mathcal C$, and the comonoid maps analogous maps in the opposite
  category. These maps all represent corelations, and so we may use the same maps as
  hypergraph structure on $\corel(\mathcal C)$.   

\begin{theorem} \label{thm.cospantocorel}
  Let $\mathcal C$ be a category with finite colimits, and let $(\mathcal E,
  \mathcal M)$ be factorisation system on $\mathcal C$ such that $\mathcal M$ is
  stable under pushout. Then there exists a hypergraph category
  $\mathrm{Corel}(\mathcal C)$ with 
  \begin{center}
    \begin{tabular}{| c | p{.65\textwidth} |}
      \hline
      \multicolumn{2}{|c|}{The hypergraph category $(\mathrm{Corel},+)$} \\
      \hline
      \textbf{objects} & the objects of $\mathcal C$ \\ 
      \textbf{morphisms} & $(\mc E,\mc M)$-corelations in $\mathcal C$\\ 
      \textbf{composition} & given by the $\mc E$-part of pushout \\
      \textbf{monoidal product} & the coproduct in $\mathcal C$ \\
      \textbf{coherence maps} & inherited from $\cospan(\mc C)$  \\
      \textbf{hypergraph maps} & inherited from $\cospan(\mc C)$ \\
      \hline
    \end{tabular}
  \end{center}  
\end{theorem}

To check this is well-defined we need to show that these structures obey the
necessary axioms: associativity, the identity laws, the monoidal coherence laws,
the special commutative Frobenius monoid laws. Instead of proving each axiom
directly, we will leverage the fact that we already know $\cospan(\mc C)$ is a
hypergraph category, and show that $\corel(\mc C)$ is the image of $\cospan(\mc
C)$ under a composition-preserving map that also respects the monoidal and
hypergraph structure. This is a proof strategy we will employ a number of times
in this paper.

Before completing this proof in the next section, we note a few examples.
\begin{examples}
  In the factorisation systems above the right factor $\mathcal M$ is stable
  under pushout.  Thus we have hypergraph categories of cospans, equivalence
  relations, and trivial maps.  
\end{examples}

\subsection{Functors between corelation categories.}

\begin{proposition} \label{prop.corelfunctors}
  Let $\mathcal C$, $\mathcal C'$ have finite colimits and respective
  factorisation systems $(\mathcal E, \mathcal M)$, $(\mathcal E', \mathcal M')$,
  such that $\mathcal M$ and $\mathcal M'$ are stable under pushout. Further let
  $A\maps \mathcal C \to \mathcal C'$ be a functor that preserves finite colimits
  and such that the image of $\mathcal M$ lies in $\mathcal M'$.

  Then we may define a hypergraph functor $\square\maps \corel(\mathcal C) \to
  \corel(\mathcal C')$ sending each object $X$ in $\corel(\mathcal C)$ to $AX$ in
  $\corel(\mc C')$ and each corelation 
  \[
    X \stackrel{i_X}{\longrightarrow} N \stackrel{o_Y}{\longleftarrow} Y 
  \]
  to the jointly-$\mc E'$-part
  \[
    AX \stackrel{Ai_X}{\longrightarrow} \widetilde{AN} \stackrel{Ao_Y}{\longleftarrow} AY.
  \]
  of the image cospan. The coherence maps are the isomorphisms
  $\kappa_{X,Y}\maps AX+AY \to A(X+Y)$ given as $A$ preserves colimits.
\end{proposition}

\begin{proof}
  Note that we still have not proved that $\corel(\mc C)$ is a category, let
  alone a hypergraph category. To begin, we will simply show that the map
  $\square$ is composition-preserving, and then that $\square$ respects the
  monoidal and hypergraph structure. Finally, specialising to the case where $A$
  is the identity functor on $\mc C$ and $(\mc E,\mc M)$ is the factorisation
  system $(\mc C, \mc I_{\mc C})$, we observe that the domain of $\square$ is
  then $\cospan(\mc C)$, and that $\square$ maps it fully
  (surjectively-on-morphisms) and bijectively-on-objects onto $\corel(\mc C')$.
  This shows that all corelation categories are indeed deserving of the name
  category and moreover hypergraph category, and hence that $\square$ is
  deserving of the name hypergraph functor.

  First, that $\square$ preserves composition. Let
  \[
    \alpha = (X \stackrel{i_X}{\longrightarrow} N \stackrel{o_Y}{\longleftarrow} Y)
    \qquad \mbox{and} \qquad 
    \beta = (Y \stackrel{i_Y}{\longrightarrow} M \stackrel{o_Z}{\longleftarrow} Z)
  \]
  be corelations in $\mathcal C$. By definition, the corelation $\square(\beta)
  \circ \square(\alpha)$ is given by the $\mathcal E$-part of the top row of the
  diagram below, while $\square(\beta \circ \alpha)$ is given by the $\mathcal
  E$-part of the bottom row:
  \[ \label{diag.eparts}
    \begin{aligned}
      \xymatrixcolsep{3pc}
      \xymatrixrowsep{2pc}
      \xymatrix{
	AX+AZ \ar[r]^{\mc E'} \ar@{=}[dd] & \widetilde{\widetilde{AN}+_{AY}\widetilde{AM}}
	\ar[rr]^{\mc M'} \ar@{<.>}[dd]^{n} && \widetilde{AN}+_{AY}\widetilde{AM}
	\ar[d]^{m'_{AN}+_{AY}m'_{AM}} \\
	&&& AN+_{AY}AM \\
	AX+AZ \ar[r]^{\mc E'} & \widetilde{A(\widetilde{N+_YM})} \ar[r]^{\mc M'} &
	A(\widetilde{N+_YM}) \ar[r]^{Am_{N+_YM}} & A(N+_YM) \ar@{<->}[u]_{\sim}
      }
    \end{aligned}
    \tag{$\ast$}
  \]
  In the above, the morphisms labelled $\mc E'$ lie in $\mc E'$, and similarly
  for $\mc M'$; these are given by the factorisation system on $\mc C'$.  The
  remaining three maps in the lower right hand corner lie in $\mc M'$ too:
  $Am_{N+_YM}$ as it is in the image of $\mc M$, $m'_{AN}+_{AY}m'_{AM}$ as $\mc
  M'$ is stable under pushout, and the morphism labelled $\sim$ as $A$ preserves
  pushouts and $\mc M'$ contains all isomorphisms. 

  Moreover, the diagram commutes as both maps $AX+AZ \to AN+_{AY}AM$ compose to
  that given by the pushout of the images of $\alpha$ and $\beta$ over $AY$.
  Thus the diagram represents two $(\mc E', \mc M')$ factorisations of the same
  morphism, and there exists an isomorphism $n$ between the corelations
  $\square(\beta) \circ \square(\alpha)$ and $\square(\beta\circ\alpha)$. This
  proves that $\square$ preserves composition.

  As $A$ preserves colimits, it follows immediately that $\square$ respects the
  monoidal and hypergraph structure. Indeed, the coherence maps are natural by
  the properties of coproducts, the symmetric monoidal coherence laws are each
  diagrams of isomorphisms of coproducts, and for example the Frobenius
  multiplication $[1_X,1_X]$ on an object $X$ of $\mc C$ obeys $A[1_X,1_X] \circ
  \kappa_{X,X}= [1_{AX},1_{AX}]$, as required. 
  
  To complete the proof then, consider the case with $\mc C = \mc C'$, $(\mc
  E,\mc M) = (\mc C, \mc I_{\mc C})$, and $A = 1_{\mc C}$. Then the domain of
  $\square$ is $\cospan(\mc C)$ by definition. In this case, the functor
  $\square\maps \cospan(\mc C) \to \corel(\mc C)$ is bijective-on-objects and
  surjective-on-morphisms. As $\cospan(\mc C)$ is a hypergraph category, and
  $\square$ preserves composition and respects the monoidal and hypergraph
  structure, $\corel(\mc C)$ is also a hypergraph category. Moreover, thus
  $\square$ is a hypergraph functor.
\end{proof}

\begin{corollary}
  There is a hypergraph functor 
  \[
    \square\maps \mathrm{Cospan}(\mathcal C) \longrightarrow \mathrm{Corel}(\mathcal C)
  \]
  that takes each object of $\cospan(\mathcal C)$ to itself as an object of
  $\corel(\mathcal C)$ and each cospan to its $\mathcal E$-part.
\end{corollary}


\subsection{Example: extraspecial commutative Frobenius monoids}
\begin{definition}
  An \define{extraspecial commutative Frobenius monoid}
  $(X,\mu,\eta,\delta,\epsilon)$ in a monoidal category $(\mathcal C, \otimes)$
  is a special commutative Frobenius monoid that further obeys the extra law
  \[
    \extral{.1\textwidth} = \extrar{.1\textwidth}
  \]
\end{definition}

The extra law is a more recent discovery, appearing first under this name in the
work of Baez and Erbele \cite{BE}, as the `bone law' in \cite{BSZ,FRS}, and as
the `irredundancy law' in \cite{Za}.

Observe that each of these equations equate string diagrams that connect
precisely the same elements of the domain and codomain. To wit, the
associativity, coassociativity, and Frobenius laws show that the order in which
we build a connected component through pairwise clustering is irrelevant, the
special law shows that having multiple connections between points is irrelevant,
and the extra law shows that `extra' components not connected to the domain or
codomain are irrelevant. 

Theorem will show that these equations are exactly those required to
have the converse: two morphisms built from the generators of an extraspecial
commutative Frobenius monoid are equal and if and only if their diagrams impose
the same connectivity relations on the disjoint union of the domain and
codomain. This formalises an extension of the well-known spider theorem for
special commutative Frobenius monoids \cite{CK,CPP}. 

\subsection{Example: linear relations as corelations in $\Vect$.}

\begin{example}
  \textbf{Linear relations.} 
  Recall that a linear relation $L\maps U \leadsto V$ is a subspace $L \subseteq
  U \oplus V$. We compose linear relations as we do relations, and vector spaces
  and linear relations form a category $\LinRel$. This category can be
  constructed as the category of relations in the category $\Vect$ of
  vector spaces and linear maps with respect to epi-mono factorisations. We show
  that they may also be constructed as corelations in $\Vect$ with respect to
  epi-mono factorisations.
  
  If we restrict to the full subcategory $\FinVect$ of finite dimensional vector
  spaces this is easy to see: after picking a basis for each vector space the
  transpose yields an equivalence of $\FinVect$ with its opposite category, so
  the category of $(\mathcal E,\mathcal M)$-corelations (jointly-epic cospans)
  is isomorphic to the category of $(\mathcal E,\mathcal M)$-relations
  (jointly-monic spans) in $\FinVect$. This fact has been fundamental in work on
  finite dimensional linear systems and signal flow diagrams \cite{BE,BSZ,FRS}.
  
  We prove the general case in detail. To begin, note $\mathrm{Vect}$ has an
  epi-mono factorisation system with monos stable under pushouts. This
  factorisation system is inherited from $\Set$: the epimorphisms in $\Vect$ are
  precisely the surjective linear maps, the monomorphisms are the injective
  linear maps, and the image of a linear map is always a subspace of the
  codomain, and so itself a vector space. Monos are stable under pushout as the
  pushout of a diagram $V \stackrel{f}{\leftarrow} U \stackrel{m}{\rightarrow}
  W$ is $V \oplus W/\im[f\; -g]$. The map $m'\maps V \to V \oplus W/\im[f\; -g]$
  into the pushout has kernel $f(\ker m)$. Thus when $m$ is a monomorphism, $m'$
  is too.
 
  Thus we have a category of corelations $\corel(\Vect)$. We show that the map
  $\corel(\Vect) \to \LinRel$ sending each vector space to itself and each
  corelation
  \[
    U \xrightarrow{f} A \xleftarrow{g} V
  \]
  to the linear subspace $\ker[f\;-g]$ is a full, faithful, and
  bijective-on-objects functor.

  Indeed, corelations $U \xrightarrow{f} A \xleftarrow{g} V$ are in one-to-one
  correspondence with surjective linear maps $U\oplus V \to A$, which are in
  turn, by the isomorphism theorem, in one-to-one correspondence with subspaces
  of $U\oplus V$. These correspondences are described by the kernel construction
  above. Thus our map is evidently full, faithful, and bijective-on-objects. It
  also maps identities to identities. It remains to check that it preserves
  composition.

  Suppose we have corelations $U \xrightarrow{f} A \xleftarrow{g} V$
  and $V \xrightarrow{h} B \xleftarrow{k} W$. Then their pushout is given by
  $P=A \oplus B/\im[g\;-h]$, and we may draw the pushout diagram
  \[
    \xymatrix{
      U \ar[dr]_{f} & & V \ar[dl]^{g}  
      \ar[dr]_{h} & & W \ar[dl]^{k} {} 
      \\
      & A \ar[dr]_{\iota_A} & & B \ar[dl]^{\iota_B}  \\
      & & P {\save*!<0cm,-.5cm>[dl]@^{|-}\restore}
    }
  \]
  We wish to show the equality of relations
  \[
    \ker[f\;-g];\ker[h\;-k] = \ker[\iota_A f\; -\iota_B g].
  \]
  Now $(\mathbf{u},\mathbf{w}) \in U \oplus W$ lies in the composite relation
  $\ker[f\;-g];\ker[h\;-k]$ iff there exists $\mathbf{v} \in V$ such that
  $f\mathbf{u} = g\mathbf{v}$ and $h\mathbf{v} = k\mathbf{w}$. But as $P$ is the
  pushout, this is true iff 
  \[
    \iota_A f \mathbf{u} = \iota_A g \mathbf{v} = \iota_B h \mathbf{v} =
    \iota_B k \mathbf{w}.
  \]
  This in turn is true iff $(\mathbf{u}, \mathbf{w}) \in \ker[\iota_Af\;
  -\iota_Bk]$, as required. 
\end{example}

Note that if both $(\mathcal E, \mathcal M)$ and $(\mathcal E', \mathcal M')$
are epi-mono factorisations, then we always have that $F(\mathcal E) \subseteq
\mathcal E'$ and $F(\mathcal M) \subseteq \mathcal M'$. Indeed, if an
(one-sided) inverse exists in the domain category, it exists in the codomain
category.



\section{Decorated corelations} \label{sec:dcorc}

When enough structure is available to us, we may decorate corelations too.
Furthermore, and key to the idea of `black-boxing', we get a hypergraph functor
from decorated cospans to decorated corelations.


\subsection{Constructing right adjoints.}
A subcategory stable under pushouts is a useful thing.

\begin{lemma}
  Let $\mathcal C$ be a category with finite colimits, and let $\mathcal M$ be a
  subcategory of $\mathcal C$ stable under pushouts and containing all
  isomorphisms. Then $(\mc M,+)$ is a monoidal category.
\end{lemma}
\begin{proof}
  It is enough to show that for all morphisms $m,m' \in \mc M$ we have $m+m'$ in
  $\mc M$. Since $\mc M$ contains all isomorphisms, the coherence maps are
  inherited from $\mc C$. The required axioms---the functoriality of the tensor
  product, the naturality of the coherence maps, and the coherence laws---are
  also inherited as they hold in $\mc C$.

  To see $m+m'$ is in $\mc M$, simply observe that we have the pushout square
  \[
    \xymatrix{
      A \ar[r]^m \ar[d]_{\iota} & B \ar[d]^{\iota} \\
      A+C \ar[r]^{m+1} & B+C
    }
  \]
  in $\mc C$. As $\mc M$ is stable under pushout, $m+1 \in \mc M$. Similarly,
  $1+m' \in \mc M$. Thus their composite $m+m'$ lies in $\mc M$, as required.
\end{proof}

\begin{proposition}
  Let $\mathcal C$ be a category with finite colimits, and let $\mathcal M$ be a
  subcategory of $\mathcal C$ stable under pushouts. Then we define the category
  $\mathcal C; \mathcal M^\opp$ as follows  
  \begin{center}
    \begin{tabular}{| c | p{.65\textwidth} |}
      \hline
      \multicolumn{2}{|c|}{The symmetric monoidal category $(\mc C;\mc M^\opp,+)$} \\
      \hline
      \textbf{objects} & the objects of $\mathcal C$ \\ 
      \textbf{morphisms} & isomorphism classes of cospans of the form
      $\stackrel{c}\rightarrow \stackrel{m}\leftarrow$, where $c$ is a morphism
      in $\mathcal C$ and $m$ a morphism in $\mathcal M$\\ 
      \textbf{composition} & given by pushout \\
      \textbf{monoidal product} & the coproduct in $\mathcal C$ \\
      \textbf{coherence maps} & the coherence maps in $\mc C$ \\
      \hline
    \end{tabular}
  \end{center}
\end{proposition}
\begin{proof}
  Composition is well-defined as $\mc M$ is stable under pushouts. Monoidal
  composition is well-defined by lemma. Necessary laws hold as they are
  inherited from $\mc C$. 
\end{proof}

This category can be viewed as a bicategory, with 2-morphisms given by maps of
cospans. In this bicategory every morphism of $\mc M$ has a right adjoint.

\begin{examples}
  \begin{itemize}
    \item Note that $\mathcal C; \mathcal C^\opp$ is by definition equal to
      $\cospan(\mathcal C)$. 

    \item Writing $\mathcal I_{\mathcal C}$ for the wide subcategory of
      isomorphisms in $\mathcal C$, note that $\mathcal C;\mathcal I_{\mathcal
      C}^\opp$ is naturally isomorphic to $\mathcal C$.
  \end{itemize}
\end{examples}

\begin{lemma}
  Let $\mathcal C$, $\mathcal C'$ be categories with finite colimits, and let
  $\mathcal M$, $\mathcal M'$ be subcategories each stable under pushouts. Let
  $A\maps \mathcal C \to \mc C'$ be functor that preserves colimits and such
  that the image of $\mc M$ lies in $\mc M'$. Then $A$ extends to a symmetric
  (strong) monoidal functor
  \[
    A\maps \mc C;\mc M^\opp \longrightarrow \mc C'; \mc M'^\opp.
  \]
  mapping $X$ to $AX$ and $\stackrel{c}\rightarrow \stackrel{m}\leftarrow$ to
  $\stackrel{Ac}\rightarrow \stackrel{Am}\leftarrow$.
\end{lemma}
\begin{proof}
  Note $A(\mc M) \subseteq \mc M'$, so $\stackrel{Ac}\rightarrow
  \stackrel{Am}\leftarrow$ is indeed a morphism in $\mc C';\mc M'$. The functor
  $A$ preserves colimits, so composition is preserved.
\end{proof}

Note this could be done more generally with any two isomorphism-containing wide
subcategories stable under pushout.

\subsection{Decorated corelations.}

\begin{definition}
  Let $\mathcal C$ be a category with finite colimits, and let $(\mathcal E,
  \mathcal M)$ be a factorisation system on $\mathcal C$. Suppose that we also
  have a lax monoidal functor
  \[
    F: (\mathcal C;\mathcal M^\opp,+) \longrightarrow (\Set, \times).
  \]
  Then we define an $F$-\define{decorated corelation} to be the isomorphism
  class of a pair
  \[
    \left(
    \begin{aligned}
      \xymatrix{
	& N \\  
	X \ar[ur]^{i} && Y \ar[ul]_{o}
      }
    \end{aligned}
    ,
    \qquad
    \begin{aligned}
      \xymatrix{
	FN \\
	1 \ar[u]_{s}
      }
    \end{aligned}
    \right)
  \]
  where the cospan is jointly-$\mathcal E$-like.
\end{definition}

Again, we will be lazy about the distinction between a decorated corelation and
its isomorphism class.

Suppose we have decorated corelations
  \[
    \left(
    \begin{aligned}
      \xymatrix{
	& N \\  
	X \ar[ur]^{i} && Y \ar[ul]_{o}
      }
    \end{aligned}
    ,
    \qquad
    \begin{aligned}
      \xymatrix{
	FN \\
	1 \ar[u]_{s}
      }
    \end{aligned}
    \right)
    \qquad
    \mbox{and}
    \qquad
    \left(
    \begin{aligned}
      \xymatrix{
	& M \\  
	Y \ar[ur]^{i} && Z \ar[ul]_{o}
      }
    \end{aligned}
    ,
    \qquad
    \begin{aligned}
      \xymatrix{
	FM \\
	1 \ar[u]_{s}
      }
    \end{aligned}
    \right).
  \]
  Then their composite is given by the composite corelation
  \[
    \xymatrix{
      & \widetilde{N+_YM} \\  
      X \ar[ur]^{i} && Z \ar[ul]_{o}
    }
  \]
  paired with the decoration
  \[
    1 \longrightarrow F(N+M) \longrightarrow F(N+_YM) \stackrel{F(m^\opp)}\longrightarrow F(\widetilde{N+_YM})
  \]
  This is well-defined.

\subsection{Categories of decorated corelations.}
\begin{theorem}
  Let $\mathcal C$ be a category with finite colimits and factorisation system
  $(\mathcal E, \mathcal M)$ with $\mathcal M$ stable under pushout, and let 
  \[
    F: (\mathcal C;\mathcal M^\opp,+) \longrightarrow (\Set, \times)
  \]
  be a symmetric lax monoidal functor.  Then we may define 
  \begin{center}
    \begin{tabular}{| c | p{.65\textwidth} |}
      \hline
      \multicolumn{2}{|c|}{The hypergraph category $(F\mathrm{Corel},+)$} \\
      \hline
      \textbf{objects} & the objects of $\mathcal C$ \\ 
      \textbf{morphisms} & isomorphism classes of $F$-decorated corelations in
      $\mathcal C$\\ 
      \textbf{composition} & given by $\mc E$-part of pushout with restricted
      decoration  \\
      \textbf{monoidal product} & the coproduct in $\mathcal C$ on objects, coproduct
      of cospans and pair of decorations on morphisms. \\
      \textbf{coherence maps} & maps from $\cospan(\mc C)$ with restricted empty
      decoration \\
      \textbf{hypergraph maps} & maps from $\cospan(\mc C)$ with restricted empty
      decoration \\
      \hline
    \end{tabular}
  \end{center}
\end{theorem}
Similar to the corelations theorem (Theorem \ref{thm.cospantocorel}), we prove
this alongside the theorem in the next subsection.
\begin{example}
  Note that decorated cospans are a special case of decorated corelations: we
  use an morphism--isomorphism factorisation system.
\end{example}

\begin{example} \label{ex.undeccorel}
  Note that `undecorated' corelations are a special case of decorated
  corelations: they are corelations decorated by the functor $1\maps \mc C;\mc
  M^\opp \to \Set$ that maps each object to the one element set $1$, and each
  morphism to the identity function on $1$. This is a symmetric monoidal functor
  with the coherence maps all also the identity function on $1$.
\end{example}

The key difference is to decorate cospans we need to know how to push
decorations up. To decorate corelations we all need to know how to pull
decorations back down. This is related to the existence of an extraspecial
commutative Frobenius monoid in our main applications.

\subsection{Functors between decorated corelation categories}

\begin{proposition}\label{prop.deccorelfunctors}
  Let $\mathcal C$, $\mathcal C'$ have finite colimits and respective
factorisation systems $(\mathcal E, \mathcal M)$, $(\mathcal E', \mathcal M')$,
such that $\mathcal M$ and $\mathcal M'$ are stable under pushout, and suppose
that we have symmetric lax monoidal functors
\[
  F: (\mathcal C;\mathcal M^\opp,+) \longrightarrow (\Set, \times)
\]
and
\[
  G: (\mathcal C';\mathcal M'^\opp,+) \longrightarrow (\Set, \times).
\]

Further let $A\maps \mathcal C \to \mathcal C'$ be a functor that preserves
finite colimits and such that the image of $\mathcal M$ lies in $\mathcal M'$.
This functor $A$ extends to a symmetric monoidal functor $\mc C;\mc M^\opp \to
\mc C';\mc M'^\opp$.

Suppose we have a monoidal natural transformation $\theta$:
\[
  \xymatrixrowsep{2ex}
  \xymatrix{
    \mc C; \mc M^\opp \ar[dd]_{A} \ar[drr]^F  \\
    &\twocell \omit{_\:\theta}& \Set \\
    \mc C'; \mc {M'}^\opp \ar[urr]_{G} 
  }
\]

Then we may define a hypergraph functor $T\maps F\mathrm{Corel} \to
G\mathrm{Corel}$ sending each object $X \in F\mathrm{Corel}$ to $AX \in
G\mathrm{Corel}$ and each decorated corelation 
\[
  X \stackrel{i_X}{\longrightarrow} N \stackrel{o_Y}{\longleftarrow} Y, \quad
  1 \stackrel{s}{\to} FN
\]
to
\[
  AX \stackrel{Ai_X}{\longrightarrow} \widetilde{AN} \stackrel{Ao_Y}{\longleftarrow} AY,
  \quad 1 \stackrel{s}{\to} FN \stackrel{\theta_N}{\to} GAN
  \stackrel{Gm_{AN}^\opp}{\to} G\widetilde{AN}.	
\]
The coherence maps are given by
\[
  \kappa_{X,Y}=          
  \left(
  \begin{aligned}
    \xymatrix{
      & \widetilde{A(X+Y)} \\  
      AX+AY \ar[ur] && A(X+Y) \ar[ul]
    }
  \end{aligned}
  ,
  \qquad
  \begin{aligned}
    \xymatrixrowsep{1.4ex}
    \xymatrix{
      G(\widetilde{A(X+Y)}) \\
      GA(X+Y) \ar[u]_{Gm_{AX+AY}^\opp} \\
      G\varnothing \ar[u]_{G!} \\
      1 \ar[u]_{\gamma_1}
    }
  \end{aligned}
  \right).
\]
\end{proposition}
\begin{proof}
  In the proof of Proposition \ref{prop.corelfunctors} we used the existence of a
  bijective-on-objects, surjective-on-morphisms, composition preserving map
  $\mathrm{Cospan}(\mathcal C) \to\mathrm{Corel}(\mathcal C)$ to prove the
  associativity and other properties of $\mathrm{Corel}(\mathcal C)$. Our proof
  strategy here is entirely analogous.

  Similar to before, we still have not proved that decorated corelations form
  well-defined hypergraph categories. So we begin by merely showing that the map
  $\square$ is composition-preserving, and then that $\square$ respects the
  monoidal and hypergraph structure. We then specialise to the case where the
  domain forms a decorated cospan category that maps surjectively onto a generic
  decorated corelations codomain. Since we know decorated cospan categories are
  well-defined hypergraph categories, we can conclude the same for decorated
  corelations categories, and hence prove that $\square$ is a hypergraph functor. 
  
  We prove that $\square$ preserves composition. Suppose we have decorated corelations
  \[
    X \stackrel{i_X}{\longrightarrow} N \stackrel{o_Y}{\longleftarrow} Y, \quad
    1 \stackrel{s}{\to} FN
    \qquad
    \mbox{and}
    \qquad 
    Y \stackrel{i_Y}{\longrightarrow} M \stackrel{o_Y}{\longleftarrow} Z, \quad
    1 \stackrel{t}{\to} FM
  \]
%  \[
%    \left(
%    \begin{aligned}
%      \xymatrix{
%	& N \\  
%	X \ar[ur]^{i} && Y \ar[ul]_{o}
%      }
%    \end{aligned}
%    ,
%    \qquad
%    \begin{aligned}
%      \xymatrix{
%	FN \\
%	1 \ar[u]_{s}
%      }
%    \end{aligned}
%    \right)
%    \qquad
%    \mbox{and}
%    \qquad
%    \left(
%    \begin{aligned}
%      \xymatrix{
%	& M \\  
%	Y \ar[ur]^{i} && Z \ar[ul]_{o}
%      }
%    \end{aligned}
%    ,
%    \qquad
%    \begin{aligned}
%      \xymatrix{
%	FM \\
%	1 \ar[u]_{s}
%      }
%    \end{aligned}
%    \right).
%  \]
  We know the functor $\square$ preserves composition on the cospan part; this
  is precisely the content of Proposition \ref{prop.corelfunctors}. It remains to
  check that $\square( g \circ f)$ and $\square g \circ \square f$ have
  isomorphic decorations. This is expressed by the commutativity of the
  following diagram:
  \[
    \xymatrixrowsep{1.1pc}
    \xymatrixcolsep{0pc}
    \xymatrix{
      G\widetilde{A(\widetilde{N+_YM})} \ar[rrrrrr]^{Gn} &&&&&&
      G\widetilde{(\widetilde{AN}+_{AY}\widetilde{AM})}\\
      \\
      GA(\widetilde{N+_YM}) \ar[uu]^{Gm^\opp_{A(\widetilde{N+_YM})}} &&&
      \textsc{\tiny($\ast\ast$)} &&& 
      G(\widetilde{AN}+_{AY}\widetilde{AM}) \ar[uu]_{Gm^\opp_{\widetilde{AN}+_{AY}\widetilde{AM}}} \\
      \\
      F(\widetilde{N+_YM}) \ar[uu]^{\theta_{\widetilde{N+_YM}}} & %\textsc{\tiny()} & 
      &
      GA(N+_YM) \ar[uull]_{GAm^\opp_{N+_YM}} && 
      G(AN+_{AY}AM) \ar[ll]_{G\sim} \ar[uurr]^{G(m_{AN}^\opp +_{AY}m_{AM}^\opp)}
      & \textsc{\tiny(c)} & 
      G(\widetilde{AN}+\widetilde{AM}) \ar[uu]_{G[j_{\widetilde{AN}},j_{\widetilde{AM}}]} 
      \\
      &\textsc{\tiny(tn)}&& \textsc{\tiny(a)} 
      \\
      F(N+_YM) \ar[uu]^{Fm_{N+_YM}^\opp} %\ar[uurr]_{\theta_{N+_YM}} & \textsc{\tiny()} & 
      &&
      GA(N+M)\ar[uu]_{GA[j_N,j_M]} && 
      G(AN+AM) \ar[uu]^{G[j_{AN},j_{AM}]} \ar[ll]^{G\alpha_{N,M}}
      \ar[uurr]^{G(m_{AN}^\opp +m_{AM}^\opp)} & \textsc{\tiny(gm)} & 
      G\widetilde{AN} \times G\widetilde{AM}
      \ar[uu]_{\gamma_{\widetilde{AN},\widetilde{AM}}} \\
      \\
      F(N+M) \ar[uu]^{F[j_N,j_M]} \ar[uurr]_{\theta_{N+M}} &&& \textsc{\tiny(tm)} &&& 
      GAN \times GAM \ar[uu]_{Gm_{AN}^\opp \times Gm_{AM}^\opp}
      \ar[uull]^{\gamma_{AN,AM}} \\
      \\
      &&&FN \times FM \ar[uulll]^{\varphi_{N,M}} \ar[uurrr]_{\theta_N \times
      \theta_M} \\\\
      &&&1 \ar[uu]_{\rho_1\circ (s \times t)}
    }
  \]
  This diagram does indeed commute. To check this, first observe that \textsc{(tm)}
  commutes by the monoidality of $\theta$, \textsc{(gm)} commutes by the
  monoidality of $G$, and \textsc{(tn)} commutes by the naturality of $\theta$.
  The remaining three diagrams commute as they are $G$-images of diagrams that
  commute in $\mc C';\mc M'^\opp$. Indeed, \textsc{(a)} commutes since $A$ preserves
  colimits and $G$ is functorial, \textsc{(c)} commutes as it is the $G$-image
  of a pushout square in $\mc C'$, so 
  \[
    \stackrel{m_{AN}^\opp+m_{AM}^\opp}{\longleftarrow} 
    \stackrel{[j_{\widetilde{AN}},j_{\widetilde{AM}}]}{\longrightarrow}
    \quad 
    \textrm{and}
    \quad
    \stackrel{[j_{AN},j_{AM}]}{\longrightarrow}
    \stackrel{m_{AN}^\opp+_{AY}m_{AM}^\opp}{\longleftarrow} 
  \]
  are equal as morphisms of $\mc C';\mc M'^\opp$, and \textsc{($\ast\ast$)} commutes
  as it is the $G$-image of the right-hand subdiagram of $(\ast)$ used to define
  $n$.
  
  It is evident that $\square$ is bijective-on-objects and
  surjective-on-morphisms. This proves the theorem.
\end{proof}

%  \[
%    \xymatrixrowsep{1pc}
%    \xymatrixcolsep{1pc}
%    \xymatrix{
%      &&&& F(N+M) \ar[dd]^{F((m_N +m_M)^\opp)} \ar[rr]^{F[j_N,j_M]} 
%      && F(N+_YM) \ar[dd]^{F((m_N+_Ym_M)^\opp)} \ar[rr]^{F((m_{N+_YM})^\opp)} 
%      && F(\widetilde{N+_YM}) \ar[dd]^{Fn}_\sim \\ 
%      1 \ar[rr]^(.4){(s\times t)\circ\lambda^{-1}} 
%      && FN\times FM \ar[urr]^{\varphi} \ar[dr]_{Fm_N^\opp\times Fm_M^\opp} &
%      \qquad\textrm{\tiny(I)}% && \textrm{\tiny(F)} && \textrm{\tiny(C)}
%      \\ 
%      &&& F\widetilde{N} \times F\widetilde{M} \ar[r]_{\varphi} 
%      & F(\widetilde{N}+\widetilde{M})
%      \ar[rr]_{F[j_{\widetilde{N}},j_{\widetilde{M}}]} 
%      && F(\widetilde{N}+_Y\widetilde{M})
%      \ar[rr]_{F((m_{\widetilde{N}+_Y\widetilde{M}})^\opp)} 
%      && F(\widetilde{\widetilde{N}+_Y\widetilde{M}})
%    }
%  \]
%  Here $n$ is the isomorphism from the proof of Proposition
%  \ref{prop.corelfunctors}.
%  The leftmost square (I) commutes by the naturality of $\varphi$, the central
%  square commutes as it is the $F$-image of a pushout square in $\mathcal C$,
%  and rightmost square commutes as it is the $F$-image of the rightmost square
%  in the commutative diagram (\ref{diag.eparts}) in $\mathcal M^\opp$.
  


\begin{corollary}
  Let $\mathcal C$ be a category with finite colimits, and let $(\mathcal E,
  \mathcal M)$ be a factorisation system on $\mathcal C$. Suppose that we also
  have a lax monoidal functor
  \[
    F: (\mathcal C;\mathcal M^\opp,+) \longrightarrow (\Set, \times).
  \]
  Then we may define a category $F\mathrm{Corel}$ with objects the objects of
  $\mathcal C$ and morphisms isomorpism classes of $F$-decorated corelations.

  Write also $F$ for the restriction of $F$ to the wide subcategory $\mathcal
  C$ of $\mathcal C;\mathcal M^\opp$. We can thus also obtain the category
  $F\mathrm{Cospan}$ of
  $F$-decorated cospans. We moreover have a functor 
  \[
    F\mathrm{Cospan} \to F\mathrm{Corel}
  \]
  which takes each object of $F\mathrm{Cospan}$ to itself as an object of
  $F\mathrm{Corel}$, and each decorated cospan
  \[
    \left(
    \begin{aligned}
      \xymatrix{
	& N \\  
	X \ar[ur]^{i} && Y \ar[ul]_{o}
      }
    \end{aligned}
    ,
    \qquad
    \begin{aligned}
      \xymatrix{
	FN \\
	1 \ar[u]_{s}
      }
    \end{aligned}
    \right)
  \]  
  to its jointly-$\mathcal E$-part
  \[
    \xymatrix{
      & \widetilde{N} \\  
      X \ar[ur]^{i} && Y \ar[ul]_{o}
    }
  \]
  decorated by the composite
  \[
    \xymatrix{
      1 \ar[r]^s & FN \ar[r]^{Fm_N^\opp} & F\widetilde{N}.
    }
  \]
\end{corollary}

\paragraph{Associativity.}

We might supply a more direct proof of the associativity of decorated
corelations through the following.
\begin{lemma}
  Let 
  \[
    \xymatrix{
      & A+_CB \\
      A \ar[ur] && B \ar[ul] \\
      & C \ar[ul] \ar[ur]
    }
  \]
  be a pushout diagram, and let $C \to B$ factor through $\widetilde B \to B$.
  Then
  \[
    \xymatrix{
      & A+_CB \\
      A+B \ar[ur] && A+_C \widetilde B \ar[ul] \\
      & A+\widetilde B \ar[ul] \ar[ur]
    }
  \]
  is a pushout diagram.
\end{lemma}

To take a decoration on $A+B$ to one on $A+_C\widetilde B$ we may either reduce to
the $\mathcal E$-part of $B$ and then pushout over $C$, or pushout over $C$ and
then reduce to the $\mathcal E$ part of $B$. This lemma implies that both
processes result in the same decoration.




\section{All hypergraph categories are decorated corelation categories}
structured categories are algebras over their graphical calculus operad
\cite{SSR}. These equivalences are 2-categorical.

Write $\int$ for the Grothendieck construction. The Grothendieck construction
is

\begin{theorem}
hypergraph categories are symmetric lax monoidal functors cospan to Set.
\[
  \mathrm{HypCat} \cong \int^{\mc O \in \Set}
  \mathrm{SymLaxMon}(\cospan(\FinSet_{\mc O}), \Set)
\]
\end{theorem}
\cite{SV}
\begin{remark}
Not all hypergraph categories are decorated \emph{cospan} categories. To see
this, we can count morphisms. The possible apices and decorations are the same
for all morphisms. So for a decorated cospan category over the prop of finite
sets, the number of morphisms $0 \to 1$ cannot be more than countably many times
those $0 \to 0$ (we just get to choose an element of the apex). But the skeletal
category of vector spaces over $\mathbb{R}$ with monoidal product the tensor
product has $\mathbb{R}$ morphisms $0 \to 0$, and $\mathbb{R}^2$ morphisms $0
\to 1$.
\end{remark}

Decorated corelation categories, however, are more powerful. We can recover all
hypergraph categories by forcing the decorations to be on the coproduct of the
domain and codomain itself. For this we use the isomorphism--morphism
factorisation. Let $\mc H$ be a hypergraph category, and let $\mc C$ be the wide
subcategory of all Frobenius morphisms. Then $\mc H$ can be recovered as the
$(\mc I_{\mc C},\mc C)$-corelations decorated by global sections in $\mc H$.

\subsection{The global sections construction.}

\begin{theorem}
  All hypergraph categories are decorated corelation categories.
\end{theorem}
\begin{proof}
  Let $\mc H$ be a hypergraph category. Without loss of generality we can assume
  $\mc H$ has objects a free monoid under the tensor product; write $\mc O$ for
  a collection of generators for this free monoid, and $\FinSet_{\mc O}$ for
  $\mc O$ labelled finite sets (ie an object is a finite set $X$ together with a
  function $X \to \mc O$). This is a finitely cocomplete category. Equivalent to
  finite lists of objects in $\mc H$. 

  Define the global sections functor 
  \begin{align*}
    G\maps \cospan(\FinSet_{\mc O}) &\longrightarrow \Set \\
    A &\longmapsto \mc H(I,A) \\
    \stackrel{f}{\to}\stackrel{g}{\leftarrow} &\longmapsto \textrm{action of
    Frobenius maps}
  \end{align*}
  This is symmetric lax monoidal functor. Note that $\mc H(I,A)$ depends on the
  order we choose to convert a multiset into an object $A$ of $\mc H$.
  Nonetheless, from any two choices $A$, $A'$ we get a canonical map $A \to A'$.
  This is really that clique in $\Set$.

  Consider the category $\FinSet_{\mc O}$ with an
  (isomorphism,morphism)-factorisation system. We get a decorated corelations
  category with objects multisets of generating objects of $\mc H$, and morphisms $A \to
  B$ trivial corelations $A \to A+B \leftarrow B$ decorated by some morphism
  $s \in \mc H(I,A+B)$. Recall that this decorated corelation is only specified
  up to isomorphism; in the following we always choose representatives such that
  the apex of the jointly-isomorphic cospan is always of the form $A+B$ for
  morphisms $A \to B$.
  
  Given morphisms $s \in \mc H(I,A+B)$ and $t \in \mc H(I,B+C)$ of types $A \to
  B$ and $B \to C$ in $G\mathrm{Corel}$, composition is given by the map
  $H(I,A+B+B+C) \to H(I,A+C)$ arising as the $G$-image of the cospan $A+B+B+C
  \stackrel{[j,j]}\rightarrow A+B+C \stackrel{m}\leftarrow A+C$ where maps come
  from the pushout square
  \[
    \xymatrix{
      && A+C \ar[d] \\
      && A+B+C \\
      & A+B \ar[ur] && B+C \ar[ul] \\
      A \ar[ur] && B \ar[ul]\ar[ur] && C \ar[ul]
    }
  \]
  
  In terms of string diagrams in $\mc H$, this means composing the maps 
  \[
    \tikzset{every path/.style={line width=1.1pt}}
\begin{tikzpicture}
	\begin{pgfonlayer}{nodelayer}
		\node [style=none] (0) at (-0.25, 0.375) {};
		\node [style=none] (1) at (0.5, 0.375) {};
		\node [style=none] (2) at (-0.25, -0.375) {};
		\node [style=none] (3) at (0.5, -0.375) {};
		\node [style=none] (4) at (0.5, 0.25) {};
		\node [style=none] (5) at (0.5, -0.25) {};
		\node [style=none] (6) at (1.25, 0.25) {};
		\node [style=none] (7) at (1.25, -0.25) {};
		\node [style=none] (8) at (0.125, -0) {$s$};
		\node [style=none] (9) at (1.5, 0.25) {$A$};
		\node [style=none] (10) at (1.5, -0.25) {$B$};
	\end{pgfonlayer}
	\begin{pgfonlayer}{edgelayer}
		\draw (0.center) to (1.center);
		\draw (1.center) to (3.center);
		\draw (3.center) to (2.center);
		\draw (2.center) to (0.center);
		\draw (4.center) to (6.center);
		\draw (5.center) to (7.center);
	\end{pgfonlayer}
\end{tikzpicture}
\qquad 
\qquad
\begin{tikzpicture}
	\begin{pgfonlayer}{nodelayer}
		\node [style=none] (0) at (-0.25, 0.375) {};
		\node [style=none] (1) at (0.5, 0.375) {};
		\node [style=none] (2) at (-0.25, -0.375) {};
		\node [style=none] (3) at (0.5, -0.375) {};
		\node [style=none] (4) at (0.5, 0.25) {};
		\node [style=none] (5) at (0.5, -0.25) {};
		\node [style=none] (6) at (1.25, 0.25) {};
		\node [style=none] (7) at (1.25, -0.25) {};
		\node [style=none] (8) at (0.125, -0) {$t$};
		\node [style=none] (9) at (1.5, 0.25) {$B$};
		\node [style=none] (10) at (1.5, -0.25) {$C$};
	\end{pgfonlayer}
	\begin{pgfonlayer}{edgelayer}
		\draw (0.center) to (1.center);
		\draw (1.center) to (3.center);
		\draw (3.center) to (2.center);
		\draw (2.center) to (0.center);
		\draw (4.center) to (6.center);
		\draw (5.center) to (7.center);
	\end{pgfonlayer}
\end{tikzpicture}
  \]
  with the Frobenius map
  \[
    \tikzset{every path/.style={line width=1.1pt}}
    \begin{aligned}
\begin{tikzpicture}
	\begin{pgfonlayer}{nodelayer}
		\node [style=none] (0) at (-0.125, 0.75) {};
		\node [style=none] (1) at (-0.125, 0.25) {};
		\node [style=none] (2) at (-0.125, -0.25) {};
		\node [style=none] (3) at (-0.125, -0.75) {};
		\node [style=none] (4) at (0.5, -0) {};
		\node [style=none] (5) at (1, 0.75) {};
		\node [style=none] (6) at (1, -0.75) {};
		\node [style=none] (8) at (-0.375, 0.75) {$A$};
		\node [style=none] (9) at (-0.375, 0.25) {$B$};
		\node [style=none] (10) at (-0.375, -0.25) {$B$};
		\node [style=none] (11) at (-0.375, -0.75) {$C$};
		\node [style=none] (12) at (1.25, 0.75) {$A$};
		\node [style=none] (13) at (1.25, -0.75) {$C$};
	\end{pgfonlayer}
	\begin{pgfonlayer}{edgelayer}
		\draw (0.center) to (5.center);
		\draw [in=90, out=0, looseness=1.00] (1.center) to (4.center);
		\draw [in=-90, out=0, looseness=1.00] (2.center) to (4.center);
		\draw (3.center) to (6.center);
	\end{pgfonlayer}
\end{tikzpicture}
\end{aligned}
\quad
=
\quad
\begin{aligned}
\begin{tikzpicture}
	\begin{pgfonlayer}{nodelayer}
		\node [style=none] (0) at (-0.125, 0.75) {};
		\node [style=none] (1) at (-0.125, 0.25) {};
		\node [style=none] (2) at (-0.125, -0.25) {};
		\node [style=none] (3) at (-0.125, -0.75) {};
		\node [style=none] (4) at (0.5, -0) {};
		\node [style=none] (5) at (1.25, 0.75) {};
		\node [style=none] (6) at (1.25, -0.75) {};
		\node [style=circ2] (7) at (1, -0) {};
		\node [style=none] (8) at (-0.375, 0.75) {$A$};
		\node [style=none] (9) at (-0.375, 0.25) {$B$};
		\node [style=none] (10) at (-0.375, -0.25) {$B$};
		\node [style=none] (11) at (-0.375, -0.75) {$C$};
		\node [style=none] (12) at (1.5, 0.75) {$A$};
		\node [style=none] (13) at (1.5, -0.75) {$C$};
	\end{pgfonlayer}
	\begin{pgfonlayer}{edgelayer}
		\draw (0.center) to (5.center);
		\draw [in=90, out=0, looseness=1.00] (1.center) to (4.center);
		\draw [in=-90, out=0, looseness=1.00] (2.center) to (4.center);
		\draw (3.center) to (6.center);
		\draw (4.center) to (7.center);
	\end{pgfonlayer}
\end{tikzpicture}
\end{aligned}
  \]
  to get 
  \[
    \tikzset{every path/.style={line width=1.1pt}}
    \begin{aligned}
\begin{tikzpicture}
	\begin{pgfonlayer}{nodelayer}
		\node [style=none] (0) at (-0.25, 0.375) {};
		\node [style=none] (1) at (0.5, 0.375) {};
		\node [style=none] (2) at (-0.25, -0.375) {};
		\node [style=none] (3) at (0.5, -0.375) {};
		\node [style=none] (4) at (0.5, 0.25) {};
		\node [style=none] (5) at (0.5, -0.25) {};
		\node [style=none] (6) at (1.25, 0.25) {};
		\node [style=none] (7) at (1.25, -0.25) {};
		\node [style=none] (8) at (0.125, -0) {$t\circ s$};
		\node [style=none] (9) at (1.5, 0.25) {$A$};
		\node [style=none] (10) at (1.5, -0.25) {$C$};
	\end{pgfonlayer}
	\begin{pgfonlayer}{edgelayer}
		\draw (0.center) to (1.center);
		\draw (1.center) to (3.center);
		\draw (3.center) to (2.center);
		\draw (2.center) to (0.center);
		\draw (4.center) to (6.center);
		\draw (5.center) to (7.center);
	\end{pgfonlayer}
\end{tikzpicture}
\end{aligned}
\quad = 
\quad
\begin{aligned}
\begin{tikzpicture}
	\begin{pgfonlayer}{nodelayer}
		\node [style=none] (0) at (-0.875, 0.875) {};
		\node [style=none] (1) at (-0.125, 0.875) {};
		\node [style=none] (2) at (-0.125, 0.125) {};
		\node [style=none] (3) at (-0.875, 0.125) {};
		\node [style=none] (4) at (-0.875, -0.125) {};
		\node [style=none] (5) at (-0.125, -0.125) {};
		\node [style=none] (6) at (-0.125, -0.875) {};
		\node [style=none] (7) at (-0.875, -0.875) {};
		\node [style=none] (8) at (-0.125, 0.75) {};
		\node [style=none] (9) at (-0.125, 0.25) {};
		\node [style=none] (10) at (-0.125, -0.25) {};
		\node [style=none] (11) at (-0.125, -0.75) {};
		\node [style=none] (12) at (0.5, -0) {};
		\node [style=none] (13) at (0.75, 0.75) {};
		\node [style=none] (14) at (0.75, -0.75) {};
		\node [style=none] (15) at (-0.5, 0.5) {$s$};
		\node [style=none] (16) at (-0.5, -0.5) {$t$};
		\node [style=none] (17) at (1, 0.75) {$A$};
		\node [style=none] (18) at (1, -0.75) {$C$};
	\end{pgfonlayer}
	\begin{pgfonlayer}{edgelayer}
		\draw (0.center) to (1.center);
		\draw (1.center) to (2.center);
		\draw (2.center) to (3.center);
		\draw (3.center) to (0.center);
		\draw (4.center) to (5.center);
		\draw (5.center) to (6.center);
		\draw (6.center) to (7.center);
		\draw (7.center) to (4.center);
		\draw (8.center) to (13.center);
		\draw [in=90, out=0, looseness=1.00] (9.center) to (12.center);
		\draw [in=-90, out=0, looseness=1.00] (10.center) to (12.center);
		\draw (11.center) to (14.center);
	\end{pgfonlayer}
\end{tikzpicture}
\end{aligned}
\]
in $\mc H(I,A+C)$.
  
The monoidal product is given by
  \[
    \tikzset{every path/.style={line width=1.1pt}}
    \begin{aligned}
\begin{tikzpicture}
	\begin{pgfonlayer}{nodelayer}
		\node [style=none] (0) at (-0.25, 0.375) {};
		\node [style=none] (1) at (0.5, 0.375) {};
		\node [style=none] (2) at (-0.25, -0.375) {};
		\node [style=none] (3) at (0.5, -0.375) {};
		\node [style=none] (4) at (0.5, 0.25) {};
		\node [style=none] (5) at (0.5, -0.25) {};
		\node [style=none] (6) at (1.25, 0.25) {};
		\node [style=none] (7) at (1.25, -0.25) {};
		\node [style=none] (8) at (0.125, -0) {$s$};
		\node [style=none] (9) at (1.5, 0.25) {$A$};
		\node [style=none] (10) at (1.5, -0.25) {$B$};
	\end{pgfonlayer}
	\begin{pgfonlayer}{edgelayer}
		\draw (0.center) to (1.center);
		\draw (1.center) to (3.center);
		\draw (3.center) to (2.center);
		\draw (2.center) to (0.center);
		\draw (4.center) to (6.center);
		\draw (5.center) to (7.center);
	\end{pgfonlayer}
\end{tikzpicture}
\end{aligned}
\quad 
+
\quad
\begin{aligned}
\begin{tikzpicture}
	\begin{pgfonlayer}{nodelayer}
		\node [style=none] (0) at (-0.25, 0.375) {};
		\node [style=none] (1) at (0.5, 0.375) {};
		\node [style=none] (2) at (-0.25, -0.375) {};
		\node [style=none] (3) at (0.5, -0.375) {};
		\node [style=none] (4) at (0.5, 0.25) {};
		\node [style=none] (5) at (0.5, -0.25) {};
		\node [style=none] (6) at (1.25, 0.25) {};
		\node [style=none] (7) at (1.25, -0.25) {};
		\node [style=none] (8) at (0.125, -0) {$t$};
		\node [style=none] (9) at (1.5, 0.25) {$C$};
		\node [style=none] (10) at (1.5, -0.25) {$D$};
	\end{pgfonlayer}
	\begin{pgfonlayer}{edgelayer}
		\draw (0.center) to (1.center);
		\draw (1.center) to (3.center);
		\draw (3.center) to (2.center);
		\draw (2.center) to (0.center);
		\draw (4.center) to (6.center);
		\draw (5.center) to (7.center);
	\end{pgfonlayer}
\end{tikzpicture}
\end{aligned}
\quad = \quad 
\begin{aligned}
\begin{tikzpicture}
	\begin{pgfonlayer}{nodelayer}
		\node [style=none] (0) at (-0.875, 0.875) {};
		\node [style=none] (1) at (-0.125, 0.875) {};
		\node [style=none] (2) at (-0.125, 0.125) {};
		\node [style=none] (3) at (-0.875, 0.125) {};
		\node [style=none] (4) at (-0.875, -0.125) {};
		\node [style=none] (5) at (-0.125, -0.125) {};
		\node [style=none] (6) at (-0.125, -0.875) {};
		\node [style=none] (7) at (-0.875, -0.875) {};
		\node [style=none] (8) at (-0.125, 0.75) {};
		\node [style=none] (9) at (-0.125, 0.25) {};
		\node [style=none] (10) at (-0.125, -0.25) {};
		\node [style=none] (11) at (-0.125, -0.75) {};
		\node [style=none] (12) at (1, 0.75) {};
		\node [style=none] (13) at (1, 0.25) {};
		\node [style=none] (14) at (1, -0.25) {};
		\node [style=none] (15) at (1, -0.75) {};
		\node [style=none] (16) at (-0.5, 0.5) {$s$};
		\node [style=none] (17) at (-0.5, -0.5) {$t$};
		\node [style=none] (18) at (1.25, 0.75) {$A$};
		\node [style=none] (19) at (1.25, 0.25) {$C$};
		\node [style=none] (20) at (1.25, -0.25) {$B$};
		\node [style=none] (21) at (1.25, -0.75) {$D$};
	\end{pgfonlayer}
	\begin{pgfonlayer}{edgelayer}
		\draw (0.center) to (1.center);
		\draw (1.center) to (2.center);
		\draw (2.center) to (3.center);
		\draw (3.center) to (0.center);
		\draw (4.center) to (5.center);
		\draw (5.center) to (6.center);
		\draw (6.center) to (7.center);
		\draw (7.center) to (4.center);
		\draw (8.center) to (12.center);
		\draw [in=180, out=0, looseness=1.00] (9.center) to (14.center);
		\draw [in=180, out=0, looseness=1.00] (10.center) to (13.center);
		\draw (11.center) to (15.center);
	\end{pgfonlayer}
\end{tikzpicture}
\end{aligned}
\]
recalling that we have chosen to represent the equivalence class of corelations
$A+C \to B+D$ with the apex $A+C+B+D$.

  Taking a hint from the compact closed structure, it is straightforward to
  construct a pair of inverse hypergraph functors between $G\mathrm{Corel}$ to $\mc H$.
  Indeed, $G\mathrm{Corel}$ is constructed to have the same collection of
  objects as $\mc H$; simply have the functors be the `identity' on objects. On
  morphisms, we take $f: A \to B$ in $\mc H$ to its `name' $\hat f: I \to A+B$
  as a morphism of $G\mathrm{Corel}$. This is a bijection. 
  
  To check it is composition and monoidal product preserving, we can easily use
  diagrammatic reasoning. For example

  Therefore $\mc H$ is isomorphic
  as a hypergraph category to $G\mathrm{Corel}$. 
\end{proof}

\begin{theorem}
  All hypergraph functors are decorated corelation functors.
\end{theorem}
\begin{proof}
  Let $\mc H$ and $\mc H'$ be hypergraph categories, and $T\maps \mc H \to \mc
  H'$ be a hypergraph functor. By the above theorem, there exist symmetric lax
  monoidal functors 
  \[
    G\maps \cospan(\FinSet_{\mc O_{\mc H}}) \to \Set 
  \]
  and 
  \[
    G'\maps \cospan(\FinSet_{\mc O_{\mc H'}}) \to \Set 
  \]
  such that $\mc H = G\mathrm{Corel}$ and $\mc H' = G'\mathrm{Corel}$.
  Furthermore, define a functor $A\maps \FinSet_{\mc O_{\mc H}} \to \FinSet_{\mc
    O_{\mc H'}}$ taking $N \to \mc O_{\mc H}$ to $N \to \mc O_{\mc H} \to \mc
  O_{\mc H'}$, where the second map is that by the functor $T$ on objects of
  $\mc H$. We claim this is a well-defined colimit-preserving functor and show
  that $T$ can be constructed from a monoidal natural transformation between $G$
  and $G'\circ A$.
\end{proof}

Compare with Spivak Vagner construction.

\subsection{Examples.}
We give some examples reproducing hypergraph categories as decorated corelations
categories.

\begin{example}
  Example: empty decorations and equivalence relations. 
  
  Consider the hypergraph category $\cospan(\FinSet)$. This is the simplest
  hypergraph category: it is free hypergraph category on the one object discrete
  category. We show how to recover it as a decorated corelation category.

  As per Example \ref{ex.undeccorel}, $\cospan(\FinSet)$ is the hypergraph
  category of undecorated (morphism-isomorphism)-corelations in $\FinSet$. It is
  also the partition-decorated (isomorphism-morphism)-corelations in $\FinSet$.   
  
  First, the global sections functor $G\maps \cospan(\FinSet) \to \Set$ takes
  each finite set $X$ to the set of (equivalence classes of) cospans $0 \to D
  \leftarrow X$; that is, to the set of functions $X \to D$ where a unique $D$
  is chosen for each finite cardinality. Given a cospan $X \stackrel{f}\to N
  \stackrel{g}\leftarrow Y$, its image under the global sections functor maps a
  function $a\maps X \to D$, to the function $Y \to N+_YD$ given by
  \[
    \xymatrix{
      & X \ar[r]^d \ar[d]_f & D \ar[d] \\
      Y \ar[r]^g  & N \ar[r] & N+_YD
    }
  \]
  where the square is a pushout square.

  The coherence maps $\gamma_1\maps 1 \to G\varnothing$ map the unique element
  of $1$ to the unique function $!\maps \varnothing \to \varnothing$, and
  $\gamma_{X,Y}$ maps a pair of functions $a\maps X \to D$, $b\maps Y \to E$ to
  $a+b\maps X+Y \to D+E$. This is a symmetric lax monoidal functor. We use this
  functor to decorate cospans in $\FinSet$. 

  A decorated cospan in $\FinSet$ with respect to this functor is a cospan $X
  \to N \leftarrow Y$ in $\FinSet$ together with a function $N \to D$ for some
  finite set. Using the isomorphism--morphism factorisation, a decorated
  corelation (a morphism in $G\mathrm{Corel}$) is a
  cospan $X \to X+Y \leftarrow Y$, together with a function $X+Y \to D$.
  This is the same as a cospan. 

  The hypergraph structure is given by the decoration $X+X \to X$ etc, as the
  shift from cospans to corelations takes the `factored out part' and puts it
  into the decoration. At this point the morphisms are specified entirely by
  their decoration.

  It is straightforward to show the two categories are isomorphic. Note that the
  identity on $\FinSet$ maps $\mc I_\FinSet$ into $\FinSet$, and so extends to
  morphism $\FinSet \to \cospan(\FinSet)$. We can define a monoidal natural
  transformation $1(X) =1 \stackrel{\theta_X}\to GX = \{X \to D\}$ mapping the
  unique element to the identity function $1_X\maps X \to X$.
  \[
    \xymatrixrowsep{2ex}
    \xymatrix{
      \FinSet = \FinSet; \mc I_{\FinSet}^\opp \ar[dd]_{\iota} \ar[drr]^1  \\
      &\twocell \omit{_\:\theta}& \Set \\
      \cospan(\FinSet) = \FinSet; \FinSet^\opp \ar[urr]_{G} 
    }
  \]
  It is easy to verify that this is a monoidal natural transformation. This
  gives the hypergraph functor we expect, mapping the undecorated cospan $X \to
  N \leftarrow Y$ to $X \to X+Y \leftarrow Y$ decorated by $X+Y \to N$. 

  Like so many examples before, it is easy to verify this is a full, faithful,
  bijective-on-objects hypergraph functor.
\end{example}

\begin{example}
  The previous example extends to any finitely cocomplete category $\mc C$: the
  hypergraph category $\cospan(\mc C)$ can always be constructed as (i)
  trivially decorated $(\mc C, \mc I_{\mc C})$-corelations, or (ii) $(\mc I_{\mc
  C}, \mc C)$-corelations decorated by equivalence classes of morphisms with
  domain the apex of the corelation, and moreover the isomorphism of these
  hypergraph categories is given by a monoidal natural transformation between
  the decorating functors.

  More general still, a category of trivially decorated $(\mc E, \mc
  M)$-corelations in $\mc C$ can always be constructed also as $(\mc I_{\mc C},
  \mc C)$-corelations decorated by equivalence classes of morphisms in $\mc E$
  with domain the apex of the corelation, and the isomorphism of these
  hypergraph categories is given by a monoidal natural transformation between
  the decorating functors.

  Most general, the theorem implies that any category of decorated
  corelations can be constructed also as $(\mc I_{\mc C}, \mc C)$-corelations
  decorated by codomain decorated morphisms in $\mc E$. The latter are specified
  by a functor $\cospan(\mc C) \to \Set$, as in the theorem.

  The latter form is good for constructing functors that have image the
  corelation category.
\end{example}

Note that with decorated corelations we have a method of constructing hypergraph
categories from other hypergraph categories. This is black boxing. 

\section{Examples} \label{sec:excor}
\subsection{Path integrals and matrices}

Let $R$ be a commutative ring. Take functor
\begin{align*}
  R^{(-)}: (\mathrm{FinSet}^\opp,+) &\longrightarrow (\Set,\times) \\
  X &\longmapsto R^X
\end{align*}

Then $R^{(-)}\mathrm{Cospan}$ is path integrals, $R^{(-)}\mathrm{Corel}$ is
matrices over $R$. 

Many aspects of this example are `atypical', regarding the intuition we have
been working towards.
Note that the monoidal product here is the tensor product of
matrices, not the biproduct. Indeed, there is no special commutative Frobenius
algebra in $\Vect$ if we use the biproduct, but if we use the tensor product
then these correspond to orthonormal bases (Vicary). The comultiplication is the
diagonal map, multiplication is codiagonal. unit produces basis.

We note that you could take decorations here in the category $R\mathrm{Mod}$ of
$R$-modules. While Proposition \ref{prop.setdecorations} shows that the
resulting decorated cospans category would be isomorphic, this hints at an
enriched version of the theory.

\subsection{Two constructions for linear relations}

We saw earlier that linear relations are epi-mono corelations in $\Vect$. The
hypergraph structure is given by addition. We show how to recover this in
another construction. We also get a hypergraph functor between them. This is
very useful for compositional linear relations semantics of diagrams.

We can also construct linear relations in $\Vect^\opp$.

\begin{align*}
  \maps\mathrm{Cospan}(\mathrm{FinSet}) &\longrightarrow \mathrm{Set} \\
  X &\longmapsto \{\mathrm{subspaces of }k^X\} \\
  f: X \to Y &\longmapsto L \mapsto \{v \mid v\circ f \in L\} \\
  f^\opp: X \to Y &\longmapsto L \mapsto \{v = u \circ f \mid u \in L\}
\end{align*}

Then $\mathrm{Cospan}$ is cospans decorated by subspaces, and $\mathrm{Corel}$
is linear relations. This is important for circuits work \cite{BF,BSZ}.

\subsection{Automata}
This construction comes immediately from Walters et al. Automata are alphabet
labelled graphs. There is a decorated cospan functor to categories enriched over
languages, and this factors nicely to get a decorated corelation category with
morphisms languages recognised between points in domain and codomain.




