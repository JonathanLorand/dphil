\chapter{Corelations: a tool for black boxing} \label{ch.corelations}

In this chapter we develop the theory of corelations.

\section{The idea of black boxing}

Thus far we have argued that network languages should be modelled using
hypergraph categories, and shown that cospans provide a good language for
talking about interconnection. We then developed the theory of decorated
cospans, which allows us to take diagrams, mark `inputs' and `outputs' using
cospans, and then compose these diagrams using pushouts. This turns a notion of
closed system into a notion of open one.

A serious limitation of using cospans alone, however, is that cospans
indiscrimately accumulate information. If we compose twenty graphs in the
category $\mathrm{GraphCospan}$, we create an open graph containing all twenty
of those graphs as subgraphs, disjoint but for a finite collection of vertices.
This is often unnecessary and inefficient.

For example, consider the following two circuits, comprised only of perfectly
conductive wires:
\[
    \tikzset{every path/.style={line width=1.1pt}}
\begin{tikzpicture}
	\begin{pgfonlayer}{nodelayer}
		\node [style=dot] (0) at (-2.5, 1.5) {};
		\node [style=dot] (1) at (-2.5, -0) {};
		\node [style=dot] (2) at (-2.5, -1.5) {};
		\node [style=none] (3) at (-2, -2) {};
		\node [style=none] (4) at (1.5, 2) {};
		\node [style=none] (5) at (-2, -0) {};
		\node [style=none] (6) at (-2, -1.5) {};
		\node [style=none] (7) at (-2, 1.5) {};
		\node [style=dot] (8) at (-2.5, -0.75) {};
		\node [style=none] (9) at (-2, -0.75) {};
		\node [style=dot] (10) at (-2.5, 0.75) {};
		\node [style=none] (11) at (-2, 0.75) {};
		\node [style=dot] (12) at (2, -1.5) {};
		\node [style=none] (13) at (1.5, -1.5) {};
		\node [style=dot] (14) at (2, -0.75) {};
		\node [style=none] (15) at (1.5, 1.5) {};
		\node [style=dot] (16) at (2, 1.5) {};
		\node [style=none] (17) at (1.5, 0.75) {};
		\node [style=dot] (18) at (2, -0) {};
		\node [style=dot] (19) at (2, 0.75) {};
		\node [style=none] (20) at (1.5, -0) {};
		\node [style=none] (21) at (1.5, -0.75) {};
		\node [style=none] (22) at (-0.75, 1.25) {};
		\node [style=none] (23) at (0, 0.25) {};
		\node [style=none] (24) at (-1, -0.25) {};
		\node [style=none] (25) at (-0.5, -0.75) {};
		\node [style=none] (26) at (-1.25, -1.25) {};
		\node [style=none] (27) at (0, 0.75) {};
		\node [style=none] (28) at (0.5, 1.25) {};
		\node [style=none] (29) at (-1.25, 0.5) {};
		\node [style=none] (30) at (-1.25, 1.5) {};
		\node [style=none] (31) at (0.5, -1.5) {};
		\node [style=none] (32) at (-1.25, 1) {};
		\node [style=none] (33) at (-1.25, -0.75) {};
		\node [style=none] (34) at (-1.75, -0.25) {};
		\node [style=none] (35) at (0, -1.25) {};
		\node [style=none] (36) at (0, -0.25) {};
		\node [style=none] (37) at (1, -1.5) {};
		\node [style=none] (38) at (-1.75, 1.25) {};
		\node [style=none] (39) at (0.5, 0.25) {};
		\node [style=none] (40) at (1, 0.5) {};
		\node [style=none] (41) at (-0.5, 1.75) {};
		\node [style=none] (42) at (0.75, 1.75) {};
		\node [style=none] (43) at (0.25, -1.25) {};
		\node [style=none] (44) at (1.25, -1) {};
		\node [style=none] (45) at (-1.75, -1.75) {};
		\node [style=none] (46) at (-1, -1.75) {};
		\node [style=none] (47) at (1.25, -1.7) {};
		\node [style=none] (48) at (-0.5, -1.5) {};
	\end{pgfonlayer}
	\begin{pgfonlayer}{edgelayer}
		\draw (0.center) to (7.center);
		\draw (1.center) to (5.center);
		\draw (2.center) to (6.center);
		\draw (8) to (9.center);
		\draw (10) to (11.center);
		\draw (16) to (15.center);
		\draw (18) to (20.center);
		\draw (12) to (13.center);
		\draw (14) to (21.center);
		\draw (19) to (17.center);
		\draw (6.center) to (26.center);
		\draw [in=180, out=-11, looseness=2.00] (26.center) to (31.center);
		\draw [in=-165, out=15, looseness=1.25] (26.center) to (25.center);
		\draw [in=1, out=180, looseness=1.00] (24.center) to (5.center);
		\draw [in=-105, out=0, looseness=1.00] (24.center) to (23.center);
		\draw [in=108, out=-135, looseness=2.25] (29.center) to (24.center);
		\draw [in=-169, out=-30, looseness=1.25] (29.center) to (27.center);
		\draw [in=165, out=60, looseness=1.25] (23.center) to (17.center);
		\draw (27.center) to (23.center);
		\draw [bend right, looseness=1.00] (23.center) to (20.center);
		\draw (28.center) to (27.center);
		\draw [in=150, out=0, looseness=1.25] (22.center) to (28.center);
		\draw (28.center) to (15.center);
		\draw (22.center) to (30.center);
		\draw (30.center) to (7.center);
		\draw (11.center) to (32.center);
		\draw (38.center) to (11.center);
		\draw (38.center) to (32.center);
		\draw [in=-78, out=-48, looseness=1.00] (39.center) to (40.center);
		\draw (31.center) to (37.center);
		\draw (37.center) to (13.center);
		\draw (9.center) to (34.center);
		\draw (34.center) to (33.center);
		\draw [in=-158, out=22, looseness=1.00] (33.center) to (36.center);
		\draw (36.center) to (21.center);
		\draw [in=86, out=15, looseness=2.00] (25.center) to (35.center);
		\draw (24.center) to (27.center);
		\draw [in=56, out=-60, looseness=1.00] (22.center) to (29.center);
		\draw (17.center) to (20.center);
		\draw [bend left=75, looseness=1.00] (39.center) to (40.center);
		\draw (41.center) to (42.center);
		\draw (45.center) to (46.center);
		\draw (46.center) to (48.center);
		\draw (46.center) to (47.center);
		\draw (43.center) to (44.center);
		\draw (28.center) to (17.center);
	\end{pgfonlayer}
	\begin{pgfonlayer}{background}
	  \filldraw [fill=black!5!white, draw=black!40!white] (3) rectangle (4);
	\end{pgfonlayer}
\end{tikzpicture}
\qquad\qquad
\begin{tikzpicture}
	\begin{pgfonlayer}{nodelayer}
		\node [style=dot] (0) at (-2.5, 1.5) {};
		\node [style=dot] (1) at (-2.5, -0) {};
		\node [style=dot] (2) at (-2.5, -1.5) {};
		\node [style=none] (3) at (-2, -2) {};
		\node [style=none] (4) at (1.5, 2) {};
		\node [style=none] (5) at (-2, -0) {};
		\node [style=none] (6) at (-2, -1.5) {};
		\node [style=none] (7) at (-2, 1.5) {};
		\node [style=dot] (8) at (-2.5, -0.75) {};
		\node [style=none] (9) at (-2, -0.75) {};
		\node [style=dot] (10) at (-2.5, 0.75) {};
		\node [style=none] (11) at (-2, 0.75) {};
		\node [style=dot] (12) at (2, -1.5) {};
		\node [style=none] (13) at (1.5, -1.5) {};
		\node [style=dot] (14) at (2, -0.75) {};
		\node [style=none] (15) at (1.5, 1.5) {};
		\node [style=dot] (16) at (2, 1.5) {};
		\node [style=none] (17) at (1.5, 0.75) {};
		\node [style=dot] (18) at (2, -0) {};
		\node [style=dot] (19) at (2, 0.75) {};
		\node [style=none] (20) at (1.5, -0) {};
		\node [style=none] (21) at (1.5, -0.75) {};
		\node [style=none] (22) at (-0.25, 0.75) {};
	\end{pgfonlayer}
	\begin{pgfonlayer}{edgelayer}
		\draw (0.center) to (7.center);
		\draw (1.center) to (5.center);
		\draw (2.center) to (6.center);
		\draw (8) to (9.center);
		\draw (10) to (11.center);
		\draw (16) to (15.center);
		\draw (18) to (20.center);
		\draw (12) to (13.center);
		\draw (14) to (21.center);
		\draw (19) to (17.center);
		\draw (9.center) to (21.center);
		\draw (6.center) to (13.center);
		\draw (7.center) to (22.center);
		\draw (5.center) to (22.center);
		\draw (22.center) to (15.center);
		\draw (22.center) to (17.center);
		\draw (22.center) to (20.center);
	\end{pgfonlayer}
	\begin{pgfonlayer}{background}
	  \filldraw [fill=black!5!white, draw=black!40!white] (3) rectangle (4);
	\end{pgfonlayer}
\end{tikzpicture}
\]
We interact with these circuits by connecting electric circuits and other
devices to the terminals. This allows us to probe, for example, whether one
terminal is connected to another. These circuits are equivalent in the following
sense: if these circuits were encased, but for their terminals, in a black box
\[
    \tikzset{every path/.style={line width=1.1pt}}
    \begin{tikzpicture}
	\begin{pgfonlayer}{nodelayer}
		\node [style=dot] (0) at (-2.5, 1.5) {};
		\node [style=dot] (1) at (-2.5, -0) {};
		\node [style=dot] (2) at (-2.5, -1.5) {};
		\node [style=none] (3) at (-2, -2) {};
		\node [style=none] (4) at (1.5, 2) {};
		\node [style=none] (5) at (-2, -0) {};
		\node [style=none] (6) at (-2, -1.5) {};
		\node [style=none] (7) at (-2, 1.5) {};
		\node [style=dot] (8) at (-2.5, -0.75) {};
		\node [style=none] (9) at (-2, -0.75) {};
		\node [style=dot] (10) at (-2.5, 0.75) {};
		\node [style=none] (11) at (-2, 0.75) {};
		\node [style=dot] (12) at (2, -1.5) {};
		\node [style=none] (13) at (1.5, -1.5) {};
		\node [style=dot] (14) at (2, -0.75) {};
		\node [style=none] (15) at (1.5, 1.5) {};
		\node [style=dot] (16) at (2, 1.5) {};
		\node [style=none] (17) at (1.5, 0.75) {};
		\node [style=dot] (18) at (2, -0) {};
		\node [style=dot] (19) at (2, 0.75) {};
		\node [style=none] (20) at (1.5, -0) {};
		\node [style=none] (21) at (1.5, -0.75) {};
	\end{pgfonlayer}
	\begin{pgfonlayer}{edgelayer}
	  \filldraw [fill=black!80!white] (3) rectangle (4);
		\draw (0.center) to (7.center);
		\draw (1.center) to (5.center);
		\draw (2.center) to (6.center);
		\draw (8) to (9.center);
		\draw (10) to (11.center);
		\draw (16) to (15.center);
		\draw (18) to (20.center);
		\draw (12) to (13.center);
		\draw (14) to (21.center);
		\draw (19) to (17.center);
	\end{pgfonlayer}
\end{tikzpicture}
  ,
\]
we would be unable to distinguish them through our electrical investigations. In
this sense these circuits are equivalent.

In this chapter and the next we pursue a notion of composition that reflects
this, discarding extraneous information as we compose our systems. Next chapter
we will extend these ideas to allow decorations and so a richer notion of
system; for now we return to an undecorated situation, so that we may focus on
the compositional aspects.

\section{Corelations}

Given sets $X$, $Y$, a relation $X \leadsto Y$ is a subset of the product $X
\times Y$. More abstractly, we might say a relation is a jointly-monic span in
the category of sets (or an isomorphism class thereof). We generalise the dual
concept.

\subsection{Factorisation systems.}
The relevant properties of jointly-monic spans come from the fact that
monomorphisms form one half of a factorisation system. A factorisation system
allows any morphism in a category to be factored into the composite of two
morphisms in a coherent way.

\begin{definition}
  A \define{factorisation system} $(\mathcal E,\mathcal M)$ in a category
  $\mathcal C$ comprises subcategories $\mathcal E$, $\mathcal M$ of $\mathcal
  C$ such that
  \begin{enumerate}[(i)]
    \item $\mathcal E$ and $\mathcal M$ contain all isomorphisms of $\mathcal
      C$.
    \item  every morphism $f \in \mathcal C$ admits a factorisation $f=m \circ
      e$, $e \in \mathcal E$, $m \in \mathcal M$.
\item given morphisms $f,f'$, with factorisations $f = m \circ e$, $f' = m' \circ
  e'$ of the above sort, for every $u$, $v$ such that the square
  \[
    \xymatrixcolsep{3pc}
    \xymatrixrowsep{3pc}
    \xymatrix{
       \ar[r]^f \ar[d]_u &  \ar[d]^v \\
       \ar[r]_{f'} & 
    }
  \]
  commutes, there exists a unique morphism $s$ such that
  \[
    \xymatrixcolsep{3pc}
    \xymatrixrowsep{3pc}
    \xymatrix{
      \ar[r]^e \ar[d]_u & \ar[r]^m \ar@{.>}[d]^{\exists! s} &  \ar[d]^v \\
       \ar[r]_{e'}& \ar[r]_{m'} & 
    }
  \]
  commutes.
  \end{enumerate}
\end{definition}

\begin{examples}
  We introduce some factorisation systems of central importance in what follows.
  \begin{itemize}
    \item Write $\mathcal I_{\mathcal C}$ for the wide subcategory of
      $\mathcal C$ containing exactly the isomorphisms of $\mathcal C$. Then
      $(\mathcal I_{\mathcal C}, \mathcal C)$ and $(\mathcal C, \mathcal
      I_{\mathcal C})$ are both factorisation systems in $\mathcal C$. 
    
    \item The prototypical example of a factorisation system is the epi-mono
      factorisation system $(\mathrm{Sur},\mathrm{Inj})$ in $\Set$. This follows
      from a more general fact, true in any category, that if every arrow can be
      factorised as an epi followed by a split mono, then this results in a
      factorisation system.  The only non-trivial part to check is the
      uniqueness condition: given epis $e_1,e_2$, split monos $m_1,m_2$, and
      commutative diagram
      \[
	\xymatrixcolsep{3pc}
	\xymatrixrowsep{3pc}
	\xymatrix{
	  \ar[r]^{e_1} \ar[d]_u & \ar[r]^{m_1} \ar@{.>}[d]^{\exists! t} &
	  \ar[d]^v \\
	  \ar[r]_{e_2}& \ar[r]_{m_2} & 
	}
      \]
      we must show that there is a unique $t$ that makes the diagram commute.
      Indeed let $t= m_2'vm_1$ where $m_2'$ satisfies $m_2'm_2=id$. 
      To see that the right square commutes, observe
      \[
	m_2 t e_1 =  m_2 m_2' v m_1 e_1 = m_2 m_2' m_2 e_2 u = m_2 e_2 u = v m_1 e_1
      \]
      and since $e_1$ is epi we have $m_2 t = v m_1$. For the left square,
      \[
	t e_1 = m_2' v m_1 e_1 = m_2' m_2 e_2 u = e_2 u.
      \] 
      Uniqueness is immediate, since, $e_1$ is epi and $m_2$ is mono. 
\end{itemize}
\end{examples}

See \cite[\textsection 14]{AHS} for more details.

\begin{definition}
Call a factorisation system $(\mc E,\mc M)$ in a monoidal category $(\mc C,\ot)$
a \define{monoidal factorisation system} if $(\mc E,\ot)$ is a monoidal
category.
\end{definition}

One might wonder why $\mc M$ does not appear in the above definition. To give a
touch more intuition for this definition, we quote a theorem of Ambler. Recall a
symmetric monoidal closed category is one in which each functor $- \ot X$ has a
specified right adjoint $[X,-]$. See Ambler for proof and further details
\cite[Lemma 5.2.2]{Am}.
\begin{proposition}
  Let $(\mc E,\mc M)$ be a factorisation system in a symmetric monoidal
  closed category $(\mc C,\ot)$. Then the following are equivalent:
  \begin{enumerate}[(i)]
    \item $(\mc E,\mc M)$ is a monoidal factorisation system.
    \item $\mc E$ is closed under $- \ot X$ for all $X \in \mc C$.
    \item $\mc M$ is closed under $[X,-]$ for all $X \in \mc C$.
\end{enumerate}
\end{proposition}

In a category with finite coproducts every factorisation system is a monoidal
factorisation system for the coproduct.
\begin{lemma} \label{lem.monfact}
  Let $\mc C$ be a category with finite coproducts, and let $(\mc E, \mc M)$ be a
  factorisation system on $\mc C$. Then $(\mc E,+)$ is a symmetric monoidal
  category.
\end{lemma}
\begin{proof}
  The only thing to check is that $\mc E$ is closed under $+$. That is, given
  $f\maps A \to B$ and $g\maps C \to D$ in $\mc E$, we wish to show that
  $f+g\maps A+C \to B+D$, defined in $\mc C$, is also a morphism in $\mc E$. 

  Let $f+g$ have factorisation $A+C \stackrel{e}\longrightarrow \overline{B+D}
  \stackrel{m}\longrightarrow B+D$, where $e \in \mc E$ and $m \in \mc
  M$. We will prove that $m$ is an isomorphism. To construct an inverse, recall
  that by definition, as $f$ and $g$ lie in $\mc E$, there exist morphisms
  $x\maps B \to \overline{B+D}$ and $y\maps D \to \overline{B+D}$ such that
  \[ \label{eq.coreltensor}
    \xymatrixcolsep{2pc}
    \xymatrixrowsep{2pc}
    \xymatrix{
      A \ar[r]^f \ar[d] & B \ar@{=}[r] \ar@{.>}[d]^x & B
      \ar[d] \\
      A+C \ar[r]_{e}&\overline{B+D} \ar[r]_{m} & B+D
    }
    \qquad \mbox{and} \qquad
    \xymatrixcolsep{2pc}
    \xymatrixrowsep{2pc}
    \xymatrix{
      C \ar[r]^g \ar[d] & D \ar@{=}[r] \ar@{.>}[d]^y & D
      \ar[d] \\
      A+C \ar[r]_{e}&\overline{B+D} \ar[r]_{m} & B+D
    }
    \tag{1}
  \]
  The copairing $[x,y]$ is an inverse to $m$. 
  
  Indeed, taking the coproduct of the top rows of the two diagrams above and the
  copairings of the vertical maps gives the commutative diagram
  \[
    \xymatrix{
      A+C \ar[r]^{f+g} \ar@{=}[d] & B+D \ar@{=}[r] \ar[d]_{[x,y]} & B+D \ar@{=}[d] \\
      A+C \ar[r]^{e} & \overline{B+D} \ar[r]^{m} & B+D
    }
  \]
  Reading the right-hand square immediately gives $m \circ [x,y] =1$.
  
  Conversely, to see that $[x,y] \circ m = 1$, remember that by definition $f+g
  = m \circ e$. So the left-hand square above implies that
  \[
    \xymatrixcolsep{2pc}
    \xymatrixrowsep{2pc}
    \xymatrix{
      A+C \ar[r]^e \ar@{=}[d] & \overline{B+D} \ar[d]^{[x,y] \circ m} \\
      A+C \ar[r]_{e}&\overline{B+D} 
    }
  \]
  commutes. But by the universal property of factorisation systems, there is a
  unique map $\overline{B+D} \to \overline{B+D}$ such that this diagram
  commutes, and clearly the identity map also suffices. Thus $[x,y] \circ m =
  1$.
\end{proof}

\subsection{Corelations}
Relations, then, may be generalised as spans such that the span maps `jointly'
belong to some class $\mc M$ of an $(\mc E,\mc M)$-factorisation system. We
define corelations in the dual manner.

\begin{definition}
  Let $\mathcal C$ be a category with finite colimits, and let $(\mathcal E,
  \mathcal M)$ be a factorisation system on $\mathcal C$. An \define{$(\mathcal
  E,\mathcal M)$-corelation} $X \leadsto Y$ is a cospan $X
  \stackrel{i}\longrightarrow N \stackrel{o}\longleftarrow Y$ such that the
  copairing $[i,o]\maps X+Y \to N$ lies in $\mathcal E$.
\end{definition}

When the factorisation system is clear from context, we simply call $(\mathcal
E,\mathcal M)$-corelations: `corelations'.

We also say that a cospan $X \stackrel{i}\longrightarrow N
\stackrel{o}\longleftarrow Y$ with the property that the copairing $[i,o]\maps X+Y \to N$ lies in $\mathcal E$ is \define{jointly $\mathcal E$-like}. Note that
if a cospan is jointly $\mc E$-like then so are all isomorphic cospans. Thus the
property of being a corelation is closed under isomorphism of cospans, and we
again are often lazy with out language, referring to both jointly $\mc E$-like
cospans and their isomorphism classes as corelations. 

If $f\maps A \to N$ is a morphism with factorisation $f = m \circ e$, write
$\overline N$ for the object such that $e\maps A \to \overline N$ and $m\maps
\overline N \to N$. Now, given a cospan $X \stackrel{i_X}{\longrightarrow} N
\stackrel{o_Y}{\longleftarrow} Y$, we may use the factorisation system to write
the copairing $[i_X,o_Y]\maps X+Y \to N$ as
\[
  X+Y \stackrel{e}{\longrightarrow} \overline{N} \stackrel{m}{\longrightarrow}
  N.
\]
From the universal property of the coproduct, we also have maps $\iota_X\maps X
\to X+Y$ and $\iota_Y\maps Y \to X+Y$. We then call the corelation 
\[
  X \stackrel{e\circ \iota_X}{\longrightarrow} \overline{N} \stackrel{e \circ
  \iota_Y}{\longleftarrow} Y
\]
the \define{$\mathcal E$-part} of the above cospan. On occasion we will also
write $e\maps X+Y \to \overline N$ for the same corelation.

\begin{examples} \label{ex.corels}
  Many examples of corelations are already familiar.
  \begin{itemize}
    \item For the morphism-isomorphism factorisation system $(\mc C,\mc I_{\mc
      C})$, corelations are just cospans.
    \item For the isomorphism-morphism factorisation $(\mc I_{\mc C}, \mc C)$,
      jointly $\mc I_{\mc C}$-like cospans $X \to Y$ are simply isomorphisms
      $X+Y \stackrel\sim\to N$. Thus there is a unique isomorphism class of
      corelation between any two objects.
    \item Note that the category $\Set$ has finite colimits and an epi-mono
      factorisation system. Epi-mono corelations from $X \to Y$ in $\mathrm{Set}$
      surjective functions $X+Y \to N$; thus their isomorphism classes are
      partitions, or equivalence relations on $X+Y$. 
  \end{itemize}
\end{examples}

\subsection{Categories of corelations}

We compose corelations by taking the $\mathcal E$-part of their composite
cospan. That is, given corelations $X \stackrel{i_X}{\longrightarrow} N
\stackrel{o_Y}{\longleftarrow} Y$ and $Y \stackrel{i_Y}{\longrightarrow} M
\stackrel{o_Z}{\longleftarrow} Z$, their composite is given by the cospan $X
\to \overline{N+_YM} \leftarrow Z$ in
\[
  \xymatrix{
    && N+_YM \\
    && \overline{N+_YM} \ar[u]_{m} \\
    & N \ar[ur]^{e \circ \iota_N} && M \ar[ul]_{e \circ \iota_M} \\
    \quad X \quad \ar[ur]^{i_X} && Y \ar[ul]_{o_Y} \ar[ur]^{i_Y} && \quad Z, \quad \ar[ul]_{o_Z}
  }
\]
where $m \circ e$ is the $(\mc E,\mc M)$-factorisation of $[j_N,j_M]\maps N+M
\to N+_YM$. 

It is well-known that this composite is unique up to isomorphism, and that when
$\mc M$ is well-behaved it defines a category with morphisms isomorphism classes
of corelations. For example, bicategorical version of the dual theorem, for
spans and relations, can be found in \cite{JW}. Nonetheless, for the sake of
completeness we sketch our own argument here.

\begin{proposition} \label{prop.corelcomp}
  Let $\mc C$ be a category with finite colimits and with a factorisation system
  $(\mc E,\mc M)$. Then the above is a well-defined composition rule on
  isomorphism classes of corelations.
\end{proposition}
\begin{proof}
  Let
  $(X \stackrel{i_X}{\longrightarrow} N \stackrel{o_Y}{\longleftarrow} Y$,
  $X \stackrel{i_X'}{\longrightarrow} N' \stackrel{o_Y'}{\longleftarrow} Y)$
%  \[
%    X \stackrel{i_X}{\longrightarrow} N \stackrel{o_Y}{\longleftarrow} Y 
%    \qquad \mbox{and} \qquad 
%    X \stackrel{i_X'}{\longrightarrow} N' \stackrel{o_Y'}{\longleftarrow} Y
%  \]
  and
  $(Y \stackrel{i_Y}{\longrightarrow} M \stackrel{o_Z}{\longleftarrow} Z$, $Y
  \stackrel{i_Y'}{\longrightarrow} M' \stackrel{o_Z'}{\longleftarrow} Z)$
%  \[
%    Y \stackrel{i_Y}{\longrightarrow} M \stackrel{o_Z}{\longleftarrow} Z 
%    \qquad \mbox{and} \qquad 
%    Y \stackrel{i_Y'}{\longrightarrow} M' \stackrel{o_Z'}{\longleftarrow} Z
%  \]
  be pairs of isomorphic jointly $\mathcal E$-like cospans. Their composites
  \emph{as cospans} 
  % $(X \longrightarrow N+_YM \longleftarrow Z)$ and $(X
  %\longrightarrow N'+_YM' \longleftarrow Z)$
%  \[
%    X \longrightarrow N+_YM \longleftarrow Z 
%    \qquad \mbox{and} \qquad 
%    X \longrightarrow N'+_YM' \longleftarrow Z
%  \]
  are isomorphic, so the factorisation system gives an isomorphism $s$ such that
  the diagram
  \[
    \xymatrixcolsep{3pc}
    \xymatrixrowsep{2pc}
    \xymatrix{
      X+Z \ar[r]^e \ar@{=}[d] & \overline{N+_YM} \ar[r]^m \ar@{.>}[d]^{s}_\sim & N+_YM
      \ar[d]^\sim \\
      X+Z \ar[r]_{e'}&\overline{N'+_YM'} \ar[r]_{m'} & N'+_YM'
    }
  \]
  commutes. This $s$ is an isomorphism of the composite corelations.
\end{proof}

Composition of corelations is \emph{not} associative in general. It is, however,
associative when $\mathcal M$ is \define{stable under pushout}: that is,
whenever
\[
  \xymatrixcolsep{3pc}
  \xymatrixrowsep{3pc}
  \xymatrix{
    \ar[r]^j & \\
    \ar[u] \ar[r]^m &  \ar[u]
  }
\]
is a pushout square such that $m \in \mathcal M$, we also have that $j \in
\mathcal M$. 

Stability under pushout is a powerful property. A key corollary, both for
associativity and in general, is that it implies $\mc M$ is also closed under
$+$. 
%For our proof sketch we rely on the following useful lemma.

\begin{lemma} \label{lem.mcoproductsmc}
  Let $\mathcal C$ be a category with finite colimits, and let $\mathcal M$ be a
  subcategory of $\mathcal C$ stable under pushouts and containing all
  isomorphisms. Then $(\mc M,+)$ is a symmetric monoidal category.
\end{lemma}
\begin{proof}
  It is enough to show that for all morphisms $m,m' \in \mc M$ we have $m+m'$ in
  $\mc M$. Since $\mc M$ contains all isomorphisms, the coherence maps are
  inherited from $\mc C$. The required axioms---the functoriality of the tensor
  product, the naturality of the coherence maps, and the coherence laws---are
  also inherited as they hold in $\mc C$.

  To see $m+m'$ is in $\mc M$, simply observe that we have the pushout square
  \[
  \xymatrixcolsep{3pc}
  \xymatrixrowsep{2pc}
    \xymatrix{
      A+C \ar[r]^{m+1} & B+C \\
      A \ar[r]^m \ar[u]^{\iota} & B \ar[u]_{\iota} \\
    }
  \]
  in $\mc C$. As $\mc M$ is stable under pushout, $m+1 \in \mc M$. Similarly,
  $1+m' \in \mc M$. Thus their composite $m+m'$ lies in $\mc M$, as required.
\end{proof}

An analogous argument shows that pushouts of maps $m+_Ym'$ also lie in $\mc M$.
Using this lemma it is not difficult to show associativity---the key point is
that factorisation `commutes' with pushouts, and that we have a category
$\corel_{(\mc E,\mc M)}(\mc C)$. Again, this is all well-known, and can be found
in \cite{JayWar00}. We will incidentally reprove these facts in the following,
while pursuing richer structure.

%\begin{proposition}
%  Let $\mc C$ be a category with finite colimits and a factorisation system
%  $(\mc E,\mc M)$ such that $\mc M$ is stable under pushouts. Then composition
%  of corelations is associative.
%\end{proposition}
%\begin{proof}[Proof (sketch)]
%  The key concern is whether factorisation `commutes' with taking pushouts.
%  Using the above lemma and the fact that $\mc M$ is stable under pushouts, the
%  pushout square 
%  \[
%  \xymatrixcolsep{3pc}
%  \xymatrixrowsep{2pc}
%  \xymatrix{
%    N+_Y\overline{M} \ar[r]^{m'} & N+_YM \\
%    N+\overline{M} \ar[u] \ar[r]^{1+m} & N+M  \ar[u]
%  }
%  \]
%  shows that the map $m'\maps N+_Y\overline{M} \to N+_YM$ is in $\mc M$
%  whenever $m\maps \overline{N} \to N$ is. It can then be shown that the
%  composite of any number of corelations can be given by taking the composite of
%  them all as cospans, and then taking the jointly $\mc E$-like part.
%\end{proof}

%The identity axioms for corelations are self-evident. We thus have a category
%$\corel_{(\mc E,\mc M)}(\mc C)$.  Again, we will drop explicit reference to the
%factorisation system when context allows, simply writing $\mathrm{Corel}(\mc
%C)$.
%
%\begin{remark}
%An aside: proving associativity is not logically necessary in the development
%of this chapter. Associativity will follow from the existence of the functor
%from $\cospan(\mc C)$ to $\corel(\mc C)$, like the other necessary coherence
%results for hypergraph categories. Nonetheless, it is terminologically easier to
%establish that $\corel(\mc C)$ is indeed a category at this point, before we
%define hypergraph structure and functors on it.
%\end{remark}
%

Indeed, for modelling networks, we require not just a category, but a hypergraph
category. Corelation categories come equipped with this extra structure.
Recall that we gave decorated cospan categories a hypergraph structure by
defining a wide embedding $\mathrm{Cospan}(\mc C) \hookrightarrow
F\mathrm{Cospan}$, via which $F\mathrm{Cospan}$ inherited the coherence and
Frobenius maps (Theorem \ref{thm:fcospans}). We will argue similarly here, after
showing that the `quotient' map
\[
  \cospan(\mc C) \longrightarrow \corel(\mc C)
\]
taking each cospan to its jointly $\mc E$-like part is functorial. Indeed, we
define the coherence and Frobenius maps of $\corel(\mc C)$ to be their image
under this map. For the monoidal product we again use the coproduct in $\mc C$;
the monoidal product of two corelations is their monoidal product as cospans.

\begin{theorem} \label{thm.cospantocorel}
  Let $\mathcal C$ be a category with finite colimits, and let $(\mathcal E,
  \mathcal M)$ be factorisation system on $\mathcal C$ such that $\mathcal M$ is
  stable under pushout. Then there exists a hypergraph category
  $\mathrm{Corel}_{(\mc E,\mc M)}(\mathcal C)$ with 
  \begin{center}
    \begin{tabular}{| c | p{.65\textwidth} |}
      \hline
      \multicolumn{2}{|c|}{The hypergraph category $(\mathrm{Corel}_{(\mc E,\mc M)}(\mc C),+)$} \\
      \hline
      \textbf{objects} & the objects of $\mathcal C$ \\ 
      \textbf{morphisms} & isomorphism classes of $(\mc E,\mc M)$-corelations in $\mathcal C$\\ 
      \textbf{composition} & given by the $\mc E$-part of pushout \\
      \textbf{monoidal product} & the coproduct in $\mathcal C$ \\
      \textbf{coherence maps} & inherited from $\cospan(\mc C)$  \\
      \textbf{hypergraph maps} & inherited from $\cospan(\mc C)$ \\
      \hline
    \end{tabular}
  \end{center}  
\end{theorem}

Again, we will drop explicit reference to the factorisation system when context allows, simply writing $\mathrm{Corel}(\mc C)$.

Proposition \ref{prop.corelcomp} shows that our composition rule is a
well-defined function, Lemma \ref{lem.monfact} shows likewise for the monoidal
product $+\maps \corel(\mc C)\times \corel(\mc C) \to \corel(\mc C)$. Thus we
have the required data for a hypergraph category. It just remains to check a
number of axioms: associativity and unitality of the categorical composition,
functoriality of the monoidal product, naturality of the coherence maps, the
coherence axioms for symmetric monoidal categories, the Frobenius laws.

Our strategy for this will simply be to show that the (surjective) function
taking a cospan to its jointly $\mc E$-part preserves composition and has
natural strict monoidal coherence maps. This implies that to evaluate an
expression in the monoidal category of corelations, we may simply evaluate it in
the monoidal category of cospans, and then take the $\mc E$-part. Thus if an
equation is true for cospans, it is true for corelations.

Instead of proving just this, however, we will prove a generalisation regarding
the analogous `reduction' map between any two corelation `categories'. This will
reduce to the desired special case by taking the domain to be the `trivial'
$(\mc C,\mc I_{\mc C})$-corelations, which we know is equal to the hypergraph
category of cospans. But the generality is not spurious: it has the advantage of
proving the existence of a class of hypergraph functors between corelation
categories in the same fell swoop.

Although a touch convoluted, this strategy is worth the pause for thought. We
will use it once again for \emph{decorated corelations}, to great economy.

%\begin{lemma}
%  The map $(-)+(-)\maps \corel(\mc C)\times\corel(\mc C) \to \corel(\mc C)$
%  induced by the coproduct in $\mc C$ is functorial.
%\end{lemma}
%\begin{proof}
%As a preliminary, note that $+$ is a well-defined function: Lemma
%\ref{lem.monfact} implies that the coproduct of two corelations is again a
%corelation. Our task is to show that, given the four corelations 
%\[
%  \xymatrixrowsep{0pt}
%  \xymatrix{
%  f= X \longrightarrow N \longleftarrow Y  & &
%  g= Y \longrightarrow M \longleftarrow Z \\
%  h = X' \longrightarrow N' \longleftarrow Y' & &
% k= Y' \longrightarrow M' \longleftarrow Z'
%}
%\]
%we have $(g \circ f) + (k \circ h) = (g + k) \circ (f + h)$. The left and
%right side of this expression are respectively given by the first arrow in the
%upper and lower rows of the commutative diagram
%\[
%  \xymatrix{
%    (X+Z)+(X'+Z') \ar[r]^{\mc E+\mc E} \ar[d]^{\sim} & 
%    (\overline{N+_YM})+(\overline{N'+_{Y'}M'}) \ar[r]^{\mc M+\mc M} \ar@{.}[d] &
%    (N+_YM)+(N'+_{Y'}M') \ar[d]^{\sim} \\
%    (X+X')+(Z+Z') \ar[r]^{\mc E} &
%    \overline{(N+N')+_{Y+Y'}(M+M')} \ar[r]^{\mc M} & 
%    (N+N')+_{Y+Y'}(M+M'). \\
%  }
%\]
%The leftmost and rightmost vertical arrows are isomorphisms by properties of
%colimits. The upper row is an $(\mc E,\mc M)$-factorisation as the first map is
%the coproduct of two maps in $\mc E$ and the second map is the coproduct of two
%maps in $\mc M$, both of which are monoidal with respect to the coproduct
%(Lemmas \ref{lem.monfact} and \ref{lem.mcoproductsmc}). The lower row is an
%$(\mc E,\mc M)$-factorisation by definition. Thus, by the properties of
%factorisation systems, the dotted map $s$ is an isomorphism, and hence $+$ is
%functorial. 
%\end{proof}
%
%To prove Theorem \ref{thm.cospantocorel} it remains to show that our proposed
%data for $\mathrm{Corel}(\mc C)$ obey the necessary axioms: the symmetric
%monoidal coherence laws, the special commutative Frobenius monoid laws. Our
%proof strategy will be a touch complicated. Again, recall that in Theorem
%\ref{thm:fcospans} we proved $F\mathrm{Cospan}$ had hypergraph structure via
%a functor from $\cospan(\mc C)$. Instead of proving each axiom directly, we will
%again leverage the fact that we already know $\cospan(\mc C)$ is a hypergraph
%category, and show that $\corel(\mc C)$ is the image of $\cospan(\mc C)$ under a
%composition-preserving map that also respects (indeed, defines) the monoidal and
%hypergraph structure.
%
%Note that $\mc C$ is trivially stable under pushout. By definition we have the
%equality of hypergraph categories $\mathrm{Corel}_{(\mc C,\mc I_{\mc C})}(\mc
%C)= \cospan(\mc C)$. Thus the functor $\cospan(\mc C) \to \mathrm{Corel}(\mc C)$
%is a special case of functors between corelation categories. Taking advantage of
%this, we will discuss functors between corelation categories in general, before
%specialising to this case to prove that $\mathrm{Corel}(\mc C)$ indeed is a
%well-defined hypergraph category.
%
%This will be done in the next subsection.



\section{Functors between corelation categories}
We have seen that to construct a functor between cospan categories one may start
with a colimit-preserving functor between the underlying categories. Corelation
categories are similar, but we remove the part of each cospan that lies in $\mc
M$. Hence for functors between corelation categories, we also require that the
target category removes at least as much information as the source category.

\begin{proposition} \label{prop.corelfunctors}
  Let $\mathcal C$, $\mathcal C'$ have finite colimits and respective
  factorisation systems $(\mathcal E, \mathcal M)$, $(\mathcal E', \mathcal M')$,
  such that $\mathcal M$ and $\mathcal M'$ are stable under pushout. Further let
  $A\maps \mathcal C \to \mathcal C'$ be a functor that preserves finite colimits
  and such that the image of $\mathcal M$ lies in $\mathcal M'$.

  Then we may define a hypergraph functor $\square\maps \corel(\mathcal C) \to
  \corel(\mathcal C')$ sending each object $X$ in $\corel(\mathcal C)$ to $AX$ in
  $\corel(\mc C')$ and each corelation 
  \[
    X \stackrel{i_X}{\longrightarrow} N \stackrel{o_Y}{\longleftarrow} Y 
  \]
  to the $\mc E'$-part
  \[
    AX \stackrel{Ai_X}{\longrightarrow} \overline{AN}
    \stackrel{Ao_Y}{\longleftarrow} AY.
  \]
  of the image cospan. The coherence maps are the $\mc E'$-part
  $\overline{\kappa_{X,Y}}$ of the isomorphisms $\kappa_{X,Y}\maps AX+AY \to
  A(X+Y)$ given as $A$ preserves colimits.
\end{proposition}

As discussed, we are still yet to prove that $\corel(\mc C)$ is a hypergraph
category. We address this first with two lemmas regarding these proposed
functors.

\begin{lemma} \label{lem.corelfuncomposition}
  The above function $\square\maps \corel(\mc C) \to \corel(\mc C')$ preserves
  composition.
\end{lemma}
\begin{proof}
  Let $f = (X \longrightarrow N \longleftarrow Y)$ and $g= (Y \longrightarrow M
  \longleftarrow Z)$ be corelations in $\mathcal C$. By definition, the
  corelations $\square(g) \circ \square(f)$ and $\square(g \circ f)$ are given
  by the first arrows in the top and bottom row respectively of the diagram:
  \[ \label{diag.eparts}
    \begin{aligned}
      \xymatrixcolsep{4pc}
      \xymatrixrowsep{2pc}
      \xymatrix{
	AX+AZ \ar[r]^{\mc E'} \ar@{=}[d] & \overline{\overline{AN}+_{AY}\overline{AM}}
	\ar[r]^{\mc M'} \ar@{<.>}[d]^{n} & \overline{AN}+_{AY}\overline{AM}
	\ar[r]^{m'_{AN}+_{AY}m'_{AM}} &
	AN+_{AY}AM \\
	AX+AZ \ar[r]^{\mc E'} & \overline{A(\overline{N+_YM})} \ar[r]^{\mc M'} &
	A(\overline{N+_YM}) \ar[r]^{Am_{N+_YM}} & A(N+_YM) \ar@{<->}[u]_{\sim}
      }
    \end{aligned}
    \tag{$\ast$}
  \]
  The morphisms labelled $\mc E'$ lie in $\mc E'$, and similarly for $\mc M'$;
  these are given by the factorisation system on $\mc C'$.  The maps
  $Am_{N+_YM}$ and $m'_{AN}+_{AY}m'_{AM}$ lie in $\mc M'$ too: $Am_{N+_YM}$ as
  it is in the image of $\mc M$, and $m'_{AN}+_{AY}m'_{AM}$ as $\mc M'$ is
  stable under pushout. 

  Moreover, the diagram commutes as both maps $AX+AZ \to AN+_{AY}AM$ compose to
  that given by the pushout of the images of $f$ and $g$ over $AY$.  Thus the
  diagram represents two $(\mc E', \mc M')$ factorisations of the same morphism,
  and there exists an isomorphism $n$ between the corelations $\square(g) \circ
  \square(f)$ and $\square(g\circ f)$. This proves that $\square$ preserves
  composition.
\end{proof}
This first lemma allows us to verify the associativity and unit laws for
$\corel(\mc C)$.
\begin{corollary}
  $\corel(\mc C)$ is well-defined as a category.
\end{corollary}
\begin{proof}
  Consider the case of Proposition \ref{prop.corelfunctors} with $\mc C = \mc
  C'$, $(\mc E,\mc M) = (\mc C, \mc I_{\mc C})$, and $A = 1_{\mc C}$. Then the
  domain of $\square$ is $\cospan(\mc C)$ by definition. In this case, the
  function $\square\maps \cospan(\mc C) \to \corel(\mc C)$ is
  bijective-on-objects and surjective-on-morphisms. Thus to compute the
  composite of any two corelations, we may consider them as cospans, compute
  their composite \emph{as cospans}, and then take the $\mc E$-part of the
  result. Since composition of cospans is associative and unital, so is
  composition of corelations, with the identity corelation just the image of the
  identity cospan.
\end{proof}

Note that the identity in $\corel(\mc C)$ may not be the identity cospan itself.
For example, with the factorisation system $(\mc I_{\mc C}, \mc C)$ the $\mc
I_{\mc C}$-part of the identity cospan is simply $X \stackrel{\iota_{X_1}}\to
X+X \stackrel{\iota_{X_2}}\leftarrow X$, where $\iota_{X_i}$ is the inclusion of
$X$ into the $X_i$ factor of the coproduct $X+X$.

This first lemma is also useful in proving the second important lemma: the
naturality of $\overline{\kappa}$.

\begin{lemma} \label{lem.corelfunmonoidal}
  The maps $\overline{\kappa_{X,Y}}$ above are natural.
\end{lemma}
\begin{proof}
  Let $f = (X \longrightarrow N \longleftarrow Y)$, $g= (Z \longrightarrow M
  \longleftarrow W)$ be corelations in $\mc C$. We wish to show that
  \[
      \xymatrixcolsep{4pc}
      \xymatrixrowsep{2pc}
    \xymatrix{
      AX+AY \ar[r]^{\square(f)+\square(g)}
      \ar[d]_{\overline{\kappa_{X,Y}}} & 
      AZ+AW \ar[d]^{\overline{\kappa_{Z,W}}} \\
      A(X+Y) \ar[r]^{\square(f+g)} & A(Z+W)
    }
  \]
  commutes in $\corel(\mc C')$. 
  
  Consider the following commutative diagram in $\mc C'$, with the outside
  square equivalent to the naturality square for the coherence maps of the
  monoidal functor $\cospan(\mc C) \to \cospan(\mc C')$:
  \[ \label{diag.natural}
    \begin{aligned}
      \xymatrixcolsep{4pc}
      \xymatrixrowsep{2pc}
    \xymatrix{
      (AX+AY)+(AZ+AW) \ar[r]^{\mc E'+\mc E'} \ar[d]_{\kappa_{X,Y}+\kappa_{Z,W}} & 
      \overline{AN}+\overline{AM} \ar[r]^{\mc M'+\mc M'} \ar@{.>}[d]^{p} & 
      AN+AM \ar[d]^{\kappa_{N,M}}\\
      A(X+Y)+A(Z+W) \ar[r]^{\mc E'} & \overline{A(N+M)} \ar[r]^{\mc M'} & A(N+M)
    }
  \end{aligned}
    \tag{$\#$}
  \]
  We have factored the top edge as the coproduct of the respective
  factorisations of $f$ and $g$, and the bottom edge simply as the factorisation
  of the coproduct $f+g$. 
  
  Note that by Lemma \ref{lem.monfact} the coproduct of two maps in $\mc E'$ is
  again in $\mc E'$, while Lemma \ref{lem.mcoproductsmc} implies the same for
  $\mc M'$. Thus the top edge is an $(\mc E',\mc M')$-factorisation, and the
  uniqueness of factorisations gives the isomorphism $n$. 
  Given that the map reducing cospans to corelations is functorial, the
  commutative square
  \[
      \xymatrixcolsep{2.5pc}
      \xymatrixrowsep{2pc}
    \xymatrix{
      (AX+AY)+A(Z+W) \ar[r]^{1+\kappa_{Z,W}^{-1}} \ar@{=}[d] & (AX+AY)+(AZ+AW)
      \ar[r]^{\mc E'+\mc E'} & 
      \overline{AN}+\overline{AM} \ar[d]^{n} \\
      (AX+AY)+A(Z+W) \ar[r]^{\kappa_{X,Y}+1} & A(X+Y)+A(Z+W) \ar[r]^{\mc E'} & 
      \overline{A(N+M)}
    }
  \]
  then implies the naturality of the maps $\overline{\kappa}$.
\end{proof}

These lemmas now imply that $\corel(\mc C)$ is a well-defined hypergraph
category.
\begin{proof}[Proof of Theorem \ref{thm.cospantocorel}]
  To complete the proof then, again consider the case of Proposition
  \ref{prop.corelfunctors} with $\mc C = \mc C'$, $(\mc E,\mc M) = (\mc C, \mc
  I_{\mc C})$, and $A = 1_{\mc C}$. Note that by definition this function maps
  the coherence and hypergraph maps of $\cospan(\mc C)$ onto the corresponding
  maps of $\corel(\mc C)$. As $\cospan(\mc C)$ is a hypergraph, and $\square$
  preserves composition and respects the monoidal and hypergraph structure,
  $\corel(\mc C)$ is also a hypergraph category. 
  
  For instance, suppose we want to check the functoriality of the monoidal
  product $+$. We then wish to show $(g \circ f) + (k \circ h) = (g + k) \circ
  (f + h)$ for corelations of the appropriate types.  But $\square$ preserves
  composition, and the naturality of $\kappa$, here the identity map, implies
  that for any two cospans the $\mc E$-part of their coproduct is equal to the
  coproduct of their $\mc E$-parts. Thus we may compute these two expressions by
  viewing $f$, $g$, $h$, and $k$ as cospans, evaluating them in the category of
  cospans, and then taking their $\mc E$-parts. Since the equality holds in the
  category of cospans, it holds in the category of corelations.
\end{proof}

\begin{corollary}
  There is a strict hypergraph functor 
  \[
    \square\maps \mathrm{Cospan}(\mathcal C) \longrightarrow \mathrm{Corel}(\mathcal C)
  \]
  that takes each object of $\cospan(\mathcal C)$ to itself as an object of
  $\corel(\mathcal C)$ and each cospan to its $\mathcal E$-part.
\end{corollary}

  Finally, we complete the proof that $\square$ is always a hypergraph functor.

\begin{proof}[Proof of Proposition \ref{prop.corelfunctors}] 
  We show $\square$ is a functor, a symmetric monoidal functor, and then finally
  a hypergraph functor.

  \paragraph{Functoriality.} First, recall that $\square$ preserves composition
  (Lemma \ref{lem.corelfuncomposition}). Thus to prove $\square$ is a functor it
  remains to show identities are mapped to identities. The general idea for this
  and for similar axioms is to recall that the special maps are given by reduced
  versions of particular colimits, and that $(\mc E',\mc M')$ reduces maps more
  than $(\mc E,\mc M)$. 

  In this case, recall the identity corelation is given by the $\mc E$-part $X+X
  \to \overline{X}$ of $[1,1]\maps X+X \to X$. Thus the image of the identity on
  $X$ and the identity on $AX$ are given by the top and bottom rows of the
  commuting square
  \[
    \xymatrixcolsep{4pc}
    \xymatrixrowsep{2pc}
    \xymatrix{
      A(X+X) \ar[d]^{\kappa^{-1}}_{\sim} \ar[r]^{\mc E'} &
      \overline{A\overline{X}} \ar[r]^{\mc M'} \ar@{.>}[d]^{n} &
      A\overline{X} \ar[r]^{A\mc M} & AX \ar@{=}[d]\\
      AX+AX \ar[r]^{\mc E'} & \overline{AX} \ar[rr]^{\mc M'} && AX
    }
  \]
  The outside square commutes as we know $A$ maps the identity cospan of $\mc C$
  to the identity cospan of $\mc C'$. The top row is the image under $A$ of the
  identity cospan in $\mc C$, factored first in $\mc C$, and then in $\mc C'$.
  The bottom row is just the factored identity cospan on $AX$ in $\mc C'$. As
  $A$ maps $\mc M$ into $\mc M'$, the map marked $A\mc M$ lies in $\mc M'$. Thus
  both rows are $(\mc E',\mc M')$-factorisations, and so we have the isomorphism
  $n$. Thus $\square$ preserves identities.

  \paragraph{Strong monoidality.} We proved in Lemma \ref{lem.corelfunmonoidal} that
    our proposed coherence maps are natural. The rest of the properties follow
    from the composition preserving map $\cospan(\mc C') \to \corel(\mc C')$.
    Since the $\kappa$ obey all the required axioms as cospans, they obey them
    as corelations too.

  \paragraph{Hypergraph structure.} The proof of preservation of the hypergraph
  structure follows the same pattern as the identity maps. 
\end{proof}

\begin{example}
Note that if both $(\mathcal E, \mathcal M)$ and $(\mathcal E', \mathcal M')$
are epi-split mono factorisations, then we always have that $F(\mathcal M)
\subseteq \mathcal M'$. Indeed, if an (one-sided) inverse exists in the domain
category, it exists in the codomain category. Thus colimit-preserving functors
between categories with finite colimits and epi-split mono factorisation systems
also induce a functor between the epi-split mono corelation categories. We will
use this in Chapter ref.
\end{example}

\section{Examples}
\subsection{Equivalence relations as corelations in $\Set$}
  
In each factorisation system of Examples \ref{ex.corels} the right factor
  $\mathcal M$ is stable under pushout.  This gives the hypergraph categories of
  cospans in $\mc C$, the indiscrete category on the objects of $\mc C$, and
  equivalence relations between finite sets. The last of these examples is
  perhaps the most instructive for black-boxing open systems. 
  
  Epi-mono corelations in $\Set$ are isomorphism classes of
  jointly epic cospans. Thus a corelation $X \to Y$ is an equivalence relation
  on $X+Y$. We might depict these as follows
\[
  \begin{tikzpicture}[circuit ee IEC]
	\begin{pgfonlayer}{nodelayer}
		\node [contact, outer sep=5pt] (0) at (-2, 1) {};
		\node [contact, outer sep=5pt] (1) at (-2, 0.5) {};
		\node [contact, outer sep=5pt] (2) at (-2, -0) {};
		\node [contact, outer sep=5pt] (3) at (-2, -0.5) {};
		\node [contact, outer sep=5pt] (4) at (-2, -1) {};
		\node [contact, outer sep=5pt] (5) at (1, 1.25) {};
		\node [contact, outer sep=5pt] (6) at (1, 0.75) {};
		\node [contact, outer sep=5pt] (7) at (1, 0.25) {};
		\node [contact, outer sep=5pt] (8) at (1, -0.25) {};
		\node [contact, outer sep=5pt] (9) at (1, -0.75) {};
		\node [contact, outer sep=5pt] (10) at (1, -1.25) {};
		\node [style=none] (11) at (-2.75, -0) {$X$};
		\node [style=none] (12) at (1.75, -0) {$Y$};
	\end{pgfonlayer}
	\begin{pgfonlayer}{edgelayer}
		\draw [rounded corners=5pt, dashed] 
   (node cs:name=0, anchor=north west) --
   (node cs:name=1, anchor=south west) --
   (node cs:name=6, anchor=south east) --
   (node cs:name=5, anchor=north east) --
   cycle;
		\draw [rounded corners=5pt, dashed] 
   (node cs:name=2, anchor=north west) --
   (node cs:name=3, anchor=south west) --
   (node cs:name=3, anchor=south east) --
   (node cs:name=2, anchor=north east) --
   cycle;
		\draw [rounded corners=5pt, dashed] 
   (node cs:name=4, anchor=north west) --
   (node cs:name=4, anchor=south west) --
   (node cs:name=10, anchor=south east) --
   (node cs:name=9, anchor=north east) --
   cycle;
   		\draw [rounded corners=5pt, dashed] 
   (node cs:name=7, anchor=north west) --
   (node cs:name=7, anchor=south west) --
   (node cs:name=7, anchor=south east) --
   (node cs:name=7, anchor=north east) --
   cycle;
   		\draw [rounded corners=5pt, dashed] 
   (node cs:name=8, anchor=north west) --
   (node cs:name=8, anchor=south west) --
   (node cs:name=8, anchor=south east) --
   (node cs:name=8, anchor=north east) --
   cycle;
	\end{pgfonlayer}
\end{tikzpicture}
\]
Here we have a corelation from a set $X$ of five elements to a set $Y$ of six
elements. Elements belonging to the same equivalence class of $X+Y$ are grouped
(`connected') by a dashed line.

Composition of corelations first takes the transitive closure of the two
partitions (the pushout in $\Set$), before restricting the partition to the new
domain and codomain (restricting to the jointly epic part). For example,
suppose in addition to the corelation $\alpha\maps X \to Y$ above we have
another corelation $\beta\maps Y \to Z$
\[
\begin{tikzpicture}[circuit ee IEC]
	\begin{pgfonlayer}{nodelayer}
		\node [style=none] (0) at (-2.75, -0) {$Y$};
		\node [style=none] (1) at (1.75, 0) {$Z$};
		\node [contact, outer sep=5pt] (2) at (-2, 1.25) {};
		\node [contact, outer sep=5pt] (3) at (-2, 0.75) {};
		\node [contact, outer sep=5pt] (4) at (-2, 0.25) {};
		\node [contact, outer sep=5pt] (5) at (-2, -0.25) {};
		\node [contact, outer sep=5pt] (6) at (-2, -0.75) {};
		\node [contact, outer sep=5pt] (7) at (-2, -1.25) {};
		\node [contact, outer sep=5pt] (8) at (1, 1) {};
		\node [contact, outer sep=5pt] (9) at (1, 0.5) {};
		\node [contact, outer sep=5pt] (10) at (1, -0) {};
		\node [contact, outer sep=5pt] (11) at (1, -0.5) {};
		\node [contact, outer sep=5pt] (12) at (1, -1) {};
	\end{pgfonlayer}
		\draw [rounded corners=5pt, dashed] 
   (node cs:name=2, anchor=north west) --
   (node cs:name=3, anchor=south west) --
   (node cs:name=8, anchor=south east) --
   (node cs:name=8, anchor=north east) --
   cycle;
		\draw [rounded corners=5pt, dashed] 
   (node cs:name=4, anchor=north west) --
   (node cs:name=4, anchor=south west) --
   (node cs:name=4, anchor=south east) --
   (node cs:name=4, anchor=north east) --
   cycle;
		\draw [rounded corners=5pt, dashed] 
   (node cs:name=5, anchor=north west) --
   (node cs:name=6, anchor=south west) --
   (node cs:name=11, anchor=south east) --
   (node cs:name=10, anchor=north east) --
   cycle;
		\draw [rounded corners=5pt, dashed] 
   (node cs:name=7, anchor=north west) --
   (node cs:name=7, anchor=south west) --
   (node cs:name=12, anchor=south east) --
   (node cs:name=12, anchor=north east) --
   cycle;
		\draw [rounded corners=5pt, dashed] 
   (node cs:name=9, anchor=north west) --
   (node cs:name=9, anchor=south west) --
   (node cs:name=9, anchor=south east) --
   (node cs:name=9, anchor=north east) --
   cycle;
\end{tikzpicture}
\]
Then the composite $\beta\circ\alpha$ of our two corelations is given by
\vspace{-1ex}
\[
  \begin{aligned}
\begin{tikzpicture}[circuit ee IEC]
	\begin{pgfonlayer}{nodelayer}
		\node [contact, outer sep=5pt] (-2) at (1, 1.25) {};
		\node [contact, outer sep=5pt] (-1) at (1, 0.75) {};
		\node [contact, outer sep=5pt] (0) at (1, 0.25) {};
		\node [contact, outer sep=5pt] (1) at (1, -0.25) {};
		\node [contact, outer sep=5pt] (2) at (1, -0.75) {};
		\node [contact, outer sep=5pt] (3) at (1, -1.25) {};
		\node [style=none] (4) at (-2.75, -0) {$X$};
		\node [style=none] (5) at (4.75, -0) {$Z$};
		\node [contact, outer sep=5pt] (6) at (-2, 1) {};
		\node [contact, outer sep=5pt] (7) at (-2, -0.5) {};
		\node [contact, outer sep=5pt] (8) at (-2, 0.5) {};
		\node [contact, outer sep=5pt] (9) at (-2, -0) {};
		\node [contact, outer sep=5pt] (10) at (-2, -1) {};
		\node [contact, outer sep=5pt] (11) at (4, -0) {};
		\node [contact, outer sep=5pt] (12) at (4, -1) {};
		\node [contact, outer sep=5pt] (13) at (4, -0.5) {};
		\node [contact, outer sep=5pt] (14) at (4, 0.5) {};
		\node [contact, outer sep=5pt] (19) at (4, 1) {};
		\node [style=none] (20) at (1, -1.75) {$Y$};
		\node [style=none] (21) at (1, 1.75) {\phantom{$Y$}};
	\end{pgfonlayer}
	\begin{pgfonlayer}{edgelayer}
		\draw [rounded corners=5pt, dashed] 
   (node cs:name=6, anchor=north west) --
   (node cs:name=8, anchor=south west) --
   (node cs:name=-1, anchor=south east) --
   (node cs:name=-2, anchor=north east) --
   cycle;
		\draw [rounded corners=5pt, dashed] 
   (node cs:name=9, anchor=north west) --
   (node cs:name=7, anchor=south west) --
   (node cs:name=7, anchor=south east) --
   (node cs:name=9, anchor=north east) --
   cycle;
		\draw [rounded corners=5pt, dashed] 
   (node cs:name=10, anchor=north west) --
   (node cs:name=10, anchor=south west) --
   (node cs:name=3, anchor=south east) --
   (node cs:name=2, anchor=north east) --
   cycle;
		\draw [rounded corners=5pt, dashed] 
   (node cs:name=-2, anchor=north west) --
   (node cs:name=-1, anchor=south west) --
   (node cs:name=19, anchor=south east) --
   (node cs:name=19, anchor=north east) --
   cycle;
		\draw [rounded corners=5pt, dashed] 
   (node cs:name=0, anchor=north west) --
   (node cs:name=0, anchor=south west) --
   (node cs:name=0, anchor=south east) --
   (node cs:name=0, anchor=north east) --
   cycle;
		\draw [rounded corners=5pt, dashed] 
   (node cs:name=1, anchor=north west) --
   (node cs:name=1, anchor=south west) --
   (node cs:name=1, anchor=south east) --
   (node cs:name=1, anchor=north east) --
   cycle;
		\draw [rounded corners=5pt, dashed] 
   (node cs:name=1, anchor=north west) --
   (node cs:name=2, anchor=south west) --
   (node cs:name=13, anchor=south east) --
   (node cs:name=11, anchor=north east) --
   cycle;
		\draw [rounded corners=5pt, dashed] 
   (node cs:name=3, anchor=north west) --
   (node cs:name=3, anchor=south west) --
   (node cs:name=12, anchor=south east) --
   (node cs:name=12, anchor=north east) --
   cycle;
		\draw [rounded corners=5pt, dashed] 
   (node cs:name=14, anchor=north west) --
   (node cs:name=14, anchor=south west) --
   (node cs:name=14, anchor=south east) --
   (node cs:name=14, anchor=north east) --
   cycle;
	\end{pgfonlayer}
\end{tikzpicture}
\end{aligned}
\:
  =
\:
\begin{aligned}
\begin{tikzpicture}[circuit ee IEC]
	\begin{pgfonlayer}{nodelayer}
		\node [style=none] (0) at (-2.75, -0) {$X$};
		\node [style=none] (1) at (1.75, -0) {$Z$};
		\node [contact, outer sep=5pt] (2) at (-2, 1) {};
		\node [contact, outer sep=5pt] (3) at (-2, -0.5) {};
		\node [contact, outer sep=5pt] (4) at (-2, 0.5) {};
		\node [contact, outer sep=5pt] (5) at (-2, -0) {};
		\node [contact, outer sep=5pt] (6) at (-2, -1) {};
		\node [contact, outer sep=5pt] (7) at (1, -0) {};
		\node [contact, outer sep=5pt] (8) at (1, -1) {};
		\node [contact, outer sep=5pt] (9) at (1, -0.5) {};
		\node [contact, outer sep=5pt] (10) at (1, 0.5) {};
		\node [contact, outer sep=5pt] (13) at (1, 1) {};
		\node [style=none] (20) at (1, -1.75) {\phantom{$Y$}};
		\node [style=none] (21) at (1, 1.75) {\phantom{$Y$}};
	\end{pgfonlayer}
	\begin{pgfonlayer}{edgelayer}
		\draw [rounded corners=5pt, dashed] 
   (node cs:name=2, anchor=north west) --
   (node cs:name=4, anchor=south west) --
   (node cs:name=13, anchor=south east) --
   (node cs:name=13, anchor=north east) --
   cycle;
		\draw [rounded corners=5pt, dashed] 
   (node cs:name=5, anchor=north west) --
   (node cs:name=3, anchor=south west) --
   (node cs:name=3, anchor=south east) --
   (node cs:name=5, anchor=north east) --
   cycle;
		\draw [rounded corners=5pt, dashed] 
   (node cs:name=6, anchor=north west) --
   (node cs:name=6, anchor=south west) --
   (node cs:name=8, anchor=south east) --
   (node cs:name=7, anchor=north east) --
   cycle;
		\draw [rounded corners=5pt, dashed] 
   (node cs:name=10, anchor=north west) --
   (node cs:name=10, anchor=south west) --
   (node cs:name=10, anchor=south east) --
   (node cs:name=10, anchor=north east) --
   cycle;
	\end{pgfonlayer}
\end{tikzpicture}
\end{aligned}
\]
Informally, this captures the idea that two elements of $X+Z$ are `connected' if
we may travel from one to the other staying within connected components of
$\alpha$ and $\beta$.
  
  Epi-mono corelations in $\Set$ can also be visualised as terminals connected by junctions of ideal wires. We draw these by marking each equivalence
class with a point (the `junction'), and then connecting each element of the
domain and codomain to their equivalence class with a `wire'. Composition then
involves collapsing connected junctions down to a point.
\vspace{-1ex}
\[
  \begin{aligned}
\begin{tikzpicture}[circuit ee IEC]
	\begin{pgfonlayer}{nodelayer}
		\node [contact, outer sep=5pt] (6) at (-2, 1) {};
		\node [contact, outer sep=5pt] (7) at (-2, -0.5) {};
		\node [contact, outer sep=5pt] (8) at (-2, 0.5) {};
		\node [contact, outer sep=5pt] (9) at (-2, -0) {};
		\node [contact, outer sep=5pt] (10) at (-2, -1) {};
		\node [style=none] (15) at (-0.5, 0.875) {};
		\node [style=none] (28) at (-0.5, 0.25) {};
		\node [style=none] (16) at (-0.5, -0.125) {};
		\node [style=none] (29) at (-0.5, -0.375) {};
		\node [style=none] (17) at (-0.5, -1) {};
		\node [contact, outer sep=5pt] (-2) at (1, 1.25) {};
		\node [contact, outer sep=5pt] (-1) at (1, 0.75) {};
		\node [contact, outer sep=5pt] (0) at (1, 0.25) {};
		\node [contact, outer sep=5pt] (1) at (1, -0.25) {};
		\node [contact, outer sep=5pt] (2) at (1, -0.75) {};
		\node [contact, outer sep=5pt] (3) at (1, -1.25) {};
		\node [style=none] (18) at (2.5, -1.125) {};
		\node [style=none] (21) at (2.5, 1) {};
		\node [style=none] (22) at (2.5, -0.375) {};
		\node [style=none] (23) at (2.5, 0.475) {};
		\node [style=none] (24) at (2.5, 0.25) {};
		\node [contact, outer sep=5pt] (19) at (4, 1) {};
		\node [contact, outer sep=5pt] (14) at (4, 0.5) {};
		\node [contact, outer sep=5pt] (11) at (4, -0) {};
		\node [contact, outer sep=5pt] (13) at (4, -0.5) {};
		\node [contact, outer sep=5pt] (12) at (4, -1) {};
		\node [style=none] (4) at (-2.75, -0) {$X$};
		\node [style=none] (5) at (4.75, -0) {$Z$};
		\node [style=none] (20) at (1, -1.75) {$Y$};
		\node [style=none] (30) at (1, 1.75) {\phantom{$Y$}};
	\end{pgfonlayer}
	\begin{pgfonlayer}{edgelayer}
		\draw [thick] (6.center) to (15.center);
		\draw [thick] (8.center) to (15.center);
		\draw [thick] (-2.center) to (15.center);
		\draw [thick] (-1.center) to (15.center);
		\draw [thick] (9.center) to (16.center);
		\draw [thick] (7.center) to (16.center);
		\draw [thick] (10.center) to (17.center);
		\draw [thick] (17.center) to (2.center);
		\draw [thick] (17.center) to (3.center);
		\draw [thick] (3.center) to (18.center);
		\draw [thick] (18.center) to (12.center);
		\draw [thick] (-2.center) to (21.center);
		\draw [thick] (-1.center) to (21.center);
		\draw [thick] (21.center) to (19.center);
		\draw [thick] (1.center) to (22.center);
		\draw [thick] (2.center) to (22.center);
		\draw [thick] (22.center) to (11.center);
		\draw [thick] (22.center) to (13.center);
		\draw [thick] (23.center) to (14.center);
		\draw [thick] (28.center) to (0.center);
		\draw [thick] (0.center) to (24.center);
		\draw [thick] (29.center) to (1.center);
		\draw [rounded corners=5pt, dashed, color=gray] 
   (node cs:name=6, anchor=north west) --
   (node cs:name=8, anchor=south west) --
   (node cs:name=-1, anchor=south east) --
   (node cs:name=-2, anchor=north east) --
   cycle;
		\draw [rounded corners=5pt, dashed, color=gray] 
   (node cs:name=9, anchor=north west) --
   (node cs:name=7, anchor=south west) --
   (node cs:name=7, anchor=south east) --
   (node cs:name=9, anchor=north east) --
   cycle;
		\draw [rounded corners=5pt, dashed, color=gray] 
   (node cs:name=10, anchor=north west) --
   (node cs:name=10, anchor=south west) --
   (node cs:name=3, anchor=south east) --
   (node cs:name=2, anchor=north east) --
   cycle;
		\draw [rounded corners=5pt, dashed, color=gray] 
   (node cs:name=-2, anchor=north west) --
   (node cs:name=-1, anchor=south west) --
   (node cs:name=19, anchor=south east) --
   (node cs:name=19, anchor=north east) --
   cycle;
		\draw [rounded corners=5pt, dashed, color=gray] 
   (node cs:name=0, anchor=north west) --
   (node cs:name=0, anchor=south west) --
   (node cs:name=0, anchor=south east) --
   (node cs:name=0, anchor=north east) --
   cycle;
		\draw [rounded corners=5pt, dashed, color=gray] 
   (node cs:name=1, anchor=north west) --
   (node cs:name=1, anchor=south west) --
   (node cs:name=1, anchor=south east) --
   (node cs:name=1, anchor=north east) --
   cycle;
		\draw [rounded corners=5pt, dashed, color=gray] 
   (node cs:name=1, anchor=north west) --
   (node cs:name=2, anchor=south west) --
   (node cs:name=13, anchor=south east) --
   (node cs:name=11, anchor=north east) --
   cycle;
		\draw [rounded corners=5pt, dashed, color=gray] 
   (node cs:name=3, anchor=north west) --
   (node cs:name=3, anchor=south west) --
   (node cs:name=12, anchor=south east) --
   (node cs:name=12, anchor=north east) --
   cycle;
		\draw [rounded corners=5pt, dashed, color=gray] 
   (node cs:name=14, anchor=north west) --
   (node cs:name=14, anchor=south west) --
   (node cs:name=14, anchor=south east) --
   (node cs:name=14, anchor=north east) --
   cycle;
	\end{pgfonlayer}
\end{tikzpicture}
\end{aligned}
\:
  =
\:
\begin{aligned}
\begin{tikzpicture}[circuit ee IEC]
	\begin{pgfonlayer}{nodelayer}
		\node [style=none] (0) at (-2.75, -0) {$X$};
		\node [style=none] (1) at (1.75, -0) {$Z$};
		\node [contact, outer sep=5pt] (2) at (-2, 1) {};
		\node [contact, outer sep=5pt] (3) at (-2, -0.5) {};
		\node [contact, outer sep=5pt] (4) at (-2, 0.5) {};
		\node [contact, outer sep=5pt] (5) at (-2, -0) {};
		\node [contact, outer sep=5pt] (6) at (-2, -1) {};
		\node [contact, outer sep=5pt] (7) at (1, -0) {};
		\node [contact, outer sep=5pt] (8) at (1, -1) {};
		\node [contact, outer sep=5pt] (9) at (1, -0.5) {};
		\node [contact, outer sep=5pt] (10) at (1, 0.5) {};
		\node [style=none] (11) at (-0.5, 0.875) {};
		\node [style=none] (12) at (-0.5, 0.3) {};
		\node [contact, outer sep=5pt] (13) at (1, 1) {};
		\node [style=none] (14) at (-0.5, -0.2) {};
		\node [style=none] (15) at (-0.5, -0.6) {};
	\end{pgfonlayer}
	\begin{pgfonlayer}{edgelayer}
		\draw [thick] (2.center) to (11.center);
		\draw [thick] (4.center) to (11.center);
		\draw [thick] (11.center) to (13.center);
		\draw [thick] (5.center) to (14.center);
		\draw [thick] (3.center) to (14.center);
		\draw [thick] (15.center) to (7.center);
		\draw [thick] (15.center) to (9.center);
		\draw [thick] (6.center) to (15.center);
		\draw [thick] (15.center) to (8.center);
		\draw [thick] (12.center) to (10.center);
		\draw [rounded corners=5pt, dashed, color=gray] 
   (node cs:name=2, anchor=north west) --
   (node cs:name=4, anchor=south west) --
   (node cs:name=13, anchor=south east) --
   (node cs:name=13, anchor=north east) --
   cycle;
		\draw [rounded corners=5pt, dashed, color=gray] 
   (node cs:name=5, anchor=north west) --
   (node cs:name=3, anchor=south west) --
   (node cs:name=3, anchor=south east) --
   (node cs:name=5, anchor=north east) --
   cycle;
		\draw [rounded corners=5pt, dashed, color=gray] 
   (node cs:name=10, anchor=north west) --
   (node cs:name=10, anchor=south west) --
   (node cs:name=10, anchor=south east) --
   (node cs:name=10, anchor=north east) --
   cycle;
		\draw [rounded corners=5pt, dashed, color=gray] 
   (node cs:name=6, anchor=north west) --
   (node cs:name=6, anchor=south west) --
   (node cs:name=8, anchor=south east) --
   (node cs:name=7, anchor=north east) --
   cycle;
	\end{pgfonlayer}
\end{tikzpicture}
\end{aligned}
\]
The composition law captures the idea that connectivity is all that
matters: as long as the wires are `ideal', the exact path does not matter. 

In Coya--Fong we formalise this idea by saying that corelations are the prop for
extraspecial commutative Frobenius monoids \cite{CoyFon16}. An
\define{extraspecial commutative Frobenius monoid}
$(X,\mu,\eta,\delta,\epsilon)$ in a monoidal category $(\mathcal C, \otimes)$ is
a special commutative Frobenius monoid that further obeys the extra law
  \[
    \extral{.1\textwidth} = \extrar{.1\textwidth}
  \]

Two morphisms built from the generators of an extraspecial commutative Frobenius
monoid are equal and if and only if their diagrams impose the same connectivity
relations on the disjoint union of the domain and codomain. This is an extension
of the spider theorem for special commutative Frobenius monoids. 


\subsection{Linear relations as corelations in $\Vect$}

Recall that a linear relation $L\maps U \leadsto V$ is a subspace $L \subseteq
U \oplus V$. We compose linear relations as we do relations, and vector spaces
and linear relations form a category $\LinRel$. It is well-known that this
category can be constructed as the category of relations in the category $\Vect$
of vector spaces and linear maps with respect to epi-mono factorisations. We
show that they may also be constructed as corelations in $\Vect$ with respect to
epi-mono factorisations.

If we restrict to the full subcategory $\FinVect$ of finite dimensional vector
spaces this is easy to see: after picking a basis for each vector space the
transpose yields an equivalence of $\FinVect$ with its opposite category, so
the category of $(\mathcal E,\mathcal M)$-corelations (jointly-epic cospans)
is isomorphic to the category of $(\mathcal E,\mathcal M)$-relations
(jointly-monic spans) in $\FinVect$. This fact has been fundamental in work on
finite dimensional linear systems and signal flow diagrams \cite{BE,BSZ,FRS}.

We prove the general case in detail. To begin, note $\mathrm{Vect}$ has an
epi-mono factorisation system with monos stable under pushouts. This
factorisation system is inherited from $\Set$: the epimorphisms in $\Vect$ are
precisely the surjective linear maps, the monomorphisms are the injective
linear maps, and the image of a linear map is always a subspace of the
codomain, and so itself a vector space. Monos are stable under pushout as the
pushout of a diagram $V \stackrel{f}{\leftarrow} U \stackrel{m}{\rightarrow}
W$ is $V \oplus W/\im[f\; -g]$. The map $m'\maps V \to V \oplus W/\im[f\; -g]$
into the pushout has kernel $f(\ker m)$. Thus when $m$ is a monomorphism, $m'$
is too.

Thus we have a category of corelations $\corel(\Vect)$. We show that the map
$\corel(\Vect) \to \LinRel$ sending each vector space to itself and each
corelation
\[
  U \xrightarrow{f} A \xleftarrow{g} V
\]
to the linear subspace $\ker[f\;-g]$ is a full, faithful, and
bijective-on-objects functor.

Indeed, corelations $U \xrightarrow{f} A \xleftarrow{g} V$ are in one-to-one
correspondence with surjective linear maps $U\oplus V \to A$, which are in
turn, by the isomorphism theorem, in one-to-one correspondence with subspaces
of $U\oplus V$. These correspondences are described by the kernel construction
above. Thus our map is evidently full, faithful, and bijective-on-objects. It
also maps identities to identities. It remains to check that it preserves
composition.

Suppose we have corelations $U \xrightarrow{f} A \xleftarrow{g} V$
and $V \xrightarrow{h} B \xleftarrow{k} W$. Then their pushout is given by
$P=A \oplus B/\im[g\;-h]$, and we may draw the pushout diagram
\[
  \xymatrix{
    U \ar[dr]_{f} & & V \ar[dl]^{g}  
    \ar[dr]_{h} & & W \ar[dl]^{k} {} 
    \\
    & A \ar[dr]_{\iota_A} & & B \ar[dl]^{\iota_B}  \\
    & & P {\save*!<0cm,-.5cm>[dl]@^{|-}\restore}
  }
\]
We wish to show the equality of relations
\[
  \ker[f\;-g];\ker[h\;-k] = \ker[\iota_A f\; -\iota_B g].
\]
Now $(\mathbf{u},\mathbf{w}) \in U \oplus W$ lies in the composite relation
$\ker[f\;-g];\ker[h\;-k]$ iff there exists $\mathbf{v} \in V$ such that
$f\mathbf{u} = g\mathbf{v}$ and $h\mathbf{v} = k\mathbf{w}$. But as $P$ is the
pushout, this is true iff 
\[
  \iota_A f \mathbf{u} = \iota_A g \mathbf{v} = \iota_B h \mathbf{v} =
  \iota_B k \mathbf{w}.
\]
This in turn is true iff $(\mathbf{u}, \mathbf{w}) \in \ker[\iota_Af\;
-\iota_Bk]$, as required. 

This corelations perspective is important as it fits the relational picture into
our philosophy of black boxing. In Chapter ref we will see it is the corelation
construction of $\LinRel$ that correctly generalises to the case of
non-controllable systems.


